\documentclass[]{article}
\usepackage{lmodern}
\usepackage{amssymb,amsmath}
\usepackage{ifxetex,ifluatex}
\usepackage{fixltx2e} % provides \textsubscript
\ifnum 0\ifxetex 1\fi\ifluatex 1\fi=0 % if pdftex
  \usepackage[T1]{fontenc}
  \usepackage[utf8]{inputenc}
\else % if luatex or xelatex
  \ifxetex
    \usepackage{mathspec}
  \else
    \usepackage{fontspec}
  \fi
  \defaultfontfeatures{Ligatures=TeX,Scale=MatchLowercase}
\fi
% use upquote if available, for straight quotes in verbatim environments
\IfFileExists{upquote.sty}{\usepackage{upquote}}{}
% use microtype if available
\IfFileExists{microtype.sty}{%
\usepackage{microtype}
\UseMicrotypeSet[protrusion]{basicmath} % disable protrusion for tt fonts
}{}
\usepackage[margin=1in]{geometry}
\usepackage{hyperref}
\hypersetup{unicode=true,
            pdftitle={Recent Articles For Pancreatobiliary Pathology Society Journal Watch},
            pdfborder={0 0 0},
            breaklinks=true}
\urlstyle{same}  % don't use monospace font for urls
\usepackage{graphicx,grffile}
\makeatletter
\def\maxwidth{\ifdim\Gin@nat@width>\linewidth\linewidth\else\Gin@nat@width\fi}
\def\maxheight{\ifdim\Gin@nat@height>\textheight\textheight\else\Gin@nat@height\fi}
\makeatother
% Scale images if necessary, so that they will not overflow the page
% margins by default, and it is still possible to overwrite the defaults
% using explicit options in \includegraphics[width, height, ...]{}
\setkeys{Gin}{width=\maxwidth,height=\maxheight,keepaspectratio}
\IfFileExists{parskip.sty}{%
\usepackage{parskip}
}{% else
\setlength{\parindent}{0pt}
\setlength{\parskip}{6pt plus 2pt minus 1pt}
}
\setlength{\emergencystretch}{3em}  % prevent overfull lines
\providecommand{\tightlist}{%
  \setlength{\itemsep}{0pt}\setlength{\parskip}{0pt}}
\setcounter{secnumdepth}{0}
% Redefines (sub)paragraphs to behave more like sections
\ifx\paragraph\undefined\else
\let\oldparagraph\paragraph
\renewcommand{\paragraph}[1]{\oldparagraph{#1}\mbox{}}
\fi
\ifx\subparagraph\undefined\else
\let\oldsubparagraph\subparagraph
\renewcommand{\subparagraph}[1]{\oldsubparagraph{#1}\mbox{}}
\fi

%%% Use protect on footnotes to avoid problems with footnotes in titles
\let\rmarkdownfootnote\footnote%
\def\footnote{\protect\rmarkdownfootnote}

%%% Change title format to be more compact
\usepackage{titling}

% Create subtitle command for use in maketitle
\providecommand{\subtitle}[1]{
  \posttitle{
    \begin{center}\large#1\end{center}
    }
}

\setlength{\droptitle}{-2em}

  \title{Recent Articles For Pancreatobiliary Pathology Society Journal Watch}
    \pretitle{\vspace{\droptitle}\centering\huge}
  \posttitle{\par}
  \subtitle{Others}
  \author{}
    \preauthor{}\postauthor{}
      \predate{\centering\large\emph}
  \postdate{\par}
    \date{Last Update on 2019-08-20}


\begin{document}
\maketitle

{
\setcounter{tocdepth}{5}
\tableofcontents
}
\begin{center}\rule{0.5\linewidth}{\linethickness}\end{center}

\hypertarget{pbpath-journal-watch-articles}{%
\section{PBPath Journal Watch
Articles}\label{pbpath-journal-watch-articles}}

\begin{center}\rule{0.5\linewidth}{\linethickness}\end{center}

\textbf{Wellcome to the PBPath Journal Watch!}

This bi-monthly journal watch features exciting recently published
pancreas and biliary pathology articles that will provide up to date
medical knowledge in our field. These articles will be showcased in
several convenient categories, including surgical pathology,
cytopathology, and molecular pathology among others. The articles in
each category are in no particular order. See the list of journals we
search regularly \href{http://pbpath.org/pbpath-journal-watch/}{here}.
Previous months' issues may be found in our
\emph{\href{http://pbpath.org/journal-watch-archive/}{archive}} and you
may see
\href{http://pbpath.org/journal-watch-upcoming-issue/}{preparation of
upcoming issue here}.

We encourage members to actively participate by recommending new
articles and providing feedback using the forms provided below.

\textbf{We hope that you will enjoy the new PBPath Journal Watch!}

\begin{center}\rule{0.5\linewidth}{\linethickness}\end{center}

\textbf{Feedback}

Please send your feedbacks using the forms below:

\href{https://docs.google.com/forms/d/e/1FAIpQLSeD3Z9J6Y7eMmiyM12f_SfAmHUlykb1zxZcwO6lg7cebGYQIQ/viewform}{Google
Feedback Form}

\hypertarget{disqus_thread}{}

Please enable JavaScript to view the comments powered by Disqus.

\begin{center}\rule{0.5\linewidth}{\linethickness}\end{center}

Back to top

\begin{center}\rule{0.5\linewidth}{\linethickness}\end{center}

\pagebreak

\hypertarget{recent-articles}{%
\subsection{Recent Articles}\label{recent-articles}}

\begin{center}\rule{0.5\linewidth}{\linethickness}\end{center}

\hypertarget{pancreas}{%
\subsubsection{Pancreas}\label{pancreas}}

\begin{center}\rule{0.5\linewidth}{\linethickness}\end{center}

 - \textbf{Response to the Letter to the Editor ``Minimally Invasive
Versus Open Distal Pancreatectomy (LEOPARD)''}

\emph{Annals of surgery 2019 Aug;():}

PubMed: \url{https://www.ncbi.nlm.nih.gov/pubmed/?term=31425333}

doi: \url{https://doi.org/10.1097/SLA.0000000000003541}

{}

{}

\begin{center}\rule{0.5\linewidth}{\linethickness}\end{center}

\begin{center}\rule{0.5\linewidth}{\linethickness}\end{center}

 - \textbf{Significance of Examined Lymph Node Number in Accurate
Staging and Long-term Survival in Resected Stage I-II Pancreatic
Cancer-More is Better? A Large International Population-based Cohort
Study}

\emph{Annals of surgery 2019 Aug;():}

PubMed: \url{https://www.ncbi.nlm.nih.gov/pubmed/?term=31425290}

OBJECTIVE: This large international cohort study aimed to investigate
the associations of examined lymph node (ELN) number with accurate
staging and long-term survival in pancreatic adenocarcinoma (PaC) and to
robustly determine the minimal and optimal ELN thresholds. SUMMARY
BACKGROUND DATA: ELN number is an important quality metric in cancer
care. The recommended minimal ELN number in PaC to accurately stage
cancer varies greatly across guidelines, and the optimal number
especially to adequately stratify patient survival has not yet been
established. METHODS: Population-based data on patients with stage I to
II PaC resected in 2003 to 2015 from the US Surveillance, Epidemiology,
and End Results (SEER)-18 Program and Netherlands National Cancer
Registry (NCR) were analyzed. Associations of ELN number with stage
migration and survival were evaluated using multivariable-adjusted
logistic and Cox regression models, respectively. The series of odds
ratios (ORs) for negative-to-positive node stage migration and hazard
ratios (HRs) for survival with more ELNs were fitted using a LOWESS
smoother, and structural breakpoints were determined by Chow test.
RESULTS: Overall 16,241 patients were analyzed. With increasing ELN
number, both cohorts exhibited significant proportional increases from
node-negative to node-positive disease {[}ORSEER-18=1.05, 95\%
confidence interval (CI) = 1.04-1.05; ORNCR = 1.10, 95\% CI =
1.08-1.12{]} and serial improvements in survival (HRSEER-18 = 0.98, 95\%
CI = 0.98-0.99; HRNCR = 0.98, 95\% CI = 0.97-0.99) per additional ELN
after controlling for confounders. Associations for stage migration and
survival remained significant in most stratifications by patient, tumor,
and treatment factors. Cut-point analyses suggested a minimal threshold
ELN number of 11 and an optimal number of 19, which were validated both
internally in the derivative US cohort and externally in the Dutch
cohort with the ability to well discriminate different probabilities of
both survival and stage migration. CONCLUSIONS: In stage I to II PaC,
more ELNs are associated with more precise nodal staging, which might
largely explain the survival association. Our observational study does
not suggest causality, and does not encourage more extended
lymphadenectomy before further randomized evidence is obtained. Our
results robustly conclude 11 ELNs as the minimal and suggest 19 ELNs as
the optimal cut-points, for evaluating quality of lymph node examination
and possibly for stratifying postoperative prognosis.

doi: \url{https://doi.org/10.1097/SLA.0000000000003558}

{}

{}

\begin{center}\rule{0.5\linewidth}{\linethickness}\end{center}

\begin{center}\rule{0.5\linewidth}{\linethickness}\end{center}

 - \textbf{Cardiovascular, mortality, and kidney outcomes with GLP-1
receptor agonists in patients with type 2 diabetes: a systematic review
and meta-analysis of cardiovascular outcome trials}

\emph{The lancet. Diabetes \& endocrinology 2019 Aug;():}

PubMed: \url{https://www.ncbi.nlm.nih.gov/pubmed/?term=31422062}

BACKGROUND: Glucagon-like peptide-1 (GLP-1) receptor agonists differ in
their structure and duration of action and have been studied in trials
of varying sizes and with different patient populations, with
inconsistent effects on cardiovascular outcomes reported. We aimed to
synthesise the available evidence by doing a systematic review and
meta-analysis of cardiovascular outcome trials of these drugs. METHODS:
We searched MEDLINE (via PubMed) and the Cochrane Central Register of
Controlled Trials for eligible placebo-controlled trials reporting major
adverse cardiovascular events (MACE; ie, cardiovascular death, stroke,
or myocardial infarction) up to June 15, 2019. We did a meta-analysis
using a random-effects model to estimate overall hazard ratios (HRs) for
MACE, its components, death from any cause, hospital admission for heart
failure, kidney outcomes, and key safety outcomes (severe hypoglycaemia,
pancreatitis, and pancreatic cancer). We also examined MACE in several
subgroups based on patient characteristics (history of cardiovascular
disease, BMI, age, baseline HbA1c, and baseline estimated glomerular
filtration rate), trial duration, treatment dosing interval, and
structural homology. FINDINGS: Of 27 publications screened, seven
trials, with a combined total of 56 004 participants, were included:
ELIXA (lixisenatide), LEADER (liraglutide), SUSTAIN-6 (semaglutide),
EXSCEL (exenatide), Harmony Outcomes (albiglutide), REWIND
(dulaglutide), and PIONEER 6 (oral semaglutide). Overall, GLP-1 receptor
agonist treatment reduced MACE by 12\% (HR 0·88, 95\% CI 0·82-0·94;
p\textless{}0·0001). There was no statistically significant
heterogeneity across the subgroups examined. HRs were 0·88 (95\% CI
0·81-0·96; p=0·003) for death from cardiovascular causes, 0·84
(0·76-0·93; p\textless{}0·0001) for fatal or non-fatal stroke, and 0·91
(0·84-1·00; p=0·043) for fatal or non-fatal myocardial infarction. GLP-1
receptor agonist treatment reduced all-cause mortality by 12\% (0·88,
0·83-0·95; p=0·001), hospital admission for heart failure by 9\% (0·91,
0·83-0·99; p=0·028), and a broad composite kidney outcome (development
of new-onset macroalbuminuria, decline in estimated glomerular
filtration rate {[}or increase in creatinine{]}, progression to
end-stage kidney disease, or death attributable to kidney causes) by
17\% (0·83, 0·78-0·89; p\textless{}0·0001), mainly due to a reduction in
urinary albumin excretion. There was no increase in risk of severe
hypoglycaemia, pancreatitis, or pancreatic cancer. INTERPRETATION:
Treatment with GLP-1 receptor agonists has beneficial effects on
cardiovascular, mortality, and kidney outcomes in patients with type 2
diabetes. FUNDING: None.

doi: \url{https://doi.org/10.1016/S2213-8587(19)30249-9}

{}

{}

\begin{center}\rule{0.5\linewidth}{\linethickness}\end{center}

\begin{center}\rule{0.5\linewidth}{\linethickness}\end{center}

 - \textbf{Long-term outcomes of therapeutic ERCP in pediatric patients
with pancreas divisum presenting with acute recurrent or chronic
pancreatitis}

\emph{Pancreatology : official journal of the International Association
of Pancreatology (IAP) \ldots{} {[}et al.{]} 2019 Aug;():}

PubMed: \url{https://www.ncbi.nlm.nih.gov/pubmed/?term=31421974}

OBJECTIVES: The aim of this study was to evaluate the long-term outcomes
of therapeutic endoscopic retrograde cholangiopancreatography (ERCP) for
pediatric patients with pancreas divisum (PD) presenting with acute
recurrent pancreatitis (ARP) or chronic pancreatitis (CP). METHODS:
Between May 2008 and August 2017, pediatric patients with PD who
received endotherapy at Ruijin Hospital were identified and grouped
according to clinical presentation, namely ARP and CP. Primary success
was defined as patients' improvement in symptoms after index ERCPs,
without further intervention or any analgesic. RESULTS: A total of 74
ERCPs were performed in 38 pediatric patients. The frequency of at least
1 genetic mutation identified in patients with ARP and CP was 44.4\% and
68.4\%, respectively. Patients with CP required more ERCPs than those
with ARP (2.4\,±\,1.7 vs.~1.1\,±\,0.4, P\,=\,0.005). The incidence of
post-ERCP complications was 14.9\%, including pancreatitis of 13.5\% and
hemorrhage of 1.4\%. During a median follow-up duration of 41 months
(range, 12-123 months), the frequency of pancreatitis episodes decreased
significantly from 2.31 to 0.45 (P\,\textless{}\,0.0001). The 25\%
recurrence and reintervention rates were estimated at 25 and 48 months,
respectively, without significant difference between patients with ARP
or CP. There was a nonsignificant trend towards a higher rate of primary
success in patients with ARP than those with CP (92.9\% vs.~69.6\%,
P\,=\,0.123). After further endotherapy, 91.3\% patients with CP
improved clinically. CONCLUSIONS: Therapeutic ERCP is an effective and
safe intervention for pediatric patients with symptomatic PD. Patients
presenting with CP seem to achieve improvement after additional ERCPs.

doi: \url{https://doi.org/10.1016/j.pan.2019.08.004}

{}

{}

\begin{center}\rule{0.5\linewidth}{\linethickness}\end{center}

\begin{center}\rule{0.5\linewidth}{\linethickness}\end{center}

 - \textbf{First-in-human Study of Mivebresib (ABBV-075), an Oral
Pan-inhibitor of Bromodomain and Extra Terminal Proteins, in Patients
with Relapsed/Refractory Solid Tumors}

\emph{Clinical cancer research : an official journal of the American
Association for Cancer Research 2019 Aug;():}

PubMed: \url{https://www.ncbi.nlm.nih.gov/pubmed/?term=31420359}

PURPOSE: Bromodomain and extra-terminal (BET) proteins play important
roles in transcriptional regulation relevant to cancer pathogenesis, and
therapeutic targeting/inhibition of BET causes apoptosis of cancer cells
in vitro In this first-in-human study of the pan-BET inhibitor
mivebresib (ABBV-075) the safety profile, maximal tolerated dose (MTD),
and recommended phase 2 dose (RP2D) were determined in patients with
advanced solid tumors. EXPERIMENTAL DESIGN: A 3+3 dose escalation for
different mivebresib dosing schedules (daily, Monday/Wednesday/Friday
{[}M-W-F{]}, 4 days on/3 off {[}4/7{]}) was followed by dose expansion
in prostate cancer patients. Endpoints were safety, tolerability,
pharmacokinetics, and preliminary antitumor activity. RESULTS:
Seventy-two patients with solid tumors {[}14\% uveal melanoma; 11\%
colorectal, 11\% breast; 8\% pancreatic; 7\% head/neck; 49\% others{]}
were treated with mivebresib during dose escalation, and 12 additional
patients with prostate cancer in expansion cohort. Most common TEAEs
related to mivebresib were dysgeusia (49\%), thrombocytopenia (48\%),
fatigue (26\%) and nausea (25\%). Most common grade 3/4 TEAEs related to
mivebresib were thrombocytopenia (35\%) and anemia (6\%). Dose-limiting
toxicities included thrombocytopenia (2mg daily; 4.5mg M-W-F),
gastrointestinal bleed (2mg daily), hypertension (2-3mg 4/7), fatigue,
decreased appetite, and aspartate aminotransferase elevation (4mg
M-W-F). Of 61 evaluable patients from dose-escalation, 26 (43\%) had
stable disease and 35 (57\%) had progressive disease. Median
progression-free survival was 1.8 months (95\%CI: 1.8,1.9). CONCLUSIONS:
Based on safety and tolerability, mivebresib RP2D is 1.5mg for the daily
schedule, 2.5mg for 4/7 and 3mg for M-W-F. Mivebresib has a tolerable
safety profile and stable disease was observed in some patients with
malignant solid tumors.

doi: \url{https://doi.org/10.1158/1078-0432.CCR-19-0578}

{}

{}

\begin{center}\rule{0.5\linewidth}{\linethickness}\end{center}

\begin{center}\rule{0.5\linewidth}{\linethickness}\end{center}

 - \textbf{Pancreatic tumor in type 1 autoimmune pancreatitis: a
diagnostic challenge}

\emph{BMC cancer 2019 Aug;19(1):814}

PubMed: \url{https://www.ncbi.nlm.nih.gov/pubmed/?term=31419961}

BACKGROUND: The co-occurrence of type 1 autoimmune pancreatitis (AIP)
and pancreatic tumor (PaT) has been previously reported. Pure AIP cases
have favorable prognosis and are primarily treated with steroids, while
AIP cases with PaT are associated with poor prognosis where the primary
management is pancreatic resection. However, it's a challenge to timely
identify the concurrent PaT in AIP because of their similar clinical and
radiological manifestations. METHODS: We retrospectively reviewed the
data in two medical centers from January 2010 to April 2019. The
inclusion criteria were as follows: 1) completion of abdominal CT
imaging before invasive procedures to the pancreas, 2) a final diagnosis
of type 1 AIP using the 2011 international consensus diagnostic
criteria, 3) follow-up duration of at least one month unless AIP and PaT
were identified simultaneously. The presence of PaT in AIP was made
based on histopathological confirmation, and the absence of PaT in AIP
was defined as no pathological or radiological evidence of concurrent
PaT. Clinical and radiological characteristics including gender, age,
surveillance period, serum IgG4 and Ca-199 levels, biopsy,
extrapancreatic involvement, CT and MR (if performed) imaging
characteristics were compared between AIP with and without PaT. The
Fisher's exact test was used for qualitative variables, and
nonparametric Mann-Whitney test for quantitative variables. A p value
≤0.05 was considered statistically significant. RESULTS: A total of 74
patients with type 1 AIP were included, of which 5 (6.7\%) had the
concurrent PaT. The subtypes were pancreatic ductal adenocarcinoma
(3/5), solitary extramedullary plasmacytoma in the pancreas (1/5) and
cholangiocarcinoma in the pancreatic segment (1/5), respectively. Gender
(p = 0.044), the pattern of pancreatic enlargement (p = 0.003),
heterogeneity (p = 0.015), low-density (p = 0.004) on CT and rim
enhancement on MRI (p = 0.050) differed significantly between AIP with
and without PaT. None of the low-density characteristics on CT or other
assessed MRI characteristics could significantly differentiate the two
groups (p\textgreater{}0.05). CONCLUSIONS: Female, focal pancreatic
enlargement, pancreatic heterogeneity, low-density on CT and rim
enhancement on MRI are suggestive of the concurrent PaT in type 1 AIP.
The characteristics of low-density on CT or other MRI characteristics
did not provide further diagnostic values.

doi: \url{https://doi.org/10.1186/s12885-019-6027-0}

{}

{}

\begin{center}\rule{0.5\linewidth}{\linethickness}\end{center}

\begin{center}\rule{0.5\linewidth}{\linethickness}\end{center}

 - \textbf{Transgenic Expression of PRSS1R122H Sensitizes Mice to
Pancreatitis}

\emph{Gastroenterology 2019 Aug;():}

PubMed: \url{https://www.ncbi.nlm.nih.gov/pubmed/?term=31419436}

BACKGROUND \& AIMS: Mutations in the trypsinogen gene (PRSS1) cause
human hereditary pancreatitis. However, it is not clear how mutant forms
of PRSS1 contribute to disease development. We studied the effects of
expressing mutant forms of human PRSS1 in mice. METHODS: We expressed
forms of PRSS1 with and without the mutation encoding R122H (PRSS1R122H)
specifically in pancreatic acinar cells under control of a full-length
pancreatic elastase gene promoter. Mice that did not express these
transgenes were used as controls. Mice were given injections of
caerulein to induce acute pancreatitis or injections of
lipopolysaccharide (LPS) to induce chronic pancreatitis. Other groups of
mice were fed ethanol or placed on a high-fat diet to induce
pancreatitis. Pancreata were collected and analyzed by histology,
immunoblots, real-time PCR, and immunohistochemistry. Trypsin enzymatic
activity and chymotrypsin enzymatic activity were measured in pancreatic
homogenates. Blood was collected and serum amylase activity was
measured. RESULTS: Pancreata from mice expressing transgenes encoding
PRSS1 or PRSS1R122H had focal areas of inflammation; these lesions were
more prominent in mice that express PRSS1R122H. Pancreata from mice that
express PRSS1 or PRSS1R122H had increased levels of HSP70 and NRF2 and
reduced levels of chymotrypsin C (CTRC), compared with control mice.
Increased expression of PRSS1 or PRSS1R122H increased focal damage in
pancreatic tissues and increased the severity of acute pancreatitis
after caerulein injection. Administration of LPS exacerbated
inflammation in mice that express PRSS1R122H compared to mice that
express PRSS1 or control mice. Mice that express PRSS1R122H developed
more severe pancreatitis after ethanol feeding or a high-fat diet than
mice that express PRSS1 or control mice. Pancreata from mice that
express PRSS1R122H had more DNA damage, apoptosis, and collagen
deposition and increased trypsin activity and infiltration by
inflammatory cells than mice that express PRSS1 or control mice.
CONCLUSIONS: Expression of a transgene encoding PRSS1R122H in mice
promoted inflammation and increase the severity of pancreatitis,
compared with mice that express PRSS1 or control mice. These mice might
be used as a model for human hereditary pancreatitis and can be studied
to determine mechanisms of induction of pancreatitis by LPS, ethanol, or
a high-fat diet.

doi: \url{https://doi.org/10.1053/j.gastro.2019.08.016}

{}

{}

\begin{center}\rule{0.5\linewidth}{\linethickness}\end{center}

\begin{center}\rule{0.5\linewidth}{\linethickness}\end{center}

 - \textbf{Perioperative Gemcitabine + Erlotinib Plus
Pancreaticoduodenectomy for Resectable Pancreatic Adenocarcinoma: ACOSOG
Z5041 (Alliance) Phase II Trial}

\emph{Annals of surgical oncology 2019 Aug;():}

PubMed: \url{https://www.ncbi.nlm.nih.gov/pubmed/?term=31418130}

BACKGROUND: There is considerable interest in a neoadjuvant approach for
resectable pancreatic ductal adenocarcinoma (PDAC). This study evaluated
perioperative gemcitabine + erlotinib (G+E) for resectable PDAC.
METHODS: A multicenter, cooperative group, single-arm, phase II trial
was conducted between April 2009 and November 2013 (ACOSOG Z5041).
Patients with biopsy-confirmed PDAC in the pancreatic head without
evidence of involvement of major mesenteric vessels (resectable) were
eligible. Patients (n~=~123) received an 8-week cycle of G+E before and
after surgery. The primary endpoint was 2-year overall survival (OS),
and secondary endpoints included toxicity, response, resection rate, and
time to progression. Resectability was assessed retrospectively by
central review. The study closed early due to slow accrual, and no
formal hypothesis testing was performed. RESULTS: Overall, 114 patients
were eligible, consented, and initiated protocol treatment. By central
radiologic review, 97 (85\%) of the 114 patients met the
protocol-defined resectability criteria. Grade 3+ toxicity was reported
in 60\% and 79\% of patients during the neoadjuvant phase and overall,
respectively. Twenty-two of 114 (19\%) patients did not proceed to
surgery; 83 patients (73\%) were successfully resected. R0 and R1
margins were obtained in 67 (81\%) and 16 (19\%) resected patients,
respectively, and 54 patients completed postoperative G+E (65\%). The
2-year OS rate for the entire cohort (n~=~114) was 40\% (95\% confidence
interval {[}CI{]} 31-50), with a median OS of 21.3 months (95\% CI
17.2-25.9). The~2-year OS rate for resected patients (n~=~83) was 52\%
(95\% CI 41-63), with a median OS of 25.4 months (95\% CI 21.8-29.6).
CONCLUSIONS: For resectable PDAC, perioperative G+E is feasible. Further
evaluation of neoadjuvant strategies in resectable PDAC is warranted
with more active systemic regimens.

doi: \url{https://doi.org/10.1245/s10434-019-07685-1}

{}

{}

\begin{center}\rule{0.5\linewidth}{\linethickness}\end{center}

\begin{center}\rule{0.5\linewidth}{\linethickness}\end{center}

 - \textbf{Cell phenotypic plasticity requires autophagic flux driven by
YAP/TAZ mechanotransduction}

\emph{Proceedings of the National Academy of Sciences of the United
States of America 2019 Aug;():}

PubMed: \url{https://www.ncbi.nlm.nih.gov/pubmed/?term=31416916}

Autophagy, besides ensuring energy metabolism and organelle renewal, is
crucial for the biology of adult normal and cancer stem cells. However,
it remains incompletely understood how autophagy connects to stemness
factors and the nature of the microenvironmental signals that pattern
autophagy in different cell types. Here we advance in these directions
by reporting that YAP/TAZ transcriptionally control autophagy, being
critical for autophagosomal degradation into autolysosomes. YAP/TAZ are
downstream effectors of cellular mechanotransduction and indeed we found
that cell mechanics, dictated by the physical property of the ECM and
cytoskeletal tension, profoundly impact on autophagic flux in a
YAP/TAZ-mediated manner. Functionally, by using pancreatic and mammary
organoid cultures, we found that YAP/TAZ-regulated autophagy is
essential in normal cells for YAP/TAZ-mediated dedifferentiation and
acquisition of self-renewing properties. In tumor cells, the
YAP/TAZ-autophagy connection is key to sustain transformed traits and
for acquisition of a cancer stem cell state by otherwise more benign
cells. Mechanistically, YAP/TAZ promote autophagic flux by directly
promoting the expression of Armus, a RAB7-GAP required for autophagosome
turnover and whose add-back rescues autophagy in YAP/TAZ-depleted cells.
These findings expand the influence of YAP/TAZ mechanotransduction to
the control of autophagy and, vice versa, the role of autophagy in
YAP/TAZ biology, and suggest a mechanism to coordinate transcriptional
rewiring with cytoplasmic restructuring during cell reprogramming.

doi: \url{https://doi.org/10.1073/pnas.1908228116}

{}

{}

\begin{center}\rule{0.5\linewidth}{\linethickness}\end{center}

\begin{center}\rule{0.5\linewidth}{\linethickness}\end{center}

 - \textbf{A multicenter, open-label, single-arm study of anamorelin
(ONO-7643) in advanced gastrointestinal cancer patients with cancer
cachexia}

\emph{Cancer 2019 Aug;():}

PubMed: \url{https://www.ncbi.nlm.nih.gov/pubmed/?term=31415709}

BACKGROUND: Cancer cachexia is characterized by weight loss and is
associated with increased morbidity and mortality in patients with
cancer. Anamorelin (ONO-7643; ANAM) is a novel and selective ghrelin
receptor agonist that improves appetite, lean body mass (LBM), body
weight, and anorexia. METHODS: This multicenter, open-label, single-arm
study investigated the efficacy and safety of 100~mg anamorelin in 50
Japanese patients with advanced and unresectable gastrointestinal
(colorectal, gastric, or pancreatic) cancer. ANAM was administered once
daily over 12~weeks. The primary endpoint was the proportion of patients
that maintained or gained LBM over the course of the study. Secondary
endpoints included changes in LBM, body weight, quality of life (QoL),
and nutritional status biomarkers. RESULTS: The proportion of patients
who responded to treatment was 63.3\% (95\% CI, 48.3\%-76.6\%), with a
least square mean~±~SE change in LBM and body weight from baseline of
1.89~±~0.36~kg and 1.41~±~0.61~kg, respectively. Appetite-related
questions on the QoL questionnaire showed that ANAM improved appetite.
Adverse events occurred in 79.6\% of patients, and the most common
treatment-related adverse events were increased γ-glutamyl
transpeptidase (8.2\%), diabetes mellitus (6.1\%), hyperglycemia
(6.1\%), and prolonged QRS complex (6.1\%). CONCLUSIONS: ANAM improved
anorexia and patients' nutritional status, resulting in rapid increases
in LBM and body weight in patients with advanced gastrointestinal cancer
who had cancer cachexia. ANAM treatment was well tolerated over
12~weeks. ANAM is a potential clinically beneficial pharmacotherapeutic
option for patients with advanced gastrointestinal cancer who have
cancer cachexia.

doi: \url{https://doi.org/10.1002/cncr.32406}

{}

{}

\begin{center}\rule{0.5\linewidth}{\linethickness}\end{center}

\begin{center}\rule{0.5\linewidth}{\linethickness}\end{center}

 - \textbf{Grading Pancreatic Neuroendocrine Tumors by Ki-67 Index
Evaluated on Fine-Needle Aspiration Cell Block Material}

\emph{American journal of clinical pathology 2019 Aug;():}

PubMed: \url{https://www.ncbi.nlm.nih.gov/pubmed/?term=31415691}

OBJECTIVES: This study aimed to determine whether Ki-67 index evaluated
on cytologic material could reliably grade pancreatic neuroendocrine
tumors (PanNETs). METHODS: Cases with adequate cell block and available
surgical specimens were included. Ki-67 index was calculated using
``eyeballing,'' ``hot spot,'' and ``complete'' counting methods.
RESULTS: The overall concordance rates between cytology and surgical
specimens were 71\%, 73\%, and 59\%, respectively, by using eyeballing,
hot spot, and complete counting approaches. All grade 1 tumors were
correctly graded on cytology, but in grade 2 tumors concordance rates
were only 36\%, 41\%, and 9\%, respectively. All grade 2 tumors were
undergraded when cell blocks contained fewer than 1,000 cells, while
concordance rate increased to 57\%, 64\%, and 14\%, respectively, in
cases with 1,000 cells or more. CONCLUSIONS: Grade 2 PanNETs can be
significantly undergraded when Ki-67 index is evaluated on cell block
material. In cases with 1,000 or more cells, the hot spot counting
method has better correlation with surgical specimens.

doi: \url{https://doi.org/10.1093/ajcp/aqz110}

{}

{}

\begin{center}\rule{0.5\linewidth}{\linethickness}\end{center}

\begin{center}\rule{0.5\linewidth}{\linethickness}\end{center}

 - \textbf{Randomized Comparison of Gastric Tube Reconstruction With and
Without Duodenal Diversion Plus Roux-en-Y Anastomosis After
Esophagectomy}

\emph{Annals of surgery 2019 Aug;():}

PubMed: \url{https://www.ncbi.nlm.nih.gov/pubmed/?term=31415003}

OBJECTIVE: This prospective randomized phase-II trial examined whether
gastric reconstruction with duodenal diversion plus Roux-en-Y
anastomosis(RY) minimized gastroduodenal reflux and delayed gastric
emptying compared with standard gastric reconstruction. SUMMARY
BACKGROUND DATA: There is no established standard surgical procedure to
prevent both gastroduodenal reflux and delayed gastric emptying
simultaneously. METHODS: Sixty patients with thoracic esophageal cancer
scheduled to undergo esophagectomy with retrosternal gastric tube
reconstruction were randomly allocated to standard gastric
reconstruction (non-RY, n = 31) or gastric reconstruction with duodenal
diversion plus RY (n = 29) groups. Primary endpoint was quality of life
assessed by DAUGS-32 score 1 year after surgery. Secondary endpoints
were the extent of postoperative duodenal juice reflux into the gastric
tube, postoperative morbidity, endoscopic findings, body weight changes,
and nutritional status. RESULTS: Preoperative clinicopathological
characteristics and postoperative morbidity did not differ significantly
between groups. However, operation time and blood loss volume were
significantly higher in the RY group. Pancreatic amylase concentrations
in the gastric conduit on postoperative days 2, 3, and 7 were higher in
the non-RY group. Postoperative endoscopic examination showed residual
gastric content in 7 of 17 patients in the non-RY group but in none in
the RY group (P = 0.012). Quality of life was significantly favorable in
the RY group with regard to reflux symptoms and food passage
dysfunction. Postoperative body weight changes, serum albumin levels,
and peripheral blood lymphocyte counts were not significantly different
between groups. CONCLUSION: Gastric reconstruction with duodenal
diversion plus RY is effective in improving both gastroduodenal reflux
and delayed gastric emptying.

doi: \url{https://doi.org/10.1097/SLA.0000000000003557}

{}

{}

\begin{center}\rule{0.5\linewidth}{\linethickness}\end{center}

\begin{center}\rule{0.5\linewidth}{\linethickness}\end{center}

 - \textbf{Rosai-Dorfman Disease of the Digestive System-Beware
Vasculopathy: A Clinicopathologic Analysis}

\emph{The American journal of surgical pathology 2019 Aug;():}

PubMed: \url{https://www.ncbi.nlm.nih.gov/pubmed/?term=31414989}

Rosai-Dorfman disease (RDD) is a rare non-Langerhans cell histiocytic
proliferation that occurs in nodal and extranodal sites. Rare examples
of the disease involving the digestive system have been described. To
characterize the digestive tract manifestations of this disease, 12
specimens from 11 patients with extranodal RDD affecting the digestive
organs were analyzed. Hematoxylin and eosin sections and available
immunohistochemical stains were reviewed, and the clinical information
was obtained from patients' electronic or submitted records. Eight
patients were female and 3 male (median age, 65 y; range, 17 to 76 y).
Abdominal pain was the most frequent symptom. Six patients had an
associated immunologic or malignant disease. Nine lesions arose in the
gastrointestinal tract (1 involving the appendix, 2 right colon, 6 left
colon), 2 in the pancreas, and 1 in the liver. Two patients had the
coexistent nodal disease, and 1 had bone and soft-tissue involvement.
The lesions were generally composed of polygonal to spindle-shaped
histiocytes with eosinophilic to clear cytoplasm admixed with
lymphoplasmacytic cells. The inflammatory cells formed lymphoid
aggregates in 7 cases and included focally scattered or small
collections of neutrophils in 6 cases. Fibrosis was variable, and 4
cases had a storiform pattern. Vasculopathy in the form of a thickened
capillary wall, medium-sized arterial wall infiltration by lesional and
inflammatory cells and phlebitis was seen in 10, 5, and 2 cases,
respectively. All cases were reactive for S100-protein. Of the 5
patients with follow-up, 1 developed immunoglobulin A nephropathy and
died of renal failure.

doi: \url{https://doi.org/10.1097/PAS.0000000000001343}

{}

{}

\begin{center}\rule{0.5\linewidth}{\linethickness}\end{center}

\begin{center}\rule{0.5\linewidth}{\linethickness}\end{center}

 - \textbf{The applications of metabolomics in the molecular diagnostics
of cancer}

\emph{Expert review of molecular diagnostics 2019 Aug;():}

PubMed: \url{https://www.ncbi.nlm.nih.gov/pubmed/?term=31414918}

Introduction: Metabolomics, the study of metabolites, is a promising
research field for cancers. The metabolic pathway in a tumor cell is
different from a normal tissue cell. There are two approaches to study
the metabolism, targeted and untargeted. The general approach is that
metabolomic data are interpreted by bioinformatics tools correlating
with metabolomic databases to obtain significant findings. With the use
of specific analysis tools, such as nuclear magnetic resonance (NMR) and
mass spectrometer (MS) combined with chromatography, metabolic profile
or metabolic fingerprint of various biological specimens could be
obtained. The applications of metabolomics are used to discover
potential cancer biomarkers and monitor the metastatic state,
therapeutic and drug response for better patient management. Areas
covered: In this review, the author introduce metabolomics and discuss
the use of metabolomics approaches in different cancers, including the
study of colorectal cancer, prostate cancer, liver cancer, pancreatic
cancer and breast cancer using NMR and MS. Expert opinion: Knowledge on
the molecular basis of cancer metabolism and its potential clinical
applications has been improving recently. However, there are still many
challenges for the technological development and integration of
metabolomics with other omics spaces such as genomics. In the near
future, it is expected that metabolomics will play an important role in
cancer molecular diagnostics.

doi: \url{https://doi.org/10.1080/14737159.2019.1656530}

{}

{}

\begin{center}\rule{0.5\linewidth}{\linethickness}\end{center}

\begin{center}\rule{0.5\linewidth}{\linethickness}\end{center}

 - \textbf{Lipid droplet velocity is a microenvironmental sensor of
aggressive tumors regulated by V-ATPase and PEDF}

\emph{Laboratory investigation; a journal of technical methods and
pathology 2019 Aug;():}

PubMed: \url{https://www.ncbi.nlm.nih.gov/pubmed/?term=31409893}

Lipid droplets (LDs) utilize microtubules (MTs) to participate in
intracellular trafficking of cargo proteins. Cancer cells accumulate LDs
and acidify their tumor microenvironment (TME) by increasing the proton
pump V-ATPase. However, it is not known whether these two metabolic
changes are mechanistically related or influence LD movement. We
postulated that LD density and velocity are progressively increased with
tumor aggressiveness and are dependent on V-ATPase and the lipolysis
regulator pigment epithelium-derived factor (PEDF). LD density was
assessed in human prostate cancer (PCa) specimens across Gleason scores
(GS) 6-8. LD distribution and velocity were analyzed in low and highly
aggressive tumors using live-cell imaging and in cells exposed to low pH
and/or treated with V-ATPase inhibitors. The MT network was disrupted
and analyzed by α-tubulin staining. LD density positively correlated
with advancing GS in human tumors. Acidification promoted peripheral
localization and clustering of LDs. Highly aggressive prostate, breast,
and pancreatic cell lines had significantly higher maximum LD velocity
(LDVmax) than less aggressive and benign cells. LDVmax was MT-dependent
and suppressed by blocking V-ATPase directly or indirectly with PEDF.
Upon lowering pH, LDs moved to the~cell periphery and carried
metalloproteinases. These results suggest that acidification of the TME
can alter intracellular LD movement and augment velocity in cancer.
Restoration of PEDF or blockade of V-ATPase can normalize LD
distribution and decrease velocity. This study identifies V-ATPase and
PEDF as new modulators of LD trafficking in the cancer microenvironment.

doi: \url{https://doi.org/10.1038/s41374-019-0296-8}

{}

{}

\begin{center}\rule{0.5\linewidth}{\linethickness}\end{center}

\begin{center}\rule{0.5\linewidth}{\linethickness}\end{center}

 - \textbf{Mouse pancreatic ductal organoid culture as a relevant model
to study exocrine pancreatic ion secretion}

\emph{Laboratory investigation; a journal of technical methods and
pathology 2019 Aug;():}

PubMed: \url{https://www.ncbi.nlm.nih.gov/pubmed/?term=31409889}

Pancreatic exocrine secretory processes are challenging to investigate
on primary epithelial cells. Pancreatic organoid cultures may help to
overcome shortcomings of the current models, however the ion secretory
processes in pancreatic organoids-and therefore their physiological
relevance or their utility in disease modeling-are not known. To answer
these questions, we provide side-by-side comparison of gene expression,
morphology, and function of epithelial cells in primary isolated
pancreatic ducts and organoids. We used mouse pancreatic ductal
fragments for experiments or were grown in Matrigel to obtain organoid
cultures. Using PCR analysis~we showed that gene expression of ion
channels and transporters remarkably overlap in primary ductal cells and
organoids. Morphological analysis with scanning electron microscopy
revealed that pancreatic organoids form polarized monolayers with brush
border on the apical membrane. Whereas the expression and localization
of key proteins involved in ductal secretion (cystic fibrosis
transmembrane conductance regulator, Na+/H+ exchanger 1 and electrogenic
Na+/HCO3- cotransporter 1) are equivalent to the primary ductal
fragments. Measurements of intracellular pH and Cl- levels revealed no
significant difference in the activities of the apical Cl-/HCO3-
exchange, or in the basolateral Na+ dependent HCO3- uptake. In summary
we found that ion transport activities in the mouse pancreatic organoids
are remarkably similar to those observed in freshly isolated primary
ductal fragments. These results suggest that organoids can be suitable
and robust model to study pancreatic ductal epithelial ion transport in
health and diseases and facilitate drug development for secretory
pancreatic disorders like cystic fibrosis, or chronic pancreatitis.

doi: \url{https://doi.org/10.1038/s41374-019-0300-3}

{}

{}

\begin{center}\rule{0.5\linewidth}{\linethickness}\end{center}

\begin{center}\rule{0.5\linewidth}{\linethickness}\end{center}

 - \textbf{Prognostic significance of neutrophil-lymphocyte ratio in
resectable~pancreatic neuroendocrine tumors with special reference~to
tumor-associated macrophages}

\emph{Pancreatology : official journal of the International Association
of Pancreatology (IAP) \ldots{} {[}et al.{]} 2019 Aug;():}

PubMed: \url{https://www.ncbi.nlm.nih.gov/pubmed/?term=31409525}

BACKGROUND: Recent studies have shown that the systemic inflammatory
response induced by cancer leads to cancer progression.
Neutrophil-to-lymphocyte ratio (NLR) is the most reliable marker to
detect systemic inflammation. In this study, we investigated the
significance of NLR in patients with well-differentiated pancreatic
neuroendocrine tumors (PanNETs) according to the World Health
Organization 2017 classification. METHODS: We retrospectively collected
data for patients with PanNET who underwent pancreatic resection with
curative intent between January 2008 and December 2017\,at six
institutions. Clinicopathological factors, recurrence, and
immunohistochemical staining of tumor-associated macrophages (TAMs) were
analyzed in a total of 55 patients in this study. RESULTS: High NLR
(\textgreater{}3.41) in patients was significantly associated with
higher white blood cell count, higher Ki-67 index, higher mitotic count,
higher grade, higher incidence of lymph node metastasis, higher
incidence of lymphatic and neural invasion, massive blood loss, and a
large number of CD163-expressing TAMs. Recurrence-free survival of
patients with high NLR was significantly poorer than that of patients
with low NLR. Multivariate analysis identified high NLR, NET Grade 2
(G2) or Grade 3 (G3), and synchronous hepatic resection as independent
risk factors for recurrence after curative resection. CONCLUSIONS: NLR
is a promising predictor of recurrence after pancreatectomy that needs
to be further investigated and that accumulation of TAMs in the tumor
could be one of the causes of NLR elevation.

doi: \url{https://doi.org/10.1016/j.pan.2019.08.003}

{}

{}

\begin{center}\rule{0.5\linewidth}{\linethickness}\end{center}

\begin{center}\rule{0.5\linewidth}{\linethickness}\end{center}

 - \textbf{Dissecting the presence of malignant squamous cells in
pancreatic cytopathology: A case series}

\emph{Diagnostic cytopathology 2019 Aug;():}

PubMed: \url{https://www.ncbi.nlm.nih.gov/pubmed/?term=31407529}

The presence of malignant squamous cells in pancreatic cytopathology is
a rare phenomenon that results either from a primary or a metastatic
process. Pancreatic adenosquamous carcinoma (PASC) represents the most
common variant of pancreatic ductal adenocarcinoma and is associated
with a dismal prognosis. Within the period of 2013-2018, the archives of
``Hygeia and Mitera Hospital'' were searched for pancreatic
cytopathology-related diagnoses that included the interpretation of
``malignant squamous cells present.'' All fine needle aspirations (FNAs)
of pancreatic lesions, including liver metastases in patients with known
pancreatic primaries, were retrieved along with their relevant clinical
information. Five pancreatic and two liver FNAs acquired from a total of
six patients were reexamined. None of these patients had any documented
history of primary squamous malignancy elsewhere. All pancreatic and one
of the two liver FNAs showed malignant squamous cells, identified based
on either morphology or immunochemistry. The other liver FNA represented
a metastatic deposit which comprised of only a glandular component,
whereas the associated pancreatic FNA exhibited both squamous and
glandular counterparts. Most cases characteristically showed necrosis
and keratinization. Of interest, two cases revealed the presence of
tumor-associated giant cells. In conclusion, the presence of malignant
squamous cells in pancreatic FNAs could mean the presence of PASC,
especially when there is no documented history of a primary malignancy
and a complete clinical and imaging workup has been performed.
Immunochemistry on cell block material could help to confirm squamous
differentiation in the absence of overt keratinization.

doi: \url{https://doi.org/10.1002/dc.24302}

{}

{}

\begin{center}\rule{0.5\linewidth}{\linethickness}\end{center}

\begin{center}\rule{0.5\linewidth}{\linethickness}\end{center}

 - \textbf{Laparoscopic Suprapancreatic Lymph Node Dissection Using a
Systematic Mesogastric Excision Concept for Gastric Cancer}

\emph{Annals of surgical oncology 2019 Aug;():}

PubMed: \url{https://www.ncbi.nlm.nih.gov/pubmed/?term=31407184}

BACKGROUND: Gastrointestinal cancer surgery requires en bloc removal of
the primary tumor and organ-specific mesentery1,2. However, this
surgical concept for gastric cancer has not yet been applied because of
the morphological complexity of the mesenteries of the stomach. Lymph
node dissection in gastric cancer surgery can be roughly performed into
three regions: lesser curvature, grater curvature, and suprapancreatic
region. In this video, we introduced laparoscopic lymphadenectomy in the
suprapancreatic region using a systematic mesogastric excision (SME),
which has been reported as a concept to perform en bloc resection3.
METHODS: This procedure was divided into three steps. First,
mesenterization of the mesogastrium was performed by dissecting the
embryological planes, and the mesogastrium was dissected from the
retroperitoneal surface (Fig.~1a). Second, soft tissue, including the
lymph node, was separated from the pancreas and the splenic artery by
tracing the inner dissectable layer (Fig.~1b). Finally, the
tumor-specific mesentery was transected according to the extent of the
lymphadenectomy (Fig.~1c). Fig.~1 Intraoperative findings during the
stepwise procedure in dissecting the lymph node in the suprapancreatic
region. The red broken line indicates the surgical outline. a The
mesogastrium is dissected from the retroperitoneal tissue. b The
mesogastrium is separated from the pancreas and splenic artery. c The
mesogastric transection line is determined on the basis of the extent of
the lymphadenectomy. Inf. phrenic a. inferior phrenic artery; PGA
posterior gastric artery; Post. epiploic a. posterior epiploic artery;
RV renal vein; SA splenic artery; SV splenic vein RESULTS: Between
January 2017 and December 2017, six patients underwent laparoscopic
distal gastrectomy with D2 lymphadenectomy using SME. The median time
required to complete the suprapancreatic lymphadenectomy was 48~min. No
patient underwent conversion to open surgery or experienced
intraoperative complications. CONCLUSIONS: We believe that this
laparoscopic suprapancreatic lymphadenectomy using SME takes advantage
of the surgical anatomy and achieves en bloc removal of the primary
tumor and gastric mesentery. This series is a proof of concept that this
procedure can be performed in a timely manner and is feasible.

doi: \url{https://doi.org/10.1245/s10434-019-07700-5}

{}

{}

\begin{center}\rule{0.5\linewidth}{\linethickness}\end{center}

\begin{center}\rule{0.5\linewidth}{\linethickness}\end{center}

 - \textbf{Extended Laparoscopic Central Pancreatectomy with Clamping of
the Mesentericoportal Vein and Resection of the Splenic Vessels for a
Large Solid Pseudopapillary Tumor}

\emph{Annals of surgical oncology 2019 Aug;():}

PubMed: \url{https://www.ncbi.nlm.nih.gov/pubmed/?term=31407182}

BACKGROUND: Solid pseudopapillary tumors (SPPTs) are low malignant
potential entities found mainly in young females.1,2 Pancreatectomy
without tumor rupture is the treatment of choice, and the laparoscopic
approach is indicated.3,4 Limited pancreatectomy is possible due to the
low risk of malignancy (\textless{} 10\%) based on the low risk of lymph
node invasion or true vascular invasion.1,2 Centrally located large
SPPTs can be treated by extended central pancreatectomy with or without
vascular resection to avoid pancreatoduodenectomy or distal
pancreatectomy. METHODS: A 24-year-old woman was admitted with abdominal
pain. A 6-cm SPPT was discovered at the neck-body junction in close
contact with the anterior aspect of the mesentericoportal vein (MPV) and
the splenic vessels, with signs of segmental portal hypertension. To
avoid an extended pancreatectomy for this young patient, an extended
central pancreatectomy was performed, with resection of the splenic
vessels, and the MPV was freed from the tumor under clamping for 10~min,
with no need for vascular reconstruction. The duration of the surgery
was 260~min, with 200~ml of blood loss and no transfusion. RESULTS: The
woman's postoperative course was uneventful, with a hospital stay of
16~days. Histology confirmed the diagnosis of a 6-cm SPPT tumor (R0 and
N0). The patient was asymptomatic 1~year later, with no tumor recurrence
and no pancreatic insufficiency. Between 2011 and 2018 the authors
performed 72 laparoscopic central pancreatectomies, with SPPT performed
for 13 patients (18\%). Laparoscopic central pancreatectomy was extended
(n = 5) or standard (n = 8) with no conversion, no recurrence, and no
pancreatic insufficiency. CONCLUSION: An SPPT tumor is a good indication
for the laparoscopic approach because this entity is found in young
patients with a low risk of malignancy. Large centrally located tumors
can be treated by extended central pancreatectomy to avoid a large
pancreatectomy with greater early and long-term disadvantages.

doi: \url{https://doi.org/10.1245/s10434-019-07689-x}

{}

{}

\begin{center}\rule{0.5\linewidth}{\linethickness}\end{center}

\begin{center}\rule{0.5\linewidth}{\linethickness}\end{center}

 - \textbf{Robotic Extended Right Hemicolectomy with Complete Mesocolic
Excision and D3 Lymph Node Dissection}

\emph{Annals of surgical oncology 2019 Aug;():}

PubMed: \url{https://www.ncbi.nlm.nih.gov/pubmed/?term=31407176}

BACKGROUND: Recent studies have shown the benefits of complete mesocolic
excision and extended lymphadenectomy (D3 lymph node dissection) in
patients with colon cancer.1-3 METHODS: We present the case of a
62-year-old male with hepatic flexure adenocarcinoma. No metastatic
disease was identified by computed tomography. A robot-assisted extended
right hemicolectomy with complete mesocolic excision, D3 lymph node
dissection, and resection of the mesentery with intact visceral
peritoneum was performed. RESULTS: The trocars are placed in the right
lower (8~mm), lower midline (8~mm), and left upper (12~mm) quadrants.
The camera port is placed superior to the umbilicus, and the assistant
port is placed in the left lower quadrant. The robotic right lower port
is used to place the cecum on tension in order to outline the ileocolic
pedicle. The assistant retracts the transverse colon cephalad to outline
the superior mesenteric artery and vein. Using two robotic arms, the
surgeon begins dissection over the superior mesenteric vein inferior to
the ileocolic pedicle. Cephalad dissection along the superior mesenteric
vein proceeds with reflection of the mesentery and D3 lymph nodes
laterally to allow en bloc resection. The ileocolic and middle colic
vessels are identified, ligated and divided at their origins. The plane
is then developed between the right colon mesentery and the
retroperitoneum, including Gerota's fascia, duodenum, and head of the
pancreas, in a medial-to-lateral fashion, with care taken to ensure an
intact visceral peritoneum is maintained. The proximal transverse colon,
hepatic flexure, and ascending colon are mobilized by taking down
lateral attachments. The intervening mesentery is transected, and
perfusion is assessed with indocyanine green fluorescence imaging. An
intracorporeal, isoperistaltic, side-to-side anastomosis is performed
using the 45-mm robotic stapler. The common enterotomy is sewn closed in
two layers. Pathology showed T3N0 adenocarcinoma with all negative
margins. CONCLUSION: Extended right hemicolectomy with complete
mesocolic excision and D3 lymph node dissection is facilitated by a
robotic approach, which improves visualization and instrument dexterity.

doi: \url{https://doi.org/10.1245/s10434-019-07692-2}

{}

{}

\begin{center}\rule{0.5\linewidth}{\linethickness}\end{center}

\begin{center}\rule{0.5\linewidth}{\linethickness}\end{center}

 - \textbf{Short-Term Outcomes of Laparoscopic and Open Total
Gastrectomy for Gastric Cancer: A Nationwide Retrospective Cohort
Analysis}

\emph{Annals of surgical oncology 2019 Aug;():}

PubMed: \url{https://www.ncbi.nlm.nih.gov/pubmed/?term=31407172}

BACKGROUND: Laparoscopic total gastrectomy is gradually gaining
popularity; however, previous studies have produced conflicting results
regarding the safety and advantages of the procedure, partly because of
small sample sizes. The purpose of this study was to compare short-term
outcomes between laparoscopic and open total gastrectomy for gastric
cancer. METHODS: We analyzed data for patients undergoing laparoscopic
or open total gastrectomy for clinical stage I-III gastric cancer from
July 2010 to March 2017, using a Japanese nationwide inpatient database.
We performed propensity-matched analyses to compare in-hospital
mortality, morbidity, duration of anesthesia, time to first oral intake,
and length of postoperative stay between the two groups. RESULTS: Among
58,689 eligible patients, propensity-score matching created 12,229
pairs. Laparoscopic total gastrectomy was associated with higher
incidences of anastomotic leakage (2.9\% vs.~1.7\%, p \textless{} 0.001)
and stenosis (0.9\% vs.~0.6\%, p = 0.02), lower incidences of pancreatic
injury (1.4\% vs.~1.8\%, p = 0.01), endoscopic hemostasis (0.9\%
vs.~1.7\%, p \textless{} 0.001), blood transfusion (9.9\% vs.~17.7\%,
p \textless{} 0.001) and 30-day readmission, a shorter interval from
surgery to first oral intake (4 vs.~5~days, p \textless{} 0.001),
shorter postoperative hospital stay (14 vs.~15~days,
p \textless{} 0.001), and a longer duration of anesthesia (323
vs.~304~min, p \textless{} 0.001). There was no significant difference
in in-hospital mortality (0.6\% vs.~0.8\%, p = 0.58). CONCLUSIONS:
Laparoscopic total gastrectomy has some advantages over open surgery for
gastric cancer in terms of time to first oral intake and postoperative
length of stay, but the incidence of anastomotic leakage was higher than
that of open total gastrectomy.

doi: \url{https://doi.org/10.1245/s10434-019-07688-y}

{}

{}

\begin{center}\rule{0.5\linewidth}{\linethickness}\end{center}

\begin{center}\rule{0.5\linewidth}{\linethickness}\end{center}

 - \textbf{Pancreas FNA}

\emph{Cytopathology : official journal of the British Society for
Clinical Cytology 2019 Aug;():}

PubMed: \url{https://www.ncbi.nlm.nih.gov/pubmed/?term=31400170}

67-year-old lady diagnosed with right breast invasive ductal carcinoma
with axillary node involvement. Staging FDG PET identified a small
enhancing 1.2cm `indeterminate' nodule within pancreatic tail, which CT
pancreas with contrast revealed to be hypervascular. This article is
protected by copyright. All rights reserved.

doi: \url{https://doi.org/10.1111/cyt.12767}

{}

{}

\begin{center}\rule{0.5\linewidth}{\linethickness}\end{center}

\begin{center}\rule{0.5\linewidth}{\linethickness}\end{center}

 - \textbf{Comparison of Tissue and Blood Concentrations of Oxaliplatin
Administrated by Different Modalities of Intraperitoneal Chemotherapy}

\emph{Annals of surgical oncology 2019 Aug;():}

PubMed: \url{https://www.ncbi.nlm.nih.gov/pubmed/?term=31399820}

BACKGROUND: Pressurized intraperitoneal aerosol chemotherapy (PIPAC) is
a new technology for delivering intraperitoneal chemotherapy. It is
generally assumed that with PIPAC, the ratio of peritoneal to systemic
drug concentration is superior to liquid hyperthermic intraperitoneal
chemotherapy (HIPEC). To date, no direct comparative data are available
supporting such an assumption. MATERIALS AND METHODS: Twelve 65-day-old
pigs were randomly separated into three groups of four pigs each, all of
which received intraperitoneal chemotherapy using the following
administration methods: PIPAC with oxaliplatin 92~mg in 150~ml dextrose
5\% (Group 1); PIPAC with electrostatic aerosol precipitation (ePIPAC;
Group 2); or laparoscopic HIPEC (L-HIPEC) with oxaliplatin 400~mg in 4~L
dextrose 5\% at 42~°C (Group 3). Serial blood and peritoneal tissue
concentrations of oxaliplatin were determined by spectrometry. RESULTS:
In all three groups, the maximum concentration of oxaliplatin in blood
was detected 50-60~min after onset of the chemotherapy experiments, with
no significant differences among the three groups (p = 0.7994). Blood
oxaliplatin concentrations (0-30~min) were significantly higher in the
L-HIPEC group compared with the ePIPAC group (p \textless{} 0.05). No
difference was found for the overall systemic oxaliplatin absorption
(area under the curve). Overall concentrations in the peritoneum were
not different among the three groups (p = 0.4725), but were
significantly higher in the visceral peritoneum in the PIPAC group
(p = 0.0242). CONCLUSIONS: Blood and tissue concentrations were
comparable between all groups; however, depending on the intraperitoneal
area examined and the time points of drug delivery, the concentrations
differed significantly between the three groups.

doi: \url{https://doi.org/10.1245/s10434-019-07695-z}

{}

{}

\begin{center}\rule{0.5\linewidth}{\linethickness}\end{center}

\begin{center}\rule{0.5\linewidth}{\linethickness}\end{center}

 - \textbf{ARF6 and AMAP1 are major targets of KRAS and TP53 mutations
to promote invasion, PD-L1 dynamics, and immune evasion of pancreatic
cancer}

\emph{Proceedings of the National Academy of Sciences of the United
States of America 2019 Aug;():}

PubMed: \url{https://www.ncbi.nlm.nih.gov/pubmed/?term=31399545}

Although KRAS and TP53 mutations are major drivers of pancreatic ductal
adenocarcinoma (PDAC), the incurable nature of this cancer still remains
largely elusive. ARF6 and its effector AMAP1 are often overexpressed in
different cancers and regulate the intracellular dynamics of integrins
and E-cadherin, thus promoting tumor invasion and metastasis when ARF6
is activated. Here we show that the ARF6-AMAP1 pathway is a major target
by which KRAS and TP53 cooperatively promote malignancy. KRAS was
identified to promote eIF4A-dependent ARF6 mRNA translation, which
contains a quadruplex structure at its 5'-untranslated region, by
inducing TEAD3 and ETV4 to suppress PDCD4; and also eIF4E-dependent
AMAP1 mRNA translation, which contains a 5'-terminal
oligopyrimidine-like sequence, via up-regulating mTORC1. TP53
facilitated ARF6 activation by platelet-derived growth factor (PDGF),
via its known function to promote the expression of PDGF receptor β
(PDGFRβ) and enzymes of the mevalonate pathway (MVP). The ARF6-AMAP1
pathway was moreover essential for PDGF-driven recycling of PD-L1, in
which KRAS, TP53, eIF4A/4E-dependent translation, mTOR, and MVP were all
integral. We moreover demonstrated that the mouse PDAC model KPC cells,
bearing KRAS/TP53 mutations, express ARF6 and AMAP1 at high levels and
that the ARF6-based pathway is closely associated with immune evasion of
KPC cells. Expression of ARF6 pathway components statistically
correlated with poor patient outcomes. Thus, the cooperation among
eIF4A/4E-dependent mRNA translation and MVP has emerged as a link by
which pancreatic driver mutations may promote tumor cell motility, PD-L1
dynamics, and immune evasion, via empowering the ARF6-based pathway and
its activation by external ligands.

doi: \url{https://doi.org/10.1073/pnas.1901765116}

{}

{}

\begin{center}\rule{0.5\linewidth}{\linethickness}\end{center}

\begin{center}\rule{0.5\linewidth}{\linethickness}\end{center}

 - \textbf{Second harmonic generation detection of Ras conformational
changes and discovery of a small molecule binder}

\emph{Proceedings of the National Academy of Sciences of the United
States of America 2019 Aug;():}

PubMed: \url{https://www.ncbi.nlm.nih.gov/pubmed/?term=31399543}

Second harmonic generation (SHG) is an emergent biophysical method that
sensitively measures real-time conformational change of biomolecules in
the presence of biological ligands and small molecules. This study
describes the successful implementation of SHG as a primary screening
platform to identify fragment ligands to oncogenic Kirsten rat sarcoma
(KRas). KRas is the most frequently mutated driver of pancreatic, colon,
and lung cancers; however, there are few well-characterized small
molecule ligands due to a lack of deep binding pockets. Using SHG, we
identified a fragment binder to KRasG12D and used 1H 15N transverse
relaxation optimized spectroscopy (TROSY) heteronuclear single-quantum
coherence (HSQC) NMR to characterize its binding site as a pocket
adjacent to the switch 2 region. The unique sensitivity of SHG furthered
our study by revealing distinct conformations induced by our hit
fragment compared with 4,6-dichloro-2-methyl-3-aminoethyl-indole (DCAI),
a Ras ligand previously described to bind the same pocket. This study
highlights SHG as a high-throughput screening platform that reveals
structural insights in addition to ligand binding.

doi: \url{https://doi.org/10.1073/pnas.1905516116}

{}

{}

\begin{center}\rule{0.5\linewidth}{\linethickness}\end{center}

\begin{center}\rule{0.5\linewidth}{\linethickness}\end{center}

 - \textbf{Tumor Microbiome Diversity and Composition Influence
Pancreatic Cancer Outcomes}

\emph{Cell 2019 Aug;178(4):795-806.e12}

PubMed: \url{https://www.ncbi.nlm.nih.gov/pubmed/?term=31398337}

Most patients diagnosed with resected pancreatic adenocarcinoma (PDAC)
survive less than 5 years, but a minor subset survives longer. Here, we
dissect the role of the tumor microbiota and the immune system in
influencing long-term survival. Using 16S rRNA gene sequencing, we
analyzed the tumor microbiome composition in PDAC patients with
short-term survival (STS) and long-term survival (LTS). We found higher
alpha-diversity in the tumor microbiome of LTS patients and identified
an intra-tumoral microbiome signature
(Pseudoxanthomonas-Streptomyces-Saccharopolyspora-Bacillus clausii)
highly predictive of long-term survivorship in both discovery and
validation cohorts. Through human-into-mice fecal microbiota
transplantation (FMT) experiments from STS, LTS, or control donors, we
were able to differentially modulate the tumor microbiome and affect
tumor growth as well as tumor immune infiltration. Our study
demonstrates that PDAC microbiome composition, which cross-talks to the
gut microbiome, influences the host immune response and natural history
of the disease.

doi: \url{https://doi.org/10.1016/j.cell.2019.07.008}

{}

{}

\begin{center}\rule{0.5\linewidth}{\linethickness}\end{center}

\begin{center}\rule{0.5\linewidth}{\linethickness}\end{center}

 - \textbf{Variation in use of open and laparoscopic distal
pancreatectomy and associated outcome metrics in a universal health care
system}

\emph{Pancreatology : official journal of the International Association
of Pancreatology (IAP) \ldots{} {[}et al.{]} 2019 Aug;():}

PubMed: \url{https://www.ncbi.nlm.nih.gov/pubmed/?term=31395453}

BACKGROUND: Universal health care (UHC) should ensure equal access to
and use of surgery, but few studies have explored variation in UHC
systems. The objective was to describe practice of distal pancreatectomy
in Norway covered exclusively by an UHC. METHODS: Data on all patients
undergoing distal pancreatectomy from the Norwegian Patient Register
over a 5-year period. Age- and gender-adjusted population-based
resection rates (adj. per million/yr) for distal pancreatectomy were
analysed across 4 regions and outcomes related to splenic salvage rate,
hospital stay, reoperation, readmissions and 90-day mortality risk
between regions. Risk is reported as odds ratio (OR) with 95\%
confidence interval (c.i.). RESULTS: Regional difference exist in terms
of absolute numbers, with the majority of procedures done in one region
(n~=~331; 59.7\%). Regional variation persisted for age- and
gender-adjusted population-rates, with highest rate at 23.8/million/yr
and lowest rate at 13.5/mill/yr (for a 176\% relative difference; or an
absolute difference of~+10.3 resections/million/yr). Overall, a lapDP
instead of an open DP was 3.5 times more likely in SouthEast compared to
all other regions combined (lapDP rate: 83\% vrs 24\%, respectively; OR
15.4, 95\% c.i. 10.1-23.5; P~\textless{}~0.001). The splenic salvage
rate was lower in SouthEast (19.9\%) compared to all other regions
(average 26.5\%; highest in Central-region at 37.0\%; P~=~0.010 for
trend). Controlled for other factors in multivariate regression,
`region' of surgery remained significantly associated with laparoscopic
access. CONCLUSION: Despite a universal health care system, considerable
variation exists in resection rates, use of laparoscopy and splenic
salvage rates across regions.

doi: \url{https://doi.org/10.1016/j.pan.2019.07.047}

{}

{}

\begin{center}\rule{0.5\linewidth}{\linethickness}\end{center}

\begin{center}\rule{0.5\linewidth}{\linethickness}\end{center}

 - \textbf{Comparison of overall survival and perioperative outcomes of
laparoscopic pancreaticoduodenectomy and open pancreaticoduodenectomy
for pancreatic ductal adenocarcinoma: a systematic review and
meta-analysis}

\emph{BMC cancer 2019 Aug;19(1):781}

PubMed: \url{https://www.ncbi.nlm.nih.gov/pubmed/?term=31391085}

BACKGROUND: The aim of this study was to compare the oncological
outcomes and clinical efficacy of laparoscopic pancreaticoduodenectomy
(LPD) and open pancreaticoduodenectomy (OPD) in patients with pancreatic
ductal adenocarcinoma (PDAC). METHODS: We systematically searched
PubMed, EMBASE, Web of Science, ClinicalTrials.gov and the Cochrane
Central Register for studies published between May 1998 and May 2018.
The included studies compared LPD and OPD for the treatment of PDAC. The
oncological outcomes and perioperative data were analyzed. RESULTS:
Eight studies involving 15,278 patients were included in our
meta-analysis. No significant difference was found in the 5-year overall
survival (OS) between patients undergoing the two types of surgery (HR:
0.97, 95\% CI 0.82-1.15, p~= 0.76). LPD resulted in a higher rate of R0
resection than OPD (OR: 1.16, 95\% CI 0.85-1.57, p~\textgreater{} 0.05).
This study showed that compared with OPD, LPD resulted in comparable
rates of postoperative pancreatic fistulas (POPFs) (OR: 1.07, 95\% CI:
0.68-1.68, p~= 0.77) and postoperative hemorrhage (OR: 1.74, 95\% CI
0.96-3.71, p~= 0.07), more harvested lymph nodes (WMD: 1.84, 95\% CI:
0.95-2.72, p~\textless{} 0.05), shorter hospital stays (WMD: -2.45, 95\%
CI: - 3.33- -1.56, p \textless{} 0.05), and less estimated blood loss
(WMD: -374.30, 95\% CI: - 513.06- -235.54, p \textless{} 0.05).
CONCLUSIONS: LPD is equivalent to OPD with respect to 5-year OS and
results in better perioperative clinical outcomes for patients with
PDAC.

doi: \url{https://doi.org/10.1186/s12885-019-6001-x}

{}

{}

\begin{center}\rule{0.5\linewidth}{\linethickness}\end{center}

\begin{center}\rule{0.5\linewidth}{\linethickness}\end{center}

 - \textbf{A 12-year trend analysis of the incidence of gastrointestinal
cancers in East Azerbaijan: last updated results of an ongoing
population-based cancer registry}

\emph{BMC cancer 2019 Aug;19(1):782}

PubMed: \url{https://www.ncbi.nlm.nih.gov/pubmed/?term=31391032}

BACKGROUND: The most recent results of Global Cancer Statistics
indicated that gastrointestinal cancers, including gastric, colorectal,
esophageal, and liver cancers, are among the most commonly diagnosed
cancers worldwide. Previous reports from cancer registries in East
Azerbaijan have shown that there is a high incidence of gastrointestinal
cancer in this region, so we performed a trend analysis to determine the
pattern of change over the last decade. METHODS: In total, 12 years of
cancer registry data were collected from different sources in East
Azerbaijan, and a data quality check was performed to ensure clean data.
Using the 2000 World Health Organization standard population, we then
generated age-standardized incidence rates (ASRs) for different cancers,
and for each year from 1383 to 1394 of the Persian calendar (i.e., 19
March 2004 to 20 March 2015). Annual percent changes (APCs) and Average
annual percent changes (AAPCs) in the ASRs for esophageal, gastric,
small intestine, colorectal, anal, liver, gallbladder, and pancreatic
cancers were calculated using Joinpoint Software (Version 4.5.0.1, June
2017). RESULTS: An increase in most types of cancer was observed during
the study period. The ASR for colorectal cancer increased from 2.9 to
13.6 per 100,000 women (APC, 9.7\%) and from 2.2 to 17.8 per 100,000 men
(APC, 10.2\%). The ASR for gastric cancer showed a slight increasing
trend from 10.5 to 13.5 per 100,000 women (APC, 1.3\%) and from 3.1 to
29.9 per 100,000 men (APC, 3.2\%). However, trend analysis showed a
decreasing pattern for the ASR of esophageal cancer in both genders
(APC,- 3\%), with APCs of - 1.1\% in females and - 0.4\% in males.
CONCLUSIONS: The latest results of the East Azerbaijan Population-Based
Cancer Registry indicate that gastrointestinal cancers remain common,
with significant increasing trends in their ASRs. Improved screening and
early detection are needed in this region.

doi: \url{https://doi.org/10.1186/s12885-019-6008-3}

{}

{}

\begin{center}\rule{0.5\linewidth}{\linethickness}\end{center}

\begin{center}\rule{0.5\linewidth}{\linethickness}\end{center}

 - \textbf{The prognostic value of lncRNA SNHG1 in cancer patients: a
meta-analysis}

\emph{BMC cancer 2019 Aug;19(1):780}

PubMed: \url{https://www.ncbi.nlm.nih.gov/pubmed/?term=31391030}

BACKGROUND: Increasing evidence revealed that high expression level of
lncRNA SNHG1 was associated with the unfavorable prognosis of cancer and
maybe used as a valuable biomarker for cancer patients. The present
meta-\textgreater{}analysis is to analyze existing data to reveal
potential clinical application of SNHG1 on cancer prognosis and tumor
progression. All of the included studies were collected through a
variety of retrieval strategies. And the articles were qualified by
MOOSE and PRISMA checklists. METHODS: Up to Mar 20, 2018, literature
collection was performed by comprehensive search through electronic
databases, including the Cochrane library, PubMed, Embase, Web of
science, Springer, Science direct, and three Chinese databases: CNKI,
Weipu, and Wanfang. We analyzed 14 studies that met the criteria, and
concluded that the increased SHNG1 level was correlated with poor OS and
tumor progression. RESULTS: The combined results indicated that elevated
SNHG1 expression level was significantly associated with poor OS
(HR = 2.06, 95\% CI: 1.69-2.52, P~\textless{} 0.01) and PFS (HR = 2.78,
95\% CI: 1.69-4.55, P~\textless{} 0.01) in various cancers. Moreover,
the promoted SNHG1 expression was also associated with tumor progression
((III/IV vs.~I/II: HR = 1.89, 95\% CI: 1.53-2.34, P~\textless{} 0.01).
In stratified analyses, a significantly unfavorable association of
elevated lncRNA SNHG1 and OS was observed in both digestive system
(HR = 2.04, 95\% CI: 1.56-2.68, P~\textless{} 0.01) and non-digestive
system (HR = 2.09, 95\% CI: 1.55-2.83, P~\textless{} 0.01) cancer
patients. CONCLUSIONS: The present analysis indicated that the increased
SNHG1 is associated with poor OS in patients with general tumors and may
be served as a useful prognostic biomarker.

doi: \url{https://doi.org/10.1186/s12885-019-5987-4}

{}

{}

\begin{center}\rule{0.5\linewidth}{\linethickness}\end{center}

\begin{center}\rule{0.5\linewidth}{\linethickness}\end{center}

 - \textbf{Dynamic serum alkaline phosphatase is an indicator of overall
survival in pancreatic cancer}

\emph{BMC cancer 2019 Aug;19(1):785}

PubMed: \url{https://www.ncbi.nlm.nih.gov/pubmed/?term=31391026}

BACKGROUND: The prognostic role of serum alkaline phosphatase (ALP) has
been found in several kinds of solid malignant tumor, but has never been
extensively discussed in pancreatic cancer, especially through the
application of dynamic survival model which incorporates the varying
nature of ALP measurements. METHODS: We conducted a retrospective study
which successfully collected 551 histopathologically confirmed
pancreatic ductal adenocarcinoma (PDAC) patients from a cancer
specialized hospital in southwest China. The association between variant
ALP which measured during the whole survival period and the overall
survival (OS) of PDAC patients was evaluated by using dynamic
Anderson-Gill (AG) model. Exhaustive sensitivity analysis was performed
by adopting continuous cut-offs of ALP. RESULTS: After adjusted for
possible confounding of serum albumin, total bilirubin and leukocyte
counts, AG model revealed that, serum ALP during the survival period was
nonlinearly associated with the OS of PDAC: for resected patients,
compared with those whose ALP results ranged within the first quartile
(\textless{}P25), patients whose ALP measurements belonged to the second
(P25-P50), the third (P50-P75), and the forth (\textgreater{}P75)
quartiles were observed 1.14 (95\% CI: 0.29-4.56), 3.93 (95\% CI:
1.23-12.60), 3.87 (95\% CI: 1.32-11.36) folds of death hazard; whereas
in un-resected PDAC patients, the hazard ratios (HRs) were 1.15 (95\%
CI: 0.79-1.68), 1.92 (95\% CI: 1.32-2.78), and 1.97 (95\% CI:
1.30-2.98), respectively. Sensitivity analysis revealed that, for both
resected and un-resected patients, the results of AG model were robust
with regard to various cut-offs of ALP, and an increased ALP was in
general associated with significantly increased hazard of death.
CONCLUSION: Serum ALP during the survival period was significantly
associated with the OS of PDAC patients, especially for resected early
stage PDAC patients. Future studies with expanded sample size and
refined prospective design should be implemented to corroborate our
major findings. Besides, the underlying mechanism for this possible
hazardous role of ALP should also be investigated.

doi: \url{https://doi.org/10.1186/s12885-019-6004-7}

{}

{}

\begin{center}\rule{0.5\linewidth}{\linethickness}\end{center}

\begin{center}\rule{0.5\linewidth}{\linethickness}\end{center}

 - \textbf{Corrigendum to ``Cystic pancreatic neuroendocrine tumors: A
more favorable lesion?'' {[}Pancreatology 19 (2) (March 2019)
372-376{]}}

\emph{Pancreatology : official journal of the International Association
of Pancreatology (IAP) \ldots{} {[}et al.{]} 2019 Aug;():}

PubMed: \url{https://www.ncbi.nlm.nih.gov/pubmed/?term=31387834}

doi: \url{https://doi.org/10.1016/j.pan.2019.07.043}

{}

{}

\begin{center}\rule{0.5\linewidth}{\linethickness}\end{center}

\begin{center}\rule{0.5\linewidth}{\linethickness}\end{center}

 - \textbf{Classification of Complication Clusters Might Vary in
Different Populations With Chronic Pancreatitis}

\emph{The American journal of gastroenterology 2019
Aug;114(8):1351-1352}

PubMed: \url{https://www.ncbi.nlm.nih.gov/pubmed/?term=31386633}

doi: \url{https://doi.org/10.14309/ajg.0000000000000292}

{}

{}

\begin{center}\rule{0.5\linewidth}{\linethickness}\end{center}

\begin{center}\rule{0.5\linewidth}{\linethickness}\end{center}

 - \textbf{Sclerosing epithelioid mesenchymal neoplasm of the
pancreas~-~a proposed new entity}

\emph{Modern pathology : an official journal of the United States and
Canadian Academy of Pathology, Inc 2019 Aug;():}

PubMed: \url{https://www.ncbi.nlm.nih.gov/pubmed/?term=31383964}

We have encountered pancreatic tumors with unique histologic features,
which do not conform to any of the known tumors of the pancreas or other
anatomical sites. We aimed to define their clinicopathologic features
and whether they are characterized by recurrent molecular signatures.
Eight cases were identified; studied histologically and by
immunohistochemistry. Selected cases were also subjected to whole-exome
sequencing (WES; n = 4), RNA-sequencing (n = 6), Archer FusionPlex assay
(n = 5), methylation profiling using the Illumina MethylationEPIC (850k)
array platform (n = 6), and TERT promoter sequencing (n = 5). Six
neoplasms occurred in females. The mean age was 43 years (range: 26-75).
Five occurred in the head/neck of the pancreas. All patients were
treated surgically; none received neoadjuvant/adjuvant therapy. All
patients are free of disease after 53 months of median follow-up (range:
8-94). The tumors were well-circumscribed, and the median size was
1.8 cm (range: 1.3-5.8). Microscopically, the unencapsulated tumors had
a geographic pattern of epithelioid cell nests alternating with spindle
cell fascicles. Some areas showed dense fibrosis, in which enmeshed
tumor cells imparted a slit-like pattern. The predominant epithelioid
cells had scant cytoplasm and round-oval nuclei with open chromatin. The
spindle cells displayed irregular, hyperchromatic nuclei. Mitoses were
rare. No lymph node metastases were identified. All tumors were positive
for vimentin, CD99 and cytokeratin (patchy), while negative for markers
of solid pseudopapillary neoplasm, neuroendocrine, acinar,
myogenic/rhabdoid, vascular, melanocytic, or lymphoid differentiation,
gastrointestinal stromal tumor as well as MUC4. Whole-exome sequencing
revealed no recurrent somatic mutations or amplifications/homozygous
deletions in any known oncogenes or tumor suppressor genes.
RNA-sequencing and the Archer FusionPlex assay did not detect any
recurrent likely pathogenic gene fusions. Single sample gene set
enrichment analysis revealed that these tumors display a likely
mesenchymal transcriptomic program. Unsupervised analysis (t-SNE) of
their methylation profiles against a set of different mesenchymal
neoplasms demonstrated a distinct methylation pattern. Here, we describe
pancreatic neoplasms with unique morphologic/immunophenotypic features
and a distinct methylation pattern, along with a lack of abnormalities
in any of key genetic drivers, supporting that these neoplasms represent
a novel entity with an indolent clinical course. Given their mesenchymal
transcriptomic features, we propose the designation of ``sclerosing
epithelioid mesenchymal neoplasm'' of the pancreas.

doi: \url{https://doi.org/10.1038/s41379-019-0334-5}

{}

{}

\begin{center}\rule{0.5\linewidth}{\linethickness}\end{center}

\begin{center}\rule{0.5\linewidth}{\linethickness}\end{center}

 - \textbf{The cost of endoscopic treatment for walled-off pancreatic
necrosis}

\emph{Pancreatology : official journal of the International Association
of Pancreatology (IAP) \ldots{} {[}et al.{]} 2019 Jul;():}

PubMed: \url{https://www.ncbi.nlm.nih.gov/pubmed/?term=31383574}

BACKGROUND: Use of minimally invasive techniques has reduced mortality
in walled-off pancreatic necrosis (WON) but may be costly. The aim of
this study was to evaluate the actual costs associated with the
endoscopic management of patients with WON. METHODS: We included a
retrospective cohort of WON patients treated with endoscopic,
transgastric drainage and necrosectomy (ETDN) during 2013-2014. Costs
were calculated for six sub-areas based on a micro-costing model.
Students T-test and non-parametric analysis of variance were performed
to evaluate costs in relation to disease etiology and outcome. RESULTS:
We included 58 patients (50\% men, median age 57 years). The most common
etiologies were gallstones (57\%) and alcohol (19\%). Nine patients
(16\%) died during admission. The median length of stay was 50 days (IQR
31 days). Eighteen patients (31\%) needed treatment in our intensive
care unit with a median length of stay of 16 days (IQR 31 days). The
mean costs and standard deviation of costs (SD) per patient were:
diagnostic imaging \$2,431 (\$2,301), laboratory tests \$3,579
(\$2,477), blood products \$982 (\$1,734), endoscopic treatment \$3,794
(\$1,777), medicine \$5,440 (\$6,656), and ward cost \$41,260
(\$35,854). The mean total cost was \$57,486 (\$46,739). Post-ERCP
pancreatitis and mortality predicted higher costs. CONCLUSIONS: This
study sheds light on the different costs associated with endoscopic
treatment of WON. As nearly three quarters of the costs are related to
ward care, initiatives aimed at reducing the length of hospital stay may
have a great impact on making endoscopic treatment more cost effective.

doi: \url{https://doi.org/10.1016/j.pan.2019.07.042}

{}

{}

\begin{center}\rule{0.5\linewidth}{\linethickness}\end{center}

\begin{center}\rule{0.5\linewidth}{\linethickness}\end{center}

 - \textbf{Coexisting pancreatic serous cystadenoma and pancreatic
ductal adenocarcinoma: A cytological-pathologic correlation with
literature review}

\emph{Annals of diagnostic pathology 2019 Jul;42():87-91}

PubMed: \url{https://www.ncbi.nlm.nih.gov/pubmed/?term=31382079}

Pancreatic serous cystadenoma (SCA) is a benign neoplastic lesion with a
distinctive gross and microscopic appearance consisting of numerous
thin-walled cysts lined by uniform epithelial cells with clear cytoplasm
and small nuclei. The vast majority of serous cystadenomas are benign.
Pancreatic SCA has rarely been reported in association with other
pancreatic lesions. We present a challenging case in which a cystic and
solid pancreatic mass was identified on imaging studies. FNA was
performed and showed clusters of atypical cells with significant nuclear
pleomorphism (\textgreater{}4:1), disorganized, overlapping nuclei, and
prominent nucleoli. The FNA diagnosis was positive for malignancy,
consistent with adenocarcinoma. The patient underwent neoadjuvant
therapy and pancreaticoduodenectomy. Final pathology showed a serous
cystadenoma associated with small foci of high-grade PanIN. The lack of
invasive adenocarcinoma in the resection specimen was most likely due to
complete response of the tumor to neoadjuvant chemoradiation therapy,
but it is also possible that only high-grade PanIN was present
initially. To our knowledge, this is the first reported case of SCA and
high grade PanIN/PDAC that was assessed by FNA. We discuss the cytologic
differential diagnosis and how to avoid potential pitfalls highlighted
by this case.

doi: \url{https://doi.org/10.1016/j.anndiagpath.2019.07.006}

{}

{}

\begin{center}\rule{0.5\linewidth}{\linethickness}\end{center}

\begin{center}\rule{0.5\linewidth}{\linethickness}\end{center}

 - \textbf{Pancreatic cancer-A disease in need: Optimizing and
integrating supportive care}

\emph{Cancer 2019 Aug;():}

PubMed: \url{https://www.ncbi.nlm.nih.gov/pubmed/?term=31381149}

Pancreatic ductal adenocarcinoma (PDAC) is an aggressive malignancy that
continues to be challenging to treat. PDAC has the lowest 5-year
relative survival rate compared with all other solid tumor malignancies
and is expected to become the second-leading cause of cancer-related
death in the United States by 2030. Given the high mortality, there is
an increasing role for concurrent anticancer and supportive care in the
management of patients with PDAC with the aims of maximizing length of
life, quality of life, and symptom control. Emerging trends in
supportive care that can be integrated into the clinical management of
patients with PDAC include standardized supportive care screening, early
integration of supportive care into routine cancer care, early
implementation of outpatient-based advance care planning, and
utilization of electronic patient-reported outcomes for improved symptom
management and quality of life. The most common symptoms experienced are
nausea, constipation, weight loss, diarrhea, anorexia, and abdominal and
back pain. This review article includes current supportive management
strategies for these and others. Common disease-related complications
include biliary and duodenal obstruction requiring endoscopic procedures
and venous thromboembolic events. Patients with PDAC continue to have a
poor prognosis. Systemic therapy options are able to palliate the high
symptom burden but have a modest impact on overall survival. Early
integration of supportive care can lead to improved outcomes.

doi: \url{https://doi.org/10.1002/cncr.32423}

{}

{}

\begin{center}\rule{0.5\linewidth}{\linethickness}\end{center}

\begin{center}\rule{0.5\linewidth}{\linethickness}\end{center}

 - \textbf{The role of abdominal drainage in pancreatic resection - A
multicenter validation study for early drain removal}

\emph{Pancreatology : official journal of the International Association
of Pancreatology (IAP) \ldots{} {[}et al.{]} 2019 Jul;():}

PubMed: \url{https://www.ncbi.nlm.nih.gov/pubmed/?term=31378583}

BACKGROUND: Abdominal drainage and the timing of drain removal in
patients undergoing pancreatic resection are under debate. Early drain
removal after pancreatic resection has been reported to be safe with a
low risk for clinical relevant postoperative pancreatic fistula
(CR-POPF) when drain amylase on POD1 is\,\textless{}\,5000U/L. The aim
of this study was to validate this algorithm in a large national cohort.
METHODS: Patients registered in the Dutch Pancreatic Cancer Audit
(2014-2016) who underwent pancreatoduodenectomy, distal pancreatectomy
or enucleation were analysed. Data on post-operative drain amylase
levels, drain removal, postoperative pancreatic fistulae were collected.
Univariate and multivariate analysis using a logistic regression model
were performed. The primary outcome measure was grade B/C pancreatic
fistula (CR-POPF). RESULTS: Among 1402 included patients, 433 patients
with a drain fluid amylase level of \textless{}5000U/L on POD1, 7\%
developed a CR-POPF. For patients with an amylase level
\textgreater{}5000U/L the CR-POPF rate was 28\%. When using a cut-off
point of 2000U/L or 1000U/L during POD1-3, the CR-POPF rates were 6\%
and 5\% respectively. For patients with an amylase level of
\textgreater{}2000U/L and \textgreater{}1000UL during POD 1-3 the
CR-POPF rates were 26\% and 22\% respectively (n\,=\,223). Drain removal
on POD4 or thereafter was associated with more complications
(p\,=\,0.004). Drain amylase level was shown to be the most
statistically significant predicting factor for CR-POPF (Wald\,=\,49.7;
p\,\textless{}\,0.001). CONCLUSION: Our data support early drain removal
after pancreatic resection. However, a cut-off of 5000U/L drain amylase
on POD1 was associated with a relatively high CR-POPF rate of 7\%. A
cut-off point of 1000U/L during POD1-3 resulted in 5\% CR-POPF and might
be a safer alternative.

doi: \url{https://doi.org/10.1016/j.pan.2019.07.041}

{}

{}

\begin{center}\rule{0.5\linewidth}{\linethickness}\end{center}

\begin{center}\rule{0.5\linewidth}{\linethickness}\end{center}

 - \textbf{Factors predicting readmission within 30 days of acute
pancreatitis attack: A prospective study}

\emph{Pancreatology : official journal of the International Association
of Pancreatology (IAP) \ldots{} {[}et al.{]} 2019 Jul;():}

PubMed: \url{https://www.ncbi.nlm.nih.gov/pubmed/?term=31378581}

doi: \url{https://doi.org/10.1016/j.pan.2019.07.044}

{}

{}

\begin{center}\rule{0.5\linewidth}{\linethickness}\end{center}

\begin{center}\rule{0.5\linewidth}{\linethickness}\end{center}

 - \textbf{The Impact of Dedicated Cancer Centers on Outcomes Among
Medicare Beneficiaries Undergoing Liver and Pancreatic Cancer Surgery}

\emph{Annals of surgical oncology 2019 Aug;():}

PubMed: \url{https://www.ncbi.nlm.nih.gov/pubmed/?term=31376033}

BACKGROUND: The Alliance of Dedicated Cancer Centers (DCCs) is comprised
of 11 institutions that are exempt from the prospective payment system
utilized by Medicare for hospital reimbursement. OBJECTIVE: The aim of
this study was to compare short- and long-term outcomes of patients
undergoing liver and pancreatic surgery for cancer at DCCs versus
non-DCCs. METHODS: Patients who underwent a liver or pancreatic
operation for a malignant indication between 2013 and 2015 were
identified using the Medicare Inpatient Standard Analytic Files.
Regression analyses and the Kaplan-Meier method were used to assess
short- and long-term outcomes of patients at DCCs versus non-DCCs.
RESULTS: Among 13,256 patients, 7.0\% of patients were treated at a DCC.
Median patient age and complexity of surgical procedures were comparable
among DCCs and non-DCCs (all p \textgreater{} 0.05). Overall
complications (16.5\% vs.~23.6\%), 90-day readmission (26.2\%
vs.~30.2\%), and 90-day mortality (3.0\% vs.~8.7\%) were lower at DCCs
compared with non-DCCs (all p \textless{} 0.001). In addition, long-term
hazards of death among patients undergoing hepatectomy {[}hazard ratio
(HR) 0.64, 95\% confidence interval (CI) 0.54-0.75{]} and pancreatectomy
(HR 0.66, 95\% CI 0.56-0.78) were lower among patients treated at DCCs
(both p \textless{}  0.05). While Medicare payments for patients
undergoing pancreatic surgery (DCC: \$22,200 vs.~non-DCC: \$22,100;
p = 0.772) were comparable among DCC and non-DCC hospitals, Medicare
payments for liver resection at DCCs were 13.9\% lower than non-DCCs
(DCC: \$16,700 vs.~non-DCC: \$19,400; p \textless{} 0.001). CONCLUSIONS:
Patients undergoing hepatopancreatic surgery at DCCs had better short-
and long-term outcomes for the same/lower level of Medicare expenditure
as non-DCC hospitals. DCCs provide higher-value surgical care for
patients undergoing liver and pancreatic cancer operations.

doi: \url{https://doi.org/10.1245/s10434-019-07677-1}

{}

{}

\begin{center}\rule{0.5\linewidth}{\linethickness}\end{center}

\begin{center}\rule{0.5\linewidth}{\linethickness}\end{center}

 - \textbf{NTRK fusion detection across multiple assays and 33,997
cases: diagnostic implications and pitfalls}

\emph{Modern pathology : an official journal of the United States and
Canadian Academy of Pathology, Inc 2019 Aug;():}

PubMed: \url{https://www.ncbi.nlm.nih.gov/pubmed/?term=31375766}

With the FDA approval of larotrectinib, NTRK fusion assessment has
recently become a standard part of management for patients with locally
advanced or metastatic cancers. Unlike somatic mutation assessment, the
detection of NTRK fusions is not straightforward, and various assays
exist at the DNA, RNA, and protein level. Here, we investigate the
performance of immunohistochemistry and DNA-based next-generation
sequencing to indirectly or directly detect NTRK fusions relative to an
RNA-based next-generation sequencing approach in the largest cohort of
NTRK fusion positive solid tumors to date. A retrospective analysis of
38,095 samples from 33,997 patients sequenced by a targeted DNA-based
next-generation sequencing panel (MSK-IMPACT), 2189 of which were also
examined by an RNA-based sequencing assay (MSK-Fusion), identified 87
patients with oncogenic NTRK1-3 fusions. All available institutional
NTRK fusion positive cases were assessed by pan-Trk immunohistochemistry
along with a cohort of control cases negative for NTRK fusions by
next-generation sequencing. DNA-based sequencing showed an overall
sensitivity and specificity of 81.1\% and 99.9\%, respectively, for the
detection of NTRK fusions when compared to RNA-based sequencing. False
negatives occurred when fusions involved breakpoints not covered by the
assay. Immunohistochemistry showed overall sensitivity of 87.9\% and
specificity of 81.1\%, with high sensitivity for NTRK1 (96\%) and NTRK2
(100\%) fusions and lower sensitivity for NTRK3 fusions (79\%).
Specificity was 100\% for carcinomas of the colon, lung, thyroid,
pancreas, and biliary tract. Decreased specificity was seen in breast
and salivary gland carcinomas (82\% and 52\%, respectively), and
positive staining was often seen in tumors with neural differentiation.
Both sensitivity and specificity were poor in sarcomas. Selection of the
appropriate assay for NTRK fusion detection therefore depends on tumor
type and genes involved, as well as consideration of other factors such
as available material, accessibility of various clinical assays, and
whether comprehensive genomic testing is needed concurrently.

doi: \url{https://doi.org/10.1038/s41379-019-0324-7}

{}

{}

\begin{center}\rule{0.5\linewidth}{\linethickness}\end{center}

\begin{center}\rule{0.5\linewidth}{\linethickness}\end{center}

 - \textbf{Tumour Growth Rate as a validated early radiological
biomarker able to reflect treatment-induced changes in Neuroendocrine
Tumours; the GREPONET-2 study}

\emph{Clinical cancer research : an official journal of the American
Association for Cancer Research 2019 Aug;():}

PubMed: \url{https://www.ncbi.nlm.nih.gov/pubmed/?term=31375514}

PURPOSE: TGR represents the percentage change in tumour volume per month
(\%/m). Previous results from the GREPONET study showed that TGR
measured after 3 months (TGR3m) of starting systemic treatment (ST) or
watch and wait (WW) was an early biomarker predicting progression-free
survival (PFS) in NETs. EXPERIMENTAL DESIGN: Pts from7 centres with
advanced grade(G) 1/2 NETs from the pancreas(P)/small bowel(SB)
initiating ST/WW were eligible. Computed tomography (CT) / magnetic
resonance imaging (MRI) performed at pre-baseline, baseline and 3(+/-1)
months of study entry were retrospectively reviewed. Aim-1: explore
treatment-induced changes in TGR (ΔTGR3m-BL) (paired T-test) and Aim-2:
validate TGR3m (\textless{}0.8\%/m vs ≥0.8\%/m) as an early biomarker in
an independent cohort (Kaplan-Meier/Cox Regression). RESULTS: Out of 785
pts screened, 127 were eligible. Mean (SD) TGR0 and TGR3m were 5.4\%/m
(14.9) and -1.4\%/m (11.8), respectively. Mean(SD) ΔTGR3m-BL
paired-difference was -6.8\%/m(19.3) (p\textless{}0.001). Most marked
ΔTGR3m-BL (mean (SD);p) were identified with targeted therapies
(-11.3\%/m(4.7);0.0237) and chemotherapy (-7.9\%/m(3.4);0.0261).
Multivariable analysis confirmed the absence of previous treatment (Odds
Ratio (OR) 4.65 (95\%CI 1.31-16.52); p-value0.018) and low TGR3m
(continuous variable; OR 1.09 (95\%CI 1.01-1.19); p-value0.042) to be
independent predictors of radiological objective response. When the
multivariable Cox Regression was adjusted to grade (p-value 0.004) and
stage (p-value0.017), TGR3m≥0.8 (vs.\textless{}0.8) maintained its
significance (p\textless{}0.001), while TGR0 and ΔTGR3m-BL did not.
TGR3m was confirmed as an independent prognosis factor for PFS (external
validation; Aim-2) (multivariable HR 2.21 (95\%CI 1.21-3.70);
p-value0.003). CONCLUSIONS: TGR has a role as biomarker for monitoring
response to therapy for early prediction of PFS and radiological
objective response.

doi: \url{https://doi.org/10.1158/1078-0432.CCR-19-0963}

{}

{}

\begin{center}\rule{0.5\linewidth}{\linethickness}\end{center}

\begin{center}\rule{0.5\linewidth}{\linethickness}\end{center}

 - \textbf{Intraductal oncocytic papillary neoplasm of the pancreas: A
systematic review}

\emph{Pancreatology : official journal of the International Association
of Pancreatology (IAP) \ldots{} {[}et al.{]} 2019 Jul;():}

PubMed: \url{https://www.ncbi.nlm.nih.gov/pubmed/?term=31375434}

BACKGROUND: Intraductal oncocytic papillary neoplasm of the pancreas
(IOPN-P) is a rare subtype of intraductal papillary mucinous neoplasm
(IPMN). This study was performed to summarize the clinicopathological
features and management of IOPN-P. METHODS: English-language articles
were searched from MEDLINE and EMBASE from the first report of IOPN-P in
1996 until 1 May 2019 following the methodology in the PRISMA
guidelines. RESULTS: In total, 66 patients from 24 full articles were
included in the final data analysis. The patients' average age was 61
years, and the male/female ratio was 1. Most lesions were large (average
size, 5.50\,cm), located in the pancreatic head, and found either
incidentally or by uncharacteristic abdominal symptoms. IOPN-P was
usually a cystic and solid lesion with or without mural nodules on
radiological examination. A definitive diagnosis was often acquired from
fine needle aspiration biopsy or postoperative pathology. All tumors
were diagnosed as carcinoma in situ or minimally invasive carcinoma,
necessitating surgical resection. The prognosis of IOPN-P was better
than that of other IPMN subtypes, even when metastasis occurred.
Recurrence after surgical resection of IOPN-P was rare. CONCLUSIONS:
IOPN-P is rare among IPMN subtypes with unique pathological
characteristics. Because of the nontypical symptoms and radiological
findings, a definitive preoperative diagnosis usually depends on
multimodal examinations. Management and surveillance of IOPN-P after
surgical resection should be differentiated from those of other
pancreatic benign cystic lesions because of its relative malignancy, but
IOPN-P should also be differentiated from other IPMN subtypes and
malignant cystic tumors because of its favorable prognosis.

doi: \url{https://doi.org/10.1016/j.pan.2019.07.040}

{}

{}

\begin{center}\rule{0.5\linewidth}{\linethickness}\end{center}

\begin{center}\rule{0.5\linewidth}{\linethickness}\end{center}

 - \textbf{Response to repeat echoendoscopic celiac plexus neurolysis in
pancreatic cancer patients: A machine learning approach}

\emph{Pancreatology : official journal of the International Association
of Pancreatology (IAP) \ldots{} {[}et al.{]} 2019 Jul;():}

PubMed: \url{https://www.ncbi.nlm.nih.gov/pubmed/?term=31375433}

BACKGROUND: /Objectives: Efficacy of repeat echoendoscopic celiac plexus
neurolysis is still unclear. Aim of the study was to assess the efficacy
of repeat celiac plexus neurolysis and to build an artificial neural
network model able to predict pain response. METHODS: Data regarding 156
patients treated with repeat celiac plexus neurolysis between 2004 and
2019 were reviewed. Artificial neural network and logistic regression
models were built to predict pain response after treatment. Performance
of the models was expressed in terms of accuracy, positive predictive
value, and positive likelihood ratio. RESULTS: Median age was 62 years
(range 39-86) and most patients were male (66\%) with pre-procedural
visual analogue score 7. Fifty-one patients (32.6\%) experienced
treatment response, of which 6 (3.8\%) complete pain suppression. Median
duration of pain relief was 6 (2-8) weeks. Tumoral stage, interval from
initial to repeat treatment, response to initial neurolysis, and tumor
progression between the two treatments resulted as significant
predictors of pain response. The performance of the artificial neural
network in predicting treatment response was higher than regression
model (area under the curve: 0.94, 0.89-0.97 versus 0.85, 0.78-0.89;
p\,\textless{}\,0.001). Positive predictive value and positive
likelihood ratio resulted 90.3\% and 19.35, respectively. Classification
error rate was 5.7\% with the artificial neural network compared to
14.7\% of regression model (p\,\textless{}\,0.001). These findings were
confirmed through ten-fold cross validation. CONCLUSIONS: Pain response
following repeat neurolysis is generally less pronounced than after
initial treatment. Artificial neural network may help to identify those
subjects likely to benefit from repeat neurolysis.

doi: \url{https://doi.org/10.1016/j.pan.2019.07.038}

{}

{}

\begin{center}\rule{0.5\linewidth}{\linethickness}\end{center}

\begin{center}\rule{0.5\linewidth}{\linethickness}\end{center}

 - \textbf{Facility Type is Another Factor in the Volume-Outcome
Relationship for Complex Hepatopancreatobiliary Procedures}

\emph{Annals of surgical oncology 2019 Aug;():}

PubMed: \url{https://www.ncbi.nlm.nih.gov/pubmed/?term=31372869}

doi: \url{https://doi.org/10.1245/s10434-019-07668-2}

{}

{}

\begin{center}\rule{0.5\linewidth}{\linethickness}\end{center}

\begin{center}\rule{0.5\linewidth}{\linethickness}\end{center}

 - \textbf{Disruption of IRE1α through its kinase domain attenuates
multiple myeloma}

\emph{Proceedings of the National Academy of Sciences of the United
States of America 2019 Aug;116(33):16420-16429}

PubMed: \url{https://www.ncbi.nlm.nih.gov/pubmed/?term=31371506}

Multiple myeloma (MM) arises from malignant immunoglobulin
(Ig)-secreting plasma cells and remains an incurable, often lethal
disease despite therapeutic advances. The unfolded-protein response
sensor IRE1α supports protein secretion by deploying a
kinase-endoribonuclease module to activate the transcription factor
XBP1s. MM cells may co-opt the IRE1α-XBP1s pathway; however, the
validity of IRE1α as a potential MM therapeutic target is controversial.
Genetic disruption of IRE1α or XBP1s, or pharmacologic IRE1α kinase
inhibition, attenuated subcutaneous or orthometastatic growth of MM
tumors in mice and augmented efficacy of two established frontline
antimyeloma agents, bortezomib and lenalidomide. Mechanistically, IRE1α
perturbation inhibited expression of key components of the endoplasmic
reticulum-associated degradation machinery, as well as secretion of Ig
light chains and of cytokines and chemokines known to promote MM growth.
Selective IRE1α kinase inhibition reduced viability of CD138+ plasma
cells while sparing CD138- cells derived from bone marrows of newly
diagnosed or posttreatment-relapsed MM patients, in both US- and
European Union-based cohorts. Effective IRE1α inhibition preserved
glucose-induced insulin secretion by pancreatic microislets and
viability of primary hepatocytes in vitro, as well as normal tissue
homeostasis in mice. These results establish a strong rationale for
developing kinase-directed inhibitors of IRE1α for MM therapy.

doi: \url{https://doi.org/10.1073/pnas.1906999116}

{}

{}

\begin{center}\rule{0.5\linewidth}{\linethickness}\end{center}

\begin{center}\rule{0.5\linewidth}{\linethickness}\end{center}

 - \textbf{Comparative effectiveness of primary tumor resection in
patients with stage III pancreatic adenocarcinoma}

\emph{BMC cancer 2019 Aug;19(1):761}

PubMed: \url{https://www.ncbi.nlm.nih.gov/pubmed/?term=31370893}

BACKGROUND: Previous studies comparing primary tumor resection (PTR) to
palliative treatment for advanced-stage pancreatic ductal adenocarcinoma
(PDA) were limited by strong selection bias. We used multiple methods to
control for confounding and selection bias to estimate the effect of PTR
on survival for late-stage PDA. METHODS: Surveillance, Epidemiology, and
End Results (SEER) 18 registry database for 2004 through 2014 was
retrieved for the present study. A total of 4322 patients with stage III
(AJCC, 6th) PDA were included in this study. Propensity score matching
(PSM) was performed to eliminate possible bias. In addition,
instrumental variable (IV) analysis was utilized to adjust for both
measured and unmeasured confounders. RESULTS: A total of 4322 patients
with stage III PDA including 552 (12.8\%) who underwent PTR, 3770
(87.2\%) without PTR, were identified. In the multivariable cohort, a
clear prognostic advantage of PTR was observed in overall survival (OS)
(P \textless{} 0.001) and disease-specific survival (DSS)
(P \textless{} 0.001) compared to patients after non-surgery therapy. In
the PSM cohort, patients in PTR group showed a better OS and DSS (both P
values \textless{} 0.001) compared to patients in non-surgery group. The
survival benefit of PTR for stage III PDA was not observed in the
two-stage residual inclusion (2SRI) model. Estimates based on this
instrument indicated that patients treated with PTR had similar OS
(P = 0.448) and DSS (P = 0.719). In IV analyses stratified by
chemotherapy and tumor location, patients undergoing PTR had similar OS
and DSS compared to patients in non-surgery group across all subgroups.
CONCLUSIONS: Survival with PTR did not differ significantly from
palliative treatment in marginal patients with stage III pancreatic
adenocarcinoma. High-quality randomized trials are needed to validate
these results.

doi: \url{https://doi.org/10.1186/s12885-019-5966-9}

{}

{}

\begin{center}\rule{0.5\linewidth}{\linethickness}\end{center}

\begin{center}\rule{0.5\linewidth}{\linethickness}\end{center}

 - \textbf{Rosai-Dorfman Disease of the Pancreas Shows Significant
Histologic Overlap With IgG4-related Disease}

\emph{The American journal of surgical pathology 2019 Jul;():}

PubMed: \url{https://www.ncbi.nlm.nih.gov/pubmed/?term=31368911}

Rosai-Dorfman disease (RDD) is a rare entity characterized by
proliferating S100-positive histiocytes. Originally described in lymph
nodes, it can involve extranodal sites. Pancreatic involvement is rare,
with \textless{}10 cases previously reported. Recent studies demonstrate
a possible overlap between RDD and the more common IgG4-related disease
(IRD), which could further complicate pathologic diagnosis. We describe
distinct morphologic characteristics as well as overlapping histologic
features of IRD in 5 cases of pancreatic RDD at our institution and
compare these to a cohort of nonpancreatic extranodal RDD cases. All
pancreatic cases were mass forming and had spindled patterns of
elongated histiocytes with smaller areas of more classical appearing
RDD; all cases had areas of storiform fibrosis and dense
lymphoplasmacytic infiltrates with no increase in IgG4-positive plasma
cells, and all cases had some degree of vasculitis (4 cases had
obliterative vasculitis). Thirteen nonpancreatic extranodal RDD cases
had dense lymphoplasmacytic infiltrates; most (85\%) had some fibrosis
with 46\% showing storiform fibrosis, 85\% had vasculitis with 31\%
demonstrating obliterative vasculitis and 2 cases had increased IgG4
staining. Extranodal (pancreatic and nonpancreatic) RDD often shows
overlapping morphologic features with IRD, including lymphoplasmacytic
inflammation, storiform fibrosis with elongated histiocytes and
vasculitis. This can create a diagnostic challenge in the pancreas where
IRD is more commonly encountered. Pathologists need to be aware that RDD
can occur in the pancreas and should include RDD in the differential of
any mass forming pancreatic lesion in which morphologic features of IRD
are present.

doi: \url{https://doi.org/10.1097/PAS.0000000000001334}

{}

{}

\begin{center}\rule{0.5\linewidth}{\linethickness}\end{center}

\begin{center}\rule{0.5\linewidth}{\linethickness}\end{center}

 - \textbf{ZEB1 promotes inflammation and progression towards
inflammation-driven carcinoma through repression of the DNA repair
glycosylase MPG in epithelial cells}

\emph{Gut 2019 Jul;():}

PubMed: \url{https://www.ncbi.nlm.nih.gov/pubmed/?term=31366457}

OBJECTIVE: Chronic inflammation is a risk factor in colorectal cancer
(CRC) and reactive oxygen species (ROS) released by the inflamed stroma
elicit DNA damage in epithelial cells. We sought to identify new drivers
of UC and inflammatory CRC. DESIGN: The study uses samples from patients
with UC, mouse models of colitis and CRC and mice deficient for the
epithelial-to-mesenchymal transition factor ZEB1 and the DNA repair
glycosylase N-methyl-purine glycosylase (MPG). Samples were analysed by
immunostaining, qRT-PCR, chromatin immunoprecipitation assays,
microbiota next-generation sequencing and ROS determination. RESULTS:
ZEB1 was induced in the colonic epithelium of UC and of mouse models of
colitis. Compared with wild-type counterparts, Zeb1-deficient mice were
partially protected from experimental colitis and, in a model of
inflammatory CRC, they developed fewer tumours and exhibited lower
levels of DNA damage (8-oxo-dG) and higher expression of MPG. Knockdown
of ZEB1 in CRC cells inhibited 8-oxo-dG induction by oxidative stress
(H2O2) and inflammatory cytokines (interleukin (IL)1β). ZEB1 bound
directly to the MPG promoter whose expression inhibited. This molecular
mechanism was validated at the genetic level and the crossing of
Zeb1-deficient and Mpg-deficient mice reverted the reduced inflammation
and tumourigenesis in the former. ZEB1 expression in CRC cells induced
ROS and IL1β production by macrophages that, in turn, lowered MPG in CRC
cells thus amplifying a positive loop between both cells to promote DNA
damage and inhibit DNA repair. CONCLUSIONS: ZEB1 promotes colitis and
inflammatory CRC through the inhibition of MPG in epithelial cells, thus
offering new therapeutic strategies to modulate inflammation and
inflammatory cancer.

doi: \url{https://doi.org/10.1136/gutjnl-2018-317294}

{}

{}

\begin{center}\rule{0.5\linewidth}{\linethickness}\end{center}

\begin{center}\rule{0.5\linewidth}{\linethickness}\end{center}

 - \textbf{Characterising the impact of body composition change during
neoadjuvant chemotherapy for pancreatic cancer}

\emph{Pancreatology : official journal of the International Association
of Pancreatology (IAP) \ldots{} {[}et al.{]} 2019 Jul;():}

PubMed: \url{https://www.ncbi.nlm.nih.gov/pubmed/?term=31362865}

BACKGROUND: Pancreatic Cancer remains a lethal disease for the majority
of patients. New chemotherapy agents such as Folfirinox offer
therapeutic potential for patients who present with Borderline
Resectable disease (BRPC). However, results to date are inconsistent,
with factors such as malnutrition limiting successful drug delivery. We
sought to determine the prevalence of sarcopenia in BRPC patients at
diagnosis, and to quantify body composition change during chemotherapy.
METHODS: The diagnostic/restaging CT scans of BRPC patients were
analysed. Body composition was measured at L3 using Tomovision
Slice-O-Matic™. Total muscle and adipose tissue mass were estimated
using validated regression equations. Sarcopenia was defined as per
gender- and body mass index (BMI)-specific lumbar skeletal muscle index
(LSMI) and muscle attenuation reference values. RESULTS: Seventy-eight
patients received neo-adjuvant chemotherapy, and 67 patients underwent
restaging CT, at which point a third were deemed resectable. Half were
sarcopenic at diagnosis, and sarcopenia was equally prevalent across all
BMI categories.. Skeletal muscle and adipose tissue (intra-muscular,
visceral and sub-cutaneous) area decreased during chemotherapy
(p\,\textless{}\,0.0001). Low muscle attenuation was observed in half of
patients at diagnosis, and was associated with increased mortality risk.
Loss of lean tissue parameters during chemotherapy was associated with
an increased mortality risk; specifically fat-free mass, HR 1.1 (95\% CI
1.03-1.17, p\,=\,0.003) and skeletal muscle mass, HR 1.21 (95\%CI
1.08-1.35, p\,=\,0.001). CONCLUSIONS: Sarcopenia was prevalent in half
of patients at the time of diagnosis with BRPC. Low muscle attenuation
at diagnosis, coupled with lean tissue loss during chemotherapy,
independently increased mortality risk.

doi: \url{https://doi.org/10.1016/j.pan.2019.07.039}

{}

{}

\begin{center}\rule{0.5\linewidth}{\linethickness}\end{center}

\begin{center}\rule{0.5\linewidth}{\linethickness}\end{center}

 - \textbf{Actual 10-Year Survival After Surgical Microwave Ablation for
Hepatocellular Carcinoma: A Single-Center Experience in Japan}

\emph{Annals of surgical oncology 2019 Jul;():}

PubMed: \url{https://www.ncbi.nlm.nih.gov/pubmed/?term=31359277}

BACKGROUND: Little evidence exists regarding long-term survival after
microwave ablation for hepatocellular carcinoma (HCC). The aim of this
study is to determine actual 10-year survival and clarify the
clinicopathological features of patients surviving ≥ 10~years after
surgical microwave ablation. PATIENTS AND METHODS: This retrospective
study identified 459 patients who underwent surgical microwave ablation
for HCC with curative intent between 2001 and 2008. We compared 100
patients who survived ≥ 10~years with 321 patients who died within
10~years. RESULTS: Median overall survival and recurrence-free survival
rates were 5.5 and 2.4~years, respectively. The actual 10-year overall
survival rate was 23.8\%, and the actual 10-year recurrence-free
survival rate was 8.1\%. Multivariate analysis showed that
age \textgreater{} 70~years {[}odds ratio 1.87, P = 0.029{]}, hepatitis
C virus positivity (OR 2.30, P = 0.004), Child-Pugh class B (OR 3.28,
P = 0.003), and platelet count \textless{} 10 × 104 /µL (OR 1.93,
P = 0.033) were independent risk factors for actual 10-year survival.
During 10-year follow-up, 66\% of the ≥ 10-year survivors developed
recurrence, and 91\% of these patients underwent further curative
treatment, including hepatic resection or local ablation, for HCC
recurrence. CONCLUSION: Ten-year survival after surgical microwave
ablation for HCC can be expected in approximately 24\% of patients, even
though nearly 2/3 of our 10-year survival patients experienced
recurrence. Close postoperative follow-up and further curative treatment
for recurrence are important for improving long-term survival.

doi: \url{https://doi.org/10.1245/s10434-019-07646-8}

{}

{}

\begin{center}\rule{0.5\linewidth}{\linethickness}\end{center}

\begin{center}\rule{0.5\linewidth}{\linethickness}\end{center}

 - \textbf{Characterization and comparison of GITR expression in solid
tumors}

\emph{Clinical cancer research : an official journal of the American
Association for Cancer Research 2019 Jul;():}

PubMed: \url{https://www.ncbi.nlm.nih.gov/pubmed/?term=31358539}

PURPOSE: Determine the differential effect of a FcgR-binding, mIgG2a
anti-GITR antibody in mouse tumor modelsand characterize the tumor
microenvironment for the frequency of GITR expression in T cell subsets
from seven different human solid tumors. EXPERIMENTAL DESIGN: For mouse
experiments, wildtype C57BL/6 mice were subcutaneously injected with
MC38 cells or B16 cells, and BALB/c mice were injected with CT26 cells.
Mice were treated with the anti-mouse GITR agonist antibody 21B6, and
tumor burden and survival were monitored. GITR expression was evaluated
at the single cell level using flow cytometry (FC). 213 samples were
evaluated for GITR expression by immunohistochemistry (IHC), 63 by FC
and 170 by both in seven human solid tumors: advanced hepatocellular
carcinoma, non-small cell lung cancer, renal cell carcinoma, pancreatic
carcinoma, head and neck carcinoma, melanoma, and ovarian carcinoma.
RESULTS: The therapeutic benefit of 21B6 was greatest in CT26 followed
by MC38, and was least in the B16 tumor model. The frequency of CD8 T
cells and effector CD4 T cells within the immune infiltrate correlated
with response to treatment with GITR antibody. Analysis of clinical
tumor samples showed that non-small cell lung cancer, renal cell
carcinoma, and melanoma had the highest proportions of GITR-expressing
cells and highest per-cell density of GITR expression on CD4-positive
Foxp3 positive Tregs. IHC and FC data showed similar trends with a good
correlation between both techniques. CONCLUSIONS: Human tumor data
suggest that NSCLC, RCC, and melanoma should be the tumor subtypes
prioritized for anti-GITR therapy development.

doi: \url{https://doi.org/10.1158/1078-0432.CCR-19-0289}

{}

{}

\begin{center}\rule{0.5\linewidth}{\linethickness}\end{center}

\begin{center}\rule{0.5\linewidth}{\linethickness}\end{center}

 - \textbf{Fungal sinusitis in simultaneous pancreas-kidney transplant}

\emph{Journal of clinical pathology 2019 Jul;():}

PubMed: \url{https://www.ncbi.nlm.nih.gov/pubmed/?term=31358535}

doi: \url{https://doi.org/10.1136/jclinpath-2018-205258}

{}

{}

\begin{center}\rule{0.5\linewidth}{\linethickness}\end{center}

\begin{center}\rule{0.5\linewidth}{\linethickness}\end{center}

 - \textbf{New Nodal Staging for Primary Pancreatic Neuroendocrine
Tumors: A Multi-institutional and National Data Analysis}

\emph{Annals of surgery 2019 Jul;():}

PubMed: \url{https://www.ncbi.nlm.nih.gov/pubmed/?term=31356277}

OBJECTIVE: To determine the prognostic role of metastatic lymph node
(LN) number and the minimal number of LNs for optimal staging of
patients with pancreatic neuroendocrine tumors (pNETs). BACKGROUND:
Prognosis relative to number of LN metastasis (LNM), and minimal number
of LNs needed to evaluate for accurate staging, have been poorly defined
for pNETs. METHODS: Number of LNM and total number of LN evaluated
(TNLE) were assessed relative to recurrence-free survival (RFS) and
overall survival (OS) in a multi-institutional database. External
validation was performed using Surveillance, Epidemiology and End
Results (SEER) registry. RESULTS: Among 854 patients who underwent
resection, 233 (27.3\%) had at least 1 LNM. Patients with 1, 2, or 3 LNM
had a comparable worse RFS versus patients with no nodal metastasis
(5-year RFS, 1 LNM 65.6\%, 2 LNM 68.2\%, 3 LNM 63.2\% vs 0 LNM 82.6\%;
all P \textless{} 0.001). In contrast, patients with ≥4 LNM (proposed
N2) had a worse RFS versus patients who either had 1 to 3 LNM (proposed
N1) or node-negative disease (5-year RFS, ≥4 LNM 43.5\% vs 1-3 LNM
66.3\%, 0 LNM 82.6\%; all P \textless{} 0.05) {[}C-statistics area under
the curve (AUC) 0.650{]}. TNLE ≥8 had the highest discriminatory power
relative to RFS (AUC 0.713) and OS (AUC 0.726) among patients who had 1
to 3 LNM, and patients who had ≥4 LNM in the multi-institutional and
SEER database (n = 2764). CONCLUSIONS: Regional lymphadenectomy of at
least 8 lymph nodes was necessary to stage patients accurately. The
proposed nodal staging of N0, N1, and N2 optimally staged patients.

doi: \url{https://doi.org/10.1097/SLA.0000000000003478}

{}

{}

\begin{center}\rule{0.5\linewidth}{\linethickness}\end{center}

\begin{center}\rule{0.5\linewidth}{\linethickness}\end{center}

 - \textbf{Left-sided Portal Hypertension After Pancreaticoduodenectomy
With Resection of the Portal Vein/Superior Mesenteric Vein Confluence in
Patients With Pancreatic Cancer: A Project Study by the Japanese Society
of Hepato-Biliary-Pancreatic Surgery}

\emph{Annals of surgery 2019 Jul;():}

PubMed: \url{https://www.ncbi.nlm.nih.gov/pubmed/?term=31356273}

OBJECTIVE: The aim of this study was to evaluate how often left-sided
portal hypertension (LPH) develops and how LPH affects the long-term
outcomes of patients with pancreatic cancer treated with
pancreaticoduodenectomy (PD) and resection of the portal vein
(PV)/superior mesenteric vein (SMV) confluence. SUMMARY BACKGROUND DATA:
Little is known about LPH after PD with resection of the PV/SMV
confluence. METHODS: Overall, 536 patients who underwent PD with PV/SMV
resection were enrolled. Among them, we mainly compared the SVp group
{[}n=285; the splenic vein (SV) was preserved{]} and the SVr group (n =
227; the SV was divided and not reconstructed). RESULTS: The incidence
of variceal formation in the SVr group increased until 3 years after PD
compared with that in the SVp group (38.7\% vs 8.3\%, P \textless{}
0.001). Variceal bleeding occurred in the SVr group (n = 9: 4.0\%) but
not in the SVp group (P \textless{} 0.001). In the multivariate
analysis, the risk factors for variceal formation were liver disease, N
factor, conventional PD, middle colic artery resection, and SV division.
The only risk factor for variceal bleeding was SV division. The platelet
count ratio at 6 months after PD was significantly lower in the SVr
group than in the SVp group (0.97 vs 0.82, P \textless{} 0.001), and the
spleen-volume ratios at 6 and 12 months were significantly higher in the
SVr group than in the SVp group (1.38 vs 1.00 and 1.54 vs 1.09; P
\textless{} 0.001 and P \textless{} 0.001, respectively). CONCLUSIONS:
PD with SV division causes variceal formation, bleeding, and
thrombocytopenia.

doi: \url{https://doi.org/10.1097/SLA.0000000000003487}

{}

{}

\begin{center}\rule{0.5\linewidth}{\linethickness}\end{center}

\begin{center}\rule{0.5\linewidth}{\linethickness}\end{center}

 - \textbf{Cytology with rapid on-site examination (ROSE) does not
improve diagnostic yield of EUS-FNA of pancreatic cystic lesions}

\emph{Diagnostic cytopathology 2019 Jul;():}

PubMed: \url{https://www.ncbi.nlm.nih.gov/pubmed/?term=31356003}

BACKGROUND: Cytology with rapid on-site evaluation (ROSE) has been shown
to increase the diagnostic accuracy of endoscopic ultrasound-guided fine
needle aspiration (EUS-FNA) for solid pancreatic lesions. No data exists
on the need for rapid onsite cytology in the evaluation of pancreatic
cystic lesions (PCLs). The purpose of this study is to determine whether
onsite cytology impacts the diagnostic yield of EUS-FNA of PCLs.
METHODS: We prospectively examined all patients with PCLs who underwent
EUS-FNA without onsite cytology over a 6-month period and compared this
to a historical cohort of patients with PCLs who underwent EUS-FNA with
ROSE in the previous 6 months. Comparison was made between the two
groups based upon patient demographics, EUS cyst characteristics, and
FNA fluid \& cytopathology results. RESULTS: A total of 100 EUS-FNA
exams for PCLs were identified: 46 with ROSE and 54 without onsite
cytology. The majority of cytology findings were negative or
nondiagnostic, 87.0\% in the ROSE group, 77.8\% in the group without
onsite cytology. There was no difference using EUS-FNA without onsite
cytology compared to ROSE when measuring total diagnostic yield (22.2\%
vs 13.0\%, P = .30), number of nondiagnostic specimens (50\% vs 54\%, P
= .69), and number of needle passes (1.51 vs 1.57, P = .68).
CONCLUSIONS: (a) The majority of cytology results from EUS-FNA of cystic
lesions are negative or nondiagnostic. (b) Having rapid onsite cytology
evaluation of cystic lesions does not affect the number of needle passes
nor diagnostic yield and is thus not recommended.

doi: \url{https://doi.org/10.1002/dc.24291}

{}

{}

\begin{center}\rule{0.5\linewidth}{\linethickness}\end{center}

\begin{center}\rule{0.5\linewidth}{\linethickness}\end{center}

 - \textbf{African American women with gum disease and tooth loss face
higher pancreatic cancer risk}

\emph{Cancer 2019 Aug;125(16):2719}

PubMed: \url{https://www.ncbi.nlm.nih.gov/pubmed/?term=31355935}

doi: \url{https://doi.org/10.1002/cncr.32413}

{}

{}

\begin{center}\rule{0.5\linewidth}{\linethickness}\end{center}

\begin{center}\rule{0.5\linewidth}{\linethickness}\end{center}

 - \textbf{Oncogenic KRAS Reduces Expression of FGF21 in Acinar Cells to
Promote Pancreatic Tumorigenesis in Mice on a High-Fat Diet}

\emph{Gastroenterology 2019 Jul;():}

PubMed: \url{https://www.ncbi.nlm.nih.gov/pubmed/?term=31352001}

BACKGROUND \& AIMS: Obesity is a risk factor for pancreatic cancer. In
mice, a high-fat diet (HFD) and expression of oncogenic KRAS lead to
development of invasive pancreatic ductal adenocarcinoma (PDAC) by
unknown mechanisms. We investigated how oncogenic KRAS regulates the
expression of fibroblast growth factor 21 (FGF21), a metabolic regulator
that prevents obesity, and the effects of recombinant human FGF21
(rhFGF21) on pancreatic tumorigenesis. METHODS: We performed
immunohistochemical analyses of FGF21 levels in human pancreatic tissue
arrays, comprising 59 PDAC specimens and 45 non-tumor tissues. We also
studied mice with tamoxifen-inducible expression of oncogenic KRAS in
acinar cells (KrasG12D/+ mice) and fElasCreERT mice (controls).
KrasG12D/+ mice were placed on a HFD or regular chow diet (control) and
given injections of rhFGF21 or vehicle; pancreata were collected and
analyzed by histology, immunoblots, quantitative PCR, and
immunohistochemistry. We measured markers of inflammation in the
pancreas, liver, and adipose tissue. Activity of RAS was measured based
on the amount of bound GTP. RESULTS: Pancreatic tissues of mice
expressed high levels of FGF21 compared with liver. FGF21 and its
receptor proteins were expressed by acinar cells. Acinar cells that
expressed KrasG12D/+ had significantly lower expression of Fgf21 mRNA,
compared with acinar cells from control mice, partly due to
downregulation of PPARG expression-a transcription factor that activates
Fgf21 transcription. Pancreata from KrasG12D/+ mice on a control diet
and given injections of rhFGF21 had reduced pancreatic inflammation,
infiltration by immune cells, and acinar-to-ductal metaplasia compared
with mice given injections of vehicle. HFD-fed KrasG12D/+ mice given
injections of vehicle accumulated abdominal fat, developed extensive
inflammation, pancreatic cysts, and high-grade pancreatic
intraepithelial neoplasias (PanINs); half the mice developed PDAC with
liver metastases. HFD-fed KrasG12D/+ mice given injections of rhFGF21
had reduced accumulation of abdominal fat and pancreatic triglycerides,
fewer pancreatic cysts, reduced systemic and pancreatic markers of
inflammation, fewer PanINs, and longer survival-only about 12\% of mice
developed PDACs and none of the mice had metastases. Pancreata from
HFD-fed KrasG12D/+ mice given injections of rhFGF21 had lower levels of
active RAS than from mice given vehicle. CONCLUSIONS: Normal acinar
cells from mice and humans express high levels of FGF21. In mice, acinar
expression of oncogenic KRAS significantly reduces FGF21 expression.
When these mice are placed on a HFD, they develop extensive
inflammation, pancreatic cysts, PanINs, and PDACs, which are reduced by
injection of FGF21. FGF21 also reduces the GTP binding capacity of RAS.
FGF21 might be used in prevention or treatment of pancreatic cancer.

doi: \url{https://doi.org/10.1053/j.gastro.2019.07.030}

{}

{}

\begin{center}\rule{0.5\linewidth}{\linethickness}\end{center}

\begin{center}\rule{0.5\linewidth}{\linethickness}\end{center}

 - \textbf{Fine needle aspiration of the liver: a ten-year single
institution retrospective review}

\emph{Human pathology 2019 Jul;():}

PubMed: \url{https://www.ncbi.nlm.nih.gov/pubmed/?term=31351156}

Fine-needle aspiration (FNA) of liver masses is a minimally invasive
means of evaluation, with diagnostic accuracy over 85\%. Given that most
of the recent literature on sampling hepatic tumors was published by
radiologists and gastroenterologists, we herein conduct a 10-year
retrospective review of a single institution's cytopathology experience
with the diagnosis of liver lesions. Electronic record review of the
cytopathology files (CoPathPlus; Cerner Corp.) was conducted for the
10-year interval January 2007 through December 2016. All cytology
specimens designated as ``liver'' and ``FNA'' were included. Associated
concurrent and subsequent surgical pathology and cytopathology cases
were identified. All FNA cases were organized into four diagnostic
categories: positive for malignancy, atypical, negative for malignancy,
and non-diagnostic. There were 713 hepatic FNAs that were categorized as
follows: positive for malignancy 467 (65.5\%), atypical 49 (6.9\%),
negative 171 (24.0\%) and non-diagnostic 26 (3.6\%). Metastatic tumors
(95.7\%) were more common that primary (4.3\%). The top two metastatic
primary sites were pancreas (30.1\%) and colon (12.7\%). A total of 166
(23.2\%) cases had concurrent core needle biopsies (CNB). 111 (66.9\%)
were concordant with the FNA diagnosis. Of the 55 discordant cases, 43
(25.9\%) had diagnostic material only on CNB and 12 (7.2\%) had
diagnostic material only on FNA. The sensitivity, specificity, positive
predictive value, negative predictive value, and diagnostic accuracy
were 93.4\%, 96.7\%, 98.2\%, 84.3\%, and 89.3\% respectively.
Irrespective of endoscopic versus percutaneous approach, hepatic FNA is
a sensitive and specific means of identifying metastatic and primary
malignancies of the liver.

doi: \url{https://doi.org/10.1016/j.humpath.2019.07.007}

{}

{}

\begin{center}\rule{0.5\linewidth}{\linethickness}\end{center}

\begin{center}\rule{0.5\linewidth}{\linethickness}\end{center}

 - \textbf{Morphologic Factors Predict Pain Relief Following Pancreatic
Head Resection in Chronic Pancreatitis Description of the Chronic
Pancreatitis Pain Relief (CPPR) Score}

\emph{Annals of surgery 2019 Jul;():}

PubMed: \url{https://www.ncbi.nlm.nih.gov/pubmed/?term=31348039}

OBJECTIVE: This study analyzes the clinicopathologic findings and their
impact on outcome of patients so as to identify which patients benefit
most from surgical treatment in chronic pancreatitis, especially in
regard to pain relief. SUMMARY BACKGROUND DATA: The predominant symptom
of chronic pancreatitis is chronic pain resulting in reduced quality of
life. It is well known that the main reason for development of the
disease is abuse of alcohol and nicotine, but only little data on
factors influencing outcome are available. METHODS: One thousand one
hundred forty-six consecutive patients who underwent surgery for chronic
pancreatitis were included. Clinicopathologic data, including morphology
of the pancreas in preoperative diagnostics and the histopathologic
results, were evaluated. A long-term follow-up including Quality of Life
and pain scores was performed. Additionally, we describe the novel
Chronic Pancreatitis Pain Relief Score (CPPR-Score) as a tool for
prediction of pain relief. RESULTS: Overall the rate of pain relief was
79.8\% after surgery. The presence of an inflammatory mass in the
pancreatic head larger than 4 cm (P \textless{} 0.001), presence of a
dilated main pancreatic duct of over 4 mm (P \textless{} 0.001),
histopathologically detected severe calcifications (P = 0.001) and
severe fibrosis (P \textless{} 0.001) as well as ethanol induced disease
(P \textless{} 0.001) found to be strong independent prognostic factors
for pain relief. The CPPR-Score (0-5 points) proved to be a very good
predictive score for pain-relief (P \textless{} 0.001). CONCLUSIONS: The
rate of pain relief after surgical treatment in chronic pancreatitis is
high and the commonly used procedures can be performed with acceptable
morbidity and mortality. The Chronic Pancreatitis Pain Relief Score
allows identifying patients who will benefit most from surgery.

doi: \url{https://doi.org/10.1097/SLA.0000000000003439}

{}

{}

\begin{center}\rule{0.5\linewidth}{\linethickness}\end{center}

\begin{center}\rule{0.5\linewidth}{\linethickness}\end{center}

 - \textbf{Reappraisal of a 2-Cm Cut-Off Size for the Management of
Cystic Pancreatic Neuroendocrine Neoplasms: A Multicenter International
Study}

\emph{Annals of surgery 2019 Jul;():}

PubMed: \url{https://www.ncbi.nlm.nih.gov/pubmed/?term=31348038}

MINI: The characteristics of cystic pancreatic neuroendocrine neoplasms
(cPanNENs) are largely unknown, and their clinical management remains
unclear; specifically, an observational strategy for asymptomatic
cPanNENs ≤2 cm has been proposed by recent guidelines, but evidence is
scarce and limited to single institutional series. In this international
cohort study of 263 resected cPanNENs from 16 institutions worldwide, a
preoperative size \textgreater{}2 cm was independently associated with
aggressive behavior both in the whole cohort and in the subset of
asymptomatic patients; notably, only 1 of 61 asymptomatic cPanNENs ≤2 cm
was aggressive. Based on these results, a watch-and-wait policy for
sporadic asymptomatic cPanNENs ≤2 cm seems justified and safe. The aim
of this study was to characterize an international cohort of resected
cystic pancreatic neuroendocrine neoplasms (cPanNENs) and identify
preoperative predictors of aggressive behavior. The characteristics of
cPanNENs are unknown and their clinical management remains unclear. An
observational strategy for asymptomatic cPanNENs ≤2 cm has been proposed
by recent guidelines, but evidence is scarce and limited to
single-institutional series. Resected cPanNENs (1995-2017) from 16
institutions worldwide were included. Solid lesions (\textgreater{}50\%
solid component), functional tumors, and MEN-1 patients were excluded.
Aggressiveness was defined as lymph node (LN) involvement, G3 grading,
distant metastases, and/or recurrence. Overall, 263 resected cPanNENs
were included, among which 177 (63.5\%) were \textgreater{}2 cm
preoperatively. A preoperative diagnosis of cPanNEN was established in
162 cases (61.6\%) and was more frequent when patients underwent
endoscopic ultrasound {[}EUS, odds ratio (OR) 2.69, 95\% confidence
interval (CI) 1.52-4.77{]} and somatostatin-receptor imaging (OR 3.681,
95\% CI 1.809-7.490), and for those managed in specialized institutions
(OR 3.12, 95\% CI 1.57-6.21). Forty-one cPanNENs (15.6\%) were
considered aggressive. In the whole cohort, LN involvement on imaging,
age \textgreater{}65 years, preoperative size \textgreater{}2 cm, and
pancreatic duct dilation were independently associated with aggressive
behavior. In asymptomatic patients, older age and a preoperative size
\textgreater{}2 cm remained independently associated with
aggressiveness. Only 1 of 61 asymptomatic cPanNENs ≤2 cm displayed an
aggressive behavior. The diagnostic accuracy of cPanNENs is increased by
the use of EUS and somatostatin-receptor imaging and is higher in
specialized institutions. Preoperative size \textgreater{}2 cm is
independently associated with aggressive behavior. Consequently, a
watch-and-wait policy for sporadic asymptomatic cPanNENs ≤2 cm seems
justified and safe for most patients.

doi: \url{https://doi.org/10.1097/SLA.0000000000003508}

{}

{}

\begin{center}\rule{0.5\linewidth}{\linethickness}\end{center}

\begin{center}\rule{0.5\linewidth}{\linethickness}\end{center}

 - \textbf{Long non-coding RNA PTTG3P functions as an oncogene by
sponging miR-383 and up-regulating CCND1 and PARP2 in hepatocellular
carcinoma}

\emph{BMC cancer 2019 Jul;19(1):731}

PubMed: \url{https://www.ncbi.nlm.nih.gov/pubmed/?term=31340767}

BACKGROUND: Emerging evidence indicates that Long non-coding RNAs
(LncRNAs) and microRNAs (miRNAs) play crucial roles in tumor
progression, including hepatocellular carcinoma (HCC). However, whether
there is a crosstalk between LncRNA pituitary tumor-transforming 3
(PTTG3P) and miR-383 in HCC remains unknown. This study is designed to
explore the underlying mechanism by which LncRNA PTTG3P sponges miR-383
during HCC progression. METHODS: qPCR and Western blot were used to
analyze LncRNA PTTG3P, miR-383 and other target genes' expression. CCK-8
assay was performed to examine cell proliferation. Annexin V-PE/PI and
PI staining were used to analyze cell apoptosis and cell cycle
distribution by flow cytometry, respectively. Transwell migration and
invasion assays were used to examine cell migration and invasion
abilities. An in vivo xenograft study was performed to detect tumor
growth. Luciferase reporter assay and RNA pull-down assay were carried
out to detect the interaction between miR-383 and LncRNA PTTG3P. RIP was
carried out to detect whether PTTG3P and miR-383 were enriched in
Ago2-immunoprecipitated complex. RESULTS: In this study, we found that
PTTG3P was up-regulated in HCC tissues and cells. Functional experiments
demonstrated that knockdown of PTTG3P inhibited cell proliferation,
migration and invasion, and promoted cell apoptosis, acting as an
oncogene. Mechanistically, PTTG3P upregulated the expression of miR-383
targets Cyclin D1 (CCND1) and poly ADP-ribose polymerase 2 (PARP2) by
sponging miR-383, acting as a competing endogenous RNA (ceRNA). The
PTTG3P-miR-383-CCND1/PARP2 axis modulated HCC phenotypes. Moreover,
PTTG3P also affected the PI3K/Akt signaling pathway. CONCLUSION: The
data indicate a novel PTTG3P-miR-383-CCND1/PARP2 axis in HCC
tumorigenesis, suggesting that PTTG3P may be used as a potential
therapeutic target in HCC.

doi: \url{https://doi.org/10.1186/s12885-019-5936-2}

{}

{}

\begin{center}\rule{0.5\linewidth}{\linethickness}\end{center}

\begin{center}\rule{0.5\linewidth}{\linethickness}\end{center}

 - \textbf{Age-related morphological changes in the pancreas and their
association with pancreatic carcinogenesis}

\emph{Pathology international 2019 Jul;():}

PubMed: \url{https://www.ncbi.nlm.nih.gov/pubmed/?term=31339204}

Age-related pathological changes in the pancreas have been unclear
because they are often minor and nonspecific. However, recent studies
have shown that they are closely related to various pathological
conditions such as pancreatic cancer and diabetes mellitus. Knowledge of
age-related changes is important to determine appropriate prevention,
detection, and treatment strategies for various diseases observed in
elderly patients. We present a review of the pathological age-related
non-neoplastic changes in the exocrine pancreas such as pancreatic fatty
replacement, lobulocentric pancreatic atrophy, pancreatic duct ectasia,
and metaplasia of exocrine pancreas, as well as changes in islet cells.
We have discussed common pancreatic neoplasms in elderly patients, such
as pancreatic intraepithelial neoplasia (PanIN), intraductal papillary
mucinous neoplasms (IPMNs), and pancreatic ductal adenocarcinoma (PDAC).
Age-related pathological changes play a key role in pancreatic
carcinogenesis via telomere dysfunction. Further studies are warranted
to clarify molecular mechanisms of pancreatic carcinogenesis in elderly
patients.

doi: \url{https://doi.org/10.1111/pin.12837}

{}

{}

\begin{center}\rule{0.5\linewidth}{\linethickness}\end{center}

\begin{center}\rule{0.5\linewidth}{\linethickness}\end{center}

 - \textbf{Acinar cell carcinoma of the pancreas with thyroid-like
follicular features: first description of a new diagnostic challenging
subtype}

\emph{Virchows Archiv : an international journal of pathology 2019
Jul;():}

PubMed: \url{https://www.ncbi.nlm.nih.gov/pubmed/?term=31338587}

Acinar cell carcinomas (ACCs) of the pancreas are a heterogeneous group
of neoplasms showing a wide spectrum of morphological features including
acinar, solid, glandular, and trabecular architecture. In addition,
uncommon cytological aspects have recently been described and include
oncocytic, spindle, clear, and pleomorphic cell types. This wide
histological spectrum represents a challenge in the diagnostic task for
pathologists. Molecular mechanisms involved in the onset and progression
of ACCs are not completely known, but, in general, they differ from
those observed in ductal adenocarcinomas or neuroendocrine neoplasms of
the pancreas and frequently include alterations in the APC/β-catenin
pathway. In the present paper, we describe a new variant of ACC showing
thyroid-like follicular features and CTNNB1 mutation. This phenotype
needs to be included in the spectrum of morphological presentation of
ACC.

doi: \url{https://doi.org/10.1007/s00428-019-02628-3}

{}

{}

\begin{center}\rule{0.5\linewidth}{\linethickness}\end{center}

\begin{center}\rule{0.5\linewidth}{\linethickness}\end{center}

 - \textbf{Overall survival in patients over 40 years old with
surgically resected pancreatic carcinoma: a SEER-based nomogram
analysis}

\emph{BMC cancer 2019 Jul;19(1):726}

PubMed: \url{https://www.ncbi.nlm.nih.gov/pubmed/?term=31337369}

BACKGROUND: The aim of this study was to identify the determinants of
overall survival (OS) within patients over 40 years old with surgically
resected pancreatic carcinoma (PC), and to develop a nomogram with the
intention of OS predicting. METHODS: A total of 6341 patients of
40 years of age or later with surgically resected PC between 2010 and
2015 were enrolled from the Surveillance, Epidemiology, and End Results
(SEER) program and randomly assigned into training set (4242 cases) and
validation set (2099 cases). A nomogram was constructed for predicting
1-, 2- and 3-years OS based on univairate and multivariate Cox
regression. The C-index and calibration plot were adopted to assess the
nomogram performance. RESULTS: Our analysis showed that age, location of
carcinoma in pancreas, tumor grade, TNM stage, size of carcinoma
together with lymph node ratio (LNR) were considered to be independent
overall survival predictors. A nomogram based on these six factors was
developed with C-index being 0.680 (95\%CI: 0.667-0.693). All
calibration curves of OS fitted well. The OS curves stratified by
nomogram-predicted probability score (≥20, 10-19 and \textless{} 10)
demonstrated statistically significant difference not only within
training set but also in validation set. CONCLUSIONS: The present
nomogram for OS predicting can serve as the efficacious
survival-predicting model and assist in accurate decision-making for
patients over 40 years old with surgically resected PC.

doi: \url{https://doi.org/10.1186/s12885-019-5958-9}

{}

{}

\begin{center}\rule{0.5\linewidth}{\linethickness}\end{center}

\begin{center}\rule{0.5\linewidth}{\linethickness}\end{center}

 - \textbf{Anticancer immunotherapy by MFAP5 blockade inhibits fibrosis
and enhances chemosensitivity in ovarian and pancreatic cancer}

\emph{Clinical cancer research : an official journal of the American
Association for Cancer Research 2019 Jul;():}

PubMed: \url{https://www.ncbi.nlm.nih.gov/pubmed/?term=31332047}

PURPOSE: Recent studies demonstrate the role of the tumor
microenvironment in tumor progression. However, strategies used to
overcome the malignant phenotypes of cancer cells modulated by the
microenvironment have not been thoroughly explored. In this study, we
evaluated the therapeutic efficacy of a newly developed monoclonal
antibody targeting microfibril associated protein 5 (MFAP5), which is
secreted predominately by CAFs, in ovarian and pancreatic cancer models.
EXPERIMENTAL DESIGN: Monoclonal antibodies were developed using human
MFAP5 recombinant protein as an antigen in mice and antibodies from
hybridoma clones were evaluated for their specificity to human and
murine MFAP5. An Octet RED384 system was used to determine the kinetics
of binding affinity and the specificity of the antibody clones, which
were followed by epitope mapping and functional characterization by in
vitro assays. The therapeutic efficacy of a lead anti-MFAP5 antibody
clone 130A in tumor suppression was evaluated by ovarian tumor- and
pancreatic tumor-bearing mouse models. RESULTS: Three hybridoma clones,
which produced antibodies with high affinity and specificity to MFAP5,
were selected for functional studies. Antibody clone 130A, which
recognizes a common epitope shared between human and murine MFAP5
protein, were further selected for in vivo studies. Results showed that
clone 130A down-regulated MFAP5-induced collagen production in CAFs,
suppressed intratumoral microvessel leakiness, and enhanced paclitaxel
bioavailability in both ovarian and pancreatic cancer mouse models.
CONCLUSIONS: These data suggest that MFAP5 blockade using an immunologic
approach inhibits fibrosis, induces tumor vessel normalization and
enhances chemosensitivity in ovarian and pancreatic cancer, and can be
used as a novel therapeutic agent.

doi: \url{https://doi.org/10.1158/1078-0432.CCR-19-0187}

{}

{}

\begin{center}\rule{0.5\linewidth}{\linethickness}\end{center}

\begin{center}\rule{0.5\linewidth}{\linethickness}\end{center}

 - \textbf{Mild maternal hyperglycemia in INSC93S transgenic pigs causes
impaired glucose tolerance and metabolic alterations in neonatal
offspring}

\emph{Disease models \& mechanisms 2019 Aug;12(8):}

PubMed: \url{https://www.ncbi.nlm.nih.gov/pubmed/?term=31308048}

Alongside the obesity epidemic, the prevalence of maternal diabetes is
rising worldwide, and adverse effects on fetal development and metabolic
disturbances in the offspring's later life have been described. To
clarify whether metabolic programming effects are due to mild maternal
hyperglycemia without confounding obesity, we investigated wild-type
offspring of INSC93S transgenic pigs, which are a novel genetically
modified large-animal model expressing mutant insulin (INS) C93S in
pancreatic β-cells. This mutation results in impaired glucose tolerance,
mild fasting hyperglycemia and insulin resistance during late pregnancy.
Compared with offspring from wild-type sows, piglets from hyperglycemic
mothers showed impaired glucose tolerance and insulin resistance
(homeostatic model assessment of insulin resistance: +3-fold in males;
+4.4-fold in females) prior to colostrum uptake. Targeted metabolomics
in the fasting and insulin-stimulated state revealed distinct
alterations in the plasma metabolic profile of piglets from
hyperglycemic mothers. They showed increased levels of acylcarnitines,
gluconeogenic precursors such as alanine, phospholipids (in particular
lyso-phosphatidylcholines) and α-aminoadipic acid, a potential biomarker
for type 2 diabetes. These observations indicate that mild gestational
hyperglycemia can cause impaired glucose tolerance, insulin resistance
and associated metabolic alterations in neonatal offspring of a
large-animal model born at a developmental maturation status comparable
to human babies.

doi: \url{https://doi.org/10.1242/dmm.039156}

{}

{}

\begin{center}\rule{0.5\linewidth}{\linethickness}\end{center}

\begin{center}\rule{0.5\linewidth}{\linethickness}\end{center}

 - \textbf{Reply to Comment on Zeng et al, Spatial Distribution of
Pancreatic Stones in Chronic Pancreatitis}

\emph{Pancreas 2019 Aug;48(7):e59}

PubMed: \url{https://www.ncbi.nlm.nih.gov/pubmed/?term=31306309}

doi: \url{https://doi.org/10.1097/MPA.0000000000001351}

{}

{}

\begin{center}\rule{0.5\linewidth}{\linethickness}\end{center}

\begin{center}\rule{0.5\linewidth}{\linethickness}\end{center}

 - \textbf{Mathematical Model and Study Design Could Be Optimized in
Spatial Distribution Analysis of Pancreatic Stones}

\emph{Pancreas 2019 Aug;48(7):e58}

PubMed: \url{https://www.ncbi.nlm.nih.gov/pubmed/?term=31306308}

doi: \url{https://doi.org/10.1097/MPA.0000000000001362}

{}

{}

\begin{center}\rule{0.5\linewidth}{\linethickness}\end{center}

\begin{center}\rule{0.5\linewidth}{\linethickness}\end{center}

 - \textbf{Reply to: The Relationship of Acute Pancreatitis and
Pancreatic Cancer}

\emph{Pancreas 2019 Aug;48(7):e57-e58}

PubMed: \url{https://www.ncbi.nlm.nih.gov/pubmed/?term=31306307}

doi: \url{https://doi.org/10.1097/MPA.0000000000001358}

{}

{}

\begin{center}\rule{0.5\linewidth}{\linethickness}\end{center}

\begin{center}\rule{0.5\linewidth}{\linethickness}\end{center}

 - \textbf{The Relationship of Acute Pancreatitis and Pancreatic Cancer}

\emph{Pancreas 2019 Aug;48(7):e57}

PubMed: \url{https://www.ncbi.nlm.nih.gov/pubmed/?term=31306306}

doi: \url{https://doi.org/10.1097/MPA.0000000000001357}

{}

{}

\begin{center}\rule{0.5\linewidth}{\linethickness}\end{center}

\begin{center}\rule{0.5\linewidth}{\linethickness}\end{center}

 - \textbf{Surgical and Oncological Outcomes of Laparoscopic Versus Open
Pancreaticoduodenectomy in Patients With Pancreatic Duct Adenocarcinoma}

\emph{Pancreas 2019 Aug;48(7):861-867}

PubMed: \url{https://www.ncbi.nlm.nih.gov/pubmed/?term=31306305}

It is not clear which of the 2 principal treatments for patients with
pancreatic duct adenocarcinoma (PDAC), laparoscopic
pancreaticoduodenectomy (LPD) and open pancreaticoduodenectomy (OPD),
has greater safety and efficacy. We performed the present meta-analysis
to assess the efficacy of both treatments for PDAC patients undergoing
LPD. Multiple electronic databases were systematically searched to
identify studies (up to October 2018) comparing LPD with OPD for PDAC.
Short- and long-term oncological outcomes were evaluated. Six studies
were qualified for inclusion criteria in this meta-analysis with a total
of 9144 PDAC participants. Regarding safety, there were fewer overall
postoperative complications associated with LPD (P = 0.005), but the
results were similar in terms of pancreatic fistula and mortality.
Laparoscopic pancreaticoduodenectomy was associated with a better trend
of performance both in R0 resection (relative risk, 1.03; 95\%
confidence interval {[}CI{]}, 1.00-1.07; P = 0.07) and preserved lymph
nodes (median, 2.14; 95\% CI, -0.21 to 4.49; P = 0.07). Long-term
overall survival was comparable between LPD and OPD (hazard ratio, 1.03;
95\% CI, 0.95-1.13; P = 0.49). In conclusion, LPD was found to be a
suitable alternative to OPD in selected PDAC patients with respect to
both surgical and oncological outcomes.

doi: \url{https://doi.org/10.1097/MPA.0000000000001363}

{}

{}

\begin{center}\rule{0.5\linewidth}{\linethickness}\end{center}

\begin{center}\rule{0.5\linewidth}{\linethickness}\end{center}

 - \textbf{Excess body weight at age \textless{}50~years is linked to
pancreatic cancer mortality}

\emph{Cancer 2019 Aug;125(15):2527}

PubMed: \url{https://www.ncbi.nlm.nih.gov/pubmed/?term=31298753}

doi: \url{https://doi.org/10.1002/cncr.32394}

{}

{}

\begin{center}\rule{0.5\linewidth}{\linethickness}\end{center}

\begin{center}\rule{0.5\linewidth}{\linethickness}\end{center}

 - \textbf{Impact of Changes in the American Joint Committee on Cancer
Staging Manual, Eighth Edition, for Pancreatic Ductal Adenocarcinoma}

\emph{Pancreas 2019 Aug;48(7):876-882}

PubMed: \url{https://www.ncbi.nlm.nih.gov/pubmed/?term=31268985}

OBJECTIVE: Consistent and reliable tumor staging is a critical factor in
determining treatment strategy, selection of patients for adjuvant
therapy, and for therapeutic clinical trials. The aim of this study was
to evaluate the number and extent of pancreatic ductal adenocarcinoma
(PDAC) cases that would have a different pT, pN, and overall stages
based on the new eighth edition American Joint Committee on Cancer
staging system when compared with the seventh edition. METHODS: Patients
diagnosed with PDAC who underwent pancreaticoduodenectomy, total
pancreatectomy, or distal pancreatectomy from 2007 to 2017 were
retrospectively reviewed. A total of 340 cases were included. RESULTS:
According to the seventh edition, the vast majority of tumors in our
cohort were staged as pT3 tumors (88.2\%). Restaging these cases with
the new size-based pT system resulted in a more equal distribution among
the 3 pT categories, with higher percentage of pT2 cases (55\%).
CONCLUSIONS: The newly adopted pT stage protocol for PDAC is clinically
relevant, ensures a more equal distribution among different stages, and
allows for a significant prognostic stratification. In contrast, the new
pN classification (pN1 and pN2) based on the number of positive lymph
nodes failed to show survival differences and remains controversial.

doi: \url{https://doi.org/10.1097/MPA.0000000000001349}

{}

{}

\begin{center}\rule{0.5\linewidth}{\linethickness}\end{center}

\begin{center}\rule{0.5\linewidth}{\linethickness}\end{center}

 - \textbf{The Surveillance Patterns of Incidentally Detected Pancreatic
Cysts Vary Widely and Infrequently Adhere to Guidelines}

\emph{Pancreas 2019 Aug;48(7):883-887}

PubMed: \url{https://www.ncbi.nlm.nih.gov/pubmed/?term=31268984}

OBJECTIVES: We aimed to determine incidental pancreatic cyst (``cyst'')
surveillance patterns, predictors of receiving surveillance, and
guideline adherence. METHODS: We performed a retrospective cohort study
of all patients receiving longitudinal care at a single tertiary care
center with a newly diagnosed incidental pancreatic cyst over a 2-year
period (2010-2011). All follow-up care was abstracted over a 5-year
period. RESULTS: Of 3241 eligible imaging studies reviewed, 100 patients
with newly diagnosed incidental cysts eligible for surveillance were
identified. A majority (53\%) received no follow-up. We identified 4
predictors of cyst surveillance: radiology report conclusion mentioning
the cyst (odds ratio {[}OR{]}, 14.9; 95\% confidence interval {[}CI{]},
1.9-119) and recommending follow-up (OR, 5.5; 95\% CI, 2.1-13.9),
pancreas main duct dilation (OR, 10.7; 95\% CI, 1.3-89), and absence of
multiple cysts (OR, 2.5; 95\% CI, 1.1-10.0). Of the 47 patients who
received surveillance, 66\% met minimum surveillance imaging intervals
of at least one guideline. Conversely, 21\% of patients met the criteria
for overutilization in at least one guideline. CONCLUSIONS: Although
guidelines recommend that surgically fit patients with incidental cysts
undergo surveillance, most patients receive no follow-up. When follow-up
occurs, surveillance patterns vary widely and infrequently conform to
guidelines. Interventions to reduce care variation require study.

doi: \url{https://doi.org/10.1097/MPA.0000000000001352}

{}

{}

\begin{center}\rule{0.5\linewidth}{\linethickness}\end{center}

\begin{center}\rule{0.5\linewidth}{\linethickness}\end{center}

 - \textbf{Alternate Week Gemcitabine and Capecitabine: An Effective
Treatment for Patients With Pancreatic Adenocarcinoma}

\emph{Pancreas 2019 Aug;48(7):927-930}

PubMed: \url{https://www.ncbi.nlm.nih.gov/pubmed/?term=31268983}

OBJECTIVE: Determine whether a regimen of fixed dose rate gemcitabine
plus capecitabine is effective and tolerable for advanced pancreatic
adenocarcinoma. METHODS: We performed a retrospective analysis of 62
patients with locally advanced or metastatic pancreatic adenocarcinoma
treated at the University of California San Francisco between 2008 and
2016. Treatment was an alternate week schedule of fixed dose rate 1000
mg/m gemcitabine and capecitabine 1000 mg/m (58 patients), 1200 mg/m (12
patients), or 650 mg/m (1 patient) for intended 12 cycles. We evaluated
overall survival (OS), progression-free survival (PFS), radiologic
response, and adverse events necessitating treatment modification.
RESULTS: For metastatic patients, median OS was 10.3 months (95\%
confidence interval {[}CI{]}, 6.7-12.1 months), and PFS was 5.6 months
(95\% CI, 2.6-7.7 months). In locally advanced patients, OS was 12.0
months (95\% CI, 4.9-17.1 months), and PFS was 5.4 months (95\% CI,
2.5-9.4 months). Radiologic response for metastatic disease (42
patients) was 19\% objective response, 45\% stable disease, and 36\%
progressive disease. Treatment required modification for 22 patients due
to adverse events, most frequently hand-foot syndrome (18 patients).
CONCLUSIONS: Alternate week schedule of fixed dose rate gemcitabine and
capecitabine was active and tolerable for advanced pancreatic
adenocarcinoma. Overall survival and PFS were comparable to first-line
treatments. Importantly, adverse effects appear less severe than
first-line treatments.

doi: \url{https://doi.org/10.1097/MPA.0000000000001354}

{}

{}

\begin{center}\rule{0.5\linewidth}{\linethickness}\end{center}

\begin{center}\rule{0.5\linewidth}{\linethickness}\end{center}

 - \textbf{Induction Therapy in Localized Pancreatic Cancer}

\emph{Pancreas 2019 Aug;48(7):913-919}

PubMed: \url{https://www.ncbi.nlm.nih.gov/pubmed/?term=31268982}

OBJECTIVES: Pancreatic cancer (PDAC) with localized stage includes
resectable (RPC), borderline resectable (BRPC), or locally advanced
unresectable (LAPC). Standard of care for RPC is adjuvant chemotherapy.
There are no prospective randomized trials for best treatment of BRPC
and LAPC. We evaluate the impact of induction chemotherapy on localized
PDAC. METHODS: Charts of PDAC patients treated at Emory University
between 2009 and 2016 were reviewed. The primary end point was overall
survival (OS). RESULTS: A total of 409 localized PDACs were identified.
Resectability was prospectively determined at a multidisciplinary tumor
conference. Median age was 67 years (range, 30-92 years), 49\% were
male, 66\% were white, 171 had RPC, 131 had BRPC, and 107 had LAPC.
Median OSs for RPC, BRPC, and LAPC were 19.5, 16.1, and 12.7 months,
respectively. Type of chemotherapy and age were predictors of OS.
Induction chemotherapy was used in 106 with BRPC (81\%) and 74 with RPC
(56.5\%); patients with BRPC who received combination chemotherapy and
resection had a median OS of 31.5 compared with 19.5 months in patients
with RPC (P = 0.0049). Patients with LAPC had a median OS of 12.7
months. CONCLUSIONS: In patients with BRPC who undergo resection after
induction treatment, the OS was significantly better than in patients
with RPC. Neoadjuvant treatment should be considered for all localized
PDACs.

doi: \url{https://doi.org/10.1097/MPA.0000000000001353}

{}

{}

\begin{center}\rule{0.5\linewidth}{\linethickness}\end{center}

\begin{center}\rule{0.5\linewidth}{\linethickness}\end{center}

 - \textbf{Acute Recurrent and Chronic Pancreatitis as Initial
Manifestations of Cystic Fibrosis and Cystic Fibrosis Transmembrane
Conductance Regulator-Related Disorders}

\emph{Pancreas 2019 Aug;48(7):888-893}

PubMed: \url{https://www.ncbi.nlm.nih.gov/pubmed/?term=31268981}

OBJECTIVES: Recurrent pancreatitis is considered a rare manifestation of
cystic fibrosis transmembrane conductance regulator (CFTR) dysfunction;
this case series highlights that pancreatitis can be a presenting
symptoms of cystic fibrosis (CF) or a CFTR-related disorder (CFTR-RD).
METHODS: Retrospective review of patients younger than 30 years
diagnosed as having acute recurrent pancreatitis (ARP) or chronic
pancreatitis (CP) and subsequently diagnosed as having CF or CFTR-RD.
RESULTS: Among 18 patients, median time from diagnosis of ARP/CP to
diagnosis of CF was 0.4 years (range, 0-33 years). Eight were classified
as having CF by elevated sweat chloride testing (SCT). Five had
intermediate SCT (30-59 mmol/L) with 2 pathogenic mutations. Five had
CFTR-RD with intermediate SCT and 0 to 1 pathogenic mutations. Eight
patients (44\%) had exocrine pancreatic insufficiency, and pancreatic
fluid collections were more common in this group. Based on the CFTR
mutation, 6 patients were eligible for CFTR potentiator therapy,
although none received it during the study period. Nine of the 18 had ≥1
other likely CF manifestations, including sinusitis (33\%), nasal polyps
(11\%), pneumonia (22\%), and gallbladder disease (22\%). CONCLUSIONS:
Cystic fibrosis or CFTR-RD can present as ARP/CP. Complete diagnostic
testing for CFTR-RD in patients with ARP/CP will broaden treatment
options and help to identify comorbid illness.

doi: \url{https://doi.org/10.1097/MPA.0000000000001350}

{}

{}

\begin{center}\rule{0.5\linewidth}{\linethickness}\end{center}

\begin{center}\rule{0.5\linewidth}{\linethickness}\end{center}

 - \textbf{Significance of Lymph Node Metastasis in Resectable
Well-differentiated Pancreatic Neuroendocrine Tumor}

\emph{Pancreas 2019 Aug;48(7):943-947}

PubMed: \url{https://www.ncbi.nlm.nih.gov/pubmed/?term=31268980}

OBJECTIVES: Understanding the effect of lymph node metastasis (LNM) on
prognosis in pancreatic neuroendocrine neoplasm is helpful for surgery
and follow-up. In this study, we investigated the significance of LNM in
well-differentiated pancreatic neuroendocrine tumors (PanNETs) according
to the World Health Organization 2017 classification. METHODS: We
retrospectively collected data for 95 consecutive patients with PanNET
who underwent pancreatic resection with curative intent between January
2008 and December 2017 at 6 institutions. The clinicopathological
factors were compared in patients with and without LNM, and prognostic
factors were analyzed. RESULTS: Lymph node metastasis was significantly
associated with malignant potential of PanNET, such as larger tumor
size, higher Ki-67 index, higher tumor grade, and higher incidence of
lymphatic, vessel, and neural invasion. Lymph node metastasis was also
associated with disease-free but not overall survival. Multivariate
analysis identified NET grade 2 (G2) and G3 as independent risk factors
for recurrence after curative resection. CONCLUSIONS: World Health
Organization 2017 classification was the most independent prognostic
factor in patients with resectable well-differentiated PanNETs. Patients
with G2 and higher-grade tumors require lymph node dissection to improve
prognosis.

doi: \url{https://doi.org/10.1097/MPA.0000000000001355}

{}

{}

\begin{center}\rule{0.5\linewidth}{\linethickness}\end{center}

\begin{center}\rule{0.5\linewidth}{\linethickness}\end{center}

 - \textbf{Gastric Emptying and Distal Gastrectomy Independently Enhance
Postprandial Glucagon-Like Peptide-1 Release After a Mixed Meal and
Improve Glycemic Control in Subjects Having Undergone
Pancreaticoduodenectomy}

\emph{Pancreas 2019 Aug;48(7):953-957}

PubMed: \url{https://www.ncbi.nlm.nih.gov/pubmed/?term=31268979}

OBJECTIVES: New-onset diabetes frequently resolves after
pancreaticoduodenectomy (PD). Glucagon-like peptide-1 (GLP-1)
conceivably is involved as its release is enhanced by rapid gastric
emptying and distal bowel exposure to nutrients. We aimed at studying
factors associated with GLP-1 release after PD. METHODS: Fifteen PD
subjects with distal gastrectomy (Whipple) and 15 with pylorus
preservation were evaluated. A test meal containing 1 g paracetamol to
measure gastric emptying was ingested. Blood for the measurement of
paracetamol, glucose, insulin, and GLP-1 was drawn at baseline and 10,
20, 30, 60, 90, 120, 150, and 180 minutes thereafter. The Matsuda index
of insulin sensitivity was calculated. RESULTS: In univariate analysis,
gastric emptying correlated with GLP-1. Glucagon-like peptide-1
responses to the modes of operation did not differ. Multiple regression
analysis confirmed gastric emptying and Whipple versus
pylorus-preserving pancreaticoduodenectomy as independent predictors of
GLP-1 release. The Matsuda index of insulin sensitivity correlated with
GLP-1 concentrations and inversely with body mass index. Patients after
Whipple procedure revealed lower glycated hemoglobin as compared with
pylorus-preserving pancreaticoduodenectomy. CONCLUSIONS: Following PD,
the postprandial GLP-1 release seems to be enhanced by rapid gastric
emptying and to improve insulin sensitivity. Partial gastrectomy versus
pylorus preservation enhanced the release of GLP-1, conceivably because
of greater distal bowel exposure to undigested nutrients.

doi: \url{https://doi.org/10.1097/MPA.0000000000001361}

{}

{}

\begin{center}\rule{0.5\linewidth}{\linethickness}\end{center}

\begin{center}\rule{0.5\linewidth}{\linethickness}\end{center}

 - \textbf{Endogenous Gastrin Collaborates With Mutant KRAS in
Pancreatic Carcinogenesis}

\emph{Pancreas 2019 Aug;48(7):894-903}

PubMed: \url{https://www.ncbi.nlm.nih.gov/pubmed/?term=31268978}

OBJECTIVE: The KRAS gene is the most frequently mutated gene in
pancreatic cancer, and no successful anti-Ras therapy has been
developed. Gastrin has been shown to stimulate pancreatic cancer in an
autocrine fashion. We hypothesized that reactivation of the peptide
gastrin collaborates with KRAS during pancreatic carcinogenesis.
METHODS: LSL-Kras; P48-Cre (KC) mutant KRAS transgenic mice were crossed
with gastrin-KO (GKO) mice to develop GKO/KC mice. Pancreata were
examined for 8 months for stage of pancreatic intraepithelial neoplasia
lesions, inflammation, fibrosis, gastrin peptide, and microRNA
expression. Pancreatic intraepithelial neoplasias from mice were
collected by laser capture microdissection and subjected to
reverse-phase protein microarray, for gastrin and protein kinases
associated with signal transduction. Gastrin mRNA was measured by RNAseq
in human pancreatic cancer tissues and compared to that in normal
pancreas. RESULTS: In the absence of gastrin, PanIN progression,
inflammation, and fibrosis were significantly decreased and signal
transduction was reversed to the canonical pathway with decreased KRAS.
Gastrin re-expression in the PanINs was mediated by miR-27a. Gastrin
mRNA expression was significantly increased in human pancreatic cancer
samples compared to normal human pancreas controls. CONCLUSIONS: This
study supports the mitogenic role of gastrin in activation of KRAS
during pancreatic carcinogenesis.

doi: \url{https://doi.org/10.1097/MPA.0000000000001360}

{}

{}

\begin{center}\rule{0.5\linewidth}{\linethickness}\end{center}

\begin{center}\rule{0.5\linewidth}{\linethickness}\end{center}

 - \textbf{New-Onset Diabetes Mellitus After Chronic Pancreatitis
Diagnosis: A Systematic Review and Meta-analysis}

\emph{Pancreas 2019 Aug;48(7):868-875}

PubMed: \url{https://www.ncbi.nlm.nih.gov/pubmed/?term=31268977}

OBJECTIVES: The aim of this study was to assess the occurrence of
new-onset diabetes mellitus (DM) after chronic pancreatitis (CP)
diagnosis via systematic review and meta-analysis. METHODS: A systematic
review of literature and meta-analysis of relevant reports were
performed. The primary outcome measures studied were newly diagnosed DM
and DM treated with insulin. For the binary outcomes, pooled prevalence
and 95\% confidence interval (CI) were calculated. METHODS: Fifteen
studies involving 8970 patients were eligible. The incidence of
new-onset DM after CP diagnosis was 30\% (95\% CI, 27\%-33\%). Among all
patients, 17\% (95\% CI, 13\%-22\%) developed insulin-dependent
new-onset DM. The prevalence of newly diagnosed DM after CP diagnosis
increased from 15\% within 36 months to 33\% after 60 months. The
proportion of alcoholic CP, sex, age, and body mass index had minimal
effect on the studied outcomes. CONCLUSIONS: This systematic review
identified a clinically relevant risk of new-onset DM after CP
diagnosis. Therefore, patients should be informed of the risk of DM and
monitored.

doi: \url{https://doi.org/10.1097/MPA.0000000000001359}

{}

{}

\begin{center}\rule{0.5\linewidth}{\linethickness}\end{center}

\begin{center}\rule{0.5\linewidth}{\linethickness}\end{center}

 - \textbf{The Importance of a Conjoint Analysis of Tumor-Associated
Macrophages and Immune Checkpoints in Pancreatic Cancer}

\emph{Pancreas 2019 Aug;48(7):904-912}

PubMed: \url{https://www.ncbi.nlm.nih.gov/pubmed/?term=31268976}

OBJECTIVES: Tumor-associated macrophages are dominant players in
establishing the inmmunosuppressive microenvironment in pancreatic
ductal adenocarcinoma (PDAC). Immune checkpoint inhibitor monotherapy
has achieved limited clinical effectiveness. To date, the interaction of
macrophages and checkpoint regulators and their correlation with
clinicopathologic characteristics in PDAC have been largely unavailable.
METHODS: Macrophages and immune checkpoint expression were assessed by
immunohistochemistry from 80 PDAC samples. Clinicopathologic features
and the prognostic value of each marker were evaluated. In vitro changes
in the expression of immune markers in cocultured macrophages and PDAC
cells were detected by Western blot and immunosorbance assays. RESULTS:
The macrophages marker CD163 and the checkpoint marker programmed
death-ligand 1 (PD-L1) remained as the independent prognostic factors
for overall survival (hazard ratio, 2.543; P = 0.017 and hazard ratio,
2.389; P = 0.021). Furthermore, integrated analysis of CD163 and PD-L1
served as more optimal indicators of survival (P = 0.000). In vitro
coculture of macrophages and PDAC cells significantly increased the
expression of CD163 and PD-L1, compared with monocultured counterpart (P
\textless{} 0.05). CONCLUSIONS: Combined analysis of CD163 and PD-L1 was
enhanced indicators of survival in PDAC patients. The interaction of
macrophages and immune checkpoints implied the value of the combination
therapy.

doi: \url{https://doi.org/10.1097/MPA.0000000000001364}

{}

{}

\begin{center}\rule{0.5\linewidth}{\linethickness}\end{center}

\begin{center}\rule{0.5\linewidth}{\linethickness}\end{center}

 - \textbf{Correlation of DOTATOC Uptake and Pathologic Grade in
Neuroendocrine Tumors}

\emph{Pancreas 2019 Aug;48(7):948-952}

PubMed: \url{https://www.ncbi.nlm.nih.gov/pubmed/?term=31268975}

OBJECTIVES: Gallium (Ga)-DOTATOC is a somatostatin analog used to detect
neuroendocrine tumors (NETs). Ki-67 proliferation index (Ki-67 PI) has
been established as a prognostic factor in NETs. We aimed to evaluate
whether a correlation exists between Ki-67 PI and somatostatin receptor
positron emission tomography (SSTR-PET) uptake. METHODS: We
retrospectively reviewed 238 DOTATOC PET scans between 2014 and 2016.
Patients were excluded if DOTATOC PET was performed more than 365 days
from the date of biopsy. Maximum standardized uptake values (SUVmax) of
SSTR-PET from biopsied lesions were measured and correlated with Ki-67
PI using the Pearson correlation coefficient. RESULTS: Among 110 lesions
from 90 patients, DOTATOC PET had 92.7\% sensitivity and 100\%
specificity (102 true positives, 8 false negatives) for detection of
NETs. Among 63 lesions from 54 patients with Ki-67 PI available, there
were 27 grade 1 lesions {[}median Ki-67 PI, 1.0\%; interquartile range
(IQR), 1.0-2.0{]}, 30 grade 2 lesions (median, Ki-67 PI 7.5\%; IQR,
5-10), and 6 grade 3 lesions (median Ki-67 PI, 30\%; IQR, 26-34). There
was a correlation between Ki-67 PI and SUVmax (r = -0.3, P = 0.018).
CONCLUSIONS: Our analysis demonstrates an inverse correlation between
Ki-67 PI and SUVmax in NETs. Somatostatin receptor-PET provides
additional information that can help guide management of NETs.

doi: \url{https://doi.org/10.1097/MPA.0000000000001356}

{}

{}

\begin{center}\rule{0.5\linewidth}{\linethickness}\end{center}

\begin{center}\rule{0.5\linewidth}{\linethickness}\end{center}

 - \textbf{Diagnostic and Management Challenges in Vasoactive Intestinal
Peptide Secreting Tumors: A Series of 15 Patients}

\emph{Pancreas 2019 Aug;48(7):934-942}

PubMed: \url{https://www.ncbi.nlm.nih.gov/pubmed/?term=31268974}

OBJECTIVES: Vasoactive intestinal peptide-secreting tumors (VIPomas) are
rare functioning neuroendocrine tumors often characterized by a
difficult-to-control secretory syndrome and high potential to develop
metastases. We hereby present the characteristics of 15 cases of VIPomas
and provide a recent literature review. METHODS: This was a
retrospective data analysis of 15 patients with VIPoma from 3 different
centers and literature research through PubMed database during the last
10 years. RESULTS: Fifteen patients with VIPomas (9 with hepatic
metastases at diagnosis) with watery diarrhea and raised VIP levels were
studied. Ten patients (67\%) had grade 2 tumors, 6 of 15 had localized
disease and underwent potentially curative surgery, whereas the
remaining 9 received multiple systemic therapies; 3 patients died during
follow-up. The median overall survival was 71 months (range, 41-154
months). Patients who were treated with curative surgery (n = 7) had
longer median overall survival compared with patients who were treated
with other therapeutic modalities (44 vs 33 months). CONCLUSIONS: The
management of VIPomas is challenging requiring the application of
multiple treatment modalities. Patients who underwent surgical treatment
with curative intent appear to have higher survival rate. Central
registration and larger prospective studies are required to evaluate the
effect of currently employed therapies in these patients.

doi: \url{https://doi.org/10.1097/MPA.0000000000001347}

{}

{}

\begin{center}\rule{0.5\linewidth}{\linethickness}\end{center}

\begin{center}\rule{0.5\linewidth}{\linethickness}\end{center}

 - \textbf{Complications to Chronic Pancreatitis and Etiological Risk
Factors: A Continental Divide?}

\emph{The American journal of gastroenterology 2019 Aug;114(8):1353}

PubMed: \url{https://www.ncbi.nlm.nih.gov/pubmed/?term=31211705}

doi: \url{https://doi.org/10.14309/ajg.0000000000000302}

{}

{}

\begin{center}\rule{0.5\linewidth}{\linethickness}\end{center}

\begin{center}\rule{0.5\linewidth}{\linethickness}\end{center}

 - \textbf{Benchmark, Textbook or Optimal Pancreatic Surgery?}

\emph{Annals of surgery 2019 Aug;270(2):219-220}

PubMed: \url{https://www.ncbi.nlm.nih.gov/pubmed/?term=31188222}

doi: \url{https://doi.org/10.1097/SLA.0000000000003377}

{}

{}

\begin{center}\rule{0.5\linewidth}{\linethickness}\end{center}

\begin{center}\rule{0.5\linewidth}{\linethickness}\end{center}

 - \textbf{Comparisons of Outcomes of Real-World Patients With Advanced
Pancreatic Cancer Treated With FOLFIRINOX Versus Gemcitabine and
Nab-Paclitaxel: A Population-Based Cohort Study}

\emph{Pancreas 2019 Aug;48(7):920-926}

PubMed: \url{https://www.ncbi.nlm.nih.gov/pubmed/?term=31180981}

OBJECTIVES: The aim of this study was to compare the efficacy and safety
of FOLFIRINOX (5-FU/leucovorin, irinotecan, and oxaliplatin) and
gemcitabine/nab-paclitaxel (GnP) in patients with advanced pancreatic
cancer. METHODS: Patients with newly diagnosed advanced pancreatic
cancer in Saskatchewan, Canada, from 2011 to 2016, who received
FOLFIRINOX or GnP were assessed. A Cox proportional multivariate
analysis was performed to evaluate prognostic variables. RESULTS: One
hundred nineteen eligible patients with median age of 61 years and
male/female ratio of 70:49 were identified. Seventy-seven percent had
metastatic disease. Of 119 patients, 86 (72\%) received FOLFIRINOX and
33 (28\%) were treated with GnP. Median progression-free survival of the
FOLFIRINOX group was 6.0 months {[}95\% confidence interval (CI),
4.5-7.5{]} versus 4.0 months (95\% CI, 2.9-5.1) with GnP (P = 0.39). The
median overall survival of the FOLFIRINOX group was 9.0 months (95\% CI,
7-11) compared with 9.0 months (95\% CI, 4.2-13.8) with GnP (P = 0.88).
On multivariate analysis, albumin {[}hazard ratio (HR), 0.63; 95\% CI,
0.41-0.97{]}, male sex (HR, 0.65; 95\% CI, 0.43-0.97), and second-line
therapy (HR, 0.50; 95\% CI, 0.28-0.86) were correlated with survival.
CONCLUSIONS: Our results showed that real-world patients with advanced
pancreatic cancer treated with FOLFIIRNOX or GnP had comparable survival
with different safety profile.

doi: \url{https://doi.org/10.1097/MPA.0000000000001340}

{}

{}

\begin{center}\rule{0.5\linewidth}{\linethickness}\end{center}

\begin{center}\rule{0.5\linewidth}{\linethickness}\end{center}

 - \textbf{Differences in Pancreatic Cancer Incidence Rates and Temporal
Trends Across Asian Subpopulations in California (1988-2015)}

\emph{Pancreas 2019 Aug;48(7):931-933}

PubMed: \url{https://www.ncbi.nlm.nih.gov/pubmed/?term=31180980}

OBJECTIVE: Ethnic disparities in pancreatic cancer (PanCan) incidence
exist, but little is known about incidence trends in heterogeneous Asian
Americans. We examined PanCan incidence and temporal patterns among
detailed ethnic populations, including Asian American subgroups.
METHODS: A total of 71,099 invasive exocrine PanCan cases were
identified using the California Cancer Registry between 1988 and 2015.
Cases were grouped into mutually exclusive groups of non-Hispanic (NH)
white, NH black, Hispanic, NH Asian/Pacific Islander (API), and NH
American Indian/Alaska Native (AIAN). Asians were further identified by
specific ethnicity. RESULTS: The age-adjusted incidence rates (AAIRs,
per 100,000) of PanCan varied significantly across racial/ethnic groups,
ranging from the highest of 10.4 in NH blacks to 7.6 in NH whites, 7.1
in Hispanics, 6.2 in NH APIs, and to the lowest of 5.2 in NH AIAN.
Despite the relatively low rate in the NH APIs, the rates across Asian
subgroups varied significantly, with rates similar to NH whites in
Japanese (8.1) and Koreans (7.5) to the low rate in South Asians (4.4).
CONCLUSIONS: Significant heterogeneity of PanCan incidence in
disaggregated Asian Americans is a novel finding. These results fill a
gap regarding PanCan burden in Asian Americans and underscore the
importance of disaggregating ethnic populations in cancer research.

doi: \url{https://doi.org/10.1097/MPA.0000000000001337}

{}

{}

\begin{center}\rule{0.5\linewidth}{\linethickness}\end{center}

\begin{center}\rule{0.5\linewidth}{\linethickness}\end{center}

 - \textbf{Oncogenic NRG1 Fusions: A New Hope for Targeted Therapy in
Pancreatic Cancer}

\emph{Clinical cancer research : an official journal of the American
Association for Cancer Research 2019 Aug;25(15):4589-4591}

PubMed: \url{https://www.ncbi.nlm.nih.gov/pubmed/?term=31164372}

Approximately 8\%-10\% of pancreatic ductal adenocarcinoma cases are
KRAS wild type. In a subset of these tumors, NRG1 gene fusions have been
identified as targetable oncogenic drivers, a discovery that highlights
the importance of deep molecular characterization for KRAS wild-type
pancreatic cancers and provides a novel treatment strategy in this
disease.See related article by Jones et al., p.~4674.

doi: \url{https://doi.org/10.1158/1078-0432.CCR-19-1280}

{}

{}

\begin{center}\rule{0.5\linewidth}{\linethickness}\end{center}

\begin{center}\rule{0.5\linewidth}{\linethickness}\end{center}

 - \textbf{Maintenance Olaparib for Germline BRCA-Mutated Metastatic
Pancreatic Cancer}

\emph{The New England journal of medicine 2019 07;381(4):317-327}

PubMed: \url{https://www.ncbi.nlm.nih.gov/pubmed/?term=31157963}

BACKGROUND: Patients with a germline BRCA1 or BRCA2 mutation make up a
small subgroup of those with metastatic pancreatic cancer. The
poly(adenosine diphosphate-ribose) polymerase (PARP) inhibitor olaparib
has had antitumor activity in this population. METHODS: We conducted a
randomized, double-blind, placebo-controlled, phase 3 trial to evaluate
the efficacy of olaparib as maintenance therapy in patients who had a
germline BRCA1 or BRCA2 mutation and metastatic pancreatic cancer and
disease that had not progressed during first-line platinum-based
chemotherapy. Patients were randomly assigned, in a 3:2 ratio, to
receive maintenance olaparib tablets (300 mg twice daily) or placebo.
The primary end point was progression-free survival, which was assessed
by blinded independent central review. RESULTS: Of the 3315 patients who
underwent screening, 154 underwent randomization and were assigned to a
trial intervention (92 to receive olaparib and 62 to receive placebo).
The median progression-free survival was significantly longer in the
olaparib group than in the placebo group (7.4 months vs.~3.8 months;
hazard ratio for disease progression or death, 0.53; 95\% confidence
interval {[}CI{]}, 0.35 to 0.82; P = 0.004). An interim analysis of
overall survival, at a data maturity of 46\%, showed no difference
between the olaparib and placebo groups (median, 18.9 months vs.~18.1
months; hazard ratio for death, 0.91; 95\% CI, 0.56 to 1.46; P = 0.68).
There was no significant between-group difference in health-related
quality of life, as indicated by the overall change from baseline in the
global quality-of-life score (on a 100-point scale, with higher scores
indicating better quality of life) based on the European Organization
for Research and Treatment of Cancer Quality of Life Questionnaire
(between-group difference, -2.47 points; 95\% CI, -7.27 to 2.33). The
incidence of grade 3 or higher adverse events was 40\% in the olaparib
group and 23\% in the placebo group (between-group difference, 16
percentage points; 95\% CI, -0.02 to 31); 5\% and 2\% of the patients,
respectively, discontinued the trial intervention because of an adverse
event. CONCLUSIONS: Among patients with a germline BRCA mutation and
metastatic pancreatic cancer, progression-free survival was longer with
maintenance olaparib than with placebo. (Funded by AstraZeneca and
others; POLO ClinicalTrials.gov number, NCT02184195.).

doi: \url{https://doi.org/10.1056/NEJMoa1903387}

{}

{}

\begin{center}\rule{0.5\linewidth}{\linethickness}\end{center}

\begin{center}\rule{0.5\linewidth}{\linethickness}\end{center}

 - \textbf{Proton Radiotherapy for Isolated Local Recurrence of Primary
Resected Pancreatic Ductal Adenocarcinoma}

\emph{Annals of surgical oncology 2019 Aug;26(8):2587-2594}

PubMed: \url{https://www.ncbi.nlm.nih.gov/pubmed/?term=31147994}

BACKGROUND: The optimal treatment for isolated local recurrence (ILR) of
pancreatic adenocarcinoma (PDAC) after surgical resection remains
unclear. This study aimed to evaluate the safety and efficacy of proton
radiotherapy (PRT) for ILR of PDAC after surgery. METHODS: The medical
records of patients with ILR of PDAC after surgery who underwent proton
beam therapy between 2011 and 2015 at Hyogo Ion Beam Medical Center were
retrospectively studied. RESULTS: The study analyzed 30 patients (14
women and 16 men) with a median age of 65~years (range 38-81~years) who
had initially undergone pancreatoduodenectomy (n = 23) or distal
pancreatectomy (n = 7) for their primary tumors. Upon ILR, PRT was
administered with a median total cumulative dose of 67.5~gray equivalent
(GyE) (range 50-67.5~GyE) using 19 to 25 fractions. For 25 patients,
concurrent chemotherapy was administered using gemcitabine (n = 18) or
S-1 (n = 7). Four patients (13.3\%) experienced acute grade ≥ 3
gastrointestinal toxicities. After a median follow-up period of
17.6~months (range 2.1-50.4~months), 23 patients had experienced tumor
progression and 10 had died. Nine patients (30\%) experienced local
tumor progression. The median overall, progression-free, and local
progression-free survival rates were 26.1, 12.3, and 41.2~months,
respectively. Pre-PRT serum levels of cancer antigen 19-9 higher than
100~U/mL and duke pancreatic monoclonal antigen type 2 higher than
150~U/mL were significantly associated with shorter progression-free
survival rates. CONCLUSIONS: Proton radiotherapy for ILR of PDAC after
surgery is well tolerated and produces good locoregional control and
should be considered for eligible patients.

doi: \url{https://doi.org/10.1245/s10434-019-07471-z}

{}

{}

\begin{center}\rule{0.5\linewidth}{\linethickness}\end{center}

\begin{center}\rule{0.5\linewidth}{\linethickness}\end{center}

 - \textbf{Circulating Tumor DNA as a Clinical Test in Resected
Pancreatic Cancer}

\emph{Clinical cancer research : an official journal of the American
Association for Cancer Research 2019 Aug;25(16):4973-4984}

PubMed: \url{https://www.ncbi.nlm.nih.gov/pubmed/?term=31142500}

PURPOSE: In research settings, circulating tumor DNA (ctDNA) shows
promise as a tumor-specific biomarker for pancreatic ductal
adenocarcinoma (PDAC). This study aims to perform analytical and
clinical validation of a KRAS ctDNA assay in a Clinical Laboratory
Improvement Amendments (CLIA) and College of American
Pathology-certified clinical laboratory. EXPERIMENTAL DESIGN:
Digital-droplet PCR was used to detect the major PDAC-associated somatic
KRAS mutations (G12D, G12V, G12R, and Q61H) in liquid biopsies. For
clinical validation, 290 preoperative and longitudinal postoperative
plasma samples were collected from 59 patients with PDAC. The utility of
ctDNA status to predict PDAC recurrence during follow-up was assessed.
RESULTS: ctDNA was detected preoperatively in 29 (49\%) patients and was
an independent predictor of decreased recurrence-free survival (RFS) and
overall survival (OS). Patients who had neoadjuvant chemotherapy were
less likely to have preoperative ctDNA than were chemo-naïve patients
(21\% vs.~69\%; P \textless{} 0.001). ctDNA levels dropped significantly
after tumor resection. Persistence of ctDNA in the immediate
postoperative period was associated with a high rate of recurrence and
poor median RFS (5 months). ctDNA detected during follow-up predicted
clinical recurrence {[}sensitivity 90\% (95\% confidence interval (CI),
74\%-98\%), specificity 88\% (95\% CI, 62\%-98\%){]} with a median lead
time of 84 days (interquartile range, 25-146). Detection of ctDNA during
postpancreatectomy follow-up was associated with a median OS of 17
months, while median OS was not yet reached at 30 months for patients
without ctDNA (P = 0.011). CONCLUSIONS: Measurement of KRAS ctDNA in a
CLIA laboratory setting can be used to predict recurrence and survival
in patients with PDAC.

doi: \url{https://doi.org/10.1158/1078-0432.CCR-19-0197}

{}

{}

\begin{center}\rule{0.5\linewidth}{\linethickness}\end{center}

\begin{center}\rule{0.5\linewidth}{\linethickness}\end{center}

 - \textbf{Controversies on the endoscopic and surgical management of
pain in patients with chronic pancreatitis: pros and cons!}

\emph{Gut 2019 08;68(8):1343-1351}

PubMed: \url{https://www.ncbi.nlm.nih.gov/pubmed/?term=31129569}

doi: \url{https://doi.org/10.1136/gutjnl-2019-318742}

{}

{}

\begin{center}\rule{0.5\linewidth}{\linethickness}\end{center}

\begin{center}\rule{0.5\linewidth}{\linethickness}\end{center}

 - \textbf{Stromal hyaluronan accumulation is associated with low tumor
grade and nodal metastases in pancreatic ductal adenocarcinoma}

\emph{Human pathology 2019 Aug;90():37-44}

PubMed: \url{https://www.ncbi.nlm.nih.gov/pubmed/?term=31121193}

Pancreatic ductal adenocarcinoma is an aggressive malignancy
characterized by abundant desmoplastic stroma. Hyaluronan is a prominent
stromal component of pancreatic ductal adenocarcinoma and is associated
with unique clinical-pathological profiles in other tumor types. The
current study aimed to delineate clinical and pathological features
associated with hyaluronan accumulation in pancreatic ductal
adenocarcinoma using a novel hyaluronan-binding assay currently being
used in a clinical trial targeting hyaluronan. Sixty-four
formalin-fixed, paraffin-embedded samples of pancreatic ductal
adenocarcinomas from 49 patients treated at a single tertiary care
hospital were stained. Fifty-two percent of tumors had high levels of
hyaluronan. High levels were associated with low tumor grade and lymph
node metastases, novel associations not previously seen in pancreatic
ductal adenocarcinoma. This study has elucidated a novel
clinical-pathological profile in pancreatic ductal adenocarcinomas using
a new assay, suggesting hyaluronan may act as a biomarker for a subset
of pancreatic tumors that could be targeted by hyaluronan-degrading
agents.

doi: \url{https://doi.org/10.1016/j.humpath.2019.05.004}

{}

{}

\begin{center}\rule{0.5\linewidth}{\linethickness}\end{center}

\begin{center}\rule{0.5\linewidth}{\linethickness}\end{center}

 - \textbf{Arginine Starvation and Docetaxel Induce c-Myc-Driven hENT1
Surface Expression to Overcome Gemcitabine Resistance in ASS1-Negative
Tumors}

\emph{Clinical cancer research : an official journal of the American
Association for Cancer Research 2019 Aug;25(16):5122-5134}

PubMed: \url{https://www.ncbi.nlm.nih.gov/pubmed/?term=31113844}

PURPOSE: The response to acute and long-term arginine starvation results
in a conditional adaptive metabolic reprogramming that can be harnessed
for therapeutic opportunities in ASS1-negative tumors. Here, we
investigate the underlying biology of priming ASS1- tumors with arginine
deiminase (ADI-PEG20) before treatment with gemcitabine (GEM) and
docetaxel (DTX) in sarcoma, pancreatic cancer, and melanoma cell lines.
EXPERIMENTAL DESIGN: ASS1- tumor cell lines were treated to create LTAT
(long-term ADI treated) cell lines (ASS1+) and used for drug combination
studies. Protein expression of ASS1, dCK, RRM2, E2F1, c-MYC, and hENT1
was measured. c-MYC activity was determined, live-cell immunofluorescent
studies for hENT1, uptake assays of FITC-cytosine probe, and rescue
studies with a c-MYC inhibitor were all determined in the presence or
absence of the ADI-PEG20:GEM:DTX. RESULTS: In examining modulations
within the pyrimidine pathway, we identified that the addition of DTX to
cells treated with ADI-PEG20 resulted in translocation of stabilized
c-Myc to the nucleus. This resulted in an increase of hENT1 cell-surface
expression and rendered the cells susceptible to GEM. In vivo studies
demonstrate that the combination of ADI-PEG20:GEM:DTX was optimal for
tumor growth inhibition, providing the preclinical mechanism and
justification for the ongoing clinical trial of ADI-PEG20, GEM, and DTX
in sarcoma. CONCLUSIONS: The priming of tumors with ADI-PEG20 and DTX
results in the stabilization of c-MYC potentiating the effect of GEM
treatment via an increase in hENT1 expression. This finding is
applicable to ASS1-deficient cancers that are currently treated with
GEM.

doi: \url{https://doi.org/10.1158/1078-0432.CCR-19-0206}

{}

{}

\begin{center}\rule{0.5\linewidth}{\linethickness}\end{center}

\begin{center}\rule{0.5\linewidth}{\linethickness}\end{center}

 - \textbf{RAS Mutation Decreases Overall Survival After Optimal
Cytoreductive Surgery and Hyperthermic Intraperitoneal Chemotherapy of
Colorectal Peritoneal Metastasis: A Modification Proposal of the
Peritoneal Surface Disease Severity Score}

\emph{Annals of surgical oncology 2019 Aug;26(8):2595-2604}

PubMed: \url{https://www.ncbi.nlm.nih.gov/pubmed/?term=31111351}

BACKGROUND: Cytoreductive surgery (CRS) and hyperthermic intraperitoneal
chemotherapy (HIPEC) are currently the most accepted treatment for
peritoneal metastases from colorectal cancer. Restrictive selection
criteria are essential to obtain the best survival benefits for this
complex procedure. The most widespread score for patient selection, the
peritoneal surface disease severity score (PSDSS), does not include
current biological factors that are known to influence on prognosis. We
investigated the impact of including RAS mutational status in the
selection criteria for these patients. METHODS: We studied the risk
factors for survival by multivariate analysis using a prospective
database of consecutive patients with carcinomatosis from colorectal
origin treated by CRS and HIPEC in our unit from 2009 to 2017. The risk
factors obtained were validated in a multicentre, international cohort,
including a total of 520 patients from 15 different reference units.
RESULTS: A total of 77 patients were selected for local análisis. Only
RAS mutational status (HR: 2.024; p = 0.045) and PSDSS stage (HR: 2.90;
p = 0.009) were shown to be independent factors for overall survival.
Early PSDSS stages I and II associated to RAS mutations impaired their
overall survival with no significant differences with PSDSS stage III
overall survival (p \textgreater{} 0.05). These results were supported
by the international multicentre validation. CONCLUSIONS: By including
RAS mutational status, we propose an updated RAS-PSDSS score that
outperforms PSDSS alone providing a quick and feasible preoperative
assessment of the expected overall survival for patients with
carcinomatosis from colorectal origin undergone to CRS + HIPEC.

doi: \url{https://doi.org/10.1245/s10434-019-07378-9}

{}

{}

\begin{center}\rule{0.5\linewidth}{\linethickness}\end{center}

\begin{center}\rule{0.5\linewidth}{\linethickness}\end{center}

 - \textbf{Nivolumab alone or in combination with cisplatin plus
gemcitabine in Japanese patients with unresectable or recurrent biliary
tract cancer: a non-randomised, multicentre, open-label, phase 1 study}

\emph{The lancet. Gastroenterology \& hepatology 2019 Aug;4(8):611-621}

PubMed: \url{https://www.ncbi.nlm.nih.gov/pubmed/?term=31109808}

BACKGROUND: This study aimed to assess the safety and tolerability of
the immune checkpoint inhibitor nivolumab, as monotherapy or combined
with chemotherapy, in Japanese patients with biliary tract cancer.
METHODS: This multicentre, open-label, phase 1 trial was done at four
cancer centres in Japan. Eligible patients were aged 20-79 years, had
biliary tract adenocarcinoma (intrahepatic bile duct cancer,
extrahepatic bile duct cancer, gallbladder cancer, or ampullary cancer),
Eastern Cooperative Oncology Group performance status 0 or 1, adequate
hepatic, renal, and haematological function, and tumour tissue samples
for PD-L1 expression analysis. Patients with unresectable or recurrent
biliary tract cancer that was refractory or intolerant to
gemcitabine-based treatment regimens received nivolumab monotherapy (240
mg every 2 weeks {[}monotherapy cohort{]}). Chemotherapy-naive patients
with unresectable or recurrent biliary tract cancer received nivolumab
(240 mg every 2 weeks) and cisplatin (25 mg/m2) plus gemcitabine (1000
mg/m2) chemotherapy (combined therapy cohort). The primary objective was
to assess tolerability and safety. The primary objective was assessed in
the safety population of all patients who had received at least one dose
of nivolumab. This study is registered with www.clinicaltrials.jp,
number JapicCTI-153098, and follow-up is ongoing. FINDINGS: 30 patients
were enrolled into each cohort between Jan 13, 2016, and April 19, 2017.
Data cutoff was Aug 31, 2017. In the monotherapy cohort, the most
frequently reported treatment-related adverse events were decreased
appetite (five {[}17\%{]}), malaise (four {[}13\%{]}), and pruritus
(four {[}13\%{]}). Grade 3-4 treatment-related adverse events were
reported by three (10\%) patients (rash, maculopapular rash, and amylase
increase) and treatment-related serious adverse events were reported by
one (3\%) patient (pleurisy). In the combined therapy cohort, the most
frequently reported treatment-related adverse events were neutrophil
count decrease (any grade 25 {[}83\%{]}; grade 3-4 in 23 {[}77\%{]}
patients) and platelet count decrease (any grade 25 {[}83\%{]} of 30;
grade 3-4 in 15 {[}50\%{]} patients). Six (20\%) patients reported 11
treatment-related serious adverse events (platelet count decrease
{[}three patients{]}, febrile neutropenia {[}two patients{]}, neutrophil
count decrease, anaemia, anaphylactic reaction, decreased appetite,
pyrexia, and myocarditis {[}one patient each{]}). In the monotherapy
cohort, median overall survival was 5·2 months (90\% CI 4·5-8·7), median
progression-free survival was 1·4 months (90\% CI 1·4-1·4), and one of
30 patients had an objective response. In the combined therapy cohort,
median overall survival was 15·4 months (90\% CI 11·8-not estimable),
median progression-free survival was 4·2 months (90\% CI 2·8-5·6), and
11 of 30 patients had an objective response. INTERPRETATION: Nivolumab
had a manageable safety profile and signs of clinical activity in
patients with unresectable or recurrent biliary tract cancer. This
initial assessment of nivolumab for the treatment of advanced biliary
tract cancer provides supportive evidence for future larger randomised
studies of nivolumab in this difficult to treat cancer. FUNDING: Ono
Pharmaceutical Co Ltd and Bristol-Myers Squibb Inc.

doi: \url{https://doi.org/10.1016/S2468-1253(19)30086-X}

{}

{}

\begin{center}\rule{0.5\linewidth}{\linethickness}\end{center}

\begin{center}\rule{0.5\linewidth}{\linethickness}\end{center}

 - \textbf{NRG1 Gene Fusions Are Recurrent, Clinically Actionable Gene
Rearrangements in KRAS Wild-Type Pancreatic Ductal Adenocarcinoma}

\emph{Clinical cancer research : an official journal of the American
Association for Cancer Research 2019 Aug;25(15):4674-4681}

PubMed: \url{https://www.ncbi.nlm.nih.gov/pubmed/?term=31068372}

PURPOSE: Gene fusions involving neuregulin 1 (NRG1) have been noted in
multiple cancer types and have potential therapeutic implications.
Although varying results have been reported in other cancer types, the
efficacy of the HER-family kinase inhibitor afatinib in the treatment of
NRG1 fusion-positive pancreatic ductal adenocarcinoma is not fully
understood. EXPERIMENTAL DESIGN: Forty-seven patients with pancreatic
ductal adenocarcinoma received comprehensive whole-genome and
transcriptome sequencing and analysis. Two patients with gene fusions
involving NRG1 received afatinib treatment, with response measured by
pretreatment and posttreatment PET/CT imaging. RESULTS: Three of 47
(6\%) patients with advanced pancreatic ductal adenocarcinoma were
identified as KRAS wild type by whole-genome sequencing. All KRAS
wild-type tumors were positive for gene fusions involving the ERBB3
ligand NRG1. Two of 3 patients with NRG1 fusion-positive tumors were
treated with afatinib and demonstrated a significant and rapid response
while on therapy. CONCLUSIONS: This work adds to a growing body of
evidence that NRG1 gene fusions are recurrent, therapeutically
actionable genomic events in pancreatic cancers. Based on the clinical
outcomes described here, patients with KRAS wild-type tumors harboring
NRG1 gene fusions may benefit from treatment with afatinib.See related
commentary by Aguirre, p.~4589.

doi: \url{https://doi.org/10.1158/1078-0432.CCR-19-0191}

{}

{}

\begin{center}\rule{0.5\linewidth}{\linethickness}\end{center}

\begin{center}\rule{0.5\linewidth}{\linethickness}\end{center}

 - \textbf{Contemporary Improvements in Postoperative Mortality After
Major Cancer Surgery are Associated with Weakening of the Volume-Outcome
Association}

\emph{Annals of surgical oncology 2019 Aug;26(8):2348-2356}

PubMed: \url{https://www.ncbi.nlm.nih.gov/pubmed/?term=31065959}

BACKGROUND: Regionalization of complex visceral surgery across the
United States has followed identification of a volume-outcome
association. However, improvements in postoperative mortality overall
during the last decade may have weakened the strength of this
association. METHODS: The National Cancer Database was used to identify
patients undergoing colon, esophageal, liver, and pancreatic surgery
from 2003 to 2011. Hospitals were divided into low-volume
(\textless{} 33rd \%tile), medium-volume (34-66th \%tile), and
high-volume (\textgreater{} 67th \%tile) groups. Annual cancer-specific
adjusted observed versus expected (O/E) ratios for 30- and 90-day
mortality for each volume strata were calculated and plotted over time.
RESULTS: In the year 2003, the O/E ratios decreased from low- to medium-
to high-volume hospitals for all cancer surgeries for both 30- and
90-day mortality, indicating a strong volume-outcome relationship. For
all volume strata, the O/E ratios trended downward from 2003 to 2011 for
both 30- and 90-day mortality for all cancer surgeries. This trend was
more pronounced for low- and medium-volume than for high-volume
hospitals. Consequently, by 2011 the confidence intervals of the O/E
ratios for the low-volume groups, and particularly for the medium-volume
groups, overlapped those for the high-volume groups for most of the
cancer surgeries studied. CONCLUSIONS: The volume-outcome association
for major cancer surgery is dynamic and has attenuated over time
primarily due to improvements in postoperative mortality at low- and
medium-volume hospitals.

doi: \url{https://doi.org/10.1245/s10434-019-07413-9}

{}

{}

\begin{center}\rule{0.5\linewidth}{\linethickness}\end{center}

\begin{center}\rule{0.5\linewidth}{\linethickness}\end{center}

 - \textbf{Mesothelin-Targeted Thorium-227 Conjugate (MSLN-TTC):
Preclinical Evaluation of a New Targeted Alpha Therapy for
Mesothelin-Positive Cancers}

\emph{Clinical cancer research : an official journal of the American
Association for Cancer Research 2019 Aug;25(15):4723-4734}

PubMed: \url{https://www.ncbi.nlm.nih.gov/pubmed/?term=31064781}

PURPOSE: Targeted thorium-227 conjugates (TTC) represent a new class of
molecules for targeted alpha therapy (TAT). Covalent attachment of a
3,2-HOPO chelator to an antibody enables specific complexation and
delivery of the alpha particle emitter thorium-227 to tumor cells.
Because of the high energy and short penetration range, TAT efficiently
induces double-strand DNA breaks (DSB) preferentially in the tumor cell
with limited damage to the surrounding tissue. We present herein the
preclinical evaluation of a mesothelin (MSLN)-targeted thorium-227
conjugate, BAY 2287411. MSLN is a GPI-anchored membrane glycoprotein
overexpressed in mesothelioma, ovarian, pancreatic, lung, and breast
cancers with limited expression in healthy tissue. EXPERIMENTAL DESIGN:
The binding activity and radiostability of BAY 2287411 were confirmed
bioanalytically. The mode-of-action and antitumor potency of BAY 2287411
were investigated in vitro and in vivo in cell line and patient-derived
xenograft models of breast, colorectal, lung, ovarian, and pancreatic
cancer. RESULTS: BAY 2287411 induced DSBs, apoptotic markers, and
oxidative stress, leading to reduced cellular viability. Furthermore,
upregulation of immunogenic cell death markers was observed. BAY 2287411
was well-tolerated and demonstrated significant antitumor efficacy when
administered via single or multiple dosing regimens in vivo. In
addition, significant survival benefit was observed in a disseminated
lung cancer model. Biodistribution studies showed specific uptake and
retention of BAY 2287411 in tumors and enabled the development of a
mechanistic pharmacokinetic/pharmacodynamic model to describe the
preclinical data. CONCLUSIONS: These promising preclinical results
supported the transition of BAY 2287411 into a clinical phase I program
in mesothelioma and ovarian cancer patients (NCT03507452).

doi: \url{https://doi.org/10.1158/1078-0432.CCR-18-3476}

{}

{}

\begin{center}\rule{0.5\linewidth}{\linethickness}\end{center}

\begin{center}\rule{0.5\linewidth}{\linethickness}\end{center}

 - \textbf{Advanced stage at diagnosis and elevated mortality among US
patients with cancer infected with HIV in the National Cancer Data Base}

\emph{Cancer 2019 Aug;125(16):2868-2876}

PubMed: \url{https://www.ncbi.nlm.nih.gov/pubmed/?term=31050361}

BACKGROUND: People living with HIV (PLWH) are at an increased risk of
developing several cancers, but to the authors' knowledge less is known
regarding how HIV impacts the rate of progression to advanced cancer or
death. METHODS: The authors compared stage of disease at the time of
presentation and mortality after diagnosis between 14,453 PLWH and
6,368,126 HIV-uninfected patients diagnosed with cancers of the oral
cavity, stomach, colorectum, anus, liver, pancreas, lung, female breast,
cervix, prostate, bladder, kidney, and thyroid and melanoma using data
from the National Cancer Data Base (2004-2014). Polytomous logistic
regression and Cox proportional hazards regression were used to evaluate
the association between HIV, cancer stage, and stage-adjusted mortality
after diagnosis, respectively. Regression models accounted for the type
of health facility at which cancer treatment was administered and the
type of individual health insurance. RESULTS: HIV-infected patients with
cancer were found to be more likely to be uninsured (HIV-infected: 5.0\%
vs HIV-uninfected: 3.3\%; P~\textless{}~.0001) and were less likely to
have private health insurance (25.4\% vs 44.7\%; P~\textless{}~.0001).
Compared with those not infected with HIV, the odds of being diagnosed
at an advanced stage of disease were significantly elevated in PLWH for
melanoma and cancers of the oral cavity, liver, female breast, prostate,
and thyroid (odds ratio for stage IV vs stage I range, 1.24-2.06). PLWH
who were diagnosed with stage I to stage III disease experienced
elevated mortality after diagnosis across 13 of the 14 cancer sites
evaluated, with hazard ratios ranging from 1.20 (95\% CI, 1.14-1.26) for
lung cancer to 1.85 (95\% CI, 1.68-2.04), 1.85 (95\% CI, 1.51-2.27), and
2.93 (95\% CI, 2.08-4.13), respectively, for cancers of the female
breast, cervix, and thyroid. CONCLUSIONS: PLWH were more likely to be
diagnosed with advanced-stage cancers and to experience elevated
mortality after a cancer diagnosis, even after accounting for health
care-related factors.

doi: \url{https://doi.org/10.1002/cncr.32158}

{}

{}

\begin{center}\rule{0.5\linewidth}{\linethickness}\end{center}

\begin{center}\rule{0.5\linewidth}{\linethickness}\end{center}

 - \textbf{GNAS but Not Extended RAS Mutations Spectrum are Associated
with a Better Prognosis in Intraductal Pancreatic Mucinous Neoplasms}

\emph{Annals of surgical oncology 2019 Aug;26(8):2640-2650}

PubMed: \url{https://www.ncbi.nlm.nih.gov/pubmed/?term=31025231}

BACKGROUND: The management of intraductal papillary mucinous neoplasms
(IPMNs) is mainly based on imaging features and clinical symptoms, and
remains challenging. OBJECTIVE: The aim of this study was to assess
GNAS, RAS family (KRAS, NRAS and HRAS), BRAF, and PIK3CA mutation status
in resected IPMNs and correlate it with clinicopathological
characteristics and patient survival. METHODS: Overall, 149 consecutive
unselected patients who underwent pancreatectomy for IPMNs were
included. After dissection from formalin-fixed and paraffin-embedded
tumors, GNAS mutational screening was assessed by allelic discrimination
using Taqman® probes and confirmed by SNaPshot analysis. RAS family,
BRAF, and PIK3CA mutational screening was assessed by high resolution
melt and Sanger sequencing. RESULTS: Gastric- and intestinal-type IPMNs
were the most frequent lesions (52\% and 41\%, respectively).
Intestinal-type IPMNs were more frequently associated high-grade
dysplasia (49\%) and were the only IPMNs associated with colloid-type
carcinoma. All pancreatobiliary IPMNs were invasive lesions, located in
the main pancreatic duct. GNAS-activating mutations were strongly
associated with the intestinal phenotype (p \textless{} 10-4), while RAS
pathway mutations were not associated with any particular phenotype.
Mutations within other members of the epidermal growth factor receptor
(EGFR) pathway were very rare (2\%). GNAS-mutated IPMNs were rarely
invasive (11\%) and almost exclusively (83\%) of the colloid type. For
invasive lesions, multivariate analyses determined that only node
negativity was associated with improved cancer-specific survival, but,
in univariate analysis, GNAS mutation was associated with prolonged
survival. CONCLUSION: In patients selected for surgery, GNAS mutation
analysis and tumor phenotype can help to better predict patient
prognosis. In the near future, a more precise mutational analysis of
IPMNs might help to better tailor their management.

doi: \url{https://doi.org/10.1245/s10434-019-07389-6}

{}

{}

\begin{center}\rule{0.5\linewidth}{\linethickness}\end{center}

\begin{center}\rule{0.5\linewidth}{\linethickness}\end{center}

 - \textbf{The Pancreas as a Site of Metastasis or Second Primary in
Patients with Small Bowel Neuroendocrine Tumors}

\emph{Annals of surgical oncology 2019 Aug;26(8):2525-2532}

PubMed: \url{https://www.ncbi.nlm.nih.gov/pubmed/?term=31011904}

BACKGROUND: The small bowel and pancreas are the most common primary
sites of neuroendocrine tumors (NETs) giving rise to metastatic disease.
Some patients with small bowel NETs (SBNETs) present with synchronous or
metachronous pancreatic NETs (PNETs), and it is unclear whether these
are separate primaries or metastases from one site to the other.
METHODS: A surgical NET database including patients undergoing
operations for SBNETs or PNETs was reviewed. Patients with synchronous
or metachronous tumors in both the small bowel and pancreas were
identified, and available tissues from primary tumors and metastases
were examined using a 4-gene quantitative polymerase chain reaction
(qPCR) and immunohistochemistry (IHC) panel developed for evaluating
NETs of unknown primary. RESULTS: Of 338 patients undergoing
exploration, 11 had NETs in both the small bowel and pancreas. Tissues
from 11 small bowel tumors, 9 pancreatic tumors, and 10 metastases were
analyzed. qPCR and IHC data revealed that three patients had separate
SBNET and PNET primaries, and five patients had SBNETs that metastasized
to the pancreas. Pancreatic tissue was unavailable in two patients, and
qPCR and IHC gave discrepant results in one patient. CONCLUSIONS: NETs
in both the small bowel and pancreas were found in 3\% of our patients.
In nearly two-thirds of evaluable patients, the pancreatic tumor was a
metastasis from the SBNET primary, while in the remaining one-third of
patients it represented a separate primary. Determining the origin of
these tumors can help guide the choice of systemic therapy and surgical
management.

doi: \url{https://doi.org/10.1245/s10434-019-07370-3}

{}

{}

\begin{center}\rule{0.5\linewidth}{\linethickness}\end{center}

\begin{center}\rule{0.5\linewidth}{\linethickness}\end{center}

 - \textbf{ASO Author Reflections: Tending Towards a Personalized
Medicine for Colorectal Carcinomatosis by Adding the RAS Mutation Status
in the Workup for CRS and HIPEC}

\emph{Annals of surgical oncology 2019 Aug;26(8):2605-2606}

PubMed: \url{https://www.ncbi.nlm.nih.gov/pubmed/?term=31004296}

doi: \url{https://doi.org/10.1245/s10434-019-07362-3}

{}

{}

\begin{center}\rule{0.5\linewidth}{\linethickness}\end{center}

\begin{center}\rule{0.5\linewidth}{\linethickness}\end{center}

 - \textbf{Defining the Role of Lymphadenectomy for Pancreatic
Neuroendocrine Tumors: An Eight-Institution Study of 695 Patients from
the US Neuroendocrine Tumor Study Group}

\emph{Annals of surgical oncology 2019 Aug;26(8):2517-2524}

PubMed: \url{https://www.ncbi.nlm.nih.gov/pubmed/?term=31004295}

BACKGROUND: Preoperative factors that reliably predict lymph node (LN)
metastases in pancreatic neuroendocrine tumors (PanNETs) are unclear.
The number of LNs needed to accurately stage PanNETs has not been
defined. METHODS: Patients who underwent curative-intent resection of
non-functional PanNETs at eight institutions from 2000 to 2016 were
analyzed. Preoperative factors associated with LN metastases were
identified. A procedure-specific target for LN retrieval to accurately
stage patients was determined. RESULTS: Of 695 patients who underwent
resection, 33\% of tumors were proximal (head/uncinate) and 67\% were
distal (neck/body/tail). Twenty-six percent of patients (n = 158) had
LN-positive disease, which was associated with a worse 5-year
recurrence-free survival (RFS; 60\% vs.~86\%; p \textless{} 0.001). The
increasing number of positive LNs was not associated with worse RFS.
Preoperative factors associated with positive LNs included tumor
size ≥ 2~cm (odds ratio {[}OR{]} 6.6; p \textless{} 0.001), proximal
location (OR 2.5; p \textless{} 0.001), moderate versus
well-differentiation (OR 2.1; p = 0.006), and Ki-67 ≥ 3\% (OR 3.1;
p \textless{} 0.001). LN metastases were also present in tumors without
these risk factors: \textless{} 2~cm (9\%), distal location (19\%),
well-differentiated (23\%), and Ki-67 \textless{} 3\% (16\%). Median LN
retrieval was 13 for pancreatoduodenectomy (PD), but only 9 for distal
pancreatectomy (DP). Given that PD routinely includes a complete
regional lymphadenectomy, a minimum number of LNs to accurately stage
patients was not identified. However, for DP, removal of less than seven
LNs failed to discriminate 5-year RFS between LN-positive and
LN-negative patients (less than seven LNs: 72\% vs.~83\%, p = 0.198;
seven or more LNs: 67\% vs.~86\%; p = 0.002). CONCLUSIONS: Tumor
size ≥ 2~cm, proximal location, moderate differentiation, and
Ki-67 ≥ 3\% are preoperative factors that predict LN positivity in
resected non-functional PanNETs. Given the 9-23\% incidence of LN
metastases in patients without such risk factors, routine regional
lymphadenectomy should be considered. PD inherently includes sufficient
LN retrieval, while DP should aim to remove seven or more LNs for
accurate staging.

doi: \url{https://doi.org/10.1245/s10434-019-07367-y}

{}

{}

\begin{center}\rule{0.5\linewidth}{\linethickness}\end{center}

\begin{center}\rule{0.5\linewidth}{\linethickness}\end{center}

 - \textbf{Detection of NRG1 Gene Fusions in Solid Tumors}

\emph{Clinical cancer research : an official journal of the American
Association for Cancer Research 2019 Aug;25(16):4966-4972}

PubMed: \url{https://www.ncbi.nlm.nih.gov/pubmed/?term=30988082}

PURPOSE: NRG1 gene fusions are rare but potentially actionable oncogenic
drivers that are present in some solid tumors. Details regarding the
incidence of these gene rearrangements are lacking. Here, we assessed
the incidence of NRG1 fusions across multiple tumor types and described
fusion partners. EXPERIMENTAL DESIGN: Tumor specimens submitted for
molecular profiling at a Clinical Laboratory Improvement Amendments
(CLIA)-certified genomics laboratory and that underwent fusion testing
by anchored multiplex PCR for targeted RNA sequencing were
retrospectively identified. The overall and tumor-specific incidence was
noted, as was the specific fusion partner. RESULTS: Out of 21,858 tumor
specimens profiled from September 2015 to December 2018, 41 cases
(0.2\%) harbored an NRG1 fusion. Multiple fusion partners were
identified. Fusion events were seen across tumor types. The greatest
incidence was in non-small cell lung cancer (NSCLC, 25), though this
represented only 0.3\% of NSCLC cases tested. Other tumor types
harboring an NRG1 fusion included gallbladder cancer, renal cell
carcinoma, bladder cancer, ovarian cancer, pancreatic cancer, breast
cancer, neuroendocrine tumor, sarcoma, and colorectal cancer.
CONCLUSIONS: NRG1 fusions can be detected at a low incidence across
multiple tumor types with significant heterogeneity in fusion
partner.See related commentary by Dimou and Camidge, p.~4865.

doi: \url{https://doi.org/10.1158/1078-0432.CCR-19-0160}

{}

{}

\begin{center}\rule{0.5\linewidth}{\linethickness}\end{center}

\begin{center}\rule{0.5\linewidth}{\linethickness}\end{center}

 - \textbf{Tspan8 is expressed in breast cancer and regulates
E-cadherin/catenin signalling and metastasis accompanied by increased
circulating extracellular vesicles}

\emph{The Journal of pathology 2019 Aug;248(4):421-437}

PubMed: \url{https://www.ncbi.nlm.nih.gov/pubmed/?term=30982971}

Tspan8 exhibits a functional role in many cancer types including
pancreatic, colorectal, oesophagus carcinoma, and melanoma. We present a
first study on the expression and function of Tspan8 in breast cancer.
Tspan8 protein was present in the majority of human primary breast
cancer lesions and metastases in the brain, bone, lung, and liver. In a
syngeneic rat breast cancer model, Tspan8+ tumours formed multiple liver
and spleen metastases, while Tspan8- tumours exhibited a significantly
diminished ability to metastasise, indicating a role of Tspan8 in
metastases. Addressing the underlying molecular mechanisms, we
discovered that Tspan8 can mediate up-regulation of E-cadherin and
down-regulation of Twist, p120-catenin, and β-catenin target genes
accompanied by the change of cell phenotype, resembling the
mesenchymal-epithelial transition. Furthermore, Tspan8+ cells exhibited
enhanced cell-cell adhesion, diminished motility, and decreased
sensitivity to irradiation. As a regulator of the content and function
of extracellular vesicles (EVs), Tspan8 mediated a several-fold increase
in EV number in cell culture and the circulation of tumour-bearing
animals. We observed increased protein levels of E-cadherin and
p120-catenin in these EVs; furthermore, Tspan8 and p120-catenin were
co-immunoprecipitated, indicating that they may interact with each
other. Altogether, our findings show the presence of Tspan8 in breast
cancer primary lesion and metastases and indicate its role as a
regulator of cell behaviour and EV release in breast cancer. © 2019 The
Authors. The Journal of Pathology published by John Wiley \& Sons Ltd on
behalf of Pathological Society of Great Britain and Ireland.

doi: \url{https://doi.org/10.1002/path.5281}

{}

{}

\begin{center}\rule{0.5\linewidth}{\linethickness}\end{center}

\begin{center}\rule{0.5\linewidth}{\linethickness}\end{center}

 - \textbf{Laparoscopic Complete Mesocolic Excision for Double Flexural
Colon Cancers}

\emph{Annals of surgical oncology 2019 Aug;26(8):2516}

PubMed: \url{https://www.ncbi.nlm.nih.gov/pubmed/?term=30927197}

BACKGROUND: Laparoscopic complete mesocolic excision (CME) for hepatic
or splenic flexural colon cancer is considered technically demanding.
The double (hepatic and splenic) flexural colon cancers are rare, and
the laparoscopic CME procedure for such disease is not standardized.
METHODS: This video presents laparoscopic CME for double (hepatic and
splenic) flexural colon cancers using a medial and cranial approach.
RESULTS: The patient was a 60-year-old woman with the diagnosis of
splenic flexure cancer (cT4N1M0) and hepatic flexure cancer (cT3N0M0).
Laparoscopic subtotal colectomy was performed. First, the left colic
artery was divided at its origin, and the inferior mesenteric vein also
was divided at the same level. The descending mesocolon was widely
separated from the retroperitoneal tissues using a medial approach.
Then, lymph node dissection along the surgical trunk was performed using
a cranial approach. Finally, the transverse mesocolon was divided at the
inferior border of the pancreas, and CME was achieved. The specimen was
extracted through a small incision at the umbilicus, and side-to-side
ileo-sigmoid anastomosis was performed extracorporeally. CONCLUSIONS:
The approach presented in the video might be useful for standardization
of laparoscopic CME for double flexural colon cancers.

doi: \url{https://doi.org/10.1245/s10434-019-07329-4}

{}

{}

\begin{center}\rule{0.5\linewidth}{\linethickness}\end{center}

\begin{center}\rule{0.5\linewidth}{\linethickness}\end{center}

 - \textbf{Response to Comment on ``Letter to Editor Re Manuscript by
Bannone et al.'' Ann Surg. 2018 Dec 20}

\emph{Annals of surgery 2019 Aug;270(2):e60-e61}

PubMed: \url{https://www.ncbi.nlm.nih.gov/pubmed/?term=30870179}

doi: \url{https://doi.org/10.1097/SLA.0000000000003259}

{}

{}

\begin{center}\rule{0.5\linewidth}{\linethickness}\end{center}

\begin{center}\rule{0.5\linewidth}{\linethickness}\end{center}

 - \textbf{Clear Cell Variant of Solid Pseudopapillary Neoplasm of the
Pancreas: A Report of a Rare Variant and Review of the Literature}

\emph{International journal of surgical pathology 2019
Aug;27(5):535-540}

PubMed: \url{https://www.ncbi.nlm.nih.gov/pubmed/?term=30845855}

The clear cell variant of solid pseudopapillary neoplasm (ccSPN) of the
pancreas was first described in 2006. In this article, we report a case
of this rare variant and review the few published reports. Both the
current and previous reports show that ccSPN has several morphologic
differences from conventional SPN, including clear vacuoles, fewer
pseudopapillary formations, more solid/diffuse architecture, less
hemorrhage, and fewer cholesterol clefts. Some of these features
peculiar to ccSPN, such as solid/diffuse architecture, have been
proposed to suggest aggressive behavior, though reports of ccSPN are
rare and often have limited clinical follow-up. ccSPN also appears to
occur more frequently in males than conventional SPNs. These clinical
and pathologic features lead to unique set of differential diagnostic
considerations for ccSPN, including metastatic renal cell carcinoma,
perivascular epithelial cell tumor, and clear cell variants of other
carcinomas. These unique features, atypical differential, and uncertain
prognostic ramifications all make ccSPN an important variant to be aware
of and report.

doi: \url{https://doi.org/10.1177/1066896919833790}

{}

{}

\begin{center}\rule{0.5\linewidth}{\linethickness}\end{center}

\begin{center}\rule{0.5\linewidth}{\linethickness}\end{center}

 - \textbf{Benchmarks in Pancreatic Surgery: A Novel Tool for Unbiased
Outcome Comparisons}

\emph{Annals of surgery 2019 Aug;270(2):211-218}

PubMed: \url{https://www.ncbi.nlm.nih.gov/pubmed/?term=30829701}

OBJECTIVE: To use the concept of benchmarking to establish robust and
standardized outcome references after pancreatico-duodenectomy (PD).
BACKGROUND: Best achievable results after PD are unknown. Consequently,
outcome comparisons among different cohorts, centers or with novel
surgical techniques remain speculative. METHODS: This multicenter study
analyzes consecutive patients (2012-2015) undergoing PD in 23
international expert centers in pancreas surgery. Outcomes in patients
without significant comorbidities and major vascular resection
(benchmark cases) were analyzed to establish 20 outcome benchmarks for
PD. These benchmarks were tested in a cohort with a poorer preoperative
physical status (ASA class ≥3) and a cohort treated by minimally
invasive approaches. RESULTS: Two thousand three hundred seventy-five
(38\%) low-risk cases out of a total of 6186 PDs were analyzed,
disclosing low in-hospital mortality (≤1.6\%) but high morbidity, with a
73\% benchmark morbidity rate cumulated within 6 months following
surgery. Benchmark cutoffs for pancreatic fistulas (B-C), severe
complications (≥ grade 3), and failure-to-rescue rate were 19\%, 30\%,
and 9\%, respectively. The ASA ≥3 cohort showed comparable morbidity but
a higher in hospital-mortality (3\% vs 1.6\%) and failure-to-rescue rate
(16\% vs 9\%) than the benchmarks. The proportion of benchmark cases
performed varied greatly across centers and continents for both open
(9\%-93\%) and minimally invasive (11\%-62\%) PD. Centers operating
mostly on complex PD cases disclosed better results than those with a
majority of low-risk cases. CONCLUSION: The proposed outcome benchmarks
for PD, established in a large-scale international patient cohort and
tested in 2 different cohorts, may allow for meaningful comparisons
between different patient cohorts, centers, countries, and surgical
techniques.

doi: \url{https://doi.org/10.1097/SLA.0000000000003223}

{}

{}

\begin{center}\rule{0.5\linewidth}{\linethickness}\end{center}

\begin{center}\rule{0.5\linewidth}{\linethickness}\end{center}

 - \textbf{The Impact of Preoperative Immune Modulating Nutrition on
Outcomes in Patients Undergoing Surgery for Gastrointestinal Cancer: A
Systematic Review and Meta-analysis}

\emph{Annals of surgery 2019 Aug;270(2):247-256}

PubMed: \url{https://www.ncbi.nlm.nih.gov/pubmed/?term=30817349}

OBJECTIVE: To define the influence of preoperative immune modulating
nutrition (IMN) on postoperative outcomes in patients undergoing surgery
for gastrointestinal cancer. BACKGROUND: Although studies have shown
that perioperative IMN may reduce postoperative infectious
complications, many of these have included patients with benign and
malignant disease, and the optimal timing of such an intervention is not
clear. METHODS: The Embase, Medline, and Cochrane databases were
searched from 2000 to 2018, for prospective randomized controlled trials
evaluating preoperative oral or enteral IMN in patients undergoing
surgery for gastrointestinal cancer. The primary endpoint was the
development of postoperative infectious complications. Secondary
endpoints included postoperative noninfectious complications, length of
stay, and up to 30-day mortality. The analysis was performed using
RevMan v5.3 software. RESULTS: Sixteen studies reporting on 1387
patients (715 IMN group, 672 control group) were included. Six of the
included studies reported on a mixed population of patients undergoing
all gastrointestinal cancer surgery. Of the remaining, 4 investigated
IMN in colorectal cancer surgery, 2 in pancreatic surgery, and another 2
in patients undergoing surgery for gastric cancer. There was 1 study
each on liver and esophageal cancer. The formulation of nutrition used
in all studies in the treated patients was Impact (Novartis/Nestlé),
which contains ω-3 fatty acids, arginine, and nucleotides. Preoperative
IMN in patients undergoing surgery for gastrointestinal cancer reduced
infectious complications {[}odds ratio (OR) 0.52, 95\% confidence
interval (CI) 0.38-0.71, P \textless{} 0.0001, I = 16\%, n = 1387{]} and
length of hospital stay (weighted mean difference -1.57 days, 95\% CI
-2.48 to -0.66, P = 0.0007, I = 34\%, n = 995) when compared with
control (isocaloric isonitrogeneous feed or normal diet). It, however,
did not affect noninfectious complications (OR 0.98, 95\% CI 0.73-1.33,
P = 0.91, I = 0\%, n = 1303) or mortality (OR 0.55, 95\% CI 0.18-1.68, P
= 0.29, I = 0\%, n = 955). CONCLUSION: Given the significant impact on
infectious complications and a tendency to shorten length of stay,
preoperative IMN should be encouraged in routine practice in patients
undergoing surgery for gastrointestinal cancer.

doi: \url{https://doi.org/10.1097/SLA.0000000000003256}

{}

{}

\begin{center}\rule{0.5\linewidth}{\linethickness}\end{center}

\begin{center}\rule{0.5\linewidth}{\linethickness}\end{center}

 - \textbf{{[}Microcystic serous cystadenoma: An uncommon neoplasm of
pancreas. Report of two cases{]}}

\emph{Annales de pathologie 2019 Aug;39(4):292-296}

PubMed: \url{https://www.ncbi.nlm.nih.gov/pubmed/?term=30711334}

Microcystic variant of serous cystadenoma of the pancreas is a rare
neoplasm; essentially located in the body or tail of the pancreas and
associated with the von Hippel-Lindau. Often, patients are asymptomatic
and the neoplasm is incidentally discovered. Usually radiographic
manifestations are characteristic. Histopathological examination
revealed uniform clear cuboidal cells; they can be confused with other
clear cell neoplasms like renal cell carcinomas, well-differentiated
neuroendocrine tumors and solid pseudopapillary tumors of the pancreas.
Immunohistochemistry can be help to establish the diagnosis and to
remove differential diagnosis. Serous cystadenoma is a benign neoplasm
whose prognosis is excellent. We herein report two cases of microcystic
serous cystadenomas of the pancreas diagnosed in two asymptomatic women
and review analysis in the literature to remind the main features of
this lesion and the main differential diagnosis.

doi: \url{https://doi.org/10.1016/j.annpat.2018.12.007}

{}

{}

\begin{center}\rule{0.5\linewidth}{\linethickness}\end{center}

\begin{center}\rule{0.5\linewidth}{\linethickness}\end{center}

 - \textbf{Response to Comment on ``Characterization and Optimal
Management of High-risk Pancreatic Anastomoses During
Pancreatoduodenectomy: Response to Goussous and Cunningham''}

\emph{Annals of surgery 2019 Aug;270(2):e58-e59}

PubMed: \url{https://www.ncbi.nlm.nih.gov/pubmed/?term=30499813}

doi: \url{https://doi.org/10.1097/SLA.0000000000003121}

{}

{}

\begin{center}\rule{0.5\linewidth}{\linethickness}\end{center}

\begin{center}\rule{0.5\linewidth}{\linethickness}\end{center}

 - \textbf{Comment on ``Interpreting Clinical Benefits of Neoadjuvant
Chemoradiation With Gemcitabine Versus Upfront Surgery in Patients With
Borderline Resectable Pancreatic Cancer (BRPC)''}

\emph{Annals of surgery 2019 Aug;270(2):e48-e50}

PubMed: \url{https://www.ncbi.nlm.nih.gov/pubmed/?term=30499804}

doi: \url{https://doi.org/10.1097/SLA.0000000000003115}

{}

{}

\begin{center}\rule{0.5\linewidth}{\linethickness}\end{center}

\begin{center}\rule{0.5\linewidth}{\linethickness}\end{center}

 - \textbf{DYRK1A modulates c-MET in pancreatic ductal adenocarcinoma to
drive tumour growth}

\emph{Gut 2019 08;68(8):1465-1476}

PubMed: \url{https://www.ncbi.nlm.nih.gov/pubmed/?term=30343272}

BACKGROUND AND AIMS: Pancreatic ductal adenocarcinoma (PDAC) is a very
aggressive tumour with a poor prognosis using current treatments.
Targeted therapies may offer a new avenue for more effective strategies.
Dual-specificity tyrosine regulated kinase 1A (DYRK1A) is a pleiotropic
kinase with contradictory roles in different tumours that is
uncharacterised in PDAC. Here, we aimed to investigate the role of
DYRK1A in pancreatic tumorigenesis. DESIGN: We analysed DYRK1A
expression in PDAC genetic mouse models and in patient samples. DYRK1A
function was assessed with knockdown experiments in pancreatic tumour
cell lines and in PDAC mouse models with genetic reduction of Dyrk1a
dosage. Furthermore, we explored a mechanistic model for DYRK1A
activity. RESULTS: We showed that DYRK1A was highly expressed in PDAC,
and that its protein level positively correlated with that of c-MET.
Inhibition of DYRK1A reduced tumour progression by limiting tumour cell
proliferation. DYRK1A stabilised the c-MET receptor through SPRY2,
leading to prolonged activation of extracellular signal-regulated kinase
signalling. CONCLUSIONS: These findings reveal that DYRK1A contributes
to tumour growth in PDAC, at least through regulation of c-MET
accumulation, suggesting that inhibition of DYRK1A could represent a
novel therapeutic target for PDAC.

doi: \url{https://doi.org/10.1136/gutjnl-2018-316128}

{}

{}

\begin{center}\rule{0.5\linewidth}{\linethickness}\end{center}

\begin{center}\rule{0.5\linewidth}{\linethickness}\end{center}

 - \textbf{A Prospective, Randomized Phase II Study of Adjuvant
Gemcitabine Versus S-1 After Major Hepatectomy for Biliary Tract Cancer
(KHBO 1208): Kansai Hepato-Biliary Oncology Group}

\emph{Annals of surgery 2019 Aug;270(2):230-237}

PubMed: \url{https://www.ncbi.nlm.nih.gov/pubmed/?term=30339627}

OBJECTIVE: To evaluate each arm independently and compare adjuvant
gemcitabine (GEM) and S-1 chemotherapy after major hepatectomy
(hemihepatectomy or trisectionectomy) for biliary tract cancer (BTC).
BACKGROUND: Standardized adjuvant therapy is not performed after major
hepatectomy for BTC, and we determined the recommended dose in the
former study (KHBO1003). METHODS: We performed a multicenter, randomized
phase II study. The primary measure was 1-year recurrence-free survival
(RFS); the secondary measures were other RFS, overall survival (OS), and
others. The following 6-month adjuvant chemotherapy was administered
within 12 weeks of R0/1: GEM (1000 mg/m) every 2 weeks; or S-1
(80 mg/m/d) for 28 days every 6 weeks. Thirty-five patients were
assigned to each arm (alpha error, 10\%; beta error, 20\%). RESULTS: No
patients were excluded for the per-protocol analysis. There were no
statistically significant differences in the patient characteristics of
the 2 arms. The 1-year RFS and 1-year OS rates of the GEM arm were
51.4\% and 80.0\%, respectively, whereas those of the S-1 group were
62.9\% and 97.1\%. The comparison of the 2 arms revealed that 2-year RFS
rate, 1 and 2-year OS rates, and OS curve of the S-1 arm were superior
to GEM. With regard to OS, the hazard ratio of the S-1 group was 0.477
(90\% confidence interval 0.245-0.927). CONCLUSION: The comparison of
the survival of the 2 groups revealed that adjuvant S-1 therapy may be
superior to adjuvant GEM therapy after major hepatectomy for BTC.

doi: \url{https://doi.org/10.1097/SLA.0000000000002865}

{}

{}

\begin{center}\rule{0.5\linewidth}{\linethickness}\end{center}

\begin{center}\rule{0.5\linewidth}{\linethickness}\end{center}

 - \textbf{Response to Comment on ``The Virtual Hepatectomy Changed the
Practice of Liver Surgery: More Details, More Significance''}

\emph{Annals of surgery 2019 Aug;270(2):e33}

PubMed: \url{https://www.ncbi.nlm.nih.gov/pubmed/?term=30138165}

doi: \url{https://doi.org/10.1097/SLA.0000000000003009}

{}

{}

\begin{center}\rule{0.5\linewidth}{\linethickness}\end{center}

\begin{center}\rule{0.5\linewidth}{\linethickness}\end{center}

 - \textbf{Determinants of Severity in Acute Pancreatitis: A Nation-wide
Multicenter Prospective Cohort Study}

\emph{Annals of surgery 2019 Aug;270(2):348-355}

PubMed: \url{https://www.ncbi.nlm.nih.gov/pubmed/?term=29672416}

OBJECTIVE: The aim of this study was to compare and validate the
different classifications of severity in acute pancreatitis (AP) and to
investigate which characteristics of the disease are associated with
worse outcomes. SUMMARY OF BACKGROUND DATA: AP is a heterogeneous
disease, ranging from uneventful cases to patients with considerable
morbidity and high mortality rates. Severity classifications based on
legitimate determinants of severity are important to correctly describe
the course of disease. METHODS: A prospective multicenter cohort study
involving patients with AP from 23 hospitals in Spain. The Atlanta
Classification (AC), Revised Atlanta Classification (RAC), and
Determinant-based Classification (DBC) were compared. Binary logistic
multivariate analysis was performed to investigate independent
determinants of severity. RESULTS: A total of 1655 patients were
included; 70 patients (4.2\%) died. RAC and DBC were equally superior to
AC for describing the clinical course of AP. Although any kind of organ
failure was associated with increased morbidity and mortality,
persistent organ failure (POF) was the most significant determinant of
severity. All local complications were associated with worse outcomes.
Infected pancreatic necrosis correlated with high morbidity, but in the
presence of POF, it was not associated to higher mortality when compared
with sterile necrotizing pancreatitis. Exacerbation of previous
comorbidity was associated with increased morbidity and mortality.
CONCLUSION: The RAC and DBC both signify an advance in the description
and differentiation of AP patients. Herein, we describe the
complications of the disease independently associated to morbidity and
mortality. Our findings are valuable not only when designing future
studies on AP but also for the improvement of current classifications.

doi: \url{https://doi.org/10.1097/SLA.0000000000002766}

{}

{}

\begin{center}\rule{0.5\linewidth}{\linethickness}\end{center}

\begin{center}\rule{0.5\linewidth}{\linethickness}\end{center}

 - \textbf{Survival in Locally Advanced Pancreatic Cancer After
Neoadjuvant Therapy and Surgical Resection}

\emph{Annals of surgery 2019 Aug;270(2):340-347}

PubMed: \url{https://www.ncbi.nlm.nih.gov/pubmed/?term=29596120}

OBJECTIVE: The aim of the study was to identify the survival of patients
with locally advanced pancreatic cancer (LAPC) and assess the effect of
surgical resection after neoadjuvant therapy on patient outcomes.
BACKGROUND: An increasing number of LAPC patients who respond favorably
to neoadjuvant therapy undergo surgical resection. The impact of surgery
on patient survival is largely unknown. MATERIALS AND METHODS: All LAPC
patients who presented to the institutional pancreatic multidisciplinary
clinic (PMDC) from January 2013 to September 2017 were included in the
study. Demographics and clinical data on neoadjuvant treatment and
surgical resection were documented. Primary tumor resection rates after
neoadjuvant therapy and overall survival (OS) were the primary study
endpoints. RESULTS: A total of 415 LAPC patients were included in the
study. Stratification of neoadjuvant therapy in FOLFIRINOX-based,
gemcitabine-based, and combination of the two, and subsequent outcome
comparison did not demonstrate significant differences in OS of 331
non-resected LAPC patients (P = 0.134). Eighty-four patients underwent
resection of the primary tumor (20\%), after a median duration of 5
months of neoadjuvant therapy. FOLFIRINOX-based therapy and stereotactic
body radiation therapy correlated with increased probability of
resection (P = 0.006). Resected patients had better performance status,
smaller median tumor size (P = 0.029), and lower median CA19-9 values (P
\textless{} 0.001) at PMDC. Patients who underwent surgical resection
had significant higher median OS compared with those who did not (35.3
vs 16.3 mo, P \textless{} 0.001). The difference remained significant
when non-resected patients were matched for time of neoadjuvant therapy
(19.9 mo, P \textless{} 0.001). CONCLUSIONS: Surgical resection of LAPC
after neoadjuvant therapy is feasible in a highly selected cohort of
patients (20\%) and is associated with significantly longer median
overall survival.

doi: \url{https://doi.org/10.1097/SLA.0000000000002753}

{}

{}

\begin{center}\rule{0.5\linewidth}{\linethickness}\end{center}

\begin{center}\rule{0.5\linewidth}{\linethickness}\end{center}

 - \textbf{Metastatic low-grade endometrial stromal sarcoma of uterus
presenting as a primary pancreatic tumor: case presentation and
literature review}

\emph{Diagnostic pathology 2019 Apr;14(1):30}

PubMed: \url{https://www.ncbi.nlm.nih.gov/pubmed/?term=31010432}

BACKGROUND: Metastatic tumors to the pancreas are uncommon, accounting
for approximately 2\% of pancreatic malignancies. The most common
primary tumors to give rise to pancreatic metastases are carcinomas.
CASE PRESENTATION: A 50-year old female patient was investigated for a
cause of abdominal discomfort. She had a 2-year history of menorrhagia
and dysmenorrhea which was ascribed to a fibroid uterus. On imaging, she
was found to have a large solid and cystic mass in the tail of the
pancreas. Imaging also confirmed a fibroid uterus. A distal
pancreatectomy and splenectomy showed a 9 cm circumscribed mass within,
and grossly confined to, the parenchyma of the pancreatic tail.
Microscopically, the pancreatic lesion was lobulated, and
well-circumscribed, but focally infiltrative. It comprised sheets of
uniform spindled to epithelioid cells with round to oval nuclei, coarse
to vesicular chromatin, visible nucleoli, nuclear grooves and clear to
eosinophilic cytoplasm. Prominent arterioles were identified. The stroma
was collagenized in areas. Occasional hemosiderin-laden macrophages were
seen, and focal cystic change was present. There was no evidence of
nuclear pleomorphism, mitotic activity or necrosis, and there was no
evidence of endometriosis despite multiple sections being taken.
Immunohistochemistry showed that the tumor cells were positive for CD10,
estrogen receptor (ER), progesterone receptor (PR), Wilms tumor-1 (WT-1)
and smooth muscle actin (SMA). RNA sequencing detected a PHF1
rearrangement. The morphological, immunohistochemical and molecular
features were of a low-grade endometrial stromal sarcoma (LG-ESS).
Subsequent total hysterectomy and bilateral salpingo-oophorectomy
3 months later, showed uterine fibroids and a 5 cm low-grade endometrial
stromal sarcoma confined to the uterus, with lymphatic invasion.
CONCLUSIONS: To the best of our knowledge, this is the first documented
case of metastatic endometrial stromal sarcoma of uterus presenting as a
primary pancreatic neoplasm. An unexpected extra-uterine location and
unusual presentation of ESS may make the diagnosis challenging, despite
classic histological features. Morphological, immunohistochemical and
molecular findings must be combined to render the correct diagnosis.

doi: \url{https://doi.org/10.1186/s13000-019-0807-3}

{}

{}

\begin{center}\rule{0.5\linewidth}{\linethickness}\end{center}

\begin{center}\rule{0.5\linewidth}{\linethickness}\end{center}

 - \textbf{{[}Heterotopic tissue in the gastrointestinal tract{]}}

\emph{Der Pathologe 2018 Sep;39(5):402-408}

PubMed: \url{https://www.ncbi.nlm.nih.gov/pubmed/?term=30105611}

Heterotopia of the gastrointestinal tract is a~common finding. This is
due to the complex embryogenesis and the relative ease to detect
heterotopic tissue during endoscopy. The reason for biopsy is mostly to
rule out neoplasms or to define specific causes of inflammation.
Heterotopic tissue can occur in any location of the gastrointestinal
tract. The most frequent are gastric heterotopia, pancreatic
heterotopia, and heterotopia of Brunner's gland. On rare occasions,
heterotopic tissue of salivary gland type as well as heterotopias of
apocrine glands, thyroid, and prostatic tissue have been described. The
most frequently involved organs are the small intestine, in particular
the duodenum, the esophagus, and the stomach. Heterotopia of the large
bowel occurs exclusively in the rectum. Most heterotopias do not cause
symptoms and are easily diagnosed by biopsy and histology. However,
depending on location, size, and the kind of underlying heterotopic
tissue, they may cause significant complications, such as inflammation,
ulceration and perforation, obstruction, intussusception, and severe
life-threatening bleeding. Another rare but significant complication is
neoplasia. Gastric heterotopias may give rise to pyloric gland adenomas
within the bowel or rarely adenocarcinomas of the esophagus. Pancreatic
heterotopia can be complicated by ductal type pancreatic
adenocarcinomas, by acinus cell carcinomas, by intraductal papillary
mucinous neoplasias, and also by endocrine tumors. The present paper
summarizes our current knowledge about heterotopias in a~topographic
clinico-pathological manner.

doi: \url{https://doi.org/10.1007/s00292-018-0466-2}

{}

{}

\begin{center}\rule{0.5\linewidth}{\linethickness}\end{center}

\begin{center}\rule{0.5\linewidth}{\linethickness}\end{center}

Back to top

\begin{center}\rule{0.5\linewidth}{\linethickness}\end{center}

\pagebreak

\hypertarget{gallbladder}{%
\subsubsection{Gallbladder}\label{gallbladder}}

\begin{center}\rule{0.5\linewidth}{\linethickness}\end{center}

 - \textbf{Transpapillary Endoscopic Removal of Gallbladder Stones
Through a Fully Covered Metallic Stent}

\emph{The American journal of gastroenterology 2019 Aug;():}

PubMed: \url{https://www.ncbi.nlm.nih.gov/pubmed/?term=31403962}

doi: \url{https://doi.org/10.14309/ajg.0000000000000367}

{}

{}

\begin{center}\rule{0.5\linewidth}{\linethickness}\end{center}

\begin{center}\rule{0.5\linewidth}{\linethickness}\end{center}

 - \textbf{A 12-year trend analysis of the incidence of gastrointestinal
cancers in East Azerbaijan: last updated results of an ongoing
population-based cancer registry}

\emph{BMC cancer 2019 Aug;19(1):782}

PubMed: \url{https://www.ncbi.nlm.nih.gov/pubmed/?term=31391032}

BACKGROUND: The most recent results of Global Cancer Statistics
indicated that gastrointestinal cancers, including gastric, colorectal,
esophageal, and liver cancers, are among the most commonly diagnosed
cancers worldwide. Previous reports from cancer registries in East
Azerbaijan have shown that there is a high incidence of gastrointestinal
cancer in this region, so we performed a trend analysis to determine the
pattern of change over the last decade. METHODS: In total, 12 years of
cancer registry data were collected from different sources in East
Azerbaijan, and a data quality check was performed to ensure clean data.
Using the 2000 World Health Organization standard population, we then
generated age-standardized incidence rates (ASRs) for different cancers,
and for each year from 1383 to 1394 of the Persian calendar (i.e., 19
March 2004 to 20 March 2015). Annual percent changes (APCs) and Average
annual percent changes (AAPCs) in the ASRs for esophageal, gastric,
small intestine, colorectal, anal, liver, gallbladder, and pancreatic
cancers were calculated using Joinpoint Software (Version 4.5.0.1, June
2017). RESULTS: An increase in most types of cancer was observed during
the study period. The ASR for colorectal cancer increased from 2.9 to
13.6 per 100,000 women (APC, 9.7\%) and from 2.2 to 17.8 per 100,000 men
(APC, 10.2\%). The ASR for gastric cancer showed a slight increasing
trend from 10.5 to 13.5 per 100,000 women (APC, 1.3\%) and from 3.1 to
29.9 per 100,000 men (APC, 3.2\%). However, trend analysis showed a
decreasing pattern for the ASR of esophageal cancer in both genders
(APC,- 3\%), with APCs of - 1.1\% in females and - 0.4\% in males.
CONCLUSIONS: The latest results of the East Azerbaijan Population-Based
Cancer Registry indicate that gastrointestinal cancers remain common,
with significant increasing trends in their ASRs. Improved screening and
early detection are needed in this region.

doi: \url{https://doi.org/10.1186/s12885-019-6008-3}

{}

{}

\begin{center}\rule{0.5\linewidth}{\linethickness}\end{center}

\begin{center}\rule{0.5\linewidth}{\linethickness}\end{center}

 - \textbf{Classification of the cystic duct patterns and endoscopic
transpapillary cannulation of the gallbladder to prevent post-ERCP
cholecystitis}

\emph{BMC gastroenterology 2019 Aug;19(1):139}

PubMed: \url{https://www.ncbi.nlm.nih.gov/pubmed/?term=31382888}

BACKGROUND: Endoscopic transpapillary cannulation of the gallbladder is
useful but challenging. This study aimed to investigate cystic duct
anatomy patterns, which may guide cystic duct cannulation. METHODS: A
total of 226 patients who underwent endoscopic transpapillary
cannulation of the gallbladder were analyzed retrospectively. RESULTS:
According to the cystic duct take-off, 226 cystic duct patterns were
divided into 3 patterns: Type I (193, 85.4\%), located on the right and
angled up; Type II (7, 3.1\%), located on the right and angled down; and
Type III (26, 11.5\%), located on the left and angled up. Type I was
further divided into three subtypes: Line type, S type (S1, not
surrounding the common bile duct; S2, surrounding the common bile duct),
and α type (α1, forward α; α2, reverse α). Types I and III cystic ducts
were easier to be cannulated with a higher success rate (85.1 and
86.4\%, respectively) compared with Type II cystic duct (75\%) despite
no statistically significant difference. The reasons for the failure of
gallbladder cannulation included invisible cyst duct take-off, severe
cyst duct stenosis, impacted stones in cyst duct or neck of the
gallbladder, sharply angled cyst duct, and markedly dilated cyst duct
with the tortuous valves of Heister. CONCLUSION: Classification of
cystic duct patterns was helpful in guiding endoscopic transpapillary
gallbladder cannulation.

doi: \url{https://doi.org/10.1186/s12876-019-1053-6}

{}

{}

\begin{center}\rule{0.5\linewidth}{\linethickness}\end{center}

\begin{center}\rule{0.5\linewidth}{\linethickness}\end{center}

 - \textbf{Efficacy and Safety of Eluxadoline in Patients With Irritable
Bowel Syndrome With Diarrhea Who Report Inadequate Symptom Control With
Loperamide: RELIEF Phase 4 Study}

\emph{The American journal of gastroenterology 2019 Jul;():}

PubMed: \url{https://www.ncbi.nlm.nih.gov/pubmed/?term=31356229}

OBJECTIVES: Irritable bowel syndrome with diarrhea (IBS-D) is a
functional gastrointestinal disorder with limited effective treatment
options. We evaluated the efficacy and safety of eluxadoline in patients
with IBS-D who reported inadequate symptom control with prior
loperamide. METHODS: Three hundred forty-six adults with IBS-D (Rome III
criteria) were randomly assigned to placebo or eluxadoline 100 mg twice
daily for 12 weeks. Patients recorded daily IBS-D symptoms, including
worst abdominal pain (WAP) and stool consistency (through Bristol Stool
Scale). The primary endpoint was proportion of composite responders,
defined as patients who met daily composite response criteria (≥40\% WAP
improvement and \textless{}5 Bristol Stool Scale score) for at least
50\% of treatment days, and recorded ≥60 days of diary entries over the
12-week period. RESULTS: Over 12 weeks, a significantly greater
proportion of eluxadoline patients achieved the primary composite
responder endpoint compared to placebo (22.7\% vs 10.3\%, P = 0.002),
and component endpoints of improvements in stool consistency (27.9\% vs
16.7\%, P = 0.01) and WAP (43.6\% vs 31.0\%, P = 0.02). Additionally, a
greater proportion of eluxadoline patients met the composite responder
endpoint assessed at monthly intervals compared to placebo (weeks 1-4:
14.0\% vs 6.9\%, P = 0.03; weeks 5-8: 26.7\% vs 14.9\%, P = 0.006; weeks
9-12: 30.8\% vs 16.7\%, P = 0.002). Rates of adverse events were
comparable in both groups (37.4\% vs 35.3\%); no treatment-related
serious adverse event, cases of sphincter of Oddi spasm, or pancreatitis
were reported. DISCUSSION: Eluxadoline appears safe and effective for
treating IBS-D symptoms in patients with an intact gallbladder reporting
inadequate relief with prior loperamide use.

doi: \url{https://doi.org/10.14309/ajg.0000000000000327}

{}

{}

\begin{center}\rule{0.5\linewidth}{\linethickness}\end{center}

\begin{center}\rule{0.5\linewidth}{\linethickness}\end{center}

 - \textbf{Acute Recurrent and Chronic Pancreatitis as Initial
Manifestations of Cystic Fibrosis and Cystic Fibrosis Transmembrane
Conductance Regulator-Related Disorders}

\emph{Pancreas 2019 Aug;48(7):888-893}

PubMed: \url{https://www.ncbi.nlm.nih.gov/pubmed/?term=31268981}

OBJECTIVES: Recurrent pancreatitis is considered a rare manifestation of
cystic fibrosis transmembrane conductance regulator (CFTR) dysfunction;
this case series highlights that pancreatitis can be a presenting
symptoms of cystic fibrosis (CF) or a CFTR-related disorder (CFTR-RD).
METHODS: Retrospective review of patients younger than 30 years
diagnosed as having acute recurrent pancreatitis (ARP) or chronic
pancreatitis (CP) and subsequently diagnosed as having CF or CFTR-RD.
RESULTS: Among 18 patients, median time from diagnosis of ARP/CP to
diagnosis of CF was 0.4 years (range, 0-33 years). Eight were classified
as having CF by elevated sweat chloride testing (SCT). Five had
intermediate SCT (30-59 mmol/L) with 2 pathogenic mutations. Five had
CFTR-RD with intermediate SCT and 0 to 1 pathogenic mutations. Eight
patients (44\%) had exocrine pancreatic insufficiency, and pancreatic
fluid collections were more common in this group. Based on the CFTR
mutation, 6 patients were eligible for CFTR potentiator therapy,
although none received it during the study period. Nine of the 18 had ≥1
other likely CF manifestations, including sinusitis (33\%), nasal polyps
(11\%), pneumonia (22\%), and gallbladder disease (22\%). CONCLUSIONS:
Cystic fibrosis or CFTR-RD can present as ARP/CP. Complete diagnostic
testing for CFTR-RD in patients with ARP/CP will broaden treatment
options and help to identify comorbid illness.

doi: \url{https://doi.org/10.1097/MPA.0000000000001350}

{}

{}

\begin{center}\rule{0.5\linewidth}{\linethickness}\end{center}

\begin{center}\rule{0.5\linewidth}{\linethickness}\end{center}

 - \textbf{Epithelial Inclusions in Gallbladder May Mimic Parasite
Infection}

\emph{American journal of clinical pathology 2019 Aug;152(3):399-402}

PubMed: \url{https://www.ncbi.nlm.nih.gov/pubmed/?term=31189015}

doi: \url{https://doi.org/10.1093/ajcp/aqz054}

{}

{}

\begin{center}\rule{0.5\linewidth}{\linethickness}\end{center}

\begin{center}\rule{0.5\linewidth}{\linethickness}\end{center}

 - \textbf{Nivolumab alone or in combination with cisplatin plus
gemcitabine in Japanese patients with unresectable or recurrent biliary
tract cancer: a non-randomised, multicentre, open-label, phase 1 study}

\emph{The lancet. Gastroenterology \& hepatology 2019 Aug;4(8):611-621}

PubMed: \url{https://www.ncbi.nlm.nih.gov/pubmed/?term=31109808}

BACKGROUND: This study aimed to assess the safety and tolerability of
the immune checkpoint inhibitor nivolumab, as monotherapy or combined
with chemotherapy, in Japanese patients with biliary tract cancer.
METHODS: This multicentre, open-label, phase 1 trial was done at four
cancer centres in Japan. Eligible patients were aged 20-79 years, had
biliary tract adenocarcinoma (intrahepatic bile duct cancer,
extrahepatic bile duct cancer, gallbladder cancer, or ampullary cancer),
Eastern Cooperative Oncology Group performance status 0 or 1, adequate
hepatic, renal, and haematological function, and tumour tissue samples
for PD-L1 expression analysis. Patients with unresectable or recurrent
biliary tract cancer that was refractory or intolerant to
gemcitabine-based treatment regimens received nivolumab monotherapy (240
mg every 2 weeks {[}monotherapy cohort{]}). Chemotherapy-naive patients
with unresectable or recurrent biliary tract cancer received nivolumab
(240 mg every 2 weeks) and cisplatin (25 mg/m2) plus gemcitabine (1000
mg/m2) chemotherapy (combined therapy cohort). The primary objective was
to assess tolerability and safety. The primary objective was assessed in
the safety population of all patients who had received at least one dose
of nivolumab. This study is registered with www.clinicaltrials.jp,
number JapicCTI-153098, and follow-up is ongoing. FINDINGS: 30 patients
were enrolled into each cohort between Jan 13, 2016, and April 19, 2017.
Data cutoff was Aug 31, 2017. In the monotherapy cohort, the most
frequently reported treatment-related adverse events were decreased
appetite (five {[}17\%{]}), malaise (four {[}13\%{]}), and pruritus
(four {[}13\%{]}). Grade 3-4 treatment-related adverse events were
reported by three (10\%) patients (rash, maculopapular rash, and amylase
increase) and treatment-related serious adverse events were reported by
one (3\%) patient (pleurisy). In the combined therapy cohort, the most
frequently reported treatment-related adverse events were neutrophil
count decrease (any grade 25 {[}83\%{]}; grade 3-4 in 23 {[}77\%{]}
patients) and platelet count decrease (any grade 25 {[}83\%{]} of 30;
grade 3-4 in 15 {[}50\%{]} patients). Six (20\%) patients reported 11
treatment-related serious adverse events (platelet count decrease
{[}three patients{]}, febrile neutropenia {[}two patients{]}, neutrophil
count decrease, anaemia, anaphylactic reaction, decreased appetite,
pyrexia, and myocarditis {[}one patient each{]}). In the monotherapy
cohort, median overall survival was 5·2 months (90\% CI 4·5-8·7), median
progression-free survival was 1·4 months (90\% CI 1·4-1·4), and one of
30 patients had an objective response. In the combined therapy cohort,
median overall survival was 15·4 months (90\% CI 11·8-not estimable),
median progression-free survival was 4·2 months (90\% CI 2·8-5·6), and
11 of 30 patients had an objective response. INTERPRETATION: Nivolumab
had a manageable safety profile and signs of clinical activity in
patients with unresectable or recurrent biliary tract cancer. This
initial assessment of nivolumab for the treatment of advanced biliary
tract cancer provides supportive evidence for future larger randomised
studies of nivolumab in this difficult to treat cancer. FUNDING: Ono
Pharmaceutical Co Ltd and Bristol-Myers Squibb Inc.

doi: \url{https://doi.org/10.1016/S2468-1253(19)30086-X}

{}

{}

\begin{center}\rule{0.5\linewidth}{\linethickness}\end{center}

\begin{center}\rule{0.5\linewidth}{\linethickness}\end{center}

 - \textbf{Detection of NRG1 Gene Fusions in Solid Tumors}

\emph{Clinical cancer research : an official journal of the American
Association for Cancer Research 2019 Aug;25(16):4966-4972}

PubMed: \url{https://www.ncbi.nlm.nih.gov/pubmed/?term=30988082}

PURPOSE: NRG1 gene fusions are rare but potentially actionable oncogenic
drivers that are present in some solid tumors. Details regarding the
incidence of these gene rearrangements are lacking. Here, we assessed
the incidence of NRG1 fusions across multiple tumor types and described
fusion partners. EXPERIMENTAL DESIGN: Tumor specimens submitted for
molecular profiling at a Clinical Laboratory Improvement Amendments
(CLIA)-certified genomics laboratory and that underwent fusion testing
by anchored multiplex PCR for targeted RNA sequencing were
retrospectively identified. The overall and tumor-specific incidence was
noted, as was the specific fusion partner. RESULTS: Out of 21,858 tumor
specimens profiled from September 2015 to December 2018, 41 cases
(0.2\%) harbored an NRG1 fusion. Multiple fusion partners were
identified. Fusion events were seen across tumor types. The greatest
incidence was in non-small cell lung cancer (NSCLC, 25), though this
represented only 0.3\% of NSCLC cases tested. Other tumor types
harboring an NRG1 fusion included gallbladder cancer, renal cell
carcinoma, bladder cancer, ovarian cancer, pancreatic cancer, breast
cancer, neuroendocrine tumor, sarcoma, and colorectal cancer.
CONCLUSIONS: NRG1 fusions can be detected at a low incidence across
multiple tumor types with significant heterogeneity in fusion
partner.See related commentary by Dimou and Camidge, p.~4865.

doi: \url{https://doi.org/10.1158/1078-0432.CCR-19-0160}

{}

{}

\begin{center}\rule{0.5\linewidth}{\linethickness}\end{center}

\begin{center}\rule{0.5\linewidth}{\linethickness}\end{center}

 - \textbf{Gallbladder Papilloma in a Child Unmasking Metachromatic
Leukodystrophy: A Case Report With Review of Literature}

\emph{Fetal and pediatric pathology 2019 Aug;38(4):345-351}

PubMed: \url{https://www.ncbi.nlm.nih.gov/pubmed/?term=30912695}

Background: Metachromatic leukodystrophy (MLD) is a lipid storage
disease characterized the accumulation of sulfatides in different
viscera including the gallbladder. Case report: A 2-year-old girl had
upper right quadrant lesion that was preoperatively thought to be a
biliary cystadenoma. Histologically, the gallbladder lesion was a
tubulo-villous papilloma with multiple foci of papillary mucosal
hyperplasia. Many storage histiocytes containing metachromatic granules,
characteristic of MLD, were present in the tips of the papillae. MLD was
later confirmed by enzyme studies. Conclusion: Gallbladder papilloma can
be the presenting feature of MLD.

doi: \url{https://doi.org/10.1080/15513815.2019.1588442}

{}

{}

\begin{center}\rule{0.5\linewidth}{\linethickness}\end{center}

\begin{center}\rule{0.5\linewidth}{\linethickness}\end{center}

 - \textbf{Incidental Hepatic Tissue Obtained via Routine
Cholecystectomy}

\emph{International journal of surgical pathology 2019
Aug;27(5):499-505}

PubMed: \url{https://www.ncbi.nlm.nih.gov/pubmed/?term=30520351}

Background. The hepatic tissue that may occupy specimens from routine
cholecystectomies has yet to be studied. Our objectives were to
determine the prevalence of hepatic tissue obtained at routine
cholecystectomy, to determine whether such hepatic tissue can
histologically withstand technical artifacts commonly associated with
cholecystectomy, and to determine whether examining such hepatic tissue
has diagnostic utility. Materials and Methods. We retrospectively
reviewed 50 specimens from routine cholecystectomies that were performed
by surgeons who lacked knowledge of our study. All 50 specimens were
grossed according to standard protocol, with only limited, nontargeted
sampling of the rough nonperitonealized margin, and were received
without fixative. Results. Twelve specimens (24.0\%) contained hepatic
tissue. The hepatic tissue measured up to 44.5-mm long and 1.8-mm wide
and contained up to 11 complete portal tracts. Hepatic tissue in 3
specimens satisfied criteria for adequacy established for core biopsies
based on number of portal tracts or size. Despite cautery and delayed
fixation, all hepatic tissue had surprisingly well-preserved histology.
Pathologic findings included nonalcoholic fatty liver disease, von
Meyenburg complex, chronic cholestasis, and senescence. Conclusions. The
hepatic tissue that accompanies specimens from routine cholecystectomies
may be relatively common, can be large, is well preserved, and can
harbor diagnostically useful information.

doi: \url{https://doi.org/10.1177/1066896918817374}

{}

{}

\begin{center}\rule{0.5\linewidth}{\linethickness}\end{center}

\begin{center}\rule{0.5\linewidth}{\linethickness}\end{center}

Back to top

\begin{center}\rule{0.5\linewidth}{\linethickness}\end{center}

\pagebreak

\hypertarget{bile-ducts}{%
\subsubsection{Bile Ducts}\label{bile-ducts}}

\begin{center}\rule{0.5\linewidth}{\linethickness}\end{center}

 - \textbf{Fragile X mental retardation protein protects against tumour
necrosis factor-mediated cell death and liver injury}

\emph{Gut 2019 Aug;():}

PubMed: \url{https://www.ncbi.nlm.nih.gov/pubmed/?term=31409605}

OBJECTIVE: The Fragile X mental retardation (FMR) syndrome is a
frequently inherited intellectual disability caused by decreased or
absent expression of the FMR protein (FMRP). Lack of FMRP is associated
with neuronal degradation and cognitive dysfunction but its role outside
the central nervous system is insufficiently studied. Here, we identify
a role of FMRP in liver disease. DESIGN: Mice lacking Fmr1 gene
expression were used to study the role of FMRP during tumour necrosis
factor (TNF)-induced liver damage in disease model systems. Liver damage
and mechanistic studies were performed using real-time PCR, Western
Blot, staining of tissue sections and clinical chemistry. RESULTS:
Fmr1null mice exhibited increased liver damage during virus-mediated
hepatitis following infection with the lymphocytic choriomeningitis
virus. Exposure to TNF resulted in severe liver damage due to increased
hepatocyte cell death. Consistently, we found increased caspase-8 and
caspase-3 activation following TNF stimulation. Furthermore, we
demonstrate FMRP to be critically important for regulating key molecules
in TNF receptor 1 (TNFR1)-dependent apoptosis and necroptosis including
CYLD, c-FLIPS and JNK, which contribute to prolonged RIPK1 expression.
Accordingly, the RIPK1 inhibitor Necrostatin-1s could reduce liver cell
death and alleviate liver damage in Fmr1null mice following TNF
exposure. Consistently, FMRP-deficient mice developed increased
pathology during acute cholestasis following bile duct ligation, which
coincided with increased hepatic expression of RIPK1, RIPK3 and
phosphorylation of MLKL. CONCLUSIONS: We show that FMRP plays a central
role in the inhibition of TNF-mediated cell death during infection and
liver disease.

doi: \url{https://doi.org/10.1136/gutjnl-2019-318215}

{}

{}

\begin{center}\rule{0.5\linewidth}{\linethickness}\end{center}

\begin{center}\rule{0.5\linewidth}{\linethickness}\end{center}

 - \textbf{Decompressive laparotomy for abdominal compartment syndrome
resulting from severe acute pancreatitis: a case report}

\emph{BMC gastroenterology 2019 Aug;19(1):141}

PubMed: \url{https://www.ncbi.nlm.nih.gov/pubmed/?term=31395017}

BACKGROUND: Abdominal compartment syndrome (ACS) is associated with
mortality in patients with critical illness such as severe acute
pancreatitis, but it remains unclear whether decompressive laparotomy
for ACS can improve the prognosis of patients. CASE PRESENTATION: A
woman in her 60s visited our hospital because of upper abdominal pain.
On the basis of her laboratory data and abdominal contrast-enhanced
computed tomography findings, acute gallstone pancreatitis was
diagnosed. She underwent endoscopic sphincterotomy for the removal of
the common bile duct stone. Then, a drainage tube was placed in the bile
duct. However, on the 5th hospital day, her intra-abdominal pressure
increased to 22 mmHg and renal dysfunction was observed, which led to
the diagnosis of ACS. As intensive medical treatments did not improve
her ACS, she underwent decompressive laparotomy on the 9th hospital day.
Postoperatively, her laboratory data and intravesical pressure improved,
and she was discharged from the hospital after abdominal closure,
continuous drainage, and antibiotic therapy. CONCLUSION: As the
effectiveness of decompressive laparotomy for ACS has not been
established, this treatment indication remains controversial.
Decompressive laparotomy is considered useful for the management of ACS,
if it is performed at an appropriate time, as in the present case.

doi: \url{https://doi.org/10.1186/s12876-019-1059-0}

{}

{}

\begin{center}\rule{0.5\linewidth}{\linethickness}\end{center}

\begin{center}\rule{0.5\linewidth}{\linethickness}\end{center}

 - \textbf{Classification of the cystic duct patterns and endoscopic
transpapillary cannulation of the gallbladder to prevent post-ERCP
cholecystitis}

\emph{BMC gastroenterology 2019 Aug;19(1):139}

PubMed: \url{https://www.ncbi.nlm.nih.gov/pubmed/?term=31382888}

BACKGROUND: Endoscopic transpapillary cannulation of the gallbladder is
useful but challenging. This study aimed to investigate cystic duct
anatomy patterns, which may guide cystic duct cannulation. METHODS: A
total of 226 patients who underwent endoscopic transpapillary
cannulation of the gallbladder were analyzed retrospectively. RESULTS:
According to the cystic duct take-off, 226 cystic duct patterns were
divided into 3 patterns: Type I (193, 85.4\%), located on the right and
angled up; Type II (7, 3.1\%), located on the right and angled down; and
Type III (26, 11.5\%), located on the left and angled up. Type I was
further divided into three subtypes: Line type, S type (S1, not
surrounding the common bile duct; S2, surrounding the common bile duct),
and α type (α1, forward α; α2, reverse α). Types I and III cystic ducts
were easier to be cannulated with a higher success rate (85.1 and
86.4\%, respectively) compared with Type II cystic duct (75\%) despite
no statistically significant difference. The reasons for the failure of
gallbladder cannulation included invisible cyst duct take-off, severe
cyst duct stenosis, impacted stones in cyst duct or neck of the
gallbladder, sharply angled cyst duct, and markedly dilated cyst duct
with the tortuous valves of Heister. CONCLUSION: Classification of
cystic duct patterns was helpful in guiding endoscopic transpapillary
gallbladder cannulation.

doi: \url{https://doi.org/10.1186/s12876-019-1053-6}

{}

{}

\begin{center}\rule{0.5\linewidth}{\linethickness}\end{center}

\begin{center}\rule{0.5\linewidth}{\linethickness}\end{center}

 - \textbf{Liver biopsy in primary biliary cholangitis: is sinusoidal
fibrosis the missing key?}

\emph{Journal of clinical pathology 2019 Aug;():}

PubMed: \url{https://www.ncbi.nlm.nih.gov/pubmed/?term=31371396}

AIMS: The role of liver biopsy in primary biliary cholangitis (PBC) is
controversial, as is the optimal method of histological assessment. We
compared the Ludwig and Ishak systems and three components of the
Japanese (Nakanuma) staging system to evaluate their clinical and
biochemical correlations and prognostic value. METHODS: We reviewed
biopsies from 106 patients with PBC, derived from a previous trial of
colchicine therapy with 24-34 years' follow-up, following which five
clinical outcomes were evaluated: hepatic decompensation, cholestatic
PBC death/liver transplant, portal hypertensive PBC death, all PBC
deaths and overall survival. RESULTS: Ludwig and Ishak stages correlated
well with prognostically significant parameters, including serum
bilirubin, and both Mayo and Child Scores. Serum aspartate
aminotransferase correlated with interface hepatitis (IFH), and alkaline
phosphatase with orcein deposition, bile duct (BD) loss and cholestasis.
Ludwig correlated with all five clinical outcomes, while Ishak stage was
only significantly correlated with two. While sinusoidal fibrosis,
orcein deposition, BD loss and cholestasis all predicted hepatic
death/transplant, after correction for Mayo Score, the only histological
parameters predictive of clinical outcomes were IFH (associated with
two) and sinusoidal fibrosis (associated with all five). CONCLUSION:
Liver biopsy is required in the diagnosis of around 20\% of patients
with PBC. The Ludwig system is of more prognostic value than both Ishak
and any of the three individual components of the Nakanuma staging
system, but the major histological parameter providing independent
prognostic value beyond the Mayo Score is sinusoidal fibrosis.

doi: \url{https://doi.org/10.1136/jclinpath-2019-205958}

{}

{}

\begin{center}\rule{0.5\linewidth}{\linethickness}\end{center}

\begin{center}\rule{0.5\linewidth}{\linethickness}\end{center}

 - \textbf{Immunoglobulin G4-related hepatobiliary disease}

\emph{Seminars in diagnostic pathology 2019 Jul;():}

PubMed: \url{https://www.ncbi.nlm.nih.gov/pubmed/?term=31358425}

Immunoglobuline G4-related disease (IgG4-RD) is a systemic disease that
can involve virtually any organs including the biliary tract and liver.
The biliary tract involvement of IgG4-RD is known as IgG4-sclerosing
cholangitis (IgG4-SC) and may or may not present with an inflammatory
pseudotumor. Large bile ducts such as extrahepatic, hilar, and perihilar
ducts are typically affected and demonstrate marked bile duct wall
thickening and develop strictures. Histologically, the involved ducts
show transmural dense lymphoplasmacytic infiltrates with storiform
fibrosis extending into peribiliary glands and periductal soft tissue.
The luminal epithelium is usually preserved. Tissue eosinophilia and
obliterative phlebitis are also frequently noted. Liver biopsy findings
of IgG4-SC are heterogeneous and rather nonspecific, but two features
specific to IgG4-SC have been described: \textgreater{}10 IgG4-positive
plasma cell/HPF and small portal-based fibroinflammatory nodules.
Secondary changes, due to downstream bile duct obstruction are often
appreciated. When considering the differential diagnosis, primary
sclerosing cholangitis and cholangiocarcinoma are great clinical and
histologic mimics of IgG4-SC. Liver involvement in IgG4-RD has not been
well characterized and includes IgG4-hepatopathy and IgG4-related
autoimmune hepatitis (AIH). IgG4-hepatopathy is a generic term covering
hepatic lesions related to IgG4-RD and /or IgG4-SC. It includes primary
liver parenchymal changes inherent to IgG4-RD, liver parenchymal
involvement of IgG4-SC, and secondary changes related to IgG4-SC.
IgG4-related AIH is characterized by clinical and histologic features of
classical AIH but with prominent (\textgreater{}10/HPF) IgG4-positive
plasma cells. It is unclear whether this represents a hepatic
manifestation of IgG4-RD or a subset of AIH with increased IgG4-positive
plasma cells at the present time. Synchronous or metachronous
involvement of other organs, offers a clue to make this distinction.
IgG4 immunohistochemistry has an important role in diagnosing IgG4-RD.
But the diagnosis cannot be made solely based on the number of
IgG4-positive plasma cells, and results need to be interpreted with
caution as increased IgG4-positive plasma cells can be seen in other
inflammatory conditions or even in malignancy.

doi: \url{https://doi.org/10.1053/j.semdp.2019.07.007}

{}

{}

\begin{center}\rule{0.5\linewidth}{\linethickness}\end{center}

\begin{center}\rule{0.5\linewidth}{\linethickness}\end{center}

 - \textbf{Checkpoint inhibitor-induced liver injury: A novel form of
liver disease emerging in the era of cancer immunotherapy}

\emph{Seminars in diagnostic pathology 2019 Jul;():}

PubMed: \url{https://www.ncbi.nlm.nih.gov/pubmed/?term=31358424}

Liver injury triggered by immune checkpoint inhibitors has been
increasingly seen in clinical practice, and the incidence is likely to
rise further in the next several years because of expanded indications
for cancer immunotherapy. Tissue damage driven by disrupted immune
tolerance against self-antigens is called an immune-related adverse
event (irAE). irAEs in the liver histologically presents panlobular
hepatitis (∼70\%), isolated central zonal necrosis (∼20\%), primarily
granulomatous hepatitis (∼5\%), and other minor forms of tissue injury
(∼5\%). Infiltrating cells are mainly lymphocytes and occasional
eosinophils. Unlike classic autoimmune hepatitis (AIH), plasma cell
infiltration is not conspicuous. Immunostaining reveals a large number
of CD8+ T lymphocytes and a markedly smaller number of CD4+ cells or
CD20+ B lymphocytes. The unique CD3+/CD20+ and CD4+/CD8+ ratios shifted
in favor of CD8+ cytotoxic T lymphocytes are helpful to discriminate
irAEs from other conditions (e.g., AIH, idiosyncratic drug-induced liver
injury). Another hepatobiliary manifestation of irAEs is sclerosing
cholangitis clinically characterized by elevations of biliary enzymes,
diffuse duct wall thickening, and duct dilatation. Lymphocytic
infiltration can be observed by endoscopic biopsies from the thick
extrahepatic bile ducts, and liver needle biopsies may also show severe
lymphocytic cholangitis resembling primary biliary cholangitis. An
important differential diagnosis of irAEs is previously asymptomatic or
subclinical liver disease unmasked by cancer immunotherapy, which is
often challenging and requires close clinicopathological correlations.

doi: \url{https://doi.org/10.1053/j.semdp.2019.07.009}

{}

{}

\begin{center}\rule{0.5\linewidth}{\linethickness}\end{center}

\begin{center}\rule{0.5\linewidth}{\linethickness}\end{center}

 - \textbf{Clinical value of DPOC for detecting and removing residual
common bile duct stones (video)}

\emph{BMC gastroenterology 2019 Jul;19(1):135}

PubMed: \url{https://www.ncbi.nlm.nih.gov/pubmed/?term=31349795}

BACKGROUND: This study aims to evaluate the efficacy and safety of
detecting and removing residual common bile duct stones (CBDS) using
direct peroralcholangioscopy (DPOC) after performing endoscopic
retrograde cholangiopancreatography (ERCP) for stone retrieval. METHODS:
From January 5, 2017 to December 27, 2017, a total of 164 cases of
choledocholithiasis were treated by ERCP for stone retrieval. According
to the inclusion and exclusion criteria, the remaining 79 cases (39
males; mean age: 63.3 years old, range: 52-79 years old) were enrolled
in the present study. The maximum transverse stone diameter was 6-15 mm
(12.7 ± 4.2 mm), as determined by ERCP. Furthermore, there were 57 cases
of multiple stones (number of stones: two in 41 cases, three in nine
cases, and ≥ 4 in seven cases), 13 cases of post-mechanical lithotripsy,
and nine cases of broken stones. RESULTS: The overall success rate of
DPOC was 94.9\% (75/79). Furthermore, 18.7\%(14/75) of cases were
directly inserted, 72\%(54/75) of cases required guide wire assistance,
and 9.3\%(7/75) of cases were successfully inserted with overtube
assistance. The average insertion time was 7-17 min (4.9 ± 2.9 min).
Residual stones were detected in 19 cases (25.3\%), and all of which
were \textless{} 5 mm in diameter. Moreover, five cases of formed stones
were removed by basket and balloon catheter, while the remaining cases
were cleaned after irrigation and suction. There were no serious
complications. CONCLUSION: DPOC is safe and effective for both the
detection and removal of residual CBDS after conventional ERCP.

doi: \url{https://doi.org/10.1186/s12876-019-1045-6}

{}

{}

\begin{center}\rule{0.5\linewidth}{\linethickness}\end{center}

\begin{center}\rule{0.5\linewidth}{\linethickness}\end{center}

 - \textbf{Development of a Theranostic Convergence
Bioradiopharmaceutical for Immuno-PET based Radioimmunotherapy of L1CAM
in Cholangiocarcinoma Model}

\emph{Clinical cancer research : an official journal of the American
Association for Cancer Research 2019 Jul;():}

PubMed: \url{https://www.ncbi.nlm.nih.gov/pubmed/?term=31337646}

PURPOSE: Cholangiocarcinoma (CCA) is a malignancy of bile duct with a
poor prognosis. Conventional chemotherapy and radiation therapy are
generally ineffective and surgical resection is the only curative
treatment for CCA. L1-cell adhesion molecule (L1CAM) has been known as a
novel prognostic marker and therapeutic target for CCA. This study aimed
to evaluate the feasibility of immuno-positron emission tomography (PET)
imaging-based radioimmunotherapy using radiolabeled anti-L1CAM antibody
in CCA xenograft model. EXPERIMENTAL DESIGN: We prepared a theranostic
convergence bioradiopharmaceutical using chimeric anti-L1CAM antibody
(cA10-A3) conjugated with 1,4,7-triazacyclononane-1,4,7-triacetic acid
(NOTA) chelator and labeled with 64Cu or 177Lu and evaluated the
immuno-PET or SPECT/CT imaging and biodistribution with
64Cu-/177Lu-cA10-A3 in various CCA xenograft models. Therapeutic
efficacy and response monitoring were performed by 177Lu-cA10-A3 and
18F-FDG-PET, respectively, and immunohistochemistry was done by TUNEL
and Ki-67. RESULTS: Radiolabeled cA10-A3 antibodies specifically
recognized L1CAM in vitro, clearly visualized CCA tumors in immuno-PET
and SPECT/CT imaging, and differentiated the L1CAM expression level in
CCA xenograft models. 177Lu-cA10-A3 (12.95 MBq/100 μg) showed
statistically significant reduction in tumor volumes (P \textless{}
0.05) and decreased glucose metabolism (P \textless{} 0.01). IHC
analysis revealed 177Lu-cA10-A3 treatment increased TUNEL-positive and
decreased Ki-67-positive cells, compared with saline, cA10-A3, or
177Lu-isotype. CONCLUSIONS: Anti-L1CAM immuno-PET imaging using
64Cu-cA10-A3 could be translated into the clinic for characterizing the
pharmacokinetics and selecting pertinent patient for radioimmunotherapy.
Radioimmunotherapy using 177Lu-cA10-A3 may provide survival benefit in
L1CAM expressing CCA tumor. Theranostic convergence
bioradiopharmaceutical strategy would be applied as an imaging biomarker
based personalized medicine in L1CAM expressing CCA patients.

doi: \url{https://doi.org/10.1158/1078-0432.CCR-19-1157}

{}

{}

\begin{center}\rule{0.5\linewidth}{\linethickness}\end{center}

\begin{center}\rule{0.5\linewidth}{\linethickness}\end{center}

 - \textbf{Nivolumab alone or in combination with cisplatin plus
gemcitabine in Japanese patients with unresectable or recurrent biliary
tract cancer: a non-randomised, multicentre, open-label, phase 1 study}

\emph{The lancet. Gastroenterology \& hepatology 2019 Aug;4(8):611-621}

PubMed: \url{https://www.ncbi.nlm.nih.gov/pubmed/?term=31109808}

BACKGROUND: This study aimed to assess the safety and tolerability of
the immune checkpoint inhibitor nivolumab, as monotherapy or combined
with chemotherapy, in Japanese patients with biliary tract cancer.
METHODS: This multicentre, open-label, phase 1 trial was done at four
cancer centres in Japan. Eligible patients were aged 20-79 years, had
biliary tract adenocarcinoma (intrahepatic bile duct cancer,
extrahepatic bile duct cancer, gallbladder cancer, or ampullary cancer),
Eastern Cooperative Oncology Group performance status 0 or 1, adequate
hepatic, renal, and haematological function, and tumour tissue samples
for PD-L1 expression analysis. Patients with unresectable or recurrent
biliary tract cancer that was refractory or intolerant to
gemcitabine-based treatment regimens received nivolumab monotherapy (240
mg every 2 weeks {[}monotherapy cohort{]}). Chemotherapy-naive patients
with unresectable or recurrent biliary tract cancer received nivolumab
(240 mg every 2 weeks) and cisplatin (25 mg/m2) plus gemcitabine (1000
mg/m2) chemotherapy (combined therapy cohort). The primary objective was
to assess tolerability and safety. The primary objective was assessed in
the safety population of all patients who had received at least one dose
of nivolumab. This study is registered with www.clinicaltrials.jp,
number JapicCTI-153098, and follow-up is ongoing. FINDINGS: 30 patients
were enrolled into each cohort between Jan 13, 2016, and April 19, 2017.
Data cutoff was Aug 31, 2017. In the monotherapy cohort, the most
frequently reported treatment-related adverse events were decreased
appetite (five {[}17\%{]}), malaise (four {[}13\%{]}), and pruritus
(four {[}13\%{]}). Grade 3-4 treatment-related adverse events were
reported by three (10\%) patients (rash, maculopapular rash, and amylase
increase) and treatment-related serious adverse events were reported by
one (3\%) patient (pleurisy). In the combined therapy cohort, the most
frequently reported treatment-related adverse events were neutrophil
count decrease (any grade 25 {[}83\%{]}; grade 3-4 in 23 {[}77\%{]}
patients) and platelet count decrease (any grade 25 {[}83\%{]} of 30;
grade 3-4 in 15 {[}50\%{]} patients). Six (20\%) patients reported 11
treatment-related serious adverse events (platelet count decrease
{[}three patients{]}, febrile neutropenia {[}two patients{]}, neutrophil
count decrease, anaemia, anaphylactic reaction, decreased appetite,
pyrexia, and myocarditis {[}one patient each{]}). In the monotherapy
cohort, median overall survival was 5·2 months (90\% CI 4·5-8·7), median
progression-free survival was 1·4 months (90\% CI 1·4-1·4), and one of
30 patients had an objective response. In the combined therapy cohort,
median overall survival was 15·4 months (90\% CI 11·8-not estimable),
median progression-free survival was 4·2 months (90\% CI 2·8-5·6), and
11 of 30 patients had an objective response. INTERPRETATION: Nivolumab
had a manageable safety profile and signs of clinical activity in
patients with unresectable or recurrent biliary tract cancer. This
initial assessment of nivolumab for the treatment of advanced biliary
tract cancer provides supportive evidence for future larger randomised
studies of nivolumab in this difficult to treat cancer. FUNDING: Ono
Pharmaceutical Co Ltd and Bristol-Myers Squibb Inc.

doi: \url{https://doi.org/10.1016/S2468-1253(19)30086-X}

{}

{}

\begin{center}\rule{0.5\linewidth}{\linethickness}\end{center}

\begin{center}\rule{0.5\linewidth}{\linethickness}\end{center}

 - \textbf{Evaluation of histologic changes in the livers of patients
with early and late hepatic artery thrombosis}

\emph{Human pathology 2019 Aug;90():8-13}

PubMed: \url{https://www.ncbi.nlm.nih.gov/pubmed/?term=31075300}

Hepatic artery thrombosis (HAT) following orthotopic liver
transplantation (OLT) can cause hepatic parenchymal necrosis and
ischemic cholangiopathy. This study investigates additional histologic
features that may suggest HAT in post-OLT liver specimens. For 94 liver
specimens (explanted allografts and biopsies) from patients with a
clinical or pathologic diagnosis of HAT, we recorded length of time
between OLT and procedure, categorizing cases into early HAT (;≤30\,days
since OLT) and late HAT (\textgreater{}30\,days since OLT). Common
histologic findings in HAT included lobular necrosis (60 cases, 64\%),
portal inflammation (68 cases, 72\%), ductular reaction (73 cases,
78\%), lobular cholestasis (70 cases, 74\%), and bile-tinged macrophages
(40 cases, 43\%). Ductular cholestasis was seen in 30 cases (32\%); 10
of those patients were clinically septic. Bile in veins was seen in 16
(17\%) cases and arteritis in 6 (6\%) cases. Findings more common in
resection than biopsy specimens included lobular necrosis
(P\,\textless{}\,.0001), hemorrhage (P\,=\,.0044), ductular cholestasis
(P\,=\,.0003), and bile-tinged macrophages (P\,\textless{}\,.0001).
Lobular necrosis was more common in early HAT (P\,=\,.0002), and
ductular reaction (P\,=\,.006) and bile in veins (P\,=\,.03) were more
common in late HAT. Histologic changes in HAT vary based on specimen
type and whether HAT is early or late. In late HAT, biliary injury might
occur after a prolonged period of ischemia, with subsequent bile duct
necrosis, bile in veins, and remodeling (eg, ductular reaction). Bile in
veins is an unusual finding that may occur in HAT, although it can be
seen in bile infarcts from other causes.

doi: \url{https://doi.org/10.1016/j.humpath.2019.04.019}

{}

{}

\begin{center}\rule{0.5\linewidth}{\linethickness}\end{center}

\begin{center}\rule{0.5\linewidth}{\linethickness}\end{center}

 - \textbf{Intraoperative Air Leak Test to Prevent Bile Leak After Right
Posterior Sectionectomy with En Bloc Diaphragm Resection for Metastatic
Teratoma}

\emph{Annals of surgical oncology 2019 Aug;26(8):2579}

PubMed: \url{https://www.ncbi.nlm.nih.gov/pubmed/?term=31065963}

BACKGROUND: The intraoperative air cholangiogram, or ``air leak test''
(ALT), at the time of hepatectomy can significantly reduce the rates of
bile leak and symptomatic fluid collection after high-risk
procedures.1,2 Because a bile leak in the setting of an en bloc
diaphragm resection and mesh reconstruction would be a particularly
dreaded complication, this video shows the technique for resection,
reconstruction, and ALT. PRESENTATION: The video presents the case of a
29-year-old woman who had metastatic teratoma with an 8 × 7-cm liver
metastasis in segment 7 and diaphragm invasion to the level of the right
hepatic vein. OPERATION: The authors performed a formal right posterior
sectionectomy with en bloc diaphragm resection. The 12 × 8-cm
diaphragmatic defect was reconstructed using biologic mesh (Surgimend,
Integra LifeSciences, Plainsboro, NJ). An intraoperative ALT (air
injection into the cystic duct with finger compression of the distal
bile duct) identified several areas of bubbles from biliary radicles on
the cut surface of the liver, which were ligated with 4-0 polypropylene.
The ALT was repeated until no bubbles remained. Because no evidence of
bubbles was observed, no surgical drain was needed. The patient did well
postoperatively with no complications. CONCLUSION: In cases of combined
liver and diaphragmatic resection, prevention of bile leak, with
subsequent contamination of the diaphragm repair and even the thoracic
cavity, is particularly vital. An easily replicated intraoperative air
leak test can mitigate the risk of bile leak and organ-space infection,
as well as associated sequelae on quality of life, return to intended
oncologic therapy, and oncologic outcomes.

doi: \url{https://doi.org/10.1245/s10434-019-07410-y}

{}

{}

\begin{center}\rule{0.5\linewidth}{\linethickness}\end{center}

\begin{center}\rule{0.5\linewidth}{\linethickness}\end{center}

 - \textbf{Microbiota as a cornerstone in the development of primary
sclerosing cholangitis: paving the path for translational diagnostic and
therapeutic approaches}

\emph{Gut 2019 08;68(8):1353-1355}

PubMed: \url{https://www.ncbi.nlm.nih.gov/pubmed/?term=31028154}

doi: \url{https://doi.org/10.1136/gutjnl-2019-318487}

{}

{}

\begin{center}\rule{0.5\linewidth}{\linethickness}\end{center}

\begin{center}\rule{0.5\linewidth}{\linethickness}\end{center}

 - \textbf{Statin use and reduced risk of biliary tract cancers in the
UK Clinical Practice Research Datalink}

\emph{Gut 2019 08;68(8):1458-1464}

PubMed: \url{https://www.ncbi.nlm.nih.gov/pubmed/?term=30448774}

OBJECTIVE: To evaluate the association between statin use and risk of
biliary tract cancers (BTC). DESIGN: This is a nested case-control study
conducted in the UK Clinical Practice Research Datalink. We included
cases diagnosed with incident primary BTCs, including cancers of the
gall bladder, bile duct (ie, both intrahepatic and extrahepatic
cholangiocarcinoma), ampulla of Vater and mixed type, between 1990 and
2017. For each case, we selected five controls who did not develop BTCs
at the time of case diagnosis, matched by sex, year of birth, calendar
time and years of enrolment in the general practice using incidence
density sampling. Exposures were defined as two or more prescription
records of statins 1 year prior to BTC diagnosis or control selection.
ORs and 95\% CIs for associations between statins and BTC overall and by
subtypes were estimated using conditional logistic regression, adjusted
for relevant confounders. RESULTS: We included 3118 BTC cases and 15 519
cancer-free controls. Current statin use versus non-use was associated
with a reduced risk of all BTCs combined (adjusted OR=0.88, 95\% CI 0.79
to 0.98). The reduced risks were most pronounced among long-term users,
as indicated by increasing number of prescriptions (ptrend=0.016) and
cumulative dose of statins (ptrend=0.008). The magnitude of association
was similar for statin use and risk of individual types of BTCs. The
reduced risk of BTCs associated with a record of current statin use
versus non-use was more pronounced among persons with diabetes (adjusted
OR=0.72, 95\% CI 0.57 to 0.91). Among non-diabetics, the adjusted OR for
current statin use versus non-use was 0.91 (95\% CI 0.81 to 1.03,
pheterogeneity=0.007). CONCLUSION: Compared with non-use of statins,
current statin use is associated with 12\% lower risk of BTCs; no
association found with former statin use. If replicated, particularly in
countries with a high incidence of BTCs, our findings could pave the way
for evaluating the value of statins for BTC chemoprevention.

doi: \url{https://doi.org/10.1136/gutjnl-2018-317504}

{}

{}

\begin{center}\rule{0.5\linewidth}{\linethickness}\end{center}

\begin{center}\rule{0.5\linewidth}{\linethickness}\end{center}

 - \textbf{Oral vancomycin induces clinical and mucosal remission of
colitis in children with primary sclerosing cholangitis-ulcerative
colitis}

\emph{Gut 2019 08;68(8):1533-1535}

PubMed: \url{https://www.ncbi.nlm.nih.gov/pubmed/?term=30131321}

doi: \url{https://doi.org/10.1136/gutjnl-2018-316599}

{}

{}

\begin{center}\rule{0.5\linewidth}{\linethickness}\end{center}

\begin{center}\rule{0.5\linewidth}{\linethickness}\end{center}

Back to top

\begin{center}\rule{0.5\linewidth}{\linethickness}\end{center}

\pagebreak

\hypertarget{ampulla}{%
\subsubsection{Ampulla}\label{ampulla}}

\begin{center}\rule{0.5\linewidth}{\linethickness}\end{center}

 - \textbf{Response to the Letter to the Editor ``Minimally Invasive
Versus Open Distal Pancreatectomy (LEOPARD)''}

\emph{Annals of surgery 2019 Aug;():}

PubMed: \url{https://www.ncbi.nlm.nih.gov/pubmed/?term=31425333}

doi: \url{https://doi.org/10.1097/SLA.0000000000003541}

{}

{}

\begin{center}\rule{0.5\linewidth}{\linethickness}\end{center}

\begin{center}\rule{0.5\linewidth}{\linethickness}\end{center}

Back to top

\begin{center}\rule{0.5\linewidth}{\linethickness}\end{center}

\pagebreak


\end{document}
