% Options for packages loaded elsewhere
\PassOptionsToPackage{unicode}{hyperref}
\PassOptionsToPackage{hyphens}{url}
%
\documentclass[
]{article}
\usepackage{lmodern}
\usepackage{amssymb,amsmath}
\usepackage{ifxetex,ifluatex}
\ifnum 0\ifxetex 1\fi\ifluatex 1\fi=0 % if pdftex
  \usepackage[T1]{fontenc}
  \usepackage[utf8]{inputenc}
  \usepackage{textcomp} % provide euro and other symbols
\else % if luatex or xetex
  \usepackage{unicode-math}
  \defaultfontfeatures{Scale=MatchLowercase}
  \defaultfontfeatures[\rmfamily]{Ligatures=TeX,Scale=1}
\fi
% Use upquote if available, for straight quotes in verbatim environments
\IfFileExists{upquote.sty}{\usepackage{upquote}}{}
\IfFileExists{microtype.sty}{% use microtype if available
  \usepackage[]{microtype}
  \UseMicrotypeSet[protrusion]{basicmath} % disable protrusion for tt fonts
}{}
\makeatletter
\@ifundefined{KOMAClassName}{% if non-KOMA class
  \IfFileExists{parskip.sty}{%
    \usepackage{parskip}
  }{% else
    \setlength{\parindent}{0pt}
    \setlength{\parskip}{6pt plus 2pt minus 1pt}}
}{% if KOMA class
  \KOMAoptions{parskip=half}}
\makeatother
\usepackage{xcolor}
\IfFileExists{xurl.sty}{\usepackage{xurl}}{} % add URL line breaks if available
\IfFileExists{bookmark.sty}{\usepackage{bookmark}}{\usepackage{hyperref}}
\hypersetup{
  pdftitle={Pancreatobiliary Pathology Society Journal Watch},
  hidelinks,
  pdfcreator={LaTeX via pandoc}}
\urlstyle{same} % disable monospaced font for URLs
\usepackage[margin=1in]{geometry}
\usepackage{graphicx,grffile}
\makeatletter
\def\maxwidth{\ifdim\Gin@nat@width>\linewidth\linewidth\else\Gin@nat@width\fi}
\def\maxheight{\ifdim\Gin@nat@height>\textheight\textheight\else\Gin@nat@height\fi}
\makeatother
% Scale images if necessary, so that they will not overflow the page
% margins by default, and it is still possible to overwrite the defaults
% using explicit options in \includegraphics[width, height, ...]{}
\setkeys{Gin}{width=\maxwidth,height=\maxheight,keepaspectratio}
% Set default figure placement to htbp
\makeatletter
\def\fps@figure{htbp}
\makeatother
\setlength{\emergencystretch}{3em} % prevent overfull lines
\providecommand{\tightlist}{%
  \setlength{\itemsep}{0pt}\setlength{\parskip}{0pt}}
\setcounter{secnumdepth}{5}
\renewcommand{\linethickness}{0.05em}
\usepackage{booktabs}
\usepackage{sectsty} \allsectionsfont{\nohang\centering \emph}

\title{Pancreatobiliary Pathology Society Journal Watch}
\usepackage{etoolbox}
\makeatletter
\providecommand{\subtitle}[1]{% add subtitle to \maketitle
  \apptocmd{\@title}{\par {\large #1 \par}}{}{}
}
\makeatother
\subtitle{Recent Articles that will be Selected}
\author{}
\date{\vspace{-2.5em}Last Update on 2019-12-27}

\begin{document}
\maketitle

{
\setcounter{tocdepth}{2}
\tableofcontents
}
\begin{center}\rule{0.5\linewidth}{\linethickness}\end{center}

\hypertarget{pbpath_jw}{%
\section{PBPath Journal Watch Articles}\label{pbpath_jw}}

\begin{center}\rule{0.5\linewidth}{\linethickness}\end{center}

\textbf{Wellcome to the PBPath Journal Watch!}

This bi-monthly journal watch features exciting recently published
pancreas and biliary pathology articles that will provide up to date
medical knowledge in our field. These articles will be showcased in
several convenient categories, including surgical pathology,
cytopathology, and molecular pathology among others. The articles in
each category are in no particular order. See the list of journals we
search regularly \href{http://pbpath.org/pbpath-journal-watch/}{here}.
Previous months' issues may be found in our
\emph{\href{http://pbpath.org/journal-watch-archive/}{archive}} and you
may see \href{http://pbpath.org/journal-watch-upcoming-issue/}{drafts of
the upcoming issue here}.

We encourage members to actively participate by recommending new
articles and providing feedback using the
\protect\hyperlink{feedback}{forms provided below.}

\textbf{We hope that you will enjoy the new PBPath Journal Watch!}

\pagebreak

\hypertarget{pancreas}{%
\subsection{Pancreas}\label{pancreas}}

\begin{center}\rule{0.5\linewidth}{\linethickness}\end{center}

- \textbf{Diabetic Kidney Disease: Past and Present}

\emph{Advances in anatomic pathology 2019 Dec;():}

PubMed: \url{https://www.ncbi.nlm.nih.gov/pubmed/?term=31876542}

Diabetes mellitus (DM) afflicting humans has been recognized as a
disease for \textgreater3000 years. However, very little was known about
its etiology and pathogenesis until about a century ago when increasing
knowledge about anatomy and physiology of the human body gradually led
to our understanding that the hormone insulin produced by the Islets of
Langerhans in the pancreas plays a crucial role in the metabolism of
glucose and maintaining the blood sugar level within a normal range. DM
is caused by inadequate insulin production (type 1) or insulin
resistance (type 2). For thousands of years, DM has been considered as a
disease of the kidney; however, with the understanding of the
pathogenesis of DM, it became clear that diabetic kidney disease (DKD)
is a complication and not a cause of DM. DKD is associated with
increased matrix expansion that manifests morphologically as a diffuse
or nodular expansion of the mesangium and diffuse thickening of the
glomerular and tubular basement membranes. Hyperglycemia plays a crucial
role in the development of pathologic changes within the kidney. Once
established, DKD usually undergoes a slow but relentless progression to
end-stage renal disease. However, recent studies have shown that its
progression can be slowed or even reversed by strict control of
hyperglycemia. Morphologically, DKD may resemble several other
glomerular diseases that must be ruled out before a definitive
diagnosis. Patients with DM may also develop nondiabetic glomerular or
interstitial diseases with or without DKD. The findings in nephrectomy
specimens and the differential diagnoses are presented in detail.

doi: \url{https://doi.org/10.1097/PAP.0000000000000257}

\begin{center}\rule{0.5\linewidth}{\linethickness}\end{center}

\begin{center}\rule{0.5\linewidth}{\linethickness}\end{center}

- \textbf{Loss of GATA4 causes ectopic pancreas in the stomach}

\emph{The Journal of pathology 2019 Dec;():}

PubMed: \url{https://www.ncbi.nlm.nih.gov/pubmed/?term=31875961}

Pancreatic heterotopia is defined as pancreatic tissue outside its
normal location in the body and anatomically separated from the
pancreas. In this work we have analyzed the stomach glandular epithelium
of Gata4flox/flox ; Pdx1-Cre mice (Gata4KO mice). We found that Gata4KO
glandular epithelium displays an atypical morphology similar to the
cornified squamous epithelium and exhibits upregulation of forestomach
markers. The developing gastric units fail to form properly, and the
glandular epithelial cells do not express markers of gastric gland in
the absence of GATA4. Interestingly, the developing glands of the
Gata4KO stomach express pancreatic cell markers. Furthermore, a mass of
pancreatic tissue located in the subserosa of the Gata4KO stomach is
observed at adult stages. Heterotopic pancreas found in Gata4-deficient
mice contains all three pancreatic cell lineages, ductal, acinar and
endocrine. Moreover, Gata4 expression is downregulated in ectopic
pancreatic tissue of some human biopsy samples. This article is
protected by copyright. All rights reserved.

doi: \url{https://doi.org/10.1002/path.5378}

\begin{center}\rule{0.5\linewidth}{\linethickness}\end{center}

\begin{center}\rule{0.5\linewidth}{\linethickness}\end{center}

Back to top

\begin{center}\rule{0.5\linewidth}{\linethickness}\end{center}

\pagebreak

\hypertarget{gallbladder}{%
\subsection{Gallbladder}\label{gallbladder}}

\begin{center}\rule{0.5\linewidth}{\linethickness}\end{center}

- \textbf{Pancreatic cancer organoids recapitulate disease and allow
personalized drug screening}

\emph{Proceedings of the National Academy of Sciences of the United
States of America 2019 Dec;():}

PubMed: \url{https://www.ncbi.nlm.nih.gov/pubmed/?term=31818951}

We report the derivation of 30 patient-derived organoid lines (PDOs)
from tumors arising in the pancreas and distal bile duct. PDOs
recapitulate tumor histology and contain genetic alterations typical of
pancreatic cancer. In vitro testing of a panel of 76 therapeutic agents
revealed sensitivities currently not exploited in the clinic, and
underscores the importance of personalized approaches for effective
cancer treatment. The PRMT5 inhibitor EZP015556, shown to target MTAP (a
gene commonly lost in pancreatic cancer)-negative tumors, was validated
as such, but also appeared to constitute an effective therapy for a
subset of MTAP-positive tumors. Taken together, the work presented here
provides a platform to identify novel therapeutics to target pancreatic
tumor cells using PDOs.

doi: \url{https://doi.org/10.1073/pnas.1911273116}

\begin{center}\rule{0.5\linewidth}{\linethickness}\end{center}

\begin{center}\rule{0.5\linewidth}{\linethickness}\end{center}

Back to top

\begin{center}\rule{0.5\linewidth}{\linethickness}\end{center}

\pagebreak

\hypertarget{bileducts}{%
\subsection{Bile Ducts}\label{bileducts}}

\begin{center}\rule{0.5\linewidth}{\linethickness}\end{center}

- \textbf{Pancreatic cancer organoids recapitulate disease and allow
personalized drug screening}

\emph{Proceedings of the National Academy of Sciences of the United
States of America 2019 Dec;():}

PubMed: \url{https://www.ncbi.nlm.nih.gov/pubmed/?term=31818951}

We report the derivation of 30 patient-derived organoid lines (PDOs)
from tumors arising in the pancreas and distal bile duct. PDOs
recapitulate tumor histology and contain genetic alterations typical of
pancreatic cancer. In vitro testing of a panel of 76 therapeutic agents
revealed sensitivities currently not exploited in the clinic, and
underscores the importance of personalized approaches for effective
cancer treatment. The PRMT5 inhibitor EZP015556, shown to target MTAP (a
gene commonly lost in pancreatic cancer)-negative tumors, was validated
as such, but also appeared to constitute an effective therapy for a
subset of MTAP-positive tumors. Taken together, the work presented here
provides a platform to identify novel therapeutics to target pancreatic
tumor cells using PDOs.

doi: \url{https://doi.org/10.1073/pnas.1911273116}

\begin{center}\rule{0.5\linewidth}{\linethickness}\end{center}

\begin{center}\rule{0.5\linewidth}{\linethickness}\end{center}

- \textbf{Adult onset of genetic disorders in bile acid transport in the
liver}

\emph{Human pathology 2019 Dec;():}

PubMed: \url{https://www.ncbi.nlm.nih.gov/pubmed/?term=31669892}

Although severe deficiencies of canalicular transporter enzymes due to
biallelic mutations are well known as causes of progressive cholestatic
liver disease in children, it is increasingly recognized that milder
disease may occur if a single, heterozygous gene mutation is present.
This mild disease, generally presenting initially in adulthood, may have
a variety of clinical and histological appearances. Bland canalicular
cholestasis is the prototypic change, but it is now clear that some gene
mutations, particularly in ABCB4 (encoding MDR3), can cause other
patterns that include early cholesterol calculus formation, bile duct
injury and disappearance, ductular reactions mimicking large duct
obstruction, and, in rare cases, progressive fibrosis. Because the
features can be subtle and not diagnostic in isolation, it is generally
the combination of a biliary pattern of injury with a suggestive
clinical and family history that allows the diagnosis to be suspected.
Increased awareness and improved access to genetic testing are likely to
result in more frequent diagnosis of these disorders.

doi: \url{https://doi.org/10.1016/j.humpath.2019.10.006}

\begin{center}\rule{0.5\linewidth}{\linethickness}\end{center}

\begin{center}\rule{0.5\linewidth}{\linethickness}\end{center}

- \textbf{Landscape of distant metastasis mode and current chemotherapy
efficacy of the advanced biliary tract cancer in the United States,
2010-2016}

\emph{Cancer medicine 2019 Dec;():}

PubMed: \url{https://www.ncbi.nlm.nih.gov/pubmed/?term=31876990}

BACKGROUND: The distant metastasis (DM) mode and treatment efficacies in
the advanced biliary tract cancer (BTC) were obscure, and a credible
evaluation is urgently needed. METHOD: A total of 6348 advanced BTC
patients (ICC, intrahepatic cholangiocarcinoma, n~=~1762; PHCC,
perihilar cholangiocarcinoma, n~=~1103; GBC, gallbladder cancer,
n~=~2580; DCC, distal cholangiocarcinoma, n~=~538; AVC, carcinoma of
Vater ampulla, n~=~365) were enrolled from the Surveillance,
Epidemiology, and End Results (SEER) database. Propensity score matching
(PSM) process was carried out for less bias. RESULT: The proportion of
M1 patients in each subtype at first diagnosis was 26.4\% (ICC), 37.2\%
(PHCC), 41. 0\% (GBC), 24.5\% (DCC), and 12.7\% (AVC), and the
constitution of DM sites in different subtypes varied apparently.
Moreover, the survival of metastasis sites was different
(P~\textless~.05 in all the subtypes) where the multi-metastasis and
distant lymph node (dLN) only always indicated the worst and best
prognosis, respectively. Chemotherapy presented the most significant
survival impact with the lowest hazard ratio by multivariate cox model
and still provided a survival improvement after PSM (all
P~\textless~.001) in all subtypes. However, the median months manifested
different between patients with and without chemotherapy among the
subtypes (ICC, from 5 to 9; PHCC, from 6 to 10; AVC, from 4 to 9; GBC,
from 6 to 7; DCC from 6 to 8). CONCLUSION: We provided a landscape about
the detailed DM mode of the advanced BTC in a large population, found
the survival differences among DM sites, and revealed the different
chemotherapy efficacies in the BTC subtypes.

doi: \url{https://doi.org/10.1002/cam4.2794}

\begin{center}\rule{0.5\linewidth}{\linethickness}\end{center}

\begin{center}\rule{0.5\linewidth}{\linethickness}\end{center}

Back to top

\begin{center}\rule{0.5\linewidth}{\linethickness}\end{center}

\pagebreak

\hypertarget{ampulla}{%
\subsection{Ampulla}\label{ampulla}}

\begin{center}\rule{0.5\linewidth}{\linethickness}\end{center}

- \textbf{Microbial bile acid metabolites modulate gut RORγ+~regulatory
T cell homeostasis}

\emph{Nature 2019 Dec;():}

PubMed: \url{https://www.ncbi.nlm.nih.gov/pubmed/?term=31875848}

The metabolic pathways encoded by the human gut microbiome constantly
interact with host gene products through numerous bioactive molecules1.
Primary bile acids (BAs) are synthesized within hepatocytes and released
into the duodenum to facilitate absorption of lipids or fat-soluble
vitamins2. Some BAs (approximately 5\%) escape into the colon, where gut
commensal bacteria convert them into various intestinal BAs2 that are
important hormones that regulate host cholesterol metabolism and energy
balance via several nuclear receptors and/or G-protein-coupled
receptors3,4. These receptors have pivotal roles in shaping host innate
immune responses1,5. However, the effect of this host-microorganism
biliary network on the adaptive immune system remains poorly
characterized. Here we report that both dietary and microbial factors
influence the composition of the gut BA pool and modulate an important
population of colonic FOXP3+ regulatory T (Treg) cells expressing the
transcription factor RORγ. Genetic abolition of BA metabolic pathways in
individual gut symbionts significantly decreases this Treg cell
population. Restoration of the intestinal BA pool increases colonic
RORγ+~Treg cell counts and ameliorates host susceptibility to
inflammatory colitis via BA nuclear receptors. Thus, a pan-genomic
biliary network interaction between hosts and their bacterial symbionts
can control host immunological homeostasis via the resulting
metabolites.

doi: \url{https://doi.org/10.1038/s41586-019-1865-0}

\begin{center}\rule{0.5\linewidth}{\linethickness}\end{center}

\begin{center}\rule{0.5\linewidth}{\linethickness}\end{center}

- \textbf{Pathologic Evaluation of Endoscopically Resected Non-Ampullary
Duodenal Lesions: A Single Center Experience}

\emph{Turk patoloji dergisi 2019 Dec;():}

PubMed: \url{https://www.ncbi.nlm.nih.gov/pubmed/?term=31825518}

OBJECTIVE: Endoscopic resections are increasingly being used for
superficial gastrointestinal lesions. However, application of these
techniques in the duodenum remains challenging, due to the technical
difficulties and high complication rates. This study projects a western
tertiary center's experience in the endoscopic treatment and diagnostic
workup of 19 cases of non-ampullary duodenal lesions. MATERIAL AND
METHOD: Specimens (12 endoscopic mucosal resections, 6 endoscopic
submucosal dissections, and one endoscopic full-thickness resection)
were processed following a strict protocol (photographed, mapped
digitally and submitted totally) for histopathologic examination.
Clinicopathologic characteristics, margin status and follow-up
information were analyzed. RESULTS: The mean age of the 16 patients was
52 years (range: 22-81). Mean lesion size was 1.4 cm (range: 0.3-3.6 cm)
for all cases, 2 cm for endoscopic submucosal dissections and 1.1 cm for
endoscopic mucosal resections. Mean number of blocks submitted was
4/case. Seven neuroendocrine tumors, 3 tubulovillous adenomas were
diagnosed along with nine benign lesions. For endoscopic submucosal
dissections, en-bloc and R0 resection rates were 100\% (n=6/6) and 83\%
(n=5/6); for endoscopic mucosal resections, they were 92\% (n=11/12) and
83\% (n=10/12), respectively. Only one patient had procedure-related
late perforation that was managed endoscopically. No mortality was
encountered. CONCLUSION: Duodenal endoscopic resections proved
successful, safe and feasible methods in a tertiary center. The
pathologist's role is to designate the accurate diagnosis, related
histopathologic parameters and margin status. The gross protocol was
found to be essential in evaluating specimen margins and orientation, as
well as in size measurement. We recommend following a standardized
approach including gross photography and digital mapping when handling
these specimens, for both diagnostic and data collection purposes.

doi: \url{https://doi.org/10.5146/tjpath.2019.01474}

\begin{center}\rule{0.5\linewidth}{\linethickness}\end{center}

\begin{center}\rule{0.5\linewidth}{\linethickness}\end{center}

Back to top

\begin{center}\rule{0.5\linewidth}{\linethickness}\end{center}

\pagebreak

\begin{center}\rule{0.5\linewidth}{\linethickness}\end{center}

Back to top

\begin{center}\rule{0.5\linewidth}{\linethickness}\end{center}

\pagebreak

\hypertarget{feedback}{%
\section{Feedback}\label{feedback}}

\href{https://docs.google.com/forms/d/e/1FAIpQLSeD3Z9J6Y7eMmiyM12f_SfAmHUlykb1zxZcwO6lg7cebGYQIQ/viewform}{Google
Feedback Form}

\hypertarget{disqus_thread}{}

Please enable JavaScript to view the comments powered by Disqus.

\begin{center}\rule{0.5\linewidth}{\linethickness}\end{center}

Back to top

\begin{center}\rule{0.5\linewidth}{\linethickness}\end{center}

\pagebreak

\end{document}
