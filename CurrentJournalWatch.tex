\documentclass[]{article}
\usepackage{lmodern}
\usepackage{amssymb,amsmath}
\usepackage{ifxetex,ifluatex}
\usepackage{fixltx2e} % provides \textsubscript
\ifnum 0\ifxetex 1\fi\ifluatex 1\fi=0 % if pdftex
  \usepackage[T1]{fontenc}
  \usepackage[utf8]{inputenc}
\else % if luatex or xelatex
  \ifxetex
    \usepackage{mathspec}
  \else
    \usepackage{fontspec}
  \fi
  \defaultfontfeatures{Ligatures=TeX,Scale=MatchLowercase}
\fi
% use upquote if available, for straight quotes in verbatim environments
\IfFileExists{upquote.sty}{\usepackage{upquote}}{}
% use microtype if available
\IfFileExists{microtype.sty}{%
\usepackage{microtype}
\UseMicrotypeSet[protrusion]{basicmath} % disable protrusion for tt fonts
}{}
\usepackage[margin=1in]{geometry}
\usepackage{hyperref}
\hypersetup{unicode=true,
            pdftitle={PBPath Journal Watch},
            pdfborder={0 0 0},
            breaklinks=true}
\urlstyle{same}  % don't use monospace font for urls
\usepackage{graphicx,grffile}
\makeatletter
\def\maxwidth{\ifdim\Gin@nat@width>\linewidth\linewidth\else\Gin@nat@width\fi}
\def\maxheight{\ifdim\Gin@nat@height>\textheight\textheight\else\Gin@nat@height\fi}
\makeatother
% Scale images if necessary, so that they will not overflow the page
% margins by default, and it is still possible to overwrite the defaults
% using explicit options in \includegraphics[width, height, ...]{}
\setkeys{Gin}{width=\maxwidth,height=\maxheight,keepaspectratio}
\IfFileExists{parskip.sty}{%
\usepackage{parskip}
}{% else
\setlength{\parindent}{0pt}
\setlength{\parskip}{6pt plus 2pt minus 1pt}
}
\setlength{\emergencystretch}{3em}  % prevent overfull lines
\providecommand{\tightlist}{%
  \setlength{\itemsep}{0pt}\setlength{\parskip}{0pt}}
\setcounter{secnumdepth}{0}
% Redefines (sub)paragraphs to behave more like sections
\ifx\paragraph\undefined\else
\let\oldparagraph\paragraph
\renewcommand{\paragraph}[1]{\oldparagraph{#1}\mbox{}}
\fi
\ifx\subparagraph\undefined\else
\let\oldsubparagraph\subparagraph
\renewcommand{\subparagraph}[1]{\oldsubparagraph{#1}\mbox{}}
\fi

%%% Use protect on footnotes to avoid problems with footnotes in titles
\let\rmarkdownfootnote\footnote%
\def\footnote{\protect\rmarkdownfootnote}

%%% Change title format to be more compact
\usepackage{titling}

% Create subtitle command for use in maketitle
\newcommand{\subtitle}[1]{
  \posttitle{
    \begin{center}\large#1\end{center}
    }
}

\setlength{\droptitle}{-2em}

  \title{PBPath Journal Watch}
    \pretitle{\vspace{\droptitle}\centering\huge}
  \posttitle{\par}
  \subtitle{June July 2018}
  \author{}
    \preauthor{}\postauthor{}
      \predate{\centering\large\emph}
  \postdate{\par}
    \date{Last Update on 2018-07-21}


\begin{document}
\maketitle

{
\setcounter{tocdepth}{2}
\tableofcontents
}
\href{http://pbpath.org/}{\includegraphics{http://pbpath.org/wp-content/uploads/2017/05/cropped-PBP_header_logo_vector_lighter-1.jpg}}

\begin{center}\rule{0.5\linewidth}{\linethickness}\end{center}

\hypertarget{pancreas}{%
\subsection{Pancreas}\label{pancreas}}

\begin{center}\rule{0.5\linewidth}{\linethickness}\end{center}

\begin{itemize}
\tightlist
\item
  \textbf{S100A10, a Novel Biomarker in Pancreatic Ductal
  Adenocarcinoma}
\end{itemize}

\emph{Molecular oncology 2018 Jul;():}

Pancreatic cancer is arguably the deadliest cancer type. The efficacy of
current therapies is often hindered by the inability to predict patient
outcome. As such, the development of tools for early detection and risk
prediction is key for improving outcome and quality of life. Here, we
introduce the plasminogen receptor S100A10 as a novel predictive
biomarker and a driver of pancreatic tumor growth and invasion. We
demonstrated that S100A10 mRNA and protein are overexpressed in human
pancreatic tumors compared to normal ducts and non-ductal stroma.
S100A10 mRNA and methylation status were predictive of overall survival
and recurrence-free survival across multiple patient cohorts. S100A10
expression was driven by promoter methylation and the oncogene KRAS.
S100A10 knockdown reduced surface plasminogen activation, invasiveness
and in vivo growth of pancreatic cancer cell lines. These findings
delineate the clinical and functional contribution of S100A10 as a
biomarker in pancreatic cancer.

PubMed: \url{https://www.ncbi.nlm.nih.gov/pubmed/?term=30009399}

doi: \url{https://doi.org/10.1002/1878-0261.12356}

{}

{}

\begin{center}\rule{0.5\linewidth}{\linethickness}\end{center}

\begin{itemize}
\tightlist
\item
  \textbf{Integrative landscape of dysregulated signaling pathways of
  clinically distinct pancreatic cancer subtypes}
\end{itemize}

\emph{Oncotarget 2018 Jun;9(49):29123-29139}

Despite modern therapeutic advances, the survival prospects of
pancreatic cancer patients have remained poor. Besides being highly
metastatic, pancreatic cancer is challenging to treat because it is
caused by a heterogeneous array of somatic mutations that impact a
variety of signaling pathways and cellular regulatory systems. Here we
use publicly available transcriptomic, copy number alteration and
mutation profiling datasets from pancreatic cancer patients together
with data on disease outcomes to show that the three major pancreatic
cancer subtypes each display distinctive aberrations in cell signaling
and metabolic pathways. Importantly, patients afflicted with these
different pancreatic cancer subtypes also exhibit distinctive survival
profiles. Within these patients, we find that dysregulation of the
phosphoinositide 3-kinase and mitogen-activated protein kinase pathways,
and p53 mediated disruptions of cell cycle processes are apparently
drivers of disease. Further, we identify for the first time the
molecular perturbations of signalling networks that are likely the
primary causes of oncogenesis in each of the three pancreatic cancer
subtypes.

PubMed: \url{https://www.ncbi.nlm.nih.gov/pubmed/?term=30018740}

doi: \url{https://doi.org/10.18632/oncotarget.25632}

{}

{}

\begin{center}\rule{0.5\linewidth}{\linethickness}\end{center}

\begin{itemize}
\tightlist
\item
  \textbf{Genomic testing for pancreatic cancer in clinical practice as
  real-world evidence}
\end{itemize}

\url{https://www.pancreatology.net/article/S1424-3903(18)30633-1/abstract}

\begin{center}\rule{0.5\linewidth}{\linethickness}\end{center}

\begin{itemize}
\tightlist
\item
  \textbf{Pitfalls of Frozen Section Diagnosis for Paraganglioma: A
  Clinicopathologic Analysis and Review of the Literature}
\end{itemize}

\emph{International journal of surgical pathology 2018
May;26(3):213-220}

Few paraganglioma (PG) cases include frozen section diagnoses, and
therefore, the accuracy of frozen section diagnosis for PG remains
unknown. To better understand the histologic characteristics and
pitfalls of frozen section findings for PG, 15 PG cases with frozen
section diagnoses were selected from 12 articles through PubMed
(1984-2015). In addition, we included 3 cases of PG for which
intraoperative consultations were requested during a 5-year period
(2012-2016) in 2 hospitals. Seven PGs were from the thyroid; 2 from the
pancreas; 4 from the mediastinum, retroperitoneum, or mesentery; 1 from
the middle ear; and 4 from the urinary bladder (UB). Out of 18 PGs,
correct diagnoses were rendered in only 2 cases, and no thyroid or UB
PGs were correctly diagnosed on intraoperative consultation. Thyroid PGs
were frequently misdiagnosed as medullary thyroid carcinomas (4/7, 57\%)
and UB PGs were frequently misdiagnosed as malignancies.

PubMed: \url{https://www.ncbi.nlm.nih.gov/pubmed/?term=29169281}

doi: \url{https://doi.org/10.1177/1066896917742199}

{}

{}

\begin{center}\rule{0.5\linewidth}{\linethickness}\end{center}

\begin{itemize}
\tightlist
\item
  \textbf{Expression of liver X receptors in normal and refractory
  carcinoma tissues of the human lung and pancreas}
\end{itemize}

\emph{Histology and histopathology 2018 May;33(5):497-505}

Liver X receptors (LXRs) participate not only in maintaining cholesterol
homeostasis but also in controlling cellular growth in many types of
normal and tumor cells. We previously reported that LXRα was aberrantly
expressed in human oral squamous cell carcinoma (HOSCC) tissues and cell
lines, and that LXR stimulation led to significant reduction of
proliferation of HOSCC cells via accelerating cholesterol efflux. Since
LXRs and downstream proteins involved in cholesterol metabolism could be
also applied as therapeutic targets in small cell lung carcinoma (SCLC)
and pancreatic ductal adenocarcinoma (PDAC), we herein analyzed the
distribution of LXR proteins in these refractory cancers as well as in
normal human lung and pancreatic tissues. LXRβ was observed in ciliated
epithelial cells, bronchial gland epithelia, type II alveolar epithelia
and alveolar macrophages of the lung, and was less expressed in
bronchial basal cells and type I alveolar epithelia. In addition, LXRβ
was detected in epithelium of the pancreatic duct and acinar cells of
the pancreas, and was weakly expressed in pancreatic islet cells. By
contrast, LXRα expression was restricted to alveolar macrophages, and
was not evident in any types of epithelial cells in the lung and
pancreas. We also demonstrated that LXRβ but not LXRα was abundantly
expressed in nine cases of SCLC and twenty cases of PDAC tissues. These
findings provide basic information for evaluating the efficacy of
LXR-targeted treatment in SCLC and PDAC.

PubMed: \url{https://www.ncbi.nlm.nih.gov/pubmed/?term=29181837}

doi: \url{https://doi.org/10.14670/HH-11-949}

{}

{}

\begin{center}\rule{0.5\linewidth}{\linethickness}\end{center}

\begin{itemize}
\tightlist
\item
  \textbf{Significance of T1a and T1b Carcinoma Arising in Mucinous
  Cystic Neoplasm of Pancreas}
\end{itemize}

\emph{The American journal of surgical pathology 2018 May;42(5):578-586}

Mucinous cystic neoplasm (MCN) of pancreas is one of the precursor
lesions of pancreatic ductal adenocarcinoma. The 5-year disease-specific
survival for noninvasive MCNs was 100\% and 20\% to 60\% for those with
pancreatic ductal adenocarcinoma arising in a MCN. However, the
significance of T1a (≤0.5 cm) and T1b (\textgreater{}0.5 and
\textless{}1.0 cm) carcinoma arising in MCN as defined by the upcoming
American Joint Committee on Cancer, eighth edition is unclear. In this
study, we examined 3 cases of MCN with T1a or T1b carcinoma and compared
their clinicopathologic characteristics and survival to 46 cases of MCN
with low-grade dysplasia (MCN-LGD), 7 cases of MCN with high-grade
dysplasia (MCN-HGD), and 7 cases of MCN with advanced invasive carcinoma
(T2 or higher T stage). The tumors from all 3 cases were submitted in
their entirety in 123, 296, and 200 blocks, respectively. All 3 patients
were alive with no recurrence during the follow-up of 20.0, 113.8, and
137.2 months, respectively. Similarly, none of the patients who had MCN
with either LGD or HGD had recurrence or died of disease. In contrast, 5
of 7 patients who had MCN with advanced invasive carcinoma had
recurrence and later died of disease with a median survival of 22.9
months (P\textless{}0.001). Our study showed that MCN with T1a and T1b
carcinoma had an excellent prognosis similar to MCNs with LGD or HGD
after complete tumor sampling for histologic examination. Our results
along with the previous studies suggest that close follow-up, rather
than aggressive systemic therapy, may be a better approach for these
patients.

PubMed: \url{https://www.ncbi.nlm.nih.gov/pubmed/?term=29462092}

doi: \url{https://doi.org/10.1097/PAS.0000000000001040}

{}

{}

\begin{center}\rule{0.5\linewidth}{\linethickness}\end{center}

\begin{itemize}
\tightlist
\item
  \textbf{Interaction of amyloidogenic proteins in pancreatic β cells
  from subjects with synucleinopathies}
\end{itemize}

\emph{Acta neuropathologica 2018 Jun;135(6):877-886}

Parkinson's disease patients experience a wide range of non-motor
symptoms that may be provoked by deposits of phosphorylated α-synuclein
in the peripheral nervous system. Pre-existing diabetes mellitus might
be a risk factor for developing Parkinson's disease, and indeed, nearly
60\% of Parkinson's disease patients are insulin resistant. Thus, we
have investigated whether phosphorylated α-synuclein is deposited in
pancreatic tissue of subjects with synucleinopathies. We studied
pancreatic tissue from 39 subjects diagnosed with Parkinson's disease,
Lewy body Dementia or incidental Lewy bodies disease, as well as that
from 34 subjects with diabetes mellitus and a normal neuropathological
examination, and 52 subjects with a normal neuropathological
examination. We examined the pancreatic accumulation of phosphorylated
α-synuclein and of the islet amyloid polypeptide precursor (IAPP), an
amyloidogenic protein that plays an unknown role in diabetes mellitus,
but that can promote α-synuclein amyloid deposition in vitro. Moreover,
we performed proximity ligation assays to assess whether these two
proteins interact in the pancreas of these subjects. Cytoplasmic
phosphorylated α-synuclein deposits were found in the pancreatic β cells
of 14 subjects with Parkinson's disease (93\%), in 11 subjects with Lewy
body Dementia (85\%) and in 8 subjects with incidental Lewy body disease
(73\%). Furthermore, we found similar phosphorylated α-synuclein
inclusions in 23 subjects with a normal neuropathological examination
but with diabetes mellitus (68\%) and in 9 control subjects (17\%). In
addition, IAPP/α-synuclein interactions appear to occur in patients with
pancreatic inclusions of phosphorylated α-synuclein. The presence of
phosphorylated α-synuclein inclusions in pancreatic β cells provides a
new evidence of a mechanism that is potentially common to the
pathogenesis of diabetes mellitus, PD and DLB. Moreover, the interaction
of IAPP and α-synuclein in the pancreatic β cells of patients may
represent a novel target for the development of strategies to treat
these diseases.

PubMed: \url{https://www.ncbi.nlm.nih.gov/pubmed/?term=29536165}

doi: \url{https://doi.org/10.1007/s00401-018-1832-0}

{}

{}

\begin{center}\rule{0.5\linewidth}{\linethickness}\end{center}

\begin{itemize}
\tightlist
\item
  \textbf{c-MYC amplification and c-myc protein expression in pancreatic
  acinar cell carcinomas. New insights into the molecular signature of
  these rare cancers}
\end{itemize}

\emph{Virchows Archiv : an international journal of pathology 2018
May;():}

The molecular alterations of pancreatic acinar cell carcinomas (ACCs)
and mixed acinar-neuroendocrine carcinomas (MANECs) are not completely
understood, and the possible role of c-MYC amplification in tumor
development, progression, and prognosis is not known. We have
investigated c-MYC gene amplification in a series of 35 ACCs and 4
MANECs to evaluate its frequency and a possible prognostic role. Gene
amplification was investigated using interphasic fluorescence in situ
hybridization analysis simultaneously hybridizing c-MYC and the
centromere of chromosome 8 probes. Protein expression was
immunohistochemically investigated using a specific monoclonal
anti-c-myc antibody. Twenty cases had clones with different polysomies
of chromosome 8 in absence of c-MYC amplification, and 5 cases had one
amplified clone and other clones with chromosome 8 polysomy, while the
remaining 14 cases were diploid for chromosome 8 and lacked c-MYC
amplification. All MANECs showed c-MYC amplification and/or polysomy
which were observed in 54\% pure ACCs. Six cases (15.3\%) showed nuclear
immunoreactivity for c-myc, but only 4/39 cases showed simultaneous
c-MYC amplification/polysomy and nuclear protein expression. c-myc
immunoreactivity as well as c-MYC amplification and/or chromosome 8
polysomy was not statistically associated with prognosis. Our study
demonstrates that a subset of ACCs shows c-MYC alterations including
gene amplification and chromosome 8 polysomy. Although they are not
associated with a different prognostic signature, the fact that these
alterations are present in all MANECs suggests a role in the
acinar-neuroendocrine differentiation possibly involved in the
pathogenesis of MANECs.

PubMed: \url{https://www.ncbi.nlm.nih.gov/pubmed/?term=29721608}

doi: \url{https://doi.org/10.1007/s00428-018-2366-5}

{}

{}

\begin{center}\rule{0.5\linewidth}{\linethickness}\end{center}

\begin{itemize}
\tightlist
\item
  \textbf{Pancreatic Lipomatous Hamartoma: A Hitherto Unrecognized
  Variant}
\end{itemize}

\emph{The American journal of surgical pathology 2018 Jul;42(7):891-897}

Pancreatic masses consisting of lipomatous components clinically include
lipoma, liposarcoma, lipomatous pseudohypertrophy of the pancreas,
fat-containing neoplasms such as perivascular epithelioid cell tumor,
and malignant neoplasm with lipoid degeneration. We present pancreatic
lipomatous hamartoma, which has not been reported hitherto. A solid
pancreatic mass was detected from a computed tomographic scan check-up
in each of 3 cases of Japanese men. Macroscopically, well-demarcated
solid lipomatous masses were detected at the uncus, body, and tail of
the pancreas, respectively. Microscopically, the masses predominantly
consisted of mature adipocytes with no atypia, but contained
characteristics components of pancreatic hamartoma, such as small ducts,
a well-preserved acinar structure, and/or fibrous stroma. On the basis
of the unique features, lack of islets and absence of periductal elastic
fibers, these tumors are a distinct variant of pancreatic hamartoma.
Furthermore, high-mobility group AT-hook 2 expression in the
fibro-adipocytes of this tumor indicated that these cells are an
integral component of the pancreatic lipomatous hamartoma. Consequently,
the unique tumors described herein are pancreatic lipomatous hamartoma,
which must be discriminated from other lipomatous lesions of the
pancreas.

PubMed: \url{https://www.ncbi.nlm.nih.gov/pubmed/?term=29738363}

doi: \url{https://doi.org/10.1097/PAS.0000000000001075}

{}

{}

\begin{center}\rule{0.5\linewidth}{\linethickness}\end{center}

\begin{itemize}
\tightlist
\item
  \textbf{Primary Hepatoid Carcinoma of the Pancreas: A
  Clinicopathological Study of 3 Cases With Review of Additional 31
  Cases in the Literature}
\end{itemize}

\emph{International journal of surgical pathology 2018
Jun;():1066896918783468}

Primary pancreatic hepatoid carcinoma (PHC) is very rare. Here, we
reported 3 such cases with review of additional 31 cases in the
literature. Our 3 patients were male (83, 72, and 54 years old,
respectively). Serum α-fetoprotein (AFP) was elevated in 1 patient (case
3, 8338 ng/mL) and not measured in the other two. The PHC in patient 1
(pathological stage pT2N0M0) and patient 2 (pT3N0M0) showed pure
hepatocellular carcinoma (HCC)-like morphology, whereas in case 3 it was
a PHC with true glandular differentiation (pT4N0M0). The diagnosis of
PHC was confirmed with positive immunohistochemical staining in the
tumor cells for AFP (2/3), Hep Par 1 (3/3), glypican-3 (2/3), arginase-1
(2/3), and Sal-like protein 4 (1/3). CD10 and polyclonal
carcinoembryonic antigen stains show focal canalicular pattern in 2/3
tumors. Patient 1 did not receive further treatment after resection and
was alive with no evidence of disease at 107 months. Patient 2 died of
postoperative complications, whereas patient 3 received postsurgical
chemoradiation and died of disease at 29 months. Our findings and
literature review indicate that PHCs can be divided into 4 histological
subtypes: with pure HCC-like morphology (n = 22), with neuroendocrine
differentiation (n = 8), with true glandular differentiation (n = 3),
and with acinar cell differentiation (n = 1). On univariate analysis,
pure HCC-like morphology was associated with better disease-specific
survival (DSS; P = .04), whereas lymph node and distant metastases were
associated with worse DSS ( P = .002 for both). Age, gender, presenting
symptoms, serum AFP level, and T stage were not associated with DSS. On
multivariate analysis, none of these parameters was significantly
associated with DSS.

PubMed: \url{https://www.ncbi.nlm.nih.gov/pubmed/?term=29961402}

doi: \url{https://doi.org/10.1177/1066896918783468}

{}

{}

\begin{center}\rule{0.5\linewidth}{\linethickness}\end{center}

\begin{itemize}
\tightlist
\item
  \textbf{Hereditary Pancreatitis in the United States: Survival and
  Rates of Pancreatic Cancer}
\end{itemize}

\emph{The American journal of gastroenterology 2018 Jul;():}

OBJECTIVES: Hereditary pancreatitis (HP), an autosomal dominant disease
typically caused by mutations in PRSS1, has a broad range of clinical
characteristics and high cumulative risk of pancreatic cancer. We
describe survival and pancreatic cancer risk in the largest HP cohort in
the US. METHODS: HP probands and family members prospectively recruited
from 1995 to 2013 completed medical and family history questionnaires,
and provided blood for DNA testing. Overall survival (until 12/31/2015)
was determined from the Social Security Death Index (SSDI), National
Death Index (NDI), and family members. Cause of death was obtained from
the NDI. RESULTS: 217 PRSS1 carriers (181 symptomatic) formed the study
cohort. The most frequently detected mutations were p.R122H (83.9\%) and
p.N29I (11.5\%). Thirty-seven PRSS1 carriers (30 symptomatic, 7
asymptomatic) were deceased at conclusion of the study (5 from
pancreatic cancer). Median overall survival was 79.3 years (IQR
72.2-85.2). Risk of pancreatic cancer was significantly greater than
age- and sex- matched SEER data (SIR 59, 95\% CI 19-138), and cumulative
risk was 7.2\% (95\% CI 0-15.4) at 70 years. DISCUSSION: We confirm
prior observations on survival and pancreatic cancer SIR in PRSS1
subjects. Although risk of pancreatic cancer was significantly high in
these patients, its cumulative risk was much lower than previous
reports.

PubMed: \url{https://www.ncbi.nlm.nih.gov/pubmed/?term=30018304}

doi: \url{https://doi.org/10.1038/s41395-018-0194-5}

{}

{}

\begin{center}\rule{0.5\linewidth}{\linethickness}\end{center}

\begin{itemize}
\tightlist
\item
  \textbf{Loss of SATB2 Expression in Colorectal Carcinoma is Associated
  With DNA Mismatch Repair Protein Deficiency and BRAF Mutation}
\end{itemize}

\emph{The American journal of surgical pathology 2018 Jul;():}

The special AT-rich sequence binding protein (SATB2) has been reported
to be a specific immunohistochemical marker for colorectal carcinoma;
however, correlation of SATB2 expression with molecular alterations
commonly assessed in colorectal carcinoma has not been performed. We
examined the immunohistochemical expression of SATB2 in 586
adenocarcinomas of the gastrointestinal (GI) tract and pancreas to
assess its utility in diagnosis and analyze the clinicopathologic and
molecular characteristics of colorectal carcinoma stratified by SATB2
expression. SATB2 and CDX2 expression were evaluated in 266
adenocarcinomas of lower GI tract origin (246 colorectal and 20
appendiceal mucinous), 208 adenocarcinomas of upper GI tract and small
intestinal origin (74 esophagus/esophagogastric junction, 103 stomach,
20 duodenal, and 11 jejunoileal), and 112 pancreatic ductal
adenocarcinomas. SATB2 expression was more frequently identified in
adenocarcinomas of lower GI tract origin (222/266, 83\%) compared with
upper GI tract, small intestinal, or pancreatic origin (26/320, 8\%)
(P\textless{}0.001). Compared with CDX2 alone, dual positive expression
for SATB2 and CDX2 (SATB2/CDX2) has a significantly higher specificity
for adenocarcinoma of lower GI tract origin (94\% vs.~57\%,
P\textless{}0.001). In colorectal carcinoma, loss of SATB2 expression
was more frequently observed in DNA mismatch repair (MMR) protein
deficient tumors (31\%) compared with MMR protein proficient tumors
(13\%) (P\textless{}0.01). A BRAF V600E mutation was more frequently
identified in colorectal carcinomas with loss of SATB2 expression
compared with those with positive SATB2 expression (29\% vs.~3\%)
(P\textless{}0.001). In summary, SATB2 expression is a relatively
specific marker of lower GI tract origin; however, loss of SATB2
expression is more commonly seen in colorectal carcinoma with MMR
protein deficiency and BRAF mutation.

PubMed: \url{https://www.ncbi.nlm.nih.gov/pubmed/?term=30001238}

doi: \url{https://doi.org/10.1097/PAS.0000000000001116}

{}

{}

\begin{center}\rule{0.5\linewidth}{\linethickness}\end{center}

\begin{itemize}
\tightlist
\item
  \textbf{Mutations in BRCA1, BRCA2, and PALB2, and a panel of 50
  cancer-associated genes in pancreatic ductal adenocarcinoma}
\end{itemize}

\emph{Scientific reports 2018 May;8(1):8105}

\url{https://www.nature.com/articles/s41598-018-26526-x}

Mutations in genes of the breast cancer susceptibility gene (BRCA)
pathway, namely, BRCA1, BRCA2, and PALB2, can provide useful information
for the efficacy of platinum-based or poly ADP-ribose polymerase
inhibitors chemotherapeutic regimens. Pancreatic ductal adenocarcinoma
(PDAC) is an important target for such precision chemotherapies because
of its dismal prognosis. We analyzed mutations in the entire coding
regions of the BRCA pathway genes, expression of breast cancer 2
(BRCA2), and mutations in hotspots of 50 cancer-associated genes in 42
surgically resected PDACs, and evaluated their associations with
clinicopathological features. We identified 13 rare germline mutations
in the BRCA pathway genes; 68 somatic mutations in KRAS, TP53, SMAD4,
CDKN2A, GNAS, SMARCB1, and RB1; and 2 germline variations in MLH1. Among
them, BRCA2S2148fs was known to be pathogenic. BRCA2R18H and BRCA2G2044V
were enriched in tumor tissues. BRCA2K799R and BRCA2R2964T were novel
germline variations. Patients harboring potentially deleterious
mutations in the BRCA pathway genes showed significantly better
prognosis than those with benign mutations or no mutation. These results
indicate that rare germline variations in BRCA pathway genes could be
found more frequently than previously anticipated and, more importantly,
potentially deleterious mutations of them could be a favorable
prognostic factor in patients with resectable PDACs.

PubMed: \url{https://www.ncbi.nlm.nih.gov/pubmed/?term=29802286}

doi: \url{https://doi.org/10.1038/s41598-018-26526-x}

{}

{}

\begin{center}\rule{0.5\linewidth}{\linethickness}\end{center}

\begin{itemize}
\tightlist
\item
  \textbf{The BRG1/SOX9 axis is critical for acinar cell-derived
  pancreatic tumorigenesis}
\end{itemize}

\emph{The Journal of clinical investigation 2018 Jul;():}

Chromatin remodeler Brahma related gene 1 (BRG1) is silenced in
approximately 10\% of human pancreatic ductal adenocarcinomas (PDAs). We
previously showed that BRG1 inhibits the formation of intraductal
pancreatic mucinous neoplasm (IPMN) and that IPMN-derived PDA originated
from ductal cells. However, the role of BRG1 in pancreatic
intraepithelial neoplasia-derived (PanIN-derived) PDA that originated
from acinar cells remains elusive. Here, we found that exclusive
elimination of Brg1 in acinar cells of Ptf1a-CreER; KrasG12D; Brg1fl/fl
mice impaired the formation of acinar-to-ductal metaplasia (ADM) and
PanIN independently of p53 mutation, while PDA formation was inhibited
in the presence of p53 mutation. BRG1 bound to regions of the Sox9
promoter to regulate its expression and was critical for recruitment of
upstream regulators, including PDX1, to the Sox9 promoter and enhancer
in acinar cells. SOX9 expression was downregulated in BRG1-depleted
ADMs/PanINs. Notably, Sox9 overexpression canceled this PanIN-attenuated
phenotype in KBC mice. Furthermore, Brg1 deletion in established PanIN
by using a dual recombinase system resulted in regression of the lesions
in mice. Finally, BRG1 expression correlated with SOX9 expression in
human PDAs. In summary, BRG1 is critical for PanIN initiation and
progression through positive regulation of SOX9. Thus, the BRG1/SOX9
axis is a potential target for PanIN-derived PDA.

PubMed: \url{https://www.ncbi.nlm.nih.gov/pubmed/?term=30010625}

doi: \url{https://doi.org/10.1172/JCI94287}

{}

{}

\begin{center}\rule{0.5\linewidth}{\linethickness}\end{center}

\begin{itemize}
\tightlist
\item
  \textbf{Immunoprevention of Pancreatic Cancer}
\end{itemize}

\emph{Current medicinal chemistry 2018 ;25(22):2576-2584}

BACKGROUND: Pancreatic cancer (PC) is considered an incurable disease
due to late diagnosis, rapid spread and negligible response treatment
methods, with a 5-year survival rate of only 7\%. Hence, there is an
urgency in developing novel strategies for PC prevention. This review is
focused on discussing the challenges in understanding complex immune
functions in tumor microenvironment and host-induced immune responses
against tumors, selection of antigens for development of preventive
vaccines, lessons from immunoprevention clinical trials and challenges
in developing future vaccines. METHODS: 65 original articles were
referenced from various sources, based on immunoprevention or criteria
pertaining to tumor antigens and immune responses in PC. All these
articles were analyzed for the method details and results obtained, and
the existing challenges were derived for successful development of
clinical immunoprevention strategies. RESULTS: The analysis of these
articles and our experience with preclinical efficacy evaluations of
various preventive approaches against PC helped in identifying specific
tumor antigens as targets which can overcome tumor cell immune
suppression. This review discussed the status of primary, secondary and
tertiary preventive vaccines and reasons for failure of therapeutic
vaccines. The key parameters for effective vaccination were identified,
including stage of the disease for vaccination efficacy, use of
appropriate animal models for development of preventive vaccines.
Potential of chemopreventive agents as adjuvants in immunoprevention was
discussed. This review identified new challenges for development of
immunopreventive vaccines. CONCLUSION: This review analyzed various
aspects of vaccine development for immunoprevention of PC and emphasized
the challenges for development of immunoprevention strategies.

PubMed: \url{https://www.ncbi.nlm.nih.gov/pubmed/?term=28240168}

doi: \url{https://doi.org/10.2174/0929867324666170223153509}

{}

{}

\begin{center}\rule{0.5\linewidth}{\linethickness}\end{center}

\hypertarget{biliary-tract}{%
\subsection{Biliary Tract}\label{biliary-tract}}

\begin{center}\rule{0.5\linewidth}{\linethickness}\end{center}

\hypertarget{gallbladder}{%
\subsubsection{Gallbladder}\label{gallbladder}}

\begin{center}\rule{0.5\linewidth}{\linethickness}\end{center}

\begin{itemize}
\tightlist
\item
  \textbf{Epithelial Inclusions in Gallbladder Specimens Mimic Parasite
  Infection: Histologic and Molecular Examination of Reported
  Cystoisopora belli Infection in Gallbladders of Immunocompetent
  Patients}
\end{itemize}

\emph{The American journal of surgical pathology 2018 Jul;():}

Recent publications have described epithelial cytoplasmic vacuoles and
inclusions incidentally noted within gallbladder epithelium and
concluded that they represent coccidian parasite infection, in
particular, Cystoisospora belli. We identified 8 gallbladder specimens
from our institution in the past 3 years in which this diagnosis was
suggested or in which similar epithelial alterations were prominent.
Molecular analysis was performed on the 8 gallbladder specimens and on 3
positive control specimens: small bowel biopsies from acquired
immunodeficiency syndrome patients with diarrhea. Polymerase chain
reaction using primers designed to amplify an internal transcribed
spacer (ITS2) in the C. belli ribosomal gene cluster was performed on
the DNA samples. All 8 gallbladder specimens were negative for
amplification, while a product consistent with C. belli was amplified
from all 3 positive controls. Histologically, the gallbladder
cytoplasmic inclusions stained diffusely positive for Grocott-Gomori's
methenamine silver and Periodic acid-Schiff with diastase. In contrast,
sections from a positive control small bowel biopsy demonstrated
organisms that were negative for Grocott-Gomori's methenamine silver and
showed a distinct capsular and punctate internal staining on Periodic
acid-Schiff with diastase in various parasite forms. Together, the lack
of molecular evidence of C. belli and the distinct morphologic and
special staining patterns in these gallbladders compared with positive
control small bowel suggest that these epithelial changes do not
represent true C. belli infection. Our results suggest that gallbladders
of immunocompetent patients may occasionally show epithelial changes
that can morphologically mimic C. belli infection. Pathologists should
be aware of this histologic variant to minimize unnecessary treatment,
testing, and patient anxiety.

\url{https://journals.lww.com/ajsp/Pages/articleviewer.aspx?year=9000\&issue=00000\&article=97799\&type=Abstract}

PubMed: \url{https://www.ncbi.nlm.nih.gov/pubmed/?term=30020094}

doi: \url{https://doi.org/10.1097/PAS.0000000000001094}

{}

{}

\begin{center}\rule{0.5\linewidth}{\linethickness}\end{center}

\begin{itemize}
\tightlist
\item
  \textbf{Pyloric Gland Adenoma (PGA) of the Gallbladder: A Unique and
  Distinct Tumor from PGAs of the Stomach, Duodenum, and Pancreas}
\end{itemize}

\emph{The American journal of surgical pathology 2018 Jul;():}

Twenty-four surgically resected, gallbladder pyloric gland adenomas
(GB-PGAs) were examined and their features were compared with the
reported features of stomach, duodenum, and pancreatic PGAs to better
understand GB-PGAs. Clinical information on background gallbladder
lesions and histologic data, including tumor grade, existence of
squamoid morules, intratumoral cholesterosis, and intracytoplasmic
mucins were collected. Immunohistochemical staining for MUC2, MUC5AC,
MUC6, CDX2, pepsinogen I, p53, and MIB-1/nuclear β-catenin were
evaluated. Targeted mutational analyses of KRAS exon2, GNAS exon 7, and
CTNNB1 exon 3 were conducted. We found that 29.2\% of the GB-PGAs were
histologically high-grade dysplasias/carcinomas; 70.8\% were low grade;
and 20.8\% and 33.3\% contained squamoid morules and intratumoral
cholesterosis, respectively. In addition, 45.8\% and 54.2\% of GB-PGAs
were mucin-rich and mucin-poor types, respectively.
Immunohistochemically, MUC6 was diffusely positive in all GB-PGAs; MUC2,
MUC5AC, and CDX2 were only focally positive, and no pepsinogen-I
positive cells were observed. Nuclear β-catenin accumulation was
observed in all cases; however, the ratio varied among cases. Mucin-poor
types were significantly associated with high histologic grade
dysplasias/carcinomas and high nuclear β-catenin labeling indices.
Mutational analyses identified CTNNB1 mutations in 100\% of GB-PGAs
(21/21), KRAS in 4.2\% (1/23), and GNAS in 0\% (0/22). The present study
clarified the unique histologic features, phenotypic differentiation,
and molecular statuses frequently associated with GB-PGAs. Altogether,
our data suggest that tumorigenesis of GB-PGA is distinct from that of
stomach, duodenum, and pancreatic PGAs.

PubMed: \url{https://www.ncbi.nlm.nih.gov/pubmed/?term=29975247}

doi: \url{https://doi.org/10.1097/PAS.0000000000001117}

{}

{}

\begin{center}\rule{0.5\linewidth}{\linethickness}\end{center}

\begin{itemize}
\tightlist
\item
  \textbf{Clinical relevance of PD-L1 expression in gallbladder cancer:
  a potential target for therapy}
\end{itemize}

\emph{Histopathology 2018 Jun;():}

AIMS: Programmed death-ligand 1 (PD-L1), a potential target for immune
checkpoint inhibitors in various solid neoplasms, has been studied in
very few cases of Gall Bladder Carcinoma (GBC). The current study aimed
to evaluate PD-L1 expression at primary and metastatic sites of GBC, and
its associations with standard prognostic clinicopathological
parameters, as well as with overall survival. METHODS AND RESULTS: One
hundred and seventy-four cases of GBC were evaluated for PD-L1
expression by the use of the SP263 clone in tissue microarrays.
Clinicopathological characteristics and survival data were correlated
with PD-L1 expression analysed at different cut-offs of ≥1\%, ≥10\% and
≥50\% in tumour cells and tumour-infiltrating lymphocytes (TILs). The
mean age of patients was 49.9 years, and the male/female ratio was
1:2.9. Of the cases, 73.6\% presented with stage 3/4 disease. Tumour
cells expressed PD-L1 in 23.0\% of cases, and TILs expressed PD-L1 in
24.1\% of cases. At a cut-off of 10\%, 14.9\% of cases expressed PD-L1,
and at a cut-off of 50\%, 7.5\% of cases expressed PD-L1. Significant
associations were seen between tumour proportion score and histological
type (P = 0.004), histological grade (P = 0.004), nuclear grade (P =
0.008), nodal metastasis (P = 0.051), higher stage (P = 0.058), and TILs
(P \textless{} 0.001). Tumour size, growth pattern, the presence of
necrosis and lymphovascular emboli showed no significant associations
with PD-L1 in tumour cells or TILs. In synchronous paired samples from
primary and metastatic lymph nodes, discordantly higher PD-L1 expression
was evident in lymph nodes. Overall survival was not associated with
PD-L1 expression (P = 0.546). CONCLUSION: PD-L1 does not appear to be a
prognostic marker or influence survival in GBC patients. However, PD-L1
expression occurs in one of four GBCs, supporting the future possibility
of immune-modulation therapy to improve the dismal overall survival.

PubMed: \url{https://www.ncbi.nlm.nih.gov/pubmed/?term=29882997}

doi: \url{https://doi.org/10.1111/his.13669}

{}

{}

\begin{center}\rule{0.5\linewidth}{\linethickness}\end{center}

\begin{itemize}
\tightlist
\item
  \textbf{Intramural capillary hemangioma of gallbladder artery:
  age-related or treatment-induced change?}
\end{itemize}

\emph{Cardiovascular pathology : the official journal of the Society for
Cardiovascular Pathology 2018 Jun;36():22-24}

PubMed: \url{https://www.ncbi.nlm.nih.gov/pubmed/?term=30005393}

doi: \url{https://doi.org/10.1016/j.carpath.2018.05.006}

{}

{}

\begin{center}\rule{0.5\linewidth}{\linethickness}\end{center}

\begin{itemize}
\tightlist
\item
  \textbf{Research gaps and unanswered questions in gallbladder cancer}
\end{itemize}

\emph{HPB : the official journal of the International Hepato Pancreato
Biliary Association 2018 May;():}

PubMed: \url{https://www.ncbi.nlm.nih.gov/pubmed/?term=29784604}

doi: \url{https://doi.org/10.1016/j.hpb.2018.03.006}

{}

{}

\begin{center}\rule{0.5\linewidth}{\linethickness}\end{center}

\begin{itemize}
\tightlist
\item
  \textbf{Contrast-enhanced ultrasound findings of gallbladder
  adenocarcinoma with sarcomatoid carcinoma accompanied by intrahepatic
  metastasis: A case report and literature review}
\end{itemize}

\emph{Medicine 2018 May;97(21):e10773}

RATIONALE: Gallbladder adenocarcinoma is the most common subtype of
malignant tumor in gallbladder carcinoma; its incidence is quite low.
Preoperative diagnosis of gallbladder adenocarcinoma appears to be very
difficult because they are normally asymptomatic and are often found
incidentally. Adenocarcinoma with sarcomatoid carcinoma of gallbladder
is exceedingly rare with no case reported in the literature to date.
PATIENT CONCERNS: A 69-year-old male was referred to our hospital due to
right upper quadrant pain. To further diagnosis, he received laboratory
tests, gray-scale US, and contrast-enhanced ultrasound (CEUS). In this
present study, ethical approval was not necessary, as this article is a
case report, which is based on the clinical information of the patient.
The patient gave his permission for publication of the case. DIAGNOSES:
Gray-scale abdominal ultrasound image demonstrated a hypoechoic
intraluminal mass of gallbladder and a heterogeneous mass at the
inferior segment of the right anterior liver lobe. In CEUS, the lesion
of gallbladder was hyper-enhanced in the arterial phase, and
hypo-enhanced in the later arterial phase. On liver, the hepatic lesion
appeared heterogenous mild enhancement in the arterial phase. In portal
and parenchymal phase, the hepatic lesion displayed hypo-enhancement,
which conformed to the hepatic metastasis. INTERVENTIONS: The patient
received cholecystectomy, right hepatectomy, and lymphadenectomy.
OUTCOMES: He had an uneventful recovery from the procedure. LESSONS: We
presented a case report with gallbladder adenocarcinoma with sarcomatoid
carcinoma accompanied by intrahepatic metastasis, focusing on the
ultrasound imaging features of this tumor, especially its enhancement
pattern on CEUS. CEUS provided helpful information for diagnosis.
Herein, we report the case; and the associated literature is further
reviewed.

PubMed: \url{https://www.ncbi.nlm.nih.gov/pubmed/?term=29794755}

doi: \url{https://doi.org/10.1097/MD.0000000000010773}

{}

{}

\begin{center}\rule{0.5\linewidth}{\linethickness}\end{center}

\hypertarget{bile-duct}{%
\subsubsection{Bile Duct}\label{bile-duct}}

\begin{center}\rule{0.5\linewidth}{\linethickness}\end{center}

\begin{itemize}
\tightlist
\item
  \textbf{Coexistence of double gallbladder with cholangiocarcinoma: A
  case report}
\end{itemize}

\emph{Medicine 2018 Jun;97(25):e11015}

RATIONALE: Gallbladder duplication is a rare congenital disorder, which
could cause an increasing risk of complications during surgery. The
coexistence of cholangiocarcinoma with double gallbladder is extremely
rare, which might lead to an even higher possibility of misdiagnosis and
postsurgery complications. PATIENT CONCERNS: A 58-year-old female was
presented with abdominal pain and jaundice. Abdominal ultrasonography
showed duplication of gallbladder, one of which with a thickened wall
and a rough surface. This was also confirmed by an abdominal computed
tomography (CT), magnetic resonance imaging (MRI) and magnetic resonance
cholangiopancreatography (MRCP) scan. During the surgery, we found a
tumor inside one bile duct. The postsurgery pathology showed
adenosquamous carcinoma. DIAGNOSES: Gallbladder duplication,
cholangiocarcinoma. INTERVENTIONS: The tumor was removed by surgery.
OUTCOMES: The patient died of tumor relapse six months after surgery.
LESSONS: This is the first reported case with coexistence of gallbladder
duplication and cholangiocarcinoma, which was diagnosed by abdominal
ultrasound, CT and MRCP, as well as further confirmed in surgery and
pathology. This case emphasized the importance of a thorough examination
of gallbladder before surgery, especially in those cases with suspected
double gallbladder, since each gallbladder could have the possibility of
an independent cholangiocarcinoma.

PubMed: \url{https://www.ncbi.nlm.nih.gov/pubmed/?term=29923988}

doi: \url{https://doi.org/10.1097/MD.0000000000011015}

{}

{}

\begin{center}\rule{0.5\linewidth}{\linethickness}\end{center}

\begin{itemize}
\tightlist
\item
  \textbf{Pyloric metaplasia, pseudopyloric metaplasia, ulcer-associated
  cell lineage and spasmolytic polypeptide-expressing metaplasia:
  reparative lineages in the gastrointestinal mucosa}
\end{itemize}

\emph{The Journal of pathology 2018 Jun;245(2):132-137}

The gastrointestinal mucosae provide a critical barrier between the
external and internal milieu. Thus, damage to the mucosa requires an
immediate response to provide appropriate wound closure and healing.
Metaplastic lineages with phenotypes similar to the mucous glands of the
distal stomach or Brunner's glands have been associated with various
injurious scenarios in the stomach, small bowel, and colon. These
lineages have been assigned various names including pyloric metaplasia,
pseudopyloric metaplasia, ulcer-associated cell lineage (UACL), and
spasmolytic polypeptide-expressing metaplasia (SPEM). A re-examination
of the literature on these various forms of mucous cell metaplasia
suggests that pyloric-type mucosal gland lineages may provide a
ubiquitous response to mucosal injury throughout the gastrointestinal
tract as well as in the pancreas, esophagus, and other mucosal surfaces.
While the cellular origin of these putative reparative lineages likely
varies in different regions of the gut, their final phenotypes may
converge on a pyloric-type gland dedicated to mucous secretion. In
addition to their healing properties in the setting of acute injury,
these pyloric-type lineages may also represent precursors to neoplastic
transitions in the face of chronic inflammatory influences. Further
investigations are needed to determine how discrete molecular profiles
relate to the origin and function of pyloric-type metaplasias previously
described by histological characteristics in multiple epithelial mucosal
systems in the setting of acute and chronic damage. Copyright © 2018
Pathological Society of Great Britain and Ireland. Published by John
Wiley \& Sons, Ltd.

PubMed: \url{https://www.ncbi.nlm.nih.gov/pubmed/?term=29508389}

doi: \url{https://doi.org/10.1002/path.5066}

{}

{}

\begin{center}\rule{0.5\linewidth}{\linethickness}\end{center}

\begin{itemize}
\tightlist
\item
  \textbf{Interleukin-33 overexpression reflects less aggressive tumour
  features in large-duct type cholangiocarcinomas}
\end{itemize}

\emph{Histopathology 2018 Aug;73(2):259-272}

AIMS: The aim of the present study was to elucidate the
clinicopathological significance of interleukin (IL)-6 and IL-33
expression in intrahepatic cholangiocarcinomas (iCCAs) and perihilar
cholangiocarcinomas (pCCAs). METHODS AND RESULTS: IL-6 and IL-33 mRNA
expression levels were examined in iCCAs (n = 55) and pCCAs (n = 32) by
the use of quantitative real-time polymerase chain reaction and a highly
sensitive in-situ hybridisation protocol (RNAscope), and expression
levels were correlated with clinicopathological features. According to a
recently proposed classification scheme, iCCAs were separated into
small-duct (n = 33) and large-duct (n = 22) types. IL-6 and IL-33
expression levels were higher in large-duct iCCAs and pCCAs than in
small-duct iCCAs, and there was a positive correlation between the
expression levels of these cytokines. Double in-situ
hybridisation/immunostaining showed that IL-6 mRNA was expressed in
actin-positive (myo)fibroblasts, whereas IL-33 mRNA was mainly produced
by CD31-positive endothelial cells. With the average expression level as
a cut-off point, cases were classified as IL-6high and IL-6low or
IL-33high and IL-33low . In the combined cohort of large-duct iCCAs and
pCCAs, IL-6high and IL-6low cholangiocarcinomas shared many features,
whereas IL-33high cases had less aggressive characteristics than
IL-33low cases, as shown by lower tumour marker concentrations, smaller
tumour sizes, less common vascular invasion, lower pT stages, and higher
lymphocyte/monocyte ratios in blood. KRAS mutations were slightly less
common in IL-33high cases than in IL-33low cases (9\% versus 29\%; P =
0.061). The strong expression of IL-33 in tissue appeared to be an
independent favourable prognostic factor. CONCLUSIONS: IL-33high
cholangiocarcinomas may represent a unique, less aggressive
carcinogenetic process of the large bile ducts.

PubMed: \url{https://www.ncbi.nlm.nih.gov/pubmed/?term=29675965}

doi: \url{https://doi.org/10.1111/his.13633}

{}

{}

\begin{center}\rule{0.5\linewidth}{\linethickness}\end{center}

\begin{itemize}
\tightlist
\item
  \textbf{B7-H3 expression and its correlation with clinicopathologic
  features, angiogenesis, and prognosis in intrahepatic
  cholangiocarcinoma}
\end{itemize}

\emph{APMIS : acta pathologica, microbiologica, et immunologica
Scandinavica 2018 May;126(5):396-402}

This study was designed to explore the expression of B7-H3 in human
intrahepatic cholangiocarcinoma (ICC) and its association with the
clinicopathologic factors. In the current study, the expression of B7-H3
in 45 patients with intrahepatic cholangiocarcinoma and 8 patients with
hepatolithiasis was analyzed by immunohistochemistry, which revealed
that B7-H3 was not expressed in hepatolithiatic tissues, but positively
expressed in 57.8\% (26/45) of the ICC cases. The expression of B7-H3
was significantly associated with lymph node metastases and venous
invasion. A positive correlation was also observed between the
expression of B7-H3 and MVD, an index for tumor angiogenesis. Further
survival analysis indicated that patients with B7-H3 negative expression
had higher overall survival (OS) and cancer-specific survival (CSS)
rates than those with B7-H3 positive expression. Multivariate analysis
revealed that B7-H3 expression was an independent prognostic indicator
for poor OS and CSS of ICC patients. Our results suggest that B7-H3 may
be a valuable biomarker in determining tumor progression and prognosis
of intrahepatic cholangiocarcinoma. It is also a potential target for
antivascular therapy of ICC.

PubMed: \url{https://www.ncbi.nlm.nih.gov/pubmed/?term=29696716}

doi: \url{https://doi.org/10.1111/apm.12837}

{}

{}

\begin{center}\rule{0.5\linewidth}{\linethickness}\end{center}

\begin{itemize}
\tightlist
\item
  \textbf{SurePath® LBC improves the diagnostic accuracy of intrahepatic
  and hilar cholangiocarcinoma}
\end{itemize}

\emph{Cytopathology : official journal of the British Society for
Clinical Cytology 2018 May;():}

INTRODUCTION: The current study aimed to compare cytology using
SurePath® (SP)-LBC and biliary tissue histology (BTH) for the diagnosis
of biliary disease. METHODS: Between January 2014 and December 2016, 57
patients underwent endoscopic retrograde cholangiopancreatography for
the diagnosis of biliary disease. Biliary cytological samples were
processed using SP-LBC and subsequently BTH was performed. A final
diagnosis was confirmed by surgery (23 malignant cases) and clinical
follow-up (34 benign and malignant cases): 18 extrahepatic
cholangiocarcinoma; 17 intrahepatic/hilar cholangiocarcinoma
(intra/H-CC); eight other malignant disease; and 14 benign biliary
disease. The diagnoses made using SP-LBC and BTH were classified into
four categories: (1) benign; (2) indeterminate; (3) suspicious for
malignancy/malignant; and (4) inadequate. In addition, diagnostic
accuracy was compared between SP-LBC and BTH. RESULTS: Although 23\%
(13/57) of BTH samples were classified as inadequate, all SP-LBC cases
were classified as adequate. Among 43 malignant cases, 11 normal, four
indeterminate and 28 suspicious for malignancy/malignant were found
using SP-LBC (26\%, 9\% and 65\%, respectively), in contrast to 10
inadequate, nine normal, 10 indeterminate and 14 suspicious for
malignancy/malignant observed using BTH (23\%, 21\%, 23\%, and 33\%,
respectively). The identification of malignant cells was strikingly
different between SP-LBC and BTH. Furthermore, limited to intra/H-CC,
accuracy was significantly higher using SP-LBC than using BTH
(P~\textless{}~.001). CONCLUSIONS: SP-LBC of the biliary tract is a
useful and reliable method for diagnosing biliary malignant disease and
has an advantage over BTH for detecting malignant cells and accurately
diagnosing intra/H-CC.

PubMed: \url{https://www.ncbi.nlm.nih.gov/pubmed/?term=29723910}

doi: \url{https://doi.org/10.1111/cyt.12565}

{}

{}

\begin{center}\rule{0.5\linewidth}{\linethickness}\end{center}

\begin{itemize}
\tightlist
\item
  \textbf{DJ-1 is a useful biomarker for invasive extrahepatic
  cholangiocarcinoma}
\end{itemize}

\emph{Human pathology 2018 Jun;76():28-36}

We have previously reported that DJ-1 protein is up-regulated in
cholangiocarcinoma compared with non-neoplastic epithelium of the bile
duct in a study using liquid-chromatography mass spectrometry-based
proteomics. The aim of this study was to clarify whether DJ-1 expression
offers a biomarker for patients with invasive extrahepatic
cholangiocarcinoma (EHCC) who undergo surgical resection with curative
intent. Positive immunohistochemical (IHC) staining of DJ-1 was
significantly more frequent in the cytoplasm of 96 invasive EHCCs
(n\,=\,28, 29.2\%) than in that of 66 non-neoplastic epithelial lesions
adjacent to invasive EHCC (n\,=\,7, 10.6\%; P\,=\,.006). No significant
difference in clinicopathological features was evident between invasive
EHCC patients with negative (n\,=\,68) and positive (n\,=\,28) IHC
staining. However, negative IHC staining for DJ-1 in cytoplasm was
selected as an independent risk factor for adverse prognosis on
multivariate analysis (P\,=\,.004, hazard ratio 2.13, 95\% confidence
interval 1.28-3.57). Serum levels of DJ-1 in 16 invasive EHCC patients
with metastasis were compared with 12 invasive EHCC patients without
metastasis. Serum levels of DJ-1 tended to be higher in 16 patients with
metastasis (median, 40.9\,ng/ml) than in 12 patients without
(27.6\,ng/ml, P\,=\,.137). In addition, patients with high serum levels
(≥ 40\,ng/ml) of DJ-1 tended to have metastasis more frequently than
those without (P\,=\,.054, Fisher's exact test). We concluded that IHC
staining pattern and serum level of DJ-1 in patients with invasive EHCC
might be predictive of prognosis and metastasis, respectively.

PubMed: \url{https://www.ncbi.nlm.nih.gov/pubmed/?term=29447925}

doi: \url{https://doi.org/10.1016/j.humpath.2018.02.010}

{}

{}

\begin{center}\rule{0.5\linewidth}{\linethickness}\end{center}

\begin{itemize}
\tightlist
\item
  \textbf{Intrahepatic Cholangiocarcinomas Have Histologically and
  Immunophenotypically Distinct Small and Large Duct Patterns}
\end{itemize}

\emph{The American journal of surgical pathology 2018 Jul;():}

Intrahepatic cholangiocarcinomas are histologically heterogenous. Using
a cohort of 184 clinically defined, resected intrahepatic
cholangiocarcinomas, we retrospectively classified the histology into 4
subtypes: large duct (LD), small duct (SD) (predominantly tubular
{[}SD1{]} or predominantly anastomosing/cholangiolar, {[}SD2{]}), or
indeterminate. Then, we tested the 4 subtypes for associations with risk
factors, patient outcomes, histology, and immunophenotypic
characteristics. SD was the most common (84\%; 24\% SD1 and 60\% SD2)
with lower proportions of LD (8\%), and indeterminate (8\%). Primary
sclerosing cholangitis was rare (2\%), but correlated with LD (P=0.005).
Chronic hepatitis, frequent alcohol use, smoking, and steatosis had no
histologic association. LD was associated with mucin production
(P\textless{}0.001), perineural invasion (P=0.002), CA19-9 staining
(P\textless{}0.001), CK7, CK19, CD56 immunophenotype (P=0.005), and
negative albumin RNA in situ hybridization (P\textless{}0.001). SD was
histologically nodular (P=0.019), sclerotic (P\textless{}0.001),
hepatoid (P=0.042), and infiltrative at the interface with hepatocytes
(P\textless{}0.001). Albumin was positive in 71\% of SD and 18\% of LD
(P=0.0021). Most albumin positive tumors (85\%) lacked extracellular
mucin (P\textless{}0.001). S100P expression did not associate with
subtype (P\textgreater{}0.05). There was no difference in
disease-specific or recurrence-free survival among the subtypes.
Periductal infiltration and American Joint Committee on Cancer eighth
edition pT stage predicted survival by multivariable analysis accounting
for gross configuration, pT stage, and histologic type. pT2 had worse
outcome relative to other pT stages. Significant differences in
histology and albumin expression distinguish LD from SD, but there is
insufficient evidence to support further subclassification of SD.

PubMed: \url{https://www.ncbi.nlm.nih.gov/pubmed/?term=30001234}

doi: \url{https://doi.org/10.1097/PAS.0000000000001118}

{}

{}

\begin{center}\rule{0.5\linewidth}{\linethickness}\end{center}

\begin{itemize}
\tightlist
\item
  \textbf{The prognostic impact of differentiation at the invasive front
  of biliary tract cancer}
\end{itemize}

\emph{Journal of surgical oncology 2018 May;117(6):1278-1287}

BACKGROUND: The invasive front of tumor can provide prognostic
information in many cancers. We investigated the prognostic
morphological factors at the invasive front including tumor
differentiation (Difinv ) and tumor budding (Bud) in biliary tract
cancer (BTC). METHODS: The resected specimen from the 299 BTC patients
were examined. Intrahepatic cholangiocarcinoma, extrahepatic
cholangiocarcinoma, gallbladder cancer, and ampulla of Vater cancer were
found in 16\%, 48\%, 17\%, and 19\%, respectively. Difinv grade (G) 3
and Bud foci ≥5 were found in 47\% and 10\%. Tumor with Difinv G3 showed
the high frequencies of Bud, vascular invasion (Ve) and nodal metastasis
(LN) compared to tumor with Difinv G1/2 (Bud: 21\% vs 0\%, Ve: 71\% vs
50\%, LN: 52\% vs 36\%). Multivariate analysis revealed that the
independent predictors were Difinv G3 (HR: 1.71), Bud foci ≥5 (HR:
2.14), Ve (HR: 1.56) and LN (HR: 2.59) in overall survival and were
positive resection margin (HR: 1.71), Difinv G3 (HR: 1.75), Ve (HR:
1.50), and LN (HR: 2.19) in relapse free survival. CONCLUSION: Poor
differentiation at the invasive front of tumor was associated with poor
prognosis and early relapse in BTC patients.

PubMed: \url{https://www.ncbi.nlm.nih.gov/pubmed/?term=29572828}

doi: \url{https://doi.org/10.1002/jso.24946}

{}

{}

\begin{center}\rule{0.5\linewidth}{\linethickness}\end{center}

\begin{itemize}
\tightlist
\item
  \textbf{Postradiation Synovial Sarcoma of the Common Bile Duct: A
  Previously Unreported Anatomic Site}
\end{itemize}

\emph{International journal of surgical pathology 2018
Aug;26(5):469-474}

Synovial sarcoma is a ubiquitous neoplasm predominantly affecting soft
tissues of young adults of any gender; few cases have been described in
the digestive system, mostly in the stomach. The (X;18)(p11.2; q11.2)
translocation yields unique SS18-SSX fusion genes. Synovial sarcoma has
been related to radiotherapy, but no synovial sarcoma has been
associated with the digestive system. This article describes the case of
a synovial sarcoma arising along the extrahepatic biliary tree, 10 years
after the application of an abdominal radiotherapy schedule due to a
retroperitoneal metastatic seminoma in a male who developed progressive
obstructive jaundice. Ninety percent of the analyzed cells carried the
SS18 gene with separation of sequences, thus denoting a translocation.
There are only 8 post-radiotherapy synovial sarcomas that have been
reported previously, and this is the first report of a
radiotherapy-related synovial sarcoma arising from the extrahepatic
biliary tree, and the second case described in this anatomic region.

PubMed: \url{https://www.ncbi.nlm.nih.gov/pubmed/?term=29336183}

doi: \url{https://doi.org/10.1177/1066896917752863}

{}

{}

\begin{center}\rule{0.5\linewidth}{\linethickness}\end{center}

\hypertarget{ampulla-of-vater}{%
\subsection{Ampulla of Vater}\label{ampulla-of-vater}}

\begin{center}\rule{0.5\linewidth}{\linethickness}\end{center}

\begin{itemize}
\tightlist
\item
  \textbf{Intraobserver and Interobserver Variability in the Assessment
  of Dysplasia in Ampullary Mucosal Biopsies}
\end{itemize}

\emph{The American journal of surgical pathology 2018
Aug;42(8):1095-1100}

Endoscopic mucosal biopsies of the ampulla of Vater (AmpBx) are obtained
to histologically assess for dysplasia or carcinoma. However, biopsy
material is often scant and a host of factors can induce histologic
changes that pose diagnostic challenges. We sought to investigate
observer variability in interpretation of AmpBx and the impact clinical
data may have on diagnostic interpretation. Thirty-one cases from
institutional archives were selected, including 12 cases of reactive
atypia (RA), 8 indefinite for dysplasia (ID), and 11 showing low-grade
dysplasia (LGD). Slides were independently reviewed at 3 time points
with and without clinical information by 6 pathologists who categorized
the biopsies RA, ID, or LGD. Following the reviews, intraobserver and
interobserver agreement was assessed. Review of AmpBx without clinical
data showed fair (κ, 0.27), poor (κ, 0.07), and good (κ, 0.42)
interobserver agreement for diagnoses of RA, ID, and LGD, respectively.
Interobserver agreement improved for LGD (κ, 0.66 and 0.73) when
clinical information was provided; however, agreement remained fair for
RA (κ, 0.4 and 0.42) and poor-to-fair for ID (κ, 0.17 and 0.25). When
follow-up data were reviewed, all cases that reached unanimous agreement
had that diagnosis substantiated by subsequent endoscopic or histologic
findings. The same was true of 13 of 19 cases that reached majority
consensus. Given the potential clinical consequences of these diagnoses
combined with the significant intraobserver and interobserver
variability found in this study, we conclude that better-defined
diagnostic criteria and consensus reads on difficult cases would assist
in the histologic assessment of these challenging cases.

PubMed: \url{https://www.ncbi.nlm.nih.gov/pubmed/?term=29738360}

doi: \url{https://doi.org/10.1097/PAS.0000000000001079}

{}

{}

\begin{center}\rule{0.5\linewidth}{\linethickness}\end{center}

\begin{itemize}
\tightlist
\item
  \textbf{Ampullary adenocarcinoma: Defining predictors of survival and
  the impact of adjuvant therapy following surgical resection for stage
  I disease}
\end{itemize}

\emph{Journal of surgical oncology 2018 Jun;117(7):1500-1508}

BACKGROUND AND OBJECTIVES: Outcomes and recommendations regarding
adjuvant therapy (AT) for stage I ampullary adenocarcinoma (AAC) are
inadequately described. We sought to determine factors associated with
survival and better define the impact of AT. METHODS: The NCDB was
queried for stage I AAC patients undergoing resection. We evaluated
variables influencing the administration of AT and affecting survival,
including the receipt of AT. RESULTS: Five hundred thirty-seven patients
were identified. 1, 3, and 5-year OS were 91.3\%, 78.8\%, and 67.4\%,
respectively. 103 received AT: 101 chemotherapy, 31 radiation, and 29 a
combination of both. AT was more commonly utilized in patients with
poorly differentiated and T2 tumors. Comorbid disease was inversely
associated with use of AT. Age ≥65 was associated with decreased
survival for stage IA and IB, while positive resection margins and
sampling of \textless{}12 LNs were associated with decreased OS for
stage IA and IB, respectively. After propensity matching key covariates,
no significant difference in OS was observed between those receiving and
not receiving AT (P = 0.449). CONCLUSION: This analysis revealed a
modest 5-year OS for stage I AAC. Age, positive resection margins, and
evaluation of \textless{}12 LNs negatively influenced OS and AT did not
convey a survival benefit.

PubMed: \url{https://www.ncbi.nlm.nih.gov/pubmed/?term=29518820}

doi: \url{https://doi.org/10.1002/jso.25021}

{}

{}

\begin{center}\rule{0.5\linewidth}{\linethickness}\end{center}

\begin{itemize}
\tightlist
\item
  \textbf{MiR-21 up-regulation in ampullary adenocarcinoma and its
  pre-invasive lesions}
\end{itemize}

\emph{Pathology, research and practice 2018 Jun;214(6):835-839}

Poor information is available on the molecular landscape characterizing
the carcinogenetic process leading to ampullary carcinoma. MiR-21 is one
of the most frequently up-regulated miRNAs in pancreatic adenocarcinoma,
a tumor sharing similar molecular features with ampullary
adenocarcinomas (AVCs), above all with the pancreatic-biliary type. We
profiled, by in situ hybridization (ISH), miR-21 expression in a series
of 26 AVCs, 50 ampullary dysplastic lesions (35 low-grade {[}LG-IEN{]}
and 15 high-grade {[}HG-IEN{]}) and 10 normal duodenal mucosa samples.
The same series was investigated by immunohistochemistry for β-catenin,
p53 and HER2 expression. HER2 gene amplification was evaluated by
chromogenic in situ hybridization. To validate miR-21 ISH results we
performed miR-21 qRT-PCR analysis in a series of 10 AVCs and their
matched normal samples. All the normal control samples showed a negative
or faint miR-21 expression, whereas a significant miR-21 up-regulation
was observed during the carcinogenetic cascade (p\,\textless{}\,0.001),
with 21/26 (80.8\%) of cancer samples showing a miR-21 overexpression.
In comparison to control samples, a significant overexpression was found
in samples of LG-IEN (p\,=\,.0003), HG-IEN (p\,=\,.0001), and AVCs
(p\,\textless{}\,0.0001). No significant difference in miR-21
overexpression was observed between LG-IEN, HG-IEN and AVCs. By qRT-PCR
analysis, AVCs showed a 1.7-fold increase over the controls
(p\,=\,.003). P53 was frequently dysregulated in both dysplastic and
carcinoma samples (44 out of 76; 57.9\%). A 20\% (10/50) of dysplastic
lesions and 11\% (3/26) of carcinomas were characterized by a nuclear
localization of β-catenin. Only 2 AVCs (7.7\%; both intestinal-type)
showed a HER2 overexpression (both 2+), which corresponded to a HER2
gene amplification at CISH analysis. This is the first study
demonstrating a miRNA dysregulation in the whole spectrum of ampullary
carcinogenesis. MiR-21 overexpression is an early molecular event during
ampullary carcinogenesis and its levels increase with the neoplastic
progression.

PubMed: \url{https://www.ncbi.nlm.nih.gov/pubmed/?term=29731265}

doi: \url{https://doi.org/10.1016/j.prp.2018.04.018}

{}

{}

\begin{center}\rule{0.5\linewidth}{\linethickness}\end{center}

\begin{itemize}
\tightlist
\item
  \textbf{Primary small bowel adenomas and adenocarcinomas-recent
  advances}
\end{itemize}

\emph{Virchows Archiv : an international journal of pathology 2018
Jul;():}

The small intestine represents 75\% of the length and 90\% of the
absorptive surface area of the gastrointestinal tract (GIT), yet only
2\% of digestive system cancers occur at this site. Adenocarcinoma
accounts for half of small bowel malignancies. There have been a number
of important recent advances in our understanding, classification and
treatment of small bowel tumours. Over recent years, ampullary tumours
have become recognised as a form of small bowel carcinoma, distinct from
head of pancreas and lower biliary tract tumours. This is reflected in
separate TNM systems and increasing interest in separating intestinal
from pancreatobiliary subtypes. The recognition of the importance of
microsatellite (MSI) status and the advent of molecular pathology has
also changed our approach to these neoplasms.

PubMed: \url{https://www.ncbi.nlm.nih.gov/pubmed/?term=29998424}

doi: \url{https://doi.org/10.1007/s00428-018-2400-7}

{}

{}

\begin{center}\rule{0.5\linewidth}{\linethickness}\end{center}

\begin{itemize}
\tightlist
\item
  \textbf{Prognostic factors and benefits of adjuvant therapy after
  pancreatoduodenectomy for ampullary adenocarcinoma: Mayo Clinic
  experience}
\end{itemize}

\emph{European journal of surgical oncology : the journal of the
European Society of Surgical Oncology and the British Association of
Surgical Oncology 2018 05;44(5):677-683}

INTRODUCTION: Ampullary adenocarcinoma is a rare entity with limited
data on prognostic factors. The aim of this study is to identify
prognostic factors and assess the benefit of adjuvant therapy in
patients with ampullary adenocarcinoma who underwent
pancreatoduodenectomy. METHODS: A cohort of 121 consecutive patients
underwent pancreatoduodenectomy for ampullary adenocarcinoma from 2006
to 2016~at Mayo Clinic in Rochester, MN. All patients were confirmed by
independent pathologic review to have ampullary carcinoma. Patient
survival and its correlation with patient and tumor variables were
evaluated by univariate and multivariate analysis. RESULTS: Fifty three
patients (45\%) received adjuvant therapy (34 patients had chemotherapy
alone, while 19 patients received both chemotherapy and radiation
therapy). Fifty seven percent of the patients were diagnosed with
advanced stage disease (Stage IIB or higher). Nearly all patients
(98.3\%) had negative surgical margins. Median overall survival (OS) was
91.8 months (95\% CI:52.6 months-not reached). In multivariate analysis,
excellent performance status (ECOG: 0), adjuvant therapy, and advanced
stage remained statistically significant. Adjuvant therapy was
independently associated with improved disease free survival (Hazard
ratio {[}HR{]}:0.52, P~=~0.04) and overall survival (HR:0.45, P~=~0.03)
in patients with advanced disease. CONCLUSIONS: Adjuvant therapy was
associated with improved survival in patients with resected ampullary
cancer, especially with advanced stage disease. A multi-institutional
randomized trial is needed to further assess the role of adjuvant
therapy in ampullary adenocarcinoma.

PubMed: \url{https://www.ncbi.nlm.nih.gov/pubmed/?term=29506768}

doi: \url{https://doi.org/10.1016/j.ejso.2018.02.008}

{}

{}

\begin{center}\rule{0.5\linewidth}{\linethickness}\end{center}

\begin{itemize}
\tightlist
\item
  \textbf{Retrospective cohort of pancreatic and Vater ampullary
  adenocarcinoma from a reference center in Mexico}
\end{itemize}

\emph{Annals of medicine and surgery (2012) 2018 Jun;30():7-12}

Introduction: Pancreatic ductal adenocarcinoma (PDAC) and ampulla of
Vater adenocarcinomas (AVAC) are periampullary tumors. These tumors have
overlapping symptoms and a common treatment, but present differences in
their survival and biology. No recent studies in Mexico have been
published that describe the clinicopathological characteristics of these
tumors. Therefore, the aim of this study was to describe the
clinicopathological characteristics of PDAC and AVAC in patients at a
reference center in Mexico. Methods: A retrospective cohort of patients
with PDAC or AVAC was analyzed at our institution (July 2007 to June
2016). Inferential analysis of the clinical data was performed with
Student's t-test or a χ2 test with odds ratios (OR) and confidence
intervals (CI), depending on the variables. Overall survival was
compared using Kaplan-Meier curves with log-rank p values. Results:
Forty patients with PDAC and 76 with AVAC were analyzed, including 77
females and 39 males with a mean age of 60.6 years and a mean evolution
time of 5.7 months. PDAC patients had more abdominal pain, a larger
tumor size and more advanced stages than AVAC patients. In contrast,
AVAC patients had more jaundice, a higher percentage of complete
resections and higher overall survival. Up to 70\% of patients were
overweight. PDAC cohort included a higher proportion of smokers.
Conclusions: Our cohort was slightly younger, had a larger percentage of
females, and a greater percentage of obese patients than those in many
international reports. A high proportion of PDAC patients are diagnosed
in advanced stages and have a low likelihood of resectability.

PubMed: \url{https://www.ncbi.nlm.nih.gov/pubmed/?term=29707208}

doi: \url{https://doi.org/10.1016/j.amsu.2018.04.002}

{}

{}

\begin{center}\rule{0.5\linewidth}{\linethickness}\end{center}

\begin{itemize}
\tightlist
\item
  \textbf{Myoepithelial Hamartoma in the Ampulla of Vater}
\end{itemize}

\emph{Clinical gastroenterology and hepatology : the official clinical
practice journal of the American Gastroenterological Association 2018
May;():}

PubMed: \url{https://www.ncbi.nlm.nih.gov/pubmed/?term=29857148}

doi: \url{https://doi.org/10.1016/j.cgh.2018.05.032}

{}

{}

\begin{center}\rule{0.5\linewidth}{\linethickness}\end{center}

\begin{itemize}
\tightlist
\item
  \textbf{Adenosquamous Carcinoma of the Ampulla of Vater: A Rare Cause
  of Obstructive Jaundice}
\end{itemize}

\emph{GE Portuguese journal of gastroenterology 2018 Jun;25(4):195-197}

PubMed: \url{https://www.ncbi.nlm.nih.gov/pubmed/?term=29998166}

doi: \url{https://doi.org/10.1159/000481457}

{}

{}

\begin{center}\rule{0.5\linewidth}{\linethickness}\end{center}

\begin{itemize}
\tightlist
\item
  \textbf{Long-term complete remission of a patient with high grade
  neuroendocrine carcinoma of ampulla of Vater}
\end{itemize}

\emph{BMJ case reports 2018 Jun;2018():}

We describe a case report of a 53-year-old man with a 5-months history
of progressive jaundice and upper abdominal pain. The patient was
further evaluated and finally diagnosed with a high-grade ampullary
neuroendocrine tumour (based on endoscopic-guided biopsy). Thereafter,
he underwent pancreatoduodenectomy and adjuvant platinum-based
chemotherapy. This extremely rare case presents his long-lasting
disease-free survival compared with similar cases; this case report
exemplifies a new, potentially efficient method for treating high-grade
papillary neuroendocrine tumour and may pave the way for further
clinical trials utilising this blueprint in the treatment of related
conditions.

PubMed: \url{https://www.ncbi.nlm.nih.gov/pubmed/?term=29950363}

doi: \url{https://doi.org/10.1136/bcr-2018-224845}

{}

{}

\begin{center}\rule{0.5\linewidth}{\linethickness}\end{center}

\begin{itemize}
\tightlist
\item
  \textbf{Validation of the eighth edition of the American Joint
  Committee on Cancer staging system for ampulla of Vater cancer}
\end{itemize}

\emph{Surgery 2018 May;163(5):1071-1079}

BACKGROUND: The American Joint Committee on Cancer recently proposed the
eighth edition of cancer staging system. Validation studies are required
to evaluate the prognostic stratification of ampulla of Vater cancer
patients. METHODS: In the study, 369 operatively resected patients with
ampullary cancers were grouped based on the eighth T (T1a, limited to
sphincter of Oddi; T1b, invasion to duodenal submucosa; T2, invasion to
duodenal proper muscle; T3a, invasion to pancreas ≤0.5 cm; T3b, invasion
to pancreas \textgreater{}0.5 cm; and T4, involvement of celiac axis or
superior mesenteric artery) and N (N0, no nodal metastasis; N1, 1-3
nodal metastasis; and N2, ≥4 nodal metastasis) category of ampullary
cancer staging. RESULTS: Overall 5-year survival rates for T and N
categories were as followed: T1a, 83\%; T1b, 71\%; T2, 46\%; T3a, 48\%;
T3b, 28.5\%, T4, 7\% (P\textless{} .001); N0, 44.8\%; N1, 20\%; N2, 4\%
(P \textless{} .001). Pair-wise comparisons demonstrated significant
differences between T1a-b (P = .005), T3a-T3b (P = .03), N0-N1 (P
\textless{} .001), and N1-N2 (P = .007) tumors, but not between T1b-T2
(P = .20), T2-T3a (P = .84), and T3b-T4 (P = .17) lesions. CONCLUSION:
The eighth edition T category for ampullary cancer does not stratify
patients accurately with regard to prognosis. Modification of the
current T category with eliminating subcategories (T1, invasion to
duodenal submucosa; T2, invasion to duodenal proper muscle; T3, invasion
to pancreas or duodenal subserosa) is a better way for determining
prognosis of ampullary cancer. The current N category segregates patient
survival well.

PubMed: \url{https://www.ncbi.nlm.nih.gov/pubmed/?term=29452703}

doi: \url{https://doi.org/10.1016/j.surg.2017.12.018}

{}

{}

\begin{center}\rule{0.5\linewidth}{\linethickness}\end{center}

\begin{itemize}
\tightlist
\item
  \textbf{Immunohistochemical Predictors for Intestinal and
  Pancreatobiliary Types of Adenocarcinoma of The Ampulla of Vater}
\end{itemize}

\emph{Journal of gastrointestinal surgery : official journal of the
Society for Surgery of the Alimentary Tract 2018 May;():}

OBJECTIVES: To investigate immunohistochemical predictors for intestinal
and pancreatobiliary types of adenocarcinoma of ampulla of Vater and
identify clinicopathological characteristics associated with the
histological types and patient survival. METHODS: Immunohistochemical
markers included MUC1, MUC2, MUC5AC, CDX2, CK7, and CK20. The data were
analyzed by univariate and multivariate methods. The two-step cluster
method was used to determine the best immunohistochemical markers to
discriminate the intestinal from the pancreatobiliary type. RESULTS:
This study identified 9 (33.3\%) intestinal and 21 (66.7\%)
pancreatobiliary tumors. CK7 and CDX2 achieved the highest value (= 1)
as predictor markers, while CK20, MUC1, and MUC2 showed degrees of
importance equal to 0.77, 0.71, and 0.68, respectively. MUC5AC did not
reach 0.50 of importance. In the univariate analysis, lymph node
involvement, staging (TNM), and angiolymphatic and perineural invasions
were associated with histological types. The independent
clinicopathological variable in the multivariate model to predict the
histological type was angiolymphatic invasion (p = 0.005), OR = 17 (95\%
CI 2.33 to 123.83). The final model showed positive nodes (N1)
associated with shorter survival (HR = 9.5; p = 0.006). Overall survival
at 12, 36, and 60~months was 88.5, 67.0, and 47.6\%, respectively.
CONCLUSIONS: CDX2 and CK7 were the immunohistochemical markers that best
discriminated the intestinal from the pancreatobiliary type. Lymph node
involvement had a high impact on survival and proved to be more frequent
in the pancreatobiliary type.

PubMed: \url{https://www.ncbi.nlm.nih.gov/pubmed/?term=29736668}

doi: \url{https://doi.org/10.1007/s11605-018-3797-7}

{}

{}

\begin{center}\rule{0.5\linewidth}{\linethickness}\end{center}

\begin{itemize}
\tightlist
\item
  \textbf{A Case of Gangliocytic Paraganglioma with Carcinoma of the
  Ampulla of Vater: A Case Report}
\end{itemize}

\emph{Internal medicine (Tokyo, Japan) 2018 May;():}

The patient was a ``73'' -year-old woman who visited our hospital with
the chief complaint of weight loss. Upper gastrointestinal endoscopy
revealed an enlarged ampulla of Vater, and a biopsy led to a diagnosis
of Group ``4'' gastric carcinoma; suspicious of adenocarcinoma. There
were no findings suggesting invasion into the muscle layer of duodenum,
despite tumor mass formation being observed in the sphincter of Oddi. We
performed endoscopic papillectomy for both diagnostic and therapeutic
purposes. Pathologically, a well-differentiated adenocarcinoma existed
in the superficial layer of the mucous membrane of the papilla of Vater,
and gangliocytic paraganglioma was present in the deep portion. The
resected margins of both lesions were negative.

PubMed: \url{https://www.ncbi.nlm.nih.gov/pubmed/?term=29780117}

doi: \url{https://doi.org/10.2169/internalmedicine.0464-17}

{}

{}

\begin{center}\rule{0.5\linewidth}{\linethickness}\end{center}

\begin{itemize}
\tightlist
\item
  \textbf{Signet Ring Cell Carcinoma of the Ampulla of Vater With Focal
  Neuroendocrine Differentiation of the Amphicrine Type: Report of a
  Case With Long-Term Survival}
\end{itemize}

\emph{International journal of surgical pathology 2018
Jul;():1066896918784666}

Carcinoma of the ampulla of Vater is an uncommon neoplasm and represents
0.5\% of all gastrointestinal malignancies, being less common than
carcinoma of the pancreas and bile ducts. The most common ampullary
tumor is the adenocarcinoma with tubular growth pattern. Signet ring
cell carcinoma is extremely rare. In this article, we report a case of
signet ring cell carcinoma of the ampulla of Vater showing focal
neuroendocrine amphicrine differentiation and intestinal phenotype,
which occurred in a 49-year-old male who is still alive 7 years after
surgery, without evidence of recurrence. This long-term survival might
be attributed not only to the early stage of the disease but also to the
neuroendocrine differentiation and the absence of genetic alterations.

PubMed: \url{https://www.ncbi.nlm.nih.gov/pubmed/?term=29996669}

doi: \url{https://doi.org/10.1177/1066896918784666}

{}

{}

\begin{center}\rule{0.5\linewidth}{\linethickness}\end{center}

\begin{itemize}
\tightlist
\item
  \textbf{Sarcomatoid Adenocarcinoma of the Ampulla of Vater}
\end{itemize}

\emph{Indian journal of surgical oncology 2018 Jun;9(2):274-277}

Sarcomatoid adenocarcinoma of ampulla of Vater is an extremely rare
malignant neoplasm that displays both carcinomatous and sarcomatous
component. A 58-year-old woman was admitted to our hospital under the
suspicion of an ampulla of Vater cancer. Abdominal computed tomography
and endoscopy demonstrated a bulging of ampulla and the biopsy specimen
revealed an adenocarcinoma, well differentiated in the background of
tubulovillous adenoma. So we performed the pylorus preserving
pancreaticoduodenectomy. At postoperative biopsy, the tumor was composed
of adenocarcinoma component and sarcomatoid component. Thus, a diagnosis
of sarcomatoid adenocarcinoma of ampulla of Vater could be made. Here,
we present a case of sarcomatoid adenocarcinoma of ampulla of Vater.

PubMed: \url{https://www.ncbi.nlm.nih.gov/pubmed/?term=29887717}

doi: \url{https://doi.org/10.1007/s13193-018-0743-9}

{}

{}

\begin{center}\rule{0.5\linewidth}{\linethickness}\end{center}

\begin{itemize}
\tightlist
\item
  \textbf{Primary small bowel adenomas and adenocarcinomas-recent
  advances}
\end{itemize}

\emph{Virchows Archiv : an international journal of pathology 2018
Jul;():}

The small intestine represents 75\% of the length and 90\% of the
absorptive surface area of the gastrointestinal tract (GIT), yet only
2\% of digestive system cancers occur at this site. Adenocarcinoma
accounts for half of small bowel malignancies. There have been a number
of important recent advances in our understanding, classification and
treatment of small bowel tumours. Over recent years, ampullary tumours
have become recognised as a form of small bowel carcinoma, distinct from
head of pancreas and lower biliary tract tumours. This is reflected in
separate TNM systems and increasing interest in separating intestinal
from pancreatobiliary subtypes. The recognition of the importance of
microsatellite (MSI) status and the advent of molecular pathology has
also changed our approach to these neoplasms.

PubMed: \url{https://www.ncbi.nlm.nih.gov/pubmed/?term=29998424}

doi: \url{https://doi.org/10.1007/s00428-018-2400-7}

{}

{}

\begin{center}\rule{0.5\linewidth}{\linethickness}\end{center}

\begin{itemize}
\tightlist
\item
  \textbf{Signet Ring Cell Carcinoma of the Ampulla of Vater With Focal
  Neuroendocrine Differentiation of the Amphicrine Type: Report of a
  Case With Long-Term Survival}
\end{itemize}

\emph{International journal of surgical pathology 2018
Jul;():1066896918784666}

Carcinoma of the ampulla of Vater is an uncommon neoplasm and represents
0.5\% of all gastrointestinal malignancies, being less common than
carcinoma of the pancreas and bile ducts. The most common ampullary
tumor is the adenocarcinoma with tubular growth pattern. Signet ring
cell carcinoma is extremely rare. In this article, we report a case of
signet ring cell carcinoma of the ampulla of Vater showing focal
neuroendocrine amphicrine differentiation and intestinal phenotype,
which occurred in a 49-year-old male who is still alive 7 years after
surgery, without evidence of recurrence. This long-term survival might
be attributed not only to the early stage of the disease but also to the
neuroendocrine differentiation and the absence of genetic alterations.

PubMed: \url{https://www.ncbi.nlm.nih.gov/pubmed/?term=29996669}

doi: \url{https://doi.org/10.1177/1066896918784666}

{}

{}

\begin{center}\rule{0.5\linewidth}{\linethickness}\end{center}

\begin{itemize}
\tightlist
\item
  \textbf{Prognostic factors of non-ampullary duodenal adenocarcinoma}
\end{itemize}

\emph{Japanese journal of clinical oncology 2018 Jun;():}

Background: Non-ampullary duodenal adenocarcinoma, excluding carcinoma
in the ampulla of Vater, is a rare disease. Although several prognostic
factors have been reported, they remain controversial due to the rarity
of non-ampullary duodenal adenocarcinoma. The aims of this study were to
investigate prognostic factors in patients with non-ampullary duodenal
adenocarcinoma and to assess chemotherapy in patients with recurrence.
Patients and methods: Records of 25 patients who underwent surgical
treatment for non-ampullary duodenal adenocarcinoma from 2004 to 2016
were retrospectively reviewed. The relationship between the
clinicopathological factors and outcomes was investigated. Results:
Serum level of CA19-9, gross appearance, tumor size, tumor invasion,
lymph node metastases, TNM stage and lymphatic and vascular invasion
were significant risk factors of recurrence. Patients with recurrence
who received chemotherapy according to regimens used to treat colorectal
cancer had a better prognosis than those without chemotherapy (P =
0.016). Conclusion: Advanced non-ampullary duodenal adenocarcinoma has a
poor prognosis, but chemotherapy possibly improves the prognosis in the
patients with recurrent non-ampullary duodenal adenocarcinoma.

PubMed: \url{https://www.ncbi.nlm.nih.gov/pubmed/?term=29931295}

doi: \url{https://doi.org/10.1093/jjco/hyy086}

{}

{}

\begin{center}\rule{0.5\linewidth}{\linethickness}\end{center}

\begin{itemize}
\tightlist
\item
  \textbf{Heterotopic Gastric Mucosa in the Common Bile Duct With
  Cholangiocarcinoma}
\end{itemize}

\emph{International journal of surgical pathology 2018
May;():1066896918775531}

Gastric heterotopia within the biliary system is extremely rare.
Moreover, the combination of gastric heterotopia in the bile duct with
cholangiocarcinoma has not been reported. We describe a case of
heterotopic gastric mucosa in the common bile duct with
cholangiocarcinoma. An 80-year-old male was admitted with abdominal
pain. Abdominal computed tomography revealed wall thickening from the
hilar duct to the distal common bile duct. Biopsy from the distal bile
duct showed only benign gastric foveolar-type epithelium and fundic
glands. Although the diagnosis of the biopsy was benign, malignancy was
strongly suspected from the radiologic findings, and excision of the
bile ducts was performed. Microscopically, the resected specimen showed
poorly formed malignant glands and gastric heterotopia also identified
in the common bile duct. Three months later, the patient's state
worsened due to recurrence, and he died. To our knowledge, this is the
first report of gastric heterotopia in the bile duct accompanying
cholangiocarcinoma.

PubMed: \url{https://www.ncbi.nlm.nih.gov/pubmed/?term=29742955}

doi: \url{https://doi.org/10.1177/1066896918775531}

{}

{}

\begin{center}\rule{0.5\linewidth}{\linethickness}\end{center}

\hypertarget{neuroendocrine}{%
\subsection{Neuroendocrine}\label{neuroendocrine}}

\begin{center}\rule{0.5\linewidth}{\linethickness}\end{center}

\begin{itemize}
\tightlist
\item
  \textbf{Fine-needle aspiration of a pancreatic neuroendocrine tumor
  with prominent rhabdoid features}
\end{itemize}

\emph{Diagnostic cytopathology 2018 Jul;46(7):600-603}

Pancreatic neuroendocrine tumors (PanNETs) are uncommon neoplasms that
conventionally possess architectural and cytomorphological features seen
in neuroendocrine neoplasms found at other sites. When present, these
features often allow rapid identification of neuroendocrine
differentiation and an accurate diagnosis. Here, we report the cytologic
findings seen on fine-needle aspiration (FNA) of a PanNET with distinct
rhabdoid features. This morphology is rare in PanNETs and has been
reported in only two case series examining surgical resection specimens
and has not been described on FNA. It is important to recognize this
morphology as this variant appears to portend an aggressive clinical
course. Furthermore, unfamiliarity with this morphologic variant may
lead to a larger initial differential and thus delay final diagnosis.

PubMed: \url{https://www.ncbi.nlm.nih.gov/pubmed/?term=29359517}

doi: \url{https://doi.org/10.1002/dc.23892}

{}

{}

\begin{center}\rule{0.5\linewidth}{\linethickness}\end{center}

\begin{itemize}
\tightlist
\item
  \textbf{Well differentiated grade 3 pancreatic neuroendocrine tumors
  compared with related neoplasms: A morphologic study}
\end{itemize}

\emph{Cancer cytopathology 2018 May;126(5):326-335}

BACKGROUND: Pancreatic neuroendocrine neoplasms with a Ki-67 labeling
index greater than 20\% were reclassified in 2017 by the World Health
Organization into well differentiated (WD) and poorly differentiated
grade 3 neuroendocrine carcinoma (NEC). The authors describe the
cytologic features of grade 3 WD pancreatic neuroendocrine neoplasms
compared with grade 2 neoplasms and NEC. METHODS: Fine-needle aspirates
from 65 pancreatic neuroendocrine neoplasms were reviewed, and their
cytomorphologic features were compared across grade 2, WD grade 3, and
PD small cell type (PD-S), large cell type (PD-L), and type not
otherwise specified (PD-NOS) neoplasms. RESULTS: The 65 aspirates
consisted of 19 grade 2 neoplasms, 32 WD grade 3 neoplasms, and 14 NECs
(6 PD-S, 5 PD-L, and 3 PD-NOS). The medians Ki-67 proliferation index
was 11\% (range, 3.2\%-17\%) in grade 2 neoplasms, 40\% (range,
21\%-89\%) in WD grade 3 neoplasms, 80\% (range, 63\%-95\%) in PD-S
neoplasms, 39\% (range, 25\%-61\%) in PD-L neoplasms, and 70\% (range,
30\%-80\%) in PD-NOS neoplasms. Both grade 2 and WD grade 3 neoplasms
were associated with plasmacytoid morphology and smooth nuclear
contours, but WD grade 3 neoplasms had significant increases in abundant
cytoplasm (72\% vs 17\%; P = .007), nuclear tangles (75\% vs 42\%; P =
.006), and apoptosis (86\% vs 58\%; P = .005). Compared with NECs, WD
grade 3 neoplasms had increased plasmacytoid morphology (75\% vs 7\%; P
\textless{} .001), smooth nuclear contours (94\% vs 64\%; P = .02),
round nuclei (59\% vs 21\%; P = .01), and less pleomorphism (13\% vs
50\%; P = .004), molding (9\% vs 79\%; P \textless{} .001), and necrosis
(13\% vs 43\%; P = .003). WD grade 3 neoplasms had less pleomorphism
(13\% vs 50\%; P = .04), less necrosis (13\% vs 60\%; P = .04), and more
plasmacytoid morphology (75\% vs 20\%; P = .03) than PD-L. CONCLUSIONS:
The prevalence of cytologic features differs in WD grade 3 pancreatic
neuroendocrine neoplasms compared with grade 2 neoplasms and NECs, and
these differences assist in the recognition of this newly classified
entity. Cancer Cytopathol 2018;126:326-35. © 2018 American Cancer
Society.

PubMed: \url{https://www.ncbi.nlm.nih.gov/pubmed/?term=29451738}

doi: \url{https://doi.org/10.1002/cncy.21982}

{}

{}

\begin{center}\rule{0.5\linewidth}{\linethickness}\end{center}

\begin{itemize}
\tightlist
\item
  \textbf{Pancreatic neuroendocrine carcinomas reveal a closer
  relationship to ductal adenocarcinomas than to neuroendocrine tumors
  G3}
\end{itemize}

\emph{Human pathology 2018 Jul;77():70-79}

Pancreatic neuroendocrine carcinoma is a rare aggressive tumor commonly
harboring TP53 and RB1 alterations and lacking neuroendocrine-related
genetic changes such as mutations in MEN1 and ATRX/DAXX. Little is known
about its genetic profile with regard to that of pancreatic ductal
adenocarcinoma. We therefore conducted a detailed genetic study in 12
pancreatic neuroendocrine carcinomas of large cell (n = 9) and small
cell type (n = 3) using massive parallel sequencing applying a 409-gene
panel on an Ion Torrent system. The genetic data were compared with
known data of pancreatic ductal adenocarcinoma and correlated with
exocrine lineage marker expression. A similar analysis was performed in
11 pancreatic neuroendocrine tumors G3. Neuroendocrine carcinomas
harbored 63 somatic mutations in 45 different genes, affecting most
commonly TP53 (8/12 cases), KRAS (5/12 cases), and RB1 (loss of
expression with or without deletion in 4/12 cases). Five carcinomas had
both TP53 and KRAS mutations. Neuroendocrine tumors G3 only shared
singular mutations in 5 different genes with neuroendocrine carcinomas,
including TP53, CDKN2A, ARID1A, LRP1B, and APC, affecting 5 different
cases. Most KRAS-positive neuroendocrine carcinomas also expressed MUC1
(4/5) and carcinoembryonic antigen (3/5) as markers of ductal
differentiation. Our data indicate that almost half of the pancreatic
neuroendocrine carcinomas are genetically and phenotypically related to
pancreatic ductal adenocarcinoma, and might therefore respond to
chemotherapies targeting the latter carcinomas.

PubMed: \url{https://www.ncbi.nlm.nih.gov/pubmed/?term=29596894}

doi: \url{https://doi.org/10.1016/j.humpath.2018.03.018}

{}

{}

\begin{center}\rule{0.5\linewidth}{\linethickness}\end{center}

\begin{itemize}
\tightlist
\item
  \textbf{Carbonic anhydrase 9 expression in well-differentiated
  pancreatic neuroendocrine neoplasms might be associated with
  aggressive behavior and poor survival}
\end{itemize}

\emph{Virchows Archiv : an international journal of pathology 2018
May;472(5):739-748}

Well-differentiated pancreatic neuroendocrine neoplasms/tumors (PanNETs)
are rare neoplasms with diverse clinical behavior. Biomarker discovery
is important for predicting clinical course and prognosis of PanNET
patients. Carbonic anhydrase 9 (CA9) and vimentin are hypoxia and
epithelial-mesenchymal transition-related proteins of which expression
in many carcinomas has been associated with poor prognosis, but their
significance in PanNET has yet to be determined. We assessed CA9 and
vimentin expression in 164 PanNETs and compared this with
clinicopathologic characteristics. CA9 expression was observed in normal
islets, while neuroendocrine microadenomas and small (\textless{} 1~cm)
PanNETs showed loss of CA9 expression. CA9 and vimentin expression was
observed in 38 (23\%) and 36 (22\%) of PanNETs, respectively. CA9
expression was associated with larger size (p = 0.001), higher grade
(p \textless{} 0.001), higher pT category (p \textless{} 0.001), lymph
node (p = 0.003) and distant (p = 0.047) metastases, higher AJCC stage
(p \textless{} 0.001), and lymphovascular (p \textless{} 0.001) and
perineural (p = 0.002) invasion. PanNET patients with CA9 expression had
a shorter recurrence-free survival (5-year survival rate 47\%) than
those without CA9 expression (76\%) by univariate (p = 0.001) but not
multivariate analysis. Vimentin expression correlated with CA9
expression (p \textless{} 0.001) but not with other clinicopathologic
factors. In conclusion, CA9 expression was observed in normal islets,
while neuroendocrine microadenomas and small (\textless{} 1~cm) PanNETs
showed CA9 expression loss. CA9 expression gradually reappeared in
larger PanNETs, and this was associated with clinical progression and
decreased patient survival by univariate but not multivariate analysis.

PubMed: \url{https://www.ncbi.nlm.nih.gov/pubmed/?term=29666945}

doi: \url{https://doi.org/10.1007/s00428-018-2353-x}

{}

{}

\begin{center}\rule{0.5\linewidth}{\linethickness}\end{center}

\begin{itemize}
\tightlist
\item
  \textbf{Pancreatic islets communicate with lymphoid tissues via
  exocytosis of insulin peptides}
\end{itemize}

\url{https://www.nature.com/articles/s41586-018-0341-6}

\begin{center}\rule{0.5\linewidth}{\linethickness}\end{center}

\begin{itemize}
\tightlist
\item
  \textbf{Hepatic micrometastases are associated with poor prognosis in
  patients with liver metastases from neuroendocrine tumors of the
  digestive tract}
\end{itemize}

\emph{Human pathology 2018 May;():}

Pathologic examination of hepatic metastasectomies from patients with
metastatic small intestinal or pancreatic neuroendocrine tumor
frequently reveals micrometastases undetectable by radiologic or
macroscopic gross examination. This finding raises the possibility that
undetectable micrometastases remain in these patients after
metastasectomy. Here we examined liver resections for micrometastases
and assessed their impact on prognosis. Hepatic metastasectomies from 65
patients with neuroendocrine tumor of the small intestine (N=43) or
pancreas (N=22) were reviewed for the presence of micrometastases, which
were defined as microscopic tumor foci ≤1mm in greatest dimension.
Medical records were also reviewed for patient demographics, clinical
history, and follow-up data. Micrometastasis was identified in 36 (55\%)
of 65 hepatic resection specimens. More hepatic micrometastases were
seen in small intestinal cases than in pancreatic cases (29/43, 67\%
versus 7/22, 32\%; P\textless{}.01). They were typically present within
portal tracts, sometimes with extension into the periportal region or
sinusoidal spaces away from the portal tracts. Patients without hepatic
micrometastases had fewer macrometastases or more R0 hepatic resections
than those with micrometastases. The presence of hepatic micrometastases
was associated with poor overall survival both before (hazard ratio
{[}HR{]} 3.43; 95\% CI 1.1410.30; P=.03) and after accounting for
confounding variables in stratified Cox regression (HR 4.82; 95\% CI
1.0621.79; P=.04). In conclusion, hepatic micrometastases are common in
patients with metastatic small intestinal or pancreatic neuroendocrine
tumor and are independently associated with poor prognosis. These data
suggest that surgical resection of hepatic metastases is likely not
curative in these patients.

PubMed: \url{https://www.ncbi.nlm.nih.gov/pubmed/?term=29763717}

doi: \url{https://doi.org/10.1016/j.humpath.2018.05.006}

{}

{}

\begin{center}\rule{0.5\linewidth}{\linethickness}\end{center}

\begin{itemize}
\tightlist
\item
  \textbf{Neoplasms of the Neuroendocrine Pancreas: An Update in the
  Classification, Definition, and Molecular Genetic Advances}
\end{itemize}

\emph{Advances in anatomic pathology 2018 Jun;():}

This review focuses on discussing the main modifications of the recently
published 2017 WHO Classification of Neoplasms of the Neuroendocrine
Pancreas (panNEN). Recent updates separate pancreatic neuroendocrine
tumors into 2 broad categories: well-differentiated pancreatic
neuroendocrine tumors (panNET) and poorly differentiated pancreatic
neuroendocrine carcinoma (panNEC), and incorporates a new subcategory of
``well-differentiated high-grade NET (G3)'' to the well-differentiated
NET category. This new classification algorithm aims to improve the
prediction of clinical outcomes and survival and help clinicians select
better therapeutic strategies for patient care and management. In
addition, these neuroendocrine neoplasms are capable of producing large
quantity of hormones leading to clinical hormone hypersecretion
syndromes. These functioning tumors include, insulinomas, glucagonomas,
somatostatinomas, gastrinomas, VIPomas, serotonin-producing tumors, and
ACTH-producing tumors. Although most panNENs arise as sporadic diseases,
a subset of these heterogeneous tumors present as parts on inherited
genetic syndromes, such as multiple endocrine neoplasia type 1, von
Hippel-Lindau, neurofibromatosis type 1, tuberous sclerosis, and
glucagon cell hyperplasia and neoplasia syndromes. Characteristic
clinical and morphologic findings for certain functioning and syndromic
panNENs should alert both pathologists and clinicians as appropriate
patient management and possible genetic counseling may be necessary.

PubMed: \url{https://www.ncbi.nlm.nih.gov/pubmed/?term=29912000}

doi: \url{https://doi.org/10.1097/PAP.0000000000000201}

{}

{}

\begin{center}\rule{0.5\linewidth}{\linethickness}\end{center}

\begin{itemize}
\tightlist
\item
  \textbf{Neuroendocrine tumor of the pancreas with rhabdoid feature}
\end{itemize}

\emph{Virchows Archiv : an international journal of pathology 2018
Jun;():}

Imaging of a 53-year-old Japanese man revealed two tumors in the liver
and a tumor in the head of the pancreas with a swelling lymph node. A
needle biopsy for the liver tumors was performed, revealing a
neuroendocrine tumor. Enucleation, lymphadenectomy, and partial
hepatectomy were performed. The microscopic examination identified many
tumor cells with intracytoplasmic inclusions arranged in a nested, cord,
or tubular fashion. The intracytoplasmic inclusions displayed densely
eosinophilic globules and displaced the nuclei toward the periphery,
which constitutes ``rhabdoid'' features. The tumor cells were positive
for synaptophysin and weakly positive for NCAM, but negative for
chromogranin A. Epithelial markers (AE1/AE3 and CAM5.2) accentuated
intracytoplasmic globules. Pancreatic neuroendocrine tumors with
rhabdoid features are very rare. Generally, rhabdoid features are
aggressive and dedifferentiated characteristics of various types of
tumor. Pancreatic neuroendocrine tumors containing rhabdoid cells tend
to display extrapancreatic spread at the time of presentation, although
some of these tumors with rhabdoid features are not always associated
with aggressive behavior.

PubMed: \url{https://www.ncbi.nlm.nih.gov/pubmed/?term=29938394}

doi: \url{https://doi.org/10.1007/s00428-018-2398-x}

{}

{}

\begin{center}\rule{0.5\linewidth}{\linethickness}\end{center}

\begin{itemize}
\tightlist
\item
  \textbf{Incidentally detected pancreatic neuroendocrine microadenoma
  with lymph node metastasis}
\end{itemize}

\url{https://link.springer.com/article/10.1007/s00428-018-2407-0}

\emph{Virchows Archiv : an international journal of pathology 2018
Jul;():}

Pancreatic neuroendocrine microadenomas (NEMAs) are non-functioning
neuroendocrine tumors \textless{} 0.5~cm with a low proliferation rate
and are considered benign. We report on a pancreatic NEMA with lymph
node metastasis. A male in his 70s had pylorus-preserving
pancreaticoduodenectomy for a distal bile duct carcinoma, which was a
2.1 cm well-differentiated-infiltrating adenocarcinoma with invasion
limited to the bile duct wall. An incidental separate 0.4 cm
well-differentiated NEMA was found in the pancreatic head with
metastatic well-differentiated neuroendocrine tumor in one
peripancreatic lymph node. Both neuroendocrine tumors in the pancreatic
head and in the lymph node were composed of nests of uniform neoplastic
cells with a fine chromatin pattern. The Ki-67 labeling index of NEMA
was 0.85\%. The neoplastic neuroendocrine cells in both the pancreas and
node were diffusely positive for synaptophysin, chromogranin, and
insulin. Therefore, this unusual case provides an exception to the
current classification system which regards NEMAs as benign lesions.

PubMed: \url{https://www.ncbi.nlm.nih.gov/pubmed/?term=30003325}

doi: \url{https://doi.org/10.1007/s00428-018-2407-0}

{}

{}

\begin{center}\rule{0.5\linewidth}{\linethickness}\end{center}

\begin{itemize}
\tightlist
\item
  \textbf{Reduced RNA-binding protein HuD in pancreatic neuroendocrine
  tumors lowers p27Kip1 levels linked to poor prognosis}
\end{itemize}

\emph{The Journal of pathology 2018 Jul;():}

For the majority of patients diagnosed with pancreatic neuroendocrine
tumors (pancreatic NETs) there is a significant malignant potential with
a poor prognosis, however the molecular abnormalities and pathogenesis
of pancreatic NETs have not been firmly established. Here, we report
that loss of RNA-binding protein HuD expression correlates with low
p27Kip1 (p27) levels and poor prognosis in pancreatic NETs. HuD
expression was frequently lost in many human pancreatic NETs and these
pancreatic NETs showed aggressive clinico-pathological phenotypes with
low p27 levels, increased tumor size, higher WHO grade and pathological
T stage of the tumor, and presence of angioinvasion. Furthermore, loss
of HuD was an independent, progress-free prognostic factor in
multivariate survival analysis. However, level of HuR, the same Hu
protein family member with HuD, was not significantly correlated with
pancreatic NET size and progression. Mechanistically, HuD enhanced p27
mRNA translation by interacting with both 5'- and 3'-untranslated
regions (UTRs) of p27 mRNA and consequently suppressed cell cycle
progression and tumor growth. In addition, HuD competed with miR-30a-3p
for binding to 3'UTR of p27 mRNA, suggesting interplay between HuD and
miR-30a-3p in controlling p27 translation. Our results identify HuD as a
pivotal suppressor of pancreatic NET growth, and propose that HuD has
potential value as a prognostic factor of pancreatic NETs. This article
is protected by copyright. All rights reserved.

PubMed: \url{https://www.ncbi.nlm.nih.gov/pubmed/?term=30014466}

doi: \url{https://doi.org/10.1002/path.5135}

{}

{}

\begin{center}\rule{0.5\linewidth}{\linethickness}\end{center}

\hypertarget{cytopathology}{%
\subsection{Cytopathology}\label{cytopathology}}

\begin{center}\rule{0.5\linewidth}{\linethickness}\end{center}

\begin{itemize}
\tightlist
\item
  \textbf{Factors Impacting the Performance Characteristics of Bile Duct
  Brushings: A Clinico-Cytopathologic Analysis of 253 Patients}
\end{itemize}

\emph{Archives of pathology \& laboratory medicine 2018
Jul;142(7):863-870}

CONTEXT: - Literature on factors impacting bile duct brushings (BDBs)
performance characteristics remain limited. OBJECTIVE: - To capture the
current state of daily practice with BDB sign-out. DESIGN: - Two hundred
fifty-three of 444 BDBs signed out by more than 7 cytopathologists, with
histopathologic and/or clinical follow-up of at least 18 months, were
examined. RESULTS: - One hundred thirty-five of 253 BDBs (53\%) had
histologically confirmed malignancies, 22 (9\%) had cancer-related
deaths, and 96 (38\%) were benign. Cytologic diagnoses in the 444 BDBs
were nondiagnostic (11 {[}2.5\%{]}), negative (284 {[}64\%{]}), atypical
(62 {[}13.9\%{]}), suspicious (34 {[}7.7\%{]}), and malignant (53
{[}11.9\%{]}). Sensitivity, specificity, positive predictive value
(PPV), negative predictive value (NPV), and accuracy of malignancy
detection were 35\%, 100\%, 100\%, 58\%, and 66\%, respectively. When
atypical, suspicious, and malignant (ASM) categories were combined,
sensitivity increased (58\%), specificity and PPV dropped (97\%), and
accuracy increased (73\%). Carcinoma type (bile-duct versus
pancreatic-ductal) had no effect on accuracy ( P = .60) or diagnostic
class ( P = .84), nor did time of performance (first 7.5 versus latter
7.5 years, P = .13). Interestingly, ThinPrep + cell block (n = 41) had
higher sensitivity (61\%) and lower specificity (80\%) than ThinPrep
only (versus 51\% and 100\%, respectively). Sensitivity and specificity
were higher (47\% and 100\%) in nonstented than stented specimens (59\%
and 97\%). Relative risk of malignancy for ``suspicious'' (2.30) and
``atypical'' (2.28) categories was lower but not very different from
that of ``malignant'' category (2.41). CONCLUSIONS: - Bile duct
brushings had fairly low sensitivity but high specificity and PPV with
no false positives. Sensitivity almost doubled and specificity dipped
minimally when ASM categories were combined, highlighting the need for
better classification criteria for atypical/suspicious cases. Higher
specificity, PPV, NPV, and accuracy but lower sensitivity in stented
BDBs suggest that they be called malignant only when evidence is
overwhelmingly convincing.

PubMed: \url{https://www.ncbi.nlm.nih.gov/pubmed/?term=29582676}

doi: \url{https://doi.org/10.5858/arpa.2017-0150-OA}

{}

{}

\begin{center}\rule{0.5\linewidth}{\linethickness}\end{center}

\begin{itemize}
\tightlist
\item
  \textbf{UroVysion Multi-Target Fluorescence in situ Hybridization
  Assay for the Detection of Malignant Bile Duct Brushing Specimens: A
  Comparison with Routine Cytology}
\end{itemize}

\emph{Acta cytologica 2018 May;():1-7}

OBJECTIVE: Routine bile duct brushing cytology is an important
diagnostic tool in the evaluation of bile duct stricture. The purpose of
this study was to evaluate the performance of the UroVysion fluorescence
in situ hybridization (FISH) assay for the detection of malignant bile
duct brushing specimens. STUDY DESIGN: Thirty-five bile duct brushing
specimens were included in the study. The FISH assay utilized the
commercially available UroVysion probes. The indeterminate cytology
results were considered as negative for statistical analysis. RESULTS:
Twenty-two of 35 patients were diagnosed as having malignancy based on
tissue diagnosis or clinical progression of disease by image assessment.
The sensitivity of routine cytology and FISH for the detection of
malignancy was 14\% (3/22) and 55\% (12/22), respectively (p = 0.003).
The specificity of routine cytology and FISH was 100\% (13/13) and 62\%
(8/13), respectively (p = 0.025). The false-positive rate for routine
cytology and FISH was 0\% (0/13) and 38\% (5/13), respectively.
CONCLUSIONS: Our study shows that FISH is significantly more sensitive
than routine cytology for the detection of malignancy in bile duct
brushing specimens. However, in our study, the specificity of FISH was
poor compared to the excellent specificity of routine cytology. The
compromised specificity of FISH may limit its utility in the detection
of malignant bile duct brushing specimens.

PubMed: \url{https://www.ncbi.nlm.nih.gov/pubmed/?term=29734171}

doi: \url{https://doi.org/10.1159/000488636}

{}

{}

\begin{center}\rule{0.5\linewidth}{\linethickness}\end{center}

\begin{itemize}
\tightlist
\item
  \textbf{Cytological features of mixed adenoneuroendocrine carcinoma of
  the ampulla of Vater: A case report with immunocytochemical analyses}
\end{itemize}

\emph{Diagnostic cytopathology 2018 Jun;46(6):540-546}

Mixed adenoneuroendocrine carcinoma (MANEC) is defined as a tumor that
has morphologically recognizable both adenocarcinoma and neuroendocrine
carcinoma components comprising at least 30\% of either components.
MANEC occurring in the ampulla of Vater is extremely rare, and only 16
cases have been reported in the English language literature. In the
present report, we describe the first case of MANEC of the ampulla of
Vater with immunocytochemical analyses. An 82-year-old Japanese male was
incidentally found to have a tumorous lesion in the ampulla of Vater.
Endoscopic ultrasound-fine needle aspiration (EUS-FNA) of the tumor was
performed. The Papanicolaou smear demonstrated the presence of different
three components. The most dominant component was cohesive clusters of
small round cells with round to oval nuclei with powdery chromatin and
scant cytoplasm, which corresponded to small cell carcinoma. The second
component was an adenocarcinoma, which was composed of irregularly
overlapping clusters of tall columnar cells with large round to oval
nuclei containing conspicuous nucleoli. The third component was an
adenoma, which was comprised of flat cohesive clusters of columnar cells
without atypia. Immunocytochemical analyses demonstrated that
synaptophysin was expressed in the small round cells, and cdx-2 was
expressed in all three components. Accordingly, a cytodiagnosis of MANEC
with adenoma component was made. Preoperative diagnosis of ampullary
MANEC is difficult. However, this report clearly demonstrates three
different components in the EUS-FNA cytological specimen. Therefore, we
suggest that cytological examination is a useful method for diagnosis of
MANEC of the ampulla of Vater.

PubMed: \url{https://www.ncbi.nlm.nih.gov/pubmed/?term=29341470}

doi: \url{https://doi.org/10.1002/dc.23891}

{}

{}

\begin{center}\rule{0.5\linewidth}{\linethickness}\end{center}

\begin{itemize}
\tightlist
\item
  \textbf{The Diagnostic Accuracy of Cytology for the Diagnosis of
  Hepatobiliary and Pancreatic Cancers}
\end{itemize}

\emph{Acta cytologica 2018 Jun;():1-6}

OBJECTIVE: Although cytology testing is considered a valuable method to
diagnose tumors that are difficult to access such as
hepato-biliary-pancreatic (HBP) malignancies, its diagnostic accuracy
remains unclear. We therefore aimed to investigate the diagnostic
accuracy of cytology testing for HBP tumors. STUDY DESIGN: We performed
a retrospective study of all cytology samples that were used to confirm
radiologically detected HBP tumors between 2002 and 2016. The cytology
techniques used in our center included fine needle aspiration (FNA),
brush cytology, and aspiration of bile. Sensitivity, specificity,
positive and negative predictive values, and likelihood ratios were
calculated in comparison to histological confirmation. RESULTS: From a
total of 133 medical records, we calculated an overall sensitivity of
76\%, specificity of 74\%, a negative likelihood ratio of 0.30, and a
positive likelihood ratio of 2.9. Cytology was more accurate in
diagnosing lesions of the liver (sensitivity 79\%, specificity 57\%) and
biliary tree (sensitivity 100\%, specificity 50\%) compared to
pancreatic (sensitivity 60\%, specificity 83\%) and gallbladder lesions
(sensitivity 50\%, specificity 85\%). Cytology was more accurate in
detecting primary cancers (sensitivity 77\%, specificity 73\%) when
compared to metastatic cancers (sensitivity 73\%, specificity 100\%).
FNA was the most frequently used cytological technique to diagnose HBP
lesions (sensitivity 78.8\%). CONCLUSION: Cytological testing is
efficient in diagnosing HBP cancers, especially for hepatobiliary
tumors. Given its relative simplicity, cost-effectiveness, and paucity
of alternative diagnostic methods, cytology should still be considered
as a first-line tool for diagnosing HBP malignancies.

PubMed: \url{https://www.ncbi.nlm.nih.gov/pubmed/?term=29898439}

doi: \url{https://doi.org/10.1159/000489549}

{}

{}

\begin{center}\rule{0.5\linewidth}{\linethickness}\end{center}

\begin{itemize}
\tightlist
\item
  \textbf{Fine-needle aspiration of a pancreatic neuroendocrine tumor
  with prominent rhabdoid features}
\end{itemize}

\emph{Diagnostic cytopathology 2018 Jul;46(7):600-603}

Pancreatic neuroendocrine tumors (PanNETs) are uncommon neoplasms that
conventionally possess architectural and cytomorphological features seen
in neuroendocrine neoplasms found at other sites. When present, these
features often allow rapid identification of neuroendocrine
differentiation and an accurate diagnosis. Here, we report the cytologic
findings seen on fine-needle aspiration (FNA) of a PanNET with distinct
rhabdoid features. This morphology is rare in PanNETs and has been
reported in only two case series examining surgical resection specimens
and has not been described on FNA. It is important to recognize this
morphology as this variant appears to portend an aggressive clinical
course. Furthermore, unfamiliarity with this morphologic variant may
lead to a larger initial differential and thus delay final diagnosis.

PubMed: \url{https://www.ncbi.nlm.nih.gov/pubmed/?term=29359517}

doi: \url{https://doi.org/10.1002/dc.23892}

{}

{}

\begin{center}\rule{0.5\linewidth}{\linethickness}\end{center}

\begin{itemize}
\tightlist
\item
  \textbf{Well differentiated grade 3 pancreatic neuroendocrine tumors
  compared with related neoplasms: A morphologic study}
\end{itemize}

\emph{Cancer cytopathology 2018 May;126(5):326-335}

BACKGROUND: Pancreatic neuroendocrine neoplasms with a Ki-67 labeling
index greater than 20\% were reclassified in 2017 by the World Health
Organization into well differentiated (WD) and poorly differentiated
grade 3 neuroendocrine carcinoma (NEC). The authors describe the
cytologic features of grade 3 WD pancreatic neuroendocrine neoplasms
compared with grade 2 neoplasms and NEC. METHODS: Fine-needle aspirates
from 65 pancreatic neuroendocrine neoplasms were reviewed, and their
cytomorphologic features were compared across grade 2, WD grade 3, and
PD small cell type (PD-S), large cell type (PD-L), and type not
otherwise specified (PD-NOS) neoplasms. RESULTS: The 65 aspirates
consisted of 19 grade 2 neoplasms, 32 WD grade 3 neoplasms, and 14 NECs
(6 PD-S, 5 PD-L, and 3 PD-NOS). The medians Ki-67 proliferation index
was 11\% (range, 3.2\%-17\%) in grade 2 neoplasms, 40\% (range,
21\%-89\%) in WD grade 3 neoplasms, 80\% (range, 63\%-95\%) in PD-S
neoplasms, 39\% (range, 25\%-61\%) in PD-L neoplasms, and 70\% (range,
30\%-80\%) in PD-NOS neoplasms. Both grade 2 and WD grade 3 neoplasms
were associated with plasmacytoid morphology and smooth nuclear
contours, but WD grade 3 neoplasms had significant increases in abundant
cytoplasm (72\% vs 17\%; P = .007), nuclear tangles (75\% vs 42\%; P =
.006), and apoptosis (86\% vs 58\%; P = .005). Compared with NECs, WD
grade 3 neoplasms had increased plasmacytoid morphology (75\% vs 7\%; P
\textless{} .001), smooth nuclear contours (94\% vs 64\%; P = .02),
round nuclei (59\% vs 21\%; P = .01), and less pleomorphism (13\% vs
50\%; P = .004), molding (9\% vs 79\%; P \textless{} .001), and necrosis
(13\% vs 43\%; P = .003). WD grade 3 neoplasms had less pleomorphism
(13\% vs 50\%; P = .04), less necrosis (13\% vs 60\%; P = .04), and more
plasmacytoid morphology (75\% vs 20\%; P = .03) than PD-L. CONCLUSIONS:
The prevalence of cytologic features differs in WD grade 3 pancreatic
neuroendocrine neoplasms compared with grade 2 neoplasms and NECs, and
these differences assist in the recognition of this newly classified
entity. Cancer Cytopathol 2018;126:326-35. © 2018 American Cancer
Society.

PubMed: \url{https://www.ncbi.nlm.nih.gov/pubmed/?term=29451738}

doi: \url{https://doi.org/10.1002/cncy.21982}

{}

{}

\begin{center}\rule{0.5\linewidth}{\linethickness}\end{center}

\begin{itemize}
\tightlist
\item
  \textbf{Inspissated cyst fluid in endoscopic ultrasound-guided fine
  needle aspiration of pancreatic cysts}
\end{itemize}

\emph{Diagnostic cytopathology 2018 May;46(5):395-399}

BACKGROUND: Inspissated cyst fluid may be identified on pancreatic cyst
aspiration cytology. We report on the cytomorphologic characteristics of
inspissated cyst fluid on EUS-FNA of pancreatic cysts and correlate this
finding with histopathology or multimodal (cytology, cyst fluid
analysis, molecular pathology, imaging) classification of cyst type.
METHODS: The department archives were searched for pancreatic cyst
fine-needle aspiration biopsies that contained dessicated, crystalline
or inspissated material on cytologic preparations. RESULTS: Twenty-eight
cases of pancreatic cysts containing inspissated material were
identified. The cytomorphology of the inspissated material ranged from
fibrillary fan-like structure (54\%), ball-like structures (57\%), and
granular material (43\%). When present, the fibrillary inspissated
material was associated with neoplastic mucinous cysts in 11/15 (73\%)
cases, but was also seen in 2 pseudocysts, 1 serous cystadenoma, and 1
cyst of uncertain type (suggestive of mucinous cyst on EUS). The
presence of fibrillary inspissated cyst material on cytology had a
positive predictive value of 79\% and a specificity of 63\% for a
neoplastic mucinous cyst. CONCLUSION: Although not highly specific, the
presence of inspissated cyst fluid with fibrillary architecture should
be recognized by cytopathologists and interpreted as at least
``atypical'' given the potential association with neoplastic mucinous
cysts of the pancreas.

PubMed: \url{https://www.ncbi.nlm.nih.gov/pubmed/?term=29476610}

doi: \url{https://doi.org/10.1002/dc.23912}

{}

{}

\begin{center}\rule{0.5\linewidth}{\linethickness}\end{center}

\begin{itemize}
\tightlist
\item
  \textbf{Telomere length determined by the fluorescence in~situ
  hybridisation distinguishes malignant and benign cells in cytological
  specimens}
\end{itemize}

\emph{Cytopathology : official journal of the British Society for
Clinical Cytology 2018 Jun;29(3):262-266}

BACKGROUND: Telomeres are tandem repeats of TTAGGG at the end of
eukaryotic chromosomes that play a key role in preventing chromosomal
instability. The aim of the present study is to determine telomere
length using fluorescence in~situ hybridisation (FISH) on cytological
specimens. METHODS: Aspiration samples (n~=~41) were smeared on glass
slides and used for FISH. RESULTS: Telomere signal intensity was
significantly lower in positive cases (cases with malignancy, n~=~25) as
compared to negative cases (cases without malignancy, n~=~16), and the
same was observed for centromere intensity. The difference in DAPI
intensity was not statistically significant. The ratio of telomere to
centromere intensity did not show a significant difference between
positive and negative cases. There was no statistical difference in the
signal intensities of aspiration samples from ascites or pleural
effusion (n~=~23) and endoscopic ultrasound-guided FNA samples from the
pancreas (n~=~18). CONCLUSIONS: The present study revealed that telomere
length can be used as an indicator to distinguish malignant and benign
cells in cytological specimens. This novel approach may help improve
diagnosis for cancer patients.

PubMed: \url{https://www.ncbi.nlm.nih.gov/pubmed/?term=29578263}

doi: \url{https://doi.org/10.1111/cyt.12535}

{}

{}

\begin{center}\rule{0.5\linewidth}{\linethickness}\end{center}

\begin{itemize}
\tightlist
\item
  \textbf{Diagnosis of paraganglioma as a pancreatic mass: A case
  report}
\end{itemize}

\emph{Diagnostic cytopathology 2018 Jun;():}

Paragangliomas are rare neoplasms that arise from the chromaffin cells
of the autonomic nervous system. Although paragangliomas can occur
anywhere paraganglia are present, they tend to occur in the head, neck,
and retroperitoneum. Rarely, paragangliomas can occur in the
peripancreatic area and present as a pancreatic mass, creating a
diagnostic challenge for the clinician, radiologist, and pathologist.
Here, we present a case of a 70-year-old woman with history of breast
carcinoma who presented with chronic constipation, early satiety, and an
abdominal mass. Her first abdominal CT described a 3.6 cm × 5 cm × 4.5
cm cystic and solid mass involving the pancreatic tail that was
suspicious for a pancreatic neoplasm. A subsequent abdominal CT
described a 5.9 cm multilobulated solid and cystic lesion close to the
pancreatic tail. Endoscopic ultrasound-guided fine-needle aspirate of
the mass demonstrated scant to moderate cellularity of a heterogeneous
population of atypical cells, some with epithelioid morphology and
others appearing neuroendocrine-like. By morphology and
immunohistochemical stains, an extra-adrenal paraganglioma or
pheochromocytoma was considered as a possible diagnosis. The surgical
resection specimen confirmed the diagnosis of paraganglioma. This case
demonstrates the importance of awareness of paragangliomas in the
differential diagnosis of a fine-needle aspiration of a pancreatic mass
to avoid erroneous diagnosis.

PubMed: \url{https://www.ncbi.nlm.nih.gov/pubmed/?term=29882285}

doi: \url{https://doi.org/10.1002/dc.23974}

{}

{}

\begin{center}\rule{0.5\linewidth}{\linethickness}\end{center}

\begin{itemize}
\tightlist
\item
  \textbf{Endoscopic ultrasound guided brush/fine-needle aspiration
  cytology: A 15-month study}
\end{itemize}

\emph{Diagnostic cytopathology 2018 Jun;46(6):461-472}

BACKGROUND: Endoscopic ultrasound-guided fine-needle aspiration
(EUS-FNA) has become increasingly popular for the diagnosis and staging
of gastrointestinal diseases and peri-gastrointestinal lesions. The
application of FNA/Brush has dramatically expanded the clinical utility
of EUS. AIMS AND OBJECTIVE: To evaluate the diagnostic accuracy, study
the spectrum of lesions encountered in EUS-FNAC/brush cytology of
gastrointestinal and peri-gastrointestinal lesions. MATERIALS AND
METHODS: Total of 124 patients during the period from August 2015 to
November 2016 was included in the study. Routine staining was done.
RESULTS: A total of 124 cases were studied with 86\% (107 cases) being
satisfactory for evaluation. M:F ratio was 1:1.03, mean age of 50.5
years. The most common site was common bile duct (CBD) (37\%) followed
by lymph node (21\%), pancreas (17.7\%), esophagus (17\%), stomach
(3.5\%), liver (1.8\%), gallbladder (1\%), and spleen (1\%). In total,
53.4\% lesions were benign, in 6.5\% atypical cells were seen, 12.1\%
were suspicious for malignancy, and 28\% cases were positive for
malignancy. Follow-up was available in 102 cases with
cyto-histopathological concordance rate of 90\%. CONCLUSION:
EUS-FNA/Brush is a reliable, sensitive, specific and minimally invasive
way to establish a diagnosis. It can be utilized as a pre-operative
procedure for the management of many intra-abdominal lesions and prevent
unnecessary invasive procedures.

PubMed: \url{https://www.ncbi.nlm.nih.gov/pubmed/?term=29504276}

doi: \url{https://doi.org/10.1002/dc.23917}

{}

{}

\begin{center}\rule{0.5\linewidth}{\linethickness}\end{center}

\begin{itemize}
\tightlist
\item
  \textbf{Immunohistochemical Staining for S100P, SMAD4, and IMP3 on
  Cell Block Preparations is Sensitive and Highly Specific for
  Pancreatic Ductal Adenocarcinoma}
\end{itemize}

\url{https://www.sciencedirect.com/science/article/pii/S2213294518301042}

\begin{center}\rule{0.5\linewidth}{\linethickness}\end{center}

\hypertarget{others}{%
\subsection{Others}\label{others}}

\hypertarget{techniques}{%
\subsubsection{Techniques}\label{techniques}}

\begin{center}\rule{0.5\linewidth}{\linethickness}\end{center}

\begin{itemize}
\tightlist
\item
  \textbf{The diagnostic and cellularity yield of reverse bevel versus
  fork-tip fine needle biopsy}
\end{itemize}

\emph{Diagnostic cytopathology 2018 May;():}

INTRODUCTION: Two new systems with a novel tip (Procore™ and SharkCore™)
have been introduced for ultrasound-guided fine needle biopsy (US-FNB).
Direct comparison of the diagnostic yield of these needles in the
evaluation of pancreatic lesions is currently under investigation. This
study aims to compare the diagnostic and cellular yields of the two
needle systems. METHODS: Consecutive patients with upper
gastrointestinal lesions undergoing EUS-FNB using 22 gauge Procore™
(reverse bevel) or SharkCore™ (fork-tip) needles were included in the
study. Cytological rapid on-site evaluation (ROSE) slides were scored on
a numerical scale of diagnostic yield relative to the number of passes.
Similarly, histology of biopsy material was assessed on diagnostic
quality using a numeric score. The final diagnosis was based on
resection specimens and/or follow-up of clinical and imaging data of the
subject. RESULTS: The diagnostic yield was similar between the fork-tip
and reverse bevel needles (125/163; 77\% vs 103/139;74\% with P = .60).
Sub-analysis for solid pancreatic masses demonstrated similar results
(69/88; 78\% vs.~83/107; 78\% with P = .88). The fork-tip needle had a
lower mean number of passes (2.5 vs 3.1; P = .04) and ROSE was utilized
in significantly less cases than in the reverse bevel needle group (77\%
versus 98.3\%). CONCLUSIONS: Although we observed no difference in the
diagnostic yield using either the fork-tip or the reverse bevel needle,
the fork-tip needle had significantly better performance with regards to
achieving more adequate cytologic specimen in fewer number of passes
while at the same time requiring fewer episodes of ROSE.

PubMed: \url{https://www.ncbi.nlm.nih.gov/pubmed/?term=29737018}

doi: \url{https://doi.org/10.1002/dc.23966}

{}

{}

\begin{center}\rule{0.5\linewidth}{\linethickness}\end{center}

\begin{itemize}
\tightlist
\item
  \textbf{Comparison of 3 Ways of Dissecting the Pancreatoduodenectomy
  Specimen and Their Impact in the Lymph Node Count and the Lymph Node
  Metastatic Ratio}
\end{itemize}

\emph{International journal of surgical pathology 2018
Jun;():1066896918780343}

BACKGROUND: Lymph node metastasis (LNM) is a strong prognostic factor in
the cancer of the pancreatobiliary tree, but it is influenced by the
number of lymph nodes (LNs). The lymph node ratio (LNR) is considered a
more reliable factor than the number of LNM. The aim was to examine the
LN retrieval and the LNR of 3 pathologic work-up strategies. METHODS:
Pancreaticoduodenectomies (n = 165) were analyzed comparing 3
pathological dissection techniques, classified as ``control,'' ``Verbeke
method,'' and ``Adsay method'' groups. RESULTS: The mean of the
dissected LNs and the number of cases with \textgreater{}20 LNs were
superior in the Adsay method group, compared with the other groups ( P
\textless{} .001). The LNR was different between the Adsay and Verbeke
groups (0.144 vs 0.069, P = .032). The median of the 3 positive LNs was
associated with decreased survival compared with an absence of LNM
(3-year specific survival of 48\% vs 22\%, P = .011). In the
multivariate analysis, LNM (hazard ratio = 6.148, 95\% confidence
interval = 2.02-8.1, P = .042) and the evaluation of \textgreater{}15
LNs (hazard ratio = 12.52, 95\% confidence interval = 5.51-21.01, P =
.001) were independent predictors of survival. CONCLUSION: The Adsay
technique for LN retrieval was associated with a better LN count, more
cases with LNM, and an LNR \textgreater{}0.1.

PubMed: \url{https://www.ncbi.nlm.nih.gov/pubmed/?term=29873282}

doi: \url{https://doi.org/10.1177/1066896918780343}

{}

{}

\begin{center}\rule{0.5\linewidth}{\linethickness}\end{center}

\begin{itemize}
\tightlist
\item
  \textbf{Immunolabeling of Cleared Human Pancreata Provides Insights
  into Three-Dimensional Pancreatic Anatomy and Pathology}
\end{itemize}

\emph{The American journal of pathology 2018 Jul;188(7):1530-1535}

Visualizing pathologies in three dimensions can provide unique insights
into the biology of human diseases. A rapid and easy-to-implement
dibenzyl ether-based technique was used to clear thick sections of
surgically resected human pancreatic parenchyma. Protocols were
applicable to both fresh and formalin-fixed, paraffin-embedded tissue.
The penetration of antibodies into dense pancreatic parenchyma was
optimized using both gradually increasing antibody concentrations and
centrifugal flow. Immunolabeling with antibodies against cytokeratin 19
was visualized using both light sheet and confocal laser scanning
microscopy. The technique was applied successfully to 26 sections of
pancreas, providing three-dimensional (3D) images of normal pancreatic
tissue, pancreatic intraepithelial neoplasia, intraductal papillary
mucinous neoplasms, and infiltrating pancreatic ductal adenocarcinomas.
3D visualization highlighted processes that are hard to conceptualize in
two dimensions, such as invasive carcinoma growing into what appeared to
be pre-existing pancreatic ducts and within venules, and the tracking of
long cords of neoplastic cells parallel to blood vessels. Expanding this
technique to formalin-fixed, paraffin-embedded tissue opens pathology
archives to 3D visualization of unique biosamples and rare diseases. The
application of immunolabeling and clearing to human pancreatic
parenchyma provides detailed visualization of normal pancreatic anatomy,
and can be used to characterize the 3D architecture of diseases
including pancreatic intraepithelial neoplasia, intraductal papillary
mucinous neoplasm, and pancreatic ductal adenocarcinomas.

\url{https://www.sciencedirect.com/science/article/pii/S0002944018300014}

PubMed: \url{https://www.ncbi.nlm.nih.gov/pubmed/?term=29684363}

doi: \url{https://doi.org/10.1016/j.ajpath.2018.04.002}

{}

{}

\begin{center}\rule{0.5\linewidth}{\linethickness}\end{center}

\hypertarget{animal-studies}{%
\subsubsection{Animal Studies}\label{animal-studies}}

\begin{center}\rule{0.5\linewidth}{\linethickness}\end{center}

\begin{itemize}
\tightlist
\item
  \textbf{Distribution of bile acid receptor TGR5 in the
  gastrointestinal tract of dogs}
\end{itemize}

\emph{Histology and histopathology 2018 Jul;():18025}

Takeda-G-protein-receptor-5 (TGR5) is a receptor for bile acids and its
expression has been described in a variety of tissues and species.
Characterization of TGR5 distribution and function has been investigated
in drug discovery for the treatment of metabolic diseases in humans.
Because dogs are one of the species used in biomedical research and
share some similarities with human gastrointestinal diseases, the
objective of this study was to characterize the distribution of TGR5
receptor in the canine species. This study characterizes the
distribution of TGR5 receptor in the gastrointestinal tract, liver,
gallbladder, and pancreas of 8 dogs. The distribution of TGR5 antigen
and mRNA expression was investigated using immunohistochemistry and RNA
in situ hybridization, respectively. TGR5 immunolabeling was located in
the cell membrane or in the cell membrane and cytoplasm. TGR5
immunolabeling was broadly distributed in macrophages, endothelial
cells, ganglion cells, and leiomyocytes throughout all the examined
tissues. Epithelial cells from tongue, stomach to rectum, as well as
from gallbladder, biliary and pancreatic ducts demonstrated TGR5
immunolabeling. In endocrine cells, TGR5 immunolabeling was observed in
intestinal enteroendocrine cells and islets of Langerhans in the
pancreas. The hepatocytes had a unique pattern of immunolabeling located
on the canalicular surface of the cell membrane. TGR5 mRNA expression
was located mainly in the nucleus and the only negative cells throughout
all examined tissues were striated muscle from tongue and esophagus,
muscularis mucosae, esophageal glands, and hepatic sinusoids. These
findings indicate that the bile acid receptor TGR5 is ubiquitously
distributed in the canine gastrointestinal tract.

PubMed: \url{https://www.ncbi.nlm.nih.gov/pubmed/?term=29999170}

doi: \url{https://doi.org/10.14670/HH-18-025}

{}

{}

\begin{center}\rule{0.5\linewidth}{\linethickness}\end{center}

\begin{itemize}
\tightlist
\item
  \textbf{Pancreatic Effects of a Bruton's Tyrosine Kinase
  Small-molecule Inhibitor in Rats Are Strain-dependent}
\end{itemize}

\emph{Toxicologic pathology 2018 Jun;46(4):460-472}

Inhibitors of Bruton's tyrosine kinase (BTK) are under development as
potential therapies for various autoimmune diseases. In repeat-dose
toxicity studies, small-molecule BTK inhibitors (BTKi) have been
reported to cause a constellation of histologic effects at the
pancreatic endocrine-exocrine interface in male rats; however, similar
findings were not reported in other species. Since the BTKi-induced
pancreatic effect is morphologically similar to well-documented
spontaneous changes (predominantly characterized by insular/peri-insular
hemorrhage, pigment deposition, chronic inflammation, and fibrosis) that
are known to vary by rat strain, we investigated potential
strain-dependent differences in the pancreatic effects of a
small-molecule BTKi, LY3337641. Following 13 weeks of LY3337641
treatment, Crl:CD(SD) rats were most sensitive, Crl:WI(Han) rats were of
intermediate sensitivity, and Hsd:SD rats were least sensitive. These
strain differences appear to be related to differences in rate of weight
gain across strains and sexes; however, a definitive mechanism was not
determined. This study demonstrated that BTKi-induced pancreatic effects
were highly dependent on rat strain and correlated with differences in
the incidence and severity of the spontaneous background change. When
considered with the lack of pancreas effects in nonrat species, these
changes in rats are unlikely predictive of similar changes in humans
administered a BTK inhibitor.

PubMed: \url{https://www.ncbi.nlm.nih.gov/pubmed/?term=29699458}

doi: \url{https://doi.org/10.1177/0192623318770163}

{}

{}

\begin{center}\rule{0.5\linewidth}{\linethickness}\end{center}

\begin{itemize}
\tightlist
\item
  \textbf{Histopathological alterations in mice under sub-acute
  treatment with Hintonia latiflora methanolic stem bark extract}
\end{itemize}

\emph{Histology and histopathology 2018 Jun;():18016}

The indiscriminate use of herbal products is increasingly growing
worldwide; nonetheless consumers are not warned about the potential
health risks that these products may cause. Hintonia latiflora (Hl) is a
tree native to the American continent belonging to the Rubiaceae family
and its stem bark is empirically used mainly to treat diabetes and
malaria; supplements containing Hl are sold in America and Europe
without medical prescription, thus scientific information regarding its
toxicity as a consequence of a regular consumption is needed. In the
present study, the histopathological effect of 200 and 1000 mg/kg of
Hintonia latiflora methanolic stem bark extract (HlMeOHe) was evaluated
in the small bowel, liver, pancreas, kidneys and brain of CD-1 male mice
after oral sub-acute treatment for 28 days. No histopathological
alterations were observed in the brain and small bowel of the treated
animals; however, mice presented diarrhea from day 2 of treatment with
both doses. No histological changes were observed in the tissues
collected from the animals treated with 200 mg /kg, except for the liver
that depicted periportal hepatitis. Animals treated with the higher dose
showed in the liver sections hydropic degeneration, hepatitis and
necrosis, kidney sections depicted tubular necrosis and in pancreas
sections, hydropic degeneration of the pancreatic islets was observed.
In conclusion, HlMeOHe damaged the liver with an oral dose of 200 mg/kg,
and at 1000 mg/kg injured the kidneys and pancreas of the CD-1 male
mice.

PubMed: \url{https://www.ncbi.nlm.nih.gov/pubmed/?term=29923593}

doi: \url{https://doi.org/10.14670/HH-18-016}

{}

{}

\begin{center}\rule{0.5\linewidth}{\linethickness}\end{center}

\begin{itemize}
\tightlist
\item
  \textbf{Orthotopic and heterotopic murine models of pancreatic cancer
  and their different responses to FOLFIRINOX chemotherapy}
\end{itemize}

\emph{Disease models \& mechanisms 2018 Jun;():}

INTRODUCTION: Syngeneic, immunocompetent allograft tumor models
recapitulate important aspects of the tumor microenvironment and have
short tumor latency with predictable growth kinetics, making them useful
for trialing novel therapeutics. We describe surgical techniques for
orthotopic and heterotopic PDAC tumor implantation and characterize
phenotypes based on implantation site. METHODS: Mice (n=8 per group)
were implanted with 104 cells in the pancreas or flank. Hy15549 and
Han4.13 cell lines were derived from primary murine PDAC (Ptf1-Cre;
LSL-KRAS-G12D; p53 Lox/+) on C57BL/6 and FVB strains, respectively.
Single cell suspension and solid tumor implants were compared. Tumors
were treated with two intravenous doses of FOLFIRINOX and responses
evaluated. RESULTS: All mice developed pancreatic tumors within 7 days.
Orthotopic tumors grew faster and larger than heterotopic tumors. By 3
weeks, orthotopic mice began losing weight, and showed declines in body
condition requiring euthanasia starting at 4 weeks. Single cell
injection into the pancreas had near 100\% engraftment, but solid tumor
implant engraftment was approximately 50\% and was associated with
growth restriction. Orthotopic tumors were significantly more responsive
to IV FOLFIRINOX compared to heterotopic tumors, with greater reductions
in size and increased apoptosis. Heterotopic tumors were more
desmoplastic and hypovascular. However, drug uptake into tumor tissue
was equivalent regardless of tumor location or degree of fibrosis,
indicating that microenvironment differences between heterotopic and
orthotopic tumors influenced response to therapy. CONCLUSION: Orthotopic
and heterotopic allograft locations confer unique microenvironments that
influence growth kinetics, desmoplastic response, and angiogenesis.
Tumor location influences chemosensitivity to FOLFIRINOX and should
inform future preclinical trials.

PubMed: \url{https://www.ncbi.nlm.nih.gov/pubmed/?term=29903803}

doi: \url{https://doi.org/10.1242/dmm.034793}

{}

{}

\begin{center}\rule{0.5\linewidth}{\linethickness}\end{center}


\end{document}
