% Options for packages loaded elsewhere
\PassOptionsToPackage{unicode}{hyperref}
\PassOptionsToPackage{hyphens}{url}
%
\documentclass[
]{article}
\usepackage{lmodern}
\usepackage{amsmath}
\usepackage{ifxetex,ifluatex}
\ifnum 0\ifxetex 1\fi\ifluatex 1\fi=0 % if pdftex
  \usepackage[T1]{fontenc}
  \usepackage[utf8]{inputenc}
  \usepackage{textcomp} % provide euro and other symbols
  \usepackage{amssymb}
\else % if luatex or xetex
  \usepackage{unicode-math}
  \defaultfontfeatures{Scale=MatchLowercase}
  \defaultfontfeatures[\rmfamily]{Ligatures=TeX,Scale=1}
\fi
% Use upquote if available, for straight quotes in verbatim environments
\IfFileExists{upquote.sty}{\usepackage{upquote}}{}
\IfFileExists{microtype.sty}{% use microtype if available
  \usepackage[]{microtype}
  \UseMicrotypeSet[protrusion]{basicmath} % disable protrusion for tt fonts
}{}
\makeatletter
\@ifundefined{KOMAClassName}{% if non-KOMA class
  \IfFileExists{parskip.sty}{%
    \usepackage{parskip}
  }{% else
    \setlength{\parindent}{0pt}
    \setlength{\parskip}{6pt plus 2pt minus 1pt}}
}{% if KOMA class
  \KOMAoptions{parskip=half}}
\makeatother
\usepackage{xcolor}
\IfFileExists{xurl.sty}{\usepackage{xurl}}{} % add URL line breaks if available
\IfFileExists{bookmark.sty}{\usepackage{bookmark}}{\usepackage{hyperref}}
\hypersetup{
  pdftitle={Pancreatobiliary Pathology Society Journal Watch},
  hidelinks,
  pdfcreator={LaTeX via pandoc}}
\urlstyle{same} % disable monospaced font for URLs
\usepackage[margin=1in]{geometry}
\usepackage{graphicx}
\makeatletter
\def\maxwidth{\ifdim\Gin@nat@width>\linewidth\linewidth\else\Gin@nat@width\fi}
\def\maxheight{\ifdim\Gin@nat@height>\textheight\textheight\else\Gin@nat@height\fi}
\makeatother
% Scale images if necessary, so that they will not overflow the page
% margins by default, and it is still possible to overwrite the defaults
% using explicit options in \includegraphics[width, height, ...]{}
\setkeys{Gin}{width=\maxwidth,height=\maxheight,keepaspectratio}
% Set default figure placement to htbp
\makeatletter
\def\fps@figure{htbp}
\makeatother
\setlength{\emergencystretch}{3em} % prevent overfull lines
\providecommand{\tightlist}{%
  \setlength{\itemsep}{0pt}\setlength{\parskip}{0pt}}
\setcounter{secnumdepth}{5}
\renewcommand{\linethickness}{0.05em}
\usepackage{booktabs}
\usepackage{sectsty} \allsectionsfont{\nohang\centering \emph}
\ifluatex
  \usepackage{selnolig}  % disable illegal ligatures
\fi

\title{Pancreatobiliary Pathology Society Journal Watch}
\usepackage{etoolbox}
\makeatletter
\providecommand{\subtitle}[1]{% add subtitle to \maketitle
  \apptocmd{\@title}{\par {\large #1 \par}}{}{}
}
\makeatother
\subtitle{September - October 2020}
\author{}
\date{\vspace{-2.5em}Last Update on 2020-12-08}

\begin{document}
\maketitle

{
\setcounter{tocdepth}{2}
\tableofcontents
}
\begin{center}\rule{0.5\linewidth}{0.5pt}\end{center}

\hypertarget{pbpath_jw}{%
\section{PBPath Journal Watch Articles}\label{pbpath_jw}}

\begin{center}\rule{0.5\linewidth}{0.5pt}\end{center}

\textbf{Welcome to the PBPath Journal Watch!}

This bi-monthly journal watch features exciting recently published
pancreas and biliary pathology articles that will provide up to date
medical knowledge in our field. These articles will be showcased in
several convenient categories, including surgical pathology,
cytopathology, molecular pathology, pancreas, gallbladder, bile ducts,
and ampulla among others. The articles in each category are in no
particular order. See the list of journals we search regularly
\href{http://pbpath.org/pbpath-journal-watch/}{here}. Previous months'
issues may be found in our
\emph{\href{http://pbpath.org/journal-watch-archive/}{archive}} and you
may see \href{http://pbpath.org/journal-watch-upcoming-issue/}{drafts of
the upcoming issue here}.

We encourage members to actively participate by recommending new
articles and providing feedback using the
\protect\hyperlink{feedback}{forms provided below.}

\textbf{We hope that you will enjoy the new PBPath Journal Watch!}

Back to top

\begin{center}\rule{0.5\linewidth}{0.5pt}\end{center}

\pagebreak

\hypertarget{pancreas}{%
\subsection{Pancreas}\label{pancreas}}

\begin{center}\rule{0.5\linewidth}{0.5pt}\end{center}

\begin{itemize}
\tightlist
\item
  \textbf{Reassessment of the Optimal Number of Examined Lymph Nodes in
  Pancreatoduodenectomy for Pancreatic Ductal Adenocarcinoma}
\end{itemize}

\emph{Annals of surgery 2020 Nov;():}

PubMed: \url{https://www.ncbi.nlm.nih.gov/pubmed/?term=33177357}

OBJECTIVE: To reappraise the optimal number of examined lymph nodes
(ELN) in pancreatoduodenectomy (PD) for pancreatic ductal adenocarcinoma
(PDAC). SUMMARY BACKGROUND DATA: The well-established threshold of 15
ELN in PD for PDAC is optimized for detecting one positive node (PLN)
per the previous 7 edition of the AJCC staging manual. In the framework
of the 8 edition, where at least four PLN are needed for an N2
diagnosis, this threshold may be inadequate for accurate staging.
METHODS: Patients who underwent upfront PD at two academic institutions
between 2000 and 2016 were analyzed. The optimal ELN threshold was
defined as the cut-point associated with a 95\% probability of
identifying at least 4 PLN in N2 patients. The results were validated
addressing the N-status distribution and stage migration. RESULTS:
Overall, 1218 patients were included. The median number of ELN was 26
(IQR 17-37). ELN was independently associated with N2-status (OR 1.27, p
\textless{} 0.001). The estimated optimal threshold of ELN was 28. This
cut-point enabled improved detection of N2 patients and stage III
disease (58\% versus 37\%, p = 0.001). The median survival was 28.6
months. There was an improved survival in N0/N1 patients when ELN
exceeded 28, suggesting a stage migration effect (47 versus 29 months,
adjusted HR 0.649, p \textless{} 0.001). In N2 patients, this threshold
was not associated with survival on multivariable analysis. CONCLUSION:
Examining at least 28 LN in PD for PDAC ensures optimal staging through
improved detection of N2/stage III disease. This may have relevant
implications for benchmarking processes and quality implementation.

doi: \url{https://doi.org/10.1097/SLA.0000000000004552}

\begin{center}\rule{0.5\linewidth}{0.5pt}\end{center}

\begin{center}\rule{0.5\linewidth}{0.5pt}\end{center}

\begin{itemize}
\tightlist
\item
  \textbf{Alterations in Ki67 Labeling Following Treatment of Poorly
  Differentiated Neuroendocrine Carcinomas: A Potential Diagnostic
  Pitfall}
\end{itemize}

\emph{The American journal of surgical pathology 2020 Nov;():}

PubMed: \url{https://www.ncbi.nlm.nih.gov/pubmed/?term=33177340}

Assessment of the Ki67 index is critical for grading well-differentiated
neuroendocrine tumors (WD-NETs), which can show a broad range of
labeling that defines the WHO grade (G1-G3). Poorly differentiated
neuroendocrine carcinomas (PD-NECs) have a relatively high Ki67 index,
\textgreater20\% in all cases and commonly exceeding 50\%. After
anecdotally observing PD-NECs with an unexpectedly low and heterogeneous
Ki67 index following chemotherapy in 5 cases, we identified 15
additional cases of treated high-grade neuroendocrine neoplasms
(HG-NENs). The study cohort comprised 20 cases; 11 PD-NECs, 8 mixed
adenoneuroendocrine carcinomas, and 1 WD-NET, G3 from various anatomic
sites (gastrointestinal tract, pancreas, larynx, lung, and breast). The
Ki67 index was evaluated on pretreatment (when available) and
posttreatment samples. Topographic heterogeneity in the Ki67 index was
expressed using a semi-quantitative score of 0 (no heterogeneity) to 5
(\textgreater80\% difference between maximal Ki67 and minimal Ki67
indices). Relative to the pretreatment group (n=9, mean Ki67 of 86.3\%,
range 80\% to 100\%), the neoplasms in the posttreatment group (n=20,
mean Ki67 of 47.7\%, range 1\% to 90\%) showed a significantly lower
Ki67 index (18/20 cases). Of the 18 cases with a relatively low Ki67
index, 15 showed heterogeneous labeling (mean heterogeneity score of
2.3, range 1 to 5) and in 3 cases it was a homogeneously low. This
phenomenon was observed in all subtypes of HG-NENs. In 6 cases, the
alterations in Ki67 index following treatment were sufficient to place
these HG-NENs in the WHO G1 or G2 grade, erroneously suggesting a
diagnosis of WD-NET, and in 9 cases there was sufficient heterogeneity
in the Ki67 index to suggest that a limited biopsy may sample an area of
low Ki67, even though hotspot regions with a Ki67 index of
\textgreater20\% persisted. In 7 cases, the alterations in the Ki67
index were accompanied by morphologic features resembling a WD-NET.
These observations suggest that there is a potential for
misinterpretation of previously treated PD-NECs as WD-NETs, or for
assigning a lower grade in G3 WD-NETs. While the prognostic significance
of treatment-associated alterations in Ki67 index is unknown, awareness
of this phenomenon is important to avoid this diagnostic pitfall when
evaluating treated NENs.

doi: \url{https://doi.org/10.1097/PAS.0000000000001602}

\begin{center}\rule{0.5\linewidth}{0.5pt}\end{center}

\begin{center}\rule{0.5\linewidth}{0.5pt}\end{center}

\begin{itemize}
\tightlist
\item
  \textbf{Efficacy of EUS-guided FNB using a Franseen needle for tissue
  acquisition and microsatellite instability evaluation in unresectable
  pancreatic lesions}
\end{itemize}

\emph{BMC cancer 2020 Nov;20(1):1094}

PubMed: \url{https://www.ncbi.nlm.nih.gov/pubmed/?term=33176750}

BACKGROUND: The efficacy of immune checkpoint blockade in the treatment
of microsatellite instability (MSI)-high tumors was recently reported.
Therefore, the acquisition of histological specimens is desired in cases
of unresectable solid pancreatic lesions (UR SPLs). This study
investigated the efficacy of endoscopic ultrasound-guided fine-needle
biopsy (EUS-FNB) using a Franseen needle for UR SPL tissue acquisition
and MSI evaluation. METHODS: A total of 195 SPL patients who underwent
EUS-guided fine-needle aspiration (EUS-FNA) or EUS-FNB (EUS-FNAB)
between January 2017 and March 2020 were enrolled in this study. Among
them, 89 SPL patients (FNB: 28, FNA: 61) underwent EUS-FNAB using a 22-G
needle (UR SPLs: 58, FNB: 22, FNA: 36) (UR SPLs after starting MSI
evaluation: 23, FNB: 9, FNA: 14). RESULTS: The puncture number was
significantly lower with FNB than with FNA (median (range): 3 (2-5) vs 4
(1-8), P \textless{} ~0.01, UR SPLs: 3 (2-5) vs 4 (1-8), P = 0.036).
Histological specimen acquisition was more commonly achieved with FNB
than with FNA (92.9\% (26/28) vs 68.9\% (42/61), P = 0.015, UR SPLs:
100\% (22/22) vs 72.2\% (26/36), P \textless{} ~0.01). The histological
specimen required for MSI evaluation was acquired more often with FNB
than with FNA (88.9\% (8/9) vs 35.7\% (5/14), P = 0.03). CONCLUSIONS:
EUS-FNB using a Franseen needle is efficient for histological specimen
acquisition and sampling the required amount of specimen for MSI
evaluation in UR SPL patients.

doi: \url{https://doi.org/10.1186/s12885-020-07588-5}

\begin{center}\rule{0.5\linewidth}{0.5pt}\end{center}

\begin{center}\rule{0.5\linewidth}{0.5pt}\end{center}

\begin{itemize}
\tightlist
\item
  \textbf{Comparative Effectiveness of Neoadjuvant Therapy and Upfront
  Resection for Patients with Resectable Pancreatic Adenocarcinoma: An
  Instrumental Variable Analysis}
\end{itemize}

\emph{Annals of surgical oncology 2020 Nov;():}

PubMed: \url{https://www.ncbi.nlm.nih.gov/pubmed/?term=33174146}

BACKGROUND: Neoadjuvant therapy (NAT) is increasingly being used in the
management of patients with resectable pancreatic ductal adenocarcinoma
(PDAC); however, there is a lack of evidence regarding the benefit among
these patients. OBJECTIVE: The aim of this study was to evaluate overall
survival (OS) in PDAC patients with resectable disease treated with NAT
or upfront resection through instrumental variable (IV) analysis.
DESIGN: A national cohort study of resectable PDAC patients in the
National Cancer Data Base (2007-2015) treated with either upfront
surgery or resection after NAT. Using multivariable modeling and IV
methods, OS was compared between those treated with NAT and upfront
resection. The IV was hospital-level NAT utilization in the most recent
year prior to treatment. RESULTS: The cohort included 16,666 patients
(14,012 upfront resection; 2654 NAT) treated at 779 hospitals. Among
those treated with upfront resection, 59.9\% received any adjuvant
therapy. NAT patients had higher median (27.9~months, 95\% confidence
interval {[}CI{]} 26.2-29.1) and 5-year OS (24.1\%, 95\% CI 21.9-26.3\%)
compared with those treated with upfront surgery (median 21.2~months,
95\% CI 20.7-21.6; 5-year survival 20.9\%, 95\% CI 20.1-21.7\%). After
multivariable modeling, NAT was associated with an approximately 20\%
decrease in the risk of death (hazard ratio {[}HR{]} 0.78, 95\% CI
0.73-0.84), and this effect was magnified in the IV analysis (HR 0.61,
95\% CI 0.47-0.79). CONCLUSIONS: In patients with resectable PDAC, NAT
is associated with improved survival relative to upfront resection.
Given the benefits of multimodality therapy and the challenges in
receiving adjuvant therapy, consideration should be given to treating
all PDAC patients with NAT.

doi: \url{https://doi.org/10.1245/s10434-020-09327-3}

\begin{center}\rule{0.5\linewidth}{0.5pt}\end{center}

\begin{center}\rule{0.5\linewidth}{0.5pt}\end{center}

\begin{itemize}
\tightlist
\item
  \textbf{Acinar cell induced autolysis is a frequent occurrence in
  CytoLyt-fixed pancreatic fine needle aspiration specimens: An analysis
  of 157 cytology samples}
\end{itemize}

\emph{Cancer cytopathology 2020 Nov;():}

PubMed: \url{https://www.ncbi.nlm.nih.gov/pubmed/?term=33136337}

BACKGROUND: Although 10\% formalin is a standard preservative in
pancreatic FNAs, the effect of CytoLyt on pancreatic tissue preservation
has not been systematically explored. METHODS: Smears and cell blocks
from CytoLyt-fixed (CF-CBs) and formalin-fixed (FF-CBs) pancreatic FNAs
were blindly reviewed without knowledge of the fixative used, and the
presence of tissue/tumor autolysis was noted. Controls included FF-CBs
from pancreatic FNAs, CF-CBs from nonpancreatic FNAs, and 4 pancreatic
FNAs with matched CF-CBs and FF-CBs. RESULTS: We found that 62 of 85
(73\%) pancreatic FNAs with CF-CBs showed significant autolysis, which
was most pronounced in acinar cells and/or tumor cells with benign
acinar cells in the background, compared with 2 of 46 (4\%) FF-CBs (P
\textless{} .0001) and 3 of 26 (12\%) CF-CBs from nonpancreatic FNAs
(73\% vs 12\%; P \textless{} .0001). Of the 4 pancreatic FNAs with
matched CF-CBs and FF-CBs, all 4 CF-CBs showed marked autolysis versus
none of the matched FF-CBs. Of the 23 (27\%) pancreatic FNAs with CF-CBs
that did not show autolysis, 10 had no acinar cells, and 7 had only
minute tissue fragments on CB. CONCLUSION: While CytoLyt~is a useful
fixative for nonpancreatic FNAs it is a suboptimal fixative for
pancreatic FNAs and is associated with tissue/tumor autolysis in the
majority of cases, influencing morphologic evaluation, and potentially
immunocytochemical staining. Autolysis appears to be due to acinar
enzymes whose effect is likely interrupted/inhibited by formalin
fixation. Cytopathologists and cytotechnologists should be mindful of
this pitfall and should avoid using CytoLyt~as a fixative for pancreatic
FNAs.

doi: \url{https://doi.org/10.1002/cncy.22378}

\begin{center}\rule{0.5\linewidth}{0.5pt}\end{center}

\begin{center}\rule{0.5\linewidth}{0.5pt}\end{center}

\begin{itemize}
\tightlist
\item
  \textbf{Clinical Implications of Pre- and Postoperative Circulating
  Tumor DNA in Patients with Resected Pancreatic Ductal Adenocarcinoma}
\end{itemize}

\emph{Annals of surgical oncology 2020 Oct;():}

PubMed: \url{https://www.ncbi.nlm.nih.gov/pubmed/?term=33128119}

BACKGROUND: The clinical implications of pre- and postoperative
KRAS-mutated circulating tumor DNA (ctDNA) present in patients with
pancreatic ductal adenocarcinoma (PDAC) have remained an unresolved
issue. This study sought to investigate the clinical significance of
pre- and postoperative ctDNA analyses and their impact on the prognosis
of PDAC patients. METHODS: Digital droplet polymerase chain reaction
detected ctDNA in pre- and postoperative plasma samples prospectively
obtained from patients with resectable and borderline-resectable PDAC.
Its associations with recurrence-free survival (RFS) and overall
survival (OS) were analyzed. The patients were sorted according to the
presence of pre- and postoperative ctDNA, and its ability to stratify
prognosis was evaluated. RESULTS: The study analyzed 97 patients. Both
pre- and postoperative ctDNA were detected in 9 patients, and neither
was detected in 55 patients. Whereas 15 patients harbored only
preoperative ctDNA, 18 patients had only postoperative ctDNA. The
multivariate analysis showed that the presence of preoperative ctDNA was
associated with poorer OS (P = 0.008) and that postoperative ctDNA was
not associated with either RFS or OS. Survival did not differ
significantly between the patients with a positive shift in ctDNA status
and those without detectable pre- or postoperative ctDNA. CONCLUSIONS:
For the patients with PDAC, the presence of preoperative ctDNA was
significantly associated poor OS, whereas postoperative ctDNA was not
associated with poor survival. A positive change in ctDNA did not affect
patients' survival.

doi: \url{https://doi.org/10.1245/s10434-020-09278-9}

\begin{center}\rule{0.5\linewidth}{0.5pt}\end{center}

\begin{center}\rule{0.5\linewidth}{0.5pt}\end{center}

\begin{itemize}
\tightlist
\item
  \textbf{Concordance Between the Ki-67 Index Cutoff Value of 55\% and
  Differentiation in Neuroendocrine Tumor and Neuroendocrine Carcinoma
  in Grade 3 Pancreatic Neuroendocrine Neoplasms}
\end{itemize}

\emph{Pancreas 2020 10;49(10):1378-1382}

PubMed: \url{https://www.ncbi.nlm.nih.gov/pubmed/?term=33122528}

OBJECTIVE: In 2017 and 2019, the World Health Organization defined grade
3 neuroendocrine tumors (G3 NETs) and neuroendocrine carcinoma (G3 NEC)
in the pancreas. The validity of this classification remains to be
verified. METHODS: Clinical data were collected and analyzed for 39 G3
pancreatic neuroendocrine neoplasms (PanNENs) patients between 2009 and
2018. RESULTS: The tumor-node-metastasis stage (P = 0.0260),
differentiation (P = 0.0115), and Ki-67 index (P = 0.0371) are
prognostic factors for G3 PanNENs by Kaplan-Meier survival analysis.
Among 39 patients, 18 had a Ki-67 index of less than 55\% and
well-differentiated morphology (G3 NET) and 16 had a Ki-67 index of 55\%
or greater and poorly differentiated morphology (G3 NEC). Grade 3
neuroendocrine tumor had a significant better prognosis than G3 NEC
(median overall survival time, 25 months {[}95\% confidence interval,
10.854-39.146 months{]} vs 12 months {[}95\% confidence interval,
6.316-17.684 months{]}, P = 0.0164). Based on Cox regression analyses,
tumor-node-metastasis stage (P = 0.016) was identified as the
independent prognostic factor for G3 PanNENs. CONCLUSIONS: The upper
Ki-67 index cutoff of 55\% might be the best cutoff value to define G3
NETs and G3 NECs for G3 PanNENs. The World Health Organization 2017 and
2019 classification system for G3 PanNENs can identify high-risk
patients with G3 PanNENs.

doi: \url{https://doi.org/10.1097/MPA.0000000000001693}

\begin{center}\rule{0.5\linewidth}{0.5pt}\end{center}

\begin{center}\rule{0.5\linewidth}{0.5pt}\end{center}

\begin{itemize}
\tightlist
\item
  \textbf{Lewis Antigen Phenotype and Survival of Patients With
  Pancreatic Cancer}
\end{itemize}

\emph{Pancreas 2020 10;49(10):1348-1354}

PubMed: \url{https://www.ncbi.nlm.nih.gov/pubmed/?term=33122524}

OBJECTIVES: The association of Lewis antigen phenotype with survival of
patients with pancreatic ductal adenocarcinoma was investigated.
METHODS: A total of 1187 patients diagnosed with pancreatic ductal
adenocarcinoma were evaluated in a prospective cohort. Patients were
classified into 3 different groups according to Lewis antigen phenotype:
Lewis antigen (1) A positive {[}Le(a+b-){]}, (2) B positive
{[}Le(a-b+){]}, and (3) negative {[}Le(a-b-){]}. Risk of mortality was
analyzed with Cox regression after adjusting for other predictors.
RESULTS: The risk of mortality increased in the order of Le(a+b-),
Le(a-b+), and Le(a-b-) {[}reference; hazard ratio (HR), 1.27; 95\%
confidence interval (CI){]}, 1.03-1.57; P = 0.02; and HR, 1.65; 95\% CI,
1.31-2.09; P \textless{} 0.001{]} after adjusting for other predictors.
Among patients with serum carbohydrate antigen (CA) 19-9 lower than 37
U/mL, the association seemed more apparent (reference; HR, 1.50; 95\%
CI, 0.77-2.29; P = 0.22; and HR, 2.10; 95\% CI, 1.10-4.02; P \textless{}
0.02). CONCLUSIONS: The risk of mortality increased in the order of
Le(a+b-), Le(a-b+), and Le(a-b-). The difference in prognosis according
to the Lewis antigen phenotype was more pronounced in the low CA 19-9
group, which suggests that the Lewis antigen phenotype works as a
biomarker predicting the prognosis of patients with pancreatic cancer
with undetectable CA 19-9 level.

doi: \url{https://doi.org/10.1097/MPA.0000000000001687}

\begin{center}\rule{0.5\linewidth}{0.5pt}\end{center}

\begin{center}\rule{0.5\linewidth}{0.5pt}\end{center}

\begin{itemize}
\tightlist
\item
  \textbf{Low expression of DDX5 is associated with poor prognosis in
  patients with pancreatic ductal adenocarcinoma}
\end{itemize}

\emph{Journal of clinical pathology 2020 Oct;():}

PubMed: \url{https://www.ncbi.nlm.nih.gov/pubmed/?term=33097588}

AIMS: Pancreatic ductal adenocarcinoma (PDAC) is one of the most fatal
malignancies. Hence, there is a need for new markers and treatment
strategies. P68/DEAD box protein 5 (DDX5) is an ATP-dependent RNA
helicase of the DEAD box protein family. It is a prognostic marker for
several cancers. In this study, we aimed to evaluate the expression and
clinical relevance of DDX5 in PDAC. METHODS: DDX5 expression in tissue
microarray blocks containing 230 PDAC samples was examined using
immunohistochemical analysis. DDX5 expression was considered high when
more than 50\% of the cells were stained and low when less than 50\% of
the cells were stained. We investigated the association between DDX5
expression and clinicopathological parameters, including patient
survival. RESULTS: The nuclei of normal pancreatic ducts, normal acinar
cells and PDAC cells were stained positive for DDX5 although the
intensity and distribution of DDX5 expression varied. Islet cells showed
strong and diffuse staining of DDX5. DDX5 expression was low and high in
148 (64.3\%) and 82 cases (35.7\%), respectively. Low DDX5 expression
was significantly associated with an advanced pT factor (pT2-pT3: tumour
size,\textgreater20 mm), lymphatic involvement, advanced
tumour-node-metastasis (TNM) stage (stages IIB, III, and IV), and venous
involvement. In addition, the multivariate analysis revealed that DDX5
expression is an independent prognostic factor for PDAC. CONCLUSION:
These results suggest that DDX5 plays an important role in tumour
invasiveness and PDAC prognosis.

doi: \url{https://doi.org/10.1136/jclinpath-2020-207002}

\begin{center}\rule{0.5\linewidth}{0.5pt}\end{center}

\begin{center}\rule{0.5\linewidth}{0.5pt}\end{center}

\begin{itemize}
\tightlist
\item
  \textbf{The histopathology of SPINK1-associated chronic pancreatitis}
\end{itemize}

\emph{Pancreatology : official journal of the International Association
of Pancreatology (IAP) \ldots{} {[}et al.{]} 2020 Dec;20(8):1648-1655}

PubMed: \url{https://www.ncbi.nlm.nih.gov/pubmed/?term=33097431}

BACKGROUND: The identification of genetic risk factors for chronic
pancreatitis, such as PRSS1, CFTR and SPINK1, provides the opportunity
to define key pathologic hallmarks and etiologic-specific changes. For
example, pancreata from PRSS1 and CFTR patients exhibit progressive
lipomatous atrophy without significant fibrosis. Considering the
pathology of SPINK1-associated pancreatitis is ill-defined, we examined
the pancreata of SPINK1 patients with chronic pancreatitis. METHODS:
Histologic sections after total pancreatectomy with islet
autotransplantation and associated clinicopathologic data were collected
from 28 patients with SPINK1 germline alterations. Clinical findings,
germline data, anatomic anomalies and pathologic findings were
descriptively evaluated. RESULTS: Patients ranged in age from 5 to 48
years (median, 21.6 years) with abdominal pain between 2 and 25 years
(median, 5.8 years). Most patients were SPINK1 heterozygous and 14
(50\%) had co-occurring CFTR (n~=~12) and CTRC (n~=~2) mutations. Other
pancreatitis risk factors included anatomic anomalies (n~=~9) and
tobacco use (n~=~1). Overall, 24 (86\%) patients had additional
pancreatitis-associated germline alterations, SPINK1 homozygosity,
anatomic anomalies or environmental factors. Examination of pancreata
revealed a sequential pattern of exocrine parenchymal loss and
replacement by prominent fibrosis, dependent on the duration of
abdominal pain. No malignancies were identified, but low-grade
pancreatic intraepithelial neoplasia was present for 2 cases.
CONCLUSIONS: Within this descriptive study, SPINK1-associated
pancreatitis is characterized by parenchymal fibrosis and suggests
divergent pathophysiologic mechanisms from PRSS1 and CFTR-associated
pancreatitis. Moreover, SPINK1 patients frequently had additional
etiologic factors that did not impact the development of pancreatic
fibrosis and may implicate SPINK1 as a disease modifier gene.

doi: \url{https://doi.org/10.1016/j.pan.2020.10.030}

\begin{center}\rule{0.5\linewidth}{0.5pt}\end{center}

\begin{center}\rule{0.5\linewidth}{0.5pt}\end{center}

\begin{itemize}
\tightlist
\item
  \textbf{Implications of Perineural Invasion on Disease Recurrence and
  Survival After Pancreatectomy for Pancreatic Head Ductal
  Adenocarcinoma}
\end{itemize}

\emph{Annals of surgery 2020 Oct;():}

PubMed: \url{https://www.ncbi.nlm.nih.gov/pubmed/?term=33086324}

OBJECTIVE: To describe PNI and to evaluate its impact on disease-free
(DFS) and overall survival (OS) in patients with resected pancreatic
ductal adenocarcinoma (PDAC). SUMMARY OF BACKGROUND DATA: Although PNI
is a prognostic factor for survival in many GI cancers, there is limited
knowledge regarding its impact on tumor recurrence, especially in
``early stage disease'' (PDAC ≤20 mm, R0/N0 PDAC). METHODS: This
multicenter retrospective study included patients undergoing PDAC
resection between 2009 and 2014. The association of PNI with DFS and OS
was analyzed using Cox proportional-hazards models. RESULTS: PNI was
found in 87\% of 778 patients included in the study, with lower rates in
PDAC ≤20 mm (78.7\%) and in R0/N0 tumors (70.6\%). PNI rate did not
differ between patients who underwent neoadjuvant therapy and upfront
surgery (88\% vs 84\%, P = 0.08). Although not significant at
multivariate analysis (P = 0.07), patients with PNI had worse DFS at
univariate analysis (median DFS: 20 vs 15 months, P \textless{} 0.01).
PNI was the only independent predictor of DFS in R0/N0 tumors (hazard
ratio {[}HR{]}: 2.2) and in PDAC ≤20 mm (HR: 1.8). PNI was an
independent predictor of OS in the entire cohort (27 vs 50 months, P =
0.01), together with G3 tumors, pN1 status, carbohydrate antigen (CA)
19.9 \textgreater37 and pain. CONCLUSIONS: PNI represents a major
determinant of tumor recurrence and patients' survival in pancreatic
cancer. The role of PNI is particularly relevant in early stages,
supporting the hypothesis that invasion of nerves by cancer cells has a
driving role in pancreatic cancer progression.

doi: \url{https://doi.org/10.1097/SLA.0000000000004464}

\begin{center}\rule{0.5\linewidth}{0.5pt}\end{center}

\begin{center}\rule{0.5\linewidth}{0.5pt}\end{center}

\begin{itemize}
\tightlist
\item
  \textbf{Endoscopic ultrasound acquired portal venous circulating tumor
  cells predict progression free survival and overall survival in
  patients with pancreaticobiliary cancers}
\end{itemize}

\emph{Pancreatology : official journal of the International Association
of Pancreatology (IAP) \ldots{} {[}et al.{]} 2020 Dec;20(8):1747-1754}

PubMed: \url{https://www.ncbi.nlm.nih.gov/pubmed/?term=33082106}

BACKGROUND AND AIMS: Despite recent advances, patients with
pancreaticobiliary cancers have a poor prognosis. We previously
demonstrated the efficacy of endoscopic ultrasound (EUS) guided
acquisition of portal vein (PV) blood for enumeration of circulating
tumor cells (CTCs). The aim of this study was to assess PV-CTCs as
potential biomarkers for the assessment of progression-free (PFS) and
overall survival (OS) in patients with pancreaticobiliary cancers.
METHODS: 17 patients with biopsy-proven pancreaticobiliary malignancy
were enrolled. CTCs were enumerated from both peripheral and PV blood.
All patients were followed until death. PFS and OS were evaluated with
the log-rank test and summarized with the use of Kaplan-Meier methods.
Unadjusted and adjusted Cox-proportional hazards models were fitted to
study the relationship between PV-CTCs and PFS and OS. RESULTS: After
3.5 years of follow-up, all patients had expired. PV-CTCs were detected
in all patients (median PV-CTCs 62.0/7.5~mL (interquartile range
{[}IQR{]} 17-132). The mean PFS in patients with PV-CTCs
\textless185/7.5~mL was significantly longer than patients with PV-CTCs
≥185/7.5~mL (43.3 weeks vs.~12.8 weeks, log-rank p~=~0.002). The mean OS
in patients with PV-CTCs \textless185/7.5~mL was significantly longer
than patients with PV-CTCs ≥185/7.5~mL (75.8 weeks vs.~29.5 weeks,
log-rank p~=~0.021). In an adjusted Cox-proportional hazards model,
PV-CTCs were significant predictors of both PFS and OS (HR 1.004,
p~=~0.037; HR 1.004, p~=~0.044 respectively). CONCLUSION: In this pilot
and feasibility study, EUS-acquired PV-CTCs predicted PFS and OS. Our
findings suggest that PV-CTCs can help provide important prognostic data
for both providers and patients.

doi: \url{https://doi.org/10.1016/j.pan.2020.10.039}

\begin{center}\rule{0.5\linewidth}{0.5pt}\end{center}

\begin{center}\rule{0.5\linewidth}{0.5pt}\end{center}

\begin{itemize}
\tightlist
\item
  \textbf{Epithelial Nr5a2 heterozygosity cooperates with mutant Kras in
  the development of pancreatic cystic lesions}
\end{itemize}

\emph{The Journal of pathology 2020 Oct;():}

PubMed: \url{https://www.ncbi.nlm.nih.gov/pubmed/?term=33079429}

Cystic neoplasms of the pancreas are an increasingly important public
health problem. The majority of these lesions are benign but some
progress to invasive pancreatic ductal adenocarcinoma (PDAC). There is a
dearth of mouse models of these conditions. The orphan nuclear receptor
NR5A2 regulates development, differentiation, and inflammation. Germline
Nr5a2 heterozygosity sensitizes mice to the oncogenic effects of mutant
Kras in the pancreas. Here, we show that - unlike constitutive Nr5a2+/-
mice - conditional Nr5a2 heterozygosity in pancreatic epithelial cells,
combined with mutant Kras (KPN+/- ), leads to a dramatic replacement of
the pancreatic parenchyma with cystic structures and an accelerated
development of high-grade PanINs and PDAC. Timed histopathological
analyses indicated that in KPN+/- mice PanINs precede the formation of
cystic lesions and the latter precede PDAC. A single episode of acute
caerulein pancreatitis is sufficient to accelerate the development of
cystic lesions in KPN+/- mice. Epithelial cells of cystic lesions of
KPN+/- mice express MUC1, MUC5AC, and MUC6, but lack expression of MUC2,
CDX2, and acinar markers, indicative of a pancreato-biliary/gastric
phenotype. In accordance with this, in human samples we found a
non-significantly decreased expression of NR5A2 in mucinous tumours,
compared with conventional PDAC. These results highlight that the
effects of loss of one Nr5a2 allele are time- and cell
context-dependent. KPN+/- mice represent a new model to study the
formation of cystic pancreatic lesions and their relationship with
PanINs and classical PDAC. Our findings suggest that pancreatitis could
also contribute to acceleration of cystic tumour progression in
patients. © 2020 The Pathological Society of Great Britain and Ireland.
Published by John Wiley \& Sons, Ltd.

doi: \url{https://doi.org/10.1002/path.5570}

\begin{center}\rule{0.5\linewidth}{0.5pt}\end{center}

\begin{center}\rule{0.5\linewidth}{0.5pt}\end{center}

\begin{itemize}
\tightlist
\item
  \textbf{A Critical Assessment of Postneoadjuvant Therapy Pancreatic
  Cancer Regression Grading Schemes With a Proposal for a Novel
  Approach}
\end{itemize}

\emph{The American journal of surgical pathology 2020 Oct;():}

PubMed: \url{https://www.ncbi.nlm.nih.gov/pubmed/?term=33074853}

Currently, there is no consensus on the optimal tumor response score
(TRS) system to assess regression in pancreatic cancers resected after
neoadjuvant therapy. We developed a novel TRS (Royal North Shore
{[}RNS{]} system) based on estimating the percentage of tumor bed
occupied by viable cancer and categorized into 3 tiers: grade 1 (≤10\%),
grade 2 (11\% to 75\%), and grade 3 (\textgreater75\%). We assessed 147
resected carcinomas with this and other TRS systems (College of American
Pathologists {[}CAP{]}, MD Anderson Cancer Center {[}MDACC{]}, and
Evans). The 3-tiered RNS system predicted median survival after surgery
for grades 1, 2, and 3 of 54, 23, and 9 months, respectively
(P\textless0.05). The CAP, MDACC, and Evans systems also predicted
survival (P\textless0.05) but less consistently. The median survival for
MDACC and CAP grade 0 (complete regression) was less than MDACC grade 1
and CAP grades 1 and 2. There was no difference in survival between CAP
grades 2 and 3 (P=0.960), Evans grades 1 and 2a (P=0.395), and Evans
grades 2a and 2b (P=0.587). Interobserver concordance was weak for CAP
(κ=0.431), moderate for MDACC (κ=0.691), minimal for Evans (κ=0.307),
and moderate to strong for RNS (κ=0.632 to 0.84). Of age, sex, size,
stage, grade, perineural and vascular invasion, extrapancreatic
extension, margin status, and RNS score, only RNS score, vascular
invasion, and extrapancreatic extension predicted survival in univariate
analysis. Only extrapancreatic extension (P=0.034) and RNS score
(P\textless0.0001) remained significant in multivariate analysis. We
conclude that the RNS system is a reproducible and powerful predictor of
survival after resection for pancreatic cancers treated with neoadjuvant
therapy and should be investigated in larger cohorts.

doi: \url{https://doi.org/10.1097/PAS.0000000000001601}

\begin{center}\rule{0.5\linewidth}{0.5pt}\end{center}

\begin{center}\rule{0.5\linewidth}{0.5pt}\end{center}

\begin{itemize}
\tightlist
\item
  \textbf{Hyalinized stroma is a characteristic feature of pancreatic
  intraductal oncocytic papillary neoplasm: An immunohistochemical
  study}
\end{itemize}

\emph{Annals of diagnostic pathology 2020 Oct;49():151639}

PubMed: \url{https://www.ncbi.nlm.nih.gov/pubmed/?term=33069084}

Hyalinized stroma (HS) is a dense, eosinophilic, and amorphous
extracellular material in the stroma. HS is observed in several tumors;
however, it has not been comprehensively studied in pancreatic
intraductal papillary mucinous neoplasm (IPMN) or intraductal oncocytic
papillary neoplasm (IOPN). Here, we aimed to evaluate the
immunohistochemical and microscopic characteristics of HS in IPMN and
IOPN. The prevalence of HS was determined in 168 cases of IPMN,
including intestinal type (IPMN-I), gastric type (IPMN-G), and
pancreatobiliary type (IPMN-PB), as well as in 11 cases of IOPN.
Immunohistochemical staining for laminin and collagen (types I, II, III,
IV, and V), as well as Congo red staining were performed in IPMN and
IOPN cases containing HS. The prevalence of HS among the IPMN and IOPN
specimens was 1.2\% (2/168 cases) and 45.5\% (5/11 cases), respectively.
The prevalence rates of HS in each IPMN subtype were as follows: 2.2\%
(2/91 cases) in IPMN-G, and 0\% in IPMN-PB and IPMN-I. All seven HS
cases were positive for collagen I, III, IV, and V but were negative for
Congo red staining. Most cases showed negative, focal, or weak
expression of laminin and type II collagen. These findings indicate that
HS is associated with IOPN and is primarily composed of collagen fibers.

doi: \url{https://doi.org/10.1016/j.anndiagpath.2020.151639}

\begin{center}\rule{0.5\linewidth}{0.5pt}\end{center}

\begin{center}\rule{0.5\linewidth}{0.5pt}\end{center}

\begin{itemize}
\tightlist
\item
  \textbf{The use of immunohistochemistry for IgG4 in the diagnosis of
  autoimmune pancreatitis: A systematic review and meta-analysis}
\end{itemize}

\emph{Pancreatology : official journal of the International Association
of Pancreatology (IAP) \ldots{} {[}et al.{]} 2020 Dec;20(8):1611-1619}

PubMed: \url{https://www.ncbi.nlm.nih.gov/pubmed/?term=33060017}

BACKGROUND: The diagnosis of autoimmune pancreatitis (AIP) remains
challenging, especially when serum IgG4 is normal or imaging features
are indeterminate. We performed a systematic review and meta-analysis to
evaluate the performance of IgG4 immunostaining of pancreatic, biliary,
and ampullary tissues as a diagnostic aid for AIP. METHODS: A
comprehensive literature search of the PubMed, EMBASE, and Ovid MEDLINE
databases was conducted until February 2020. The methodological quality
of each study was assessed according to the Quality Assessment of
Diagnostic Accuracy Studies checklist. A random-effects model was used
to summarize the diagnostic odds ratio and other measures of accuracy.
RESULTS: The meta-analysis included 20 studies comprising 346 patients
with AIP and 590 patients with other pancreatobiliary diseases,
including 371 pancreatobiliary malignancies. The summary estimates for
tissue IgG4 in discriminating AIP and controls were as follows:
diagnostic odds ratio 38.86 (95\% confidence interval (CI),
18.70-80.75); sensitivity 0.64 (95\% CI, 0.59-0.69); specificity 0.93
(95\% CI, 0.91-0.95). The area under the curve was 0.939 for tissue IgG4
in discriminating AIP and controls. Subgroup analysis revealed no
significant difference in diagnostic accuracy according to control
groups (pancreatobiliary cancer versus other chronic pancreatitis) and
sampling site (pancreas versus bile duct/ampulla). CONCLUSIONS: Current
data demonstrate that IgG4 immunostaining of pancreatic, biliary, and
ampullary tissue has a high specificity but moderate sensitivity for
diagnosing AIP. IgG4 immunostaining may be useful in supporting a
diagnosis of AIP when AIP is clinically suspected, but a combination of
imaging and serology does not provide a conclusive diagnosis.

doi: \url{https://doi.org/10.1016/j.pan.2020.10.028}

\begin{center}\rule{0.5\linewidth}{0.5pt}\end{center}

\begin{center}\rule{0.5\linewidth}{0.5pt}\end{center}

\begin{itemize}
\tightlist
\item
  \textbf{Expression of the EWSR1-FLI1 fusion oncogene in pancreas cells
  drives pancreatic atrophy and lipomatosis}
\end{itemize}

\emph{Pancreatology : official journal of the International Association
of Pancreatology (IAP) \ldots{} {[}et al.{]} 2020 Dec;20(8):1673-1681}

PubMed: \url{https://www.ncbi.nlm.nih.gov/pubmed/?term=33051146}

BACKGROUND: Pancreatic ductal adenocarcinoma (PDAC) harbors mutant KRAS
as the most common driver mutation. Studies on mouse models have
uncovered the tumorigenic characteristics of the Kras oncogene driving
pancreatic carcinogenesis. Similarly, Ewing sarcoma predominantly
depends on the occurrence of the EWSR1-FLI1 fusion oncogene. The
expression of EWSR1-FLI1 affects pro-tumorigenic pathways and induces
cell transformation. In this study, we investigated whether mutant Kras
could be exchanged by another potent oncogene, such as EWSR1-FLI1, to
initiate pancreatic cancer development. METHODS: We generated two
conditional mouse models expressing mutant KrasG12D (KC) or the
EWSR1-FLI1 oncogene (E/F) in pancreas cells. Pancreatic tissue was
collected from the mice at 4-6 weeks and 11-13 weeks of age as well as
from survival cohorts to determine the development of spontaneous
acinar-to-ductal metaplasia (ADM) and neoplastic lesions.
Immunohistochemistry and immunofluorescence staining were performed to
characterize and quantify changes in tissue morphology. RESULTS: The
expression of the EWSR1-FLI1 fusion protein in pancreas cells was
confirmed by positive FLI1 immunohistochemistry staining. Notably, the
EWSR1-FLI1 expression in pancreas cells resulted in a strong depletion
of the acinar cell mass and an extensive lipomatosis. Although the E/F
mice exhibited spontaneous ADM formation and a shorter overall survival
rate compared to KC mice, no development of neoplastic lesion was
observed in aging E/F mice. CONCLUSIONS: The expression of the
EWSR1-FLI1 oncogene leads to a strong pancreatic atrophy and
lipomatosis. ADM formation indicates that pancreatic acinar cells are
susceptible for EWSR1-FLI1-mediated oncogenic transformation to a
limited extent. However, the EWSR1-FLI1 oncogene is insufficient to
induce pancreatic cancer development.

doi: \url{https://doi.org/10.1016/j.pan.2020.10.033}

\begin{center}\rule{0.5\linewidth}{0.5pt}\end{center}

\begin{center}\rule{0.5\linewidth}{0.5pt}\end{center}

\begin{itemize}
\tightlist
\item
  \textbf{Three distinct stroma types in human pancreatic cancer
  identified by image analysis of fibroblast subpopulations and
  collagen}
\end{itemize}

\emph{Clinical cancer research : an official journal of the American
Association for Cancer Research 2020 Oct;():}

PubMed: \url{https://www.ncbi.nlm.nih.gov/pubmed/?term=33046515}

PURPOSE: Cancer-associated fibroblasts have emerged to be highly
heterogenous and can play multifaced roles in dictating pancreatic
ductal adenocarcinoma (PDAC) progression, immunosuppression and
therapeutic response, highlighting the need for a deeper understanding
of stromal heterogeneity between patients and even within a single
tumor. We hypothesized that image analysis of fibroblast subpopulations
and collagen in PDAC tissues might guide stroma-based patient
stratification to predict clinical outcomes and tumor characteristics.
EXPERIMENTAL DESIGN: A novel multiplex immunohistochemistry-based image
analysis system was established to digitally differentiate fibroblast
subpopulations. Using whole-tissue slides from 215 treatment-naïve
PDACs, we performed concurrent quantification of principal fibroblast
subpopulations and collagen and defined three stroma types:
collagen-rich stroma, fibroblast activation protein a (FAP)-dominant
fibroblast-rich stroma and a smooth muscle actin (ACTA2)-dominant
fibroblast-rich stroma. These stroma types were assessed for the
associations with cancer-specific survival by multivariable Cox
regression analyses, and with clinicopathological factors including CD8+
cell density. RESULTS: FAP-dominant fibroblasts and ACTA2-dominant
fibroblasts represented the principal distinct fibroblast subpopulations
in tumor stroma. Stroma types were associated with patient survival,
SMAD4 status and transcriptome signatures. Compared with FAP-dominant
fibroblast-rich stroma, collagen-rich stroma correlated with prolonged
survival (HR, 0.57; 95\% CI, 0.33-0.99), while ACTA2-dominant
fibroblast-rich stroma exhibited poorer prognosis (HR, 1.65; 95\% CI,
1.06-2.58). FAP-dominant fibroblast-rich stroma was additionally
characterized by restricted CD8+-cell infiltrates and intense neutrophil
infiltration. CONCLUSIONS: This study identified three distinct stroma
types differentially associated with survival, immunity and molecular
features, thereby underscoring the importance of stromal heterogeneity
in subtyping pancreatic cancers and supporting the development of
anti-stromal therapies.

doi: \url{https://doi.org/10.1158/1078-0432.CCR-20-2298}

\begin{center}\rule{0.5\linewidth}{0.5pt}\end{center}

\begin{center}\rule{0.5\linewidth}{0.5pt}\end{center}

\begin{itemize}
\tightlist
\item
  \textbf{Amsterdam International Consensus Meeting: tumor response
  scoring in the pathology assessment of resected pancreatic cancer
  after neoadjuvant therapy}
\end{itemize}

\emph{Modern pathology : an official journal of the United States and
Canadian Academy of Pathology, Inc 2020 Oct;():}

PubMed: \url{https://www.ncbi.nlm.nih.gov/pubmed/?term=33041332}

Histopathologically scoring the response of pancreatic ductal
adenocarcinoma (PDAC) to neoadjuvant treatment can guide the selection
of adjuvant therapy and improve prognostic stratification. However,
several tumor response scoring (TRS) systems exist, and consensus is
lacking as to which system represents best practice. An international
consensus meeting on TRS took place in November 2019 in Amsterdam, The
Netherlands. Here, we provide an overview of the outcomes and consensus
statements that originated from this meeting. Consensus (≥80\%
agreement) was reached on a total of seven statements: (1) TRS is
important because it provides information about the effect of
neoadjuvant treatment that is not provided by other histopathology-based
descriptors. (2) TRS for resected PDAC following neoadjuvant therapy
should assess residual (viable) tumor burden instead of tumor
regression. (3) The CAP scoring system is considered the most adequate
scoring system to date because it is based on the presence and amount of
residual cancer cells instead of tumor regression. (4) The defining
criteria of the categories in the CAP scoring system should be improved
by replacing subjective terms including ``minimal'' or ``extensive''
with objective criteria to evaluate the extent of viable tumor. (5) The
improved, consensus-based system should be validated retrospectively and
prospectively. (6) Prospective studies should determine the extent of
tissue sampling that is required to ensure adequate assessment of the
residual cancer burden, taking into account the heterogeneity of tumor
response. (7) In future scientific publications, the extent of tissue
sampling should be described in detail in the ``Materials and methods''
section.

doi: \url{https://doi.org/10.1038/s41379-020-00683-9}

\begin{center}\rule{0.5\linewidth}{0.5pt}\end{center}

\begin{center}\rule{0.5\linewidth}{0.5pt}\end{center}

\begin{itemize}
\tightlist
\item
  \textbf{Establishing a living biobank of patient-derived organoids of
  intraductal papillary mucinous neoplasms of the pancreas}
\end{itemize}

\emph{Laboratory investigation; a journal of technical methods and
pathology 2020 Oct;():}

PubMed: \url{https://www.ncbi.nlm.nih.gov/pubmed/?term=33037322}

Pancreatic cancer (PaCa) is the third leading cause of cancer-related
deaths in the United States. There is an unmet need to develop
strategies to detect PaCa at an early, operable stage and prevent its
progression. Intraductal papillary mucinous neoplasms (IPMNs) are cystic
PaCa precursors that comprise nearly 50\% of pancreatic cysts detected
incidentally via cross-sectional imaging. Since IPMNs can progress from
low- and moderate-grade dysplasia to high-grade dysplasia and invasion,
the study of these lesions offers a prime opportunity to develop early
detection and prevention strategies. Organoids are an ideal preclinical
platform to study IPMNs, and the objective of the current investigation
was to establish a living biobank of patient-derived organoids (PDO)
from IPMNs. IPMN tumors and adjacent normal pancreatic tissues were
successfully harvested from 15 patients with IPMNs undergoing pancreatic
surgical resection at Moffitt Cancer Center \& Research Institute
(Tampa, FL) between May of 2017 and March of 2019. Organoid cultures
were also generated from cryopreserved tissues. Organoid count and size
were determined over time by both Image-Pro Premier 3D Version 9.1
digital platform and Matlab application of a Circular Hough Transform
algorithm, and histologic and genomic characterization of a subset of
the organoids was performed using immunohistochemistry and targeted
sequencing, respectively. The success rates for organoid generation from
IPMN tumor and adjacent normal pancreatic tissues were 81\% and 87\%,
respectively. IPMN organoids derived from different epithelial subtypes
showed different morphologies in vitro, and organoids recapitulated
histologic and genomic characteristics of the parental IPMN tumor. In
summary, this preclinical model has the potential to provide new
opportunities to unveil mechanisms of IPMN progression to invasion and
to shed insight into novel biomarkers for early detection and targets
for chemoprevention.

doi: \url{https://doi.org/10.1038/s41374-020-00494-1}

\begin{center}\rule{0.5\linewidth}{0.5pt}\end{center}

\begin{center}\rule{0.5\linewidth}{0.5pt}\end{center}

\begin{itemize}
\tightlist
\item
  \textbf{Multiregion whole-exome sequencing of intraductal papillary
  mucinous neoplasms reveals frequent somatic KLF4 mutations
  predominantly in low-grade regions}
\end{itemize}

\emph{Gut 2020 Oct;():}

PubMed: \url{https://www.ncbi.nlm.nih.gov/pubmed/?term=33028669}

OBJECTIVE: Intraductal papillary mucinous neoplasms (IPMNs) are
non-invasive precursor lesions that can progress to invasive pancreatic
cancer and are classified as low-grade or high-grade based on the
morphology of the neoplastic epithelium. We aimed to compare genetic
alterations in low-grade and high-grade regions of the same IPMN in
order to identify molecular alterations underlying neoplastic
progression. DESIGN: We performed multiregion whole exome sequencing on
tissue samples from 17 IPMNs with both low-grade and high-grade
dysplasia (76 IPMN regions, including 49 from low-grade dysplasia and 27
from high-grade dysplasia). We reconstructed the phylogeny for each
case, and we assessed mutations in a novel driver gene in an independent
cohort of 63 IPMN cyst fluid samples. RESULTS: Our multiregion whole
exome sequencing identified KLF4, a previously unreported genetic driver
of IPMN tumorigenesis, with hotspot mutations in one of two codons
identified in \textgreater50\% of the analyzed IPMNs. Mutations in KLF4
were significantly more prevalent in low-grade regions in our sequenced
cases. Phylogenetic analyses of whole exome sequencing data demonstrated
diverse patterns of IPMN initiation and progression. Hotspot mutations
in KLF4 were also identified in an independent cohort of IPMN cyst fluid
samples, again with a significantly higher prevalence in low-grade
IPMNs. CONCLUSION: Hotspot mutations in KLF4 occur at high prevalence in
IPMNs. Unique among pancreatic driver genes, KLF4 mutations are enriched
in low-grade IPMNs. These data highlight distinct molecular features of
low-grade and high-grade dysplasia and suggest diverse pathways to
high-grade dysplasia via the IPMN pathway.

doi: \url{https://doi.org/10.1136/gutjnl-2020-321217}

\begin{center}\rule{0.5\linewidth}{0.5pt}\end{center}

\begin{center}\rule{0.5\linewidth}{0.5pt}\end{center}

\begin{itemize}
\tightlist
\item
  \textbf{Effect of Germline Mutations in Homologous Recombination
  Repair Genes on Overall Survival of Patients with Pancreatic
  Adenocarcinoma}
\end{itemize}

\emph{Clinical cancer research : an official journal of the American
Association for Cancer Research 2020 Oct;():}

PubMed: \url{https://www.ncbi.nlm.nih.gov/pubmed/?term=33028596}

PURPOSE: To compare the clinical characteristics and overall survival
(OS) of germline mutation carriers in homologous recombination repair
(HRR) genes and noncarriers with pancreatic ductal adenocarcinoma
(PDAC). METHODS: Germline DNA from 3,078 patients with PDAC enrolled in
a prospective registry at Mayo Clinic between 2000 and 2017 was analyzed
for mutations in 37 cancer predisposition genes. Characteristics and OS
of patients with mutations in eight genes (ATM, BARD1, BRCA1, BRCA2,
BRIP1, PALB2, RAD51C, and RAD51D) involved in HRR were compared with
patients testing negative for mutations in all 37 genes. RESULTS: The
175 HRR mutation carriers and 2,730 noncarriers in the study had a
median duration of follow-up of 9.9 years. HRR mutation carriers were
younger (median age at diagnosis: 63 vs.~66 years, P \textless{} 0.001)
and more likely to have metastatic disease at diagnosis (46\% vs.~36\%,
P = 0.004). In a multivariable model adjusting for sex, age at
diagnosis, and tumor staging, patients with germline HRR mutations had a
significantly longer OS compared with noncarriers {[}HR, 0.83; 95\%
confidence interval (CI), 0.70-0.97; P = 0.02{]}. Further gene-level
analysis demonstrated that germline ATM mutation carriers had longer OS
compared with patients without germline mutations in any of the 37 genes
(HR, 0.72; 95\% CI, 0.55-0.94; P = 0.01). CONCLUSIONS: This study
demonstrates that germline mutation carrier status in PDAC is associated
with longer OS compared with noncarriers. Further research into tumor
biology and response to platinum-based chemotherapy in germline mutation
carriers with PDAC are needed to better understand the association with
longer OS.

doi: \url{https://doi.org/10.1158/1078-0432.CCR-20-1788}

\begin{center}\rule{0.5\linewidth}{0.5pt}\end{center}

\begin{center}\rule{0.5\linewidth}{0.5pt}\end{center}

\begin{itemize}
\tightlist
\item
  \textbf{Kras mutation rate precisely orchestrates ductal derived
  pancreatic intraepithelial neoplasia and pancreatic cancer}
\end{itemize}

\emph{Laboratory investigation; a journal of technical methods and
pathology 2020 Oct;():}

PubMed: \url{https://www.ncbi.nlm.nih.gov/pubmed/?term=33009500}

Pancreatic ductal adenocarcinoma (PDAC) is the third leading cause of
cancer-related death in the United States. Despite the high prevalence
of Kras mutations in pancreatic cancer patients, murine models
expressing the oncogenic mutant Kras (Krasmut) in mature pancreatic
cells develop PDAC at a low frequency. Independent of cell of origin, a
second genetic hit (loss of tumor suppressor TP53 or PTEN) is important
for development of PDAC in mice. We hypothesized ectopic expression and
elevated levels of oncogenic mutant Kras would promote PanIN arising in
pancreatic ducts. To test our hypothesis, the significance of elevating
levels of K-Ras and Ras activity has been explored by expression of a
CAG driven LGSL-KrasG12V allele (cKras) in pancreatic ducts, which
promotes ectopic Kras expression. We predicted expression of cKras in
pancreatic ducts would generate neoplasia and PDAC. To test our
hypothesis, we employed tamoxifen dependent CreERT2 mediated
recombination. Hnf1b:CreERT2;KrasG12V (cKrasHnf1b/+) mice received 1
(Low), 5 (Mod) or 10 (High) mg per 20 g body weight to recombine cKras
in low (cKrasLow), moderate (cKrasMod), and high (cKrasHigh) percentages
of pancreatic ducts. Our histologic analysis revealed poorly
differentiated aggressive tumors in cKrasHigh mice. cKrasMod mice had
grades of Pancreatic Intraepithelial Neoplasia (PanIN), recapitulating
early and advanced PanIN observed in human PDAC. Proteomics analysis
revealed significant differences in PTEN/AKT and MAPK pathways between
wild type, cKrasLow, cKrasMod, and cKrasHigh mice. In conclusion, in
this study, we provide evidence that ectopic expression of oncogenic
mutant K-Ras in pancreatic ducts generates early and late PanIN as well
as PDAC. This Ras rheostat model provides evidence~that AKT signaling is
an important early driver of invasive ductal derived PDAC.

doi: \url{https://doi.org/10.1038/s41374-020-00490-5}

\begin{center}\rule{0.5\linewidth}{0.5pt}\end{center}

\begin{center}\rule{0.5\linewidth}{0.5pt}\end{center}

\begin{itemize}
\tightlist
\item
  \textbf{Evidence of a common cell origin in a case of pancreatic mixed
  intraductal papillary mucinous neoplasm-neuroendocrine tumor}
\end{itemize}

\emph{Virchows Archiv : an international journal of pathology 2020
Oct;():}

PubMed: \url{https://www.ncbi.nlm.nih.gov/pubmed/?term=33005981}

Recently, the term mixed neuroendocrine non-neuroendocrine neoplasms
(MiNEN) has been proposed as an umbrella definition covering different
possible combinations of mixed neuroendocrine-exocrine neoplasms. Among
these, the adenoma plus neuroendocrine tumor (NET) combination is among
the rarest and not formally recognized by the 2019 WHO Classification.
In this setting, the debate between either collision tumors or true
mixed neoplasms is still unsolved. In this report, a pancreatic
intraductal papillary mucinous neoplasm (IPMN) plus a NET is described,
and the molecular investigations showed the presence in both populations
of the same KRAS, GNAS, and CDKN2A mutations and the amplification of
the CCND1 gene. These data prove clonality and support a common origin
of both components, therefore confirming the true mixed nature. For this
reason, mixed neuroendocrine-exocrine neoplasms, in which the exocrine
component is represented by a glandular precursor lesion (adenoma/IPMN)
only, should be included into the MiNEN family.

doi: \url{https://doi.org/10.1007/s00428-020-02942-1}

\begin{center}\rule{0.5\linewidth}{0.5pt}\end{center}

\begin{center}\rule{0.5\linewidth}{0.5pt}\end{center}

\begin{itemize}
\tightlist
\item
  \textbf{GLUT-1 as a predictor of worse prognosis in pancreatic
  adenocarcinoma: immunohistochemistry study showing the correlation
  between expression and survival}
\end{itemize}

\emph{BMC cancer 2020 Sep;20(1):909}

PubMed: \url{https://www.ncbi.nlm.nih.gov/pubmed/?term=32967636}

BACKGROUND: Various parameters have been considered for predicting
survival in pancreatic ductal adenocarcinoma. Information about western
population is missing. The aim of this study is to assess the
association between Glucose transporter type 1 (GLUT-1) expression and
prognosis for patients with PDAC submitted for surgical resection in a
European cohort. METHODS: Retrospective analysis of PDAC specimens after
pancreatoduodenectomy assessing GLUT-1 expression according to intensity
(weak vs strong) and extension (low if \textless{} 80\% cells were
stained, high if \textgreater{} 80\%) was performed. Statistical
analysis was performed using the exact Fisher test, Student t test or
the Mann-Whitney U test. Survival was analysed using the Kaplan-Meier
method and compared with the Log-rank test. The differences were
considered significant at a two-sided p value of \textless{} 0.05. All
statistical analyses were performed using SPSS® 23.0 for Windows (SPSS
Inc., Chicago, IL, USA). RESULTS: Our study consisted of 39 patients of
which 58.9\% presented with weak and 41.1\% with strong intensity. The
median extension was 90\%: 28.2\% cases presented with a low extension
and 71.8\% with a high extension. No significant differences related to
intensity were found. The high-extension group showed a higher
percentage of T3 PDAC (92.9\% vs 63.6\%, p~= 0.042) and LNR20 (35.7\% vs
0\%, p~= 0.037) as well as shorter disease-free survival (17.58 vs
54.46 months; p~= 0.048). CONCLUSIONS: Our findings suggest that GLUT-1
could be related to higher aggressivity in PDAC and could be used as a
prognostic marker, identifying patients with a worse response to current
therapies who could benefit from more aggressive treatments.

doi: \url{https://doi.org/10.1186/s12885-020-07409-9}

\begin{center}\rule{0.5\linewidth}{0.5pt}\end{center}

\begin{center}\rule{0.5\linewidth}{0.5pt}\end{center}

\begin{itemize}
\tightlist
\item
  \textbf{Comprehensive histological evaluation with clinical analysis
  of venous invasion in pancreatic ductal adenocarcinoma: From histology
  to clinical implications}
\end{itemize}

\emph{Pancreatology : official journal of the International Association
of Pancreatology (IAP) \ldots{} {[}et al.{]} 2020 Oct;20(7):1486-1494}

PubMed: \url{https://www.ncbi.nlm.nih.gov/pubmed/?term=32948429}

OBJECTIVES: Venous invasion is a poor prognostic factor for pancreatic
ductal adenocarcinoma (PDAC). However, our understanding of various
features of venous invasion is limited. Our aim is to comprehensively
evaluate various histopathologic features of venous invasion, including
status, type (lymphatic or venous), number of invasion foci, and
histologic pattern (pancreatic intraepithelial neoplasia
{[}PanIN{]}-like, conventional) in PDACs. METHODS: Various features of
venous invasion, including status, number of invasion foci, histologic
patterns {[}pancreatic intraepithelial neoplasia (PanIN)-like,
conventional{]}, and size of involved vessels in 471 surgically resected
PDACs were evaluated with all available hematoxylin and eosin
(H\&E)-stained slides. RESULTS: Venous invasion was observed in 319
cases (67.7\%) and was more frequently associated with increased tumor
size, extrapancreatic extension, resection margin involvement, diffuse
tumor distribution, lymph node metastasis, and perineural invasion (all
Ps~\textless~.05). High frequency (≥3 foci) of venous invasion was
associated with shorter overall survival both in the entire group and in
the early stage subgroup (stage I; all Ps~\textless~.05). Multivariate
analysis indicated that a high frequency (≥3 foci) of venous invasion,
large tumor size (\textgreater4~cm), higher histologic grade, and lymph
node metastasis, were independent prognostic factors of worse overall
survival (all Ps~\textless~.05). CONCLUSION: Precise evaluation of
venous invasion status, including foci number of invasion, can provide
additional prognostic information for patients undergoing surgical
resection of PDAC, especially for those with early disease stage.

doi: \url{https://doi.org/10.1016/j.pan.2020.08.015}

\begin{center}\rule{0.5\linewidth}{0.5pt}\end{center}

\begin{center}\rule{0.5\linewidth}{0.5pt}\end{center}

\begin{itemize}
\tightlist
\item
  \textbf{Positivity for SATB2 distinguishes Islet1 positive rectal
  neuroendocrine tumours from pancreaticoduodenal neuroendocrine
  tumours}
\end{itemize}

\emph{Journal of clinical pathology 2020 Sep;():}

PubMed: \url{https://www.ncbi.nlm.nih.gov/pubmed/?term=32934105}

AIMS: Determining the site of origin of a metastatic neuroendocrine
tumour (NET) can be challenging and has important prognostic and
therapeutic implications. An immunohistochemical (IHC) panel consisting
of TTF1, CDX2, PAX8/PAX6 and Islet1 is often employed. However, there
can be a significant IHC overlap among different primary sites. Herein,
we sought to determine the utility of including Special AT-rich sequence
binding protein-2 (SATB2) in the IHC panel that is used for determining
the site of origin of a metastatic NET. METHODS: Paraffin tissue
microarrays consisting of 137 primary NETs (26 lung, 22 jejunoileal, 8
appendix, 5 stomach, 4 duodenum, 17 rectum and 55 pancreas) were stained
for SATB2, in addition to the well-described lineage-associated markers,
such as TTF1, CDX2, PAX6 and Islet1. Additionally, a tissue microarray
consisting of 21 metastatic NETs (1 lung, 1 stomach, 8 jejunoileal and
11 pancreas) was stained for TTF1, CDX2, SATB2 and Islet1. The results
were recorded as no staining, weak staining and moderate to strong
staining. RESULTS: All appendiceal NETs and majority (88\%) of the
rectal NETs were positive for SATB2. All primary foregut NETs (stomach,
pancreas, duodenum and lung) were negative for SATB2, except for one
pulmonary NET with weak staining. However, among the metastatic tumours,
5 of 11 pancreatic NETs, 1 stomach NET, 1 lung NET and 2 of 8
jejunoileal NETs showed weak staining. Receiver operating characteristic
analysis incorporating sensitivity and specificity data of IHC panel,
considering moderate to strong staining as truly positive cases, showed
that inclusion of SATB2 to the previously described NET IHC panel
outperformed the panel without SATB2, raising the specificity for
pancreaticoduodenal NETs from 81.2\% to 100\%, with a positive
predictive value (PPV) of 100\% and negative predictive value (NPV) of
82.22\% (p\textless0.0001); for appendiceal NETs the specificity changed
from 99.1\% to 98.5\% and sensitivity increased from 11.8\% to 80\%,
with a PPV and NPV of 66.67\% and 99.26\%, respectively
(p\textless0.0001); and for rectal NETs the specificity increased from
97.6\% to 99.3\% and sensitivity raised from 7.1\% to 66.7\%, with a PPV
and NPV of 80\% and 98.53\%, respectively (p\textless0.0001).
CONCLUSIONS: SATB2 stain is useful in differentiatingIslet1/PAX6
positive pancreatic and rectal NETs, as rectal NETs are typically
moderately to strongly positive for SATB2 and pancreatic NETs are
usually negative or weakly positive for SATB2. Moderate to strong
staining for SATB2 is suggestive of an appendiceal or a rectal primary.
SATB2 may complement the panel of CDX2, TTF1 and Islet1 in determining
the site of origin of an NET in a metastatic setting.

doi: \url{https://doi.org/10.1136/jclinpath-2020-206645}

\begin{center}\rule{0.5\linewidth}{0.5pt}\end{center}

\begin{center}\rule{0.5\linewidth}{0.5pt}\end{center}

\begin{itemize}
\tightlist
\item
  \textbf{Clinical and genomic characterisation of mismatch repair
  deficient pancreatic adenocarcinoma}
\end{itemize}

\emph{Gut 2020 Sep;():}

PubMed: \url{https://www.ncbi.nlm.nih.gov/pubmed/?term=32933947}

OBJECTIVE: To describe the clinical, pathological and genomic
characteristics of pancreatic cancer with DNA mismatch repair deficiency
(MMRD) and proficiency (MMRP). DESIGN: We identified patients with MMRD
and MMRP pancreatic cancer in a clinical cohort (N=1213, 519 with
genetic testing, 53 with immunohistochemistry (IHC)) and a genomic
cohort (N=288 with whole-genome sequencing (WGS)). RESULTS: 12 out of
1213 (1.0\%) in the clinical cohort were MMRD by IHC or WGS. Of the 14
patients with Lynch syndrome, 3 (21.4\%) had an MMRP pancreatic cancer
by IHC, and 4 (28.6\%) were excluded because tissue was unavailable for
testing. MMRD cancers had longer overall survival after surgery
(weighted HR after coarsened exact matching 0.11, 95\% CI 0.02 to 0.78,
p=0.001). One patient with an unresectable MMRD cancer has an ongoing
partial response 3 years after starting treatment with PD-L1/CTLA-4
inhibition. This tumour showed none of the classical histopathological
features of MMRD. 9 out of 288 (3.1\%) tumours with WGS were MMRD.
Despite markedly higher tumour mutational burden and neoantigen loads,
MMRD cancers were significantly less likely to have mutations in usual
pancreatic cancer driver genes like KRAS and SMAD4, but more likely to
have mutations in genes that drive cancers with microsatellite
instability like ACV2RA and JAK1. MMRD tumours were significantly more
likely to have a basal-like transcriptional programme and elevated
transcriptional markers of immunogenicity. CONCLUSIONS: MMRD pancreatic
cancers have distinct clinical, pathological and genomic profiles.
Patients with MMRD pancreatic cancer should be considered for basket
trials targeting enhanced immunogenicity or the unique genomic drivers
in these malignancies.

doi: \url{https://doi.org/10.1136/gutjnl-2020-320730}

\begin{center}\rule{0.5\linewidth}{0.5pt}\end{center}

\begin{center}\rule{0.5\linewidth}{0.5pt}\end{center}

\begin{itemize}
\tightlist
\item
  \textbf{SMAD4-Expressing Pancreatic Ductal Adenocarcinomas Have Better
  Response to Neoadjuvant Therapy and Significantly Lower Lymph Node
  Metastasis Rates}
\end{itemize}

\emph{Pancreas 2020 10;49(9):1153-1160}

PubMed: \url{https://www.ncbi.nlm.nih.gov/pubmed/?term=32897998}

OBJECTIVE: For many patients whose pancreatic ductal adenocarcinoma
(PDAC) is locally advanced, neoadjuvant therapy has been proposed as a
way to decrease tumor burden. Pancreatic ductal adenocarcinoma is
generally thought to be resistant to chemotherapy and radiation,
however, response to neoadjuvant therapy in PDAC has been described in a
subset of patients. The SMAD4 status is considered to be an important
molecular feature which distinguishes two subsets of PDAC,
SMAD4-positive and -negative tumors. The objective of this study was to
evaluate the neoadjuvant treatment response rate as well as compare the
different clinicopathologic variables between SMAD4-positive and
-negative tumors. METHODS: We analyzed the data of patients who
underwent surgical resection for PDAC from 2009-2019. Our cohort from a
single institution included 233 patients. RESULTS: Of the 233 cases, 143
(61.4\%) were SMAD4-negative and 90 (38.6\%) were SMAD4-positive.
Overall, SMAD4-positive tumors with neoadjuvant therapy had better
treatment response and better tumor regression scores. In addition,
SMAD4-positive tumors had a significantly lower lymph node metastasis
rate in both the neoadjuvant and nonneoadjuvant setting. CONCLUSIONS:
Further characterization of the role of SMAD4 within the context of
neoadjuvant therapy will lead to improved personalized therapeutic
strategies.

doi: \url{https://doi.org/10.1097/MPA.0000000000001636}

\begin{center}\rule{0.5\linewidth}{0.5pt}\end{center}

\begin{center}\rule{0.5\linewidth}{0.5pt}\end{center}

\begin{itemize}
\tightlist
\item
  \textbf{Clinical Routine Application of the Second-generation
  Neuroendocrine Markers ISL1, INSM1, and Secretagogin in Neuroendocrine
  Neoplasia: Staining Outcomes and Potential Clues for Determining Tumor
  Origin}
\end{itemize}

\emph{Endocrine pathology 2020 Dec;31(4):401-410}

PubMed: \url{https://www.ncbi.nlm.nih.gov/pubmed/?term=32813226}

Neuroendocrine neoplasms (NENs) have traditionally been identified via
expression of proteins associated to the regulation of secretory
vesicles and granules. We report the clinical usage of the
``second-generation'' proteins ISL LIM homeobox 1 (ISL1), INSM
transcriptional repressor 1 (INSM1), and secretagogin (SECG) as
immunohistochemical markers of neuroendocrine differentiation since
their introduction in clinical routine and compare the results with the
established proteins chromogranin A (CGA) and synaptophysin (SYP). In
total, 161 tumors, including 139 NENs and 22 ``non-NENs'' (unrelated
tumors with an initial suspicion of NEN), were informatively stained for
ISL1, and subsets were also interrogated for INSM1 and/or SECG. Diffuse
or focal positive immunoreactivity was noted for ISL1 in 91/139 NENs
(65\%) and in 6/22 (27\%) non-NENs, for INSM1 in 76/85 NENs (89\%) and
in 2/5 (40\%) non-NENs, and for SECG in 49 out of 64 NENs (77\%) and in
0/5 non-NENs (0\%). Generally, ISL1, INSM1, and SECG exhibited
sensitivities in line with or slightly below that of CGA and SYP-largely
attributable to tissue-specific patterns regarding tumoral origin.
Moreover, for pancreatic and small intestinal NENs, the two largest
subgroups, ISL1 staining results were consistent irrespectively of tumor
source and WHO grade. We verify previously suggested immunohistochemical
schemes of neuroendocrine markers of first- and second-generations to
facilitate the diagnostic process for NENs and confirm that the
second-generation neuroendocrine markers display tissue-specific
patterns. We therefore recommend their implementation in tertiary
endocrine pathology centers, not least to aid in the identification of
primary tumors when analyzing metastases.

doi: \url{https://doi.org/10.1007/s12022-020-09645-y}

\begin{center}\rule{0.5\linewidth}{0.5pt}\end{center}

\begin{center}\rule{0.5\linewidth}{0.5pt}\end{center}

\begin{itemize}
\tightlist
\item
  \textbf{Intraductal pancreatic cancer is less responsive than cancer
  in the stroma to neoadjuvant chemotherapy}
\end{itemize}

\emph{Modern pathology : an official journal of the United States and
Canadian Academy of Pathology, Inc 2020 10;33(10):2026-2034}

PubMed: \url{https://www.ncbi.nlm.nih.gov/pubmed/?term=32457408}

Neoadjuvant chemotherapy (NAC) is often the treatment of choice for
borderline resectable and locally advanced invasive pancreatic ductal
adenocarcinoma (PDAC); however, most cancers only partially respond to
therapy. We hypothesized that the location of residual neoplastic cells
in resected specimens following NAC could provide a clue as to the
mechanisms of resistance. PDAC cells invade the stroma but can also
invade back into and spread via the pancreatic ducts, which has been
referred to as ``cancerization of ducts'' (COD). We compared the
responsiveness to chemotherapy between PDAC cells in the stroma and PDAC
cells in the duct. Pancreatic resections from a total of 174 PDAC
patients (NAC, n = 97; immediate surgery, n = 77) were reviewed. On
hematoxylin and eosin sections, COD was identified at the same
prevalence in both groups (NAC: 50/97 cases, 52\%; immediate surgery:
39/77 cases, 51\%; p = 0.879, Fisher's exact test). However, using
quantitative image analysis of CK19 immunohistochemistry, we found that
the proportion of cancer cells that were intraductal was significantly
different between the NAC and immediate surgery groups (median; 12.7\%
vs.~1.99\%, p \textless{} 0.0001, Mann-Whitney U test). This proportion
was highest in patients with marked therapy responses (36.2\%) compared
with patients with moderate or poor responses (7.21 \& 7.91\%). In
summary, our data suggest that intraductal components in PDAC are less
responsive to chemotherapy than the remainder of the tumor, which could
have important implications for therapeutic resistance.

doi: \url{https://doi.org/10.1038/s41379-020-0572-6}

\begin{center}\rule{0.5\linewidth}{0.5pt}\end{center}

\begin{center}\rule{0.5\linewidth}{0.5pt}\end{center}

\begin{itemize}
\tightlist
\item
  \textbf{Negative prognostic impact of PD-L1 expression in tumor cells
  of undifferentiated (anaplastic) carcinoma with osteoclast-like giant
  cells of the pancreas: study of 13 cases comparing ductal pancreatic
  carcinoma and review of the literature}
\end{itemize}

\emph{Virchows Archiv : an international journal of pathology 2020
Nov;477(5):687-696}

PubMed: \url{https://www.ncbi.nlm.nih.gov/pubmed/?term=32424767}

Pancreatic carcinoma remains one of the leading cancer-related causes of
death worldwide and is generally characterized by a dismal prognosis and
limited potential for oncologic treatment. A rare subvariant of
pancreatic cancer, undifferentiated carcinoma with osteoclast-like giant
cells (UCOGC), has an unpredictable prognosis according to many previous
studies, with unexpectedly long survival in individual cases. In this
study, we collected, retrospectively, 13 cases of well-documented UCOGCs
and performed immunohistochemistry focused on the expression of the
programmed death-ligand 1 (PD-L1) and several other potential
therapeutic and predictive markers (PanTRK, p53, MSH2, PMS2, and the
number of tumor-infiltrating lymphocytes), to explore their correlation
with the follow-up of the patients. As a control group, we examined 24
cases of conventional pancreatic ductal adenocarcinoma (PDAC). In our
results, PanTRK was negative in all 24 cases. P53 did not show any
significant differences between UCOGCs and PDACs, and the entire cohort
was MSH2, MLH1, PMS2, and MSH6 positive. Significant differences were
present in the analysis of PD-L1: UCOGCs were found to express PD-L1
significantly more frequently and have a higher number of
tumor-infiltrating lymphocytes than PDAC. The expression of PD-L1 was
related to significantly shorter survival in patients with UCOGC and in
the entire cohort. Patients with PD-L1 negative UCOGCs displayed
surprisingly long survival in comparison to PD-L1 positive UCOGCs and
PDACs (both PD-L1+ and PD-L1-). We compared our results with previously
published data, and, after statistical analysis, we were able to
identify PD-L1 as an effective prognostic marker of UCOGC and suggest a
strong need for a clinical trial of immune checkpoint immunotherapy in
patients with advanced PD-L1 positive UCOGC.

doi: \url{https://doi.org/10.1007/s00428-020-02830-8}

\begin{center}\rule{0.5\linewidth}{0.5pt}\end{center}

\begin{center}\rule{0.5\linewidth}{0.5pt}\end{center}

\begin{itemize}
\tightlist
\item
  \textbf{RAF1 rearrangements are common in pancreatic acinar cell
  carcinomas}
\end{itemize}

\emph{Modern pathology : an official journal of the United States and
Canadian Academy of Pathology, Inc 2020 09;33(9):1811-1821}

PubMed: \url{https://www.ncbi.nlm.nih.gov/pubmed/?term=32358589}

There is now evidence that gene fusions activating the MAPK pathway are
relatively common in pancreatic acinar cell carcinoma with potentially
actionable BRAF or RET fusions being found in \textasciitilde30\%. We
sought to investigate the incidence of RAF1 fusions in pancreatic
malignancies with acinar cell differentiation. FISH testing for RAF1 was
undertaken on 30 tumors comprising 25 `pure' acinar cell carcinomas, 2
mixed pancreatic acinar-neuroendocrine carcinomas, 1 mixed acinar
cell-low grade neuroendocrine tumor and 2 pancreatoblastomas. RAF1
rearrangements were identified in 5 cases and confirmed by DNA and RNA
sequencing to represent oncogenic fusions (GATM-RAF1, GOLGA4-RAF1,
PDZRN3-RAF1, HERPUD1-RAF1 and TRIM33-RAF1) and to be mutually exclusive
with BRAF and RET fusions, as well as KRAS mutations. Large genome-wide
copy number changes were common and included 1q gain and/or 1p loss in
all five RAF1 FISH-positive acinar cell carcinomas. RAF1 expression by
immunohistochemistry was found in 3 of 5 (60\%) of fusion-positive cases
and no FISH-negative cases. Phospho-ERK1/2 expression was found in 4 of
5 RAF1-fusion-positive cases. Expression of both RAF1 and phospho-ERK1/2
was heterogeneous and often only detected at the tumor-stroma interface,
thus limiting their clinical utility. We conclude that RAF1 gene
rearrangements are relatively common in pancreatic acinar cell
carcinomas (14.3\% to 18.5\% of cases) and can be effectively identified
by FISH with follow up molecular testing. The combined results of
several studies now indicate that BRAF, RET or RAF1 fusions occur in
between one third and one-half of these tumors but are extremely rare in
other pancreatic malignancies. As these fusions are potentially
actionable with currently available therapies, a strong argument can be
made to perform FISH or molecular testing on all pancreatic acinar cell
carcinomas.

doi: \url{https://doi.org/10.1038/s41379-020-0545-9}

\begin{center}\rule{0.5\linewidth}{0.5pt}\end{center}

\begin{center}\rule{0.5\linewidth}{0.5pt}\end{center}

\begin{itemize}
\tightlist
\item
  \textbf{Genetic and clinical correlates of entosis in pancreatic
  ductal adenocarcinoma}
\end{itemize}

\emph{Modern pathology : an official journal of the United States and
Canadian Academy of Pathology, Inc 2020 09;33(9):1822-1831}

PubMed: \url{https://www.ncbi.nlm.nih.gov/pubmed/?term=32350415}

Entosis is a type of regulated cell death that promotes cancer cell
competition. Though several studies have revealed the molecular
mechanisms that govern entosis, the clinical and genetic correlates of
entosis in human tumors is less well understood. Here we reviewed
entotic cell-in-cell (CIC) patterns in a large single institution
sequencing cohort (MSK IMPACT clinical sequencing cohort) of more than
1600 human pancreatic ductal adenocarcinoma (PDAC) samples to identify
the genetic and clinical correlates of this cellular feature. After case
selection, 516 conventional PDACs and 21 ASCs entered this study and
\textasciitilde45,000 HPFs (median 80 HPFs per sample) were reviewed;
549 entotic-CICs were detected through our cohort. We observed that
entotic-CIC occurred more frequently in liver metastasis compared with
primary in PDAC. Moreover, poorly differentiated adenocarcinoma or
adenosquamous carcinoma had more entotic-CIC than well or moderately
differentiated adenocarcinoma. With respect to genetic features TP53
mutations, KRAS amplification, and MYC amplification were significantly
associated with entosis in PDAC tissues. From a clinical standpoint
entotic CICs were independently associated with a poor prognosis by
multivariate Cox regression analysis when considering all cases or
primary PDACs specifically. These results provide a contextual basis for
understanding entosis in PDAC, a highly aggressive cancer for which
molecular insights are needed to improve survival.

doi: \url{https://doi.org/10.1038/s41379-020-0549-5}

\begin{center}\rule{0.5\linewidth}{0.5pt}\end{center}

\begin{center}\rule{0.5\linewidth}{0.5pt}\end{center}

\begin{itemize}
\tightlist
\item
  \textbf{Does Site Matter? Impact of Tumor Location on Pathologic
  Characteristics, Recurrence, and Survival of Resected Pancreatic
  Ductal Adenocarcinoma}
\end{itemize}

\emph{Annals of surgical oncology 2020 Oct;27(10):3898-3912}

PubMed: \url{https://www.ncbi.nlm.nih.gov/pubmed/?term=32307617}

BACKGROUND: The authors hypothesized that in resected pancreatic
adenocarcinoma (PDAC), pathologic characteristics, oncologic outcomes,
prognostic factors, and the accuracy of the American Joint Committee on
Cancer (AJCC) staging system might differ based on tumor location.
METHODS: Patients undergoing pancreatectomy for PDAC at two academic
institutions from 2000 to 2015 were retrieved. A comparative analysis
between head (H-PDAC) and body-tail (BT-PDAC) tumors was performed using
uni- and multivariable models. The accuracy of the eighth AJCC staging
system was analyzed using C-statistics. RESULTS: Among 1466 patients,
264 (18\%) had BT-PDAC, which displayed greater tumor size but
significantly lower rates of perineural invasion and G3/4 grading.
Furthermore, BT-PDAC was associated with a lower frequency of nodal
involvement and a greater representation of earlier stages. The
recurrence-free survival and disease-specific survival times were longer
for BT-PDAC (16 vs 14~months {[}p = 0.020{]} and 33 vs 26~months
{[}p = 0.026{]}, respectively), but tumor location was not an
independent predictor of recurrence or survival in the multivariable
analyses. The recurrence patterns did not differ. Certain prognostic
factors (i.e., CA 19.9, grading, R-status, and adjuvant treatment) were
common, whereas others were site-specific (i.e., preoperative pain,
diabetes, and multivisceral resection). The performances of the AJCC
staging system were similar (C-statistics of 0.573 for H-PDAC and 0.597
for BT-PDAC, respectively). CONCLUSIONS: Despite differences in
pathologic profile found to be in favor of BT-PDAC, tumor location was
not an independent predictor of recurrence or survival after
pancreatectomy. An array of site-specific prognostic factors was
identified, but the AJCC staging system displayed similar prognostic
power regardless of primary tumor location.

doi: \url{https://doi.org/10.1245/s10434-020-08354-4}

\begin{center}\rule{0.5\linewidth}{0.5pt}\end{center}

\begin{center}\rule{0.5\linewidth}{0.5pt}\end{center}

Back to top

\begin{center}\rule{0.5\linewidth}{0.5pt}\end{center}

\pagebreak

\hypertarget{gallbladder}{%
\subsection{Gallbladder}\label{gallbladder}}

\begin{center}\rule{0.5\linewidth}{0.5pt}\end{center}

\begin{itemize}
\tightlist
\item
  \textbf{Mural Intracholecystic Neoplasms Arising in Adenomyomatous
  Nodules of the Gallbladder: An Analysis of 19 Examples of a
  Clinicopathologically Distinct Entity}
\end{itemize}

\emph{The American journal of surgical pathology 2020
Dec;44(12):1649-1657}

PubMed: \url{https://www.ncbi.nlm.nih.gov/pubmed/?term=33060404}

Intracholecystic neoplasms (ICNs) (pyloric gland adenomas and
intracholecystic papillary neoplasms, collectively also called
intracholecystic papillary/tubular neoplasms) form multifocal, extensive
proliferations on the gallbladder mucosa and have a high propensity for
invasion (\textgreater50\%). In this study, 19 examples of a poorly
characterized phenomenon, mural papillary mucinous lesions that arise in
adenomyomatous nodules and form localized ICNs, were analyzed. Two of
these were identified in 1750 consecutive cholecystectomies reviewed
specifically for this purpose, placing its incidence at 0.1\%. Median
age was 68 years. Unlike other gallbladder lesions, these were slightly
more common in men (female/male=0.8), and 55\% had documented
cholelithiasis. All were characterized by a compact multilocular,
demarcated, cystic lesion with papillary proliferations and mucinous
epithelial lining. The lesions' architecture, distribution, location,
and typical size were suggestive of evolution from an underlying
adenomyomatous nodule. All had gastric/endocervical-like mucinous
epithelium, but 5 also had a focal intestinal-like epithelium. Cytologic
atypia was graded as 1 to 3 and defined as 1A: mucinous, without
cytoarchitectural atypia (n=3), 1B: mild (n=7), 2: moderate (n=2), and
3: severe atypia (n=7, 3 of which also had invasive carcinoma, 16\%).
Background gallbladder mucosal involvement was absent in all but 2
cases, both of which had multifocal papillary mucosal nodules. In
conclusion, these cases highlight a distinct clinicopathologic entity,
that is, mural ICNs arising in adenomyomatous nodules, which, by
essentially sparing the ``main'' mucosa, not displaying
``field-effect/defect'' phenomenon, and only rarely (16\%) showing
carcinomatous transformation, are analogous to pancreatic branch duct
intraductal papillary mucinous neoplasms.

doi: \url{https://doi.org/10.1097/PAS.0000000000001603}

\begin{center}\rule{0.5\linewidth}{0.5pt}\end{center}

\begin{center}\rule{0.5\linewidth}{0.5pt}\end{center}

\begin{itemize}
\tightlist
\item
  \textbf{Pitfalls in Morphologic Diagnosis of Pathogens: Lessons
  Learned From the Pseudo-Cystoisospora Epidemic}
\end{itemize}

\emph{International journal of surgical pathology 2020
Oct;():1066896920960813}

PubMed: \url{https://www.ncbi.nlm.nih.gov/pubmed/?term=33016162}

Multiple groups have recently reported involvement of the gallbladder
mucosa of immunocompetent patients by cystoisospora organisms. However,
this has recently been disproved with the support of molecular and
ultrastructural studies. Here we present a summary of these events,
recounting how this pseudo-Cystoisospora epidemic began and ended. This
review also highlights the important role played by ancillary techniques
in supplementing the morphologic diagnosis of pathogens.

doi: \url{https://doi.org/10.1177/1066896920960813}

\begin{center}\rule{0.5\linewidth}{0.5pt}\end{center}

\begin{center}\rule{0.5\linewidth}{0.5pt}\end{center}

\begin{itemize}
\tightlist
\item
  \textbf{Incidence and significance of GATA3 positivity in gallbladder
  adenocarcinoma}
\end{itemize}

\emph{Human pathology 2020 Dec;106():39-44}

PubMed: \url{https://www.ncbi.nlm.nih.gov/pubmed/?term=32991930}

GATA3 immunostaining is a sensitive marker for mammary and urothelial
carcinomas. It is routinely used in surgical pathology during workup of
carcinomas of unknown origin. To the best of our knowledge, this is the
first focused study of GATA3 expression in gallbladder adenocarcinomas.
In this study, we evaluated GATA3 expression in 38 gallbladder
adenocarcinomas. Eight of 38 (21\%) gallbladder adenocarcinomas were
positive for GATA3. The expression of GATA3 tended to be moderate to
strong when present. It was patchy (\textless50\% positivity) in 4
cases, characterized by discrete clusters or groups of malignant cells
with areas of intervening negative tumor cells, whereas it was diffuse
(\textgreater50\% positivity) in the other 4 cases. GATA3 expression did
not show any significant correlation with clinicopathologic features
such as sex, histologic grade, perineural invasion, vascular invasion,
pathologic stage, or distance metastasis. The results of our study show
that a subset of gallbladder adenocarcinomas (21\%) can be GATA3
positive. Awareness of this phenomenon is important while working up
GATA3-positive carcinomas immunohistochemically.

doi: \url{https://doi.org/10.1016/j.humpath.2020.09.012}

\begin{center}\rule{0.5\linewidth}{0.5pt}\end{center}

\begin{center}\rule{0.5\linewidth}{0.5pt}\end{center}

\begin{itemize}
\tightlist
\item
  \textbf{Claudin-18 as a Marker for Identifying the Stomach and
  Pancreatobiliary Tract as the Primary Sites of Metastatic
  Adenocarcinoma}
\end{itemize}

\emph{The American journal of surgical pathology 2020
Dec;44(12):1643-1648}

PubMed: \url{https://www.ncbi.nlm.nih.gov/pubmed/?term=32925194}

Identification of the primary site of cancer is essential for the
treatment of patients with cancer. Numerous immunohistochemical markers
have been developed to determine the differentiation of tumor cells and
suggest possible primary sites, but markers of gastric and pancreatic
adenocarcinomas are still lacking. Claudin-18 is a tight-junction
protein uniquely expressed in gastric epithelial cells and has been
shown to be expressed in gastric and pancreatic adenocarcinoma. Whether
claudin-18 can be used as a marker for identifying the primary site of
cancer is still unclear. In this study, we used the immunohistochemical
method to stain claudin-18 in tissue arrays containing 575 carcinomas
from different anatomic sites and representative sections of 157
metastatic adenocarcinomas. In the group of primary tumors, claudin-18
was frequently expressed in gastric, pancreatic, and pulmonary mucinous
adenocarcinomas. Half of cholangiocarcinomas and ovarian mucinous
carcinomas and some colorectal and pulmonary adenocarcinomas were also
positive for claudin-18. In the metastatic cohort, 15 of 17 (88\%)
gastric adenocarcinomas, 18 of 23 (78\%) pancreatic adenocarcinomas, and
4 of 7 (57\%) cholangiocarcinomas and gallbladder adenocarcinomas were
positive for claudin-18. Only 4 tumors that originated outside the
stomach and pancreatobiliary tract were positive for claudin-18. After
normalization to the tumor frequency, the sensitivity of claudin-18 for
identifying the stomach and pancreatobiliary tract as primary tumor
sites was 79\%, and the specificity was 93\%. The positive and negative
predictive values were 76\% and 94\%, respectively. In conclusion,
claudin-18 represents a sensitive and specific marker for stomach and
pancreatobiliary adenocarcinoma that may be a useful diagnostic tool in
routine surgical pathology.

doi: \url{https://doi.org/10.1097/PAS.0000000000001583}

\begin{center}\rule{0.5\linewidth}{0.5pt}\end{center}

\begin{center}\rule{0.5\linewidth}{0.5pt}\end{center}

\begin{itemize}
\tightlist
\item
  \textbf{Reduced MFAP5 expression in stroma of gallbladder
  adenocarcinoma and its potential diagnostic utility}
\end{itemize}

\emph{Virchows Archiv : an international journal of pathology 2020
Sep;():}

PubMed: \url{https://www.ncbi.nlm.nih.gov/pubmed/?term=32895766}

The diagnosis of invasive adenocarcinoma of the gallbladder can
sometimes be challenging. The presence of true desmoplastic reaction
facilitates the diagnosis of invasion. However, desmoplasia-like changes
can be observed in benign gallbladder conditions, and recognition of
desmoplasia may be challenging based on morphology. In this study, we
tested the expression pattern of microfibril-associated protein 5
(MFAP5), a promising immunohistochemical marker for desmoplasia, in
benign gallbladders with desmoplasia-like reaction and gallbladders with
invasive adenocarcinoma. We also evaluated the diagnostic utility of
MFAP5 in challenging cases with an interobserver agreement study. The
results showed that all benign cases retained intact/positive MFAP5
staining pattern in periglandular connective tissue, whereas 79.3\% (23
out of 29) of cases of adenocarcinomas demonstrated diffuse and complete
loss of MFAP5 staining in the tumor stroma. Interobserver agreement was
improved by 2.66 times when images of MFAP5 immunohistochemistry were
provided. In conclusion, MFAP5 expression is downregulated in the
desmoplastic stroma of gallbladder adenocarcinoma and may provide a
useful diagnostic marker in difficult cases.

doi: \url{https://doi.org/10.1007/s00428-020-02925-2}

\begin{center}\rule{0.5\linewidth}{0.5pt}\end{center}

\begin{center}\rule{0.5\linewidth}{0.5pt}\end{center}

\begin{itemize}
\tightlist
\item
  \textbf{Details of human epidermal growth factor receptor 2 status in
  454 cases of biliary tract cancer}
\end{itemize}

\emph{Human pathology 2020 Nov;105():9-19}

PubMed: \url{https://www.ncbi.nlm.nih.gov/pubmed/?term=32891647}

Human epidermal growth factor receptor 2 (HER2)-targeted therapy has
improved clinical outcomes in patients with HER2-positive breast and
gastric cancers, although ineffective or recurrent cases are present.
One reason for this is the heterogeneity of HER2 expression in cancer
cells. The aim of this study was to investigate the clinicopathological
characteristics and HER2 status of patients with biliary tract cancers
(BTCs). We examined HER2 protein expression by immunohistochemistry,
HER2 gene amplification by fluorescence in situ hybridization, and both
HER2 protein and gene levels simultaneously by gene-protein assay.
Samples were collected from 454 patients who underwent surgical
resection for BTCs (110 intrahepatic cholangiocarcinomas {[}ICC{]}, 67
perihilar extrahepatic cholangiocarcinomas {[}ECC-Bp{]}, 119 distal
extrahepatic cholangiocarcinomas {[}ECC-Bd{]}, 80 gallbladder carcinomas
{[}GBC{]}, and 79 ampullary carcinomas {[}AVC{]}). HER2 status was
assessed according to the guidelines for HER2 testing in
gastroesophageal adenocarcinoma. HER2-positive status was detected in
14.5\% of BTCs (3.7\% of ICC, 3.0\% of ECC-Bp, 18.5\% of ECC-Bd, 31.3\%
of GBC, and 16.4\% of AVC). Furthermore, HER2-positivity tended to
correlate with low histological grade, tumor histology, and macroscopic
features in certain tumors. HER2 heterogeneity was common and highly
frequent (83\%) in BTC cases. Reduced HER2 protein expression in the
deeper invasive areas with simultaneous dedifferentiation was frequently
observed in HER2-positive cancer cells. The findings of this study
suggest that a large subgroup of HER2-positive BTC cases can be
considered for HER2-targeted therapy. Moreover, the HER2 status in BTCs
should be determined carefully using a sensitive approach toward larger
cancer tissues.

doi: \url{https://doi.org/10.1016/j.humpath.2020.08.006}

\begin{center}\rule{0.5\linewidth}{0.5pt}\end{center}

\begin{center}\rule{0.5\linewidth}{0.5pt}\end{center}

\begin{itemize}
\tightlist
\item
  \textbf{Testing for ROS1, ALK, MET, and HER2 rearrangements and
  amplifications in a large series of biliary tract adenocarcinomas}
\end{itemize}

\emph{Virchows Archiv : an international journal of pathology 2020
Jul;477(1):33-45}

PubMed: \url{https://www.ncbi.nlm.nih.gov/pubmed/?term=32447492}

Biliary tract carcinomas are divided into intrahepatic, perihilar,
distal extrahepatic cholangiocarcinomas, and gallbladder
adenocarcinomas. Therapies targeting ROS1, ALK, MET, and HER2
alterations are currently evaluated in clinical trials. We assessed ROS1
and ALK translocations/amplifications as well as MET and HER2
amplifications for each tumor subtype by fluorescent in situ
hybridization (FISH) and immunohistochemistry (IHC) in 73 intrahepatic,
40 perihilar bile duct, 36 distal extrahepatic cholangiocarcinomas, and
45 gallbladder adenocarcinomas (n = 194). By FISH, we detected
targetable alterations in 5.2\% of cases (n = 10): HER2 and MET
amplifications were found in 4.1\% (n = 8) and 1.0\% (n = 2),
respectively. The HER2-amplified cases were mostly gallbladder
adenocarcinomas (n = 5). The MET- and HER2-amplified cases were all
positive by IHC. Fourteen cases without MET amplification were positive
by IHC, whereas HER2 over-expression was detected by IHC only in
HER2-amplified cases. We detected no ALK or ROS1 translocation or
amplification. Several alterations were consistent with aneuploidy: 24
cases showed only one copy of ROS1 gene, 4 cases displayed a profile of
chromosomal instability, and an over-representation of centromeric
alpha-satellite sequences was found in five cases. We confirm a
relatively high rate of HER2 amplifications in gallbladder
adenocarcinomas and the efficacy of IHC to screen these cases. Our
results also suggest the value of IHC to screen MET amplification.
Contrary to initial publications, ROS1 rearrangements seem to be very
rare in biliary tract adenocarcinomas. We confirm a relatively high
frequency of aneuploidy and chromosomal instability and reveal the
over-representation of centromeric alpha-satellite sequences in
intrahepatic cholangiocarcinomas.

doi: \url{https://doi.org/10.1007/s00428-020-02822-8}

\begin{center}\rule{0.5\linewidth}{0.5pt}\end{center}

\begin{center}\rule{0.5\linewidth}{0.5pt}\end{center}

Back to top

\begin{center}\rule{0.5\linewidth}{0.5pt}\end{center}

\pagebreak

\hypertarget{bileducts}{%
\subsection{Bile Ducts}\label{bileducts}}

\begin{center}\rule{0.5\linewidth}{0.5pt}\end{center}

\begin{itemize}
\tightlist
\item
  \textbf{Interobserver agreement in pathologic evaluation of bile duct
  biopsies}
\end{itemize}

\emph{Human pathology 2020 Oct;107():29-38}

PubMed: \url{https://www.ncbi.nlm.nih.gov/pubmed/?term=33129823}

Intraductal biopsy is commonly used for preoperative evaluation of the
etiology of biliary strictures. Interpretation of intraductal biopsies
is frequently challenging. The diagnosis often suffers from
interobserver disagreement, which has not been studied in the
literature. We sought to assess interobserver concordance in the
interpretation of intraductal biopsies. Eighty-five biopsies were
retrieved, falling into five diagnostic categories: negative for
dysplasia (NED), indefinite for dysplasia (IND), low-grade dysplasia
(LGD), high-grade dysplasia (HGD), and carcinoma (CA). Eight
gastrointestinal pathologists blindly reviewed all the slides. Agreement
among pathologists was analyzed using Fleiss κ and weighted concordance
coefficient S∗. A face-to-face consensus/training session was held to
discuss the classification criteria, followed by a second round review.
The overall interobserver agreement was fair in the first round review
(κ~=~0.39; S∗~=~0.56) and improved to moderate in the second round
review (κ~=~0.48; S∗~=~0.69). The agreement before and after consensus
meeting was substantial to nearly perfect for CA (κ~=~0.65, S∗~=~0.83;
and κ~=~0.80, S∗~=~0.91), fair for HGD (κ~=~0.28, S∗~=~0.69; and
κ~=~0.40, S∗~=~0.63), and moderate for NED (κ~=~0.47, S∗~=~0.50; and
κ~=~0.47, S∗~=~0.53). Agreement improved from fair to moderate for LGD
(κ~=~0.36, S∗~=~0.61; and κ~=~0.49, S∗~=~0.71) and slight to fair for
IND (κ~=~0.16, S∗~=~0.51; and κ~=~0.33, S∗~=~0.50). Compared with
Hollande's fixed specimens, the agreement was higher in almost all
diagnostic categories in formalin-fixed biopsies. Overall, interobserver
concordance was improved after a consensus/training session.
Interobserver reproducibility was high at the end of the diagnostic
spectrum (CA) but fair to moderate for other diagnostic categories.

doi: \url{https://doi.org/10.1016/j.humpath.2020.10.003}

\begin{center}\rule{0.5\linewidth}{0.5pt}\end{center}

\begin{center}\rule{0.5\linewidth}{0.5pt}\end{center}

\begin{itemize}
\tightlist
\item
  \textbf{Histological evaluation of tumor differentiation score and
  prognosis of extrahepatic bile duct cancer: A proposal for a new
  histological grading system}
\end{itemize}

\emph{Pathology international 2020 Nov;70(11):857-864}

PubMed: \url{https://www.ncbi.nlm.nih.gov/pubmed/?term=32909295}

Extrahepatic bile duct (EHBD) cancer is a devastating cancer, and more
common in Asian countries than in Western countries. Histological
grading continues to be a highly relevant factor in prognosis and
management of many kinds of cancer, however no uniform histological
grading system exists for EHBD cancer. Histological heterogeneity within
tumors is a problem in the evaluation of EHBD cancer. We developed an
EHBD histological grading scheme to evaluate tumor differentiation
pattern, and statistically analyzed its relationship with prognosis. In
the present study, 257 surgically resected EHBD cancers were reviewed
and their histological glandular differentiation (HGD) pattern was
scored, and then we summed up the most and second most predominant
scores. These scores were statistically analyzed for their relationship
with patient prognosis. Patients showed a trend of shortening
recurrence-free survival (RFS) and overall survival (OS) in association
with higher HGD scores. In multivariate analyses, HGD score was
determined to be an influential factor in RFS (P = 0.00041) and OS
(P \textless{} 0.0001). In addition, combining HGD score and lymph node
status correctly stratified patient prognosis in RFS. In conclusion,
this new HGD scoring system is highly practical and has powerful
prognostic value for EHBD cancer.

doi: \url{https://doi.org/10.1111/pin.13014}

\begin{center}\rule{0.5\linewidth}{0.5pt}\end{center}

\begin{center}\rule{0.5\linewidth}{0.5pt}\end{center}

\begin{itemize}
\tightlist
\item
  \textbf{Prognostic effects of histology-based tumour microenvironment
  scores in resected distal bile duct cancer}
\end{itemize}

\emph{Histopathology 2020 Sep;77(3):402-412}

PubMed: \url{https://www.ncbi.nlm.nih.gov/pubmed/?term=32473032}

AIMS: Histology-based tumour microenvironment (TME) scores are useful in
predicting the prognosis of gastrointestinal cancer. However, their
prognostic roles in distal bile duct cancer (DBDC) have not been
previously studied. This study aimed to evaluate the prognostic
significance of the TME scores using the Klintrup-Mäkinen (KM) grade,
tumour stroma percentage (TSP) and the Glasgow microenvironment score
(GMS) in resected DBDC. METHODS AND RESULTS: Eighty-one patients with
DBDC who underwent curative resection were enrolled. DBDC was graded
according to KM grade, TSP and GMS. A high KM grade was found in
19~patients (24\%) and a high TSP was found in 47~patients (58\%). A
high TSP was significantly correlated with a low KM grade
(P~\textless~0.001). The distribution of the GMS, which was developed by
combining the KM grade and TSP, was as follows: 0 (n~=~19, 24\%), 1
(n~=~19, 24\%) and 2 (n~=~43, 52\%). A low KM grade, high TSP and high
GMS were significantly associated with short overall survival (OS)
(P~\textless~0.001) and relapse-free survival (RFS) (P~\textless~0.001).
Furthermore, multivariate analysis showed that a low KM grade {[}hazard
ratio (HR)~=~3.826; confidence interval (CI)~=~1.650-8.869;
P~=~0.014{]}, high TSP (HR~=~2.193; CI~=~1.173-4.100, P~=~0.002) and
high GMS (HR~=~7.148; CI~=~2.811-18.173) were independent prognostic
factors for short RFS; a low KM grade (HR~=~4.324; CI~=~1.594-11.733)
and high GMS (HR~=~6.332; CI~=~2.743-14.594) were independent prognostic
factors for short OS. CONCLUSION: Histology-based TME scores, including
the KM grade, TSP and GMS, are useful for predicting the survival of
patients with resected DBDC.

doi: \url{https://doi.org/10.1111/his.14163}

\begin{center}\rule{0.5\linewidth}{0.5pt}\end{center}

\begin{center}\rule{0.5\linewidth}{0.5pt}\end{center}

\begin{itemize}
\tightlist
\item
  \textbf{Tiny but mighty: use of next generation sequencing on
  discarded cytocentrifuged bile duct brushing specimens to increase
  sensitivity of cytological diagnosis}
\end{itemize}

\emph{Modern pathology : an official journal of the United States and
Canadian Academy of Pathology, Inc 2020 10;33(10):2019-2025}

PubMed: \url{https://www.ncbi.nlm.nih.gov/pubmed/?term=32457409}

Bile duct brushing (BDB) is used to evaluate pancreatobiliary lesions as
it widely samples lesions with a low complication rate. Cytological
evaluation of BDB is a specific but insensitive test. There is limited
literature on the use of post-cytocentrifuged (PCC) samples, which are
usually discarded, for next-generation sequencing (NGS) as an adjunct to
cytological diagnosis of BDB. In this study we investigate whether
molecular analysis by NGS of PCC specimens improves the sensitivity of
diagnosis. PCC samples from 100 consecutive BDB specimens spanning 93
unique patients were retained. DNA was extracted and mutational analysis
was performed agnostic of morphologic or clinical findings. Each BDB
specimen was characterized as negative, atypical or positive based on
morphological analysis by trained cytopathologists. Performance
characteristics for mutational profiling and morphological analysis were
calculated on the basis of clinicopathologic follow-up. There was
sufficient clinicopathologic follow-up to classify 94 of 100 cases as
either malignant (n = 43) or benign (n = 51). Based on morphologic
analysis of cytology, these 94 cases were classified as either benign
(n = 55), atypical (n = 18), or as at least suspicious or positive for
malignancy (n = 21). Morphologic analysis of cytology showed a
sensitivity of 49\% and a specificity of 100\% if atypical cases were
considered negative. NGS revealed oncogenic alterations in 40/43 (93\%)
of malignant cases based on clinicopathologic follow-up. The most common
alterations were in KRAS and TP53, observed in 77\% and 49\% of
malignant cases respectively. No alterations were observed in the 51
benign cases classified based on clinicopathologic follow-up.
Supplementing cytomorphologic analysis with molecular profiling of PCC
by targeted NGS analysis increased the sensitivity to 93\% and
maintained specificity at 100\%. This study provides evidence for the
utility of NGS molecular profiling of PCC specimens to increase the
sensitivity of BDB cytology samples, although studies with larger
cohorts are needed to verify these findings.

doi: \url{https://doi.org/10.1038/s41379-020-0577-1}

\begin{center}\rule{0.5\linewidth}{0.5pt}\end{center}

\begin{center}\rule{0.5\linewidth}{0.5pt}\end{center}

\begin{itemize}
\tightlist
\item
  \textbf{Utility of DNA flow cytometry in distinguishing between
  malignant and benign intrahepatic biliary lesions}
\end{itemize}

\emph{Virchows Archiv : an international journal of pathology 2020
Oct;477(4):527-534}

PubMed: \url{https://www.ncbi.nlm.nih.gov/pubmed/?term=32296928}

The distinction between well-differentiated intrahepatic
cholangiocarcinoma (iCCA) from its morphological mimics such as bile
duct adenoma (BDA) and hamartoma (BDH) can be challenging, particularly
in small biopsies. Although a few cases of BDA and BDH have been
reported to undergo malignant transformation into iCCA, their neoplastic
versus benign nature remains debated. DNA flow cytometry was performed
on 47 formalin-fixed paraffin-embedded samples of iCCA, 14 BDA, and 18
BDH. Aneuploidy was detected in 22 iCCA (47\%) but in none of the 32 BDA
and BDH samples. Among the 34 iCCA patients who underwent complete
resection and were followed up to tumor recurrence, tumor-related death,
or at least for 1~year, the overall recurrence or death rates
(regardless of flow cytometric results) were 18, 56, and 71\% within 1,
3, and 5~years, respectively. The 1-, 3-, and 5-year recurrence or death
rates in 18 iCCA patients with aneuploidy were 28, 66, and 66\%,
respectively, whereas 16 iCCA patients in the setting of normal DNA
content had 1-, 3-, and 5-year rates of 6, 44, and 72\%, respectively.
Although aneuploid tumors were associated with worse outcomes during the
first 3~years, this difference was not statistically significant (hazard
ratio = 1.4, p = 0.473) in the present sample size. In conclusion, the
frequency of aneuploidy was significantly higher in iCCA (47\%) than in
its benign morphological mimics (0\%), suggesting that it may
potentially serve as a diagnostic marker of malignancy in challenging
situations. Our findings also suggest that most BDAs and BDHs, if not
all, are benign entities and may not represent precursor lesions to
iCCAs that often harbor aneuploidy. Although a larger cohort will be
necessary to further determine the prognostic significance of aneuploidy
in iCCA patients after resection, the patients with aneuploid tumors may
have a higher risk for tumor progression, especially during the first
3~years.

doi: \url{https://doi.org/10.1007/s00428-020-02812-w}

\begin{center}\rule{0.5\linewidth}{0.5pt}\end{center}

\begin{center}\rule{0.5\linewidth}{0.5pt}\end{center}

Back to top

\begin{center}\rule{0.5\linewidth}{0.5pt}\end{center}

\pagebreak

\hypertarget{feedback}{%
\section{Feedback}\label{feedback}}

\href{https://docs.google.com/forms/d/e/1FAIpQLSeD3Z9J6Y7eMmiyM12f_SfAmHUlykb1zxZcwO6lg7cebGYQIQ/viewform}{Google
Feedback Form}

\hypertarget{disqus_thread}{}

Please enable JavaScript to view the comments powered by Disqus.

\begin{center}\rule{0.5\linewidth}{0.5pt}\end{center}

Back to top

\begin{center}\rule{0.5\linewidth}{0.5pt}\end{center}

\pagebreak

\end{document}
