\documentclass[]{article}
\usepackage{lmodern}
\usepackage{amssymb,amsmath}
\usepackage{ifxetex,ifluatex}
\usepackage{fixltx2e} % provides \textsubscript
\ifnum 0\ifxetex 1\fi\ifluatex 1\fi=0 % if pdftex
  \usepackage[T1]{fontenc}
  \usepackage[utf8]{inputenc}
\else % if luatex or xelatex
  \ifxetex
    \usepackage{mathspec}
  \else
    \usepackage{fontspec}
  \fi
  \defaultfontfeatures{Ligatures=TeX,Scale=MatchLowercase}
\fi
% use upquote if available, for straight quotes in verbatim environments
\IfFileExists{upquote.sty}{\usepackage{upquote}}{}
% use microtype if available
\IfFileExists{microtype.sty}{%
\usepackage{microtype}
\UseMicrotypeSet[protrusion]{basicmath} % disable protrusion for tt fonts
}{}
\usepackage[margin=1in]{geometry}
\usepackage{hyperref}
\hypersetup{unicode=true,
            pdftitle={Pancreatobiliary Pathology Society Journal Watch},
            pdfborder={0 0 0},
            breaklinks=true}
\urlstyle{same}  % don't use monospace font for urls
\usepackage{graphicx,grffile}
\makeatletter
\def\maxwidth{\ifdim\Gin@nat@width>\linewidth\linewidth\else\Gin@nat@width\fi}
\def\maxheight{\ifdim\Gin@nat@height>\textheight\textheight\else\Gin@nat@height\fi}
\makeatother
% Scale images if necessary, so that they will not overflow the page
% margins by default, and it is still possible to overwrite the defaults
% using explicit options in \includegraphics[width, height, ...]{}
\setkeys{Gin}{width=\maxwidth,height=\maxheight,keepaspectratio}
\IfFileExists{parskip.sty}{%
\usepackage{parskip}
}{% else
\setlength{\parindent}{0pt}
\setlength{\parskip}{6pt plus 2pt minus 1pt}
}
\setlength{\emergencystretch}{3em}  % prevent overfull lines
\providecommand{\tightlist}{%
  \setlength{\itemsep}{0pt}\setlength{\parskip}{0pt}}
\setcounter{secnumdepth}{0}
% Redefines (sub)paragraphs to behave more like sections
\ifx\paragraph\undefined\else
\let\oldparagraph\paragraph
\renewcommand{\paragraph}[1]{\oldparagraph{#1}\mbox{}}
\fi
\ifx\subparagraph\undefined\else
\let\oldsubparagraph\subparagraph
\renewcommand{\subparagraph}[1]{\oldsubparagraph{#1}\mbox{}}
\fi

%%% Use protect on footnotes to avoid problems with footnotes in titles
\let\rmarkdownfootnote\footnote%
\def\footnote{\protect\rmarkdownfootnote}

%%% Change title format to be more compact
\usepackage{titling}

% Create subtitle command for use in maketitle
\newcommand{\subtitle}[1]{
  \posttitle{
    \begin{center}\large#1\end{center}
    }
}

\setlength{\droptitle}{-2em}

  \title{Pancreatobiliary Pathology Society Journal Watch}
    \pretitle{\vspace{\droptitle}\centering\huge}
  \posttitle{\par}
  \subtitle{August September 2018}
  \author{}
    \preauthor{}\postauthor{}
      \predate{\centering\large\emph}
  \postdate{\par}
    \date{Last Update on 2018-10-26}


\begin{document}
\maketitle

{
\setcounter{tocdepth}{5}
\tableofcontents
}
\href{http://pbpath.org/}{\includegraphics{figures/PBP_header_logo.jpg}}

\begin{center}\rule{0.5\linewidth}{\linethickness}\end{center}

\hypertarget{the-current-pbpath-journal-watch-articles}{%
\section{The Current PBPath Journal Watch
Articles}\label{the-current-pbpath-journal-watch-articles}}

\begin{center}\rule{0.5\linewidth}{\linethickness}\end{center}

\textbf{Wellcome to the PBPath Journal Watch!}

This bi-monthly journal watch features exciting recently published
pancreas and biliary pathology articles that will provide up to date
medical knowledge in our field. These articles will be showcased in
several convenient categories, including surgical pathology, molecular
pathology and cytopathology among others. The articles in each category
are in no particular order.

Previous months' issues may be found in our
\emph{\href{http://pbpath.org/journal-watch-archive/}{archive}}.

We encourage members to actively participate by recommending new
articles and providing feedback using
\emph{\href{https://docs.google.com/forms/d/e/1FAIpQLSeD3Z9J6Y7eMmiyM12f_SfAmHUlykb1zxZcwO6lg7cebGYQIQ/viewform}{the
forms provided}}.

\textbf{We hope that you will enjoy the new PBPath Journal Watch!}

\pagebreak

\begin{center}\rule{0.5\linewidth}{\linethickness}\end{center}

\hypertarget{surgical-pathology}{%
\subsection{Surgical Pathology}\label{surgical-pathology}}

\begin{center}\rule{0.5\linewidth}{\linethickness}\end{center}

\hypertarget{pancreas}{%
\subsubsection{Pancreas}\label{pancreas}}

\begin{center}\rule{0.5\linewidth}{\linethickness}\end{center}

 - \textbf{Comparison of Tumor Regression Grading of Residual Pancreatic
Ductal Adenocarcinoma Following Neoadjuvant Chemotherapy Without
Radiation: Would Fewer Tier-Stratification Be Favorable Toward
Standardization?}

\emph{The American journal of surgical pathology 2018 Sep;():}

To assess whether the College of American Pathologists (CAP) and the
Evans grading systems for neoadjuvant chemotherapy without
radiation-treated pancreatectomy specimens are prognostic, and if a
3-tier stratification scheme preserves data granularity. Conducted
retrospective review of 32 patients with ordinary pancreatic ductal
adenocarcinoma treated with neoadjuvant therapy without radiation
followed by surgical resection. Final pathologic tumor category (AJCC
eighth edition) was 46.9\% ypT1, 34.4\% ypT2, and 18.7\% ypT3. Median
follow-up time was 29.8 months, median disease-free survival (DFS) was
19.6 months, and median overall survival (OS) was 34.2 months. CAP score
1, 2, 3 were present in 5 (15.6\%), 18 (56.3\%), and 9 (28.1\%)
patients, respectively. Evans grade III, IIb, IIa, and I were present in
10 (31.2\%), 8 (25.0\%), 7 (21.9\%), and 7 (21.9\%) patients,
respectively. OS (CAP: P=0.005; Evans: P=0.001) and DFS (CAP: P=0.003;
Evans: P=0.04) were statistically significant for both CAP and Evans.
Stratified CAP scores 1 and 2 versus CAP score 3 was statistically
significant for both OS (P=0.002) and DFS (P=0.002). Stratified Evans
grades I, IIa, and IIb versus Evans grade III was statistically
significant for both OS (P=0.04) and DFS (P=0.02). CAP, Evans, and
3-tier stratification are prognostic of OS and DFS.

PubMed: \url{https://www.ncbi.nlm.nih.gov/pubmed/?term=30211728}

{}

{}

\begin{center}\rule{0.5\linewidth}{\linethickness}\end{center}

 - \textbf{Cancerization of the Pancreatic Ducts: Demonstration of a
Common and Under-recognized Process Using Immunolabeling of Paired Duct
Lesions and Invasive Pancreatic Ductal Adenocarcinoma for p53 and Smad4
Expression}

\emph{The American journal of surgical pathology 2018
Nov;42(11):1556-1561}

Invasive pancreatic ductal adenocarcinoma (PDAC) can infiltrate back
into and spread along preexisting pancreatic ducts and ductules in a
process known as cancerization of ducts (COD). Histologically COD can
mimic high-grade pancreatic intraepithelial neoplasia (HG-PanIN). We
reviewed pancreatic resections from 100 patients with PDAC for the
presence or absence of ducts with histologic features of COD. Features
supporting COD included adjacent histologically similar invasive PDAC
and an abrupt transition between markedly atypical intraductal
epithelium and normal duct epithelium or circumferential involvement of
a duct. As the TP53 and SMAD4 genes are frequently targeted in invasive
PDAC but not HG-PanIN, paired PDAC and histologically suspected COD
lesions were immunolabeled with antibodies to the p53 and Smad4
proteins. Suspected COD was identified on hematoxylin and eosin sections
in 89 (89\%) of the cases. Immunolabeling for p53 and Smad4 was
performed in 68 (76\%) of 89 cases. p53 was interpretable in 55 cases
and all 55 (100\%) cases showed concordant labeling between COD and
invasive PDAC. There was matched aberrant p53 immunolabeling in 37
(67\%) cases including overexpression in 30 (55\%) cases and lack of
expression in 7 (13\%) cases. Smad4 immunolabeling was interpretable in
61 cases and 59 (97\%) cases showed concordant labeling between COD and
invasive PDAC. Matched loss of Smad4 was seen in 28 (46\%) cases. The
immunolabeling of invasive PDAC and COD for p53 and Smad4 supports the
high prevalence of COD observed on hematoxylin and eosin and highlights
the utility of p53 and Smad4 immunolabeling in differentiating COD and
HG-PanIN.

PubMed: \url{https://www.ncbi.nlm.nih.gov/pubmed/?term=30212393}

{}

{}

\begin{center}\rule{0.5\linewidth}{\linethickness}\end{center}

 - \textbf{Proposal of a modified American Joint Committee on Cancer
staging scheme for resectable pancreatic ductal adenocarcinoma with a
lymph node ratio-based N classification: A retrospective cohort study}

\emph{Medicine 2018 Aug;97(34):e12094}

The recently launched 8th edition of the American Joint Committee on
Cancer (AJCC) staging scheme for pancreatic ductal adenocarcinoma (PDAC)
did not account for the impact of the total examined lymph node count on
prognostic accuracy. In this population-based cohort study, we proposed
a modified AJCC staging scheme by incorporating a lymph node ratio
(LNR)-based N classification for patients with resectable PDAC.We
analyzed 8615 patients with resectable PDAC from the Surveillance,
Epidemiology, and End Results database between 2004 and 2013. The
optimal cut-off points for LNR were identified by recursive
partitioning, and an LNR-based N classification was designed
accordingly.The LNR-based N classification could further stratify
patients with the 8th AJCC N1 and N2 disease into subgroups with
significantly different overall survival (P \textless{} .001 for both).
By replacing the 8th AJCC N classification with the corresponding
LNR-based N classification, we further proposed a modified AJCC staging
scheme. The modified AJCC staging outperformed the 8th AJCC staging in
terms of the discriminatory capacity measured by the concordance index
and Akaike information criterion, and the prognostic homogeneity
assessed by using the likelihood ratio chi-squared test and stratified
survival analysis.Replacing the 8th AJCC N classification with the
LNR-based N classification can improve the prognostic performance of the
8th AJCC staging scheme for PDAC.

PubMed: \url{https://www.ncbi.nlm.nih.gov/pubmed/?term=30142869}

{}

{}

\begin{center}\rule{0.5\linewidth}{\linethickness}\end{center}

 - \textbf{Residual Tumor Index: A Prognostically Significant Pathologic
Parameter in Neoadjuvant-treated Pancreatic Ductal Adenocarcinoma}

\emph{The American journal of surgical pathology 2018
Nov;42(11):1480-1487}

In the setting of neoadjuvant therapy (NAT) for pancreatic ductual
adenocarcinoma (PDAC), accurate measurement of tumor size, and
consequently, staging based on AJCC eighth edition, is difficult.
Attempts to address the limitations of tumor size in the NAT setting
have included correlation of residual tumor percent with survival.
However, only cases with complete pathologic response or minimal
residual disease have shown better prognosis compared with all other
groups. To date, no studies have simultaneously evaluated the prognostic
value of tumor size and tumor regression in the setting of PDAC status
post NAT (NAT-PDAC). Our aim was to study the prognostic value of
residual tumor index (RTI), a metric combining residual tumor percent
and tumor bed size as an interaction term (\% residual tumor×tumor bed
size {[}cm{]}). In a cohort of 105 cases of NAT-PDAC, we show that RTI
supersedes the prognostic value of AJCC eighth edition T staging via
multivariate cox regression. At a binary cutoff of 0.35 for RTI, the
hazard ratio for recurrence-free survival is 3.26 (95\% confidence
interval, 1.51-7.04), P\textless{}0.01. We further identified cutoffs of
≤0.2, 0.2 to 2 and \textgreater{}2 that stratified our cases into 3
groups via RTI, which were statistically significant in Kaplan-Meier
curve analysis of recurrence-free survival (P\textless{}0.01) and
overall survival (P\textless{}0.01). RTI represents a novel metric for
combining the prognostic value of tumor size and residual tumor in
NAT-PDAC.

PubMed: \url{https://www.ncbi.nlm.nih.gov/pubmed/?term=30179901}

{}

{}

\begin{center}\rule{0.5\linewidth}{\linethickness}\end{center}

 - \textbf{Is an atypical flat lesion (AFL) a precursor lesion of the
pancreatic ductal adenocarcinoma in human?}

\emph{Pathology international 2018 Apr;():}

PubMed: \url{https://www.ncbi.nlm.nih.gov/pubmed/?term=29664180}

{}

{}

\begin{center}\rule{0.5\linewidth}{\linethickness}\end{center}

 - \textbf{Tumor grade as significant prognostic factor in pancreatic
cancer: validation of a novel TNMG staging system}

\emph{Neoplasma 2018 ;65(4):637-643}

Aim of the study was to asses the tumor grade prognostic value in the
Czech pancreatic cancer patients and to evaluate the accuracy of TNMG
prognostic model. Retrospective analysis of 431 pancreatic cancer
patients undergoing pancreatic resection in seven Czech oncological
centers between 2003 and 2013 was performed. The impact of tumor grade
and the accuracy of TNMG prognostic model were evaluated. Lymph node
status, tumor size, tumor stage and grade were proved as statistically
significant survival predictors. The lower tumor differentiation (grade
3 and 4) was associated with poorer prognosis in all stages (stage I: HR
2.23 {[}1.14; 4.36, CI 95\%{]} p=0.019, stage II: HR 3.09 {[}2.01; 4.77,
CI 95\%{]} p=0.001, stage III and IV: HR 3.52 {[}1.73; 7.18, CI 95\%{]}
p=0.001). Kaplan-Meier analysis verified statistically significant
impact of new TNMG stages on survival after resection for pancreatic
cancer (p=0.001). In conclusion, we can state that the tumor grade was
confirmed as statistically significant prognostic factor in pancreatic
cancer. Its incorporation into the current TNM classification enables
more accurate prognosis prediction within particular clinical stages.
That is why an inclusion of the grade to the standard TNM classification
should be discussed.

PubMed: \url{https://www.ncbi.nlm.nih.gov/pubmed/?term=30064236}

{}

{}

\begin{center}\rule{0.5\linewidth}{\linethickness}\end{center}

 - \textbf{Pathologic Evaluation of Surgical Margins in Pancreatic
Cancer Specimens Using Color Coding With Tissue Marking Dyes}

\emph{Pancreas 2018 Aug;47(7):830-836}

OBJECTIVES: Processing of pancreatoduodenectomy specimens is not
standardized; the clinical impact of pathologic surgical margins remains
controversial. We used the color-coding method using tissue-marking dyes
to evaluate margin status of resected specimens to assess its
association with postoperative recurrence. METHODS: We developed a
unified processing approach to assess pancreatoduodenectomy specimens.
Five surgical margins of resected pancreatic specimens were marked with
5 colors. Microscopic resection margin distance (RMD) from margin
closest to the tumor was evaluated for each surgical margin. Forty
patients assessed using nonunified protocols, and 98 patients assessed
using unified protocols were included. RESULTS: The frequency of tumors
with RMD of 1 mm or less in posterior margin was significantly lower and
that in portal vein/superior mesenteric vein margin was significantly
higher in unified protocol group than in nonunified protocol group (P
\textless{} 0.001). In unified protocol group, tumors with RMD of 1 mm
or less correlated with locoregional recurrence (P = 0.025) and
recurrence-free survival (P = 0.030). Multivariate analysis revealed
that tumor size and lymph node metastasis were independent indicators
for disease recurrence. CONCLUSIONS: Resection margin distance of 1 mm
or less was a predictor for disease recurrence, particularly for
locoregional recurrence. Early detection of small-sized tumors without
lymph node metastasis is necessary for improved clinical outcomes in
pancreas cancers.

PubMed: \url{https://www.ncbi.nlm.nih.gov/pubmed/?term=29975353}

{}

{}

\begin{center}\rule{0.5\linewidth}{\linethickness}\end{center}

 - \textbf{High-grade PanIN presenting with localised stricture of the
main pancreatic duct: A clinicopathological and molecular study of 10
cases suggests a clue for the early detection of pancreatic cancer}

\emph{Histopathology 2018 Aug;73(2):247-258}

AIMS: This study aimed to identify the pathological features of
high-grade PanIN that presents with imaging-detectable abnormalities.
METHODS AND RESULTS: Ten cases of isolated, main-duct, high-grade PanIN
as the primary clinical presentation were identified. All patients
presented with stenosis of the main pancreatic duct, with two being
associated with extensive upstream duct dilatation (\textgreater{}5 mm
in diameter). Pancreatic juice cytology suggested adenocarcinoma in all
seven cases examined. In resected specimens, high-grade PanIN was
present chiefly in the main pancreatic duct, with longitudinal extension
ranging between 3 and 40 mm in length (median = 18 mm). In four cases,
in which hypoechoic or hypovascular masses were observed on imaging,
radiopathology correlations suggested that they represented parenchymal
atrophy and subsequent fibrosis around affected ducts, but not invasive
malignancy. On immunohistochemistry, the loss of p16 expression was
found in five (50\%), p53 overexpression in two (20\%) and loss of SMAD4
expression in none (0\%). KRAS mutations were detected in nine cases,
with two dominant clones being found in three by ultrasensitive droplet
digital polymerase chain reaction, suggesting the genetic heterogeneity
of dysplastic cells composing individual lesions. Mutant GNAS was also
observed in one case. CONCLUSIONS: Isolated high-grade PanIN may present
with pancreatic duct stenosis. Therefore, intensive investigations
including pancreatic juice cytology will be required for patients with
unexplained pancreatic duct stenosis. The abnormal expression of p53 and
SMAD4 is infrequent, while GNAS may be mutated in premalignant lesions
mainly affecting the main pancreatic duct, similar to KRAS.

PubMed: \url{https://www.ncbi.nlm.nih.gov/pubmed/?term=29660164}

{}

{}

\begin{center}\rule{0.5\linewidth}{\linethickness}\end{center}

 - \textbf{The ``T'' now Matters: The Eighth Edition of the Union for
International Cancer Control Classification of Pancreatic
Adenocarcinoma}

\emph{Annals of surgery 2018 Aug;268(2):e36-e37}

PubMed: \url{https://www.ncbi.nlm.nih.gov/pubmed/?term=28938271}

{}

{}

\begin{center}\rule{0.5\linewidth}{\linethickness}\end{center}

 - \textbf{Significance of microcystic, elongated, and fragmented
glandular-like features in intraductal papillary mucinous neoplasm of
the pancreas}

\emph{Human pathology 2018 Aug;78():18-27}

Microcystic, elongated, and fragmented (MELF) glandular features are
associated with epithelial-mesenchymal transition, invasion, and
progression in endometrioid adenocarcinoma of the uterus. Similar
histological features are also observed at the periphery of pancreatic
intraductal papillary mucinous neoplasms (IPMNs). However, the
clinicopathological significance of MELF-like features-particularly
whether they represent regenerative or truly neoplastic conditions-in
IPMNs remains unclear. We assessed a total of 152 surgically resected
IPMNs. Fifty cases exhibited MELF-like features, including 26 cases of
IPMNs with accompanying adenocarcinomas and 24 cases of IPMNs without
accompanying adenocarcinomas. MELF-like features were more frequently
observed in IPMN cases with accompanying adenocarcinomas, larger tumors,
main-duct type, and non-gastric histologic subtype. A positive
correlation between the presence of MELF-like features and high-grade
dysplasia was observed in IPMNs without accompanying adenocarcinomas.
Moreover, DPC4 loss and p53 overexpression in MELF-like glands were more
commonly observed in IPMNs with high-grade dysplasia. IPMN patients with
MELF-like features had worse overall and disease-specific survival by
univariate analyses. Our observations suggest that MELF-like features in
some IPMNs with high-grade dysplasia could be related to stromal
invasion. Hence, when MELF-like features are observed in IPMNs,
pathologists should carefully evaluate the results of microscopic
examinations to identify the invasive components; and,
immunohistochemical staining for DPC4 and p53 could help clarify its
clinicopathological significance.

PubMed: \url{https://www.ncbi.nlm.nih.gov/pubmed/?term=29410139}

{}

{}

\begin{center}\rule{0.5\linewidth}{\linethickness}\end{center}

 - \textbf{Immune Cell and Stromal Signature Associated With
Progression-free Survival of Patients With Resected Pancreatic Ductal
Adenocarcinoma}

\emph{Gastroenterology 2018 Aug;():}

BACKGROUND \& AIMS: Changes to the microenvironment of~pancreatic ductal
adenocarcinomas (PDACs) have been associated with poor outcomes of
patients. We studied the associations between composition of the
pancreatic stroma (fibrogenic, inert, dormant, or fibrolytic stroma) and
infiltration by inflammatory cells and times of progression-free
survival (PFS) of patients with PDACs after resection. METHODS: We
obtained 1824 tissue microarray specimens from 385 patients included in
the European Study Group for Pancreatic Cancer trial 1 and 3 and
performed immunohistochemistry to detect alpha smooth muscle actin, type
1 collagen, CD3, CD4, CD8, CD68, CD206, and neutrophils. Tumors that
expressed high and~low levels of these markers were compared with
patient outcomes using Kaplan-Meier curves and multivariable recursive
partitioning for discrete-time survival tree analysis. Prognostic index
was delineated by a multivariable Cox proportional hazards model of
immune cell and stromal markers and PFS. Findings were validated using
279 tissue microarray specimens from 93 patients in a separate cohort.
RESULTS: Levels of CD3, CD4, CD8, CD68, and CD206 were independently
associated with tumor recurrence. Recursive partitioning for
discrete-time survival tree analysis identified a high level of CD3 as
the strongest independent predictor for longer PFS. Tumors with levels
of CD3 and high levels of CD206 associated with a median PFS time of
16.6 months and a median prognostic index of -0.32 (95\% confidence
interval {[}CI{]} -0.35 to -0.31), whereas tumors with low level of CD3
cell and low level of CD8 and high level of CD68 associated with a
median PFS time of 7.9 months and a prognostic index of 0.32 (95\% CI
0.050-0.32); we called these patterns histologic signatures. Stroma
composition, when unassociated with inflammatory cell~markers, did not
associate significantly with PFS. In the~validation cohort, the
histologic signature resulted in an error matrix accuracy of predicted
response of 0.75 (95\% CI 0.64-0.83; accuracy P \textless{} .001).
CONCLUSIONS: In an analysis of PDAC tissue microarray specimens, we
identified and validated a histologic signature, based on leukocyte and
stromal factors, that associates with PFS times of patients with
resected PDACs. Immune cells might affect the composition of the
pancreatic stroma to affect progression of PDAC. These findings provide
new insights into the immune response to PDAC.

PubMed: \url{https://www.ncbi.nlm.nih.gov/pubmed/?term=30092175}

{}

{}

\begin{center}\rule{0.5\linewidth}{\linethickness}\end{center}

 - \textbf{Tumour origin and R1 rates in pancreatic resections: towards
consilience in pathology reporting}

\emph{Virchows Archiv : an international journal of pathology 2018
Sep;473(3):293-303}

To evaluate differences in the R1 rates of ampullary (AC), pancreatic
(PC), and distal bile duct (DBD) cancers in pancreatoduodenectomies (PD)
using standardised pathology assessment. Data of PD (2010-2011) analysed
in accordance with the Royal College of Pathologists (UK) protocol, were
retrieved. Clinicopathologic features, including frequency, topography,
and mode of margin involvement in AC (n = 87), PC (n = 18), and DBD
(n = 5) cancers were evaluated. The R1 rate was 7\%, 67\%, and 20\% in
the AC, PC, and DBD cancers (p \textless{} 0.001). Within the PC cohort,
R1 rate was heterogeneous (chemo-naïve, 77\%; post-neoadjuvant, 40\%).
Commonest involved margins were as follows: posterior in overall PD
(35\%), AC (43\%), overall PC (33\%), and post-neoadjuvant PC (100\%);
superior mesenteric artery margin in chemo-naïve PC (38\%) and common
bile duct margin in DBD (100\%) cancers. In AC, majority (66\%) of R1
were signet ring cell type. Indirect margin involvement due to tumour
within lymph node, perineural sheath or lymphovascular space was
observed in 26\% cases, and altered R1 rate in AC, PC, and DBD cohorts
by 1\%, 12\%, and 0\%, respectively. Although not statistically
significant, patients with R1 had lower disease-free survival than those
with R0 (mean, 25.4~months versus 44.4~months). Tumour origin impacts R1
data in PD necessitating its accurate classification by pathologists.
Indirect involvement, histology, and neoadjuvant therapy influence the
R1 rate, albeit in a minority of cases. Generating cogent R1 data based
on standardised pathology reporting is the foremost need of the hour.

PubMed: \url{https://www.ncbi.nlm.nih.gov/pubmed/?term=30091124}

\url{https://link.springer.com/article/10.1007/s00428-018-2429-7}

{}

{}

\begin{center}\rule{0.5\linewidth}{\linethickness}\end{center}

 - \textbf{From somatic mutation to early detection: Insights from
molecular characterization of pancreatic cancer precursor lesions}

\emph{The Journal of pathology 2018 Aug;():}

Pancreatic cancer arises from non-invasive precursor lesions, including
pan creatic intraepithelial neoplasia (PanIN), intraductal papillary
mucinous neoplasm (IPMN) and mucinous cystic neoplasm (MCN), which are
curable if detected early enough. Recently, these types of precursor
lesions have been extensively characterized at the molecular level,
defining the timing of critical genetic alterations in tumorigenesis
pathways. The results of these studies deepen our understanding of
tumorigenesis in the pancreas, providing novel insights into tumor
initiation and progression. Perhaps more importantly, they also provide
a rational foundation for early detection approaches that could allow
clinical intervention prior to malignant transformation. In this review,
we summarize the results of comprehensive molecular characterization of
PanINs, IPMNs, and MCNs, and discuss the implications for cancer biology
as well as early detection. This article is protected by copyright. All
rights reserved.

PubMed: \url{https://www.ncbi.nlm.nih.gov/pubmed/?term=30105857}

{}

{}

\begin{center}\rule{0.5\linewidth}{\linethickness}\end{center}

 - \textbf{Multinational validation of the American Joint Committee on
Cancer 8th edition pancreatic cancer staging system in a pancreas head
cancer cohort}

\emph{Journal of hepato-biliary-pancreatic sciences 2018
Sep;25(9):418-427}

BACKGROUND: The aim of the present study was to compare the 7th and 8th
editions of the American Joint Committee on Cancer (AJCC) staging system
for pancreas head cancer and to validate the 8th edition using three
multinational tertiary center data. METHODS: Data of 2,864 patients with
pancreas head cancer were collected from Korea (571), Japan (824), and
the USA (1,469). Survival analysis was performed to compare the 7th and
8th editions. Validation was performed by log-rank tests and test for
trend repeated 1,000 times with random sets. RESULTS: In the 7th
edition, 4.1\%, 3.1\%, 18.6\%, 67.5\%, 3.6\%, and 3.1\% were stage IA,
IB, IIA, IIB, III, and IV. In the 8th edition, 8.8\%, 13.9\%, 3.1\%,
38.2\%, 32.9\%, and 3.1\% were stage IA, IB, IIA, IIB, III, and IV,
respectively. The change in T category downstaged 459 patients from IIA
to the new IA and IB. The new N2 category upstaged 856 patients from the
former IIB to III. The 7th edition reversely stratified IA and IB. The
8th edition corrected this mis-stratification of the 7th edition, but
lacked discriminatory power between IB and IIA (P = 0.271). Validation
using the log-rank showed that the 8th edition provided better
discrimination in 6.387 test sets among 10 tests. The test for trend
validated the 8th edition to stratify stages in correct order more often
(7.815/10). CONCLUSION: The 8th edition provides more even distribution
with more powerful discrimination compared to the 7th edition.

PubMed: \url{https://www.ncbi.nlm.nih.gov/pubmed/?term=30118171}

{}

{}

\begin{center}\rule{0.5\linewidth}{\linethickness}\end{center}

 - \textbf{Solid-pseudopapillary neoplasms of the pancreas do not
express major pancreatic markers in pediatric patients}

\emph{Human pathology 2018 Aug;():}

Solid pseudopapillary neoplasms of the pancreas (SPN) are classified as
``exocrine'' pancreatic tumors by the World Health Organization.
However, despite numerous studies using immunohistochemistry, electron
microscopy, animal models and molecular biology, the histogenesis of SPN
remains unclear. At the same time, our knowledge of human pancreas
development has significantly increased. It is now well known that the
undifferentiated PDX1+ pancreatic progenitors proliferate and
differentiate into endocrine, ductal, and acinar cells, thanks to the
expression of numerous transcription factors, which can be used to
better characterize pancreatic tumors. In a series of 14 pediatric SPN,
we investigated the expression of four transcription factors associated
with pancreatic development (PDX1, SOX9, PTF1A and NKX2.2) to obtain new
insights into the pathogenesis of SPN. In addition, we tested the
expression of different markers of epithelial, endocrine, exocrine, and
neural differentiation, using both immunohistochemical and
immunofluorescence analyses. All tumors displayed the typical
histological features of SPN, with both pseudopapillary and solid
patterns. The immunoprofile was characterized by immunoreactivity for
ß-catenin (100\%), progesterone receptor (100\%), cyclin D1 (100\%),
synaptophysin (65\%) and S100 (15\%). In all cases, tumor cells were
negative for the following markers: PDX1, SOX9, PTF1A, NKX2.2,
chromogranin A, glucagon, insulin, somatostatin, ghrelin, pancreatic
polypeptide, amylase, GFAP, calretinin, EPCAM and estrogen receptor α.
To conclude, SPN do not express major transcription factors involved in
pancreatic development and differentiation, which does not allow to
precise pancreatic lineage of tumor cells. Thus additional studies are
still required to determine origin of SPN.

PubMed: \url{https://www.ncbi.nlm.nih.gov/pubmed/?term=30130629}

{}

{}

\begin{center}\rule{0.5\linewidth}{\linethickness}\end{center}

 - \textbf{CD200 expression is a feature of solid pseudopapillary
neoplasms of the pancreas}

\emph{Virchows Archiv : an international journal of pathology 2018
Aug;():}

CD200 has been recently indicated as a robust marker of
well-differentiated neuroendocrine neoplasms. Here, we evaluate its role
in differential diagnosis of solid pancreatic neoplasms. We
immunostained for CD200 22 solid pseudopapillary neoplasms (SPNs), 8
acinar carcinomas (ACs), 2 pancreatoblastomas (PBs), 138 neuroendocrine
tumors (PanNETs), and 48 ductal adenocarcinomas. All SPNs showed strong
cytoplasmic and membranous staining for CD200, while only one case of AC
had focal positivity. The two PBs showed focal CD200 positivity, mainly
located in squamoid nests. The vast majority of PanNETs (96\%) showed
strong cytoplasmic and membranous staining for CD200, whereas all PDACs
were negative. As both PanNETs and SPNs express CD200, it has no role in
the differential diagnosis between these two entities.

PubMed: \url{https://www.ncbi.nlm.nih.gov/pubmed/?term=30132130}

{}

{}

\begin{center}\rule{0.5\linewidth}{\linethickness}\end{center}

 - \textbf{``Pancreatic Mucoepidermoid Carcinoma'' Is not a Pancreatic
Counterpart of CRTC1/3-MAML2 Fusion Gene-related Mucoepidermoid
Carcinoma of the Salivary Gland, and May More Appropriately be Termed
Pancreatic Adenosquamous Carcinoma With Mucoepidermoid Carcinoma-like
Features}

\emph{The American journal of surgical pathology 2018
Nov;42(11):1419-1428}

``Mucoepidermoid carcinoma (MEC)'' has been accepted as a synonym for
pancreatic adenosquamous carcinoma (ASC). Pancreatic ASC can show
salivary gland-type MEC-like morphology. CRTC1/3-MAML2 fusion gene is a
characteristic molecular feature of MEC of the salivary gland. We
conducted this study to clarify whether the pancreatic ASC with salivary
gland-type MEC-like morphology (Pan-MEC) is a pancreatic counterpart of
salivary gland-type MEC (Sal-MEC). We retrospectively analyzed 37
pancreatic ASCs including 16 Pan-MECs and 21 tumors without MEC-like
features (ASC-NOS {[}not otherwise specified{]}), and we investigated
(1) clinicopathologic features, (2) the presence of CRTC1/3-MAML2 fusion
gene by reverse transcription polymerase chain reaction, (3) the
presence of rearrangement of MAML2 gene by fluorescence in situ
hybridization, and (4) mucin core proteins by immunohistochemistry. We
also compared 16 Pan-MECs with 20 Sal-MECs by immunohistochemistry for
mucin core protein. There were no significant differences of any
clinicopathologic characteristics and survival analysis between the
Pan-MECs and ASCs-NOS. Of note, the pancreatic ASCs (including Pan-MEC
and ASC-NOS) were significantly more aggressive than conventional
pancreatic ductal adenocarcinoma. In addition, all Pan-MECs were
histologically high-grade. CRTC1/3-MAML2 fusion gene and MAML2 gene
rearrangement were not detected in any ASCs including Pan-MECs. There
were significant differences of MUC5AC and MUC6 between the Pan-MECs and
Sal-MECs, but no significant differences of mucin core protein between
the Pan-MECs and pancreatic ASCs-NOS. Pan-MEC is histologically and
biologically high-grade and unrelated to CRTC1/3-MAML2 fusion gene,
unlike Sal-MEC which is related to CRTC1/3-MAML2 fusion gene. Pan-MEC is
not a pancreatic counterpart of CRTC1/3-MAML2 fusion gene-related
Sal-MEC.

PubMed: \url{https://www.ncbi.nlm.nih.gov/pubmed/?term=30138216}

{}

{}

\begin{center}\rule{0.5\linewidth}{\linethickness}\end{center}

 - \textbf{Tumor-Infiltrating Platelets Predict Postsurgical Survival in
Patients with Pancreatic Ductal Adenocarcinoma}

\emph{Annals of surgical oncology 2018 Aug;():}

BACKGROUND: Platelets are believed to promote tumor growth and
metastasis in several tumor types. The prognostic role of blood
platelets in pancreatic ductal adenocarcinoma (PDAC) remains
controversial, and the prognostic value of tumor-infiltrating platelets
(TIPs) remains unknown. METHODS: A total of 303 patients who underwent
curative pancreatectomy for PDAC were enrolled from two independent
centers in China and divided into three cohorts. Paired preoperative
blood samples and surgical specimens from all patients were analyzed.
The correlations between patient outcomes and preoperative blood
platelet counts and the presence of TIPs, respectively, were analyzed.
TIPs were identified by immunohistochemical staining of CD42b.
Prognostic accuracy was estimated by concordance index (C-index) and
Akaike information criterion (AIC). RESULTS: TIPs, but not preoperative
blood platelet counts, were associated with overall survival (OS; all
P \textless{} 0.001) and recurrence-free survival (RFS; all
P \textless{} 0.001) in the training, testing, and validation sets.
Positive CD42b expression predicted poor postsurgical survival.
Incorporation of TIPs improved the predictive accuracy of the 8th
edition American Joint Committee on Cancer (AJCC) tumor-node-metastasis
(TNM) staging system for OS in each of the three cohorts (C-index:
0.7164, 0.7569, and 0.7050, respectively; AIC: 472, 386, and 1019,
respectively). The new predictor system was validated by incorporating
TIPs with the 7th edition AJCC TNM staging system (C-index: 0.7052,
0.7623, and 0.7157; AIC: 476, 386, and 1015). CONCLUSION: TIPs were an
independent prognostic factor that could be incorporated into the AJCC
TNM staging system to refine risk stratification and predict surgical
outcomes of patients with PDAC.

PubMed: \url{https://www.ncbi.nlm.nih.gov/pubmed/?term=30171511}

{}

{}

\begin{center}\rule{0.5\linewidth}{\linethickness}\end{center}

 - \textbf{Ki67 and P53 in Relation to Disease Progression in Metastatic
Pancreatic Cancer: a Single Institution Analysis}

\emph{Pathology oncology research : POR 2018 Sep;():}

We investigated the expression patterns of Ki67 and p53 in metastatic
pancreatic adenocarcinomas and analyzed their relationship with disease
progression-free survival (PFS) and overall survival (OS) in the overall
study population and in patients treated with a gemcitabine-containing
chemotherapy versus FOLFIRINOX chemotherapy. Patients with
histologically confirmed stage IV adenocarcinoma of the pancreas treated
at AUBMC were included after obtaining institutional review board
approval (IRB ID: IM.ST.05). The ROC was plotted to identify the
threshold Ki-67, p53 and CA19-9 value for disease progression, the
identified value was further used in Kaplan Meier curves to compare PFS
for both groups (gemcitabine versus FOLFIRINOX). A value of
p \textless{} 0.05 was considered significant in all analyses. On
univariate analysis, patients who had a Ki-67 \textgreater{} 12.5\% or a
p53 \textgreater{} 15\% had significantly shorter PFS (p = 0.034 and
p = 0.016, respectively). This effect was restricted to Gemcitabine or
gemcitabine-combination treated patients. A decrease in CA19-9 levels
6-8~weeks after chemotherapy of \textgreater{}58\% had significantly
longer PFS (p = 0.027). On multivariate analysis after controlling for
grade, age and P53, Ki-67 remained significant, for every one unit
increase in Ki-67 the progression risk increases by 1.017 times. Our
study highlights the negative impact of high P53 expression and Ki67
proliferation index on PFS in patients with metastatic pancreatic
cancer.

PubMed: \url{https://www.ncbi.nlm.nih.gov/pubmed/?term=30187215}

{}

{}

\begin{center}\rule{0.5\linewidth}{\linethickness}\end{center}

\begin{itemize}
\tightlist
\item
  \textbf{Stromal biology and therapy in pancreatic cancer: ready for
  clinical translation?}
\end{itemize}

\url{https://gut.bmj.com/content/early/2018/09/03/gutjnl-2018-316451}

\begin{center}\rule{0.5\linewidth}{\linethickness}\end{center}

 - \textbf{Clinical Implications of Extensive Lymph Node Metastases for
Resected Pancreatic Cancer}

\emph{Annals of surgical oncology 2018 Sep;():}

BACKGROUND: Outcomes of patients with resected pancreatic ductal
adenocarcinoma (PDA) and extensive lymph node metastases have not been
fully characterized. METHODS: A total of 637 patients underwent
resection for pancreatic ductal adenocarcinoma (PDA) between 2002 and
2014 at the Thomas Jefferson University. Positive lymph node count (LNC)
and positive lymph node ratio (LNR) were analyzed as predictors of
cancer-specific outcomes, with a focus on outcomes of patients with
extensive lymph node burden. RESULTS: Resected patients with regional
lymph node metastases had a median survival of 17.1~months (n = 425,
70\%) compared with 25.5~months (n = 185, 30\%) for patients without
lymph node spread (N0) (hazard ratio {[}HR{]} = 1.9,
p \textless{} 0.001). Overall survival decremented with increased lymph
node spread, but plateaued for LNC ≥ 4 (HR 2.4 vs.~N0,
p \textless{} 0.001) and LNR ≥ 0.4 (HR 2.2, p \textless{} 0.001).
Compared with historical cohorts with macroscopic metastatic disease, as
opposed to microscopic, superior long-term survival was achieved in
patients with extensive lymph node metastases (LNC ≥ 4); 24- and
36-month survivals were 25\% (vs.~16\%, p \textless{} 0.001) and 12\%
(vs.~6\%, p \textless{} 0.001), respectively. Extensive lymph node
burden was associated with increased baseline postoperative serum CA
19-9 (p = 0.044) and systemic recurrence (p \textless{} 0.001).
CONCLUSIONS: The prognostic impact of extensive lymph node spread after
resection for PDA plateaus above a specific threshold (LNC ≥ 4 or
LNR ≥ 0.4), supporting the new 8th edition AJCC criteria for N2 disease.
Clinically, lymph node spread above this threshold seems to correlate
with occult systemic disease (elevated postoperative CA 19-9 and
systemic pattern of failure).

PubMed: \url{https://www.ncbi.nlm.nih.gov/pubmed/?term=30225835}

{}

{}

\begin{center}\rule{0.5\linewidth}{\linethickness}\end{center}

\begin{itemize}
\tightlist
\item
  \textbf{Clinical Implications of Extensive Lymph Node Metastases for
  Resected Pancreatic Cancer}
\end{itemize}

\url{https://link.springer.com/article/10.1245/s10434-018-6763-4}

\url{https://www.researchgate.net/publication/327713995_Clinical_Implications_of_Extensive_Lymph_Node_Metastases_for_Resected_Pancreatic_Cancer}

\begin{center}\rule{0.5\linewidth}{\linethickness}\end{center}

\begin{itemize}
\tightlist
\item
  \textbf{Serous Neoplasms of the Pancreas: A Comprehensive Review}
\end{itemize}

\url{http://www.archivesofpathology.org/doi/full/10.5858/arpa.2017-0195-RS?code=coap-site}

\begin{center}\rule{0.5\linewidth}{\linethickness}\end{center}

 - \textbf{The Prognostic Significance of Resection Margins After
Pancreaticoduodenectomy}

\emph{Annals of surgical oncology 2018 Sep;():}

PubMed: \url{https://www.ncbi.nlm.nih.gov/pubmed/?term=30264255}

{}

{}

\begin{center}\rule{0.5\linewidth}{\linethickness}\end{center}

\begin{center}\rule{0.5\linewidth}{\linethickness}\end{center}

 - \textbf{Does Surgical Margin Impact Recurrence in Noninvasive
Intraductal Papillary Mucinous Neoplasms?: A Multi-institutional Study}

\emph{Annals of surgery 2018 Sep;268(3):469-478}

OBJECTIVE: The relevance of margin positivity on recurrence after
resection of intraductal papillary mucinous neoplasms (IPMNs) is poorly
defined and represents one reason controversy remains regarding optimal
surveillance recommendations. METHODS: Patients undergoing surgery for
noninvasive IPMN at 8 academic medical centers from the Central Pancreas
Consortium were analyzed. A positive margin was defined as presence of
IPMN or pancreatic intraepithelial neoplasia. RESULTS: Five hundred two
patients underwent surgery for IPMN; 330 (66\%) did not have invasive
cancer on final pathology and form the study cohort. Of these, 20\%
harbored high grade dysplasia. A positive margin was found in 20\% of
cases and was associated with multifocal disease (P = 0.02). The
majority of positive margins were associated with low grade dysplasia.
At a median follow-up of 36 months, 34 (10.3\%) patients recurred, with
6.7\% developing recurrent cystic disease and 3.6\% developing invasive
cancer. On multivariate analysis, margin positivity was not associated
with recurrence of either IPMN or invasive cancer (P \textgreater{}
0.05). No association between margin status and development of
recurrence at the margin was found. Only 6\% of recurrences developed at
the resection margin and median time to recurrence was 22 months. Of
note, 18\% of recurrences occurred \textgreater{} 5 years following
surgery. CONCLUSION: Margin positivity after resection for noninvasive
IPMNs is primarily due to low grade dysplasia and is not associated with
developing recurrence in the remnant pancreas or at the resection
margin. Long-term surveillance is required for all patients, as a
significant number of recurrences developed over 5 years after the index
operation.

PubMed: \url{https://www.ncbi.nlm.nih.gov/pubmed/?term=30063495}

\url{https://journals.lww.com/annalsofsurgery/Abstract/2018/09000/Does_Surgical_Margin_Impact_Recurrence_in.10.aspx}

{}

{}

\begin{center}\rule{0.5\linewidth}{\linethickness}\end{center}

\begin{itemize}
\tightlist
\item
  \textbf{Transmembrane mucin MUC13 distinguishes intraductal papillary
  mucinous neoplasms from non-mucinous cysts and is associated with
  high-risk lesions}
\end{itemize}

\url{https://www.sciencedirect.com/science/article/pii/S1365182X18326947}

\begin{center}\rule{0.5\linewidth}{\linethickness}\end{center}

\begin{itemize}
\tightlist
\item
  \textbf{Importance of main pancreatic duct dilatation in IPMN
  undergoing surveillance}
\end{itemize}

\url{https://onlinelibrary.wiley.com/doi/abs/10.1002/bjs.10948}

\begin{center}\rule{0.5\linewidth}{\linethickness}\end{center}

 - \textbf{Comparison of the Survival Outcomes of Pancreatic Cancer and
Intraductal Papillary Mucinous Neoplasms}

\emph{Pancreas 2018 Sep;47(8):974-979}

OBJECTIVES: The aims of the study were to compare survival outcomes
between patients with pancreatic ductal adenocarcinoma (PDAC) and
invasive intraductal papillary mucinous neoplasms (IPMN) and to
determine candidates for adjuvant chemotherapy. METHODS: A total of 579
consecutive patients, including 375 PDAC and 204 IPMN patients, were
reviewed. Stage-matched comparisons of survival data were conducted
using the Cox proportional hazards model and propensity analysis. To
evaluate prognostic factors, univariate and multivariate Cox regression
analyses were performed. RESULTS: The overall survival for invasive IPMN
was significantly longer than that for PDAC (hazard ratio, 2.34; P =
0.0001). When the analysis was limited to stage I patients, the 5-year
overall survival rate of invasive IPMN patients was significantly better
than that of PDAC patients (100\% vs 74.1\%, P = 0.0092); however, no
difference was observed between stage II patients with invasive IPMN and
PDAC (hazard ratio, 1.49; P = 0.09). The Cox proportional hazards model
and propensity analysis demonstrated no difference in stage-matched
survival. Multivariate analysis revealed that only T (≥3) was an
independent prognostic factor for invasive IPMN. CONCLUSIONS:
Stage-matched analysis did not show a significant survival difference
between invasive IPMN and PDAC patients, and T3 or higher was an
independent prognostic factor for invasive IPMN.

PubMed: \url{https://www.ncbi.nlm.nih.gov/pubmed/?term=30028445}

{}

{}

\begin{center}\rule{0.5\linewidth}{\linethickness}\end{center}

\hypertarget{bile-ducts}{%
\subsubsection{Bile Ducts}\label{bile-ducts}}

\begin{center}\rule{0.5\linewidth}{\linethickness}\end{center}

\begin{itemize}
\tightlist
\item
  \textbf{Data set for the reporting of intrahepatic cholangiocarcinoma,
  perihilar cholangiocarcinoma and hepatocellular carcinoma:
  recommendations from the International Collaboration on Cancer
  Reporting (ICCR)}
\end{itemize}

\url{https://onlinelibrary.wiley.com/doi/abs/10.1111/his.13520?campaign=woletoc}

\begin{center}\rule{0.5\linewidth}{\linethickness}\end{center}

 - \textbf{Interleukin-33 overexpression reflects less aggressive tumour
features in large-duct type cholangiocarcinomas}

\emph{Histopathology 2018 Aug;73(2):259-272}

AIMS: The aim of the present study was to elucidate the
clinicopathological significance of interleukin (IL)-6 and IL-33
expression in intrahepatic cholangiocarcinomas (iCCAs) and perihilar
cholangiocarcinomas (pCCAs). METHODS AND RESULTS: IL-6 and IL-33 mRNA
expression levels were examined in iCCAs (n = 55) and pCCAs (n = 32) by
the use of quantitative real-time polymerase chain reaction and a highly
sensitive in-situ hybridisation protocol (RNAscope), and expression
levels were correlated with clinicopathological features. According to a
recently proposed classification scheme, iCCAs were separated into
small-duct (n = 33) and large-duct (n = 22) types. IL-6 and IL-33
expression levels were higher in large-duct iCCAs and pCCAs than in
small-duct iCCAs, and there was a positive correlation between the
expression levels of these cytokines. Double in-situ
hybridisation/immunostaining showed that IL-6 mRNA was expressed in
actin-positive (myo)fibroblasts, whereas IL-33 mRNA was mainly produced
by CD31-positive endothelial cells. With the average expression level as
a cut-off point, cases were classified as IL-6high and IL-6low or
IL-33high and IL-33low . In the combined cohort of large-duct iCCAs and
pCCAs, IL-6high and IL-6low cholangiocarcinomas shared many features,
whereas IL-33high cases had less aggressive characteristics than
IL-33low cases, as shown by lower tumour marker concentrations, smaller
tumour sizes, less common vascular invasion, lower pT stages, and higher
lymphocyte/monocyte ratios in blood. KRAS mutations were slightly less
common in IL-33high cases than in IL-33low cases (9\% versus 29\%; P =
0.061). The strong expression of IL-33 in tissue appeared to be an
independent favourable prognostic factor. CONCLUSIONS: IL-33high
cholangiocarcinomas may represent a unique, less aggressive
carcinogenetic process of the large bile ducts.

PubMed: \url{https://www.ncbi.nlm.nih.gov/pubmed/?term=29675965}

doi: \url{https://doi.org/10.1111/his.13633}

{}

{}

\begin{center}\rule{0.5\linewidth}{\linethickness}\end{center}

 - \textbf{Postradiation Synovial Sarcoma of the Common Bile Duct: A
Previously Unreported Anatomic Site}\\
 \emph{International journal of surgical pathology 2018
Aug;26(5):469-474}

Synovial sarcoma is a ubiquitous neoplasm predominantly affecting soft
tissues of young adults of any gender; few cases have been described in
the digestive system, mostly in the stomach. The (X;18)(p11.2; q11.2)
translocation yields unique SS18-SSX fusion genes. Synovial sarcoma has
been related to radiotherapy, but no synovial sarcoma has been
associated with the digestive system. This article describes the case of
a synovial sarcoma arising along the extrahepatic biliary tree, 10 years
after the application of an abdominal radiotherapy schedule due to a
retroperitoneal metastatic seminoma in a male who developed progressive
obstructive jaundice. Ninety percent of the analyzed cells carried the
SS18 gene with separation of sequences, thus denoting a translocation.
There are only 8 post-radiotherapy synovial sarcomas that have been
reported previously, and this is the first report of a
radiotherapy-related synovial sarcoma arising from the extrahepatic
biliary tree, and the second case described in this anatomic region.

PubMed: \url{https://www.ncbi.nlm.nih.gov/pubmed/?term=29336183}

doi: \url{https://doi.org/10.1177/1066896917752863}

{}

{}

\begin{center}\rule{0.5\linewidth}{\linethickness}\end{center}

 - \textbf{Mucinous intrahepatic cholangiocarcinoma: a distinct variant}

\emph{Human pathology 2018 Aug;78():131-137}

Mucinous variant of intrahepatic cholangiocarcinoma (iCC) is rare, and
its clinicopathological features and prognosis are far less clear. Six
patients who had iCCs with more than 50\% of mucinous component and 79
conventional iCCs were included in the study. The mean size of mucinous
and conventional iCCs was 6.2 and 6.0\,cm, respectively. Most patients
(83\%) with mucinous iCC presented at T3 stage or above compared with
28\% of the conventional group (P\,\textless{}\,.01). Three patients
with mucinous iCC (50\%) died within 1\,year. The average survival time
of patients with mucinous iCCs was significantly reduced compared with
that of the conventional group (9\,months versus 2\,years;
P\,\textless{}\,.001). Immunohistochemistry was performed on 6 mucinous
and 12 conventional iCCs with matched age, sex, and stage, which
revealed positive immunoreactivity in MUC1 (83\% versus 58\%), MUC2
(33\% versus 17\%), MUC5AC (100\% versus 42\%), MUC6 (50\% versus 0),
CK7 (83\% versus 83\%), CK20 (0 versus 17\%), CDX2 (17\% versus 0), p53
(67\% versus 67\%), Smad4 (67\% versus 58\%), and EGFR (83\% versus
42\%) in mucinous and conventional iCCs, respectively. Molecular studies
showed one mucinous iCC with KRAS G12C mutation and no BRAF or IDH1/2
mutations. Mucinous iCC is a unique variant that constitutes 7\% of
iCCs. It is more immunoreactive for MUC1, MUC2, MUC5AC, and MUC6. Unlike
adenocarcinomas of colorectal primary, mucinous iCCs are often
CK7+/CK20-/CDX2- and microsatellite stable. Patients with mucinous iCC
likely present at advanced stage upon diagnosis with shorter survival
time compared with the conventional counterparts.

PubMed: \url{https://www.ncbi.nlm.nih.gov/pubmed/?term=29698701}

{}

{}

\begin{center}\rule{0.5\linewidth}{\linethickness}\end{center}

\hypertarget{gallbladder}{%
\subsubsection{Gallbladder}\label{gallbladder}}

\begin{center}\rule{0.5\linewidth}{\linethickness}\end{center}

 - \textbf{Gallbladder carcinoma and epithelial dysplasia: Appropriate
sampling for histopathology}

\emph{Annals of diagnostic pathology 2018 Sep;37():7-11}

Gallbladder carcinoma (GC) is an uncommon malignancy with an overall
5-year survival of \textless{}5\%. Due to overlap of clinical
presentation with the more common cholecystitis, an estimated 50-65\% of
all GCs are found incidentally. Epithelial dysplasia is identified in
\textasciitilde{}50\% of specimens with invasive carcinoma. Recent
expert panel guidelines have recommended histologic examination of the
entire gallbladder in cases where initial sampling reveals dysplasia. 89
cases of GC, 34 high grade dysplasia (HGD), and 60 low grade dysplasia
(LGD) were identified in cholecystectomy specimens assessed at our
institution over the last 15\,years. Pre-operative imaging (either
ultrasound or CT) only identified 52\% of mass lesions in GC cases.
Among gallbladder specimens with epithelial dysplasia only at initial
sampling, additional sectioning was performed in 59\% of HGD and 55\% of
LGD. Additional sectioning of gallbladder specimens with HGD had a
higher yield (10\%) for identifying invasive carcinoma than those with
LGD (0 of 28). The diagnostic yield of additional sectioning is
significantly higher in the setting of high grade as compared to low
grade dysplasia, suggesting that sampling at the discretion of the
pathologist may be sufficient for the latter.

PubMed: \url{https://www.ncbi.nlm.nih.gov/pubmed/?term=30216818}

{}

{}

\begin{center}\rule{0.5\linewidth}{\linethickness}\end{center}

\begin{itemize}
\tightlist
\item
  \textbf{Gallbladder carcinoma and epithelial dysplasia: Appropriate
  sampling for histopathology}
\end{itemize}

\url{https://www.sciencedirect.com/science/article/pii/S1092913418301163}

\begin{center}\rule{0.5\linewidth}{\linethickness}\end{center}

\begin{itemize}
\tightlist
\item
  \textbf{Significance of HER2 and Ki-67 in Preneoplastic Lesions and
  Carcinoma of Gallbladder}
\end{itemize}

\url{https://link.springer.com/article/10.1007/s12029-018-0162-8}

\begin{center}\rule{0.5\linewidth}{\linethickness}\end{center}

 - \textbf{Validation of American Joint Committee on Cancer eighth
staging system for gallbladder cancer and its lymphadenectomy
guidelines}

\emph{The Journal of surgical research 2018 Oct;230():148-154}

BACKGROUND: For gallbladder cancer (GBC), the American Joint Committee
on Cancer eighth edition (AJCC 8) staging system classifies lymph node
(LN) stage by the number of positive LN and recommends sampling of ≥6
LNs. We evaluated the prognostic capability of the AJCC 8 for patients
undergoing resection and the current national trends in LN staging in
the context of these new recommendations for nodal (N) sampling.
METHODS: Utilizing the National Cancer Data Base, we identified all
gallbladder adenocarcinoma patients treated with surgical resection in
2004-2014. Cox regression modeling was used to calculate the concordance
index of AJCC 8 in predicting overall survival. N sampling and
positivity rates were analyzed over the study period. RESULTS: In our
cohort, predicted 5-year overall survival by AJCC 8 was: stage I,
62.5\%; II, 50.2\%; IIIA, 25.7\%; IIIB, 22.1\%; IVA, 15.7\%; IVB, 6.7\%
(P~\textless{}~0.01). The concordance index for the staging system was
0.832. Only 50.7\% of the patients had any LN sampling to determine the
N stage. LN sampling rates improved from 45.6\% in 2004 to 55.1\% in
2013 (P~\textless{}~0.001). However, only 24.5\% of patients with any LN
sampling had ≥6 LNs resected (12.4\% of eligible cohort), with a median
LN sample of two. CONCLUSIONS: AJCC 8 offers adequate discrimination for
GBC staging, especially for node-positive patients. With actual GBC LN
sampling rates at 50.7\%, and far short of the ≥6 LN threshold, quality
improvement measures may need to focus on requiring any LN sampling
before raising the minimum to six LNs.

PubMed: \url{https://www.ncbi.nlm.nih.gov/pubmed/?term=30100032}

{}

{}

\begin{center}\rule{0.5\linewidth}{\linethickness}\end{center}

\begin{itemize}
\tightlist
\item
  \textbf{Clinicopathological features and survival of gallbladder
  squamous cell carcinoma: analysis of 121 cases}
\end{itemize}

\url{http://www.ijcep.com/files/ijcep0076184.pdf}

\begin{center}\rule{0.5\linewidth}{\linethickness}\end{center}

 - \textbf{Pyloric Gland Adenoma (PGA) of the Gallbladder: A Unique and
Distinct Tumor from PGAs of the Stomach, Duodenum, and Pancreas}

\emph{The American journal of surgical pathology 2018
Sep;42(9):1237-1245}

Twenty-four surgically resected, gallbladder pyloric gland adenomas
(GB-PGAs) were examined and their features were compared with the
reported features of stomach, duodenum, and pancreatic PGAs to better
understand GB-PGAs. Clinical information on background gallbladder
lesions and histologic data, including tumor grade, existence of
squamoid morules, intratumoral cholesterosis, and intracytoplasmic
mucins were collected. Immunohistochemical staining for MUC2, MUC5AC,
MUC6, CDX2, pepsinogen I, p53, and MIB-1/nuclear β-catenin were
evaluated. Targeted mutational analyses of KRAS exon2, GNAS exon 7, and
CTNNB1 exon 3 were conducted. We found that 29.2\% of the GB-PGAs were
histologically high-grade dysplasias/carcinomas; 70.8\% were low grade;
and 20.8\% and 33.3\% contained squamoid morules and intratumoral
cholesterosis, respectively. In addition, 45.8\% and 54.2\% of GB-PGAs
were mucin-rich and mucin-poor types, respectively.
Immunohistochemically, MUC6 was diffusely positive in all GB-PGAs; MUC2,
MUC5AC, and CDX2 were only focally positive, and no pepsinogen-I
positive cells were observed. Nuclear β-catenin accumulation was
observed in all cases; however, the ratio varied among cases. Mucin-poor
types were significantly associated with high histologic grade
dysplasias/carcinomas and high nuclear β-catenin labeling indices.
Mutational analyses identified CTNNB1 mutations in 100\% of GB-PGAs
(21/21), KRAS in 4.2\% (1/23), and GNAS in 0\% (0/22). The present study
clarified the unique histologic features, phenotypic differentiation,
and molecular statuses frequently associated with GB-PGAs. Altogether,
our data suggest that tumorigenesis of GB-PGA is distinct from that of
stomach, duodenum, and pancreatic PGAs.

PubMed: \url{https://www.ncbi.nlm.nih.gov/pubmed/?term=29975247}

\url{https://journals.lww.com/ajsp/Abstract/2018/09000/Pyloric_Gland_Adenoma__PGA__of_the_Gallbladder__A.12.aspx}

{}

{}

\begin{center}\rule{0.5\linewidth}{\linethickness}\end{center}

 - \textbf{Prognostic validation of the updated 8th edition
Tumor-Node-Metastasis classification by the Union for International
Cancer Control: Survival analyses of 307 patients with surgically
treated gallbladder carcinoma}

\emph{Oncology letters 2018 Oct;16(4):4427-4433}

In December 2016, the Union for International Cancer Control (UICC)
published the 8th edition of the Tumor-Node-Metastasis (TNM)
classification of malignant tumors, including a number of vital changes
in the definitions of the T2 category, the N category and the stages of
gallbladder cancer (GBC). The clinical value of this newly updated
classification in patients with surgically treated GBC has not been
rigorously validated. The present study aimed to analyze the prognosis
of patients with GBC in a high-volume surgical unit, and to validate the
prognostic value of the new UICC TNM classification, particularly the
main changes in the stages of GBC. Data from 307 patients who were
surgically treated and histopathologically diagnosed with GBC between
January 2011 and July 2016 in The West China Hospital (Chengdu, Sichuan,
China) were retrospectively collected and analyzed. The new UICC
criteria distributed 32, 60, 99 and 116 eligible patients in stages I,
II, III and IV, respectively. The differences in overall survival time
between each stage (I-IV) demonstrated statistical significance
(P\textless{}0.05). As a result of the main change of this
classification, the novel definitions of T2a and T2b effectively
stratified the prognosis of patients with T2 GBC (P\textless{}0.001).
Furthermore, patients with stage IIa tumors also obtained significantly
improved overall survival time compared with patients with stage IIb
tumors (P=0.04), whereas the comparison between patients with stage IIb
and IIIa tumors did not present any notable difference (P=0.20).
Additionally, the new N category stratified the survival of the patients
effectively (P\textless{}0.001). Together with curative resection, this
latest classification was indicated to be an independent predictor of
survival via multivariate analysis (hazard ratio, 6.25; 95\% confidence
interval, 3.81-10.26; P\textless{}0.001). In conclusion, the newly
updated UICC TNM classification could effectively reflect the clinical
outcome of patients with surgically treated GBC. Furthermore, tumor
location could predict the survival of surgically treated T2 GBC. The
novel classification of the N category by the number of lymph nodes
involved was also demonstrated to be valid.

PubMed: \url{https://www.ncbi.nlm.nih.gov/pubmed/?term=30214577}

{}

{}

\begin{center}\rule{0.5\linewidth}{\linethickness}\end{center}

\hypertarget{ampulla-of-vater}{%
\subsubsection{Ampulla of Vater}\label{ampulla-of-vater}}

\begin{center}\rule{0.5\linewidth}{\linethickness}\end{center}

 - \textbf{High-grade precursor lesions can be used as surrogate markers
to identify the epicenter of periampullary carcinomas}

\emph{Human pathology 2018 Sep;():}

Identifying the accurate origin of periampullary cancers is important,
because different origins may trigger different clinicopathologic
behaviors. The presence of intraepithelial precursor lesions, including
high-grade pancreatic intraepithelial neoplasias (PanINs) and/or
high-grade biliary intraepithelial neoplasias (BilINs), may be
suggestive of the origin of the periampullary carcinoma in challenging
cases. To prove the usefulness of high-grade intraepithelial precursor
lesions to identify the origin of ambiguous periampullary cancers, the
status and grades of PanINs and BilINs were evaluated in 256
periampullary carcinomas with a well-defined cancer origin as a test
set, including 114 pancreatic cancers, 82 distal bile duct cancers, 54
ampullary cancers, and 6 duodenal cancers, and 112 periampullary
carcinomas with clinically equivocal epicenter either by radiologic
imaging used as a validation set. High-grade PanINs were found more
commonly in pancreatic cancers than in distal bile duct, ampullary, and
duodenal cancers both in test (P=.002) and validation sets
(P\textless{}.001). Similarly, high-grade BilINs were identified more
frequently in distal bile duct cancers than in ampullary, pancreatic,
and duodenal cancers both in test (P\textless{}.001) and validation sets
(P=.039). High-grade PanINs were found most commonly in pancreatic
cancers, while high-grade BilINs were seen most frequently in distal
bile duct cancers. In addition, both high-grade PanINs and high-grade
BilINs are uncommonly noted in ampullary or duodenal cancers. The
recognition of high-grade intraepithelial lesions can help identify the
primary origin of periampullary cancers, especially when the epicenter
of the periampullary cancer is ambiguous.

PubMed: \url{https://www.ncbi.nlm.nih.gov/pubmed/?term=30261192}

{}

{}

\begin{center}\rule{0.5\linewidth}{\linethickness}\end{center}

 - \textbf{Intraobserver and Interobserver Variability in the Assessment
of Dysplasia in Ampullary Mucosal Biopsies}

\emph{The American journal of surgical pathology 2018
Aug;42(8):1095-1100}

Endoscopic mucosal biopsies of the ampulla of Vater (AmpBx) are obtained
to histologically assess for dysplasia or carcinoma. However, biopsy
material is often scant and a host of factors can induce histologic
changes that pose diagnostic challenges. We sought to investigate
observer variability in interpretation of AmpBx and the impact clinical
data may have on diagnostic interpretation. Thirty-one cases from
institutional archives were selected, including 12 cases of reactive
atypia (RA), 8 indefinite for dysplasia (ID), and 11 showing low-grade
dysplasia (LGD). Slides were independently reviewed at 3 time points
with and without clinical information by 6 pathologists who categorized
the biopsies RA, ID, or LGD. Following the reviews, intraobserver and
interobserver agreement was assessed. Review of AmpBx without clinical
data showed fair (κ, 0.27), poor (κ, 0.07), and good (κ, 0.42)
interobserver agreement for diagnoses of RA, ID, and LGD, respectively.
Interobserver agreement improved for LGD (κ, 0.66 and 0.73) when
clinical information was provided; however, agreement remained fair for
RA (κ, 0.4 and 0.42) and poor-to-fair for ID (κ, 0.17 and 0.25). When
follow-up data were reviewed, all cases that reached unanimous agreement
had that diagnosis substantiated by subsequent endoscopic or histologic
findings. The same was true of 13 of 19 cases that reached majority
consensus. Given the potential clinical consequences of these diagnoses
combined with the significant intraobserver and interobserver
variability found in this study, we conclude that better-defined
diagnostic criteria and consensus reads on difficult cases would assist
in the histologic assessment of these challenging cases.

PubMed: \url{https://www.ncbi.nlm.nih.gov/pubmed/?term=29738360}

doi: \url{https://doi.org/10.1097/PAS.0000000000001079}

{}

{}

\begin{center}\rule{0.5\linewidth}{\linethickness}\end{center}

 - \textbf{Clinicopathological features related to survival in
adenocarcinoma of the vaterian system in a Mexican population}

\emph{Human pathology 2018 Sep;():}

Adenocarcinomas of the ampulla of Vater account for 0.5\% of malignant
neoplasms of the gastrointestinal tract, and 6-20\% of malignant
periampullary neoplasms, the majority of patients being candidates to
elective surgery. Our objective was to evaluate the clinicopathological
prognostic factors of ampullary adenocarcinomas after surgical
resection, in the Mexican population. From the records of the Department
of Pathology at the Instituto Nacional de Cancerología, México, cases
diagnosed as adenocarcinomas of the ampulla of Vater were selected over
a period of 11years, from January 2005 to September 2015. Cases with a
pancreaticoduodenectomy report were included, and from each case,
demographic and pathological data of the surgical specimen, were
obtained. Univariate and multivariate statistical analyses were
performed using the Log-rank test and Cox regression.Of 157 cases
diagnosed as ampullary adenocarcinomas, 104 patients were excluded as
not elegible for surgical treatment at the time of diagnosis. In the
remaining 53 patients, a pancreaticoduodenectomy was performed. The mean
age of the entire group was 55.4years and the majority were men.
Intestinal-type adenocarcinomas were more frequent (77.4\%) than
pancreatobiliary-type (15.1\%), most without perineural invasion, well
to moderately differentiated, and less than 3cm in size. Lymph node
metastasis and age over 65years, had a negative impact on overall
survival of the patients.The most convenient classification of malignant
epithelial tumors of the Vaterian system, is according to the
histopathological phenotype into intestinal-, pancreatobiliary and
mixed-type adenocarcinomas, as well as uncommon variants.

PubMed: \url{https://www.ncbi.nlm.nih.gov/pubmed/?term=30179685}

{}

{}

\begin{center}\rule{0.5\linewidth}{\linethickness}\end{center}

\begin{itemize}
\tightlist
\item
  \textbf{Clinicopathological features related to survival in
  adenocarcinoma of the vaterian system in a Mexican population}
\end{itemize}

\url{https://www.sciencedirect.com/science/article/pii/S004681771830340X}

\begin{center}\rule{0.5\linewidth}{\linethickness}\end{center}

\hypertarget{neuroendocrine}{%
\subsubsection{Neuroendocrine}\label{neuroendocrine}}

\begin{center}\rule{0.5\linewidth}{\linethickness}\end{center}

\begin{itemize}
\tightlist
\item
  \textbf{The expression of TTF1, CDX2 and ISL1 in 74 poorly
  differentiated neuroendocrine carcinomas}
\end{itemize}

\url{https://www.sciencedirect.com/science/article/pii/S1092913418302272}

\begin{center}\rule{0.5\linewidth}{\linethickness}\end{center}

\begin{itemize}
\tightlist
\item
  \textbf{Unmet needs in high-grade gastroenteropancreatic
  neuroendocrine neoplasms (WHO G3).}
\end{itemize}

\url{https://www.karger.com/Article/Abstract/493318}

\begin{center}\rule{0.5\linewidth}{\linethickness}\end{center}

 - \textbf{A common classification framework for neuroendocrine
neoplasms: an International Agency for Research on Cancer (IARC) and
World Health Organization (WHO) expert consensus proposal}

\emph{Modern pathology : an official journal of the United States and
Canadian Academy of Pathology, Inc 2018 Aug;():}

The classification of neuroendocrine neoplasms (NENs) differs between
organ systems and currently causes considerable confusion. A uniform
classification framework for NENs at any anatomical location may reduce
inconsistencies and contradictions among the various systems currently
in use. The classification suggested here is intended to allow
pathologists and clinicians to manage their patients with NENs
consistently, while acknowledging organ-specific differences in
classification criteria, tumor biology, and prognostic factors. The
classification suggested is based on a consensus conference held at the
International Agency for Research on Cancer (IARC) in November 2017 and
subsequent discussion with additional experts. The key feature of the
new classification is a distinction between differentiated
neuroendocrine tumors (NETs), also designated carcinoid tumors in some
systems, and poorly differentiated NECs, as they both share common
expression of neuroendocrine markers. This dichotomous morphological
subdivision into NETs and NECs is supported by genetic evidence at
specific anatomic sites as well as clinical, epidemiologic, histologic,
and prognostic differences. In many organ systems, NETs are graded as
G1, G2, or G3 based on mitotic count and/or Ki-67 labeling index, and/or
the presence of necrosis; NECs are considered high grade by definition.
We believe this conceptual approach can form the basis for the next
generation of NEN classifications and will allow more consistent
taxonomy to understand how neoplasms from different organ systems
inter-relate clinically and genetically.

PubMed: \url{https://www.ncbi.nlm.nih.gov/pubmed/?term=30140036}

{}

{}

\begin{center}\rule{0.5\linewidth}{\linethickness}\end{center}

\begin{itemize}
\tightlist
\item
  \textbf{ATRX loss is an independent predictor of poor survival in
  pancreatic neuroendocrine tumours}
\end{itemize}

\url{https://www.sciencedirect.com/science/article/pii/S0046817718303009}

\begin{center}\rule{0.5\linewidth}{\linethickness}\end{center}

 - \textbf{ATRX loss is an independent predictor of poor survival in
pancreatic neuroendocrine tumours}

\emph{Human pathology 2018 Aug;():}

Pancreatic neuroendocrine tumours (PanNETs) are rare neoplasms
accounting for 1-2\% of all pancreatic tumours. The biological behaviour
of PanNETs is heterogeneous and unpredictable, adding to the
difficulties of clinical management. The DAXX (death domain associated
protein) and ATRX (alpha-thalassemia/mental retardation syndrome
X-linked) genes encode proteins involved in SWI/SNF-like chromatin
remodelling. Somatic inactivating mutations in DAXX and ATRX are
frequent in PanNETs, mutually exclusive, and associated with telomere
dysfunction resulting in genomic instability and alternate lengthening
of telomeres. We sought to assess the clinical significance of the loss
of the ATRX and DAXX proteins as determined by immunohistochemistry
(IHC) in patients with PanNET. From an unselected cohort of 105
patients, we found ATRX loss in 10 tumours (9.5\%) and DAXX loss in 16
(15.2\%). DAXX and ATRX loss were confirmed mutually exclusive and
associated with other adverse clinicopathological variables and poor
survival in univariate analysis. In addition ATRX loss was also
associated with higher AJCC stage and infiltrative tumour borders.
However only ATRX loss, lymphovascular invasion and perineural spread
were independent predictors of poor overall survival in multivariate
analysis. In conclusion, loss of expression of ATRX as determined by IHC
is a useful independent predictor of poor overall survival in PanNETs.
Given its relative availability, ATRX loss as determined by IHC may have
a role in routine clinical practice to refine prognostication in
patients with PanNET.

PubMed: \url{https://www.ncbi.nlm.nih.gov/pubmed/?term=30081149}

{}

{}

\begin{center}\rule{0.5\linewidth}{\linethickness}\end{center}

 - \textbf{Recurrence of Pancreatic Neuroendocrine Tumors and Survival
Predicted by Ki67}

\emph{Annals of surgical oncology 2018 Aug;25(8):2467-2474}

BACKGROUND: Despite evidence of different malignant potentials,
postoperative follow-up assessment is similar for G1 and G2 pancreatic
neuroendocrine tumors (panNETs) and adjuvant treatment currently is not
indicated. This study investigated the role of Ki67 with regard to
recurrence and survival after curative resection of panNET. METHODS:
Patients with resected non-functioning panNET diagnosed between 1992 and
2016 from three institutions were retrospectively analyzed. Patients who
had G1 or G2 tumor without distant metastases or hereditary syndromes
were included in the study. The patients were re-categorized into Ki67
0-5 and Ki67 6-20\%. Cox regression analysis with log-rank testing for
recurrence and survival was performed. RESULTS: The study enrolled 241
patients (86\%) with Ki67 0-5\% and 39 patients (14\%) with Ki67 6-20\%.
Recurrence was seen in 34 patients (14\%) with Ki67 0-5\% after a median
period of 34~months and in 16 patients (41\%) with Ki67 6-20\% after a
median period of 16~months (p \textless{} 0.001). The 5-year
recurrence-free and 10-year disease-specific survival periods were
respectively 90 and 91\% for Ki67 0-5\% and respectively 55 and 26\% for
Ki67 6-20\% (p \textless{} 0.001). The overall survival period after
recurrence was 44.9~months, which was comparable between the two groups
(p = 0.283). In addition to a Ki67 rate higher than 5\%, tumor larger
than 4~cm and lymph node metastases were independently associated with
recurrence. CONCLUSIONS: Patients at high risk for recurrence after
curative resection of G1 or G2 panNET can be identified by a Ki67 rate
higher than 5\%. These patients should be more closely monitored
postoperatively to detect recurrence early and might benefit from
adjuvant treatment. A clear postoperative follow-up regimen is proposed.

PubMed: \url{https://www.ncbi.nlm.nih.gov/pubmed/?term=29789972}

{}

{}

\begin{center}\rule{0.5\linewidth}{\linethickness}\end{center}

 - \textbf{Neuroendocrine tumor of the pancreas with rhabdoid feature}

\emph{Virchows Archiv : an international journal of pathology 2018
Jun;():}

Imaging of a 53-year-old Japanese man revealed two tumors in the liver
and a tumor in the head of the pancreas with a swelling lymph node. A
needle biopsy for the liver tumors was performed, revealing a
neuroendocrine tumor. Enucleation, lymphadenectomy, and partial
hepatectomy were performed. The microscopic examination identified many
tumor cells with intracytoplasmic inclusions arranged in a nested, cord,
or tubular fashion. The intracytoplasmic inclusions displayed densely
eosinophilic globules and displaced the nuclei toward the periphery,
which constitutes ``rhabdoid'' features. The tumor cells were positive
for synaptophysin and weakly positive for NCAM, but negative for
chromogranin A. Epithelial markers (AE1/AE3 and CAM5.2) accentuated
intracytoplasmic globules. Pancreatic neuroendocrine tumors with
rhabdoid features are very rare. Generally, rhabdoid features are
aggressive and dedifferentiated characteristics of various types of
tumor. Pancreatic neuroendocrine tumors containing rhabdoid cells tend
to display extrapancreatic spread at the time of presentation, although
some of these tumors with rhabdoid features are not always associated
with aggressive behavior.

PubMed: \url{https://www.ncbi.nlm.nih.gov/pubmed/?term=29938394}

{}

{}

\begin{center}\rule{0.5\linewidth}{\linethickness}\end{center}

 - \textbf{Prognostic Significance of Preoperative
Neutrophil-to-Lymphocyte Ratio in Surgically Resectable Pancreatic
Neuroendocrine Tumors}

\emph{Medical science monitor : international medical journal of
experimental and clinical research 2017 Nov;23():5574-5588}

BACKGROUND The aim of this study was to evaluate the predictive and
prognostic value of the preoperative neutrophil-to-lymphocyte ratio
(NLR) and platelet-to-lymphocyte ratio (PLR) in pancreatic
neuroendocrine tumor (PNET) patients undergoing potentially curative
resection. MATERIAL AND METHODS A retrospective review of 172 patients
with PNETs was conducted. Kaplan-Meier curves and multivariate Cox
proportional models were used to calculate overall survival (OS) and
disease-free survival (DFS). The predictive performance of the NLR was
compared with other inflammation-based scores and conventional
stratification systems using receiver operating characteristic (ROC)
curve analysis. RESULTS Elevated NLR and PLR were both associated with
advanced AJCC stage and high grade. In the univariate analysis, elevated
NLR and PLR were both significantly associated with decreased OS and
DFS. In the multivariate analysis, the preoperative NLR, but not the
PLR, was an independent risk factor for OS (HR=4.471, 95\% CI
1.531-13.054, p=0.006) and DFS (HR=2.531, 95\% CI 1.202-5.329, p=0.015).
The discriminatory capability of the NLR was superior to that of other
inflammation-based scores in OS prediction. Furthermore, the predictive
range was expanded by incorporating the NLR into the conventional
stratification systems, including the AJCC stage and WHO classification
systems. CONCLUSIONS As an independent prognostic factor, an elevated
preoperative NLR is superior to the PLR with respect to predicting
clinical outcomes in PNET patients undergoing potentially curative
resection. The incorporation of the NLR into the existing conventional
stratification systems improved the predictive accuracy.

PubMed: \url{https://www.ncbi.nlm.nih.gov/pubmed/?term=29168979}

{}

{}

\begin{center}\rule{0.5\linewidth}{\linethickness}\end{center}

 - \textbf{Variability of the Ki-67 proliferation index in
gastroenteropancreatic neuroendocrine neoplasms - a single-center
retrospective study}

\url{https://link.springer.com/article/10.1186/s12902-018-0274-y}

\begin{center}\rule{0.5\linewidth}{\linethickness}\end{center}

 - \textbf{APOBEC3B High Expression in Gastroenteropancreatic
Neuroendocrine Neoplasms and Association With Lymph Metastasis}

\emph{Applied immunohistochemistry \& molecular morphology : AIMM 2018
Aug;():}

PURPOSE: Apolipoprotein B mRNA editing enzyme catalytic polypeptide-like
3B (APOBEC3B) is known as a source of mutations in multiple cancers.
Gastroenteropancreatic neuroendocrine neoplasms (GEP-NENs) are a group
of heterogeneous tumors. However, the expression and significance of
APOBEC3B in GEP-NENs remains unclear. MATERIALS AND METHODS: A total of
158 cases of GEP-NENs, including 78 cases of biopsy or endoscopic
submucosal dissection resection specimens and 83 cases of surgical
resection specimens were collected in this study. The cases were grouped
according to tumor classification grade, including 42 cases of
neuroendocrine tumors G1 (NET G1), 36 cases of NET G2, 36 cases of NET
G3, 44 cases of neuroendocrine carcinoma (NEC). All of the 158 tumors
were immunohistochemically studied using a polyclonal antibody against
APOBEC3B. We evaluated APOBEC3B expression in GEP-NENs and investigated
the relationships among the immunoreactivity of APOBEC3B, clinical and
pathologic features, such as age, sex, tumor site, Ki67 cell
proliferation index, and lymph metastasis. RESULTS: A total of 33 cases
(78.6\%) of NET G1 showed high expression of APOBEC3B. A total of 28
cases (77.8\%) of NET G2 demonstrated high expression of APOBEC3B. In
NET G3 and NEC cases, the positive rates were 52.8\% and 2.3\%,
respectively. The expression of APOBEC3B in NETs was significantly
higher than that in NECs, NET G1 and NET G2 were higher than NET G3, and
the difference was statistically significant. APOBEC3B high expression
cases have lower lymph node metastasis rate, lower Ki67 cell
proliferation index. CONCLUSIONS: In this study, APOBEC3B is highly
expressed in GEP-NETs and is a predictor of lymph node metastasis in NET
G3 and NEC cases. These findings might provide new insights into the
biological mechanisms of GEP-NENs tumorigenesis and progression.

PubMed: \url{https://www.ncbi.nlm.nih.gov/pubmed/?term=30095460}

{}

{}

\begin{center}\rule{0.5\linewidth}{\linethickness}\end{center}

 - \textbf{Profiling the Tumour Immune Microenvironment in Pancreatic
Neuroendocrine Neoplasms with Multispectral Imaging Indicates Distinct
Subpopulation Characteristics Concordant with WHO 2017 Classification}

\emph{Scientific reports 2018 Sep;8(1):13166}

We successfully determined the difference of immune microenvironments
between pNENs and pancreatic ductal adenocarcinomas (PDACs), and the
histology-dependent variability among pNENs using multispectral
fluorescent imaging system. Tumour tissue samples including 52 pNENs and
18 PDACs were investigated. The tumour-infiltrating lymphocytes (TILs),
their PD-1 and PD-L1 expression in the pNENs were comprehensively and
quantitatively analysed and were subsequently compared with those in
PDACs. A principal component analysis revealed that the tissue immune
profile is related to tumour histology, with distinct groups being
observed for NETs, NECs, and PDACs. While NECs and some PDACs had hot
immune microenvironments with abundant TILs, NETs had a cold immune
microenvironment with few TILs. Moreover, in NETs, the numbers of
intraepithelial PD-1high T cells and PD-L1high Type-II macrophages were
elevated according to the grade. Univariate analysis revealed that lymph
node metastasis, grade, stage, PD-1high T cells, and PD-L1high Type-II
macrophages were predictors for recurrence-free survival (RFS), while
grade and PD-1high T cells were prognostic factors for overall survival
(OS). We also showed that PD-1high T cells and PD-L1high Type-II
macrophages were associated with worse outcome in pNENs. Our results
support the WHO 2017 tumour classification criteria, which distinguish
between G3 NETs and NECs.

PubMed: \url{https://www.ncbi.nlm.nih.gov/pubmed/?term=30177687}

{}

{}

\begin{center}\rule{0.5\linewidth}{\linethickness}\end{center}

 - \textbf{Characterization of the Neuroendocrine Tumor Immune
Microenvironment}

\emph{Pancreas 2018 Oct;47(9):1123-1129}

OBJECTIVES: The immune environment and the potential for neuroendocrine
tumors (NETs) to respond to immune checkpoint inhibitors remain largely
unexplored. We assessed immune checkpoint marker expression, lymphocytic
infiltrate, and associated mutational profiles in a cohort of small
intestine and pancreatic NETs. METHODS: We assessed expression of PDCD1
(PD-1), CD274 (PD-L1), and PDCD1LG2 (PD-L2) in archival tissue from 64
small intestine (SINETs) and 31 pancreatic NETs (pNET). We additionally
assessed T-cell infiltrates, categorizing T-cell subsets based on
expression of the T-cell markers CD3, CD8, CD45RO (PTPRC), or FOXP3.
Finally, we explored associations between immune checkpoint marker
expression, lymphocytic infiltrate, and tumor mutational profiles.
RESULTS: Expression of PD-1 or PD-L1 in small intestine or pancreatic
NET was rare, whereas expression of PD-L2 was common in both NET
subtypes. T-cell infiltrates were more abundant in pNET than in SINET.
We found no clear associations between immune checkpoint marker
expression, immune infiltrates, and specific mutational profile within
each tumor type. CONCLUSIONS: Our findings provide an initial assessment
of the immune environment of well-differentiated NETs. Further studies
to define the immunologic differences between pNET and SINET, as well as
the role of PD-L2 in these tumors, are warranted.

PubMed: \url{https://www.ncbi.nlm.nih.gov/pubmed/?term=30153220}

{}

{}

\begin{center}\rule{0.5\linewidth}{\linethickness}\end{center}

 - \textbf{The expression of TTF1, CDX2 and ISL1 in 74 poorly
differentiated neuroendocrine carcinomas}

\emph{Annals of diagnostic pathology 2018 Sep;37():30-34}

BACKGROUND: The expression profile of immunohistochemical markers of
origin in poorly differentiated neuroendocrine carcinoma (PDNEC) is not
well studied. MATERIALS AND METHODS: Seventy-four PDNECs from
gastroenteropancreatic (GEP) organs and the lung, including 48 large
cell NEC (LCNEC) and 26 small cell carcinomas (SmCC), were subject to
immunohistochemical staining for CDX2, TTF1 and ISL1. The staining
intensity (1 to 3) and percentage of positive tumor cells {[}0
(negative), 1 (\textless{}50\%) and 2 (≥50\%){]} were assessed. The
multiplicative index (maximum 6) was calculated and the average total
score (aTS) was determined for each primary site and histologic subtype.
RESULTS: In the 38 GEP and 36 lung PDNECs, CDX2, TTF1 and ISL1 staining
was observed in 71\% (aTS 2.8), 16\% (aTS 0.4), 63\% (aTS 1.9), and 22\%
(aTS 0.6), 72\% (aTS 2.9) and 92\% (aTS 3.8), respectively. GEP PDNECs
showed a higher aTS for CDX2 and lower aTS for TTF1 and ISL1, compared
to that of lung PDNECs (Student's t-test, p\,\textless{}\,0.001). SmCC
had a higher aTS for TTF1 and ISL1 (p\,\textless{}\,0.001) and lower aTS
for CDX2 (p\,\textless{}\,0.002) than that of LCNEC. CONCLUSIONS: CDX2
and TTF1 demonstrate potential utility in suggesting the primary site of
PDNEC. In addition, CDX2 may be useful in supporting the diagnosis of
LCNEC in cases with overlapping or borderline morphology. Utility of
ISL1 as an adjunctive diagnostic marker of SmCC remains to be studied.

PubMed: \url{https://www.ncbi.nlm.nih.gov/pubmed/?term=30236546}

{}

{}

\begin{center}\rule{0.5\linewidth}{\linethickness}\end{center}

\begin{itemize}
\tightlist
\item
  \textbf{Characterization of the Neuroendocrine Tumor Immune
  Microenvironment}
\end{itemize}

\url{https://journals.lww.com/pancreasjournal/Abstract/2018/10000/Characterization_of_the_Neuroendocrine_Tumor.10.aspx}

\begin{center}\rule{0.5\linewidth}{\linethickness}\end{center}

Back to top

\pagebreak

\hypertarget{cytopathology}{%
\subsection{Cytopathology}\label{cytopathology}}

\begin{center}\rule{0.5\linewidth}{\linethickness}\end{center}

\hypertarget{pancreas-1}{%
\subsubsection{Pancreas}\label{pancreas-1}}

\begin{center}\rule{0.5\linewidth}{\linethickness}\end{center}

 - \textbf{Pancreatic Cytology}

\emph{Surgical pathology clinics 2018 Sep;11(3):563-588}

The diagnostic approach to pancreaticobiliary disease requires a
multidisciplinary team in which the cytopathologist plays a crucial
role. Fine-needle aspiration, obtained by endoscopic ultrasound, is the
diagnostic test of choice for pancreatic lesions. Preoperative clinical
management depends on many factors, many of which rely on accurate
cytologic assessment. Pancreaticobiliary cytology is wrought with
diagnostic pitfalls. Clinical history, imaging studies, cytology
samples, and ancillary tests, including immunohistochemistry,
biochemical analysis, and genetic sequencing, are integral to forming a
complete diagnosis and guiding optimal patient management. This article
reviews clinical aspects and the diagnostic work-up of commonly
encountered diagnostic entities within the field of pancreatic cytology.

PubMed: \url{https://www.ncbi.nlm.nih.gov/pubmed/?term=30190141}

{}

{}

\url{https://www.surgpath.theclinics.com/article/S1875-9181(18)30029-1/fulltext}

\begin{center}\rule{0.5\linewidth}{\linethickness}\end{center}

 - \textbf{Acute Pancreatitis History Carries Higher Risk in Endoscopic
Ultrasound-Guided Fine-Needle Aspiration of Pancreatic Lesions}

\emph{Pancreas 2018 Aug;47(7):e38-e40}

PubMed: \url{https://www.ncbi.nlm.nih.gov/pubmed/?term=29985847}

{}

{}

\begin{center}\rule{0.5\linewidth}{\linethickness}\end{center}

 - \textbf{Cystic pancreatic schwannoma diagnosed by endoscopic
ultrasound-guided fine needle aspiration}

\emph{Diagnostic cytopathology 2018 Aug;():}

PubMed: \url{https://www.ncbi.nlm.nih.gov/pubmed/?term=30146793}

{}

{}

\begin{center}\rule{0.5\linewidth}{\linethickness}\end{center}

 - \textbf{Ancillary tests in the diagnosis of liver and pancreatic
neoplasms}

\emph{Cancer cytopathology 2018 Aug;126 Suppl 8():672-690}

Ancillary tests in the diagnosis of liver and pancreatic neoplasms
include a wide array of immunostains and molecular diagnostic tests, and
the selection of tests is based on the differential diagnosis. This
review discusses ancillary tests in the diagnosis of liver tumors,
including benign and malignant primary tumors as well as metastatic
tumors to the liver. In addition, ancillary tests for the diagnosis of
both solid and cystic pancreatic neoplasms, including molecular tests in
these lesions are also reviewed.

PubMed: \url{https://www.ncbi.nlm.nih.gov/pubmed/?term=30156777}

{}

{}

\begin{center}\rule{0.5\linewidth}{\linethickness}\end{center}

\begin{itemize}
\tightlist
\item
  \textbf{Ancillary tests in the diagnosis of liver and pancreatic
  neoplasms}
\end{itemize}

\url{https://onlinelibrary.wiley.com/doi/abs/10.1002/cncy.22009}

\begin{center}\rule{0.5\linewidth}{\linethickness}\end{center}

\begin{itemize}
\tightlist
\item
  \textbf{Comparison between groove carcinoma and groove pancreatitis}
\end{itemize}

\url{https://www.sciencedirect.com/science/article/pii/S142439031830663X}

\begin{center}\rule{0.5\linewidth}{\linethickness}\end{center}

 - \textbf{Synchronous pancreatic tumors in a patient with history of
Wilms tumor: A case of pancreatic adenocarcinoma and lipid-rich
neuroendocrine tumor diagnosed by cytopathology}

\emph{Diagnostic cytopathology 2018 Sep;():}

Synchronous tumors represent a very small portion of pancreatic tumors.
Although there is a higher incidence of secondary malignant neoplasms
(SMN) in patients with history of Wilms tumor (WT), pancreatic tumors
are very infrequent SMNs in this population. We report the first case of
synchronous pancreatic tumors in a patient with history of WT. Two
separated pancreatic lesions were identified by abdominal computerized
tomography (CT) scan. Fine-needle aspiration of both lesions was
performed for cytopathology examination. A pancreatic adenocarcinoma was
diagnosed in the head of pancreas, and the pancreatic body lesion was
found to be a neuroendocrine tumor (NET). The NET had characteristic
vacuolated lipid-rich cytoplasm. Further molecular testing was done on
both tumors, but no common cancer-associated mutation was found.

PubMed: \url{https://www.ncbi.nlm.nih.gov/pubmed/?term=30194916}

{}

{}

\begin{center}\rule{0.5\linewidth}{\linethickness}\end{center}

 - \textbf{Comparison between groove carcinoma and groove pancreatitis}

\emph{Pancreatology : official journal of the International Association
of Pancreatology (IAP) \ldots{} {[}et al.{]} 2018 Oct;18(7):805-811}

BACKGROUND/OBJECTIVES: The pancreatoduodenal groove (anatomical groove)
is a potential space bordered by the head of the pancreas, duodenum, and
common bile duct. Discerning between groove carcinoma (GC) and groove
pancreatitis (GP) is often difficult, but clinically important. We
retrospectively analyzed and compared the findings of computed
tomography (CT), laboratory tests, and endoscopic ultrasound-fine needle
aspiration (EUS-FNA) for GC and GP. METHODS: GC (n\,=\,36) and GP
(n\,=\,44) patients at Asan Medical Center from January 1, 2000, to May
31, 2017 were retrospectively reviewed. MDCT findings, baseline
characteristics, laboratory test results, and EUS and EUS-FNA findings
of GC and GP patients were compared. RESULTS: CT showed no significant
difference in groove enhancement between the groups. Mass-like lesions,
cystic groove lesions, and calcification were observed in 86.1\% and
15.9\%, 38.9\% and 75\%, and 2.8\% and 29.5\% of GC and GP patients,
respectively. Patients were tested for total bilirubin (GC: 2.0 vs.~GP:
0.6\,mg/dL), cancer antigen 19-9 (CA19-9) (GC: 76 vs.~GP: 12.5 U/mL),
and carcinoembryonic antigen (GC: 2.4 vs.~GP: 2\,ng/mL). Three GP
patients died, and one GP patient was diagnosed with GC. However, among
30\,GC patients with at least 1-year follow-up, 20 died. In multivariate
logistic regression, CA19-9, and mass-like lesion on multidetector CT
(MDCT) were discriminating factors between GC and GP. Among 23 (10\,GC,
13\,GP) patients who underwent EUS-FNA, the diagnostic sensitivity,
specificity, positive predictive value, negative predictive value,
accuracy of EUS-FNA were 90\%, 100\%, 100\%, 92.86\%, and 95.65\%,
respectively. CONCLUSIONS: Several MDCT and laboratory findings favor GC
over GP. EUS-FNA should be considered in patients with elevated CA19-9
levels and mass-like lesions on MDCT.

PubMed: \url{https://www.ncbi.nlm.nih.gov/pubmed/?term=30224296}

{}

{}

\begin{center}\rule{0.5\linewidth}{\linethickness}\end{center}

 - \textbf{Risk of malignancy in pancreatic cysts with cytology of
high-grade epithelial atypia}

\emph{Cancer cytopathology 2018 Sep;126(9):773-781}

BACKGROUND: The risk of malignancy is weighed against the attendant
risks of surgery in the clinical management of pancreatic cysts. The
latter are a group of histologically diverse and prognostically variable
entities, and the risk of malignancy therein is primarily based on
imaging characteristics-with or without high-grade atypia. Cytologic
criteria for high-grade atypia in intraductal papillary mucinous
neoplasms have recently been defined, and its recognition in all
pancreatic cysts may help to guide management. METHODS: All patients who
underwent endoscopic ultrasound-guided fine-needle aspiration for a
pancreatic cyst at Massachusetts General Hospital from June 2015 to
October 2016 were prospectively evaluated. Clinical data, radiographic
impressions, biochemical analyses, and cytologic diagnoses of 118
pancreatic cyst fine-needle aspiration biopsy specimens from 106
patients were reviewed. Clinical and radiologic data were used as
follow-up for 86 patients, and histology was obtained in 20 cases. Cysts
were classified by imaging as high-risk, worrisome, or low-risk as
defined by the 2012 Fukuoka consensus guidelines. Cytology was
categorized as low-grade or high-grade. Malignant histology included
mucinous cysts with high-grade dysplasia, invasive adenocarcinomas, and
neuroendocrine tumors. The risk of malignancy (ROM) was determined by
histological outcome. RESULTS: The presence of high-grade cytology (P
\textless{} .01) was the only statistically significant predictor of
malignancy and was 89\% sensitive and 98\% specific for malignancy. The
positive predictive value (ie, ROM) of high-grade atypia on cytology was
80\%. CONCLUSIONS: High-grade atypia is both sensitive and specific for
identifying high-risk pancreatic cysts and is associated with a high
risk of malignancy, and thus resection is warranted.

PubMed: \url{https://www.ncbi.nlm.nih.gov/pubmed/?term=30257067}

{}

{}

\begin{center}\rule{0.5\linewidth}{\linethickness}\end{center}

\hypertarget{bile-ducts-1}{%
\subsubsection{Bile Ducts}\label{bile-ducts-1}}

\begin{center}\rule{0.5\linewidth}{\linethickness}\end{center}

 - \textbf{Cytologic diagnosis of adenocarcinoma on bile duct brushings
in the presence of stent associated changes: A retrospective analysis}

\emph{Diagnostic cytopathology 2018 Aug;():}

BACKGROUND: Bile duct brushing (BDB) cytology, for the characterization
of bile duct strictures, can be challenging to interpret when associated
with a stent. Our study aims to identify the cytologic criteria for the
diagnosis of adenocarcinoma in BDBs in the presence of a stent. METHODS:
A database search (January 2010-December 2015) identified three groups
of BDBs-negative with stent, malignant with stent, malignant without
stent. All malignant cases had histologic and/or cytologic evidence of
malignancy within 1 month of the brushing sample. All reactive cases had
≥6 months of benign clinical follow-up. ThinPrep slides were reviewed by
two cytopathologists and cytologic features were recorded. Statistical
analysis was performed using Fisher's exact test. RESULTS: The study
cohort included 12 reactive cases with stent, 17 malignant cases with
stent and 32 malignant cases without stent. Among the stented cases, the
cytologic features that reached statistical significance were 3D
architecture, anisonucleosis to the extent of ≥1:6, coarse chromatin
distribution and the presence of single atypical cells in the malignant
group in contrast to the benign group. Cases that were diagnosed
malignant in the presence of a stent revealed a spectrum of cell
populations more frequently as compared with the malignant cases without
stent (76\% vs 16\%). CONCLUSION: Our findings reveal that the cytologic
features of malignancy in non-stented BDBs mostly hold true for stented
specimens as well. Application of these criteria in the presence of a
stent can improve diagnostic accuracy and thereby patient care.

PubMed: \url{https://www.ncbi.nlm.nih.gov/pubmed/?term=30144340}

{}

{}

\begin{center}\rule{0.5\linewidth}{\linethickness}\end{center}

 - \textbf{Cytological diagnosis of cryptococcosis in a biliary
specimen: Report of a rare case with brief review of literature}

\emph{Cytopathology : official journal of the British Society for
Clinical Cytology 2018 Aug;():}

PubMed: \url{https://www.ncbi.nlm.nih.gov/pubmed/?term=30153349}

{}

{}

\begin{center}\rule{0.5\linewidth}{\linethickness}\end{center}

\begin{center}\rule{0.5\linewidth}{\linethickness}\end{center}

\begin{center}\rule{0.5\linewidth}{\linethickness}\end{center}

\hypertarget{neuroendocrine-1}{%
\subsubsection{Neuroendocrine}\label{neuroendocrine-1}}

\begin{center}\rule{0.5\linewidth}{\linethickness}\end{center}

 - ****

\begin{itemize}
\tightlist
\item
  ;():*
\end{itemize}

PubMed: \url{https://www.ncbi.nlm.nih.gov/pubmed/?term=}

{}

{}

\begin{center}\rule{0.5\linewidth}{\linethickness}\end{center}

Back to top

\pagebreak

\hypertarget{molecular-pathology}{%
\subsection{Molecular Pathology}\label{molecular-pathology}}

\begin{center}\rule{0.5\linewidth}{\linethickness}\end{center}

\hypertarget{pancreas-2}{%
\subsubsection{Pancreas}\label{pancreas-2}}

\begin{center}\rule{0.5\linewidth}{\linethickness}\end{center}

 - \textbf{Precancerous neoplastic cells can move through the pancreatic
ductal system}

\emph{Nature 2018 Sep;561(7722):201-205}

Most adult carcinomas develop from noninvasive precursor lesions, a
progression that is supported by genetic analysis. However, the
evolutionary and genetic relationships among co-existing lesions are
unclear. Here we analysed the somatic variants of pancreatic cancers and
precursor lesions sampled from distinct regions of the same pancreas.
After inferring evolutionary relationships, we found that the ancestral
cell had initiated and clonally expanded to form one or more lesions,
and that subsequent driver gene mutations eventually led to invasive
pancreatic cancer. We estimate that this multi-step progression
generally spans many years. These new data reframe the step-wise
progression model of pancreatic cancer by illustrating that independent,
high-grade pancreatic precursor lesions observed in a single pancreas
often represent a single neoplasm that has colonized the ductal system,
accumulating spatial and genetic divergence over time.

PubMed: \url{https://www.ncbi.nlm.nih.gov/pubmed/?term=30177826}

{}

{}

\begin{center}\rule{0.5\linewidth}{\linethickness}\end{center}

 - \textbf{Stratification of Pancreatic Ductal Adenocarcinomas Based on
Tumor and Microenvironment Features}

\emph{Gastroenterology 2018 Aug;():}

BACKGROUND \& AIMS: Genomic studies have revealed subtypes of pancreatic
ductal adenocarcinoma (PDA) based on their molecular features, but
different studies have reported different classification systems. It is
a challenge to obtain high-quality, freshly frozen tissue for clinical
analysis and determination of PDA subtypes. We aimed to redefine
subtypes of PDA using a large number of formalin-fixed and
paraffin-embedded PDA samples, which are more amenable to routine
clinical evaluation. METHODS: We collected PDA samples from 309
consecutive patients who underwent surgery from September 1996 through
December 2010 at 4 academic hospitals in Europe; non-tumor tissue
samples were not included. Samples were formalin fixed and paraffin
embedded. DNA and RNA were isolated; gene expression, targeted DNA
sequencing, and immunohistochemical analyses were performed. We used
independent component analysis to deconvolute normal, tumor, and
microenvironment transcriptome patterns in samples. We devised
classification systems from an unsupervised analysis using a consensus
clustering approach of our dataset after removal of normal contamination
components. We associated subtypes with overall survival and
disease-free survival of patients using Cox proportional hazards
regression with estimation of hazard ratios and 95\% CIs. We used The
Cancer Genome Consortium (TCGA) and International Cancer Genome
Consortium (ICGC) PDA datasets as validation cohorts. RESULTS: We
validated the previously reported basal-like and classical
tumor-specific subtypes of PDAs. We identified features of the PDA,
including microenvironment gene expression patterns, that allowed tumors
to be categorized into 5 subtypes, called pure basal like, stroma
activated, desmoplastic, pure classical, and immune classical. These PDA
subtypes have features of cancer cells and immune cells that could be
targeted by pharmacologic agents. Tumor subtypes associated with patient
outcomes, based on analysis of our dataset and the ICGC and TCGA PDA
datasets. We also observed an exocrine signal associated with acinar
cell contamination (from pancreatic tissue). CONCLUSIONS: We identified
a classification system based on gene expression analysis of
formalin-fixed PDA samples. We identified 5 PDA subtypes, based on
features of cancer cells and the tumor microenvironment. This system
might be used to select therapies and predict patient outcomes. We found
evidence that the previously reported exocrine-like (called ADEX) tumor
subtype resulted from contamination with pancreatic acinar cells.

PubMed: \url{https://www.ncbi.nlm.nih.gov/pubmed/?term=30165049}

{}

{}

\begin{center}\rule{0.5\linewidth}{\linethickness}\end{center}

 - \textbf{WIPF1 antagonizes the tumor suppressive effect of
miR-141/200c and is associated with poor survival in patients with PDAC}
\emph{Journal of experimental \& clinical cancer research : CR 2018
Jul;37(1):167}

BACKGROUND: Aberrant expression of Wiskott-Aldrich syndrome protein
interacting protein family member 1 (WIPF1) contributes to the invasion
and metastasis of several malignancies. However, the role of WIPF1 in
human pancreatic ductal adenocarcinoma (PDAC) remains poorly understood.
METHODS: Human pancreatic cancer samples from PDAC patients were
collected for methylation analysis. Bioinformatic prediction program and
luciferase reporter assay were used to identify microRNAs regulating
WIPF1 expression. The association between WIPF1 expression and the
overall survival (OS) of patients with PDAC was evaluated by using The
Cancer Genome Atlas (TCGA) database. The roles of miR-141/200c and WIPF1
on the invasion and metastasis of PDAC cells were investigated both in
vitro and in vivo. RESULTS: We found that compared to the surrounding
non-cancerous tissues, there was significantly increased methylation of
miR-200c and miR-141 in human PDAC tissues that resulted in their
silencing, whereas the members of the other cluster of miR-200 family
including miR-200a, miR-200b and miR-429 were hypomethylated. Our data
show that forced expression of miR-141 or miR-200c suppressed invasion
and metastasis of PDAC cells both in vitro and in xenograft and
identified WIPF1 as a direct target of miR-141 and miR-200c. Both
miR-141 and miR-200c inhibit WIPF1 by directly interacting with its
3'-untranslated region. Remarkably, silencing of WIPF1 blocked PDAC
growth and metastasis both in vitro and in vivo, whereas forced WIPF1
overexpression antagonized the tumor suppressive effect of miR-141/200c.
Additionally, by using TCGA database we showed that high expression of
WIPF1 correlated with poor survival in patients with PDAC. Moreover, we
show that miR-141 and miR-200c blocked YAP/TAZ expression by suppressing
WIPF1. CONCLUSIONS: We have identified WIPF1 as an oncoprotein in PDAC
and a direct target of miR-141/miR-200c. We have also defined the
miR-141/200c-WIPF1-YAP/TAZ as a novel signaling pathway that is involved
in the regulation of the invasion and metastasis of human PDAC cells.

PubMed: \url{https://www.ncbi.nlm.nih.gov/pubmed/?term=30041660}

{}

{}

\begin{center}\rule{0.5\linewidth}{\linethickness}\end{center}

 - \textbf{Liquid Biopsies for Management of Pancreatic Cancer}

\url{https://www.sciencedirect.com/science/article/pii/S1931524418301099}

\begin{center}\rule{0.5\linewidth}{\linethickness}\end{center}

 - \textbf{Smad4/DPC4}

\emph{Journal of clinical pathology 2018 Aug;71(8):661-664}

Smad4 or DPC4 belongs to a family of signal transduction proteins that
are phosphorylated and activated by transmembrane serine-threonine
receptor kinases in response to transforming growth factor beta (TGF-β)
signaling via several pathways. The gene acts as a tumour suppressor
gene and inactivation of smad4/DPC4 is best recognised in pancreatic
cancer. However, smad4/DPC4 is also mutated in other conditions and
cancers such as juvenile polyposis syndrome with and without hereditary
haemorrhagic telangiectasia, colorectal and prostate
cancers.Immunohistochemistry for smad4/DPC4 protein is most useful in
separating benign/reactive conditions from pancreatic cancer in
needle/core biopsies. In normal and reactive states, the protein is
localised to the cytoplasm and nucleus, while the protein is lost in
high-grade pancreatic intraepithelial neoplasia/carcinoma in situ and
pancreatic cancer.

PubMed: \url{https://www.ncbi.nlm.nih.gov/pubmed/?term=29720405}

{}

{}

\begin{center}\rule{0.5\linewidth}{\linethickness}\end{center}

 - \textbf{Prospective study of germline genetic testing in incident
cases of pancreatic adenocarcinoma}

\emph{Cancer 2018 Sep;124(17):3520-3527}

BACKGROUND: The objective of this study was to investigate the
prevalence of pathogenic germline variants (PGVs) in 32 cancer
susceptibility genes in individuals with newly diagnosed pancreatic
ductal adenocarcinoma (PDAC). A key secondary objective was to evaluate
how often PGVs would have been undetected with existing genetic testing
criteria. METHODS: From May 2016 through May 2017, this multicenter
cohort study enrolled consecutive patients aged 18 to 89 years with
histologically confirmed PDAC diagnosed within the previous 12 weeks.
Demographics, medical histories, and 3-generation pedigrees were
collected from participants who provided samples for germline DNA
analysis. RESULTS: Four hundred nineteen patients were deemed eligible,
302 were enrolled, and 298 were included in the final cohort. Clinically
actionable variants were reported in 29 PDAC patients (9.7\%), with 23
(7.7\%) having a PGV associated with an increased risk for PDAC. Six of
23 individuals (26\%) with PDAC-associated gene mutations did not meet
currently established genetic testing criteria. According to
guideline-based genetic testing, only 11 of the 23 PGVs (48\%) in known
PDAC genes would have been detected. Six additional patients (2\%) had
PGVs associated with an increased risk for other cancers. CONCLUSIONS:
These findings support the significant prevalence of PGVs associated
with PDAC and the limitations of current paradigms for selecting
patients for genetic testing, and they thereby lend support for
universal germline multigene genetic testing in this population.

PubMed: \url{https://www.ncbi.nlm.nih.gov/pubmed/?term=30067863}

{}

{}

\begin{center}\rule{0.5\linewidth}{\linethickness}\end{center}

 - \textbf{Multi-institutional Validation Study of Pancreatic Cyst Fluid
Protein Analysis for Prediction of High-risk Intraductal Papillary
Mucinous Neoplasms of the Pancreas}

\emph{Annals of surgery 2018 Aug;268(2):340-347}

OBJECTIVE: Preliminary work by our group suggested that proteins within
the pancreatic cyst fluid (CF) may discriminate degree of IPMN
dysplasia. We sought to externally validate these markers and determine
whether their inclusion in a preoperative clinical nomogram could
increase diagnostic accuracy. SUMMARY BACKGROUND DATA: IPMN is the most
common radiographically identifiable precursor to pancreatic cancer;
however, the timing and frequency of its malignant progression are
unknown, and there are currently no reliable preoperative tests that can
determine the grade of dysplasia in IPMN. METHODS: Clinical and
radiographic data, as well as CF samples, were obtained from 149
patients who underwent resection for IPMN at 1 of 3 institutions.
High-risk disease was defined as the presence of high-grade dysplasia or
invasive carcinoma. Multianalyte bead array analysis (Luminex) of CF was
performed for 4 protein markers that were previously associated with
high-risk disease. Logistic regression models were fit on training data,
with and without adjustment for a previously developed clinical nomogram
and validated with an external testing set. The models incorporating
clinical risk score were presented graphically as nomograms. RESULTS:
Within the group of 149 resected patients, 89 (60\%) had low-risk
disease, and 60 (40\%) had high-risk disease. All 4 CF markers (MMP9,
CA72-4, sFASL, and IL-4) were overexpressed in patients with high-risk
IPMN (P \textless{} 0.05). Two predictive models based on preselected
combinations of CF markers had concordance indices of 0.76 (Model-1) and
0.80 (Model-2). Integration of each CF marker model into a previously
described clinical nomogram leads to increased discrimination compared
with either the CF models or nomogram alone (c-indices of 0.84 and 0.83,
respectively). CONCLUSIONS: This multi-institutional study validated 2
CF protein marker models for preoperative identification of high-risk
IPMN. When combined with a clinical nomogram, the ability to predict
high-grade dysplasia was even stronger.

PubMed: \url{https://www.ncbi.nlm.nih.gov/pubmed/?term=28700444}

{}

{}

\begin{center}\rule{0.5\linewidth}{\linethickness}\end{center}

 - \textbf{Regulation of Epithelial Plasticity Determines Metastatic
Organotropism in Pancreatic Cancer}

\emph{Developmental cell 2018 06;45(6):696-711.e8}

The regulation of metastatic organotropism in pancreatic ductal a
denocarcinoma (PDAC) remains poorly understood. We demonstrate, using
multiple mouse models, that liver and lung metastatic organotropism is
dependent upon p120catenin (p120ctn)-mediated epithelial identity.
Mono-allelic p120ctn loss accelerates KrasG12D-driven pancreatic cancer
formation and liver metastasis. Importantly, one p120ctn allele is
sufficient for E-CADHERIN-mediated cell adhesion. By contrast, cells
with bi-allelic p120ctn loss demonstrate marked lung organotropism;
however, rescue with p120ctn isoform 1A restores liver metastasis. In~a
p120ctn-independent PDAC model, mosaic loss of E-CADHERIN expression
reveals selective pressure for E-CADHERIN-positive liver metastasis and
E-CADHERIN-negative lung metastasis. Furthermore, human PDAC and liver
metastases support the premise that liver metastases exhibit
predominantly epithelial characteristics. RNA-seq demonstrates
differential induction of pathways associated with metastasis and
epithelial-to-mesenchymal transition in p120ctn-deficient versus
p120ctn-wild-type cells. Taken together, P120CTN and E-CADHERIN mediated
epithelial plasticity is an addition to the conceptual framework
underlying metastatic organotropism in pancreatic cancer.

PubMed: \url{https://www.ncbi.nlm.nih.gov/pubmed/?term=29920275}

{}

{}

\begin{center}\rule{0.5\linewidth}{\linethickness}\end{center}

 - \textbf{A Highly Verified Assay for KRAS Mutation Detection in Tissue
and Plasma of Lung, Colorectal, and Pancreatic Cancer}

\emph{Archives of pathology \& laboratory medicine 2018 Aug;():}

CONTEXT: - KRAS Mutation Test v2 is used for the qualitative detection
and identification of 28 mutations in exons 2, 3, and 4 of the human
KRAS gene. OBJECTIVE: - To verify the performance of KRAS Mutation Test
v2 and to evaluate its accuracy by correlation with a next-generation
sequencing method on Illumina MiSeq. DESIGN: - In the study, we used
formalin-fixed, paraffin-embedded tissue and plasma specimens from
non-small cell lung cancer, colorectal cancer, and pancreatic cancer
patients. Results of specificity, precision, analytical sensitivity, and
accuracy as compared with a MiSeq method are reported. RESULTS: - The
KRAS Mutation Test v2 demonstrated exquisite sensitivity and specificity
and broad coverage of KRAS mutations. Precision was 100\% (108 of 108)
across all samples, operators, and instruments for formalin-fixed,
paraffin-embedded tissue and 99.8\% (615 of 616) for plasma. Analytical
sensitivity was high with detection of 1\% mutant sequence in
formalin-fixed, paraffin-embedded tissue samples and as low as 25 mutant
sequence copies/mL for plasma samples. The test also showed high overall
concordance for formalin-fixed, paraffin-embedded tumor tissue as well
as for plasma specimens when compared with MiSeq sequencing results.
CONCLUSIONS: - The KRAS Mutation Test v2 is a highly robust,
reproducible, and sensitive test for the qualitative detection of 28
mutations in exons 2, 3, and 4 of the KRAS gene in both solid (tissue)
and liquid (plasma) biopsies from colorectal cancer, non-small cell lung
cancer, and pancreatic cancer, and is a convenient option for KRAS
mutation testing.

PubMed: \url{https://www.ncbi.nlm.nih.gov/pubmed/?term=30088781}

{}

{}

\begin{center}\rule{0.5\linewidth}{\linethickness}\end{center}

\begin{itemize}
\tightlist
\item
  \textbf{A systematic review on metabolomics-based diagnostic biomarker
  discovery and validation in pancreatic cancer}
\end{itemize}

\url{https://link.springer.com/article/10.1007/s11306-018-1404-2}

\begin{center}\rule{0.5\linewidth}{\linethickness}\end{center}

\begin{itemize}
\tightlist
\item
  \textbf{Circulating Tumor Cells Dynamics in Pancreatic Adenocarcinoma
  Correlate With Disease Status: Results of the Prospective CLUSTER
  Study}
\end{itemize}

\url{https://journals.lww.com/annalsofsurgery/Fulltext/2018/09000/Circulating_Tumor_Cells_Dynamics_in_Pancreatic.4.aspx}

\begin{center}\rule{0.5\linewidth}{\linethickness}\end{center}

\begin{itemize}
\tightlist
\item
  \textbf{From somatic mutation to early detection: Insights from
  molecular characterization of pancreatic cancer precursor lesions}
\end{itemize}

\url{https://onlinelibrary.wiley.com/doi/abs/10.1002/path.5154}

\begin{center}\rule{0.5\linewidth}{\linethickness}\end{center}

 - \textbf{Germline Variants and Risk for Pancreatic Cancer: A
Systematic Review and Emerging Concepts}

\emph{Pancreas 2018 Sep;47(8):924-936}

Pancreatic cancer requires many genetic mutations. Combinations of
underlying germline variants and environmental factors may increase the
risk of cancer and accelerate the oncogenic process. We systematically
reviewed, annotated, and classified previously reported pancreatic
cancer-associated germline variants in established risk genes. Variants
were scored using multiple criteria and binned by evidence for
pathogenicity, then annotated with published functional studies and
associated biological systems/pathways. Twenty-two previously identified
pancreatic cancer risk genes and 337 germline variants were identified
from 97 informative studies that met our inclusion criteria. Fifteen of
these genes contained 66 variants predicted to be pathogenic (APC, ATM,
BRCA1, BRCA2, CDKN2A, CFTR, CHEK2, MLH1, MSH2, NBN, PALB2, PALLD, PRSS1,
SPINK1, TP53). Pancreatic cancer risk genes were organized into key
biological mechanisms that promote pancreatic oncogenesis within an
oncogenic model. Development of precision medicine approaches requires
updated variant information within the framework of an oncogenic
progression model. Complex risk modeling may improve interpretation of
early biomarkers and guide pathway-specific treatment for pancreatic
cancer in the future. Precision medicine is within reach.

PubMed: \url{https://www.ncbi.nlm.nih.gov/pubmed/?term=30113427}

{}

{}

\begin{center}\rule{0.5\linewidth}{\linethickness}\end{center}

\begin{itemize}
\tightlist
\item
  \textbf{A Highly Verified Assay for KRAS Mutation Detection in Tissue
  and Plasma of Lung, Colorectal, and Pancreatic Cancer}
\end{itemize}

\url{http://www.archivesofpathology.org/doi/pdf/10.5858/arpa.2017-0471-OA?code=coap-site}

\begin{center}\rule{0.5\linewidth}{\linethickness}\end{center}

\begin{itemize}
\tightlist
\item
  \textbf{PAR1 signaling on tumor cells limits tumor growth by
  maintaining a mesenchymal phenotype in pancreatic cancer}
\end{itemize}

\url{https://www.researchgate.net/profile/Cansu_Tekin2/publication/326967265_PAR1_signaling_on_tumor_cells_limits_tumor_growth_by_maintaining_a_mesenchymal_phenotype_in_pancreatic_cancer/links/5b73188245851546c90320f1/PAR1-signaling-on-tumor-cells-limits-tumor-growth-by-maintaining-a-mesenchymal-phenotype-in-pancreatic-cancer.pdf}

\begin{center}\rule{0.5\linewidth}{\linethickness}\end{center}

 - \textbf{Pancreatitis-Associated Genes and Pancreatic Cancer Risk: A
Systematic Review and Meta-analysis}

\emph{Pancreas 2018 Oct;47(9):1078-1086}

OBJECTIVE: The aim of this study was to evaluate the connection between
pancreatic cancer (PC) and genetic variants associated with chronic
pancreatitis via systematic review and meta-analysis. METHODS: The data
search was performed in 3 major databases (PubMed, Embase, and Cochrane
Library). The selected studies have looked into the presence of the
pancreatitis-associated genes in patients with PC and in control
subjects, the outcome being the frequency of the mutations in the 2
groups. For the binary outcomes, pooled odds ratio (OR) and 95\%
confidence interval (CI) were calculated. RESULTS: Ten articles proved
to be eligible for the qualitative synthesis, and 8 articles were
suitable for statistical analysis. Six case-control studies, comprising
929 PC cases and 1890 control subjects for serine protease inhibitor
Kazal type 1 (SPINK1) mutations, and 5 case-control studies, comprising
1674 PC cases and 19,036 control subjects for CFTR mutations, were
enrolled in our analysis. SPINK1 mutations showed no association with PC
(OR, 1.52; 95\% CI, 0.67-3.45; P = 0.315), whereas mutations in CFTR
modestly increased the risk of PC (OR, 1.41; 95\% CI, 1.07-1.84; P =
0.013). CONCLUSION: Our meta-analysis showed that mutations in CFTR
modestly increase the risk of PC, whereas no association was found
between SPINK1 and PC.

PubMed: \url{https://www.ncbi.nlm.nih.gov/pubmed/?term=30134356}

{}

{}

\begin{center}\rule{0.5\linewidth}{\linethickness}\end{center}

 - \textbf{Germline and Somatic DNA Damage Repair Gene Mutations and
Overall Survival in Metastatic Pancreatic Adenocarcinoma Patients
Treated with FOLFIRINOX}

\emph{Clinical cancer research : an official journal of the American
Association for Cancer Research 2018 Aug;():}

Purpose: Pancreatic ductal adenocarcinoma (PDAC) is a lethal cancer with
lack of predictive biomarkers. We conducted a study to assess DNA damage
repair (DDR) gene mutations as a predictive biomarker in PDAC patients
treated with FOLFIRINOX.Experimental Design: Indiana University Simon
Cancer Center pancreatic cancer database was used to identify patients
with metastatic PDAC, treated with FOLFIRINOX and had tissue available
for DNA sequencing. Baseline demographic, clinical, and pathologic
information was gathered. DNA isolation and targeted sequencing was
performed using the Ion AmpliSeq protocol. Overall survival (OS)
analysis was conducted using Kaplan-Meier, logistic regression and Cox
proportional hazard methods. Multivariate models were adjusted for age,
gender, margin status, CA 19-9, adjuvant chemotherapy, tumor and nodal
stage.Results: Overall, 36 patients were sequenced. DDR gene mutations
were found in 12 patients. Mutations were seen in BRCA1 (N = 7), BRCA2
(N = 5), PALB2 (N = 3), MSH2 (N = 1), and FANCF (N = 1) of all the DDR
genes sequenced. Median age was 65.5 years, 58\% were male, 97.2\% were
Caucasian and 51.4\% had any family history of cancer. The median OS was
near significantly superior in those with DDR gene mutations present
vs.~absent {[}14 vs.~5 months; HR, 0.58; 95\% confidence interval (CI),
0.29-1.14; log-rank P = 0.08{]}. Multivariate logistic (OR, 1.47; 95\%
CI, 1.04-2.06; P = 0.04) and Cox regression (HR, 0.37; 95\% CI,
0.15-0.94; P = 0.04) showed presence of DDR gene mutations was
associated with improved OS.Conclusions: In a single institution,
retrospective study, we found that the presence of DDR gene mutations
are associated with improved OS in PDAC patients treated with
FOLFIRINOX. Clin Cancer Res; 1-8. ©2018 AACR.

PubMed: \url{https://www.ncbi.nlm.nih.gov/pubmed/?term=30131383}

{}

{}

\begin{center}\rule{0.5\linewidth}{\linethickness}\end{center}

\begin{itemize}
\tightlist
\item
  \textbf{Genomic testing for pancreatic cancer in clinical practice as
  real-world evidence}
\end{itemize}

\url{https://www.sciencedirect.com/science/article/pii/S1424390318306331}

\begin{center}\rule{0.5\linewidth}{\linethickness}\end{center}

\begin{itemize}
\tightlist
\item
  \textbf{Improving the accuracy of pancreatic cancer clinical staging
  by exploitation of nanoparticle-blood interactions: A pilot study}
\end{itemize}

\url{https://www.sciencedirect.com/science/article/pii/S1424390318306070}

\begin{center}\rule{0.5\linewidth}{\linethickness}\end{center}

\begin{itemize}
\tightlist
\item
  \textbf{DNA profile components predict malignant outcomes in select
  cases of intraductal papillary mucinous neoplasm with negative
  cytology}
\end{itemize}

\url{https://www.sciencedirect.com/science/article/pii/S0039606018302836}

\begin{center}\rule{0.5\linewidth}{\linethickness}\end{center}

\begin{itemize}
\tightlist
\item
  \textbf{Stratification of Pancreatic Ductal Adenocarcinomas Based on
  Tumor and Microenvironment Features}
\end{itemize}

\url{https://www.sciencedirect.com/science/article/pii/S0016508518349199}

\begin{center}\rule{0.5\linewidth}{\linethickness}\end{center}

 - \textbf{Galectin-3 Mediates Tumor Cell-Stroma Interactions by
Activating Pancreatic Stellate Cells to Produce Cytokines via~Integrin
Signaling}

\emph{Gastroenterology 2018 04;154(5):1524-1537.e6}

BACKGROUND \& AIMS: Pancreatic ductal adenocarcinoma (PDAC) is
characterized by activated pancreatic stellate cells (PSCs), abundance
of extracellular matrix (ECM), and production of cytokines and
chemokines. Galectin 3 (GAL3), a β-galactoside-specific lectin,
contributes to PDAC development but its effects on the stroma and
cytokine production are unclear. METHODS: The effect of recombinant
human GAL3 (rGAL3) on activation of PSCs, production of cytokines, and
ECM proteins was determined by proliferation, invasion, cytokine array,
and quantitative polymerase chain reaction. We assessed co-cultures of
PDAC cells with GAL3 genetic alterations with PSCs. Production of
interleukin 8 (IL8) and activities of nuclear factor (NF)-κB were
determined by enzyme-linked immunosorbent assay and luciferase reporter
analyses. We studied the effects of inhibitors of NF-κB and
integrin-linked kinase (ILK) on pathways activated by rGAL3. RESULTS: In
analyses of the Gene Expression Omnibus database and our dataset, we
observed higher levels of GAL3, IL8, and other cytokines in PDAC than in
nontumor tissues. Production of IL8, granulocyte-macrophage
colony-stimulating factor, chemokine ligand 1, and C-C motif chemokine
ligand 2 increased in PSCs exposed to rGAL3 compared with controls.
Culture of PSCs with PDAC cells that express different levels of GAL3
resulted in proliferation and invasion of PSCs that increased with level
of GAL3. GAL3 stimulated transcription of IL8 through integrin subunit
beta 1 (ITGB1) on PSCs, which activates NF-κB through ILK. Inhibitors of
ILK or NF-κB or a neutralizing antibody against ITGB1 blocked
transcription and production of IL8 from PSCs induced by rGAL3. The GAL3
inhibitor significantly reduced growth and metastases of orthotopic
tumors that formed from PDAC and PSC cells co-implanted in mice.
CONCLUSION: GAL3 activates PSC cells to produce inflammatory cytokines
via ITGB1signaling to ILK and activation of NF-κB. Inhibition of this
pathway reduced growth and metastases of pancreatic orthotopic tumors in
mice.

PubMed: \url{https://www.ncbi.nlm.nih.gov/pubmed/?term=29274868}

{}

{}

\begin{center}\rule{0.5\linewidth}{\linethickness}\end{center}

 - \textbf{Possible Autocrine Function of~Galectin-3 in Pancreatic
Stellate Cells}

\emph{Gastroenterology 2018 09;155(3):933-934}

PubMed: \url{https://www.ncbi.nlm.nih.gov/pubmed/?term=30092185}

{}

{}

\begin{center}\rule{0.5\linewidth}{\linethickness}\end{center}

 - \textbf{Reply}

\emph{Gastroenterology 2018 09;155(3):934-935}

PubMed: \url{https://www.ncbi.nlm.nih.gov/pubmed/?term=30098926}

{}

{}

\begin{center}\rule{0.5\linewidth}{\linethickness}\end{center}

 - \textbf{Circulating Tumor Cells are an Independent Predictor of
Shorter Survival in Patients Undergoing Resection for Pancreatic and
Periampullary Adenocarcinoma}

\emph{Annals of surgery 2018 Sep;():}

OBJECTIVE: We evaluated the prognostic impact of circulating tumor cells
(CTCs) for patients with presumed resectable pancreatic and
periampullary cancers. SUMMARY OF BACKGROUND DATA: Initial treatment
decisions for this group are currently taken without a reliable
prognostic marker. The CellSearch system allows standardized CTC-testing
and has shown excellent specificity and prognostic value in other
applications. METHODS: Preoperative blood samples from 242 patients
between September 2009 and December 2014 were analyzed. One hundred
seventy-nine patients underwent tumor resection, of whom 30 with stage-I
tumors and duodenal cancer were assigned to the low-risk group, and the
others to the high-risk group. Further 33 had advanced disease, 30
benign histology. Observation ended in December 2016. Cancer-specific
survival (CSS) and disease-free survival (DFS) were calculated by
log-rank and Cox regression. RESULTS: CTCs (CTC-positive; ≥1 CTC/7.5 mL)
were detected in 6.8\% (10/147) of the high-risk patients and 6.2\%
(2/33) with advanced disease. No CTCs (CTC-negative) were detected in
the low-risk patients or benign disease. In high-risk patients, median
CSS for CTC-positive versus CTC-negative was 8.1 versus 20.0 months (P
\textless{} 0.0001), and DFS 4.0 versus 10.5 months (P \textless{}
0.001). Median CSS in advanced disease was 7.7 months. Univariate hazard
ratio (HR) of CTC-positivity was 3.4 (P \textless{} 0.001). In
multivariable analysis, CTC-status remained independent (HR: 2.4, P =
0.009) when corrected for histological type (HR: 2.7, P = 0.030), nodal
status (HR: 1.7, P = 0.016), and vascular infiltration (HR: 1.7, P =
0.001). CONCLUSION: Patients testing CTC-positive preoperatively showed
a detrimental outcome despite successful tumor resections. Although the
low CTC-rate seems a limiting factor, results indicate high specificity.
Thus, preoperative analysis of CTCs by this test may guide treatment
decisions and warrants further testing in clinical trials.

PubMed: \url{https://www.ncbi.nlm.nih.gov/pubmed/?term=30216219}

{}

{}

\begin{center}\rule{0.5\linewidth}{\linethickness}\end{center}

 - \textbf{Early Loss of Forkhead Transcription Factor, O Subgroup,
Member 1 Protein in the Development of Pancreatic Ductal Adenocarcinoma}

\emph{Pathobiology : journal of immunopathology, molecular and cellular
biology 2018 Sep;():1-6}

OBJECTIVES: Forkhead transcription factor, O subgroup, member 1 (FOXO1)
is a regulatory protein that plays an essential role in cellular
homeostasis. A biological function as a tumor suppressor has been
proposed. Here, we examined FOXO1 expression in human pancreatic ductal
adenocarcinoma (PDAC) and its precursor lesions. METHODS: We
immunohistochemically labeled tissue samples from 47 patients with PDAC
for FOXO1 protein. In addition, we extracted data from the Cancer Genome
Atlas and the Cancer Cell Line Encyclopedia and studied a potential
association with well-established genetic variants. A publicly available
microarray dataset of 102 PDAC samples was used to explore the influence
of FOXO1 expression on patients' clinical outcome. RESULTS: Normal
ductal epithelium universally expressed nuclear and cytoplasmic FOXO1.
Reduced expression was observed in PanIN lesions and PDAC of all cases.
Analysis of several datasets showed that the FOXO1 gene transcript
levels do not correlate with KRAS, TP53, SMAD4, or CDKN2A mutation
status, but positively correlate with patients' outcomes. CONCLUSIONS:
Loss of FOXO1 protein is identified as an early event during PDAC
development and may be independent of the top 4 mutated cancer genes.
Because of its strong expression in normal ductal cells,
immunohistochemical detection of FOXO1 can function as a valuable test
to establish the diagnosis of transformation and malignancy in
pancreatic tissues.

PubMed: \url{https://www.ncbi.nlm.nih.gov/pubmed/?term=30227407}

{}

{}

\begin{center}\rule{0.5\linewidth}{\linethickness}\end{center}

 - \textbf{Circulating Nucleic Acids Associate with Outcomes of Patients
with Pancreatic Cancer}

\emph{Gastroenterology 2018 Sep;():}

BACKGROUND \& AIMS: We aimed to investigate the clinical utility of
circulating tumor cell DNA (ctDNA) and exosome DNA (exoDNA) in
pancreatic cancer. METHODS: We collected liquid biopsies from 194
patients undergoing treatment for localized or metastatic pancreatic
adenocarcinoma from April 7, 2015 through October 13, 2017 (425 blood
samples collected before (baseline) and during therapy). Additional
liquid biopsies were collected from 37 disease controls. Droplet digital
PCR was used to determine KRAS mutant allele fraction (MAF) from ctDNA
and exoDNA purified from plasma. For the longitudinal analysis, we
analyzed exoDNA and ctDNA in 123 serial blood samples, from 34 patients.
We performed analysis including cox regression, Fisher exact test, and
Bayesian inference to associate KRAS MAFs in exoDNA and ctDNA with
prognostic and predictive outcomes. RESULTS: In the 34 patients with
potentially resectable tumors, an increase level of exoDNA following
neoadjuvant therapy significantly associated with disease progression
(P=.003), while ctDNA did not reveal correlations with outcomes.
Concordance rates of KRAS mutations present in surgically resected
tissue and detected in liquid biopsies was over 95\%. On univariate
analysis, patients with metastases and detectable ctDNA at baseline
status had significantly shorter times of progression-free survival
(hazard ratio {[}HR{]} for death, 1.8; 95\% CI, 1.1-3.0; P=.019), and
overall survival (OS) (HR, 2.8; 95\% CI, 1.4-5.7; P=.0045) compared to
patients without detectable ctDNA. On multivariate analysis, MAFs ≥5\%
in exoDNA was a significant predictor of progression-free survival (HR,
2.28; 95\% CI 1.18-4.40; P=.014) and OS (HR, 3.46; 95\% CI, 1.40-8.50;
P=.007). A multi-analyte approach revealed detection of both ctDNA and
exoDNA MAF ≥5\% at baseline status as a significant predictor of OS (HR
7.73, 95\% CI 2.61-22.91, P=0.00002) on multivariate analysis. In the
longitudinal analysis, a MAF peak above 1\% in exoDNA was significantly
associated with radiological progression (P=.0003). CONCLUSIONS: In a
prospective cohort of pancreatic cancer patients, we demonstrate how
longitudinal monitoring using liquid biopsies through exoDNA and ctDNA,
provides both predictive and prognostic information relevant towards
therapeutic stratification.

PubMed: \url{https://www.ncbi.nlm.nih.gov/pubmed/?term=30240661}

{}

{}

\begin{center}\rule{0.5\linewidth}{\linethickness}\end{center}

\begin{itemize}
\tightlist
\item
  \textbf{Next Generation Sequencing of the cellular and liquid fraction
  of pancreatic cyst fluid supports discrimination of IPMN from
  pseudocysts and reveals cases with multiple mutated driver clones ‐
  first findings from the prospective ZYSTEUS biomarker study}
\end{itemize}

\url{https://onlinelibrary.wiley.com/doi/abs/10.1002/gcc.22682}

\begin{center}\rule{0.5\linewidth}{\linethickness}\end{center}

\begin{itemize}
\tightlist
\item
  \textbf{ARID1A, a SWI/SNF subunit, is critical to acinar cell
  homeostasis and regeneration and is a barrier to transformation and
  epithelial-mesenchymal transition in the pancreas}
\end{itemize}

\url{https://gut.bmj.com/content/early/2018/09/18/gutjnl-2017-315541}

\begin{center}\rule{0.5\linewidth}{\linethickness}\end{center}

\begin{itemize}
\tightlist
\item
  \textbf{Gene expression analysis of embryonic pancreas development
  master regulators and terminal cell fate markers in resected
  pancreatic cancer: A correlation with clinical outcome}
\end{itemize}

\url{https://www.sciencedirect.com/science/article/pii/S1424390318306835}

\begin{center}\rule{0.5\linewidth}{\linethickness}\end{center}

\hypertarget{bile-ducts-2}{%
\subsubsection{Bile Ducts}\label{bile-ducts-2}}

\begin{center}\rule{0.5\linewidth}{\linethickness}\end{center}

 - \textbf{Recurrent Mutations in APC and CTNNB1 and Activated
Wnt/β-catenin Signaling in Intraductal Papillary Neoplasms of the Bile
Duct: A Whole Exome Sequencing Study}

\emph{The American journal of surgical pathology 2018 Sep;():}

This study aimed to elucidate the genetic landscape of biliary papillary
neoplasms. Of 28 cases examined, 7 underwent whole exome sequencing,
while the remaining 21 were used for validation studies with targeted
sequencing. In the whole exome sequencing study, 4/7 cases had mutations
in either APC or CTNNB1, both of which belong to the Wnt/β-catenin
pathway. Somatic mutations were also identified in genes involved in RAS
signaling (KRAS, BRAF), a cell cycle regulator (CDC27), histone
methyltransferase (KMT2C, KMT2D), and DNA mismatch repair (MSH3, MSH6,
PMS1). Combined with discovery and validation cohorts, mutations in APC
or CTNNB1 were observed in 6/28 subjects (21\%) and were mutually
exclusive. When the cases were classified into intraductal papillary
neoplasms of the bile duct (IPNBs, n=14) and papillary
cholangiocarcinomas (n=14) based on the recently proposed classification
criteria, mutations in APC and CTNNB1 appeared to be entirely restricted
to IPNBs with 6/14 cases (43\%) harboring mutations in either gene.
These genetic alterations were detected across the 3 nonintestinal
histologic types. In immunohistochemistry, the aberrant cytoplasmic
and/or nuclear expression of β-catenin was found in not only 5/6 IPNBs
with APC or CTNNB1 mutations, but also 6/8 cases with wild-type APC and
CTNNB1 (total 79\%). In addition, APC and CTNNB1 alterations were
exceptional in nonpapillary cholangiocarcinomas (n=29) with a single
case harboring CTNNB1 mutation (3\%). This study demonstrated recurrent
mutations in APC and CTNNB1 in nonintestinal-type IPNBs, suggesting that
activation of the Wnt/β-catenin signaling pathway is relevant to the
development and progression of IPNBs.

PubMed: \url{https://www.ncbi.nlm.nih.gov/pubmed/?term=30212390}

{}

{}

\begin{center}\rule{0.5\linewidth}{\linethickness}\end{center}

 - \textbf{Hypoxia-Induced PLOD2 is a Key Regulator in
Epithelial-Mesenchymal Transition and Chemoresistance in Biliary Tract
Cancer}

\emph{Annals of surgical oncology 2018 Nov;25(12):3728-3737}

BACKGROUND: The prognosis of biliary tract cancer (BTC) is unfavorable
due to its chemoresistance. Hypoxia triggers epithelial-to-mesenchymal
transition (EMT), which is closely related to drug resistance. In this
study, we focused on the functional roles of procollagen-lysine,
2-oxoglutarate 5-dioxygenase 2 (PLOD2), a hypoxia-induced gene, in BTC,
and assessed the clinical significance of PLOD2. METHODS: The expression
of PLOD2 under hypoxia was assessed in BTC cell lines.
Gemcitabine-resistant (GR) BTC cell lines were transfected with small
interfering RNA (siRNA) against PLOD2, and EMT markers and
chemoresistance were evaluated. PLOD2 expression was also characterized
using immunohistochemistry in BTC clinical specimens following
resection. Patient survival was analyzed and the role of PLOD2
expression was examined. RESULTS: The expression of PLOD2 was induced by
hypoxia in vitro and was upregulated in BTC-GR cell lines, which had low
expression of epithelial markers and high expression of mesenchymal
markers. Downregulation of PLOD2 by siRNA resulted in improved
chemoresistance, recovery of epithelial markers, and reduction of
mesenchymal markers. In the resected BTC samples, PLOD2 expression was
significantly correlated with lymph node metastasis (p = 0.037) and
stage (p = 0.001). Recurrence-free survival (p = 0.011) and overall
survival (p \textless{} 0.001) rates were significantly lower in
patients with high expression of PLOD2. PLOD2 expression was an
independent prognostic factor for overall survival (p = 0.019).
CONCLUSIONS: The expression of PLOD2 influenced chemoresistance through
EMT, and high expression of PLOD2 was a significant unfavorable
prognostic factor in BTC patients. PLOD2 might be a potential
therapeutic target for overcoming chemoresistance.

PubMed: \url{https://www.ncbi.nlm.nih.gov/pubmed/?term=30105440}

{}

{}

\begin{center}\rule{0.5\linewidth}{\linethickness}\end{center}

\begin{center}\rule{0.5\linewidth}{\linethickness}\end{center}

\begin{center}\rule{0.5\linewidth}{\linethickness}\end{center}

\hypertarget{neuroendocrine-2}{%
\subsubsection{Neuroendocrine}\label{neuroendocrine-2}}

\begin{center}\rule{0.5\linewidth}{\linethickness}\end{center}

\begin{itemize}
\tightlist
\item
  \textbf{Molecular Genetic Studies of Pancreatic Neuroendocrine Tumors}
\end{itemize}

\url{https://www.endo.theclinics.com/article/S0889-8529(18)30519-X/abstract}

\begin{center}\rule{0.5\linewidth}{\linethickness}\end{center}

Back to top

\pagebreak

\hypertarget{others}{%
\subsection{Others}\label{others}}

\begin{center}\rule{0.5\linewidth}{\linethickness}\end{center}

\hypertarget{pancreas-3}{%
\subsubsection{Pancreas}\label{pancreas-3}}

\begin{center}\rule{0.5\linewidth}{\linethickness}\end{center}

 - \textbf{Adjuvant Treatment in Potentially Curable Pancreatic Cancer:
Need to Include Tumor Location in the Equation?}

\emph{Pancreas 2018 Sep;47(8):e50-e52}

PubMed: \url{https://www.ncbi.nlm.nih.gov/pubmed/?term=30113430}

{}

{}

\begin{center}\rule{0.5\linewidth}{\linethickness}\end{center}

 - \textbf{Loss of PDPK1 abrogates resistance to gemcitabine in
label-retaining pancreatic cancer cells}

\emph{BMC cancer 2018 Jul;18(1):772}

BACKGROUND: Label-retaining cancer cells (LRCC) have been proposed as a
model of slowly cycling cancer stem cells (CSC) which mediate resistance
to chemotherapy, tumor recurrence, and metastasis. The molecular
mechanisms of chemoresistance in LRCC remain to-date incompletely
understood. This study aims to identify molecular targets in LRCC that
can be exploited to overcome resistance to gemcitabine, a standard
chemotherapy agent for the treatment of pancreas cancer. METHODS: LRCC
were isolated following Cy5-dUTP staining by flow cytometry from
pancreatic cancer cell lines. Gene expression profiles obtained from
LRCC, non-LRCC (NLRCC), and bulk tumor cells were used to generate
differentially regulated pathway networks. Loss of upregulated targets
in LRCC on gemcitabine sensitivity was assessed via RNAi experiments and
pharmacological inhibition. Expression patterns of PDPK1, one of the
upregulated targets in LRCC, was studied in patients' tumor samples and
correlated with pathological variables and clinical outcome. RESULTS:
LRCC are significantly more resistant to gemcitabine than the bulk tumor
cell population. Non-canonical EGF (epidermal growth factor)-mediated
signal transduction emerged as the top upregulated network in LRCC
compared to non-LRCC, and knock down of EGF signaling effectors PDPK1
(3-phosphoinositide dependent protein kinase-1), BMX (BMX non-receptor
tyrosine kinase), and NTRK2 (neurotrophic receptor tyrosine kinase 2) or
treatment with PDPK1 inhibitors increased growth inhibition and
induction of apoptosis in response to gemcitabine. Knockdown of PDPK1
preferentially increased growth inhibition and reduced resistance to
induction of apoptosis upon gemcitabine treatment in the LRCC vs
non-LRCC population. These findings are accompanied by lower expression
levels of PDPK1 in tumors compared to matched uninvolved pancreas in
surgical resection specimens and a negative association of membranous
localization on IHC with high nuclear grade (p \textless{} 0.01).
CONCLUSION: Pancreatic cancer cell-derived LRCC are relatively resistant
to gemcitabine and harbor a unique transcriptomic profile compared to
bulk tumor cells. PDPK1, one of the members of an upregulated
EGF-signaling network in LRCC, mediates resistance to gemcitabine, is
found to be dysregulated in pancreas cancer specimens, and might be an
attractive molecular target for combination therapy studies.

PubMed: \url{https://www.ncbi.nlm.nih.gov/pubmed/?term=30064387}

{}

{}

\begin{center}\rule{0.5\linewidth}{\linethickness}\end{center}

 - \textbf{Intraductal papillary mucinous neoplasms of the pancreas - a
cost-effectiveness analysis of management strategies for the branch-duct
subtype}

\emph{HPB : the official journal of the International Hepato Pancreato
Biliary Association 2018 Jul;():}

BACKGROUND: Branch-duct intraductal papillary mucinous neoplasm
(BD-IPMN) presents a clinical conundrum. Rigorous long-term surveillance
or surgical resection is recommended. The economic consequences of the
management have not been fully investigated. METHODS: A Markov decision
model compared 4 strategies for low-risk BD-IPMN: I~=~upfront total
pancreatectomy, II~=~upfront partial pancreatectomy, III~=~initial
surveillance, IV~=~watchful waiting. Surveillance was based on the
Swedish Guidelines for Pancreatic Cancer. Probabilities and costs were
obtained from the participating unit and from the scientific literature.
The incremental cost-effectiveness ratios (ICERs) were calculated and
sensitivity analyses were performed by varying relevant parameters.
Survival was reported in quality-adjusted life-years (QALYs). RESULTS:
Strategy III was the most cost-effective strategy with an ICER of €31
682 compared to strategy IV. Strategy I was the most expensive but
yielded the best QALY (9.32). Total number of years, annual risk of
pancreatic cancer and annual risk of a low-risk BD-IPMN turning into a
high-risk lesion had the greatest impact in the model. CONCLUSIONS:
Initial surveillance seems to be the most cost-effective strategy in the
management of low-risk asymptomatic BD-IPMN. However, the possibility of
personalized approaches remains to be investigated.

PubMed: \url{https://www.ncbi.nlm.nih.gov/pubmed/?term=30064727}

{}

{}

\begin{center}\rule{0.5\linewidth}{\linethickness}\end{center}

 - \textbf{The Lymph Node Ratio Is an Independent Prognostic Factor in
Pancreatic Cancer Patients Who Receive Curative Resection Followed by
Adjuvant Chemotherapy}

\emph{Anticancer research 2018 Aug;38(8):4877-4882}

BACKGROUND/AIM: The present study investigated the impact of the lymph
node ratio (LNR) on survival and recurrence in patients with pancreatic
cancer after curative surgery followed by adjuvant chemotherapy.
PATIENTS AND METHODS: This study included 189 patients who underwent
curative surgery followed by adjuvant chemotherapy for pancreatic cancer
between 2005 and 2014. The risk factors for overall survival (OS) and
recurrence-free survival (RFS) were identified. RESULTS: A lymph node
ratio of 0.1 was considered to be the optimal cut-off point for
classification based on the 3-year and 5-year survival rates. The OS
rates at three and five years after surgery were 34.4\% and 28.2\% in
the LNR \textless{}0.1 group, respectively, and 23.1\% and 5.8\% in the
LNR ≥0.1 group, which amounted to a statistically significant difference
(p=0.003). The RFS rates at one and three years after surgery were
26.6\% and 20.5\% in the LNR \textless{}0.1 group, respectively, and
8.0\% and 0\% in the LNR ≥0.1 group, which was a significant difference
(p=0.001). A multivariate analysis demonstrated that the LNR was a
significant independent risk factor for both the OS and RFS. CONCLUSION:
The LNR was a risk factor for overall survival in patients who underwent
curative surgery followed by adjuvant chemotherapy for pancreatic
cancer. It is necessary to develop strategies to effectively utilize the
lymph node metastasis status.

PubMed: \url{https://www.ncbi.nlm.nih.gov/pubmed/?term=30061263}

{}

{}

\begin{center}\rule{0.5\linewidth}{\linethickness}\end{center}

 - \textbf{Well differentiated liposarcoma, sclerosing type, of the
pancreas a case report}

\emph{Experimental and molecular pathology 2016 12;101(3):320-322}

PubMed: \url{https://www.ncbi.nlm.nih.gov/pubmed/?term=27840110}

{}

{}

\begin{center}\rule{0.5\linewidth}{\linethickness}\end{center}

 - \textbf{Growth rate of serous pancreatic neoplasms in~vivo: a
retrospective, observational study}

\emph{Acta radiologica (Stockholm, Sweden : 1987) 2018
Jul;():284185118787350}

Background Determining the growth rate of pancreatic cystic lesions on
follow-up imaging is important in managing patients with these lesions.
However, the growth rates of serous pancreatic neoplasms (SPNs) have
been reported to vary among studies. Purpose To determine the in~vivo
growth rate of SPNs. Material and Methods This retrospective,
single-institutional study included patients diagnosed with SPNs during
2006-2015. The diagnosis of SPNs was based on the results of surgery,
endoscopic ultrasonography (EUS)-guided fine needle aspiration (FNA) or
core needle biopsy (CNB), or typical radiologic features of SPN. A
linear mixed-effects model was utilized to determine whether the
diagnostic intervention was associated with tumor growth rate in all
patients. The in~vivo growth rate of SPNs was estimated from patients
without or before diagnostic intervention. SPN growth rates were
compared before and after diagnostic intervention. Results SPN growth
rates in the overall patient cohort (n = 304) differed significantly
between patients who did and did not undergo diagnostic interventions.
The in~vivo SPN growth rate in 204 patients without or before diagnostic
intervention was 1.9 mm/year (95\% confidence interval
{[}CI{]} = 1.6-2.2). In the 130 patients who underwent diagnostic
intervention, the SPN growth rate was significantly greater before than
after diagnostic intervention (1.8 vs.~0.2 mm/year). Conclusions In the
absence of diagnostic intervention, the in~vivo growth rate of SPNs was
1.9 mm/year (95\% CI = 1.6-2.2). EUS-guided FNA or CNB may affect the
growth rate of SPNs.

PubMed: \url{https://www.ncbi.nlm.nih.gov/pubmed/?term=30056739}

{}

{}

\begin{center}\rule{0.5\linewidth}{\linethickness}\end{center}

 - \textbf{Progression to pancreatic ductal adenocarcinoma from
pancreatic intraepithelial neoplasia: Results of a simulation model}

\url{https://www.pancreatology.net/article/S1424-3903(18)30636-7/abstract}

\begin{center}\rule{0.5\linewidth}{\linethickness}\end{center}

 - \textbf{Serotonin uptake is required for Rac1 activation in
Kras-induced acinar-to-ductal metaplasia in the pancreas}

\emph{The Journal of pathology 2018 Nov;246(3):352-365}

Pancreatic ductal adenocarcinoma (PDAC), which is the primary cause of
pancreatic cancer mortality, is poorly responsive to currently available
interventions. Identifying new targets that drive PDAC formation and
progression is critical for developing alternative therapeutic
strategies to treat this lethal malignancy. Using genetic and
pharmacological approaches, we investigated in vivo and in vitro whether
uptake of the monoamine serotonin {[}5-hydroxytryptamine (5-HT){]} is
required for PDAC development. We demonstrated that pancreatic acinar
cells have the ability to readily take up 5-HT in a transport-mediated
manner. 5-HT uptake promoted activation of the small GTPase Ras-related
C3 botulinum toxin substrate 1 (Rac1), which is required for
transdifferentiation of acinar cells into acinar-to-ductal metaplasia
(ADM), a key determinant in PDAC development. Consistent with the
central role played by Rac1 in ADM formation, inhibition of the 5-HT
transporter Sert (Slc6a4) with fluoxetine reduced ADM formation both in
vitro and in vivo in a cell-autonomous manner. In addition, fluoxetine
treatment profoundly compromised the stromal reaction and affected the
proliferation and lipid metabolism of malignant PDAC cells. We propose
that Sert is a promising therapeutic target to counteract the early
event of ADM, with the potential to stall the initiation and progression
of pancreatic carcinogenesis. Copyright © 2018 Pathological Society of
Great Britain and Ireland. Published by John Wiley \& Sons, Ltd.

PubMed: \url{https://www.ncbi.nlm.nih.gov/pubmed/?term=30058725}

{}

{}

\begin{center}\rule{0.5\linewidth}{\linethickness}\end{center}

 - \textbf{Clinical Features and Prognosis of Patients With the Bone
Metastasis of Pancreatic Cancer: A Single-Institutional Cohort Study}

\emph{Pancreas 2018 Aug;47(7):e43-e46}

PubMed: \url{https://www.ncbi.nlm.nih.gov/pubmed/?term=29985850}

{}

{}

\begin{center}\rule{0.5\linewidth}{\linethickness}\end{center}

 - \textbf{Spatial Distribution of Pancreatic Stones in Chronic
Pancreatitis}

\emph{Pancreas 2018 Aug;47(7):864-870}

OBJECTIVES: The aim of this study was to establish a standard to
describe the spatial distribution of pancreatic stones in chronic
pancreatitis (CP). METHODS: Two hundred forty-seven CP patients with
pancreatic stones from June to December 2012 were enrolled.
Two-dimensional images from coronal projection of 3-dimensional computed
tomography images of pancreatic stones were gained. The number (n) of
all stones and the geometric standard deviation (σ) of distances between
the centroid of all stones and the centroids of every stone that
represented the spatial distribution nonuniformity were calculated by
Stone Reconstruction and Identification Programming System. RESULTS: The
mean value of n and σ were 13.6 and 22.5; n \textgreater{} 13.6 and σ
\textgreater{} 22.5 were determined as ``multistones'' and
``nonuniform,'' respectively. Compared with alcoholic CP, idiopathic CP
was less prone to multistones (odds ratio {[}OR{]}, 0.310) and more
prone to nonuniform (OR, 3.247). Pancreatic pseudocyst (OR, 2.211) in CP
course was a risk factor of multistones, whereas diabetes mellitus in
first-/second-/third-degree relatives (OR, 0.382) was a protective
factor. Age at diagnosis of pancreatic stones (OR, 1.022) was a risk
factor of nonuniformity. CONCLUSIONS: Compared with idiopathic CP,
alcoholic CP patients were prone to more pancreatic stones that
distribute more uniformly.

PubMed: \url{https://www.ncbi.nlm.nih.gov/pubmed/?term=29975348}

{}

{}

\begin{center}\rule{0.5\linewidth}{\linethickness}\end{center}

 - \textbf{Sweet Predictions Speak Volumes for Early Detection of
Pancreatic Cancer}

\emph{Gastroenterology 2018 08;155(2):265-268}

PubMed: \url{https://www.ncbi.nlm.nih.gov/pubmed/?term=29966611}

{}

{}

\begin{center}\rule{0.5\linewidth}{\linethickness}\end{center}

 - \textbf{Implications of the Pattern of Disease Recurrence on Survival
Following Pancreatectomy for Pancreatic Ductal Adenocarcinoma}

\emph{Annals of surgical oncology 2018 Aug;25(8):2475-2483}

BACKGROUND: After radical resection of pancreatic ductal adenocarcinoma
(PDAC), approximately 80\% of patients will develop disease recurrence.
It remains unclear to what extent the location of recurrence carries
prognostic significance. Additionally, stratifying the pattern of
recurrence may lead to a deeper understanding of the heterogeneous
biological behavior of PDAC. OBJECTIVE: The aim of this study was to
characterize the relationship of recurrence patterns with survival in
patients with resected PDAC. METHODS: This single-center cohort study
included patients undergoing pancreatectomy at the Johns Hopkins
Hospital between 2000 and 2013. Exclusion criteria were neoadjuvant
therapy and incomplete follow-up. Sites of first recurrence were
stratified into five groups and survival outcomes were estimated using
Kaplan-Meier curves. The association of specific recurrence locations
with overall survival (OS) was analyzed using Cox proportional-hazards
models with and without landmark analysis. RESULTS: Accurate follow-up
data were available for 877 patients, 662 (75.5\%) of whom had
documented recurrence at last follow-up. Patients with multiple-site
(n = 227, 4.7~months) or liver-only recurrence (n = 166, 7.2~months) had
significantly worse median survival after recurrence when compared with
lung- (n = 93) or local-only (n = 158) recurrence (15.4 and 9.7~months,
respectively). On multivariable analysis, the unique recurrence patterns
had variable predictive values for OS. Landmark analyses, with landmarks
set at 12, 18, and 24~months, confirmed these findings. CONCLUSIONS:
This study demonstrates that specific patterns of PDAC recurrence result
in different survival outcomes. Furthermore, distinct first recurrence
locations have unique independent predictive values for OS, which could
help with prognostic stratification and decisions regarding treatment
after the diagnosis of recurrence.

PubMed: \url{https://www.ncbi.nlm.nih.gov/pubmed/?term=29948425}

{}

{}

\begin{center}\rule{0.5\linewidth}{\linethickness}\end{center}

 - \textbf{Vanishing Pancreas}

\emph{Gastroenterology 2018 08;155(2):280-281}

PubMed: \url{https://www.ncbi.nlm.nih.gov/pubmed/?term=29409874}

{}

{}

\begin{center}\rule{0.5\linewidth}{\linethickness}\end{center}

 - \textbf{Don't Mess With the Pancreas (Wherever It May Be): Acute
Pancreatic Rest``itis'' Presenting as a Submucosal Mass With Gastric
Outlet Obstruction}

\emph{Gastroenterology 2018 08;155(2):e1-e2}

PubMed: \url{https://www.ncbi.nlm.nih.gov/pubmed/?term=29409828}

{}

{}

\begin{center}\rule{0.5\linewidth}{\linethickness}\end{center}

\begin{itemize}
\tightlist
\item
  \textbf{The Use of Biomarkers in the Risk Stratification of Cystic
  Neoplasms}
\end{itemize}

\url{https://www.giendo.theclinics.com/article/S1052-5157(18)30725-6/abstract}

\begin{center}\rule{0.5\linewidth}{\linethickness}\end{center}

\begin{itemize}
\tightlist
\item
  \textbf{Current Guideline Controversies in the Management of
  Pancreatic Cystic Neoplasms}
\end{itemize}

\url{https://www.giendo.theclinics.com/article/S1052-5157(18)30724-4/abstract}

\begin{center}\rule{0.5\linewidth}{\linethickness}\end{center}

\begin{itemize}
\tightlist
\item
  \textbf{To resect or not to resect: a review of pancreatic cyst
  disease management}
\end{itemize}

\url{https://journals.lww.com/co-gastroenterology/Abstract/2018/09000/To_resect_or_not_to_resect___a_review_of.13.aspx}

\begin{center}\rule{0.5\linewidth}{\linethickness}\end{center}

 - \textbf{Epithelial-Mesenchymal Transition in Pancreatic Cancer: A
Review}

\emph{BioMed research international 2017 ;2017():2646148}

Pancreatic ductal adenocarcinoma (PDAC) is one of the most aggressive
solid malignancies and is characterized by its insensitivity to current
therapy. The invasion and metastasis of solid tumors such as PDAC are
complex processes involving many factors. Recent insights into the role
of cancer stem cells (CSCs) and the epithelial-mesenchymal transition
(EMT) in tumorigenesis have increased the knowledge base and highlighted
new therapeutic targets of this disease. The process of EMT is regulated
by a complex network of cytokines, transcription factors, growth
factors, signaling pathways, and the tumor microenvironment, exhibiting
CSC-like properties. The transition of solid cancer cells from an
epithelial to a mesenchymal phenotype increases their migratory and
invasive properties, thus promoting metastasis. In PDAC, the exact
influence of EMT on the biological behaviors of cancer cells and its
impact on clinical therapy remain controversial, but the therapeutic
strategy of combining EMT inhibition with chemotherapy deserves
attention. Alternatively, anti-inflammatory therapy that targets the
interaction between inflammation and EMT is a valid strategy for
treating the premalignant stage of tumor progression. In this review, we
summarize the latest research on EMT and the potential relationship
between EMT and PDAC.

PubMed: \url{https://www.ncbi.nlm.nih.gov/pubmed/?term=29379795}

{}

{}

\begin{center}\rule{0.5\linewidth}{\linethickness}\end{center}

 - \textbf{A Phase II Clinical Trial of Molecular Profiled Neoadjuvant
Therapy for Localized Pancreatic Ductal Adenocarcinoma}

\emph{Annals of surgery 2018 Oct;268(4):610-619}

OBJECTIVES: One facet of precision medicine is the use of tumor
molecular profiling to guide chemotherapeutic selection. We conducted
the first prospective clinical trial of molecular profiling to guide
neoadjuvant therapy in patients with operable pancreatic ductal
adenocarcinoma (PDAC). We hypothesized that more effective systemic
therapy would prevent disease progression during neoadjuvant therapy
and, therefore, allow more patients to undergo surgery. METHODS: In
patients with resectable and borderline resectable (BLR) PDAC, molecular
profiling consisted of immunocytochemical staining of pretreatment
endoscopic ultrasound-guided fine needle aspiration tumor biopsies using
6 biomarkers. Neoadjuvant systemic therapy was selected based on the
molecular profiling results. The primary endpoint was the completion of
all intended neoadjuvant therapy and surgery. RESULTS: The trial
enrolled 130 patients; 61 (47\%) resectable and 69 (53\%) BLR. Molecular
profiling was reported within a median of 5 business days (IQR: 3). Of
the 130 patient samples, 95 (73\%) had adequate cellularity for
molecular profiling and 92 (71\%) patients received molecular
profile-directed therapy. Of the 92 patients who had predictive
profiling, 74 (80\%) received fluoropyrimidine-based therapy and 18
(20\%) received gemcitabine-based therapies. Of the 130 patients, 107
(82\%) completed all intended neoadjuvant therapy and surgery; 56 (92\%)
of the 61 with resectable PDAC and 51 (74\%) of 69 with BLR PDAC.
CONCLUSIONS: We report the first prospective clinical trial that
utilized molecular profiling to select neoadjuvant therapy in patients
with operable PDAC. Such high resectability rates have not been observed
in prior neoadjuvant trials, suggesting that molecular profiling may
improve the efficacy of chemotherapy in these patients.

PubMed: \url{https://www.ncbi.nlm.nih.gov/pubmed/?term=30080723}

{}

{}

\begin{center}\rule{0.5\linewidth}{\linethickness}\end{center}

 - \textbf{Functions of the CXC ligand family in the pancreatic tumor
microenvironment}

\emph{Pancreatology : official journal of the International Association
of Pancreatology (IAP) \ldots{} {[}et al.{]} 2018 Oct;18(7):705-716}

Therapeutic resistance is the major contributor to the poor prognosis of
and low survival from pancreatic cancer (PC). Cancer progression is a
complex process reliant on interactions between the tumor and the tumor
microenvironment (TME). Members of the CXCL family of chemokines are
present in the pancreatic TME and seem to play a vital role in
regulating PC progression. As pancreatic tumors interact with the TME
and with PC stem cells (CSCs), determining the roles of specific members
of the CXCL family is vital to the development of improved therapies.
This review highlights the roles of selected CXCLs in the interactions
between pancreatic tumor and its stroma, and in CSC phenotypes, which
can be used to identify potential treatment targets.

PubMed: \url{https://www.ncbi.nlm.nih.gov/pubmed/?term=30078614}

{}

{}

\begin{center}\rule{0.5\linewidth}{\linethickness}\end{center}

 - \textbf{Retrospective evaluation of patients diagnosed solid
pseudopapillary neoplasms of the pancreas}

\emph{Current problems in cancer 2018 Jul;():}

PURPOSE: Solid pseudopapillary neoplasm (SPN) is a rare, low-grade
neoplasm with excellent prognosis. In this study, we evaluated
clinicopathological characteristics of patients diagnosed with SPN
retrospectively. METHODS: This is a retrospective study intended to
characterize patients with the diagnosis of SPN between 2005 and 2015.
Clinicopathological features, recurrence rate, and overall survival of
28 patients were recorded. Malignant SPN criteria were defined as the
presence of distant metastasis (developed at diagnosis or during follow
up) or lymph node involvement. RESULTS: The mean age at diagnosis was 42
(range: 17-41). Among patients, 82\% (n\,=\,23) were female and 17.9\%
(n\,=\,5) were male. The mean size of tumor was 5.81 cm (range: 2-15).
The mean follow up period was 55.6 months, 1-year survival was 96.5\%
and 5-year survival rate was 88\%. A total of 25 patients were alive at
the end of follow-up period and 3 of the patients became exitus due to
disease. Two patients had a metastatic presentation in livers at the
diagnosis and metastasis developed in 3 patients during follow-up (liver
of 1 patient, peritoneum in 1 patient and liver and peritoneum in 1
patient). The reason of admission was headache in 68\% patients. The
type of operation was frequently subtotal pancreatectomy (n\,=\,11,
39.3\%) and distal pancreatectomy (n\,=\,10, 35.7\%). Tumors were
located frequently in body and tail regions (n\,=\,18, 64.3\%) and the
number of patients with malignant criteria was 6 (21.4\%). Although the
mean age of malignant patients was significantly higher than benign
patients (P\,=\,0.046), there was no significant difference between 2
groups in terms of gender, tumor size, capsule invasion, perineural
invasion, vascular invasion, and margin status. CONCLUSION: SPN is a
rarely seen tumor with low malignity potential. Surgical resection
provides long-term survival rate even in local invasion or metastasis
conditions.

PubMed: \url{https://www.ncbi.nlm.nih.gov/pubmed/?term=30104029}

{}

{}

\begin{center}\rule{0.5\linewidth}{\linethickness}\end{center}

\begin{itemize}
\tightlist
\item
  \textbf{Immune Checkpoint Inhibition for Pancreatic Ductal
  Adenocarcinoma: Current Limitations and Future Options}
\end{itemize}

\url{https://www.frontiersin.org/articles/10.3389/fimmu.2018.01878/full}

\begin{center}\rule{0.5\linewidth}{\linethickness}\end{center}

 - \textbf{Gut Microbiota Promotes Tumor Growth in Mice by Modulating
Immune Response}

\emph{Gastroenterology 2018 07;155(1):33-37.e6}

We studied the effects of gut microbiome depletion by oral antibiotics
on tumor growth in subcutaneous and liver metastases models of
pancreatic cancer, colon cancer, and melanoma. Gut microbiome depletion
significantly reduced tumor burden in all the models tested. However,
depletion of gut microbiome did not reduce tumor growth in Rag1-knockout
mice, which lack mature T and B cells. Flow cytometry analyses
demonstrated that gut microbiome depletion led to significant increase
in interferon gamma-producing T cells with corresponding decrease in
interleukin 17A and interleukin 10-producing T cells. Our results
suggest that gut microbiome modulation could emerge as a novel
immunotherapeutic strategy.

PubMed: \url{https://www.ncbi.nlm.nih.gov/pubmed/?term=29630898}

{}

{}

\begin{center}\rule{0.5\linewidth}{\linethickness}\end{center}

 - \textbf{Gut microbiome-immune crosstalk affects progression of
cancer}

\emph{Translational gastroenterology and hepatology 2018 ;3():34}

PubMed: \url{https://www.ncbi.nlm.nih.gov/pubmed/?term=30050994}

{}

{}

\begin{center}\rule{0.5\linewidth}{\linethickness}\end{center}

\begin{itemize}
\tightlist
\item
  \textbf{Intrapancreatic recurrence of intraductal tubulopapillary
  neoplasm (ITPN) 16 years after the initial surgery for noninvasive
  ITPN: a case report}
\end{itemize}

\url{https://surgicalcasereports.springeropen.com/articles/10.1186/s40792-018-0497-1}

\begin{center}\rule{0.5\linewidth}{\linethickness}\end{center}

 - \textbf{Pancreatic cystosis in patients with cystic fibrosis: A
qualitative systematic review}

\emph{Pancreatology : official journal of the International Association
of Pancreatology (IAP) \ldots{} {[}et al.{]} 2018 Oct;18(7):700-704}

BACKGROUND: Patients with cystic fibrosis (CF) and a CFTR gene mutation
may present with a variety of pancreatic disorders. The presence of
multiple macrocysts (\textgreater{}1\,cm) replacing the entire
pancreatic parenchyma is termed pancreatic cystosis. Lack of clear data
makes clinical decision making challenging and controversial. The aim of
this review is to perform a qualitative systematic analysis of the
literature with intention to evaluate management plans. METHODS:
Electronic databases MEDLINE, Embase, and Scopus were searched for
relevant studies, and 19 studies describing patients with pancreatic
cystosis were included and analyzed for clinical features and therapy
offered. RESULTS: The data of 24 patients were collected from included
studies. Eight cases (33\%) had a documented CFTR gene mutation and 10
(42\%) were symptomatic at presentation. Imaging modalities included
ultrasound in 18 (75\%), CT in 12 (50\%), and MRI in 8 (33\%) cases. An
average size of the largest cyst was 5.4\,cm. 6 (25\%) patients were
offered therapy that described surgical (3), endoscopic (1), or medical
therapy (2). Surgeries offered included total pancreatectomy, partial
pancreatic resection of uncertain extent, and complex cyst resection.
Endoscopic treatment was cystogastrostomy. Novel medical treatment was
utilized with Doxepin, Propantheline, and Clonidine, resulting in
reduction in cyst size and overall clinical improvement. CONCLUSION:
Patients with pancreatic cystosis should not be denied treatment when
necessary. This literature review is the most comprehensive thus far of
cystic fibrosis and pancreatic cystosis, and it did not provide
identification of a definitive treatment plan or demonstrate
contraindication to specific therapies.

PubMed: \url{https://www.ncbi.nlm.nih.gov/pubmed/?term=30139657}

{}

{}

\begin{center}\rule{0.5\linewidth}{\linethickness}\end{center}

\begin{itemize}
\tightlist
\item
  \textbf{The Gut Microbiome in Pancreatic Disease}
\end{itemize}

\url{https://www.sciencedirect.com/science/article/pii/S1542356518308838}

\begin{center}\rule{0.5\linewidth}{\linethickness}\end{center}

 - \textbf{HNF1A is a novel oncogene that regulates human pancreatic
cancer stem cell properties}

\emph{eLife 2018 Aug;7():}

The biological properties of pancreatic cancer stem cells (PCSCs) remain
incompletely defined and the central regulators are unknown. By
bioinformatic analysis of a human PCSC-enriched gene signature, we
identified the transcription factor HNF1A as a putative central
regulator of PCSC function. Levels of HNF1A and its target genes were
found to be elevated in PCSCs and tumorspheres, and depletion of HNF1A
resulted in growth inhibition, apoptosis, impaired tumorsphere
formation, decreased PCSC marker expression, and downregulation of
POU5F1/OCT4 expression. Conversely, HNF1A overexpression increased PCSC
marker expression and tumorsphere formation in pancreatic cancer cells
and drove pancreatic ductal adenocarcinoma (PDA) cell growth.
Importantly, depletion of HNF1A in xenografts impaired tumor growth and
depleted PCSC marker-positive cells in vivo. Finally, we established an
HNF1A-dependent gene signature in PDA cells that significantly
correlated with reduced survivability in patients. These findings
identify HNF1A as a central transcriptional regulator of PCSC properties
and novel oncogene in PDA.

PubMed: \url{https://www.ncbi.nlm.nih.gov/pubmed/?term=30074477}

{}

{}

\begin{center}\rule{0.5\linewidth}{\linethickness}\end{center}

\begin{itemize}
\tightlist
\item
  \textbf{Genomic Profiling and Potentially Targetable Alterations in
  Pancreatic Ductal Adenocarcinoma}
\end{itemize}

\url{https://link.springer.com/article/10.1007/s11938-018-0195-x}

\begin{center}\rule{0.5\linewidth}{\linethickness}\end{center}

\begin{itemize}
\tightlist
\item
  \textbf{Real-time Genomic Characterization of Advanced Pancreatic
  Cancer to Enable Precision Medicine}
\end{itemize}

\url{http://cancerdiscovery.aacrjournals.org/content/8/9/1096?iss=9}

\begin{center}\rule{0.5\linewidth}{\linethickness}\end{center}

\begin{itemize}
\tightlist
\item
  \textbf{Organoid Profiling Identifies Common Responders to
  Chemotherapy in Pancreatic Cancer}
\end{itemize}

\url{http://cancerdiscovery.aacrjournals.org/content/8/9/1112?iss=9}

\begin{center}\rule{0.5\linewidth}{\linethickness}\end{center}

\begin{itemize}
\tightlist
\item
  \textbf{Lymph Node Ratio in Pancreatic Adenocarcinoma After
  Preoperative Chemotherapy vs.~Preoperative Chemoradiation and Its
  Utility in Decisions About Postoperative Chemotherapy}
\end{itemize}

\url{https://link.springer.com/article/10.1007/s11605-018-3953-0}

\begin{center}\rule{0.5\linewidth}{\linethickness}\end{center}

 - \textbf{p21-activated kinase signalling in pancreatic cancer: New
insights into tumour biology and immune modulation}

\emph{World journal of gastroenterology 2018 Sep;24(33):3709-3723}

Pancreatic cancer is one of the most aggressive and lethal malignancies
worldwide, with a very poor prognosis and a five-year survival rate less
than 8\%. This dismal outcome is largely due to delayed diagnosis, early
distant dissemination and resistance to conventional chemo-therapies.
Kras mutation is a well-defined hallmark of pancreatic cancer, with over
95\% of cases harbouring Kras mutations that give rise to constitutively
active forms of Kras. As important down-stream effectors of Kras,
p21-activated kinases (PAKs) are involved in regulating cell
proliferation, apoptosis, invasion/migration and chemo-resistance.
Immunotherapy is now emerging as a promising treatment modality in the
era of personalized anti-cancer therapeutics. In this review, basic
knowledge of PAK structure and regulation is briefly summarised and the
pivotal role of PAKs in Kras-driven pancreatic cancer is highlighted in
terms of tumour biology and chemo-resistance. Finally, the involvement
of PAKs in immune modulation in the tumour microenvironment is discussed
and the potential advantages of targeting PAKs are explored.

PubMed: \url{https://www.ncbi.nlm.nih.gov/pubmed/?term=30197477}

{}

{}

\begin{center}\rule{0.5\linewidth}{\linethickness}\end{center}

\begin{itemize}
\tightlist
\item
  \textbf{Fluorescent‐guided surgery for sentinel lymph node detection
  in gastric cancer and carcinoembryonic antigen targeted
  fluorescent‐guided surgery in colorectal and pancreatic cancer}
\end{itemize}

\url{https://onlinelibrary.wiley.com/doi/10.1002/jso.25139}

\begin{center}\rule{0.5\linewidth}{\linethickness}\end{center}

 - \textbf{Does Autophagy Promote or~Protect Against the Pathogenesis of
Pancreatitis?}

\emph{Gastroenterology 2018 10;155(4):1273-1274}

PubMed: \url{https://www.ncbi.nlm.nih.gov/pubmed/?term=30222940}

{}

{}

\begin{center}\rule{0.5\linewidth}{\linethickness}\end{center}

\begin{itemize}
\tightlist
\item
  \textbf{Pentose conversions support the tumorigenesis of pancreatic
  cancer distant metastases}\\
  Oncogenevolume 37, pages5248--5256 (2018)
\end{itemize}

\url{https://www.nature.com/articles/s41388-018-0346-5}

\begin{verbatim}
30236651
\end{verbatim}

\begin{center}\rule{0.5\linewidth}{\linethickness}\end{center}

 - \textbf{Diagnosis, treatment and long-term outcome of autoimmune
pancreatitis in Sweden}

\emph{Pancreatology : official journal of the International Association
of Pancreatology (IAP) \ldots{} {[}et al.{]} 2018 Sep;():}

INTRODUCTION: Autoimmune pancreatitis (AIP) is a pancreatic inflammatory
process characterized by a strong inflammatory cell infiltration and two
histopathologically distinct subtypes: type 1 and type 2. Diagnosis is
often challenging and requires a combination of clinical, laboratory and
imaging data. AIP can mimic pancreatic tumours leading to unnecessary
resections if not correctly diagnosed. Short- and long-term outcomes of
AIP have been poorly investigated so far and no large series have been
previously reported from Sweden. METHODS: A single-centre,
retrospective, cohort study of patients with histologically confirmed or
highly probable diagnosis of AIP according to ICDC criteria.
Demographic, clinical and radiological characteristics, type of
treatment and its outcomes were collected and analysed. RESULTS:
Seventy-one patients with AIP (87\% with type 1), were evaluated at
Karolinska University Hospital between 2004 and 2018; 49\% males, mean
age 49 years (range 44-53). Among them, 28\% were histologically
confirmed, 35\% presented with jaundice, 22\% with acute pancreatitis,
39\% had non-specific symptoms such as weight loss or abdominal pain,
84\% showed other organ involvement (OOI). Radiologically, 76\% showed a
focal pancreatic enlargement, 27\% diffuse enlargement, 27\% signs of
acute pancreatitis and 10\% of chronic pancreatitis. Overall, 58
patients (81\%) underwent treatment with different medications: 46
(79\%) cortisone, 7 (12\%) azathioprine, 5 (8\%) other immunosuppressive
drugs. Twenty-six (36\%) underwent biliary stenting and 12 (16\%) were
given surgery. In total, 47\% of patients developed pancreatic exocrine
insufficiency (PEI), of whom 76\% had a severe form (faecal
elastase-1\,\textless{}\,100\,μg/g) and 21\% of patients developed
diabetes mellitus (pancreatic endocrine insufficiency), of whom 73\%
required insulin. CONCLUSIONS: AIP is a challenging disease for
diagnosis and treatment. Cortisone treatment is generally successful and
provides clinical remission in the large majority of patients
(\textgreater{}90\%). In the further course of the disease, a
considerable number of patients develop PEI and diabetes. Only
one-quarter of patients exhibit on imaging the characteristic
``sausage-like'' pancreas (diffuse enlargement), approximately
three-quarters had a focal mass that could be misdiagnosed as pancreatic
malignancy.

PubMed: \url{https://www.ncbi.nlm.nih.gov/pubmed/?term=30236651}

{}

{}

\begin{center}\rule{0.5\linewidth}{\linethickness}\end{center}

 - \textbf{A Hypervascular Pancreatic Tumor}

\emph{Gastroenterology 2018 Sep;():}

PubMed: \url{https://www.ncbi.nlm.nih.gov/pubmed/?term=30240665}

{}

{}

\begin{center}\rule{0.5\linewidth}{\linethickness}\end{center}

\begin{center}\rule{0.5\linewidth}{\linethickness}\end{center}

 - \textbf{Complex HuR function in pancreatic cancer cells}

\emph{Wiley interdisciplinary reviews. RNA 2018 05;9(3):e1469}

Pancreatic ductal adenocarcinoma (PDAC) is one of the most lethal
cancers with dismal patient outcomes. The underlying core genetic
drivers of disease have been identified in human tumor specimens and
described in genetically engineered mouse models. These genetic drivers
of PDAC include KRAS signaling, TP53 mutations, and genetic loss of the
SMAD4 tumor suppressor protein. Beyond the known mutational landscape of
PDAC genomes, alternative disrupted targets that extend beyond
conventional genetic mutations have been elusive and understudied in the
context of PDAC cell therapeutic resistance and survival. This last
point is important because PDAC tumors have a unique and complex tumor
microenvironment that includes hypoxic and nutrient-deprived niches that
could select for cell populations that garner therapeutic resistance,
explaining tumor heterogeneity in regards to response to different
therapies. We and others have embarked in a line of investigation
focused on the key molecular mechanism of posttranscriptional gene
regulation that is altered in PDAC cells and supports this pro-survival
phenotype intrinsic to PDAC cells. Specifically, the key regulator of
this mechanism is a RNA-binding protein, HuR (ELAVL1), first described
in cancer nearly two decades ago. Herein, we will provide a brief
overview of the work demonstrating the importance of this RNA-binding
protein in PDAC biology and then provide insight into ongoing work
developing therapeutic strategies aimed at targeting this molecule in
PDAC cells. This article is categorized under: RNA in Disease and
Development \textgreater{} RNA in Disease.

PubMed: \url{https://www.ncbi.nlm.nih.gov/pubmed/?term=29452455}

{}

{}

\begin{center}\rule{0.5\linewidth}{\linethickness}\end{center}

\begin{itemize}
\tightlist
\item
  \textbf{Norepinephrine enhances cell viability and invasion, and
  inhibits apoptosis of pancreatic cancer cells in a Notch‑1‑dependent
  manner}
\end{itemize}

\url{https://www.spandidos-publications.com/or/40/5/3015}

\begin{center}\rule{0.5\linewidth}{\linethickness}\end{center}

 - \textbf{Prognostic impact of postoperative complication after
pancreatoduodenectomy for pancreatic adenocarcinoma stratified by the
resectability status}

\emph{Journal of surgical oncology 2018 Jun;():}

PubMed: \url{https://www.ncbi.nlm.nih.gov/pubmed/?term=29878355}

BACKGROUND AND OBJECTIVES: The aim of this study was to investigate the
prognostic impact of postoperative complications after
pancreatoduodenectomy (PD) for pancreatic ductal adenocarcinoma (PDAC)
stratified by resectability status. METHODS: Medical records of 226
patients with pancreatic head carcinoma who underwent PD, including 115
with resectable (R) and 111 with borderline resectable/unresectable
(BR/UR) PDAC, were reviewed retrospectively. Major complications were
defined as grade III or IV based on the Clavien-Dindo classification
system. The prognostic impact of major complications on overall survival
(OS) was analyzed using univariate and multivariate analyses with
stratification by resectability status. RESULTS: A multivariate analysis
in the BR/UR group identified R1 resection (P = 0.03), T 3/4 stage
(P = 0.03), and incidence of major complications (P = 0.03) as
independent risk factors for poor survival, whereas major complications
did not affect survival in the R group. Initiation of adjuvant
gemcitabine plus S-1 chemotherapy occurred significantly less frequently
for patients with major complications than for those without major
complications in the BR/UR group (P = 0.02). CONCLUSION: A negative
prognostic impact of postoperative major complications after PD was
observed in patients with BR/UR PDAC, whereas the prognostic impact was
unclear in patients with R PDAC.

{}

{}

\begin{center}\rule{0.5\linewidth}{\linethickness}\end{center}

\begin{center}\rule{0.5\linewidth}{\linethickness}\end{center}

\begin{itemize}
\tightlist
\item
  \textbf{Surgical overtreatment of pancreatic intraductal papillary
  mucinous neoplasms: Do the 2017 International Consensus Guidelines
  improve clinical decision making?}
\end{itemize}

\url{https://www.surgjournal.com/article/S0039-6060(18)30459-8/fulltext}

\begin{center}\rule{0.5\linewidth}{\linethickness}\end{center}

\begin{itemize}
\tightlist
\item
  \textbf{Management of Intraductal Papillary Mucinous Neoplasms:
  Controversies in Guidelines and Future Perspectives}
\end{itemize}

\url{https://link.springer.com/article/10.1007/s11938-018-0190-2}

\begin{center}\rule{0.5\linewidth}{\linethickness}\end{center}

\begin{itemize}
\tightlist
\item
  \textbf{Clinical utility of the guidelines for intraductal papillary
  mucinous neoplasm: A case series from a medical center in central
  Taiwan}
\end{itemize}

\url{https://onlinelibrary.wiley.com/doi/abs/10.1002/aid2.13095}

\begin{center}\rule{0.5\linewidth}{\linethickness}\end{center}

\begin{center}\rule{0.5\linewidth}{\linethickness}\end{center}

 - \textbf{Is It Acute Pancreatitis or Recurrent Acute Pancreatitis
Leading to Chronic Pancreatitis that Increases Pancreatic Cancer Risk?}

\emph{Gastroenterology 2018 10;155(4):1279-1280}

PubMed: \url{https://www.ncbi.nlm.nih.gov/pubmed/?term=30236555}

{}

{}

\begin{center}\rule{0.5\linewidth}{\linethickness}\end{center}

 - \textbf{Stromal biology and therapy in pancreatic cancer: ready for
clinical translation?}

\emph{Gut 2018 Sep;():}

Pancreatic ductal adenocarcinoma (PDA) is notoriously aggressive and
hard to treat. The tumour microenvironment (TME) in PDA is highly
dynamic and has been found to promote tumour progression, metastasis
niche formation and therapeutic resistance. Intensive research of recent
years has revealed an incredible heterogeneity and complexity of the
different components of the TME, including cancer-associated
fibroblasts, immune cells, extracellular matrix components, tumour
vessels and nerves. It has been hypothesised that paracrine interactions
between neoplastic epithelial cells and TME compartments may result in
either tumour-promoting or tumour-restraining consequences. A better
preclinical understanding of such complex and dynamic network systems is
required to develop more powerful treatment strategies for patients.
Scientific activity and the number of compelling findings has virtually
exploded during recent years. Here, we provide an update of the most
recent findings in this area and discuss their translational and
clinical implications for basic scientists and clinicians alike.

PubMed: \url{https://www.ncbi.nlm.nih.gov/pubmed/?term=30177543}

{}

{}

\begin{center}\rule{0.5\linewidth}{\linethickness}\end{center}

 - \textbf{Overexpression of S100A4 protein may be associated with the
development and progression of pancreatic cancer}

\emph{Journal of cancer research and therapeutics 2018
;14(Supplement):S159-S166}

Aim: Accumulated evidence has suggested a relationship between S100A4
protein expression and the development and progression of pancreatic
cancer (PC) while its role in diagnosis and prognosis of PC still keeps
inconsistent. To obtain definitive associations between S100A4 and PC, a
meta-analysis was conducted. Materials and Methods: The PubMed and
Chinese National Knowledge Infrastructure databases were electronically
searched to identify studies reporting an association between S100A4
protein and PC. Statistical analyses were undergone with the utilization
of STATA version 12.0 software. Results: Nine clinical studies with a
total of 545 tumor samples were included in the meta-analysis. Results
revealed that increased S100A4 expression were associated with the
tumor-node-metastasis stages of PC (III-IV vs.~I-II: odds ratio {[}OR{]}
=5.50, 95\% confidence interval {[}95\% CI{]} =3.13-9.67, P \textless{}
0.001). Also, compared with 1-2 histologic grade of PC samples, S100A4
protein was expressed more frequently in samples with 3-4 histologic
grade (grades 1-2 vs.~grades 3-4: OR = 2.57, 95\% CI = 1.05-6.24, P =
0.038). Conclusion: This meta-analysis showed that overexpression of
S100A4 seems to be associated with tumor progression and poor prognosis
of PC patients.

PubMed: \url{https://www.ncbi.nlm.nih.gov/pubmed/?term=29578167}

{}

{}

\begin{center}\rule{0.5\linewidth}{\linethickness}\end{center}

\begin{itemize}
\tightlist
\item
  \textbf{Prognostic stratification of resected pancreatic ductal
  adenocarcinoma: past, present, and future}
\end{itemize}

\url{https://www.sciencedirect.com/science/article/pii/S1590865818308892}

\begin{center}\rule{0.5\linewidth}{\linethickness}\end{center}

\begin{itemize}
\tightlist
\item
  \textbf{Pancreatic cancer: French Clinical Practice Guidelines for
  diagnosis, treatment and follow-up (SNFGE, FFCD, GERCOR, UNICANCER,
  SFCD, SFED, SFRO, ACHBT, AFC)}
\end{itemize}

\url{https://www.sciencedirect.com/science/article/pii/S1590865818308880}

\begin{center}\rule{0.5\linewidth}{\linethickness}\end{center}

\begin{itemize}
\tightlist
\item
  \textbf{Serous Neoplasms of the Pancreas: A Comprehensive Review}
\end{itemize}

\url{http://www.archivesofpathology.org/doi/abs/10.5858/arpa.2017-0195-RS}

\begin{center}\rule{0.5\linewidth}{\linethickness}\end{center}

\begin{center}\rule{0.5\linewidth}{\linethickness}\end{center}

 - \textbf{Intraductal Papillary Mucinous Neoplasms of Minor Salivary
Glands With AKT1 p.Glu17Lys Mutation}

\emph{The American journal of surgical pathology 2018
Aug;42(8):1076-1082}

The spectrum of low-grade intraductal papillary proliferations of the
salivary glands is heterogenous, and reproducible morphologic diagnostic
criteria have not yet been established. Recognized types are sialadenoma
papilliferum, inverted ductal papilloma, and intraductal papilloma, but
some lesions have been possibly included in the morphologic spectrum of
cystadenoma or low-grade intraductal carcinomas. We herein present
detailed morphologic, immunophenotypic, and genotypic features of 3
minor salivary gland neoplasms affecting 2 men (aged 65 and 71 y) and 1
woman (aged 78 y). They ranged in size from 1 to 2.5 cm. All tumors
showed atypical papillary intraductal growth that presented either as
uninodular/unicystic lesions (intraductal papilloma-like; n=2) or as a
discontinuous growth along the ductal system in a manner similar to
pancreatic intraductal papillary mucinous neoplasm (n=1). Variable
cytologic and architectural atypia was observed, ranging from bland
intraductal papilloma-like features, to areas mimicking atypical ductal
hyperplasia and low-grade ductal carcinoma in situ of the breast.
Amplicon-based massive parallel sequencing revealed an identical AKT1
p.Glu17Lys mutation in all 3 cases, but absence of concurring mutations
in other genes of the RAS or PI3K pathway. This small series represents
the first genetic study on salivary intraductal papillary neoplasms. Our
cases showed significant variation in the degree of cytologic and
architectural atypia, which overlaps with intraductal papillomas at the
one end and with low-grade intraductal carcinoma at the other end of the
spectrum, suggesting a disease continuum. As the full biological and
morphologic characteristics of these ductal papillary lesions remain to
be defined, the noncommitted term ``intraductal papillary neoplasms''
might be more appropriate. Our novel genetic findings mirror similar
activating mutations of AKT1 and other PI3K pathway members in
intraductal papillary lesions of the breast and anogenital glands.

PubMed: \url{https://www.ncbi.nlm.nih.gov/pubmed/?term=29738361}

{}

{}

\begin{center}\rule{0.5\linewidth}{\linethickness}\end{center}

 - \textbf{Low-grade intraductal carcinoma of the salivary gland with
prominent oncocytic change: a newly described variant}

\emph{Histopathology 2018 Aug;73(2):314-320}

AIMS: Low-grade intraductal carcinoma (LG-IDC) is a clinically indolent
malignant tumour of the salivary glands. Because of its rarity, the
histological variants of LG-IDC have not been well characterised.
Herein, we describe five LG-IDC cases with prominent oncocytic change in
the major salivary glands. METHODS AND RESULTS: We examined five cases,
three males and two females (mean age~=~63~years), of LG-IDC with
oncocytic change. The sites affected by LG-IDC were the parotid and
submandibular glands. The lesions were macroscopically unilocular or
multilocular cysts with a solid tumour arising from the cyst wall.
Smaller tumour cell nests were also observed. As with classic LG-IDC,
the cyst wall was surrounded by myoepithelial cells with no invasive
component. The tumour cells had abundant oncocytic cytoplasm and
proliferated in a low-papillary, tubular or cribriform pattern.
Immunohistochemically, the tumour cells were diffusely positive for
pan-cytokeratin, S100, mammaglobin and antimitochondria antibody, and
were negative for androgen receptor and gross cystic disease fluid
protein-15. Unlike classic LG-IDC, some of these cases demonstrated
focal invagination of myoepithelial cells in the intraductal tumour.
CONCLUSION: Oncocytic LG-IDC should be recognised as a histologically
unique variant of LG-IDC. Awareness of this entity is important to avoid
erroneous diagnosis and inappropriate treatment for histological mimics.

PubMed: \url{https://www.ncbi.nlm.nih.gov/pubmed/?term=29574881}

{}

{}

\begin{center}\rule{0.5\linewidth}{\linethickness}\end{center}

 - \textbf{Solid pseudopapillary neoplasm (SPN) of the testis:
Comprehensive mutational analysis of 6 testicular and 8 pancreatic SPNs}

\emph{Annals of diagnostic pathology 2018 Aug;35():42-47}

BACKGROUND: Recently, we came with the theory of a possible relationship
between a group of testicular and pancreatic tumors. We used one case of
a pancreatic analogue solid pseudopapillary neoplasm of the testis
composed partially of areas reminiscent of solid pseudopapillary
neoplasm (SPN) of the pancreas and partially of structures identical to
primary signet ring stromal tumor of the testis (PSRSTT) as a connecting
link between these two entities. After demonstrating that PSRSTT and
pancreatic analogue SPN of the testis share the same immunoprofile and
genetic features characteristic for pancreatic SPN, we came to the
conclusion that pancreatic analogue SPN of the testis and PSRSTT
represent a morphological spectrum of a single entity and that both are
related to the pancreatic SPN. DESIGN: The aim of this study is to
present a series of 6 cases of testicular tumors, which lacked the
signet ring cell component and were thus morphologically very similar to
the SPN of the pancreas. The goal of this study is to compare the
genetic background of these testicular tumors that are obviously related
to the PSRSTT/pancreatic analogue SPN of the testis with the series of 8
pancreatic SPN. RESULTS: The mutational analysis revealed an oncogenic
somatic mutation in the exon 3 of the CTNNB1 (β-catenin) gene in all
analyzable (5/6) testicular and all pancreatic (8/8) tumors. The
immunoprofile (positivity with β-catenin, CD10, vimentin, NSE, CD56, and
negativity with inhibin, calretinin, chromogranin) was identical in all
testicular and pancreatic tumors. CONCLUSION: This study expanded the
morphological spectrum of the PSRSTT/pancreatic analogue SPN of the
testis by adding 6 cases without the signet ring cell component.
Considering the obvious analogy of PSRSTT/pancreatic analogue SPN of the
testis/SPN of the testis and their relationship to the pancreatic SPN we
propose the collective term ``solid pseudopapillary neoplasm of the
testis'' for these tumors. The mutational profile of the SPN of the
testis and pancreas was the same in both groups of tumors which we
consider as a final proof that SPN of the testis is identical to the SPN
of the pancreas.

PubMed: \url{https://www.ncbi.nlm.nih.gov/pubmed/?term=29705715}

{}

{}

\begin{center}\rule{0.5\linewidth}{\linethickness}\end{center}

 - \textbf{Pseudo-``solid pseudopapillary neoplasms'' of the testis: in
reality Sertoli cell tumors}

\emph{Human pathology 2018 Aug;():}

PubMed: \url{https://www.ncbi.nlm.nih.gov/pubmed/?term=30110596}

{}

{}

\begin{center}\rule{0.5\linewidth}{\linethickness}\end{center}

 - \textbf{Pseudo-``solid pseudopapillary neoplasms'' of the testis: in
reality Sertoli cell tumors-reply}

\emph{Human pathology 2018 Aug;():}

PubMed: \url{https://www.ncbi.nlm.nih.gov/pubmed/?term=30120970}

{}

{}

\begin{center}\rule{0.5\linewidth}{\linethickness}\end{center}

 - \textbf{Systems Oncology: Bridging Pancreatic and Castrate Resistant
Prostate Cancer}

\emph{Pathology oncology research : POR 2018 Sep;():}

Large investments by pharmaceutical companies in the development of new
antineoplastic drugs have not been resulting in adequate advances of new
therapies. Despite the introduction of new methods, technologies,
translational medicine and bioinformatics, the usage of collected
knowledge is unsatisfactory. In this paper, using examples of pancreatic
ductal adenocarcinoma (PaC) and castrate-resistant prostate cancer
(CRPC), we proposed a concept showing that, in order to improve
applicability of current knowledge in oncology, the re-clustering of
clinical and scientific data is crucial. Such an approach, based on
systems oncology, would include bridging of data on biomarkers and
pathways between different cancer types. Proposed concept would
introduce a new matrix, which enables combining of already approved
therapies between cancer types. Paper provides a (a) detailed analysis
of similarities in mechanisms of etiology and progression between PaC
and CRPC, (b) diabetes as common hallmark of both cancer types and (c)
knowledge gaps and directions of future investigations. Proposed
horizontal and vertical matrix in cancer profiling has potency to
improve current antineoplastic therapy efficacy. Systems biology map
using Systems Biology Graphical Notation Language is used for
summarizing complex interactions and similarities of mechanisms in
biology of PaC and CRPC.

PubMed: \url{https://www.ncbi.nlm.nih.gov/pubmed/?term=30220022}

{}

{}

\begin{center}\rule{0.5\linewidth}{\linethickness}\end{center}

 - \textbf{Primordial germ cells as a potential shared cell of origin
for mucinous cystic neoplasms of the pancreas and mucinous ovarian
tumors}

\emph{The Journal of pathology 2018 Sep;():}

Mucinous ovarian tumors (MOTs) morphologically and epidemiologically
resemble mucinous cystic neoplasms (MCNs) of the pancreas, sharing a
similar stroma and both occurring disproportionately among young
females. Additionally, MOTs and MCNs share similar clinical
characteristics and immunohistochemical phenotypes. Exome sequencing has
revealed frequent recurrent mutations in KRAS and RNF43 in both MOTs and
MCNs. The cell of origin for these tumors remains unclear, but MOTs
sometimes arise in the context of mature cystic teratomas and other
primordial germ cell (PGC) tumors. We undertook the present study to
investigate whether non-teratoma-associated MOTs and MCNs share a common
cell of origin. Comparisons of the gene expression profiles of MOTs
{[}including both the mucinous borderline ovarian tumors (MBOTs) and
invasive mucinous ovarian carcinomas (MOCs){]}, high-grade serous
ovarian carcinomas, ovarian surface epithelium, Fallopian tube
epithelium, normal pancreatic tissue, pancreatic duct adenocarcinomas,
MCNs, and single-cell RNA-sequencing of PGCs revealed that both MOTs and
MCNs are more closely related to PGCs than to either eutopic epithelial
tumors or normal epithelia. We hypothesize that MCNs may arise from PGCs
that stopped in the dorsal pancreas during their descent to the gonads
during early human embryogenesis, while MOTs arise from PGCs in the
ovary. Together, these data suggest a common pathway for the development
of MCNs and MOTs, and suggest that these tumors may be more properly
classified as germ cell tumor variants. Copyright © 2018 Pathological
Society of Great Britain and Ireland. Published by John Wiley \& Sons,
Ltd.

PubMed: \url{https://www.ncbi.nlm.nih.gov/pubmed/?term=30229909}

{}

{}

\begin{center}\rule{0.5\linewidth}{\linethickness}\end{center}

\begin{itemize}
\tightlist
\item
  \textbf{Primordial germ cells as a potential shared cell of origin for
  mucinous cystic neoplasms of the pancreas and mucinous ovarian tumors}
\end{itemize}

\url{https://onlinelibrary.wiley.com/doi/abs/10.1002/path.5161}

\begin{center}\rule{0.5\linewidth}{\linethickness}\end{center}

\hypertarget{bile-ducts-3}{%
\subsubsection{Bile Ducts}\label{bile-ducts-3}}

\begin{center}\rule{0.5\linewidth}{\linethickness}\end{center}

 - \textbf{Biliary Tract Cancer: State of the Art and potential role of
DNA Damage Repair}

\emph{Cancer treatment reviews 2018 Sep;70():168-177}

Biliary tract cancers (BTCs), including cholangiocarcinoma, gallbladder
cancer and ampullary cancers, are poor-prognosis malignancies. Most
patients are diagnosed with advanced disease, when treatment is limited
to palliative chemotherapy. First line chemotherapy is usually
administered in the form of cisplatin and gemcitabine. Benefit from
second line chemotherapy is still to be confirmed. Even though new
systemic treatment targets have been recognised, especially in patients
with intrahepatic cholangiocarcinoma (e.g.~IDH and FGFR), there is an
urgent need for novel treatment strategies. Genomic profiling of BTC is
progressively becoming a reality which allows a better understanding of
their biology and potential new targets. This review provides an insight
into DNA Damage Repair (DDR) mechanisms, prevalence of DDR-deficient
tumours in BTC, and the potential role of DDR in cancer development.
Some form of DDR deficiency is expected to be present in around 25\% of
patients with BTC, and this knowledge could be exploited to potentially
increase response to currently-available treatment strategies
(chemotherapy, radiotherapy or immunotherapy). For patients with
DDR-proficient tumours, drug inhibition of DDR could be instituted.

PubMed: \url{https://www.ncbi.nlm.nih.gov/pubmed/?term=30218788}

{}

{}

\begin{center}\rule{0.5\linewidth}{\linethickness}\end{center}

 - \textbf{Combined CDK4/6 and Pan-mTOR Inhibition Is Synergistic
Against Intrahepatic Cholangiocarcinoma}

\emph{Clinical cancer research : an official journal of the American
Association for Cancer Research 2018 Jul;():}

Purpose: Intrahepatic cholangiocarcinoma (ICC) is an aggressive cancer
type, lacking effective therapies and associated with a dismal
prognosis. Palbociclib is a selective CDK4/6 inhibitor, which has been
shown to suppress cell proliferation in many experimental cancer models.
Recently, we demonstrated that pan-mTOR inhibitors, such as MLN0128,
effectively induce apoptosis, although have limited efficacy in
restraining proliferation of ICC cells. Here, we tested the hypothesis
that palbociclib, due to its antproliferative properties in many cancer
types, might synergize with MLN0128 to impair ICC growth.Experimental
Design: Human ICC cell lines and the AKT/YapS127A ICC mouse model were
used to test the therapeutic efficacy of palbociclib and MLN0128, either
alone or in combination.Results: Administration of palbociclib
suppressed in vitro ICC cell growth by inhibiting cell-cycle
progression. Concomitant administration of palbociclib and MLN0128 led
to a pronounced, synergistic growth constraint of ICC cell lines.
Furthermore, while treatment with palbociclib or MLN0128 alone resulted
in tumor growth reduction in AKT/YapS127A mice, a remarkable tumor
regression was achieved when the two drugs were administered
simultaneously. Mechanistically, palbociclib was found to potentiate
MLN0128 mTOR inhibition activity, whereas MLN0128 prevented the
upregulation of cyclin D1 induced by palbociclib treatment.Conclusions:
Our study indicates the synergistic activity of palbociclib and MLN0128
in inhibiting ICC cell proliferation. Thus, combination of CDK4/6 and
mTOR inhibitors might represent a novel, promising, and effective
therapeutic approach against human ICC. Clin Cancer Res; 1-11. ©2018
AACR.

PubMed: \url{https://www.ncbi.nlm.nih.gov/pubmed/?term=30084835}

{}

{}

\begin{center}\rule{0.5\linewidth}{\linethickness}\end{center}

 - \textbf{Preoperative CEA levels are supplementary to CA19-9 levels in
predicting prognosis in patients with resectable intrahepatic
cholangiocarcinoma.}

\url{https://www.researchgate.net/publication/326829300_Preoperative_CEA_levels_are_supplementary_to_CA19-9_levels_in_predicting_prognosis_in_patients_with_resectable_intrahepatic_cholangiocarcinoma}

\begin{center}\rule{0.5\linewidth}{\linethickness}\end{center}

 - \textbf{Improved Survival in Surgically Resected Distal
Cholangiocarcinoma Treated with Adjuvant Therapy: a Propensity Score
Matched Analysis} \emph{Journal of gastrointestinal surgery : official
journal of the Society for Surgery of the Alimentary Tract 2018 Jul;():}

BACKGROUND: Data on the efficacy of adjuvant therapy (AT) in distal
cholangiocarcinoma (dCCA) is limited. This study aimed to determine the
role of AT in resected dCCA and identify subgroups that benefit from AT.
METHODS: We conducted a retrospective review of surgically resected dCCA
in the NCDB from 2004 to 2013. Patients who received AT or observation
(OB) were matched by propensity score. Log-rank test was used to compare
OS. RESULTS: Of 1782 patients with resected dCCA, 840 (47\%) were in the
OB group and 942 (53\%) in the AT group. AT was younger (64.0
vs.~68.7~years, p \textless{} 0.001), had less comorbidities (Charlson
Deyo score 0) (74.6 vs.~68.0\%, p \textless{} 0.001), and more likely to
have private insurance (p \textless{} 0.001). AT was more likely to
present with T3/T4 stage (72 vs.~57\%, p \textless{} 0.001), N1/N2
disease (58 vs.~37\%, p \textless{} 0.001), and positive surgical
margins (26 vs.~16\%, p \textless{} 0.001). After 1:1 propensity score
matching, 500 OB and 500 AT patients were compared. AT was associated
with better OS (HR 0.79; 95\% CI 0.67-0.93). Median OS was 31 and
25~months for the AT and OB (p = 0.006). The 1-, 3-, and 5-year survival
rates were 87, 46, and 31\% for AT; 79, 39, and 24\% for OB. Subgroup
analysis revealed an associated survival advantage for AT in T3/T4
tumors (HR = 0.72; 95\% CI 0.59-0.89), node positive disease (HR 0.70;
95\% CI 0.56-0.87), and positive margins (HR 0.58; 95\% CI 0.42-0.81).
CONCLUSION: AT is associated with improved OS in resected dCCA,
especially in T3/T4 tumors, node positive disease, and positive margins.

PubMed: \url{https://www.ncbi.nlm.nih.gov/pubmed/?term=30030718}

{}

{}

\begin{center}\rule{0.5\linewidth}{\linethickness}\end{center}

 - \textbf{Prognosis and Adherence with the National Comprehensive
Cancer Network Guidelines of Patients with Biliary Tract Cancers: an
Analysis of the National Cancer Database}

\emph{Journal of gastrointestinal surgery : official journal of the
Society for Surgery of the Alimentary Tract 2018 Aug;():}

BACKGROUND: The National Comprehensive Cancer Network (NCCN) guidelines
recommend chemotherapy for patients with inoperable biliary tract
cancers (BTC), as well as patients following resection of BTC with lymph
node metastasis (N1)/positive margins (R1). We sought to define overall
adherence, as well as long-term outcomes, with the NCCN guidelines for
BTC using the National Cancer Database (NCDB). METHODS: A total of
176,536 patients diagnosed with BTC at a hospital participating in the
NCDB between 2004 and 2015 were identified. RESULTS: Among all patients,
63\% of patients received medical therapy (chemotherapy or best
supportive care), 11\% underwent surgical palliation, and 26\% underwent
curative-intent surgery. According to the NCCN guidelines, 86\%
(n = 152,245) of patients were eligible for chemotherapy, yet, only
42.2\% (n = 64,615) received chemotherapy. Factors associated with a
lower adherence with NCCN guidelines included patient age
(\textgreater{} 65~years: OR = 1.02), ethnicity (Black: OR = 1.14,
Hispanic: OR = 1.21, Asian: OR = 1.24), and insurance status
(non-private: OR = 1.45, all p \textless{} 0.001). A smaller subset of
patients was either recommended chemotherapy but refused (n = 9269,
10.6\%) or had medical factors that contraindicated chemotherapy
(n = 8275, 9.4\%). On multivariable analysis, adjusting for clinical and
tumor-specific factors, adherence with NCCN guidelines was associated
with a survival benefit for patients receiving medical therapies
(HR = 0.74) or undergoing curative-intent surgery (HR = 0.73, both
p \textless{} 0.001). CONCLUSION: Less than half of patients with BTC
received systemic chemotherapy in adherence with NCCN guidelines. While
a subset of patients had contraindications or refused chemotherapy,
other factors such as insurance status and ethnicity were associated
with adherence. Adherence with chemotherapy guidelines may influence
long-term outcomes.

PubMed: \url{https://www.ncbi.nlm.nih.gov/pubmed/?term=30112703}

{}

{}

\begin{center}\rule{0.5\linewidth}{\linethickness}\end{center}

 - \textbf{Intrahepatic cholangiocarcinoma: the AJCC/UICC 8th edition
updates}

\emph{Chinese clinical oncology 2018 Jul;():}

Intrahepatic cholangiocarcinoma accounts for 5\% to 30\% of all primary
liver cancers, and its incidence has increased in the last 3 decades.
Surgical resection remains the only potentially curative treatment but
is associated with high tumor recurrence rates. The 7th edition of the
American Joint Committee on Cancer (AJCC) Staging Manual introduced a
new staging system for intrahepatic cholangiocarcinoma, which was
previously staged the same as hepatocellular carcinoma. The recently
published 8th edition has subdivided the T1 category to T1a and T1b
based on a size cutoff of 5 cm, removed periductal invasion from the T4
category, and downstaged T4 tumors and regional lymph node metastasis
from stage IV to IIIB. Continued international efforts to accurately
stratify prognosis are important to counsel patients and guide treatment
decisions.

PubMed: \url{https://www.ncbi.nlm.nih.gov/pubmed/?term=30180751}

{}

{}

\begin{center}\rule{0.5\linewidth}{\linethickness}\end{center}

\begin{itemize}
\tightlist
\item
  \textbf{Novel targeted treatment options for advanced
  cholangiocarcinoma}
\end{itemize}

\url{https://www.tandfonline.com/doi/abs/10.1080/13543784.2018.1512581}

\begin{center}\rule{0.5\linewidth}{\linethickness}\end{center}

\begin{itemize}
\tightlist
\item
  \textbf{Biliary Tract Cancer: State of the Art and potential role of
  DNA Damage Repair}
\end{itemize}

\url{https://www.cancertreatmentreviews.com/article/S0305-7372(18)30156-7/fulltext}

\begin{center}\rule{0.5\linewidth}{\linethickness}\end{center}

\hypertarget{gallbladder-1}{%
\subsubsection{Gallbladder}\label{gallbladder-1}}

\begin{center}\rule{0.5\linewidth}{\linethickness}\end{center}

\begin{itemize}
\tightlist
\item
  \textbf{Incidental Gallbladder Cancer: How Residual Disease Affects
  Outcome in Two Referral HPB Centers from South America}
\end{itemize}

\url{https://link.springer.com/article/10.1007/s00268-018-4762-z}

\begin{center}\rule{0.5\linewidth}{\linethickness}\end{center}

\begin{itemize}
\tightlist
\item
  \textbf{The association between preoperative serum interleukin‐6
  levels and postoperative prognosis in patients with T2 gallbladder
  cancer}
\end{itemize}

\url{https://onlinelibrary.wiley.com/doi/10.1002/jso.25085}

\begin{center}\rule{0.5\linewidth}{\linethickness}\end{center}

 - \textbf{Gallbladder adenocarcinoma diagnosed from cutaneous
metastases occurring along the tract of a ventriculoperitoneal shunt}

\emph{Journal of cutaneous pathology 2018 Nov;45(11):870-873}

PubMed: \url{https://www.ncbi.nlm.nih.gov/pubmed/?term=30054926}

{}

{}

\begin{center}\rule{0.5\linewidth}{\linethickness}\end{center}

 - \textbf{Incidental gallbladder cancer at cholecystectomy}

\emph{Annali italiani di chirurgia 2017 ;6():399-402}

BACKGROUND: Gallbladder tumours rank fifth in the world among
gastrointestinal system tumours. Coincidental gallbladder tumours are
diagnosed during cholecystectomies, or by examining the cholecystectomy
material. AIMS: In this study, we aimed to evaluate the incidence of
gallbladder cancer among patients undergoing cholecystectomies due to
gallbladder disease. STUDY DESIGN: Retrospective study METHODS: The
files of 341 patients who had undergone routine cholecystectomy
operations between January 2013 and March 2016 were reviewed, and their
pathology results were recorded. Those patients with gallbladder
carcinomas were evaluated in terms of age, gender, preoperative
findings, existing symptoms, radiological findings, surgical findings
and follow-up. The cancer invasion depth was classified according to the
American Joint Commission on Cancer (AJCC) atlas, and this study was
approved by the ethical committee of our university. RESULTS: Among the
341 patients who participated in this study, 253 (74.41\%) were female,
88 (25.80\%) were male, and their average age was 49.61 years old
(17-86). Seven of the patients (2.05\%) had gallbladder tumours; six of
which were female, one was male and their average age was 67.71 years
old (62-76). One tumour was diagnosed as a frozen specimen during the
operation, while the others were diagnosed during the postoperation
phase. Three of the patients had T1b and four had T2 tumours.
CONCLUSION: Gallbladder tumours detected incidentally could extend
survival rates with proper surgical intervention and chemotherapy. The
possibility of a tumour should not be dismissed in those patients with
advanced age, females or patients with gallbladder stones. Frozen
specimens should be created during a cholecystectomy, and if there is
any doubt about the diagnosis, a postoperative histopathological
examination of the gallbladder should be conducted. KEY WORDS:
Cholecystectomy, Gall bladder stone, Incidental gallbladder carcinoma.

PubMed: \url{https://www.ncbi.nlm.nih.gov/pubmed/?term=29197189}

{}

{}

\begin{center}\rule{0.5\linewidth}{\linethickness}\end{center}

\begin{itemize}
\tightlist
\item
  \textbf{Gallbladder Cancer in Eastern Province of Saudi Arabia: A
  Retrospective Cohort Study}
\end{itemize}

\url{https://www.sciencedirect.com/science/article/pii/S2049080118301845}

\begin{center}\rule{0.5\linewidth}{\linethickness}\end{center}

\hypertarget{ampulla-of-vater-1}{%
\subsubsection{Ampulla of Vater}\label{ampulla-of-vater-1}}

\begin{center}\rule{0.5\linewidth}{\linethickness}\end{center}

 - \textbf{Prognostic factors of non-ampullary duodenal adenocarcinoma}

\emph{Japanese journal of clinical oncology 2018 Aug;48(8):743-747}

Background: Non-ampullary duodenal adenocarcinoma, excluding carcinoma
in the ampulla of Vater, is a rare disease. Although several prognostic
factors have been reported, they remain controversial due to the rarity
of non-ampullary duodenal adenocarcinoma. The aims of this study were to
investigate prognostic factors in patients with non-ampullary duodenal
adenocarcinoma and to assess chemotherapy in patients with recurrence.
Patients and methods: Records of 25 patients who underwent surgical
treatment for non-ampullary duodenal adenocarcinoma from 2004 to 2016
were retrospectively reviewed. The relationship between the
clinicopathological factors and outcomes was investigated. Results:
Serum level of CA19-9, gross appearance, tumor size, tumor invasion,
lymph node metastases, TNM stage and lymphatic and vascular invasion
were significant risk factors of recurrence. Patients with recurrence
who received chemotherapy according to regimens used to treat colorectal
cancer had a better prognosis than those without chemotherapy (P =
0.016). Conclusion: Advanced non-ampullary duodenal adenocarcinoma has a
poor prognosis, but chemotherapy possibly improves the prognosis in the
patients with recurrent non-ampullary duodenal adenocarcinoma.

PubMed: \url{https://www.ncbi.nlm.nih.gov/pubmed/?term=29931295}

{}

{}

\begin{center}\rule{0.5\linewidth}{\linethickness}\end{center}

 - \textbf{Using an endoscopic distal cap to collect pancreatic fluid
from~the ampulla (with video)}

\emph{Gastrointestinal endoscopy 2017 Dec;86(6):1152-1156.e2}

BACKGROUND AND AIMS: Duodenal collections of pancreatic fluid can be
used as a source of mutations and other markers of pancreatic ductal
neoplasia, but admixing pancreatic juice with duodenal contents lowers
the concentrations of mutations. Collecting pancreatic fluid directly
from the ampulla could yield a purer sample of pancreatic fluid.
METHODS: We used an endoscopic distal cap attachment to ``cap'' the
ampulla and collect secretin-stimulated pancreatic fluid samples for 5
minutes from 81 patients undergoing pancreatic evaluation as part of the
Cancer of the Pancreas Screening studies. We compared mutation
concentrations (K-ras and GNAS) measured by droplet-digital PCR (ddPCR)
in ``cap-collected juice'' samples to those found in juice samples
obtained from 77 patients collected by aspiration from the duodenal
lumen without capping the ampulla. RESULTS: Among all subjects, mutation
concentrations were higher in pancreatic juice samples collected using
the endoscopic cap method (median, .028\%; IQR, 0-.077) compared with
the noncap-collected (median, .019\%; IQR, 0-.044; P~= .055). Among
pancreatic juice samples with detectable mutations, mutation
concentrations were higher in the cap-collected juice samples than in
those collected without the cap (.055\%; IQR, .026-.092 vs .032\%; IQR,
.020-.066; P~= .031). CONCLUSIONS: Collecting pancreatic juice directly
from the ampulla using an endoscopic distal cap yields higher
concentrations of pancreatic fluid mutations.

PubMed: \url{https://www.ncbi.nlm.nih.gov/pubmed/?term=28259593}

{}

{}

\begin{center}\rule{0.5\linewidth}{\linethickness}\end{center}

 - \textbf{Outcomes and Treatment Options for Duodenal Adenocarcinoma: A
Systematic Review and Meta-Analysis}

\emph{Annals of surgical oncology 2018 Sep;25(9):2681-2692}

BACKGROUND: Duodenal adenocarcinoma (DA) is a rare tumor for which
survival data per treatment modality and disease stage are unclear. This
systematic review and meta-analysis aims to summarize the current
literature on patient outcome after surgical, (neo)adjuvant, and
palliative treatment in patients with DA. METHODS: A systematic search
was performed according to the preferred reporting items for systematic
reviews and meta-analyses guidelines, to 25 April 2017. Primary outcome
was overall survival (OS), specified for treatment strategy or disease
stage. Random-effects models were used for the calculation of pooled
odds ratios per treatment modality. Included papers were also screened
for prognostic factors. RESULTS: A total of 26 observational studies,
comprising 6438 patients with DA, were included. Of these, resection
with curative intent was performed in 71\% (range 53-100\%) of patients,
and 29\% received palliative treatment (range 0-61\%). The pooled 5-year
OS rate was 46\% after curative resection, compared with 1\% in
palliative-treated patients (OR 0.04, 95\% confidence interval {[}CI{]}
0.02-0.09, p \textless{} 0.0001). Both segmental resection and
pancreaticoduodenectomy allowed adequate assessment of lymph node
involvement and resulted in similar OS. Lymph node involvement
correlated with worse OS (pooled 5-year survival rate 21\% for nodal
metastases vs.~65\% for node-negative disease; OR 0.17, 95\% CI
0.11-0.27, p \textless{} 0.0001). In the current literature, no survival
benefit for adjuvant therapy after curative resection was found.
CONCLUSION: Resection with curative intent, either
pancreaticoduodenectomy or segmental resection, and lack of nodal
metastases, favors survival for DA. Further studies exploring
multimodality (neo)adjuvant therapy are warranted to investigate their
benefit.

PubMed: \url{https://www.ncbi.nlm.nih.gov/pubmed/?term=29946997}

{}

{}

\begin{center}\rule{0.5\linewidth}{\linethickness}\end{center}

 - \textbf{Mixed mucinous adenocarcinoma and somatostatinoma of the
ampulla of Vater associated with neurofibromatosis type 1}

\emph{Pathology 2017 Aug;49(5):553-555}

PubMed: \url{https://www.ncbi.nlm.nih.gov/pubmed/?term=28693748}

{}

{}

\begin{center}\rule{0.5\linewidth}{\linethickness}\end{center}

 - \textbf{Clinicopathological Stratification and Long-term Follow-up of
Patients with Periampullary Carcinomas}

\emph{Anticancer research 2018 Sep;38(9):5379-5386}

BACKGROUND: Periampullary carcinomas generally confer a poor outcome.
Choosing the most effective treatment regimen for each sub-entity proves
challenging and is usually based on experience from pancreatic
adenocarcinoma (PDAC). PATIENTS AND METHODS: The long-term follow-up is
presented of 472 patients with periampullary tumors {[}PDAC, distal
cholangiocarcinoma (dCC) and ampullary carcinomas (AC){]} who underwent
radical resection considering clinical characteristics, paraclinical
findings and histopathological features in order to define factors of
prognostic relevance. RESULTS: Patients with PDACs presented with larger
tumor sizes, more frequent R1 resection, higher rate of nodal and
perineural invasion, higher tumor stage according to the classification
of tumors of the Union Internationale contre le Cancer when compared to
those with dCCs and ACs. In a multivariate analysis, age
\textgreater{}65 years, postoperative complications and higher grading
of the tumor proved to be independent prognostic factors for survival.
CONCLUSION: Patients suffering from PDAC have the worst prognosis and
greatest benefit from radical resection of all patients with
periampullary tumors. More detailed studies are warranted to better
distinguish between the different entities.

PubMed: \url{https://www.ncbi.nlm.nih.gov/pubmed/?term=30194192}

{}

{}

\begin{center}\rule{0.5\linewidth}{\linethickness}\end{center}

 - \textbf{Pancreaticoduodenectomy for periampullary cancer: does the
tumour entity influence perioperative morbidity and long-term outcome?}

\emph{Acta chirurgica Belgica 2018 Sep;():1-7}

BACKGROUND: Malignant tumours of the periampullary region include ductal
adenocarcinoma of the pancreas (Pan-Ca), distal bile duct cancer (DBDC)
and adenocarcinoma of the ampulla (Amp-Ca). The present retrospective
clinical study was designed to evaluate the influence of tumour entity
on postoperative complications and identify risk factors predicting
survival and morbidity. METHODS: We retrospectively analysed data from
all patients who underwent pancreatic resection for periampullary cancer
with curative intent (R0 or R1). Demographic data, risk factors,
perioperative complications and survival rates for the different
subtypes were assessed. RESULTS: A total of 225 patients with
periampullary cancer were identified: 124 (55.1\%) had Pan-Ca, 55
(24.4\%) had DBDC and 46 had (20.4\%) Amp-Ca. Sixty-nine patients
(30.7\%) had major complications (grade IIIb-V). Patients with DBDC had
significantly more grade C pancreatic fistulas. Univariate analysis
revealed male gender, BMI \textgreater{}30, R1-status, and low-grade
tumour differentiation as risk factors for major complications. Overall
in-hospital-mortality was 6.7\%. CONCLUSIONS: Further research will be
needed to implement more individualized therapy.

PubMed: \url{https://www.ncbi.nlm.nih.gov/pubmed/?term=30203717}

{}

{}

\begin{center}\rule{0.5\linewidth}{\linethickness}\end{center}

\begin{itemize}
\tightlist
\item
  \textbf{Periampullary neuroendocrine tumor with large intracellular
  mucin globules, an amphicrine tumor?}
\end{itemize}

\url{https://www.sciencedirect.com/science/article/pii/S2214330018300063}

\begin{center}\rule{0.5\linewidth}{\linethickness}\end{center}

\begin{itemize}
\tightlist
\item
  \textbf{Ampullary micropapillary adenocarcinoma widely metastatic to
  the lymph nodes: A case report}
\end{itemize}

\url{https://www.sciencedirect.com/science/article/pii/S221433001730158X}

\begin{center}\rule{0.5\linewidth}{\linethickness}\end{center}

 - \textbf{Cytological features of mixed adenoneuroendocrine carcinoma
of the ampulla of Vater: A case report with immunocytochemical analyses}

\emph{Diagnostic cytopathology 2018 Jun;46(6):540-546}

Mixed adenoneuroendocrine carcinoma (MANEC) is defined as a tumor that
has morphologically recognizable both adenocarcinoma and neuroendocrine
carcinoma components comprising at least 30\% of either components.
MANEC occurring in the ampulla of Vater is extremely rare, and only 16
cases have been reported in the English language literature. In the
present report, we describe the first case of MANEC of the ampulla of
Vater with immunocytochemical analyses. An 82-year-old Japanese male was
incidentally found to have a tumorous lesion in the ampulla of Vater.
Endoscopic ultrasound-fine needle aspiration (EUS-FNA) of the tumor was
performed. The Papanicolaou smear demonstrated the presence of different
three components. The most dominant component was cohesive clusters of
small round cells with round to oval nuclei with powdery chromatin and
scant cytoplasm, which corresponded to small cell carcinoma. The second
component was an adenocarcinoma, which was composed of irregularly
overlapping clusters of tall columnar cells with large round to oval
nuclei containing conspicuous nucleoli. The third component was an
adenoma, which was comprised of flat cohesive clusters of columnar cells
without atypia. Immunocytochemical analyses demonstrated that
synaptophysin was expressed in the small round cells, and cdx-2 was
expressed in all three components. Accordingly, a cytodiagnosis of MANEC
with adenoma component was made. Preoperative diagnosis of ampullary
MANEC is difficult. However, this report clearly demonstrates three
different components in the EUS-FNA cytological specimen. Therefore, we
suggest that cytological examination is a useful method for diagnosis of
MANEC of the ampulla of Vater.

PubMed: \url{https://www.ncbi.nlm.nih.gov/pubmed/?term=29341470}

{}

{}

\begin{center}\rule{0.5\linewidth}{\linethickness}\end{center}

 - \textbf{Obstructive jaundice caused by myeloid sarcoma in duodenal
ampulla}

\emph{Digestive and liver disease : official journal of the Italian
Society of Gastroenterology and the Italian Association for the Study of
the Liver 2018 Sep;():}

PubMed: \url{https://www.ncbi.nlm.nih.gov/pubmed/?term=30253977}

{}

{}

\begin{center}\rule{0.5\linewidth}{\linethickness}\end{center}

 - \textbf{`Bleeding Dilemma': The Story of a Periampullary Mass}

\emph{Cureus 2018 Jul;10(7):e3035}

Periampullary malignancies arise in the vicinity of the ampulla of
Vater, a common passage for biliary and pancreatic secretions.
Determining the anatomical origin of these tumors represents a
diagnostic challenge. This is especially true for large tumors due to
the transitional nature of this region, proximity to different
structures, anatomical variations, and overlapping features among
constituting structures. This determination has significant prognostic
and therapeutic implications. Among them, primary ampullary
adenocarcinoma is a rare malignancy that has the best overall prognosis
with high rates of potentially curative resection and possible survival
even in advanced disease. Due to its rarity, it is also a vague
territory with no definitive guidelines regarding management and
surveillance currently available. Acute gastrointestinal hemorrhage is a
rare presentation of ampullary carcinoma that occurs secondary to tumor
ulceration. We report an elderly male with a previously known large,
initially asymptomatic periampullary mass who came for evaluation of
melena and was noted to be hypotensive secondary to acute blood loss
from the large tumor, later determined to be adenocarcinoma of the
ampulla of Vater.

PubMed: \url{https://www.ncbi.nlm.nih.gov/pubmed/?term=30254824}

{}

{}

\begin{center}\rule{0.5\linewidth}{\linethickness}\end{center}

\hypertarget{neuroendocrine-3}{%
\subsubsection{Neuroendocrine}\label{neuroendocrine-3}}

\begin{center}\rule{0.5\linewidth}{\linethickness}\end{center}

\begin{itemize}
\tightlist
\item
  \textbf{Periampullary neuroendocrine tumor with large intracellular
  mucin globules, an amphicrine tumor?}
\end{itemize}

\url{https://www.sciencedirect.com/science/article/pii/S2214330018300063\#}

\begin{center}\rule{0.5\linewidth}{\linethickness}\end{center}

\begin{itemize}
\tightlist
\item
  \textbf{The Evolving Treatment Algorithm for Advanced Neuroendocrine
  Neoplasms: Diversity and Commonalities Across Tumor Types}
\end{itemize}

\url{http://theoncologist.alphamedpress.org/content/early/2018/08/13/theoncologist.2018-0187.abstract}

\begin{center}\rule{0.5\linewidth}{\linethickness}\end{center}

 - \textbf{Pancreatic islets communicate with lymphoid tissues via
exocytosis of insulin peptides}

\emph{Nature 2018 Aug;560(7716):107-111}

Tissue-specific autoimmunity occurs when selected antigens presented by
susceptible alleles of the major histocompatibility complex are
recognized by T cells. However, the reason why certain specific
self-antigens dominate the response and are indispensable for triggering
autoreactivity is unclear. Spontaneous presentation of insulin is
essential for initiating autoimmune type 1 diabetes in non-obese
diabetic mice1,2. A major set of pathogenic CD4 T cells specifically
recognizes the 12-20 segment of the insulin B-chain (B:12-20), an
epitope that is generated from direct presentation of insulin peptides
by antigen-presenting cells3,4. These T cells do not respond to
antigen-presenting cells that have taken up insulin that, after
processing, leads to presentation of a different segment representing a
one-residue shift, B:13-214. CD4 T cells that recognize B:12-20 escape
negative selection in the thymus and cause diabetes, whereas those that
recognize B:13-21 have only a minor role in autoimmunity3-5. Although
presentation of B:12-20 is evident in the islets3,6, insulin-specific
germinal centres can be formed in various lymphoid tissues, suggesting
that insulin presentation is widespread7,8. Here we use live imaging to
document the distribution of insulin recognition by CD4 T cells
throughout various lymph nodes. Furthermore, we identify catabolized
insulin peptide fragments containing defined pathogenic epitopes in
β-cell granules from mice and humans. Upon glucose challenge, these
fragments are released into the circulation and are recognized by CD4 T
cells, leading to an activation state that results in transcriptional
reprogramming and enhanced diabetogenicity. Therefore, a tissue such as
pancreatic islets, by releasing catabolized products, imposes a constant
threat to self-tolerance. These findings reveal a self-recognition
pathway underlying a primary autoantigen and provide a foundation for
assessing antigenic targets that precipitate pathogenic outcomes by
systemically sensitizing lymphoid tissues.

PubMed: \url{https://www.ncbi.nlm.nih.gov/pubmed/?term=30022165}

{}

{}

\begin{center}\rule{0.5\linewidth}{\linethickness}\end{center}

 - \textbf{Comparison Between Modified Extracellular-Type
Trehalose-Containing Kyoto Solution and University of Wisconsin Solution
in 18-Hour Pancreas Preservation for Islet Transplantation}

\emph{Pancreas 2018 Aug;47(7):e46-e47}

PubMed: \url{https://www.ncbi.nlm.nih.gov/pubmed/?term=29985851}

{}

{}

\begin{center}\rule{0.5\linewidth}{\linethickness}\end{center}

 - \textbf{β-Catenin Expression in Glucagon-Producing α Cells of Human
Fetal Pancreatic Islets Suggests Wnt Signaling-Dependent Development}

\emph{Pancreas 2018 Sep;47(8):e54-e55}

PubMed: \url{https://www.ncbi.nlm.nih.gov/pubmed/?term=30113433}

{}

{}

\begin{center}\rule{0.5\linewidth}{\linethickness}\end{center}

 - \textbf{ASO Author Reflections: Serum Elastase 1 Level as a Risk
Factor for Postoperative Recurrence in Patients with Well-Differentiated
Pancreatic Neuroendocrine Neoplasms}

\emph{Annals of surgical oncology 2018 Aug;():}

PubMed: \url{https://www.ncbi.nlm.nih.gov/pubmed/?term=30136123}

{}

{}

\begin{center}\rule{0.5\linewidth}{\linethickness}\end{center}

 - \textbf{Surfactants Improve Live Cell Imaging of Human Pancreatic
Islets}

\emph{Pancreas 2018 Oct;47(9):1093-1100}

OBJECTIVES: Newport Green is a zinc-specific fluorescent dye developed
to monitor cellular zinc transport. In pancreatic islets with zinc-rich
β-cells, Newport Green is expected to be useful as an islet-specific
indicator for live imaging. However, the low penetration of Newport
Green into islets hinders clear detection. The aim of this study was to
develop a practical method of live islet imaging by using surfactants to
enhance the penetration efficiency. METHODS: Surfactants (F127, Tween
20, and Triton X-100) were co-incubated with Newport Green for
fluorescent imaging of live isolated human islet and nonislet tissues.
Toxicity, enhancement of Newport Green fluorescence, and effects on
specificity to islets were examined. RESULTS: Newport Green fluorescent
intensity was increased after co-incubation with all surfactants tested
(0.2-3.2 mM); however, surfactants were toxic to islets at high
concentrations. Within the nontoxic range, high specificity to islets
was observed when co-incubated with Tween 20 at 0.2-0.4 mM, compared
with F127 and Triton X-100. This optimized range successfully
distinguished islets from nonislet tissues using statistically
calculated cutoff value of Newport Green fluorescent intensity.
CONCLUSIONS: Surfactants, particularly Tween 20 in the optimized range,
effectively and selectively enhanced Newport Green fluorescence in live
islets without increasing islet toxicity.

PubMed: \url{https://www.ncbi.nlm.nih.gov/pubmed/?term=30142118}

{}

{}

\begin{center}\rule{0.5\linewidth}{\linethickness}\end{center}

\begin{itemize}
\tightlist
\item
  \textbf{Tissue heterogeneity contributes to suboptimal precision of
  WHO 2010 scoring criteria for Ki67 labeling index in a subset of
  neuroendocrine neoplasms of the pancreas}
\end{itemize}

\url{https://www.termedia.pl/Tissue-heterogeneity-contributes-to-suboptimal-precision-of-WHO-2010-scoring-criteria-for-Ki67-labeling-index-in-a-subset-of-neuroendocrine-neoplasms-of-the-pancreas,55,29474,1,1.html}

\begin{center}\rule{0.5\linewidth}{\linethickness}\end{center}

 - \textbf{ASO Author Reflections: Heterogeneity of Duodenal
Neuroendocrine Tumors}

\emph{Annals of surgical oncology 2018 Aug;():}

PubMed: \url{https://www.ncbi.nlm.nih.gov/pubmed/?term=30145647}

{}

{}

\begin{center}\rule{0.5\linewidth}{\linethickness}\end{center}

\begin{itemize}
\tightlist
\item
  \textbf{Pancreatic islet of Langerhans' cytoarchitecture and
  ultrastructure in normal glucose tolerance and in type 2 diabetes
  mellitus}
\end{itemize}

\url{https://onlinelibrary.wiley.com/doi/pdf/10.1111/dom.13380}

\begin{center}\rule{0.5\linewidth}{\linethickness}\end{center}

 - \textbf{Lymphadenectomy in pancreatic neuroendocrine neoplasms: Why
are we still debating?}

\emph{Pancreatology : official journal of the International Association
of Pancreatology (IAP) \ldots{} {[}et al.{]} 2018 Sep;():}

Pancreatic Neuroendocrine Neoplasms (PNEN) are rare tumours exhibiting
very heterogeneous behaviour. For these reasons, studies with high level
of evidence are lacking. Whether lymphadenectomy should be performed for
PNEN is a matter of debate. In this review, we perform a critical
analysis of the available literature regarding the clinical significance
of lymphnode metastases, the importance of lymphadenectomy, and the
implications on disease-specific survival.

PubMed: \url{https://www.ncbi.nlm.nih.gov/pubmed/?term=30253923}

{}

{}

\begin{center}\rule{0.5\linewidth}{\linethickness}\end{center}

\hypertarget{techniques-research-methods}{%
\subsubsection{Techniques \& Research
Methods}\label{techniques-research-methods}}

\begin{center}\rule{0.5\linewidth}{\linethickness}\end{center}

 - \textbf{A ``Clearer'' View of Pancreatic Pathology: A Review of
Tissue Clearing and Advanced Microscopy Techniques}

\emph{Advances in anatomic pathology 2018 Sep;():}

Although pathologic lesions in the pancreas are 3-dimensional (3D)
complex structures, we currently use thin 2D hematoxylin and eosin
stained slides to study and diagnose pancreatic pathology. Two
technologies, tissue clearing and advanced microscopy, have recently
converged, and when used together they open the remarkable world of 3D
anatomy and pathology to pathologists. Advances in tissue clearing and
antibody penetration now make even dense fibrotic tissues amenable to
clearing, and light sheet and confocal microscopies allow labeled cells
deep within these cleared tissues to be visualized. Clearing techniques
can be categorized as solvent-based or aqueous-based techniques, but
both clearing methods consist of 4 fundamental steps, including
pretreatment of specimens, permeabilization and/or removal of lipid,
immunolabeling with antibody penetration, and clearing by refractive
index matching. Specialized microscopes, including the light sheet
microscope, the 2-photon microscope, and the confocal microscope, can
then be used to visualize and evaluate the 3D histology. Both endocrine
and exocrine pancreas pathology can then be visualized. The application
of labeling and clearing to surgically resected human pancreatic
parenchyma can provide detailed visualization of the complexities of
normal pancreatic anatomy. It also can be used to characterize the 3D
architecture of disease processes ranging from precursor lesions, such
as pancreatic intraepithelial neoplasia lesions and intraductal
papillary mucinous neoplasms, to infiltrating pancreatic ductal
adenocarcinomas. The evaluation of 3D histopathology, including
pathology of the pancreatic lesions, will provide new insights into
lesions that previously were seen, and thought of, only in 2 dimensions.

PubMed: \url{https://www.ncbi.nlm.nih.gov/pubmed/?term=30256228}

{}

{}

\begin{center}\rule{0.5\linewidth}{\linethickness}\end{center}

\begin{itemize}
\tightlist
\item
  \textbf{Accurate 3D Reconstruction of a Whole Pancreatic Cancer Tumor
  from Pathology Images with Different Stains}
\end{itemize}

\url{https://link.springer.com/chapter/10.1007/978-3-030-00949-6_5}

\begin{center}\rule{0.5\linewidth}{\linethickness}\end{center}

\begin{itemize}
\tightlist
\item
  \textbf{Construction of a Generative Model of H\&E Stained Pathology
  Images of Pancreas Tumors Conditioned by a Voxel Value of MRI Image}
\end{itemize}

\url{https://link.springer.com/chapter/10.1007/978-3-030-00949-6_4}

\begin{center}\rule{0.5\linewidth}{\linethickness}\end{center}

 - \textbf{Organoidomics - falling star or new galaxy in pancreatic
cancer?}

\emph{Nature reviews. Gastroenterology \& hepatology 2018
Oct;15(10):586-587}

PubMed: \url{https://www.ncbi.nlm.nih.gov/pubmed/?term=30046146}

{}

{}

\begin{center}\rule{0.5\linewidth}{\linethickness}\end{center}

 - \textbf{The Research of Acellular Pancreatic Bioscaffold as a Natural
3-Dimensional Platform In Vitro}

\emph{Pancreas 2018 Sep;47(8):1040-1049}

OBJECTIVE: The aim of the study was to investigate the biochemical and
functional properties of a rat acellular pancreatic bioscaffolds (APBs).
METHODS: Fresh pancreata from 10 rats were soaked and perfused through
portal veins using Easy-Load Digital Drive peristaltic pumps. The
histological structure, extracellular matrix composition, and the DNA
content of the APBs were evaluated. Biocompatibility studies had also
been performed. The proliferation and differentiation of AR42J
pancreatic acinar cells were assessed. RESULTS: The pancreatic tissue
became translucent after decellularization. There were no visible
vascular endothelial cells, cellular components, or cracked cellular
debris. The extracellular matrix components were not decreased after
decellularization (P \textgreater{} 0.05); however, the DNA content was
decreased significantly (P \textless{} 0.05). The subcutaneous
implantation sites showed low immunological response and low
cytotoxicity around the APB. The proliferation rate was higher and the
apoptosis rate was lower when AR42J cells were cultured on APB (P
\textless{} 0.05). The gene expression and the protein expression were
higher for the APB group (P \textless{} 0.001). CONCLUSIONS: Our
findings support the biological utility of whole pancreas APBs as
biomaterial scaffolds, which provides an improved approach for
regenerative medicine.

PubMed: \url{https://www.ncbi.nlm.nih.gov/pubmed/?term=30086100}

{}

{}

\begin{center}\rule{0.5\linewidth}{\linethickness}\end{center}

 - \textbf{Transforming growth factor-β modulates pancreatic cancer
associated fibroblasts cell shape, stiffness and invasion}

\emph{Biochimica et biophysica acta. General subjects 2018
Jul;1862(7):1537-1546}

BACKGROUND: Tumor microenvironment consists of the extracellular matrix
(ECM), stromal cells, such as fibroblasts (FBs) and cancer associated
fibroblasts (CAFs), and a myriad of soluble factors. In many tumor
types, including pancreatic tumors, the interplay between stromal cells
and the other tumor microenvironment components leads to desmoplasia, a
cancer-specific type of fibrosis that hinders treatment. Transforming
growth factor beta (TGF-β) and CAFs are thought to play a crucial role
in this tumor desmoplastic reaction, although the involved mechanisms
are unknown. METHODS: Optical/fluorescence microscopy, atomic force
microscopy, image processing techniques, invasion assay in 3D collagen I
gels and real-time PCR were employed to investigate the effect of TGF-β
on normal pancreatic FBs and CAFs with regard to crucial cellular
morphodynamic characteristics and relevant gene expression involved in
tumor progression and metastasis. RESULTS: CAFs present specific
myofibroblast-like characteristics, such as α-smooth muscle actin
expression and cell elongation, they also form more lamellipodia and are
softer than FBs. TGF-β treatment increases cell stiffness (Young's
modulus) of both FBs and CAFs and increases CAF's (but not FB's)
elongation, cell spreading, lamellipodia formation and spheroid
invasion. Gene expression analysis shows that these morphodynamic
characteristics are mediated by Rac, RhoA and ROCK expression in CAFs
treated with TGF-β. CONCLUSIONS: TGF-β modulates CAFs', but not FBs',
cell shape, stiffness and invasion. GENERAL SIGNIFICANCE: Our findings
elucidate on the effects of TGF-β on CAFs' behavior and stiffness
providing new insights into the mechanisms involved.

PubMed: \url{https://www.ncbi.nlm.nih.gov/pubmed/?term=29477748}

{}

{}

\begin{center}\rule{0.5\linewidth}{\linethickness}\end{center}

\begin{itemize}
\tightlist
\item
  \textbf{Human Organoids Share Structural and Genetic Features with
  Primary Pancreatic Adenocarcinoma Tumors}
\end{itemize}

\url{http://mcr.aacrjournals.org/content/molcanres/early/2018/08/31/1541-7786.MCR-18-0531.full.pdf}

\begin{center}\rule{0.5\linewidth}{\linethickness}\end{center}

\hypertarget{animal-studies}{%
\subsubsection{Animal Studies}\label{animal-studies}}

\begin{center}\rule{0.5\linewidth}{\linethickness}\end{center}

 - \textbf{Inverse Correlation of STAT3 and MEK Signaling Mediates
Resistance to RAS Pathway Inhibition in Pancreatic Cancer}

\emph{Cancer research 2018 Aug;():}

Major contributors to therapeutic resistance in pancreatic cancer (PDAC)
include Kras mutations, a dense desmoplastic stroma that prevents drug
delivery to the tumor, and activation of redundant signaling pathways.
We have previously identified a mechanistic rationale for targeting
STAT3 signaling to overcome therapeutic resistance in PDAC. In this
study, we investigate the molecular mechanisms underlying the
heterogeneous response to STAT3 and RAS pathway inhibition in PDAC.
Effects of JAK/STAT3 inhibition (STAT3i) or MEK inhibition (MEKi) were
established in Ptf1acre/+;LSL-KrasG12D/+;Tgfbr2flox/flox (PKT) mice and
patient-derived xenografts (PDX). Amphiregulin (AREG) levels were
determined in serum from human PDAC patients,
LSL-KrasG12D/+;Trp53R172H/+;Pdx1Cre/+ (KPC), and PKT mice.
MEKi/STAT3i-treated tumors were analyzed for integrity of the pancreas
and the presence of cancer stem cells (CSC). We observed an inverse
correlation between ERK and STAT3 phosphorylation. MEKi resulted in
immediate activation of STAT3, while STAT3i resulted in TACE-induced,
AREG-dependent activation of EGFR and ERK. Combined MEKi/STAT3i
sustained blockade of ERK, EGFR, and STAT3 signaling, overcoming
resistance to individual MEKi or STAT3i. This combined inhibition
attenuated tumor growth in PDX and increased survival of PKT mice while
reducing serum AREG levels. Furthermore, MEKi/STAT3i altered the PDAC
tumor microenvironment by depleting tumor fibrosis, maintaining
pancreatic integrity, and downregulating CD44+ and CD133+ CSC. These
results demonstrate that resistance to MEKi is mediated through
activation of STAT3, while TACE-AREG-EGFR-dependent activation of RAS
pathway signaling confers resistance to STAT3 inhibition. Combined
MEKi/STAT3i overcomes these resistances and provides a novel therapeutic
strategy to target the RAS and STAT3 pathway in PDAC.

PubMed: \url{https://www.ncbi.nlm.nih.gov/pubmed/?term=30154150}

{}

{}

\begin{center}\rule{0.5\linewidth}{\linethickness}\end{center}

 - \textbf{GNASR201C Induces Pancreatic Cystic Neoplasms in Mice That
Express Activated KRAS by Inhibiting YAP1 Signaling}

\emph{Gastroenterology 2018 Aug;():}

BACKGROUND \& AIMS: Mutations at hotspots in GNAS, which encodes
stimulatory G-protein, α subunits, are detected in approximately 60\% of
intraductal papillary mucinous neoplasms (IPMNs) of the pancreas. We
generated mice with KRAS-induced IPMNs that also express a
constitutively active form of GNAS in pancreas and studied tumor
development. METHODS: We generated p48-Cre; LSL-KrasG12D;
Rosa26R-LSL-rtTA-TetO-GnasR201C mice (Kras;Gnas mice); pancreatic
tissues of these mice express activated KRAS and also express a mutant
form of GNAS (GNASR201C) upon doxycycline administration. Mice that were
not given doxycycline were used as controls, and survival times were
compared by Kaplan-Meier analysis. Pancreata were collected at different
time points after doxycycline administration and analyzed by histology.
Pancreatic ductal adenocarcinomas (PDACs) were isolated from mice and
used to generate cell lines, which were analyzed by reverse
transcription polymerase chain reaction, immunoblotting,
immunohistochemistry, and colony formation and invasion assays.
Full-length and mutant forms of yes-associated protein (YAP) were
expressed in PDAC cells. IPMN specimens were obtained from 13 patients
with IPMN undergoing surgery and analyzed by immunohistochemistry.
RESULTS: All Kras;Gnas mice developed pancreatic cystic lesions that
resemble human IPMNs; the grade of epithelial dysplasia increased with
time. None of the control mice developed cystic lesions. Approximately
one third of Kras;Gnas mice developed PDACs at a median of 30 weeks
after doxycycline administration, whereas 33\% of control mice~developed
PDACs. Expression of GNASR201C did not accelerate the development of
PDACs compared with control mice. However, the neoplasms observed in
Kras;Gnas mice were more differentiated, and expressed more genes
associated with ductal phenotypes, than in control mice. PDACs isolated
from Kras;Gnas mice had activation of the Hippo pathway; in cells from
these tumors, phosphorylated YAP1 was sequestered in the cytoplasm, and
this was also observed in human IPMNs with GNAS mutations. Sequestration
of YAP1~was not observed in PDAC cells from control mice. CONCLUSIONS:
In mice that express activated KRAS in the pancreas, we found expression
of GNASR201C to cause development of more differentiated tumors, with
gene expression pattern associated with the ductal phenotype. Expression
of mutant GNAS caused phosphorylated YAP1 to be sequestered in the
cytoplasm, altering tumor progression.

PubMed: \url{https://www.ncbi.nlm.nih.gov/pubmed/?term=30142336}

{}

{}

\begin{center}\rule{0.5\linewidth}{\linethickness}\end{center}

 - \textbf{Characterization of Peribiliary Gland-Constituting Cells
Based on Differential Expression of Trophoblast Cell Surface Protein 2
in Biliary Tract}

\emph{The American journal of pathology 2018 Sep;188(9):2059-2073}

Peribiliary glands (PBGs) are accessory glands with mucinous and serous
acini in the biliary tree. The PBG is composed of a heterogeneous cell
population, such as mucus- and pancreatic enzyme-producing epithelial
cells, whereas it constitutes niches for multipotential stem/progenitor
cells in the human extrahepatic bile duct (EHBD). By contrast, the
nature of PBGs in the mouse EHBD remains unclear. Our aim was to
establish a method for isolating and characterizing PBG-constituting
cells in the mouse EHBD. We found that trophoblast cell surface protein
2 (Trop2) was expressed in the luminal epithelium of mouse EHBD
exclusively, but not in the PBG. On the basis of the differential
expression profile of Trop2, lumen-forming biliary epithelial cells
(LBECs) and PBG-constituting epithelial cells (PBECs) were separately
isolated for further characterization. Gene expression analysis revealed
that the isolated mouse PBECs expressed several marker genes related to
human PBGs. In the colony formation assay, PBECs showed significantly
higher colony formation capacity than LBECs. In the organoid formation
assay, PBECs formed cystic organoid with LBEC-like phenotype.
Interestingly, PBECs proliferated, accompanied by reexpression of Trop2
in~vivo after bile duct ligation. Furthermore, the unique expression
profile of Trop2 was conserved in human EHBD. Our findings indicate that
Trop2 is a useful marker in investigating the pathophysiological roles
and characteristics of mouse and human PBGs in biliary diseases.

PubMed: \url{https://www.ncbi.nlm.nih.gov/pubmed/?term=30126547}

{}

{}

\begin{center}\rule{0.5\linewidth}{\linethickness}\end{center}

 - \textbf{Disruption of C1galt1 Gene Promotes Development and
Metastasis of Pancreatic Adenocarcinomas in Mice}

\emph{Gastroenterology 2018 Aug;():}

BACKGROUND \& AIMS: Pancreatic ductal adenocarcinomas (PDACs) produce
higher levels of truncated O-glycan structures (such as Tn and sTn) than
normal pancreata. Dysregulated activity of core 1 synthase
glycoprotein-N-acetylgalactosamine 3-β-galactosyltransferase 1 (C1GALT1)
leads to increased expression of these truncated O-glycans. We
investigated whether and how truncated O-glycans contributes to the
development and progression of PDAC using mice with disruption of
C1galt1. METHODS: We crossed C1galt1 floxed mice (C1galt1loxP/loxP) with
KrasG12D/+; Trp53R172H/+; Pdx1-Cre (KPC) mice to create KPCC mice.
Growth and progression of pancreatic tumors were compared between KPC
and KPCC mice; pancreatic tissues were collected and analyzed by
immunohistochemistry; immunofluorescence; and Sirius red, alcian blue,
and lectin staining. We used the CRISPR/Cas9 system to disrupt C1galt1
in human PDAC cells (T3M4 and CD18/HPAF) and levels of O-glycans were
analyzed by lectin blotting, mass spectrometry, and lectin pulldown
assay. Orthotopic studies and RNA sequencing analyses were performed
with control and C1galt1 knockout PDAC cells. C1galt1 expression was
analyzed in well-differentiated (n~= 36) and poorly differentiated (n~=
23) PDAC samples by immunohistochemistry. RESULTS: KPCC mice had
significantly shorter survival times (median 102 days) than KPC mice
(median 200 days) and developed early pancreatic intraepithelial
neoplasias at 3 weeks, PDAC at 5 weeks, and metastases at 10 weeks
compared with KPC mice. Pancreatic tumors that developed in KPCC mice
were more aggressive than those in KPC mice (more invasive and
metastases), had a decreased amount of stroma, and had increased
production of Tn. Poorly differentiated PDAC specimens had significantly
lower levels of C1GALT1 than well-differentiated PDACs. Human PDAC cells
with knockout of C1galt1 had aberrant glycosylation of MUC16 compared
with control cells and increased expression of genes that regulate
tumorigenesis and metastasis. CONCLUSIONS: In studies of KPC mice with
disruption of C1galt1, we found that loss of C1galt1 promotes
development of aggressive PDACs and increased metastasis. Knockout of
C1galt1 leads to increased tumorigenicity and truncation of
O-glycosylation on MUC16, which could contribute to increased
aggressiveness.

PubMed: \url{https://www.ncbi.nlm.nih.gov/pubmed/?term=30086262}

{}

{}

\begin{center}\rule{0.5\linewidth}{\linethickness}\end{center}

 - \textbf{Local Phototherapy Synergizes with Immunoadjuvant for
Treatment of Pancreatic Cancer through Induced Immunogenic Tumor
Vaccine}

\emph{Clinical cancer research : an official journal of the American
Association for Cancer Research 2018 Aug;():}

Purpose: To develop a synergistic combination therapy for advanced
pancreatic cancer, using local phototherapy and immunotherapy, and to
determine the efficacy and mechanism of the novel combination therapy
using a highly metastatic pancreatic tumor model in mice.Experimental
Design: Mice bearing Panc02-H7 pancreatic tumors (both subcutaneous and
orthotopic) were treated with noninvasive or interventional photothermal
therapy, followed by local application of an immunoadjuvant. Tumor
growth and animal survival were assessed. Immune cell populations within
spleen and tumors were evaluated by FACS and IHC, and cytokine levels
were determined by ELISA.Results: Up to 75\% of mice bearing
subcutaneous tumors treated with combination therapy had complete tumor
regression. Local photothermal therapy exposed/released
damage-associated molecular patterns, which initiated an immunogenic
tumor cell death, resulting in infiltration of antigen-presenting cells
and Th1 immunity. Concomitant application of immunoadjuvant amplified
Th1 immunity, especially the tumor-specific cytotoxic T lymphocyte
response, with increased quantity and quality of T cells. Combination
therapy also induced tumor-specific immune memory, as demonstrated by
resistance to tumor rechallenge and production of memory T cells. For
the treatment of orthotopic tumor, the combination therapy significantly
reduced the primary tumors and metastases, and prolonged the animal
survival time.Conclusions: This study indicated that combination of
local phototherapy and immunotherapy induced a systemic immunity against
established tumors and metastases in an aggressive, preclinical
pancreatic tumor model, leading to a potential clinical method for
patients with advanced pancreatic cancer. Clin Cancer Res; 1-12. ©2018
AACR.

PubMed: \url{https://www.ncbi.nlm.nih.gov/pubmed/?term=30068705}

{}

{}

\begin{center}\rule{0.5\linewidth}{\linethickness}\end{center}

 - \textbf{Systemic Depletion of Nerve Growth Factor Inhibits Disease
Progression in a Genetically Engineered Model of Pancreatic Ductal
Adenocarcinoma}

\emph{Pancreas 2018 Aug;47(7):856-863}

OBJECTIVES: In patients with pancreatic ductal adenocarcinoma (PDAC),
increased expression of proinflammatory neurotrophic growth factors (eg,
nerve growth factor {[}NGF{]}) correlates with a poorer prognosis,
perineural invasion, and, with regard to NGF, pain severity. We
hypothesized that NGF sequestration would reduce inflammation and
disease in the KPC mouse model of PDAC. METHODS: Following biweekly
injections of NGF antibody or control immunoglobulin G, beginning at 4
or 8 weeks of age, inflammation and disease stage were assessed using
histological, protein expression, and quantitative polymerase chain
reaction analyses. RESULTS: In the 8-week anti-NGF group, indicators of
neurogenic inflammation in the dorsal root ganglia (substance P and
calcitonin gene-related peptide) and spinal cord (glial fibrillary
acidic protein) were significantly reduced. In the 4-week anti-NGF
group, TRPA1 mRNA in dorsal root ganglia and spinal phosphorylated ERK
protein were elevated, but glial fibrillary acidic protein expression
was unaffected. In the 8-week anti-NGF group, there was a 40\% reduction
in the proportion of mice with microscopic perineural invasion, and no
macrometastases were observed. CONCLUSIONS: Anti-NGF treatment beginning
at 4 weeks may increase inflammation and negatively impact disease.
Treatment starting at 8 weeks (after disease onset), however, reduces
neural inflammation, neural invasion, and metastasis. These data
indicate that NGF impacts PDAC progression and metastasis in a
temporally dependent manner.

PubMed: \url{https://www.ncbi.nlm.nih.gov/pubmed/?term=29975347}

{}

{}

\begin{center}\rule{0.5\linewidth}{\linethickness}\end{center}

 - \textbf{ARID1A, a SWI/SNF subunit, is critical to acinar cell
homeostasis and regeneration and is a barrier to transformation and
epithelial-mesenchymal transition in the pancreas}

\emph{Gut 2018 Sep;():}

OBJECTIVE: Here, we evaluate the contribution of AT-rich interaction
domain-containing protein 1A (ARID1A), the most frequently mutated
member of the SWItch/sucrose non-fermentable (SWI/SNF) complex, in
pancreatic homeostasis and pancreatic ductal adenocarcinoma (PDAC)
pathogenesis using mouse models. DESIGN: Mice with a targeted deletion
of Arid1a in the pancreas by itself and in the context of two common
genetic alterations in PDAC, Kras and p53, were followed longitudinally.
Pancreases were examined and analysed for proliferation, response to
injury and tumourigenesis. Cancer cell lines derived from these models
were analysed for clonogenic, migratory, invasive and transcriptomic
changes. RESULTS: Arid1a deletion in the pancreas results in progressive
acinar-to-ductal metaplasia (ADM), loss of acinar mass, diminished
acinar regeneration in response to injury and ductal cell expansion.
Mutant Kras cooperates with homozygous deletion of Arid1a, leading to
intraductal papillary mucinous neoplasm (IPMN). Arid1a loss in the
context of mutant Kras and p53 leads to shorter tumour latency, with the
resulting tumours being poorly differentiated. Cancer cell lines derived
from Arid1a-mutant tumours are more mesenchymal, migratory, invasive and
capable of anchorage-independent growth; gene expression analysis showed
activation of epithelial-mesenchymal transition (EMT) and stem cell
identity pathways that are partially dependent on Arid1a loss for
dysregulation. CONCLUSIONS: ARID1A plays a key role in pancreatic acinar
homeostasis and response to injury. Furthermore, ARID1A restrains
oncogenic KRAS-driven formation of premalignant proliferative IPMN.
Arid1a-deficient PDACs are poorly differentiated and have mesenchymal
features conferring migratory/invasive and stem-like properties.

PubMed: \url{https://www.ncbi.nlm.nih.gov/pubmed/?term=30228219}

{}

{}

\begin{center}\rule{0.5\linewidth}{\linethickness}\end{center}

\begin{itemize}
\tightlist
\item
  \textbf{The loss of ATRX increases susceptibility to pancreatic injury
  and oncogenic KRAS in female but not male mice}
\end{itemize}

\url{https://ac.els-cdn.com/S2352345X18301255/1-s2.0-S2352345X18301255-main.pdf?_tid=14d4be90-2c1b-4f3e-a7f1-e6472ebc4325\&acdnat=1537522502_724a83ab9e728e07620a6f391eb108b4}

\begin{center}\rule{0.5\linewidth}{\linethickness}\end{center}

\begin{itemize}
\tightlist
\item
  \textbf{Hmga2 is dispensable for pancreatic cancer development,
  metastasis, and therapy resistance}
\end{itemize}

\url{https://www.nature.com/articles/s41598-018-32159-x}

\begin{center}\rule{0.5\linewidth}{\linethickness}\end{center}

\begin{itemize}
\tightlist
\item
  \textbf{Inhibition of 15-PGDH causes Kras-driven tumor expansion
  through prostaglandin E2-ALDH1 signaling in the pancreas}
\end{itemize}

\url{https://www.nature.com/articles/s41388-018-0510-y}

\begin{center}\rule{0.5\linewidth}{\linethickness}\end{center}

\begin{center}\rule{0.5\linewidth}{\linethickness}\end{center}

\hypertarget{database-studies}{%
\subsubsection{Database Studies}\label{database-studies}}

\begin{center}\rule{0.5\linewidth}{\linethickness}\end{center}

 - \textbf{Elevated expression of NFE2L3 predicts the poor prognosis of
pancreatic cancer patients}

\emph{Cell cycle (Georgetown, Tex.) 2018 ;17(17):2164-2174}

The highly malignant feature and difficulties for early diagnosis are
the key reasons contributing to the poor prognosis of pancreatic cancer
(PC) patients. NFE2L3 is a nuclear transcription factor, which has been
reported an important biomarker of several tumors. But the role of
NFE2L3 in PC remained undefined. Herein, through qPCR and
immunohistochemistry, we found a significantly increased NFE2L3 in PC
tissues as compared with adjacent non-tumor tissues. While reducing
NFE2L3 expression suppressed the invasion abilities of PC cells,
elevated NFE2L3 was found associated with lymph node metastasis
(P~=~0.001; HR~=~3.95; 95\% CI: 1.70 - 9.17) and advanced TNM stages
(P~\textless{}~0.001; HR~=~4.06; 95\% CI: 1.74 - 9.46). Consistently,
data from both our two cohorts and the TCGA database revealed that
higher NFE2L3 PC carriers had worse outcomes than those lower NFE2L3
expressers. Lastly, we confirmed the regulatory role of NFE2L3 on VEGFA,
an important player involved in tumor angiogenesis. Collectively, our
investigations suggested the oncogenic role of NFE2L3 in PC development
and provided the rational for future adding NFE2L3 for the risk
assessment of PC patients. NFE2L3: NF-E2-related factor 3; UHMK1: U2AF
homology motif kinase 1; VEGFA: vascular endothelial growth factor A;
GEO: gene expression omnibus; TCGA: The Cancer Genome Atlas; HPDE: human
pancreas duct cells; OS: overall survival; IHC: immunohistochemistry;
FFPE: formalin-fixed and paraffin-embedded; SEM: standard error of mean.

PubMed: \url{https://www.ncbi.nlm.nih.gov/pubmed/?term=30196752}

{}

{}

\begin{center}\rule{0.5\linewidth}{\linethickness}\end{center}

 - \textbf{A Prognostic Nomogram for Disease-Specific Survival in
Patients with Pancreatic Ductal Adenocarcinoma of the Head of the
Pancreas Following Pancreaticoduodenectomy}

\emph{Medical science monitor : international medical journal of
experimental and clinical research 2018 Sep;24():6313-6321}

BACKGROUND This study developed and validated a nomogram to predict
patient prognosis for pancreatic ductal adenocarcinoma (PDAC) of the
head of the pancreas following pancreaticoduodenectomy. MATERIAL AND
METHODS Retrospective data were obtained from 4,383 patients with PDAC
of the head of the pancreas who underwent pancreaticoduodenectomy
between 2004-2013 from 11 Registries Research Data of the Surveillance,
Epidemiology,and End Results (SEER) database. Cox proportional hazards
model was used to identify independent risk factors. The predictive
accuracy of the nomogram was determined by the concordance index
(C-index) and calibration curve. The results were externally validated
by comparison with data from 1,743 patients from 7 other Registries
Research Data. RESULTS Of the 4,383 patients in the training dataset,
median disease-specific survival (DSS) was 17.0 months (range, 1.0-131
months), and postoperative 1-year, 3-year, and 5-year DSS rates were
70.3\%, 26.1\%, and 16.8\%, respectively. Multivariate analysis showed
that patient sex, age, tumor grade, regional lymph nodes examined,
positive regional lymph nodes, tumor size, extent of local invasion, and
tumor metastases were independent risk factors for DSS. The C-index of
the internal validation dataset for prediction of DSS was 0.64 (95\% CI,
0.63-0.65), which was superior to the American Joint Committee on Cancer
(AJCC) staging, 0.57 (95\% CI, 0.56-0.58) (P\textless{}0.001). The
5-year DSS rates and median DSS time for patients in the low-risk group
were significantly greater compared with high-risk group
(P\textless{}0.001). CONCLUSIONS A validated prognostic disease-specific
nomogram for patient survival in PDAC of the head of the pancreas
following pancreaticoduodenectomy was developed.

PubMed: \url{https://www.ncbi.nlm.nih.gov/pubmed/?term=30198517}

{}

{}

\begin{center}\rule{0.5\linewidth}{\linethickness}\end{center}

 - \textbf{Associations between autoimmune conditions and hepatobiliary
cancer risk among elderly US adults}

\emph{International journal of cancer 2018 Aug;():}

Growing evidence suggests that people with autoimmune conditions may be
at increased risk of hepatobiliary tumors. In the present study, we
evaluated associations between autoimmune conditions and hepatobiliary
cancers among adults aged ≥66 in the United States. We used
Surveillance, Epidemiology, and End Results (SEER)-Medicare data
(1992-2013) to conduct a population-based, case-control study. Cases
(n=32,443) had primary hepatobiliary cancer. Controls (n=200,000) were
randomly selected, cancer-free adults frequency-matched to cases by sex,
age, and year of selection. Using multivariate logistic regression, we
calculated odds ratios (ORs) and 95\% confidence intervals (CIs) for
associations with 39 autoimmune conditions identified via Medicare
claims. We also conducted separate analyses for diagnoses obtained via
inpatient versus outpatient claims. Sixteen conditions were associated
with at least one hepatobiliary cancer. The strongest risk estimates
were for primary biliary cholangitis with hepatocellular carcinoma (OR:
31.33 {[}95\% CI: 23.63-41.56{]}) and primary sclerosing cholangitis
with intrahepatic cholangiocarcinoma (7.53 {[}5.73-10.57{]}),
extrahepatic cholangiocarcinoma (5.59 {[}4.03-7.75{]}), gallbladder
cancer (2.06 {[}1.27-3.33{]}), and ampulla of Vater cancer (6.29
{[}4.29-9.22{]}). Associations with hepatobiliary-related conditions as
a group were observed across nearly all cancer sites (ORs ranging from
4.53 {[}95\% CI: 3.30-6.21{]} for extrahepatic cholangiocarcinoma to
7.18 {[}5.94-8.67{]} for hepatocellular carcinoma). Restricting to
autoimmune conditions diagnosed via inpatient claims, 6 conditions
remained associated with at least one hepatobiliary cancer, and several
risk estimates increased. In the outpatient restricted analysis, 12
conditions remained associated. Multiple autoimmune conditions are
associated with hepatobiliary cancer risk in the US Medicare population,
supporting a shared immuno-inflammatory etiology to these cancers. This
article is protected by copyright. All rights reserved.

PubMed: \url{https://www.ncbi.nlm.nih.gov/pubmed/?term=30155920}

{}

{}

\begin{center}\rule{0.5\linewidth}{\linethickness}\end{center}

\begin{itemize}
\tightlist
\item
  Comparative outcomes of adenosquamous carcinoma of the pancreas: An
  analysis of the National Cancer Database
\end{itemize}

\url{https://onlinelibrary.wiley.com/doi/10.1002/jso.25112}

\begin{center}\rule{0.5\linewidth}{\linethickness}\end{center}

 - \textbf{Integrating MicroRNA Expression Profiling Studies to
Systematically Evaluate the Diagnostic Value of MicroRNAs in Pancreatic
Cancer and Validate Their Prognostic Significance with the Cancer Genome
Atlas Data}

\emph{Cellular physiology and biochemistry : international journal of
experimental cellular physiology, biochemistry, and pharmacology 2018
;49(2):678-695}

BACKGROUND/AIMS: MicroRNAs (miRNAs) are promising biomarkers for
pancreatic cancer (PaCa). However, systemic and unified evaluations of
the diagnostic value of miRNAs are lacking. Therefore, we performed a
systematic evaluation based on miRNA expression profiling studies.
METHODS: We obtained miRNA expression profiling studies from Gene
Expression Omnibus (GEO) and ArrayExpress (AE) databases and calculated
the pooled sensitivity, specificity, and summary area under a receiver
operating characteristic (ROC) curve for every miRNA. According to the
area under the curve (AUC), we identified the miRNAs with diagnostic
potentiality and validated their prognostic role in The Cancer Genome
Atlas (TCGA) data. Gene Ontology (GO) annotations and pathway
enrichments of the target genes of the miRNAs were evaluated using
bioinformatics tools. RESULTS: Ten miRNA expression profiling studies
including 958 patients were used in this diagnostic meta-analysis. A
total of 693 miRNAs were measured in more than 9 studies. The top 50
miRNAs with high predictive values for PaCa were identified. Among them,
miR-130b had the best predictive value for PaCa (pooled sensitivity:
0.73 {[}95\% confidence intervals (CI) 0.44-0.91{]}, specificity: 0.81
{[}95\% CI 0.59-0.93{]}, and AUC: 0.84 {[}95\% CI 0.73-0.95{]}). We
identified nine miRNAs (miR-23a, miR-30a, miR-125a, miR-129-1,
miR-181b-1, miR-203, miR-221, miR-222, and miR-1301) associated with
overall survival in PaCa patients by combining our results with TCGA
data. The results of a Cox model revealed that two miRNAs (miR-30a
{[}hazard ratio (HR)=2.43, 95\% CI 1.05-5.59; p=0.037{]} and miR-203
{[}HR=3.14, 95\% CI 1.28-7.71; p=0.012{]}) were independent risk factors
for prognosis in PaCa patients. In total, 405 target genes of the nine
miRNAs were enriched with Kyoto Encyclopedia of Genes and Genomes (KEGG)
pathways, and cancer-associated pathways such as Ras signaling pathways,
phospholipase D signaling pathway, and AMP-activated protein kinase
(AMPK) signaling pathway were revealed among the top 20 enriched
pathways. There were significant negative correlations between
miR-181b-1 and miR-125a expression levels and the methylation status of
their promoter region. CONCLUSION: Our study performed a systematic
evaluation of the diagnostic value of miRNAs based on miRNA expression
profiling studies. We identified that miR-23a, miR-30a, miR-125a,
miR-129-1, miR-181b-1, miR-203, miR-221, miR-222, and miR-1301 had
moderate diagnostic value for PaCa and predicted overall survival in
PaCa patients.

PubMed: \url{https://www.ncbi.nlm.nih.gov/pubmed/?term=30165365}

{}

{}

\begin{center}\rule{0.5\linewidth}{\linethickness}\end{center}

 - \textbf{Proposal of a modified American Joint Committee on Cancer
staging scheme for resectable pancreatic ductal adenocarcinoma with a
lymph node ratio-based N classification: A retrospective cohort study}

\emph{Medicine 2018 Aug;97(34):e12094}

The recently launched 8th edition of the American Joint Committee on
Cancer (AJCC) staging scheme for pancreatic ductal adenocarcinoma (PDAC)
did not account for the impact of the total examined lymph node count on
prognostic accuracy. In this population-based cohort study, we proposed
a modified AJCC staging scheme by incorporating a lymph node ratio
(LNR)-based N classification for patients with resectable PDAC.We
analyzed 8615 patients with resectable PDAC from the Surveillance,
Epidemiology, and End Results database between 2004 and 2013. The
optimal cut-off points for LNR were identified by recursive
partitioning, and an LNR-based N classification was designed
accordingly.The LNR-based N classification could further stratify
patients with the 8th AJCC N1 and N2 disease into subgroups with
significantly different overall survival (P \textless{} .001 for both).
By replacing the 8th AJCC N classification with the corresponding
LNR-based N classification, we further proposed a modified AJCC staging
scheme. The modified AJCC staging outperformed the 8th AJCC staging in
terms of the discriminatory capacity measured by the concordance index
and Akaike information criterion, and the prognostic homogeneity
assessed by using the likelihood ratio chi-squared test and stratified
survival analysis.Replacing the 8th AJCC N classification with the
LNR-based N classification can improve the prognostic performance of the
8th AJCC staging scheme for PDAC.

PubMed: \url{https://www.ncbi.nlm.nih.gov/pubmed/?term=30142869}

{}

{}

\begin{center}\rule{0.5\linewidth}{\linethickness}\end{center}

\begin{itemize}
\tightlist
\item
  \textbf{Pancreatic cancer survival analysis defines a signature that
  predicts outcome}
\end{itemize}

\url{http://journals.plos.org/plosone/article?id=10.1371/journal.pone.0201751}

\begin{center}\rule{0.5\linewidth}{\linethickness}\end{center}

 - \textbf{Pancreatic cancer survival analysis defines a signature that
predicts outcome}

\emph{PloS one 2018 ;13(8):e0201751}

Pancreatic ductal adenocarcinoma (PDAC) is the third leading cause of
cancer death in the US. Despite multiple large-scale genetic sequencing
studies, identification of predictors of patient survival remains
challenging. We performed a comprehensive assessment and integrative
analysis of large-scale gene expression datasets, across multiple
platforms, to enable discovery of a prognostic gene signature for
patient survival in pancreatic cancer. PDAC RNA-Sequencing data from The
Cancer Genome Atlas was stratified into Survival+ (\textgreater{}2-year
survival) and Survival-(\textless{}1-year survival) cohorts (n = 47).
Comparisons of RNA expression profiles between survival groups and
normal pancreatic tissue expression data from the Gene Expression
Omnibus generated an initial PDAC specific prognostic differential
expression gene list. The candidate prognostic gene list was then
trained on the Australian pancreatic cancer dataset from the ICGC
database (n = 103), using iterative sampling based algorithms, to derive
a gene signature predictive of patient survival. The gene signature was
validated in 2 independent patient cohorts and against existing PDAC
subtype classifications. We identified 707 candidate prognostic genes
exhibiting differential expression in tumor versus normal tissue. A
substantial fraction of these genes was also found to be differentially
methylated between survival groups. From the candidate gene list, a
5-gene signature (ADM, ASPM, DCBLD2, E2F7, and KRT6A) was identified.
Our signature demonstrated significant power to predict patient survival
in two distinct patient cohorts and was independent of AJCC TNM staging.
Cross-validation of our gene signature reported a better ROC AUC (≥ 0.8)
when compared to existing PDAC survival signatures. Furthermore,
validation of our signature through immunohistochemical analysis of
patient tumor tissue and existing gene expression subtyping data in
PDAC, demonstrated a correlation to the presence of vascular invasion
and the aggressive squamous tumor subtype. Assessment of these genes in
patient biopsies could help further inform risk-stratification and
treatment decisions in pancreatic cancer.

PubMed: \url{https://www.ncbi.nlm.nih.gov/pubmed/?term=30092011}

{}

{}

\begin{center}\rule{0.5\linewidth}{\linethickness}\end{center}

 - \textbf{A simple matrix to predict treatment success and long-term
survival among patients undergoing pancreatectomy}

\emph{HPB : the official journal of the International Hepato Pancreato
Biliary Association 2018 Aug;():}

BACKGROUND: A more accurate measure of long-term survival among patients
who have undergone a successful resection for pancreatic adenocarcinoma
may be computed by accounting for time already survived during the
initial treatment window. METHODS: Patients diagnosed with pancreatic
adenocarcinoma, from 2004 through 2013, were identified from the
American College of Surgeons National Cancer Database (NCDB). A
risk-stratification matrix was constructed including age,
histopathologic factors and the use of adjuvant therapy, given
successful treatment and survival at 3-month following diagnosis.
RESULTS: A total of 25,897 patients (50\% male, 53\%~\textgreater{}65
years of age) presented with stage I-III pancreatic cancer. The majority
of patients had tumors \textgreater{}2~cm size (82\%), grade I/II
(65\%), lymphatic invasion (LI) (66\%), and negative margins (76\%). A
survival advantage for adjuvant therapy was observed among all patients,
independent of their risk-profile. For example, a patient ≤65 years of
age, with early stage cancer (size ≤2~cm, grade I/II,~-ve LI,~-ve
margins) who received adjuvant therapy had a 62\% probability of being
alive beyond three years (95\%CI~=~59\%-66\%). In contrast, the survival
probability decreased to 53\% (95\%CI~=~59\%-66\%) without adjuvant
therapy. CONCLUSIONS: These results provide surgeons and patients with
more accurate information regarding long-term survival, as well as the
benefit of opting for adjuvant therapy after successful pancreatic
surgery.

PubMed: \url{https://www.ncbi.nlm.nih.gov/pubmed/?term=30087052}

{}

{}

\begin{center}\rule{0.5\linewidth}{\linethickness}\end{center}

 - \textbf{Underutilization of Surgery in Periampullary Cancer
Treatment}

\emph{Journal of gastrointestinal surgery : official journal of the
Society for Surgery of the Alimentary Tract 2018 Aug;():}

BACKGROUND: Site-specific outcomes of resection for periampullary cancer
have not been analyzed on a large, registry-based scale. METHODS: We
assessed data on periampullary cancers from the SEER database. Site- and
stage-specific outcomes were analyzed. Resection was compared to no
resection. RESULTS: Resection was the main therapy in stages 1 and 2
(resection vs.~no resection, 8644 vs.~7208 patients), was less frequent
in stage 3 (1248 vs.~2783 patients) and was rarely-but still-used in
stage 4 disease (541 vs.~11,212 patients). Pancreatic head (75.7\%),
11.4\% distal bile duct, 7.7\% ampullary, and 5.3\% duodenal cancers.
Cancer subtype-independent median survival was 22.0 (resection)
vs.~7.0~months (no resection) in stages 1 and 2, 21.0 vs.~8.0~months in
stage 3, and 10.0 vs.~3.0~months in stage 4. Subtype-dependent median
survival (resection vs.~no resection) was 18.0 vs.~5.0~months in
pancreatic head, 19.0 vs 4.0~months in distal bile duct, 41.0 vs
7.0~months in ampullary, and 38.0 vs 4.0~months in duodenal
adenocarcinoma. On multivariable analysis, patient comorbidities,
marital and insurance status, and income all influenced the decision to
undergo resection. CONCLUSIONS: Surgery is still underutilized in the
treatment of periampullary cancers. Patients with cancers originating
from the duodenum or the ampulla of Vater benefit most from resectional
surgery.

PubMed: \url{https://www.ncbi.nlm.nih.gov/pubmed/?term=30088190}

{}

{}

\begin{center}\rule{0.5\linewidth}{\linethickness}\end{center}

 - \textbf{Long-Term Survivors of Pancreatic Cancer: A California
Population-Based Study}

\emph{Pancreas 2018 Sep;47(8):958-966}

OBJECTIVES: Pancreatic cancer continues to carry a poor prognosis with
survival rates that have had minimal improvement over the past 4
decades. We report a population-based, comprehensive analysis of
long-term survivors of pancreatic adenocarcinoma diagnosed in the
diverse population of California. METHODS: Data from the California
Cancer Registry were used to evaluate long-term survival. A total of
70,442 patients diagnosed with pancreatic adenocarcinoma between 1988
and 2009 were identified. Logistic regression was used to identify
factors associated with achieving 5-year survival. RESULTS: The overall
5-year survival was 2.5\%, with minimal incremental improvements
throughout the 3 decades. Age, stage, degree of differentiation, and
surgical resection were associated with 5-year survival. Furthermore,
younger age and receiving care at a National Cancer Institute-designated
cancer center were similarly correlated with 5-year survival regardless
of surgical intervention. In addition, we identified stage,
differentiation, and adjuvant chemotherapy as significant factors for
long-term survival in surgically resected patients. In the unresectable
patients, Asian/Pacific islanders and Hispanics were significantly more
likely to reach the 5-year milestone than non-Hispanic whites.
CONCLUSIONS: Although pancreatic cancer mortality remains high, our
study highlights baseline characteristics, treatment, biological
factors, and ethnicity that are associated with long-term survival.
These findings may serve as a springboard for further investigation.

PubMed: \url{https://www.ncbi.nlm.nih.gov/pubmed/?term=30074526}

{}

{}

\begin{center}\rule{0.5\linewidth}{\linethickness}\end{center}

 - \textbf{Overall survival and cancer-specific survival in patients
with surgically resected pancreatic head adenocarcinoma: A competing
risk nomogram analysis}

\url{https://www.researchgate.net/publication/326829548_Overall_survival_and_cancer-specific_survival_in_patients_with_surgically_resected_pancreatic_head_adenocarcinoma_A_competing_risk_nomogram_analysis}

\begin{center}\rule{0.5\linewidth}{\linethickness}\end{center}

 - \textbf{Race and Health Disparities in Patient Refusal of Surgery for
Early-Stage Pancreatic Cancer: An NCDB Cohort Study}

\emph{Annals of surgical oncology 2018 Nov;25(12):3427-3435}

AIM: To identify factors associated with refusal of surgery in patients
with early-stage pancreatic cancer and estimate the impact of this
decision on survival. METHODS: Using the National Cancer Data Base,
26,358 patients were identified with potentially resectable tumors
(pretreatment clinical stage I: T1 or T2 N0M0). Multivariate models were
employed to identify factors predicting failure to undergo surgery and
assess the impact on survival. RESULTS: Of early-stage patients who were
recommended surgery, 7.8\% (N = 992) refused surgery for resectable
early-stage pancreatic cancer. On multivariable analysis, patients were
more likely to refuse surgery if they were older {[}odds ratio
(OR) = 1.18; 95\% confidence interval (CI) 1.16-1.19{]}, female
(OR = 1.52; 95\% CI 1.33-1.73), African American (vs White, OR = 1.79;
95\% CI 1.37-2.34), on Medicare/Medicaid (vs private, OR = 2.75; 95\% CI
1.54-4.92) or had higher Charlson-Deyo score (2 vs 0, OR = 1.33; 95\% CI
1.03-1.72). Patients were also significantly more likely to refuse
surgery if they were seen at a center that is not an academic/research
program (OR 1.9; 95\% CI 1.6-2.27). Patients who were recommended
surgery but refused had significantly worse survival than those with
stage I who received surgery {[}median survival 6.8 vs 24~months, Cox
hazard ratio (HR) 3.41; 95\% CI 3.12-3.60{]}. CONCLUSIONS: The
percentage of patients refusing surgery for operable early-stage
pancreatic cancer has been decreasing in the last decade but remains a
significant issue that affects survival. Disparities in refusal of
surgery are independently associated with several variables including
gender, race, and insurance. To mitigate national disparities in
surgical care, future studies should focus on exploring potential
reasons for refusal and developing communication interventions.

PubMed: \url{https://www.ncbi.nlm.nih.gov/pubmed/?term=30043318}

{}

{}

\begin{center}\rule{0.5\linewidth}{\linethickness}\end{center}

 - \textbf{Association Between Hepatitis B Infection and Pancreatic
Cancer: A Population-Based Analysis in the United States}

\emph{Pancreas 2018 Aug;47(7):849-855}

OBJECTIVES: The aim of this study was to assess the role of hepatitis B
(HepB) infection in the causation of pancreatic cancer and the
predictors of pancreatic cancer and mortality. METHODS: We identified
pancreatic cancer patients 11 to 70 years of age from the 2013-2014
National Inpatient Sample. Pearson χ test and Student's t-test were used
for categorical and continuous variables, respectively. We assessed the
association of HepB and pancreatic cancer and the independent mortality
predictors by multivariate analyses. RESULTS: Of 69,210 pancreatic
cancer patients, 175 patients with a history of HepB and 69,035 patients
without a history of HepB were identified. Compared with the pancreatic
cancer-non-HepB group, the pancreatic cancer-HepB group consisted more
of younger (mean, 60.4 {[}standard deviation, 7.4{]} years vs 68.2
{[}standard deviation, 12.1{]} years), male, black, and Asian patients
with low household income and nonelective admissions. The odds of
developing pancreatic cancer among the HepB patients were significantly
higher (adjusted odds ratio, 1.24; 95\% confidence interval,
1.056-1.449; P = 0.008). Black race, age ≥ 65 years, and male sex
demonstrated greater odds of mortality. CONCLUSIONS: This study
concluded up to a 24\% increased likelihood of pancreatic cancer among
the HepB patients. Blacks showed greater odds of pancreatic cancer and
related mortality.

PubMed: \url{https://www.ncbi.nlm.nih.gov/pubmed/?term=29939908}

{}

{}

\begin{center}\rule{0.5\linewidth}{\linethickness}\end{center}

 - \textbf{Impact of Prior Malignancy on Survival Outcomes of Stage IV
Pancreatic Adenocarcinoma: SEER-Based Cohort}

\emph{Journal of gastrointestinal cancer 2018 Aug;():}

PURPOSE: Pancreatic cancer is one of the most fatal malignancies and the
fourth leading cause of cancer-related mortality in the USA. Most
clinical trials involving pancreatic adenocarcinoma (PAC) patients
exclude subjects with a prior malignancy because of the possible effect
of prior malignancies on survival. However, no data in the medical
literature support this assumption. In this paper, we aim to study the
impact of having a prior malignancy on the survival outcomes of stage IV
PAC. METHODS: We used the surveillance, epidemiology, and end results
database to review patients with stage IV PAC diagnosed between 1973 and
2014. We calculated overall and pancreatic cancer-specific survival of
these patients using unadjusted Kaplan-Meier test and multivariable
covariate-adjusted Cox models. RESULTS: We reviewed 66,874 stage IV PAC
patients, of which 4942 had a prior malignancy. Kaplan-Meier and Cox
models showed that a history of prior malignancy did not cause
significant difference in overall survival (HR = 0.938,
95\%CI = 0.880-1.000, p = .052). However, a prior malignancy was
associated with a better pancreatic cancer-specific survival
(HR = 0.855, 95\% CI = 0.796-0.918, p \textless{} .001). CONCLUSION: A
prior malignancy before stage IV PAC was not associated with worse
survival outcomes. Researchers should take these results into
consideration when including/excluding patients to improve the
generalizability and accuracy of their results.

PubMed: \url{https://www.ncbi.nlm.nih.gov/pubmed/?term=30105523}

{}

{}

\begin{center}\rule{0.5\linewidth}{\linethickness}\end{center}

\begin{itemize}
\tightlist
\item
  \textbf{Treatment Outcomes in Patients with Metastatic Neuroendocrine
  Tumors: a Retrospective Analysis of a Community Oncology Database}
\end{itemize}

\url{https://link.springer.com/article/10.1007/s12029-018-0160-x}

\begin{center}\rule{0.5\linewidth}{\linethickness}\end{center}

 - \textbf{Genome-scale analysis to identify prognostic microRNA
biomarkers in patients with early stage pancreatic ductal adenocarcinoma
after pancreaticoduodenectomy}

\emph{Cancer management and research 2018 ;10():2537-2551}

Background: The aim of the study was to investigate potential prognostic
microRNA (miRNA) biomarkers for patients with early stage pancreatic
ductal adenocarcinoma (PDAC) after pancreaticoduodenectomy using a
miRNA-sequencing (miRNA-seq) data set from The Cancer Genome Atlas
(TCGA). A miRNA expression-based prognostic signature was generated, and
the potential role of target genes in overall survival (OS) in patients
with PDAC was examined. Methods: A miRNA-seq data set of 112 PDAC
patients who underwent pancreaticoduodenectomy was obtained from TCGA.
Survival analysis was performed to identify potential prognostic
biomarkers. Results: Eleven miRNAs (hsa-mir-501, hsa-mir-4521,
hsa-mir-5091, hsa-mir-24-1, hsa-mir-126, hsa-mir-30e, hsa-mir-3157,
hsa-let-7a-3, hsa-mir-133a-1, hsa-mir-4709, and hsa-mir-421) were used
to construct a prognostic signature using the step function. The
11-miRNA prognostic signature showed good performance for prognosis
prediction (adjusted P\textless{}0.0001, adjusted hazard ratio =4.285,
95\% confidence interval =2.146-8.554), and the time-dependent receiver
operating characteristic analysis showed an area under the curve of
0.864, 0.877, and 0.787 for 1-, 2-, and 3-year PDAC OS predictions,
respectively. Comprehensive survival analysis suggested that the
prognostic signature could serve as an independent prognostic factor for
PDAC OS and performs better in prognosis prediction than other
traditional clinical indicators. Functional assessment of the target
genes of the miRNAs indicated that they were significantly enriched in
multiple biological processes and pathways, including cell
proliferation, cell cycle biological processes, the forkhead box O,
mitogen-activated protein kinase, Janus kinase/signal transducers and
activators of transcription signaling pathways, pathways in cancer, and
the ErbB signaling pathway. Several target genes of these miRNAs were
also associated with PDAC OS. Conclusion: The present study identified a
novel miRNA expression signature that showed potential as a prognostic
biomarker for PDAC after pancreaticoduodenectomy.

PubMed: \url{https://www.ncbi.nlm.nih.gov/pubmed/?term=30127641}

{}

{}

\begin{center}\rule{0.5\linewidth}{\linethickness}\end{center}

 - \textbf{Tumor grade and sex should influence the utilization of
portal lymphadenectomy for early stage intrahepatic cholangiocarcinoma}

\emph{HPB : the official journal of the International Hepato Pancreato
Biliary Association 2018 Aug;():}

BACKGROUND: Portal lymphadenectomy for intrahepatic cholangiocarcinoma
(ICC) is encouraged for staging purposes, though it is under-utilized
for clinically early-stage tumors. We sought to determine if any factor
knowable prior to resection influences rates of portal lymph node
metastases. METHODS: The Surveillance, Epidemiology, and End Results
(SEER) Program (1973-2014) database was queried to identify patients
with T1/T2 ICC undergoing resection. Patients were stratified by lymph
node (LN) status. Patients deemed LN negative required examination of
six or more LNs (AJCC guidelines). RESULTS: One-hundred and fifty-two
patients were included in the analysis (LN negative: 38, LN positive:
114). Patients with LN negative cancers experienced prolonged overall
survival as compared to patients with positive LNs (median 77 months vs
19 months, respectively p~\textless{}~0.001). Twelve patients had
well-differentiated tumors (G1), 92 patients had
moderately-differentiated tumors (G2) and 58 patients had
poorly-differentiated tumors (G3). Tumor grade (OR 3.9, CI 1.1-13.7,
p~=~0.031) and male sex (OR 2.6, CI 1.1-6.1, p~=~0.022) were associated
with positive LNs on multivariable logistic regression analysis.
CONCLUSION: Intermediate/High grade and male sex are associated with
high rates of lymph node metastasis for patients with early-stage ICC,
which portends abbreviated overall survival.

PubMed: \url{https://www.ncbi.nlm.nih.gov/pubmed/?term=30139566}

{}

{}

\begin{center}\rule{0.5\linewidth}{\linethickness}\end{center}

 - \textbf{Progression to pancreatic ductal adenocarcinoma from
pancreatic intraepithelial neoplasia: Results of a simulation model}

\emph{Pancreatology : official journal of the International Association
of Pancreatology (IAP) \ldots{} {[}et al.{]} 2018 Jul;():}

OBJECTIVES: To gain insight into the natural history and carcinogenesis
pathway of Pancreatic Intraepithelial Neoplasia (PanIN) lesions by
building a calibrated simulation model of PanIN progression to
pancreatic ductal adenocarcinoma (PDAC) METHODS: We revised a previously
validated simulation model of solid PDAC, calibrating the model to fit
data from the National Cancer Institute's Surveillance, Epidemiology,
and End Results program and published literature on PanIN prevalence by
age. We estimated the likelihood of progression from PanIN states (1, 2,
and 3) to PDAC and the time between PanIN onset and PDAC (dwell time).
We evaluated a hypothetical intervention to test for and treat PanIN 3
lesions to estimate the potential benefits from PanIN detection.
RESULTS: We estimated the lifetime probability of progressing from PanIN
1 to PDAC to be 1.5\% (men), 1.3\% (women). Progression from PanIN 1 to
PDAC took 33.6 years and 35.3 years, respectively, and from PanIN 3 to
PDAC took 11.3 years and 12.3 years. A hypothetical test for PanIN 3
detection and treatment could provide a maximum, average life expectancy
gain of 40 days. CONCLUSIONS: Our modeling analysis estimates PanINs
have a relatively indolent course to PDAC, supporting the feasibility of
potential future early detection strategies.

PubMed: \url{https://www.ncbi.nlm.nih.gov/pubmed/?term=30143405}

{}

{}

\begin{center}\rule{0.5\linewidth}{\linethickness}\end{center}

\begin{itemize}
\tightlist
\item
  Gallbladder carcinoma: An analysis of the national cancer data base to
  examine hispanic influence
\end{itemize}

\url{https://onlinelibrary.wiley.com/doi/10.1002/jso.25050}

\begin{center}\rule{0.5\linewidth}{\linethickness}\end{center}

 - \textbf{Prognostic value of retrieved lymph node counts in patients
with node-negative perihilar cholangiocarcinomas}

\emph{ANZ journal of surgery 2018 Sep;():}

BACKGROUND: This study aimed to find out the prognostic value and
optimal cut-off value of retrieved lymph node (LN) counts in patients
with node-negative perihilar cholangiocarcinomas. METHODS: The
Surveillance, Epidemiology and End Results (SEER) database was used to
screen out patients with perihilar cholangiocarcinoma. The cut-off
number of retrieved LNs was determined by the X-tile programme.
Kaplan-Meier methods with log-rank tests and Cox regression analysis
were used for survival analysis. RESULTS: A total of 778 patients with
perihilar cholangiocarcinoma (2004-2014) met the inclusion criteria for
this research, and there were 403 patients without LN metastases (N0)
among them. The cut-off numbers of retrieved LNs, which were determined
using the X-tile programme, were 8 and 18. Both results of univariate
and multivariate survival analyses in N0 patients showed that patients
with ≥18 retrieved LNs had a significantly better survival rate than
patients with 1-7 retrieved LNs and patients with 8-17 retrieved LNs. In
the subgroup of patients with early-stage tumours, patients with at
least 13 retrieved LNs had a significantly better overall and
cancer-specific survival than patients with fewer retrieved LNs.
CONCLUSIONS: The retrieved LN counts are an independent prognostic
factor for patients with node-negative perihilar cholangiocarcinoma.
Patients with at least 18 retrieved LNs had a better overall and
cancer-specific survival than patients with fewer retrieved LNs. The
minimum requirement for retrieving of LNs should reach 18 in perihilar
cholangiocarcinoma.

PubMed: \url{https://www.ncbi.nlm.nih.gov/pubmed/?term=30207026}

{}

{}

\begin{center}\rule{0.5\linewidth}{\linethickness}\end{center}

 - \textbf{Pancreaticoduodenectomy and metastasectomy for metastatic
pancreatic neuroendocrine tumors}

\emph{Journal of surgical oncology 2018 Nov;118(6):983-990}

BACKGROUND AND OBJECTIVES: Various treatment options exist for patients
with metastatic pancreatic neuroendocrine tumors (PNETs). Surgical
resection with pancreaticoduodenectomy (PD) typically reserved for
patients with limited disease. Definitive data are lacking to support
either the resection of primary PNET in the metastatic setting or for
surgical debulking of metastatic lesions. METHODS: We conducted an
analysis of the National Cancer Database (NCDB) using the pancreatic
cancer Participant User File. Thirty- and 90-day mortality rates and
survival rates were determined for patients undergoing PD for primary
tumor resection and compared with patients who had no surgery or
metastasectomy. The Kaplan-Meier method was used to compare survival
time. Cox regression models were used to assess factors independently
associated with overall survival time. RESULTS: Resection of the primary
tumor or metastatic disease each significantly improved overall survival
time compared with no resection. Adding metastasectomy to PD resulted in
an incremental increase in overall survival time. Both PD and
metastasectomy are independently associated with overall survival time.
CONCLUSIONS: Our report highlights the potential for survival time
benefit in appropriately selected patients who undergo PD in the setting
of metastatic PNET.

PubMed: \url{https://www.ncbi.nlm.nih.gov/pubmed/?term=30212595}

{}

{}

\begin{center}\rule{0.5\linewidth}{\linethickness}\end{center}

 - \textbf{Overall survival and cancer-specific survival in patients
with surgically resected pancreatic head adenocarcinoma: A competing
risk nomogram analysis}

\emph{Journal of Cancer 2018 ;9(17):3156-3167}

Background: The objective of this study was to estimate probabilities of
overall survival (OS) and cancer-specific survival (CSS) in patients
with pancreatic head adenocarcinoma after surgery. In addition, we
attempted to build nomograms to predict prognosis of these patients.
Methods: Patients diagnosed with surgically resected pancreatic head
adenocarcinoma between 2004 and 2014 were selected for the study from
the Surveillance, Epidemiology, and End Results (SEER) database.
Nomograms were established for estimating 1-, 2- and 3-year OS and CSS
based on Cox regression model and Fine and Grey's model. The performance
of the nomogram was measured by concordance index (C-index) and the area
under receiver operating characteristic (ROC) curve (AUC). Results: A
total of 2374 patients were retrospectively collected from the SEER
database. The discrimination of nomogram for OS prediction was superior
to that of the Tumor-Node-Metastasis (TNM) 7th or 8th edition stage
systems (C-index = 0.640, 95\% CI, 0.618 - 0.662 vs 0.573, 95\% CI,
0.554 - 0.593, P \textless{} 0.001; 0.640, 95\% CI, 0.618 - 0.662 vs
0.596, 95\% CI, 0.586 - 0.607, P \textless{} 0.001, respectively). The
comparisons of values of AUC showed that the established nomograms
displayed better discrimination power than TNM 7th or 8th stage systems
for predicting both OS and CSS. Conclusions: The nomograms which could
predict 1-, 2- and 3-year OS and CSS were established in this study. Our
nomograms showed a relatively good performance and could be served as an
effective tool for prognostic evaluation of patients with pancreatic
head adenocarcinoma after surgery.

PubMed: \url{https://www.ncbi.nlm.nih.gov/pubmed/?term=30210639}

{}

{}

\begin{center}\rule{0.5\linewidth}{\linethickness}\end{center}

\begin{itemize}
\tightlist
\item
  \textbf{A Prognostic Nomogram for Disease-Specific Survival in
  Patients with Pancreatic Ductal Adenocarcinoma of the Head of the
  Pancreas Following Pancreaticoduodenectomy}
\end{itemize}

\url{https://www.medscimonit.com/abstract/index/idArt/909649/act/2}

\begin{center}\rule{0.5\linewidth}{\linethickness}\end{center}

\begin{itemize}
\tightlist
\item
  \textbf{Pancreaticoduodenectomy and metastasectomy for metastatic
  pancreatic neuroendocrine tumors}
\end{itemize}

\url{https://onlinelibrary.wiley.com/doi/10.1002/jso.25219}

\begin{center}\rule{0.5\linewidth}{\linethickness}\end{center}

 - \textbf{Life expectancy in pancreatic neuroendocrine cancer}

\emph{Clinics and research in hepatology and gastroenterology 2018
Sep;():}

BACKGROUND: The prognoses widely reported for pancreatic cancer reflect
the very poor survival associated with the most common histological
type, exocrine adenocarcinoma. We calculated life expectancies for
patients with less common pancreatic neuroendocrine tumors (PNETs), and
also for the subsets of these patients who survive 1 and 5 years
post-diagnosis, all of which carry a significantly better prognosis.
Results for 1- and 5-year PNET survivors appear not to have been
previously reported, nor have life expectancies (average long-term
survival times) been given. METHODS: We identified 5287 cases of PNET in
the SEER US national database, 1973-2013. The Kaplan-Meier estimator was
used to compute empirical survival probabilities and median survival
times for functioning (n = 279) and non-functioning PNET (n = 5008)
cases. The Cox proportional hazards regression model was used to examine
univariate associations of survival with covariates including patient
age, sex, race, cancer stage, tumor grade, surgical treatment, and
calendar year. A multivariate multiplicative hazard Poisson regression
model estimated mortality rates for all combinations of the covariates.
The rates were used to construct actuarial life tables, which gave life
expectancies for male and female patients according to age, cancer
stage, tumor grade, histology (functioning versus non-functioning),
surgical treatment status, and time since diagnosis. These life
expectancies were compared with age- and sex-specific figures from the
US general population. RESULTS: Life expectancy in PNET is lower than
that of the US general population and varies significantly according to
patient age, cancer stage, tumor grade, mode of treatment, and time
since diagnosis. For example, it is near normal for persons aged 70 and
older who undergo surgical resection of localized well-differentiated
(i.e., grade I) tumors. By contrast, persons with metastatic high-grade
tumors not amenable to surgery have life expectancies of only 1 to 4
years depending on patient age. Functioning PNETs were associated with
somewhat lower mortality than non-functioning within the first few years
after diagnosis, though no major differences were observed long-term.
Positive factors for survival were younger age, localized stage, low
tumor grade, and surgical treatment. Survival improved over the
1973-2013 study period: on average mortality rates fell by 1.2\% per
year after controlling for changes in the patient population. Life
expectancy increased markedly with time since diagnosis: those surviving
1 and 5 years post-diagnosis had longer additional life expectancies.
CONCLUSIONS: Life expectancies of patients with PNETs may be markedly
reduced from normal, but even in the worst cases their prognoses remain
significantly better than that of patients with the more common
pancreatic adenocarcinomas. In some very favorable cases, the life
expectancy is near-normal, especially amongst 1- and 5-year survivors.
This information can be used to counsel patients.

PubMed: \url{https://www.ncbi.nlm.nih.gov/pubmed/?term=30220478}

{}

{}

\begin{center}\rule{0.5\linewidth}{\linethickness}\end{center}

\begin{itemize}
\tightlist
\item
  \textbf{A Refined Staging Model for Resectable Pancreatic Ductal
  Adenocarcinoma Incorporating Examined Lymph Nodes, Location of Tumor
  and Positive Lymph Nodes Ratio}
\end{itemize}

\url{https://www.researchgate.net/profile/Chaobin_He/publication/327545050_A_Refined_Staging_Model_for_Resectable_Pancreatic_Ductal_Adenocarcinoma_Incorporating_Examined_Lymph_Nodes_Location_of_Tumor_and_Positive_Lymph_Nodes_Ratio/links/5b952e98299bf14739317b7e/A-Refined-Staging-Model-for-Resectable-Pancreatic-Ductal-Adenocarcinoma-Incorporating-Examined-Lymph-Nodes-Location-of-Tumor-and-Positive-Lymph-Nodes-Ratio.pdf}

\begin{center}\rule{0.5\linewidth}{\linethickness}\end{center}

\begin{itemize}
\tightlist
\item
  \textbf{Life expectancy in pancreatic neuroendocrine cancer}
\end{itemize}

\url{https://www.sciencedirect.com/science/article/pii/S2210740118301517}

\begin{center}\rule{0.5\linewidth}{\linethickness}\end{center}

 - \textbf{Prognostic value of minichromosome maintenance mRNA
expression in early-stage pancreatic ductal adenocarcinoma patients
after pancreaticoduodenectomy}

\emph{Cancer management and research 2018 ;10():3255-3271}

Background: The aim of the current study was to investigate the
potential prognostic value of minichromosome maintenance (MCM) genes in
patients with early-stage pancreatic ductal adenocarcinoma (PDAC) after
pancreaticoduodenectomy by using the RNA-sequencing dataset from The
Cancer Genome Atlas (TCGA). Methods: An RNA-sequencing dataset of 112
early-stage PDAC patients who received a pancreaticoduodenectomy was
obtained from TCGA. Survival analysis was used to identify potential
prognostic values of MCM genes in PDAC overall survival (OS). Results:
Through mining public databases, we observed that MCM genes (MCM2, MCM3,
MCM4, MCM5, MCM6, and MCM7) were upregulated in pancreatic cancer tumor
tissue and have a strong positive coexpression with each other.
Multivariate survival analysis indicated that a high expression of MCM4
significantly increased the risk of death in patients with PDAC, and
time-dependent receiver operating characteristic analysis showed an area
under the curve of 0.655, 0.587, and 0.509 for a 1-, 2-, and 3-year PDAC
OS prediction, respectively. Comprehensive survival analysis of MCM4
using stratified and joint effects survival analysis suggests that MCM4
may be an independent prognostic indicator for PDAC OS. Gene set
enrichment analysis indicated that MCM4 may participate in multiple
biologic processes and pathways, including DNA replication, cell cycle,
tumor protein p53, and Notch signaling pathways, thereby affecting
prognosis of PDAC patients. Conclusions: Our study indicates that MCM2-7
were upregulated in pancreatic cancer tumor tissues, and mRNA expression
of MCM4 may serve as an independent prognostic indicator for PDAC OS
prediction after pancreaticoduodenectomy.

PubMed: \url{https://www.ncbi.nlm.nih.gov/pubmed/?term=30233242}

{}

{}

\begin{center}\rule{0.5\linewidth}{\linethickness}\end{center}

 - \textbf{Subgroup analysis reveals molecular heterogeneity and
provides potential precise treatment for pancreatic cancers}

\emph{OncoTargets and therapy 2018 ;11():5811-5819}

Background: The relationship between molecular heterogeneity and
clinical features of pancreatic cancer remains unclear. In this study,
pancreatic cancer was divided into different subgroups to explore its
specific molecular characteristics and potential therapeutic targets.
Patients and methods: Expression profiling data were downloaded from The
Cancer Genome Atlas database and standardized. Bioinformatics techniques
such as unsupervised hierarchical clustering was used to explore the
optimal molecular subgroups in pancreatic cancer. Clinical pathological
features and pathways in each subgroup were also analyzed to find out
the potential clinical applications and initial promotive mechanisms of
pancreatic cancer. Results: Pancreatic cancer was divided into three
subgroups based on different gene expression features. Patients included
in each subgroup had specific biological features and responded
significantly different to chemotherapy. Conclusion: Three distinct
subgroups of pancreatic cancer were identified, which means that
patients in each subgroup might benefit from targeted individual
management.

PubMed: \url{https://www.ncbi.nlm.nih.gov/pubmed/?term=30254473}

{}

{}

\begin{center}\rule{0.5\linewidth}{\linethickness}\end{center}

\begin{center}\rule{0.5\linewidth}{\linethickness}\end{center}

Back to top

\pagebreak

\hypertarget{journals-reviewed}{%
\subsection{Journals Reviewed}\label{journals-reviewed}}

\begin{center}\rule{0.5\linewidth}{\linethickness}\end{center}

Advances in Anatomic Pathology

American Journal of Clinical Pathology

The American Journal of Gastroenterology

The American Journal of Pathology

American Journal of Surgical Pathology

Annals of Diagnostic Pathology

Annals of Surgery

Annals of Surgical Oncology

Annual Review of Pathology-Mechanisms of Disease

APMIS

Applied Immunohistochemistry \& Molecular Morphology

Archives of Pathology \& Laboratory Medicine

Cancer

Cancer Cell

Cancer Cytopathology

Cell

Cellular Oncology

Clinical Cancer Research

Cochrane Database Systematic Reviews

Cytojournal

Cytopathology

Diagnostic Cytopathology

Diagnostic Pathology

Endocrine Pathology

Experimental and Molecular Pathology

Expert Review of Molecular Diagnostics

Gastroenterology

Gut

Histology and Histopathology

Histopathology

Human Pathology

International Journal of Surgical Pathology

International Journal of Clinical and Experimental Pathology

Journal of Clinical Pathology

Journal of Molecular Diagnostics

Journal of Pathology

Laboratory investigation

Lancet

Medical Molecular Morphology

Modern Pathology

Nature

Nature Reviews Gastroenterology \& Hepatology

NEJM

Pancreas

Pancreatology

Pathobiology

Pathologie Biologie

Pathology

Pathology \& Oncology Research

Pathology International

Pathology Research and Practice

PNAS

Science

Seminars in Diagnostic Pathology

Seminars in Immunopathology

Surgical pathology clinics

Tissue Antigens

Trends in Cancer

Virchows Archiv

\begin{center}\rule{0.5\linewidth}{\linethickness}\end{center}

Back to top

\pagebreak

\hypertarget{feedback}{%
\subsection{Feedback}\label{feedback}}

\begin{center}\rule{0.5\linewidth}{\linethickness}\end{center}

Please send your feedbacks using the form below:

\href{https://docs.google.com/forms/d/e/1FAIpQLSeD3Z9J6Y7eMmiyM12f_SfAmHUlykb1zxZcwO6lg7cebGYQIQ/viewform}{Click
here for the Feedback Form}

\begin{center}\rule{0.5\linewidth}{\linethickness}\end{center}

Back to top

\pagebreak

\hypertarget{archive}{%
\subsection{Archive}\label{archive}}

\begin{center}\rule{0.5\linewidth}{\linethickness}\end{center}

\textbf{The PBPath Journal Archive}

\begin{itemize}
\item
  \href{http://pbpath.org/journal-watch/}{Current Issue}
\item
  \href{http://pbpath.org/journal-watch-archive/}{Older Issues}

  \begin{itemize}
  \item
    \href{http://pbpath.org/pbpath-journal-watch-june-july-2018/}{June-July-2018}
  \item
    October - November issue is being prepared. Unofficial version is
    \href{http://pbpath.org/journal-watch-upcoming-issue/}{here}.
  \end{itemize}
\end{itemize}

\begin{center}\rule{0.5\linewidth}{\linethickness}\end{center}

Back to top


\end{document}
