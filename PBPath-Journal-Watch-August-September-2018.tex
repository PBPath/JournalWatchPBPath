\documentclass[]{article}
\usepackage{lmodern}
\usepackage{amssymb,amsmath}
\usepackage{ifxetex,ifluatex}
\usepackage{fixltx2e} % provides \textsubscript
\ifnum 0\ifxetex 1\fi\ifluatex 1\fi=0 % if pdftex
  \usepackage[T1]{fontenc}
  \usepackage[utf8]{inputenc}
\else % if luatex or xelatex
  \ifxetex
    \usepackage{mathspec}
  \else
    \usepackage{fontspec}
  \fi
  \defaultfontfeatures{Ligatures=TeX,Scale=MatchLowercase}
\fi
% use upquote if available, for straight quotes in verbatim environments
\IfFileExists{upquote.sty}{\usepackage{upquote}}{}
% use microtype if available
\IfFileExists{microtype.sty}{%
\usepackage{microtype}
\UseMicrotypeSet[protrusion]{basicmath} % disable protrusion for tt fonts
}{}
\usepackage[margin=1in]{geometry}
\usepackage{hyperref}
\hypersetup{unicode=true,
            pdftitle={Pancreatobiliary Pathology Society Journal Watch},
            pdfborder={0 0 0},
            breaklinks=true}
\urlstyle{same}  % don't use monospace font for urls
\usepackage{graphicx,grffile}
\makeatletter
\def\maxwidth{\ifdim\Gin@nat@width>\linewidth\linewidth\else\Gin@nat@width\fi}
\def\maxheight{\ifdim\Gin@nat@height>\textheight\textheight\else\Gin@nat@height\fi}
\makeatother
% Scale images if necessary, so that they will not overflow the page
% margins by default, and it is still possible to overwrite the defaults
% using explicit options in \includegraphics[width, height, ...]{}
\setkeys{Gin}{width=\maxwidth,height=\maxheight,keepaspectratio}
\IfFileExists{parskip.sty}{%
\usepackage{parskip}
}{% else
\setlength{\parindent}{0pt}
\setlength{\parskip}{6pt plus 2pt minus 1pt}
}
\setlength{\emergencystretch}{3em}  % prevent overfull lines
\providecommand{\tightlist}{%
  \setlength{\itemsep}{0pt}\setlength{\parskip}{0pt}}
\setcounter{secnumdepth}{0}
% Redefines (sub)paragraphs to behave more like sections
\ifx\paragraph\undefined\else
\let\oldparagraph\paragraph
\renewcommand{\paragraph}[1]{\oldparagraph{#1}\mbox{}}
\fi
\ifx\subparagraph\undefined\else
\let\oldsubparagraph\subparagraph
\renewcommand{\subparagraph}[1]{\oldsubparagraph{#1}\mbox{}}
\fi

%%% Use protect on footnotes to avoid problems with footnotes in titles
\let\rmarkdownfootnote\footnote%
\def\footnote{\protect\rmarkdownfootnote}

%%% Change title format to be more compact
\usepackage{titling}

% Create subtitle command for use in maketitle
\newcommand{\subtitle}[1]{
  \posttitle{
    \begin{center}\large#1\end{center}
    }
}

\setlength{\droptitle}{-2em}

  \title{Pancreatobiliary Pathology Society Journal Watch}
    \pretitle{\vspace{\droptitle}\centering\huge}
  \posttitle{\par}
  \subtitle{August September 2018}
  \author{}
    \preauthor{}\postauthor{}
      \predate{\centering\large\emph}
  \postdate{\par}
    \date{Last Update on 2018-08-24}


\begin{document}
\maketitle

{
\setcounter{tocdepth}{2}
\tableofcontents
}
\href{http://pbpath.org/}{\includegraphics{figures/PBP_header_logo.jpg}}

\begin{center}\rule{0.5\linewidth}{\linethickness}\end{center}

\hypertarget{the-current-pbpath-journal-watch-articles}{%
\section{The Current PBPath Journal Watch
Articles}\label{the-current-pbpath-journal-watch-articles}}

\textbf{Wellcome to our journal watch for pancreatobiliary pathology
articles, which is released every other month.}

\emph{We have created several categories for convenience; however,
articles in each category are in no particular order.}

\emph{Please feel free to fill out our feedback form. You may also
recommend articles to be included.}

\pagebreak

\begin{center}\rule{0.5\linewidth}{\linethickness}\end{center}

\hypertarget{surgical-pathology}{%
\subsection{Surgical Pathology}\label{surgical-pathology}}

\begin{center}\rule{0.5\linewidth}{\linethickness}\end{center}

\hypertarget{pancreas}{%
\subsubsection{Pancreas}\label{pancreas}}

\begin{center}\rule{0.5\linewidth}{\linethickness}\end{center}

 - \textbf{Is an atypical flat lesion (AFL) a precursor lesion of the
pancreatic ductal adenocarcinoma in human?}

\emph{Pathology international 2018 Apr;():}

PubMed: \url{https://www.ncbi.nlm.nih.gov/pubmed/?term=29664180}

{}

{}

\begin{center}\rule{0.5\linewidth}{\linethickness}\end{center}

 - \textbf{Tumor grade as significant prognostic factor in pancreatic
cancer: validation of a novel TNMG staging system}

\emph{Neoplasma 2018 ;65(4):637-643}

Aim of the study was to asses the tumor grade prognostic value in the
Czech pancreatic cancer patients and to evaluate the accuracy of TNMG
prognostic model. Retrospective analysis of 431 pancreatic cancer
patients undergoing pancreatic resection in seven Czech oncological
centers between 2003 and 2013 was performed. The impact of tumor grade
and the accuracy of TNMG prognostic model were evaluated. Lymph node
status, tumor size, tumor stage and grade were proved as statistically
significant survival predictors. The lower tumor differentiation (grade
3 and 4) was associated with poorer prognosis in all stages (stage I: HR
2.23 {[}1.14; 4.36, CI 95\%{]} p=0.019, stage II: HR 3.09 {[}2.01; 4.77,
CI 95\%{]} p=0.001, stage III and IV: HR 3.52 {[}1.73; 7.18, CI 95\%{]}
p=0.001). Kaplan-Meier analysis verified statistically significant
impact of new TNMG stages on survival after resection for pancreatic
cancer (p=0.001). In conclusion, we can state that the tumor grade was
confirmed as statistically significant prognostic factor in pancreatic
cancer. Its incorporation into the current TNM classification enables
more accurate prognosis prediction within particular clinical stages.
That is why an inclusion of the grade to the standard TNM classification
should be discussed.

PubMed: \url{https://www.ncbi.nlm.nih.gov/pubmed/?term=30064236}

{}

{}

\begin{center}\rule{0.5\linewidth}{\linethickness}\end{center}

 - \textbf{Pathologic Evaluation of Surgical Margins in Pancreatic
Cancer Specimens Using Color Coding With Tissue Marking Dyes}

\emph{Pancreas 2018 Aug;47(7):830-836}

OBJECTIVES: Processing of pancreatoduodenectomy specimens is not
standardized; the clinical impact of pathologic surgical margins remains
controversial. We used the color-coding method using tissue-marking dyes
to evaluate margin status of resected specimens to assess its
association with postoperative recurrence. METHODS: We developed a
unified processing approach to assess pancreatoduodenectomy specimens.
Five surgical margins of resected pancreatic specimens were marked with
5 colors. Microscopic resection margin distance (RMD) from margin
closest to the tumor was evaluated for each surgical margin. Forty
patients assessed using nonunified protocols, and 98 patients assessed
using unified protocols were included. RESULTS: The frequency of tumors
with RMD of 1 mm or less in posterior margin was significantly lower and
that in portal vein/superior mesenteric vein margin was significantly
higher in unified protocol group than in nonunified protocol group (P
\textless{} 0.001). In unified protocol group, tumors with RMD of 1 mm
or less correlated with locoregional recurrence (P = 0.025) and
recurrence-free survival (P = 0.030). Multivariate analysis revealed
that tumor size and lymph node metastasis were independent indicators
for disease recurrence. CONCLUSIONS: Resection margin distance of 1 mm
or less was a predictor for disease recurrence, particularly for
locoregional recurrence. Early detection of small-sized tumors without
lymph node metastasis is necessary for improved clinical outcomes in
pancreas cancers.

PubMed: \url{https://www.ncbi.nlm.nih.gov/pubmed/?term=29975353}

{}

{}

\begin{center}\rule{0.5\linewidth}{\linethickness}\end{center}

 - \textbf{High-grade PanIN presenting with localised stricture of the
main pancreatic duct: A clinicopathological and molecular study of 10
cases suggests a clue for the early detection of pancreatic cancer}

\emph{Histopathology 2018 Aug;73(2):247-258}

AIMS: This study aimed to identify the pathological features of
high-grade PanIN that presents with imaging-detectable abnormalities.
METHODS AND RESULTS: Ten cases of isolated, main-duct, high-grade PanIN
as the primary clinical presentation were identified. All patients
presented with stenosis of the main pancreatic duct, with two being
associated with extensive upstream duct dilatation (\textgreater{}5 mm
in diameter). Pancreatic juice cytology suggested adenocarcinoma in all
seven cases examined. In resected specimens, high-grade PanIN was
present chiefly in the main pancreatic duct, with longitudinal extension
ranging between 3 and 40 mm in length (median = 18 mm). In four cases,
in which hypoechoic or hypovascular masses were observed on imaging,
radiopathology correlations suggested that they represented parenchymal
atrophy and subsequent fibrosis around affected ducts, but not invasive
malignancy. On immunohistochemistry, the loss of p16 expression was
found in five (50\%), p53 overexpression in two (20\%) and loss of SMAD4
expression in none (0\%). KRAS mutations were detected in nine cases,
with two dominant clones being found in three by ultrasensitive droplet
digital polymerase chain reaction, suggesting the genetic heterogeneity
of dysplastic cells composing individual lesions. Mutant GNAS was also
observed in one case. CONCLUSIONS: Isolated high-grade PanIN may present
with pancreatic duct stenosis. Therefore, intensive investigations
including pancreatic juice cytology will be required for patients with
unexplained pancreatic duct stenosis. The abnormal expression of p53 and
SMAD4 is infrequent, while GNAS may be mutated in premalignant lesions
mainly affecting the main pancreatic duct, similar to KRAS.

PubMed: \url{https://www.ncbi.nlm.nih.gov/pubmed/?term=29660164}

{}

{}

\begin{center}\rule{0.5\linewidth}{\linethickness}\end{center}

 - \textbf{The ``T'' now Matters: The Eighth Edition of the Union for
International Cancer Control Classification of Pancreatic
Adenocarcinoma}

\emph{Annals of surgery 2018 Aug;268(2):e36-e37}

PubMed: \url{https://www.ncbi.nlm.nih.gov/pubmed/?term=28938271}

{}

{}

\begin{center}\rule{0.5\linewidth}{\linethickness}\end{center}

 - \textbf{Significance of microcystic, elongated, and fragmented
glandular-like features in intraductal papillary mucinous neoplasm of
the pancreas}

\emph{Human pathology 2018 Aug;78():18-27}

Microcystic, elongated, and fragmented (MELF) glandular features are
associated with epithelial-mesenchymal transition, invasion, and
progression in endometrioid adenocarcinoma of the uterus. Similar
histological features are also observed at the periphery of pancreatic
intraductal papillary mucinous neoplasms (IPMNs). However, the
clinicopathological significance of MELF-like features-particularly
whether they represent regenerative or truly neoplastic conditions-in
IPMNs remains unclear. We assessed a total of 152 surgically resected
IPMNs. Fifty cases exhibited MELF-like features, including 26 cases of
IPMNs with accompanying adenocarcinomas and 24 cases of IPMNs without
accompanying adenocarcinomas. MELF-like features were more frequently
observed in IPMN cases with accompanying adenocarcinomas, larger tumors,
main-duct type, and non-gastric histologic subtype. A positive
correlation between the presence of MELF-like features and high-grade
dysplasia was observed in IPMNs without accompanying adenocarcinomas.
Moreover, DPC4 loss and p53 overexpression in MELF-like glands were more
commonly observed in IPMNs with high-grade dysplasia. IPMN patients with
MELF-like features had worse overall and disease-specific survival by
univariate analyses. Our observations suggest that MELF-like features in
some IPMNs with high-grade dysplasia could be related to stromal
invasion. Hence, when MELF-like features are observed in IPMNs,
pathologists should carefully evaluate the results of microscopic
examinations to identify the invasive components; and,
immunohistochemical staining for DPC4 and p53 could help clarify its
clinicopathological significance.

PubMed: \url{https://www.ncbi.nlm.nih.gov/pubmed/?term=29410139}

{}

{}

\begin{center}\rule{0.5\linewidth}{\linethickness}\end{center}

 - \textbf{Immune Cell and Stromal Signature Associated with
Progression-free Survival of Patients with Resected Pancreatic Ductal
Adenocarcinoma}

\emph{Gastroenterology 2018 Aug;():}

BACKGROUND \& AIMS: Changes to the microenvironment of pancreatic ductal
adenocarcinomas (PDACs) have been associated with poor outcomes of
patients. We studied the associations between composition of the
pancreatic stroma (fibrogenic, inert, dormant, or fibrolytic stroma) and
infiltration by inflammatory cells and times of progression-free
survival (PFS) of patients with PDACs after resection. METHODS: We
obtained 1824 tissue microarray specimens from 385 patients included in
the European Study Group for Pancreatic Cancer trial 1 and 3 and
performed immunohistochemistry to detect alpha smooth muscle actin, type
1 collagen, CD3, CD4, CD8, CD68, CD206, and neutrophils. Tumors that
expressed high and low levels of these markers were compared with
patient outcomes using Kaplan-Meier curves and multivariable recursive
partitioning for discrete-time survival tree analysis. Prognostic index
was delineated by a multivariable Cox-proportional-hazards-model of
immune cell and stromal markers and PFS. Findings were validated using
279 tissue microarray specimens from 93 patients in a separate cohort.
RESULTS: Levels of CD3, CD4, CD8, CD68, and CD206 were independently
associated with tumor recurrence. Recursive partitioning for
discrete-time survival tree analysis identified a high level of CD3 as
the strongest independent predictor for longer PFS. Tumors with levels
of CD3 and high levels of CD206 associated with a median PFS time of
16.6 months and a median prognostic index of -0.32 (95\% CI, -0.35 to
-0.31), whereas tumors with low level of CD3 cell and low level of CD8
and high level of CD68 associated with a median PFS time of 7.9 month
and a prognostic index of 0.32 (95\% CI, 0.050-0.32)-we called these
patterns histologic signatures. Stroma composition, when unassociated
with inflammatory cell markers, did not associate significantly with
PFS. In the validation cohort, the histologic signature resulted in an
error matrix accuracy of predicted response of 0.75 (95\% CI, 0.64-0.83;
accuracy P\textless{}.001). CONCLUSIONS: In an analysis of PDAC tissue
microarray specimens, we identified and validated a histologic
signature, based on leukocyte and stromal factors, that associates with
PFS times of patients with resected PDACs. Immune cells might affect the
composition of the pancreatic stroma to affect progression of PDAC.
These findings provide new insights into the immune response to PDAC.

PubMed: \url{https://www.ncbi.nlm.nih.gov/pubmed/?term=30092175}

{}

{}

\begin{center}\rule{0.5\linewidth}{\linethickness}\end{center}

 - \textbf{Tumour origin and R1 rates in pancreatic resections: towards
consilience in pathology reporting}

\emph{Virchows Archiv : an international journal of pathology 2018
Aug;():}

To evaluate differences in the R1 rates of ampullary (AC), pancreatic
(PC), and distal bile duct (DBD) cancers in pancreatoduodenectomies (PD)
using standardised pathology assessment. Data of PD (2010-2011) analysed
in accordance with the Royal College of Pathologists (UK) protocol, were
retrieved. Clinicopathologic features, including frequency, topography,
and mode of margin involvement in AC (n = 87), PC (n = 18), and DBD
(n = 5) cancers were evaluated. The R1 rate was 7\%, 67\%, and 20\% in
the AC, PC, and DBD cancers (p \textless{} 0.001). Within the PC cohort,
R1 rate was heterogeneous (chemo-naïve, 77\%; post-neoadjuvant, 40\%).
Commonest involved margins were as follows: posterior in overall PD
(35\%), AC (43\%), overall PC (33\%), and post-neoadjuvant PC (100\%);
superior mesenteric artery margin in chemo-naïve PC (38\%) and common
bile duct margin in DBD (100\%) cancers. In AC, majority (66\%) of R1
were signet ring cell type. Indirect margin involvement due to tumour
within lymph node, perineural sheath or lymphovascular space was
observed in 26\% cases, and altered R1 rate in AC, PC, and DBD cohorts
by 1\%, 12\%, and 0\%, respectively. Although not statistically
significant, patients with R1 had lower disease-free survival than those
with R0 (mean, 25.4~months versus 44.4~months). Tumour origin impacts R1
data in PD necessitating its accurate classification by pathologists.
Indirect involvement, histology, and neoadjuvant therapy influence the
R1 rate, albeit in a minority of cases. Generating cogent R1 data based
on standardised pathology reporting is the foremost need of the hour.

PubMed: \url{https://www.ncbi.nlm.nih.gov/pubmed/?term=30091124}

\url{https://link.springer.com/article/10.1007/s00428-018-2429-7}

{}

{}

\begin{center}\rule{0.5\linewidth}{\linethickness}\end{center}

 - \textbf{From somatic mutation to early detection: Insights from
molecular characterization of pancreatic cancer precursor lesions}

\emph{The Journal of pathology 2018 Aug;():}

Pancreatic cancer arises from non-invasive precursor lesions, including
pan creatic intraepithelial neoplasia (PanIN), intraductal papillary
mucinous neoplasm (IPMN) and mucinous cystic neoplasm (MCN), which are
curable if detected early enough. Recently, these types of precursor
lesions have been extensively characterized at the molecular level,
defining the timing of critical genetic alterations in tumorigenesis
pathways. The results of these studies deepen our understanding of
tumorigenesis in the pancreas, providing novel insights into tumor
initiation and progression. Perhaps more importantly, they also provide
a rational foundation for early detection approaches that could allow
clinical intervention prior to malignant transformation. In this review,
we summarize the results of comprehensive molecular characterization of
PanINs, IPMNs, and MCNs, and discuss the implications for cancer biology
as well as early detection. This article is protected by copyright. All
rights reserved.

PubMed: \url{https://www.ncbi.nlm.nih.gov/pubmed/?term=30105857}

{}

{}

\begin{center}\rule{0.5\linewidth}{\linethickness}\end{center}

 - \textbf{Multinational validation of the AJCC 8th edition pancreatic
cancer staging system in a pancreas head cancer cohort}

\emph{Journal of hepato-biliary-pancreatic sciences 2018 Aug;():}

BACKGROUND: The aim was to compare the 7th and 8th editions of the AJCC
staging system for pancreas head cancer and to validate the 8th edition
using three multinational tertiary center data. METHODS: Data of 2,864
patients with pancreas head cancer were collected from Korea (571),
Japan (824), and the USA (1,469). Survival analysis was performed to
compare the 7th and 8th editions. Validation was performed by log-rank
tests and test for trend repeated 1,000 times with random sets. RESULTS:
In the 7th edition, 4.1\%, 3.1\%, 18.6\%, 67.5\%, 3.6\%, and 3.1\% were
stage IA, IB, IIA, IIB, III, and IV. In the 8th edition, 8.8\%, 13.9\%,
3.1\%, 38.2\%, 32.9\%, and 3.1\% were stage IA, IB, IIA, IIB, III, and
IV, respectively. The change in T category down-staged 459 patients from
IIA to the new IA and IB. The new N2 category upstaged 856 patients from
the former IIB to III. The 7th edition reversely stratified IA and IB.
The 8th edition corrected this mis-stratification of the 7th edition,
but lacked discriminatory power between IB and IIA (p=0.271). Validation
using the log-rank showed that the 8th edition provided better
discrimination in 6.387 test sets among 10 tests. The test for trend
validated the 8th edition to stratify stages in correct order more often
(7.815/10). CONCLUSION: The 8th edition provides more even distribution
with more powerful discrimination compared to the 7th edition. IRB
REGISTRATION NUMBER BY SEOUL NATIONAL UNIVERSITY: H-1504-062-664: This
article is protected by copyright. All rights reserved.

PubMed: \url{https://www.ncbi.nlm.nih.gov/pubmed/?term=30118171}

{}

{}

\begin{center}\rule{0.5\linewidth}{\linethickness}\end{center}

 - \textbf{Solid-pseudopapillary neoplasms of the pancreas do not
express major pancreatic markers in pediatric patients}

\emph{Human pathology 2018 Aug;():}

Solid pseudopapillary neoplasms of the pancreas (SPN) are classified as
``exocrine'' pancreatic tumors by the World Health Organization.
However, despite numerous studies using immunohistochemistry, electron
microscopy, animal models and molecular biology, the histogenesis of SPN
remains unclear. At the same time, our knowledge of human pancreas
development has significantly increased. It is now well known that the
undifferentiated PDX1+ pancreatic progenitors proliferate and
differentiate into endocrine, ductal, and acinar cells, thanks to the
expression of numerous transcription factors, which can be used to
better characterize pancreatic tumors. In a series of 14 pediatric SPN,
we investigated the expression of four transcription factors associated
with pancreatic development (PDX1, SOX9, PTF1A and NKX2.2) to obtain new
insights into the pathogenesis of SPN. In addition, we tested the
expression of different markers of epithelial, endocrine, exocrine, and
neural differentiation, using both immunohistochemical and
immunofluorescence analyses. All tumors displayed the typical
histological features of SPN, with both pseudopapillary and solid
patterns. The immunoprofile was characterized by immunoreactivity for
ß-catenin (100\%), progesterone receptor (100\%), cyclin D1 (100\%),
synaptophysin (65\%) and S100 (15\%). In all cases, tumor cells were
negative for the following markers: PDX1, SOX9, PTF1A, NKX2.2,
chromogranin A, glucagon, insulin, somatostatin, ghrelin, pancreatic
polypeptide, amylase, GFAP, calretinin, EPCAM and estrogen receptor α.
To conclude, SPN do not express major transcription factors involved in
pancreatic development and differentiation, which does not allow to
precise pancreatic lineage of tumor cells. Thus additional studies are
still required to determine origin of SPN.

PubMed: \url{https://www.ncbi.nlm.nih.gov/pubmed/?term=30130629}

{}

{}

\begin{center}\rule{0.5\linewidth}{\linethickness}\end{center}

 - \textbf{CD200 expression is a feature of solid pseudopapillary
neoplasms of the pancreas}

\emph{Virchows Archiv : an international journal of pathology 2018
Aug;():}

CD200 has been recently indicated as a robust marker of
well-differentiated neuroendocrine neoplasms. Here, we evaluate its role
in differential diagnosis of solid pancreatic neoplasms. We
immunostained for CD200 22 solid pseudopapillary neoplasms (SPNs), 8
acinar carcinomas (ACs), 2 pancreatoblastomas (PBs), 138 neuroendocrine
tumors (PanNETs), and 48 ductal adenocarcinomas. All SPNs showed strong
cytoplasmic and membranous staining for CD200, while only one case of AC
had focal positivity. The two PBs showed focal CD200 positivity, mainly
located in squamoid nests. The vast majority of PanNETs (96\%) showed
strong cytoplasmic and membranous staining for CD200, whereas all PDACs
were negative. As both PanNETs and SPNs express CD200, it has no role in
the differential diagnosis between these two entities.

PubMed: \url{https://www.ncbi.nlm.nih.gov/pubmed/?term=30132130}

{}

{}

\begin{center}\rule{0.5\linewidth}{\linethickness}\end{center}

 - \textbf{``Pancreatic Mucoepidermoid Carcinoma'' is not a Pancreatic
Counterpart of CRTC1/3-MAML2 Fusion Gene-related Mucoepidermoid
Carcinoma of the Salivary Gland, and May More Appropriately be Termed
Pancreatic Adenosquamous Carcinoma With Mucoepidermoid Carcinoma-like
Features}

\emph{The American journal of surgical pathology 2018 Aug;():}

``Mucoepidermoid carcinoma (MEC)'' has been accepted as a synonym for
pancreatic adenosquamous carcinoma (ASC). Pancreatic ASC can show
salivary gland-type MEC-like morphology. CRTC1/3-MAML2 fusion gene is a
characteristic molecular feature of MEC of the salivary gland. We
conducted this study to clarify whether the pancreatic ASC with salivary
gland-type MEC-like morphology (Pan-MEC) is a pancreatic counterpart of
salivary gland-type MEC (Sal-MEC). We retrospectively analyzed 37
pancreatic ASCs including 16 Pan-MECs and 21 tumors without MEC-like
features (ASC-NOS {[}not otherwise specified{]}), and we investigated
(1) clinicopathologic features, (2) the presence of CRTC1/3-MAML2 fusion
gene by reverse transcription polymerase chain reaction, (3) the
presence of rearrangement of MAML2 gene by fluorescence in situ
hybridization, and (4) mucin core proteins by immunohistochemistry. We
also compared 16 Pan-MECs with 20 Sal-MECs by immunohistochemistry for
mucin core protein. There were no significant differences of any
clinicopathologic characteristics and survival analysis between the
Pan-MECs and ASCs-NOS. Of note, the pancreatic ASCs (including Pan-MEC
and ASC-NOS) were significantly more aggressive than conventional
pancreatic ductal adenocarcinoma. In addition, all Pan-MECs were
histologically high-grade. CRTC1/3-MAML2 fusion gene and MAML2 gene
rearrangement were not detected in any ASCs including Pan-MECs. There
were significant differences of MUC5AC and MUC6 between the Pan-MECs and
Sal-MECs, but no significant differences of mucin core protein between
the Pan-MECs and pancreatic ASCs-NOS. Pan-MEC is histologically and
biologically high-grade and unrelated to CRTC1/3-MAML2 fusion gene,
unlike Sal-MEC which is related to CRTC1/3-MAML2 fusion gene. Pan-MEC is
not a pancreatic counterpart of CRTC1/3-MAML2 fusion gene-related
Sal-MEC.

PubMed: \url{https://www.ncbi.nlm.nih.gov/pubmed/?term=30138216}

{}

{}

\begin{center}\rule{0.5\linewidth}{\linethickness}\end{center}

\hypertarget{ipmn}{%
\subsubsection{IPMN}\label{ipmn}}

\begin{center}\rule{0.5\linewidth}{\linethickness}\end{center}

 - \textbf{Does Surgical Margin Impact Recurrence in Noninvasive
Intraductal Papillary Mucinous Neoplasms?: A Multi-institutional Study}

\emph{Annals of surgery 2018 Sep;268(3):469-478}

OBJECTIVE: The relevance of margin positivity on recurrence after
resection of intraductal papillary mucinous neoplasms (IPMNs) is poorly
defined and represents one reason controversy remains regarding optimal
surveillance recommendations. METHODS: Patients undergoing surgery for
noninvasive IPMN at 8 academic medical centers from the Central Pancreas
Consortium were analyzed. A positive margin was defined as presence of
IPMN or pancreatic intraepithelial neoplasia. RESULTS: Five hundred two
patients underwent surgery for IPMN; 330 (66\%) did not have invasive
cancer on final pathology and form the study cohort. Of these, 20\%
harbored high grade dysplasia. A positive margin was found in 20\% of
cases and was associated with multifocal disease (P = 0.02). The
majority of positive margins were associated with low grade dysplasia.
At a median follow-up of 36 months, 34 (10.3\%) patients recurred, with
6.7\% developing recurrent cystic disease and 3.6\% developing invasive
cancer. On multivariate analysis, margin positivity was not associated
with recurrence of either IPMN or invasive cancer (P \textgreater{}
0.05). No association between margin status and development of
recurrence at the margin was found. Only 6\% of recurrences developed at
the resection margin and median time to recurrence was 22 months. Of
note, 18\% of recurrences occurred \textgreater{} 5 years following
surgery. CONCLUSION: Margin positivity after resection for noninvasive
IPMNs is primarily due to low grade dysplasia and is not associated with
developing recurrence in the remnant pancreas or at the resection
margin. Long-term surveillance is required for all patients, as a
significant number of recurrences developed over 5 years after the index
operation.

PubMed: \url{https://www.ncbi.nlm.nih.gov/pubmed/?term=30063495}

\url{https://journals.lww.com/annalsofsurgery/Abstract/2018/09000/Does_Surgical_Margin_Impact_Recurrence_in.10.aspx}

{}

{}

\begin{center}\rule{0.5\linewidth}{\linethickness}\end{center}

\begin{itemize}
\tightlist
\item
  \textbf{Transmembrane mucin MUC13 distinguishes intraductal papillary
  mucinous neoplasms from non-mucinous cysts and is associated with
  high-risk lesions}
\end{itemize}

\url{https://www.sciencedirect.com/science/article/pii/S1365182X18326947}

\begin{center}\rule{0.5\linewidth}{\linethickness}\end{center}

\begin{itemize}
\tightlist
\item
  \textbf{Importance of main pancreatic duct dilatation in IPMN
  undergoing surveillance}
\end{itemize}

\url{https://onlinelibrary.wiley.com/doi/abs/10.1002/bjs.10948}

\begin{center}\rule{0.5\linewidth}{\linethickness}\end{center}

 - \textbf{Comparison of the Survival Outcomes of Pancreatic Cancer and
Intraductal Papillary Mucinous Neoplasms}

\emph{Pancreas 2018 Sep;47(8):974-979}

OBJECTIVES: The aims of the study were to compare survival outcomes
between patients with pancreatic ductal adenocarcinoma (PDAC) and
invasive intraductal papillary mucinous neoplasms (IPMN) and to
determine candidates for adjuvant chemotherapy. METHODS: A total of 579
consecutive patients, including 375 PDAC and 204 IPMN patients, were
reviewed. Stage-matched comparisons of survival data were conducted
using the Cox proportional hazards model and propensity analysis. To
evaluate prognostic factors, univariate and multivariate Cox regression
analyses were performed. RESULTS: The overall survival for invasive IPMN
was significantly longer than that for PDAC (hazard ratio, 2.34; P =
0.0001). When the analysis was limited to stage I patients, the 5-year
overall survival rate of invasive IPMN patients was significantly better
than that of PDAC patients (100\% vs 74.1\%, P = 0.0092); however, no
difference was observed between stage II patients with invasive IPMN and
PDAC (hazard ratio, 1.49; P = 0.09). The Cox proportional hazards model
and propensity analysis demonstrated no difference in stage-matched
survival. Multivariate analysis revealed that only T (≥3) was an
independent prognostic factor for invasive IPMN. CONCLUSIONS:
Stage-matched analysis did not show a significant survival difference
between invasive IPMN and PDAC patients, and T3 or higher was an
independent prognostic factor for invasive IPMN.

PubMed: \url{https://www.ncbi.nlm.nih.gov/pubmed/?term=30028445}

{}

{}

\begin{center}\rule{0.5\linewidth}{\linethickness}\end{center}

\hypertarget{biliary-tract}{%
\subsubsection{Biliary Tract}\label{biliary-tract}}

\begin{center}\rule{0.5\linewidth}{\linethickness}\end{center}

\begin{itemize}
\tightlist
\item
  \textbf{Data set for the reporting of intrahepatic cholangiocarcinoma,
  perihilar cholangiocarcinoma and hepatocellular carcinoma:
  recommendations from the International Collaboration on Cancer
  Reporting (ICCR)}
\end{itemize}

\url{https://onlinelibrary.wiley.com/doi/abs/10.1111/his.13520?campaign=woletoc}

\begin{center}\rule{0.5\linewidth}{\linethickness}\end{center}

 - \textbf{Interleukin-33 overexpression reflects less aggressive tumour
features in large-duct type cholangiocarcinomas}

\emph{Histopathology 2018 Aug;73(2):259-272}

AIMS: The aim of the present study was to elucidate the
clinicopathological significance of interleukin (IL)-6 and IL-33
expression in intrahepatic cholangiocarcinomas (iCCAs) and perihilar
cholangiocarcinomas (pCCAs). METHODS AND RESULTS: IL-6 and IL-33 mRNA
expression levels were examined in iCCAs (n = 55) and pCCAs (n = 32) by
the use of quantitative real-time polymerase chain reaction and a highly
sensitive in-situ hybridisation protocol (RNAscope), and expression
levels were correlated with clinicopathological features. According to a
recently proposed classification scheme, iCCAs were separated into
small-duct (n = 33) and large-duct (n = 22) types. IL-6 and IL-33
expression levels were higher in large-duct iCCAs and pCCAs than in
small-duct iCCAs, and there was a positive correlation between the
expression levels of these cytokines. Double in-situ
hybridisation/immunostaining showed that IL-6 mRNA was expressed in
actin-positive (myo)fibroblasts, whereas IL-33 mRNA was mainly produced
by CD31-positive endothelial cells. With the average expression level as
a cut-off point, cases were classified as IL-6high and IL-6low or
IL-33high and IL-33low . In the combined cohort of large-duct iCCAs and
pCCAs, IL-6high and IL-6low cholangiocarcinomas shared many features,
whereas IL-33high cases had less aggressive characteristics than
IL-33low cases, as shown by lower tumour marker concentrations, smaller
tumour sizes, less common vascular invasion, lower pT stages, and higher
lymphocyte/monocyte ratios in blood. KRAS mutations were slightly less
common in IL-33high cases than in IL-33low cases (9\% versus 29\%; P =
0.061). The strong expression of IL-33 in tissue appeared to be an
independent favourable prognostic factor. CONCLUSIONS: IL-33high
cholangiocarcinomas may represent a unique, less aggressive
carcinogenetic process of the large bile ducts.

PubMed: \url{https://www.ncbi.nlm.nih.gov/pubmed/?term=29675965}

doi: \url{https://doi.org/10.1111/his.13633}

{}

{}

\begin{center}\rule{0.5\linewidth}{\linethickness}\end{center}

 - \textbf{Postradiation Synovial Sarcoma of the Common Bile Duct: A
Previously Unreported Anatomic Site} \emph{International journal of
surgical pathology 2018 Aug;26(5):469-474}

Synovial sarcoma is a ubiquitous neoplasm predominantly affecting soft
tissues of young adults of any gender; few cases have been described in
the digestive system, mostly in the stomach. The (X;18)(p11.2; q11.2)
translocation yields unique SS18-SSX fusion genes. Synovial sarcoma has
been related to radiotherapy, but no synovial sarcoma has been
associated with the digestive system. This article describes the case of
a synovial sarcoma arising along the extrahepatic biliary tree, 10 years
after the application of an abdominal radiotherapy schedule due to a
retroperitoneal metastatic seminoma in a male who developed progressive
obstructive jaundice. Ninety percent of the analyzed cells carried the
SS18 gene with separation of sequences, thus denoting a translocation.
There are only 8 post-radiotherapy synovial sarcomas that have been
reported previously, and this is the first report of a
radiotherapy-related synovial sarcoma arising from the extrahepatic
biliary tree, and the second case described in this anatomic region.

PubMed: \url{https://www.ncbi.nlm.nih.gov/pubmed/?term=29336183}

doi: \url{https://doi.org/10.1177/1066896917752863}

{}

{}

\begin{center}\rule{0.5\linewidth}{\linethickness}\end{center}

 - \textbf{Mucinous intrahepatic cholangiocarcinoma: a distinct variant}

\emph{Human pathology 2018 Aug;78():131-137}

Mucinous variant of intrahepatic cholangiocarcinoma (iCC) is rare, and
its clinicopathological features and prognosis are far less clear. Six
patients who had iCCs with more than 50\% of mucinous component and 79
conventional iCCs were included in the study. The mean size of mucinous
and conventional iCCs was 6.2 and 6.0\,cm, respectively. Most patients
(83\%) with mucinous iCC presented at T3 stage or above compared with
28\% of the conventional group (P\,\textless{}\,.01). Three patients
with mucinous iCC (50\%) died within 1\,year. The average survival time
of patients with mucinous iCCs was significantly reduced compared with
that of the conventional group (9\,months versus 2\,years;
P\,\textless{}\,.001). Immunohistochemistry was performed on 6 mucinous
and 12 conventional iCCs with matched age, sex, and stage, which
revealed positive immunoreactivity in MUC1 (83\% versus 58\%), MUC2
(33\% versus 17\%), MUC5AC (100\% versus 42\%), MUC6 (50\% versus 0),
CK7 (83\% versus 83\%), CK20 (0 versus 17\%), CDX2 (17\% versus 0), p53
(67\% versus 67\%), Smad4 (67\% versus 58\%), and EGFR (83\% versus
42\%) in mucinous and conventional iCCs, respectively. Molecular studies
showed one mucinous iCC with KRAS G12C mutation and no BRAF or IDH1/2
mutations. Mucinous iCC is a unique variant that constitutes 7\% of
iCCs. It is more immunoreactive for MUC1, MUC2, MUC5AC, and MUC6. Unlike
adenocarcinomas of colorectal primary, mucinous iCCs are often
CK7+/CK20-/CDX2- and microsatellite stable. Patients with mucinous iCC
likely present at advanced stage upon diagnosis with shorter survival
time compared with the conventional counterparts.

PubMed: \url{https://www.ncbi.nlm.nih.gov/pubmed/?term=29698701}

{}

{}

\begin{center}\rule{0.5\linewidth}{\linethickness}\end{center}

\hypertarget{gallbladder}{%
\subsubsection{Gallbladder}\label{gallbladder}}

\begin{center}\rule{0.5\linewidth}{\linethickness}\end{center}

 - \textbf{Validation of American Joint Committee on Cancer eighth
staging system for gallbladder cancer and its lymphadenectomy
guidelines}

\emph{The Journal of surgical research 2018 Oct;230():148-154}

BACKGROUND: For gallbladder cancer (GBC), the American Joint Committee
on Cancer eighth edition (AJCC 8) staging system classifies lymph node
(LN) stage by the number of positive LN and recommends sampling of ≥6
LNs. We evaluated the prognostic capability of the AJCC 8 for patients
undergoing resection and the current national trends in LN staging in
the context of these new recommendations for nodal (N) sampling.
METHODS: Utilizing the National Cancer Data Base, we identified all
gallbladder adenocarcinoma patients treated with surgical resection in
2004-2014. Cox regression modeling was used to calculate the concordance
index of AJCC 8 in predicting overall survival. N sampling and
positivity rates were analyzed over the study period. RESULTS: In our
cohort, predicted 5-year overall survival by AJCC 8 was: stage I,
62.5\%; II, 50.2\%; IIIA, 25.7\%; IIIB, 22.1\%; IVA, 15.7\%; IVB, 6.7\%
(P~\textless{}~0.01). The concordance index for the staging system was
0.832. Only 50.7\% of the patients had any LN sampling to determine the
N stage. LN sampling rates improved from 45.6\% in 2004 to 55.1\% in
2013 (P~\textless{}~0.001). However, only 24.5\% of patients with any LN
sampling had ≥6 LNs resected (12.4\% of eligible cohort), with a median
LN sample of two. CONCLUSIONS: AJCC 8 offers adequate discrimination for
GBC staging, especially for node-positive patients. With actual GBC LN
sampling rates at 50.7\%, and far short of the ≥6 LN threshold, quality
improvement measures may need to focus on requiring any LN sampling
before raising the minimum to six LNs.

PubMed: \url{https://www.ncbi.nlm.nih.gov/pubmed/?term=30100032}

{}

{}

\begin{center}\rule{0.5\linewidth}{\linethickness}\end{center}

\begin{itemize}
\tightlist
\item
  \textbf{Clinicopathological features and survival of gallbladder
  squamous cell carcinoma: analysis of 121 cases}
\end{itemize}

\url{http://www.ijcep.com/files/ijcep0076184.pdf}

\begin{center}\rule{0.5\linewidth}{\linethickness}\end{center}

 - \textbf{Pyloric Gland Adenoma (PGA) of the Gallbladder: A Unique and
Distinct Tumor from PGAs of the Stomach, Duodenum, and Pancreas}

\emph{The American journal of surgical pathology 2018
Sep;42(9):1237-1245}

Twenty-four surgically resected, gallbladder pyloric gland adenomas
(GB-PGAs) were examined and their features were compared with the
reported features of stomach, duodenum, and pancreatic PGAs to better
understand GB-PGAs. Clinical information on background gallbladder
lesions and histologic data, including tumor grade, existence of
squamoid morules, intratumoral cholesterosis, and intracytoplasmic
mucins were collected. Immunohistochemical staining for MUC2, MUC5AC,
MUC6, CDX2, pepsinogen I, p53, and MIB-1/nuclear β-catenin were
evaluated. Targeted mutational analyses of KRAS exon2, GNAS exon 7, and
CTNNB1 exon 3 were conducted. We found that 29.2\% of the GB-PGAs were
histologically high-grade dysplasias/carcinomas; 70.8\% were low grade;
and 20.8\% and 33.3\% contained squamoid morules and intratumoral
cholesterosis, respectively. In addition, 45.8\% and 54.2\% of GB-PGAs
were mucin-rich and mucin-poor types, respectively.
Immunohistochemically, MUC6 was diffusely positive in all GB-PGAs; MUC2,
MUC5AC, and CDX2 were only focally positive, and no pepsinogen-I
positive cells were observed. Nuclear β-catenin accumulation was
observed in all cases; however, the ratio varied among cases. Mucin-poor
types were significantly associated with high histologic grade
dysplasias/carcinomas and high nuclear β-catenin labeling indices.
Mutational analyses identified CTNNB1 mutations in 100\% of GB-PGAs
(21/21), KRAS in 4.2\% (1/23), and GNAS in 0\% (0/22). The present study
clarified the unique histologic features, phenotypic differentiation,
and molecular statuses frequently associated with GB-PGAs. Altogether,
our data suggest that tumorigenesis of GB-PGA is distinct from that of
stomach, duodenum, and pancreatic PGAs.

PubMed: \url{https://www.ncbi.nlm.nih.gov/pubmed/?term=29975247}

\url{https://journals.lww.com/ajsp/Abstract/2018/09000/Pyloric_Gland_Adenoma__PGA__of_the_Gallbladder__A.12.aspx}

{}

{}

\begin{center}\rule{0.5\linewidth}{\linethickness}\end{center}

\hypertarget{ampulla-of-vater}{%
\subsubsection{Ampulla of Vater}\label{ampulla-of-vater}}

\begin{center}\rule{0.5\linewidth}{\linethickness}\end{center}

 - \textbf{Intraobserver and Interobserver Variability in the Assessment
of Dysplasia in Ampullary Mucosal Biopsies} \emph{The American journal
of surgical pathology 2018 Aug;42(8):1095-1100}

Endoscopic mucosal biopsies of the ampulla of Vater (AmpBx) are obtained
to histologically assess for dysplasia or carcinoma. However, biopsy
material is often scant and a host of factors can induce histologic
changes that pose diagnostic challenges. We sought to investigate
observer variability in interpretation of AmpBx and the impact clinical
data may have on diagnostic interpretation. Thirty-one cases from
institutional archives were selected, including 12 cases of reactive
atypia (RA), 8 indefinite for dysplasia (ID), and 11 showing low-grade
dysplasia (LGD). Slides were independently reviewed at 3 time points
with and without clinical information by 6 pathologists who categorized
the biopsies RA, ID, or LGD. Following the reviews, intraobserver and
interobserver agreement was assessed. Review of AmpBx without clinical
data showed fair (κ, 0.27), poor (κ, 0.07), and good (κ, 0.42)
interobserver agreement for diagnoses of RA, ID, and LGD, respectively.
Interobserver agreement improved for LGD (κ, 0.66 and 0.73) when
clinical information was provided; however, agreement remained fair for
RA (κ, 0.4 and 0.42) and poor-to-fair for ID (κ, 0.17 and 0.25). When
follow-up data were reviewed, all cases that reached unanimous agreement
had that diagnosis substantiated by subsequent endoscopic or histologic
findings. The same was true of 13 of 19 cases that reached majority
consensus. Given the potential clinical consequences of these diagnoses
combined with the significant intraobserver and interobserver
variability found in this study, we conclude that better-defined
diagnostic criteria and consensus reads on difficult cases would assist
in the histologic assessment of these challenging cases.

PubMed: \url{https://www.ncbi.nlm.nih.gov/pubmed/?term=29738360}

doi: \url{https://doi.org/10.1097/PAS.0000000000001079}

{}

{}

\begin{center}\rule{0.5\linewidth}{\linethickness}\end{center}

\hypertarget{pannet-neuroendocrine}{%
\subsubsection{PanNET Neuroendocrine}\label{pannet-neuroendocrine}}

\begin{center}\rule{0.5\linewidth}{\linethickness}\end{center}

\begin{itemize}
\tightlist
\item
  \textbf{ATRX loss is an independent predictor of poor survival in
  pancreatic neuroendocrine tumours}
\end{itemize}

\url{https://www.sciencedirect.com/science/article/pii/S0046817718303009}

\begin{center}\rule{0.5\linewidth}{\linethickness}\end{center}

 - \textbf{ATRX loss is an independent predictor of poor survival in
pancreatic neuroendocrine tumours}

\emph{Human pathology 2018 Aug;():}

Pancreatic neuroendocrine tumours (PanNETs) are rare neoplasms
accounting for 1-2\% of all pancreatic tumours. The biological behaviour
of PanNETs is heterogeneous and unpredictable, adding to the
difficulties of clinical management. The DAXX (death domain associated
protein) and ATRX (alpha-thalassemia/mental retardation syndrome
X-linked) genes encode proteins involved in SWI/SNF-like chromatin
remodelling. Somatic inactivating mutations in DAXX and ATRX are
frequent in PanNETs, mutually exclusive, and associated with telomere
dysfunction resulting in genomic instability and alternate lengthening
of telomeres. We sought to assess the clinical significance of the loss
of the ATRX and DAXX proteins as determined by immunohistochemistry
(IHC) in patients with PanNET. From an unselected cohort of 105
patients, we found ATRX loss in 10 tumours (9.5\%) and DAXX loss in 16
(15.2\%). DAXX and ATRX loss were confirmed mutually exclusive and
associated with other adverse clinicopathological variables and poor
survival in univariate analysis. In addition ATRX loss was also
associated with higher AJCC stage and infiltrative tumour borders.
However only ATRX loss, lymphovascular invasion and perineural spread
were independent predictors of poor overall survival in multivariate
analysis. In conclusion, loss of expression of ATRX as determined by IHC
is a useful independent predictor of poor overall survival in PanNETs.
Given its relative availability, ATRX loss as determined by IHC may have
a role in routine clinical practice to refine prognostication in
patients with PanNET.

PubMed: \url{https://www.ncbi.nlm.nih.gov/pubmed/?term=30081149}

{}

{}

\begin{center}\rule{0.5\linewidth}{\linethickness}\end{center}

 - \textbf{Recurrence of Pancreatic Neuroendocrine Tumors and Survival
Predicted by Ki67}

\emph{Annals of surgical oncology 2018 Aug;25(8):2467-2474}

BACKGROUND: Despite evidence of different malignant potentials,
postoperative follow-up assessment is similar for G1 and G2 pancreatic
neuroendocrine tumors (panNETs) and adjuvant treatment currently is not
indicated. This study investigated the role of Ki67 with regard to
recurrence and survival after curative resection of panNET. METHODS:
Patients with resected non-functioning panNET diagnosed between 1992 and
2016 from three institutions were retrospectively analyzed. Patients who
had G1 or G2 tumor without distant metastases or hereditary syndromes
were included in the study. The patients were re-categorized into Ki67
0-5 and Ki67 6-20\%. Cox regression analysis with log-rank testing for
recurrence and survival was performed. RESULTS: The study enrolled 241
patients (86\%) with Ki67 0-5\% and 39 patients (14\%) with Ki67 6-20\%.
Recurrence was seen in 34 patients (14\%) with Ki67 0-5\% after a median
period of 34~months and in 16 patients (41\%) with Ki67 6-20\% after a
median period of 16~months (p \textless{} 0.001). The 5-year
recurrence-free and 10-year disease-specific survival periods were
respectively 90 and 91\% for Ki67 0-5\% and respectively 55 and 26\% for
Ki67 6-20\% (p \textless{} 0.001). The overall survival period after
recurrence was 44.9~months, which was comparable between the two groups
(p = 0.283). In addition to a Ki67 rate higher than 5\%, tumor larger
than 4~cm and lymph node metastases were independently associated with
recurrence. CONCLUSIONS: Patients at high risk for recurrence after
curative resection of G1 or G2 panNET can be identified by a Ki67 rate
higher than 5\%. These patients should be more closely monitored
postoperatively to detect recurrence early and might benefit from
adjuvant treatment. A clear postoperative follow-up regimen is proposed.

PubMed: \url{https://www.ncbi.nlm.nih.gov/pubmed/?term=29789972}

{}

{}

\begin{center}\rule{0.5\linewidth}{\linethickness}\end{center}

 - \textbf{Neuroendocrine tumor of the pancreas with rhabdoid feature}

\emph{Virchows Archiv : an international journal of pathology 2018
Jun;():}

Imaging of a 53-year-old Japanese man revealed two tumors in the liver
and a tumor in the head of the pancreas with a swelling lymph node. A
needle biopsy for the liver tumors was performed, revealing a
neuroendocrine tumor. Enucleation, lymphadenectomy, and partial
hepatectomy were performed. The microscopic examination identified many
tumor cells with intracytoplasmic inclusions arranged in a nested, cord,
or tubular fashion. The intracytoplasmic inclusions displayed densely
eosinophilic globules and displaced the nuclei toward the periphery,
which constitutes ``rhabdoid'' features. The tumor cells were positive
for synaptophysin and weakly positive for NCAM, but negative for
chromogranin A. Epithelial markers (AE1/AE3 and CAM5.2) accentuated
intracytoplasmic globules. Pancreatic neuroendocrine tumors with
rhabdoid features are very rare. Generally, rhabdoid features are
aggressive and dedifferentiated characteristics of various types of
tumor. Pancreatic neuroendocrine tumors containing rhabdoid cells tend
to display extrapancreatic spread at the time of presentation, although
some of these tumors with rhabdoid features are not always associated
with aggressive behavior.

PubMed: \url{https://www.ncbi.nlm.nih.gov/pubmed/?term=29938394}

{}

{}

\begin{center}\rule{0.5\linewidth}{\linethickness}\end{center}

 - \textbf{Prognostic Significance of Preoperative
Neutrophil-to-Lymphocyte Ratio in Surgically Resectable Pancreatic
Neuroendocrine Tumors}

\emph{Medical science monitor : international medical journal of
experimental and clinical research 2017 Nov;23():5574-5588}

BACKGROUND The aim of this study was to evaluate the predictive and
prognostic value of the preoperative neutrophil-to-lymphocyte ratio
(NLR) and platelet-to-lymphocyte ratio (PLR) in pancreatic
neuroendocrine tumor (PNET) patients undergoing potentially curative
resection. MATERIAL AND METHODS A retrospective review of 172 patients
with PNETs was conducted. Kaplan-Meier curves and multivariate Cox
proportional models were used to calculate overall survival (OS) and
disease-free survival (DFS). The predictive performance of the NLR was
compared with other inflammation-based scores and conventional
stratification systems using receiver operating characteristic (ROC)
curve analysis. RESULTS Elevated NLR and PLR were both associated with
advanced AJCC stage and high grade. In the univariate analysis, elevated
NLR and PLR were both significantly associated with decreased OS and
DFS. In the multivariate analysis, the preoperative NLR, but not the
PLR, was an independent risk factor for OS (HR=4.471, 95\% CI
1.531-13.054, p=0.006) and DFS (HR=2.531, 95\% CI 1.202-5.329, p=0.015).
The discriminatory capability of the NLR was superior to that of other
inflammation-based scores in OS prediction. Furthermore, the predictive
range was expanded by incorporating the NLR into the conventional
stratification systems, including the AJCC stage and WHO classification
systems. CONCLUSIONS As an independent prognostic factor, an elevated
preoperative NLR is superior to the PLR with respect to predicting
clinical outcomes in PNET patients undergoing potentially curative
resection. The incorporation of the NLR into the existing conventional
stratification systems improved the predictive accuracy.

PubMed: \url{https://www.ncbi.nlm.nih.gov/pubmed/?term=29168979}

{}

{}

\begin{center}\rule{0.5\linewidth}{\linethickness}\end{center}

 - \textbf{Variability of the Ki-67 proliferation index in
gastroenteropancreatic neuroendocrine neoplasms - a single-center
retrospective study}
\url{https://link.springer.com/article/10.1186/s12902-018-0274-y}

\begin{center}\rule{0.5\linewidth}{\linethickness}\end{center}

 - \textbf{APOBEC3B High Expression in Gastroenteropancreatic
Neuroendocrine Neoplasms and Association With Lymph Metastasis}

\emph{Applied immunohistochemistry \& molecular morphology : AIMM 2018
Aug;():}

PURPOSE: Apolipoprotein B mRNA editing enzyme catalytic polypeptide-like
3B (APOBEC3B) is known as a source of mutations in multiple cancers.
Gastroenteropancreatic neuroendocrine neoplasms (GEP-NENs) are a group
of heterogeneous tumors. However, the expression and significance of
APOBEC3B in GEP-NENs remains unclear. MATERIALS AND METHODS: A total of
158 cases of GEP-NENs, including 78 cases of biopsy or endoscopic
submucosal dissection resection specimens and 83 cases of surgical
resection specimens were collected in this study. The cases were grouped
according to tumor classification grade, including 42 cases of
neuroendocrine tumors G1 (NET G1), 36 cases of NET G2, 36 cases of NET
G3, 44 cases of neuroendocrine carcinoma (NEC). All of the 158 tumors
were immunohistochemically studied using a polyclonal antibody against
APOBEC3B. We evaluated APOBEC3B expression in GEP-NENs and investigated
the relationships among the immunoreactivity of APOBEC3B, clinical and
pathologic features, such as age, sex, tumor site, Ki67 cell
proliferation index, and lymph metastasis. RESULTS: A total of 33 cases
(78.6\%) of NET G1 showed high expression of APOBEC3B. A total of 28
cases (77.8\%) of NET G2 demonstrated high expression of APOBEC3B. In
NET G3 and NEC cases, the positive rates were 52.8\% and 2.3\%,
respectively. The expression of APOBEC3B in NETs was significantly
higher than that in NECs, NET G1 and NET G2 were higher than NET G3, and
the difference was statistically significant. APOBEC3B high expression
cases have lower lymph node metastasis rate, lower Ki67 cell
proliferation index. CONCLUSIONS: In this study, APOBEC3B is highly
expressed in GEP-NETs and is a predictor of lymph node metastasis in NET
G3 and NEC cases. These findings might provide new insights into the
biological mechanisms of GEP-NENs tumorigenesis and progression.

PubMed: \url{https://www.ncbi.nlm.nih.gov/pubmed/?term=30095460}

{}

{}

\begin{center}\rule{0.5\linewidth}{\linethickness}\end{center}

Back to top

\pagebreak

\hypertarget{cytopathology}{%
\subsection{Cytopathology}\label{cytopathology}}

\begin{center}\rule{0.5\linewidth}{\linethickness}\end{center}

\hypertarget{pancreas-1}{%
\subsubsection{Pancreas}\label{pancreas-1}}

\begin{center}\rule{0.5\linewidth}{\linethickness}\end{center}

 - \textbf{Acute Pancreatitis History Carries Higher Risk in Endoscopic
Ultrasound-Guided Fine-Needle Aspiration of Pancreatic Lesions}

\emph{Pancreas 2018 Aug;47(7):e38-e40}

PubMed: \url{https://www.ncbi.nlm.nih.gov/pubmed/?term=29985847}

{}

{}

\begin{center}\rule{0.5\linewidth}{\linethickness}\end{center}

\hypertarget{biliary-tract-1}{%
\subsubsection{Biliary Tract}\label{biliary-tract-1}}

\begin{center}\rule{0.5\linewidth}{\linethickness}\end{center}

\hypertarget{gallbladder-1}{%
\subsubsection{Gallbladder}\label{gallbladder-1}}

\begin{center}\rule{0.5\linewidth}{\linethickness}\end{center}

\hypertarget{ampulla-of-vater-1}{%
\subsubsection{Ampulla of Vater}\label{ampulla-of-vater-1}}

\begin{center}\rule{0.5\linewidth}{\linethickness}\end{center}

\hypertarget{pannet-neuroendocrine-1}{%
\subsubsection{PanNET Neuroendocrine}\label{pannet-neuroendocrine-1}}

\begin{center}\rule{0.5\linewidth}{\linethickness}\end{center}

Back to top

\pagebreak

\hypertarget{molecular-pathology}{%
\subsection{Molecular Pathology}\label{molecular-pathology}}

\begin{center}\rule{0.5\linewidth}{\linethickness}\end{center}

\hypertarget{pancreas-2}{%
\subsubsection{Pancreas}\label{pancreas-2}}

\begin{center}\rule{0.5\linewidth}{\linethickness}\end{center}

 - \textbf{WIPF1 antagonizes the tumor suppressive effect of
miR-141/200c and is associated with poor survival in patients with PDAC}
\emph{Journal of experimental \& clinical cancer research : CR 2018
Jul;37(1):167}

BACKGROUND: Aberrant expression of Wiskott-Aldrich syndrome protein
interacting protein family member 1 (WIPF1) contributes to the invasion
and metastasis of several malignancies. However, the role of WIPF1 in
human pancreatic ductal adenocarcinoma (PDAC) remains poorly understood.
METHODS: Human pancreatic cancer samples from PDAC patients were
collected for methylation analysis. Bioinformatic prediction program and
luciferase reporter assay were used to identify microRNAs regulating
WIPF1 expression. The association between WIPF1 expression and the
overall survival (OS) of patients with PDAC was evaluated by using The
Cancer Genome Atlas (TCGA) database. The roles of miR-141/200c and WIPF1
on the invasion and metastasis of PDAC cells were investigated both in
vitro and in vivo. RESULTS: We found that compared to the surrounding
non-cancerous tissues, there was significantly increased methylation of
miR-200c and miR-141 in human PDAC tissues that resulted in their
silencing, whereas the members of the other cluster of miR-200 family
including miR-200a, miR-200b and miR-429 were hypomethylated. Our data
show that forced expression of miR-141 or miR-200c suppressed invasion
and metastasis of PDAC cells both in vitro and in xenograft and
identified WIPF1 as a direct target of miR-141 and miR-200c. Both
miR-141 and miR-200c inhibit WIPF1 by directly interacting with its
3'-untranslated region. Remarkably, silencing of WIPF1 blocked PDAC
growth and metastasis both in vitro and in vivo, whereas forced WIPF1
overexpression antagonized the tumor suppressive effect of miR-141/200c.
Additionally, by using TCGA database we showed that high expression of
WIPF1 correlated with poor survival in patients with PDAC. Moreover, we
show that miR-141 and miR-200c blocked YAP/TAZ expression by suppressing
WIPF1. CONCLUSIONS: We have identified WIPF1 as an oncoprotein in PDAC
and a direct target of miR-141/miR-200c. We have also defined the
miR-141/200c-WIPF1-YAP/TAZ as a novel signaling pathway that is involved
in the regulation of the invasion and metastasis of human PDAC cells.

PubMed: \url{https://www.ncbi.nlm.nih.gov/pubmed/?term=30041660}

{}

{}

\begin{center}\rule{0.5\linewidth}{\linethickness}\end{center}

 - \textbf{Liquid Biopsies for Management of Pancreatic Cancer}

\url{https://www.sciencedirect.com/science/article/pii/S1931524418301099}

\begin{center}\rule{0.5\linewidth}{\linethickness}\end{center}

 - \textbf{Smad4/DPC4}

\emph{Journal of clinical pathology 2018 Aug;71(8):661-664}

Smad4 or DPC4 belongs to a family of signal transduction proteins that
are phosphorylated and activated by transmembrane serine-threonine
receptor kinases in response to transforming growth factor beta (TGF-β)
signaling via several pathways. The gene acts as a tumour suppressor
gene and inactivation of smad4/DPC4 is best recognised in pancreatic
cancer. However, smad4/DPC4 is also mutated in other conditions and
cancers such as juvenile polyposis syndrome with and without hereditary
haemorrhagic telangiectasia, colorectal and prostate
cancers.Immunohistochemistry for smad4/DPC4 protein is most useful in
separating benign/reactive conditions from pancreatic cancer in
needle/core biopsies. In normal and reactive states, the protein is
localised to the cytoplasm and nucleus, while the protein is lost in
high-grade pancreatic intraepithelial neoplasia/carcinoma in situ and
pancreatic cancer.

PubMed: \url{https://www.ncbi.nlm.nih.gov/pubmed/?term=29720405}

{}

{}

\begin{center}\rule{0.5\linewidth}{\linethickness}\end{center}

 - \textbf{Prospective Study of Germline Genetic Testing in Incident
Cases of Pancreatic Adenocarcinoma}

\emph{Cancer 2018 Aug;():}

BACKGROUND: The objective of this study was to investigate the
prevalence of pathogenic germline variants (PGVs) in 32 cancer
susceptibility genes in individuals with newly diagnosed pancreatic
ductal adenocarcinoma (PDAC). A key secondary objective was to evaluate
how often PGVs would have been undetected with existing genetic testing
criteria. METHODS: From May 2016 through May 2017, this multicenter
cohort study enrolled consecutive patients aged 18 to 89 years with
histologically confirmed PDAC diagnosed within the previous 12 weeks.
Demographics, medical histories, and 3-generation pedigrees were
collected from participants who provided samples for germline DNA
analysis. RESULTS: Four hundred nineteen patients were deemed eligible,
302 were enrolled, and 298 were included in the final cohort. Clinically
actionable variants were reported in 29 PDAC patients (9.7\%), with 23
(7.7\%) having a PGV associated with an increased risk for PDAC. Six of
23 individuals (26\%) with PDAC-associated gene mutations did not meet
currently established genetic testing criteria. According to
guideline-based genetic testing, only 11 of the 23 PGVs (48\%) in known
PDAC genes would have been detected. Six additional patients (2\%) had
PGVs associated with an increased risk for other cancers. CONCLUSIONS:
These findings support the significant prevalence of PGVs associated
with PDAC and the limitations of current paradigms for selecting
patients for genetic testing, and they thereby lend support for
universal germline multigene genetic testing in this population. Cancer
2018;000:000-000. © 2018 American Cancer Society.

PubMed: \url{https://www.ncbi.nlm.nih.gov/pubmed/?term=30067863}

{}

{}

\begin{center}\rule{0.5\linewidth}{\linethickness}\end{center}

 - \textbf{Multi-institutional Validation Study of Pancreatic Cyst Fluid
Protein Analysis for Prediction of High-risk Intraductal Papillary
Mucinous Neoplasms of the Pancreas}

\emph{Annals of surgery 2018 Aug;268(2):340-347}

OBJECTIVE: Preliminary work by our group suggested that proteins within
the pancreatic cyst fluid (CF) may discriminate degree of IPMN
dysplasia. We sought to externally validate these markers and determine
whether their inclusion in a preoperative clinical nomogram could
increase diagnostic accuracy. SUMMARY BACKGROUND DATA: IPMN is the most
common radiographically identifiable precursor to pancreatic cancer;
however, the timing and frequency of its malignant progression are
unknown, and there are currently no reliable preoperative tests that can
determine the grade of dysplasia in IPMN. METHODS: Clinical and
radiographic data, as well as CF samples, were obtained from 149
patients who underwent resection for IPMN at 1 of 3 institutions.
High-risk disease was defined as the presence of high-grade dysplasia or
invasive carcinoma. Multianalyte bead array analysis (Luminex) of CF was
performed for 4 protein markers that were previously associated with
high-risk disease. Logistic regression models were fit on training data,
with and without adjustment for a previously developed clinical nomogram
and validated with an external testing set. The models incorporating
clinical risk score were presented graphically as nomograms. RESULTS:
Within the group of 149 resected patients, 89 (60\%) had low-risk
disease, and 60 (40\%) had high-risk disease. All 4 CF markers (MMP9,
CA72-4, sFASL, and IL-4) were overexpressed in patients with high-risk
IPMN (P \textless{} 0.05). Two predictive models based on preselected
combinations of CF markers had concordance indices of 0.76 (Model-1) and
0.80 (Model-2). Integration of each CF marker model into a previously
described clinical nomogram leads to increased discrimination compared
with either the CF models or nomogram alone (c-indices of 0.84 and 0.83,
respectively). CONCLUSIONS: This multi-institutional study validated 2
CF protein marker models for preoperative identification of high-risk
IPMN. When combined with a clinical nomogram, the ability to predict
high-grade dysplasia was even stronger.

PubMed: \url{https://www.ncbi.nlm.nih.gov/pubmed/?term=28700444}

{}

{}

\begin{center}\rule{0.5\linewidth}{\linethickness}\end{center}

 - \textbf{Regulation of Epithelial Plasticity Determines Metastatic
Organotropism in Pancreatic Cancer}

\emph{Developmental cell 2018 06;45(6):696-711.e8}

The regulation of metastatic organotropism in pancreatic ductal a
denocarcinoma (PDAC) remains poorly understood. We demonstrate, using
multiple mouse models, that liver and lung metastatic organotropism is
dependent upon p120catenin (p120ctn)-mediated epithelial identity.
Mono-allelic p120ctn loss accelerates KrasG12D-driven pancreatic cancer
formation and liver metastasis. Importantly, one p120ctn allele is
sufficient for E-CADHERIN-mediated cell adhesion. By contrast, cells
with bi-allelic p120ctn loss demonstrate marked lung organotropism;
however, rescue with p120ctn isoform 1A restores liver metastasis. In~a
p120ctn-independent PDAC model, mosaic loss of E-CADHERIN expression
reveals selective pressure for E-CADHERIN-positive liver metastasis and
E-CADHERIN-negative lung metastasis. Furthermore, human PDAC and liver
metastases support the premise that liver metastases exhibit
predominantly epithelial characteristics. RNA-seq demonstrates
differential induction of pathways associated with metastasis and
epithelial-to-mesenchymal transition in p120ctn-deficient versus
p120ctn-wild-type cells. Taken together, P120CTN and E-CADHERIN mediated
epithelial plasticity is an addition to the conceptual framework
underlying metastatic organotropism in pancreatic cancer.

PubMed: \url{https://www.ncbi.nlm.nih.gov/pubmed/?term=29920275}

{}

{}

\begin{center}\rule{0.5\linewidth}{\linethickness}\end{center}

 - \textbf{A Highly Verified Assay for KRAS Mutation Detection in Tissue
and Plasma of Lung, Colorectal, and Pancreatic Cancer}

\emph{Archives of pathology \& laboratory medicine 2018 Aug;():}

CONTEXT: - KRAS Mutation Test v2 is used for the qualitative detection
and identification of 28 mutations in exons 2, 3, and 4 of the human
KRAS gene. OBJECTIVE: - To verify the performance of KRAS Mutation Test
v2 and to evaluate its accuracy by correlation with a next-generation
sequencing method on Illumina MiSeq. DESIGN: - In the study, we used
formalin-fixed, paraffin-embedded tissue and plasma specimens from
non-small cell lung cancer, colorectal cancer, and pancreatic cancer
patients. Results of specificity, precision, analytical sensitivity, and
accuracy as compared with a MiSeq method are reported. RESULTS: - The
KRAS Mutation Test v2 demonstrated exquisite sensitivity and specificity
and broad coverage of KRAS mutations. Precision was 100\% (108 of 108)
across all samples, operators, and instruments for formalin-fixed,
paraffin-embedded tissue and 99.8\% (615 of 616) for plasma. Analytical
sensitivity was high with detection of 1\% mutant sequence in
formalin-fixed, paraffin-embedded tissue samples and as low as 25 mutant
sequence copies/mL for plasma samples. The test also showed high overall
concordance for formalin-fixed, paraffin-embedded tumor tissue as well
as for plasma specimens when compared with MiSeq sequencing results.
CONCLUSIONS: - The KRAS Mutation Test v2 is a highly robust,
reproducible, and sensitive test for the qualitative detection of 28
mutations in exons 2, 3, and 4 of the KRAS gene in both solid (tissue)
and liquid (plasma) biopsies from colorectal cancer, non-small cell lung
cancer, and pancreatic cancer, and is a convenient option for KRAS
mutation testing.

PubMed: \url{https://www.ncbi.nlm.nih.gov/pubmed/?term=30088781}

{}

{}

\begin{center}\rule{0.5\linewidth}{\linethickness}\end{center}

\begin{itemize}
\tightlist
\item
  \textbf{A systematic review on metabolomics-based diagnostic biomarker
  discovery and validation in pancreatic cancer}
\end{itemize}

\url{https://link.springer.com/article/10.1007/s11306-018-1404-2}

\begin{center}\rule{0.5\linewidth}{\linethickness}\end{center}

\begin{itemize}
\tightlist
\item
  \textbf{Circulating Tumor Cells Dynamics in Pancreatic Adenocarcinoma
  Correlate With Disease Status: Results of the Prospective CLUSTER
  Study}
\end{itemize}

\url{https://journals.lww.com/annalsofsurgery/Fulltext/2018/09000/Circulating_Tumor_Cells_Dynamics_in_Pancreatic.4.aspx}

\begin{center}\rule{0.5\linewidth}{\linethickness}\end{center}

\begin{itemize}
\tightlist
\item
  \textbf{From somatic mutation to early detection: Insights from
  molecular characterization of pancreatic cancer precursor lesions}
\end{itemize}

\url{https://onlinelibrary.wiley.com/doi/abs/10.1002/path.5154}

\begin{center}\rule{0.5\linewidth}{\linethickness}\end{center}

 - \textbf{Germline Variants and Risk for Pancreatic Cancer: A
Systematic Review and Emerging Concepts}

\emph{Pancreas 2018 Sep;47(8):924-936}

Pancreatic cancer requires many genetic mutations. Combinations of
underlying germline variants and environmental factors may increase the
risk of cancer and accelerate the oncogenic process. We systematically
reviewed, annotated, and classified previously reported pancreatic
cancer-associated germline variants in established risk genes. Variants
were scored using multiple criteria and binned by evidence for
pathogenicity, then annotated with published functional studies and
associated biological systems/pathways. Twenty-two previously identified
pancreatic cancer risk genes and 337 germline variants were identified
from 97 informative studies that met our inclusion criteria. Fifteen of
these genes contained 66 variants predicted to be pathogenic (APC, ATM,
BRCA1, BRCA2, CDKN2A, CFTR, CHEK2, MLH1, MSH2, NBN, PALB2, PALLD, PRSS1,
SPINK1, TP53). Pancreatic cancer risk genes were organized into key
biological mechanisms that promote pancreatic oncogenesis within an
oncogenic model. Development of precision medicine approaches requires
updated variant information within the framework of an oncogenic
progression model. Complex risk modeling may improve interpretation of
early biomarkers and guide pathway-specific treatment for pancreatic
cancer in the future. Precision medicine is within reach.

PubMed: \url{https://www.ncbi.nlm.nih.gov/pubmed/?term=30113427}

{}

{}

\begin{center}\rule{0.5\linewidth}{\linethickness}\end{center}

\begin{itemize}
\tightlist
\item
  \textbf{A Highly Verified Assay for KRAS Mutation Detection in Tissue
  and Plasma of Lung, Colorectal, and Pancreatic Cancer}
\end{itemize}

\url{http://www.archivesofpathology.org/doi/pdf/10.5858/arpa.2017-0471-OA?code=coap-site}

\begin{center}\rule{0.5\linewidth}{\linethickness}\end{center}

\begin{itemize}
\tightlist
\item
  \textbf{PAR1 signaling on tumor cells limits tumor growth by
  maintaining a mesenchymal phenotype in pancreatic cancer}
\end{itemize}

\url{https://www.researchgate.net/profile/Cansu_Tekin2/publication/326967265_PAR1_signaling_on_tumor_cells_limits_tumor_growth_by_maintaining_a_mesenchymal_phenotype_in_pancreatic_cancer/links/5b73188245851546c90320f1/PAR1-signaling-on-tumor-cells-limits-tumor-growth-by-maintaining-a-mesenchymal-phenotype-in-pancreatic-cancer.pdf}

\begin{center}\rule{0.5\linewidth}{\linethickness}\end{center}

 - \textbf{Pancreatitis-Associated Genes and Pancreatic Cancer Risk: A
Systematic Review and Meta-analysis}

\emph{Pancreas 2018 Aug;():}

OBJECTIVE: The aim of this study was to evaluate the connection between
pancreatic cancer (PC) and genetic variants associated with chronic
pancreatitis via systematic review and meta-analysis. METHODS: The data
search was performed in 3 major databases (PubMed, EMBASE, and Cochrane
Library). The selected studies have looked into the presence of the
pancreatitis-associated genes in patients with PC and in control
subjects, the outcome being the frequency of the mutations in the 2
groups. For the binary outcomes, pooled odds ratio (OR) and 95\%
confidence interval (CI) were calculated. RESULTS: Ten articles proved
to be eligible for the qualitative synthesis, and 8 articles were
suitable for statistical analysis. Six case-control studies, comprising
929 PC cases and 1890 control subjects for serine protease inhibitor
Kazal type 1 (SPINK1) mutations, and 5 case-control studies, comprising
1674 PC cases and 19,036 control subjects for CFTR mutations, were
enrolled in our analysis. SPINK1 mutations showed no association with PC
(OR, 1.52; 95\% CI, 0.67-3.45; P = 0.315), whereas mutations in CFTR
modestly increased the risk of PC (OR, 1.41; 95\% CI, 1.07-1.84; P =
0.013). CONCLUSION: Our meta-analysis showed that mutations in CFTR
modestly increase the risk of PC, whereas no association was found
between SPINK1 and PC.This is an open-access article distributed under
the terms of the Creative Commons Attribution-Non Commercial-No
Derivatives License 4.0 (CCBY-NC-ND), where it is permissible to
download and share the work provided it is properly cited. The work
cannot be changed in any way or used commercially without permission
from the journal.

PubMed: \url{https://www.ncbi.nlm.nih.gov/pubmed/?term=30134356}

{}

{}

\begin{center}\rule{0.5\linewidth}{\linethickness}\end{center}

 - \textbf{Germline and Somatic DNA Damage Repair Gene Mutations and
Overall Survival in Metastatic Pancreatic Adenocarcinoma Patients
Treated with FOLFIRINOX}

\emph{Clinical cancer research : an official journal of the American
Association for Cancer Research 2018 Aug;():}

PURPOSE: Pancreatic ductal adenocarcinoma (PDAC) is a lethal cancer with
lack of predictive biomarkers. We conducted a study to assess DNA damage
repair (DDR) gene mutations as a predictive biomarker in PDAC patients
treated with FOLFIRINOX. EXPERIMENTAL DESIGN: Indiana University Simon
Cancer Center pancreatic cancer database was used to identify patients
with metastatic PDAC, treated with FOLFIRINOX and had tissue available
for DNA sequencing. Baseline demographic, clinical and pathologic
information was gathered. DNA isolation and targeted sequencing was
performed using the Ion AmpliSeq protocol. Overall survival (OS)
analyses was conducted using Kaplan-Meier, logistic regression and Cox
proportional hazard methods. Multivariate models were adjusted for age,
gender, margin status, CA 19-9, adjuvant chemotherapy, tumor and nodal
stage. RESULTS: Overall, 36 patients were sequenced. DDR gene mutations
were found in 12 patients. Mutations were seen in BRCA1 (N=7), BRCA2
(N=5), PALB2 (N=3), MSH2 (N=1) and FANCF (N=1) of all the DDR genes
sequenced. Median age was 65.5 years, 58\% were male, 97.2\% were
Caucasian and 51.4\% had any family history of cancer. The median OS was
near significantly superior in those with DDR gene mutations present
vs.~absent (14 vs.~5 months; HR 0.58 {[}0.29-1.14{]}, log-rank p=0.08).
Multivariate logistic (OR 1.47 {[}1.04-2.06{]}, p = 0.04) and Cox
regression (HR 0.37 {[}0.15-0.94{]}, p = 0.04) showed presence of DDR
gene mutations was associated with improved OS. CONCLUSION: In a single
institution, retrospective study, we found that the presence of DDR gene
mutations are associated with improved OS in PDAC patients treated with
FOLFIRINOX.

PubMed: \url{https://www.ncbi.nlm.nih.gov/pubmed/?term=30131383}

{}

{}

\begin{center}\rule{0.5\linewidth}{\linethickness}\end{center}

\hypertarget{biliary-tract-2}{%
\subsubsection{Biliary Tract}\label{biliary-tract-2}}

\begin{center}\rule{0.5\linewidth}{\linethickness}\end{center}

 - \textbf{Hypoxia-Induced PLOD2 is a Key Regulator in
Epithelial-Mesenchymal Transition and Chemoresistance in Biliary Tract
Cancer}

\emph{Annals of surgical oncology 2018 Aug;():}

BACKGROUND: The prognosis of biliary tract cancer (BTC) is unfavorable
due to its chemoresistance. Hypoxia triggers epithelial-to-mesenchymal
transition (EMT), which is closely related to drug resistance. In this
study, we focused on the functional roles of procollagen-lysine,
2-oxoglutarate 5-dioxygenase 2 (PLOD2), a hypoxia-induced gene, in BTC,
and assessed the clinical significance of PLOD2. METHODS: The expression
of PLOD2 under hypoxia was assessed in BTC cell lines.
Gemcitabine-resistant (GR) BTC cell lines were transfected with small
interfering RNA (siRNA) against PLOD2, and EMT markers and
chemoresistance were evaluated. PLOD2 expression was also characterized
using immunohistochemistry in BTC clinical specimens following
resection. Patient survival was analyzed and the role of PLOD2
expression was examined. RESULTS: The expression of PLOD2 was induced by
hypoxia in vitro and was upregulated in BTC-GR cell lines, which had low
expression of epithelial markers and high expression of mesenchymal
markers. Downregulation of PLOD2 by siRNA resulted in improved
chemoresistance, recovery of epithelial markers, and reduction of
mesenchymal markers. In the resected BTC samples, PLOD2 expression was
significantly correlated with lymph node metastasis (p = 0.037) and
stage (p = 0.001). Recurrence-free survival (p = 0.011) and overall
survival (p \textless{} 0.001) rates were significantly lower in
patients with high expression of PLOD2. PLOD2 expression was an
independent prognostic factor for overall survival (p = 0.019).
CONCLUSIONS: The expression of PLOD2 influenced chemoresistance through
EMT, and high expression of PLOD2 was a significant unfavorable
prognostic factor in BTC patients. PLOD2 might be a potential
therapeutic target for overcoming chemoresistance.

PubMed: \url{https://www.ncbi.nlm.nih.gov/pubmed/?term=30105440}

{}

{}

\begin{center}\rule{0.5\linewidth}{\linethickness}\end{center}

\hypertarget{gallbladder-2}{%
\subsubsection{Gallbladder}\label{gallbladder-2}}

\begin{center}\rule{0.5\linewidth}{\linethickness}\end{center}

\hypertarget{ampulla-of-vater-2}{%
\subsubsection{Ampulla of Vater}\label{ampulla-of-vater-2}}

\begin{center}\rule{0.5\linewidth}{\linethickness}\end{center}

\hypertarget{pannet-neuroendocrine-2}{%
\subsubsection{PanNET Neuroendocrine}\label{pannet-neuroendocrine-2}}

\begin{center}\rule{0.5\linewidth}{\linethickness}\end{center}

\begin{itemize}
\tightlist
\item
  \textbf{Molecular Genetic Studies of Pancreatic Neuroendocrine Tumors}
\end{itemize}

\url{https://www.endo.theclinics.com/article/S0889-8529(18)30519-X/abstract}

\begin{center}\rule{0.5\linewidth}{\linethickness}\end{center}

Back to top

\pagebreak

\hypertarget{others}{%
\subsection{Others}\label{others}}

\begin{center}\rule{0.5\linewidth}{\linethickness}\end{center}

\hypertarget{pancreas-3}{%
\subsubsection{Pancreas}\label{pancreas-3}}

\begin{center}\rule{0.5\linewidth}{\linethickness}\end{center}

 - \textbf{Adjuvant Treatment in Potentially Curable Pancreatic Cancer:
Need to Include Tumor Location in the Equation?}

\emph{Pancreas 2018 Sep;47(8):e50-e52}

PubMed: \url{https://www.ncbi.nlm.nih.gov/pubmed/?term=30113430}

{}

{}

\begin{center}\rule{0.5\linewidth}{\linethickness}\end{center}

 - \textbf{Loss of PDPK1 abrogates resistance to gemcitabine in
label-retaining pancreatic cancer cells}

\emph{BMC cancer 2018 Jul;18(1):772}

BACKGROUND: Label-retaining cancer cells (LRCC) have been proposed as a
model of slowly cycling cancer stem cells (CSC) which mediate resistance
to chemotherapy, tumor recurrence, and metastasis. The molecular
mechanisms of chemoresistance in LRCC remain to-date incompletely
understood. This study aims to identify molecular targets in LRCC that
can be exploited to overcome resistance to gemcitabine, a standard
chemotherapy agent for the treatment of pancreas cancer. METHODS: LRCC
were isolated following Cy5-dUTP staining by flow cytometry from
pancreatic cancer cell lines. Gene expression profiles obtained from
LRCC, non-LRCC (NLRCC), and bulk tumor cells were used to generate
differentially regulated pathway networks. Loss of upregulated targets
in LRCC on gemcitabine sensitivity was assessed via RNAi experiments and
pharmacological inhibition. Expression patterns of PDPK1, one of the
upregulated targets in LRCC, was studied in patients' tumor samples and
correlated with pathological variables and clinical outcome. RESULTS:
LRCC are significantly more resistant to gemcitabine than the bulk tumor
cell population. Non-canonical EGF (epidermal growth factor)-mediated
signal transduction emerged as the top upregulated network in LRCC
compared to non-LRCC, and knock down of EGF signaling effectors PDPK1
(3-phosphoinositide dependent protein kinase-1), BMX (BMX non-receptor
tyrosine kinase), and NTRK2 (neurotrophic receptor tyrosine kinase 2) or
treatment with PDPK1 inhibitors increased growth inhibition and
induction of apoptosis in response to gemcitabine. Knockdown of PDPK1
preferentially increased growth inhibition and reduced resistance to
induction of apoptosis upon gemcitabine treatment in the LRCC vs
non-LRCC population. These findings are accompanied by lower expression
levels of PDPK1 in tumors compared to matched uninvolved pancreas in
surgical resection specimens and a negative association of membranous
localization on IHC with high nuclear grade (p \textless{} 0.01).
CONCLUSION: Pancreatic cancer cell-derived LRCC are relatively resistant
to gemcitabine and harbor a unique transcriptomic profile compared to
bulk tumor cells. PDPK1, one of the members of an upregulated
EGF-signaling network in LRCC, mediates resistance to gemcitabine, is
found to be dysregulated in pancreas cancer specimens, and might be an
attractive molecular target for combination therapy studies.

PubMed: \url{https://www.ncbi.nlm.nih.gov/pubmed/?term=30064387}

{}

{}

\begin{center}\rule{0.5\linewidth}{\linethickness}\end{center}

 - \textbf{Intraductal papillary mucinous neoplasms of the pancreas - a
cost-effectiveness analysis of management strategies for the branch-duct
subtype}

\emph{HPB : the official journal of the International Hepato Pancreato
Biliary Association 2018 Jul;():}

BACKGROUND: Branch-duct intraductal papillary mucinous neoplasm
(BD-IPMN) presents a clinical conundrum. Rigorous long-term surveillance
or surgical resection is recommended. The economic consequences of the
management have not been fully investigated. METHODS: A Markov decision
model compared 4 strategies for low-risk BD-IPMN: I~=~upfront total
pancreatectomy, II~=~upfront partial pancreatectomy, III~=~initial
surveillance, IV~=~watchful waiting. Surveillance was based on the
Swedish Guidelines for Pancreatic Cancer. Probabilities and costs were
obtained from the participating unit and from the scientific literature.
The incremental cost-effectiveness ratios (ICERs) were calculated and
sensitivity analyses were performed by varying relevant parameters.
Survival was reported in quality-adjusted life-years (QALYs). RESULTS:
Strategy III was the most cost-effective strategy with an ICER of €31
682 compared to strategy IV. Strategy I was the most expensive but
yielded the best QALY (9.32). Total number of years, annual risk of
pancreatic cancer and annual risk of a low-risk BD-IPMN turning into a
high-risk lesion had the greatest impact in the model. CONCLUSIONS:
Initial surveillance seems to be the most cost-effective strategy in the
management of low-risk asymptomatic BD-IPMN. However, the possibility of
personalized approaches remains to be investigated.

PubMed: \url{https://www.ncbi.nlm.nih.gov/pubmed/?term=30064727}

{}

{}

\begin{center}\rule{0.5\linewidth}{\linethickness}\end{center}

 - \textbf{The Lymph Node Ratio Is an Independent Prognostic Factor in
Pancreatic Cancer Patients Who Receive Curative Resection Followed by
Adjuvant Chemotherapy}

\emph{Anticancer research 2018 Aug;38(8):4877-4882}

BACKGROUND/AIM: The present study investigated the impact of the lymph
node ratio (LNR) on survival and recurrence in patients with pancreatic
cancer after curative surgery followed by adjuvant chemotherapy.
PATIENTS AND METHODS: This study included 189 patients who underwent
curative surgery followed by adjuvant chemotherapy for pancreatic cancer
between 2005 and 2014. The risk factors for overall survival (OS) and
recurrence-free survival (RFS) were identified. RESULTS: A lymph node
ratio of 0.1 was considered to be the optimal cut-off point for
classification based on the 3-year and 5-year survival rates. The OS
rates at three and five years after surgery were 34.4\% and 28.2\% in
the LNR \textless{}0.1 group, respectively, and 23.1\% and 5.8\% in the
LNR ≥0.1 group, which amounted to a statistically significant difference
(p=0.003). The RFS rates at one and three years after surgery were
26.6\% and 20.5\% in the LNR \textless{}0.1 group, respectively, and
8.0\% and 0\% in the LNR ≥0.1 group, which was a significant difference
(p=0.001). A multivariate analysis demonstrated that the LNR was a
significant independent risk factor for both the OS and RFS. CONCLUSION:
The LNR was a risk factor for overall survival in patients who underwent
curative surgery followed by adjuvant chemotherapy for pancreatic
cancer. It is necessary to develop strategies to effectively utilize the
lymph node metastasis status.

PubMed: \url{https://www.ncbi.nlm.nih.gov/pubmed/?term=30061263}

{}

{}

\begin{center}\rule{0.5\linewidth}{\linethickness}\end{center}

 - \textbf{Well differentiated liposarcoma, sclerosing type, of the
pancreas a case report}

\emph{Experimental and molecular pathology 2016 12;101(3):320-322}

PubMed: \url{https://www.ncbi.nlm.nih.gov/pubmed/?term=27840110}

{}

{}

\begin{center}\rule{0.5\linewidth}{\linethickness}\end{center}

 - \textbf{Growth rate of serous pancreatic neoplasms in~vivo: a
retrospective, observational study}

\emph{Acta radiologica (Stockholm, Sweden : 1987) 2018
Jul;():284185118787350}

Background Determining the growth rate of pancreatic cystic lesions on
follow-up imaging is important in managing patients with these lesions.
However, the growth rates of serous pancreatic neoplasms (SPNs) have
been reported to vary among studies. Purpose To determine the in~vivo
growth rate of SPNs. Material and Methods This retrospective,
single-institutional study included patients diagnosed with SPNs during
2006-2015. The diagnosis of SPNs was based on the results of surgery,
endoscopic ultrasonography (EUS)-guided fine needle aspiration (FNA) or
core needle biopsy (CNB), or typical radiologic features of SPN. A
linear mixed-effects model was utilized to determine whether the
diagnostic intervention was associated with tumor growth rate in all
patients. The in~vivo growth rate of SPNs was estimated from patients
without or before diagnostic intervention. SPN growth rates were
compared before and after diagnostic intervention. Results SPN growth
rates in the overall patient cohort (n = 304) differed significantly
between patients who did and did not undergo diagnostic interventions.
The in~vivo SPN growth rate in 204 patients without or before diagnostic
intervention was 1.9 mm/year (95\% confidence interval
{[}CI{]} = 1.6-2.2). In the 130 patients who underwent diagnostic
intervention, the SPN growth rate was significantly greater before than
after diagnostic intervention (1.8 vs.~0.2 mm/year). Conclusions In the
absence of diagnostic intervention, the in~vivo growth rate of SPNs was
1.9 mm/year (95\% CI = 1.6-2.2). EUS-guided FNA or CNB may affect the
growth rate of SPNs.

PubMed: \url{https://www.ncbi.nlm.nih.gov/pubmed/?term=30056739}

{}

{}

\begin{center}\rule{0.5\linewidth}{\linethickness}\end{center}

 - \textbf{Progression to pancreatic ductal adenocarcinoma from
pancreatic intraepithelial neoplasia: Results of a simulation model}

\url{https://www.pancreatology.net/article/S1424-3903(18)30636-7/abstract}

\begin{center}\rule{0.5\linewidth}{\linethickness}\end{center}

 - \textbf{Serotonin uptake is required for Rac1 activation in
Kras-induced acinar-to-ductal metaplasia in the pancreas}

\emph{The Journal of pathology 2018 Jul;():}

Pancreatic ductal adenocarcinoma (PDAC), the primary cause of pancreatic
cancer mortality, is poorly responsive to currently available
interventions. Identifying new targets that drive PDAC formation and
progression is critical to develop alternative therapeutic strategies to
treat this lethal malignancy. Using genetic and pharmacologic
approaches, we investigated in vivo and in vitro whether uptake of the
monoamine serotonin is required for PDAC development. We demonstrated
that pancreatic acinar cells have the ability to readily take up
serotonin in a transport-mediated manner. Serotonin uptake promoted the
activation of the small GTPase Ras-Related C3 Botulinum Toxin Substrate
1 (Rac1), which is required for trans-differentiation of acinar cells
into acinar-to-ductal metaplasia (ADM), a key determinant in PDAC
development. Consistent with the central role played by Rac1 in ADM
formation, inhibition of the serotonin transporter Sert (Slc6a4) with
fluoxetine reduced ADM formation both in vitro and in vivo in a cell
autonomous manner. In addition, fluoxetine treatment profoundly
compromised the stromal reaction and affected proliferation and lipid
metabolism of malignant PDAC cells. We propose that Sert is a promising
therapeutic target to counteract the early event of acinar-to-ductal
metaplasia with the potential to stall initiation and progression of
pancreatic carcinogenesis. This article is protected by copyright. All
rights reserved.

PubMed: \url{https://www.ncbi.nlm.nih.gov/pubmed/?term=30058725}

{}

{}

\begin{center}\rule{0.5\linewidth}{\linethickness}\end{center}

 - \textbf{Clinical Features and Prognosis of Patients With the Bone
Metastasis of Pancreatic Cancer: A Single-Institutional Cohort Study}

\emph{Pancreas 2018 Aug;47(7):e43-e46}

PubMed: \url{https://www.ncbi.nlm.nih.gov/pubmed/?term=29985850}

{}

{}

\begin{center}\rule{0.5\linewidth}{\linethickness}\end{center}

 - \textbf{Spatial Distribution of Pancreatic Stones in Chronic
Pancreatitis}

\emph{Pancreas 2018 Aug;47(7):864-870}

OBJECTIVES: The aim of this study was to establish a standard to
describe the spatial distribution of pancreatic stones in chronic
pancreatitis (CP). METHODS: Two hundred forty-seven CP patients with
pancreatic stones from June to December 2012 were enrolled.
Two-dimensional images from coronal projection of 3-dimensional computed
tomography images of pancreatic stones were gained. The number (n) of
all stones and the geometric standard deviation (σ) of distances between
the centroid of all stones and the centroids of every stone that
represented the spatial distribution nonuniformity were calculated by
Stone Reconstruction and Identification Programming System. RESULTS: The
mean value of n and σ were 13.6 and 22.5; n \textgreater{} 13.6 and σ
\textgreater{} 22.5 were determined as ``multistones'' and
``nonuniform,'' respectively. Compared with alcoholic CP, idiopathic CP
was less prone to multistones (odds ratio {[}OR{]}, 0.310) and more
prone to nonuniform (OR, 3.247). Pancreatic pseudocyst (OR, 2.211) in CP
course was a risk factor of multistones, whereas diabetes mellitus in
first-/second-/third-degree relatives (OR, 0.382) was a protective
factor. Age at diagnosis of pancreatic stones (OR, 1.022) was a risk
factor of nonuniformity. CONCLUSIONS: Compared with idiopathic CP,
alcoholic CP patients were prone to more pancreatic stones that
distribute more uniformly.

PubMed: \url{https://www.ncbi.nlm.nih.gov/pubmed/?term=29975348}

{}

{}

\begin{center}\rule{0.5\linewidth}{\linethickness}\end{center}

 - \textbf{Sweet Predictions Speak Volumes for Early Detection of
Pancreatic Cancer}

\emph{Gastroenterology 2018 08;155(2):265-268}

PubMed: \url{https://www.ncbi.nlm.nih.gov/pubmed/?term=29966611}

{}

{}

\begin{center}\rule{0.5\linewidth}{\linethickness}\end{center}

 - \textbf{Implications of the Pattern of Disease Recurrence on Survival
Following Pancreatectomy for Pancreatic Ductal Adenocarcinoma}

\emph{Annals of surgical oncology 2018 Aug;25(8):2475-2483}

BACKGROUND: After radical resection of pancreatic ductal adenocarcinoma
(PDAC), approximately 80\% of patients will develop disease recurrence.
It remains unclear to what extent the location of recurrence carries
prognostic significance. Additionally, stratifying the pattern of
recurrence may lead to a deeper understanding of the heterogeneous
biological behavior of PDAC. OBJECTIVE: The aim of this study was to
characterize the relationship of recurrence patterns with survival in
patients with resected PDAC. METHODS: This single-center cohort study
included patients undergoing pancreatectomy at the Johns Hopkins
Hospital between 2000 and 2013. Exclusion criteria were neoadjuvant
therapy and incomplete follow-up. Sites of first recurrence were
stratified into five groups and survival outcomes were estimated using
Kaplan-Meier curves. The association of specific recurrence locations
with overall survival (OS) was analyzed using Cox proportional-hazards
models with and without landmark analysis. RESULTS: Accurate follow-up
data were available for 877 patients, 662 (75.5\%) of whom had
documented recurrence at last follow-up. Patients with multiple-site
(n = 227, 4.7~months) or liver-only recurrence (n = 166, 7.2~months) had
significantly worse median survival after recurrence when compared with
lung- (n = 93) or local-only (n = 158) recurrence (15.4 and 9.7~months,
respectively). On multivariable analysis, the unique recurrence patterns
had variable predictive values for OS. Landmark analyses, with landmarks
set at 12, 18, and 24~months, confirmed these findings. CONCLUSIONS:
This study demonstrates that specific patterns of PDAC recurrence result
in different survival outcomes. Furthermore, distinct first recurrence
locations have unique independent predictive values for OS, which could
help with prognostic stratification and decisions regarding treatment
after the diagnosis of recurrence.

PubMed: \url{https://www.ncbi.nlm.nih.gov/pubmed/?term=29948425}

{}

{}

\begin{center}\rule{0.5\linewidth}{\linethickness}\end{center}

 - \textbf{Vanishing Pancreas}

\emph{Gastroenterology 2018 08;155(2):280-281}

PubMed: \url{https://www.ncbi.nlm.nih.gov/pubmed/?term=29409874}

{}

{}

\begin{center}\rule{0.5\linewidth}{\linethickness}\end{center}

 - \textbf{Don't Mess With the Pancreas (Wherever It May Be): Acute
Pancreatic Rest``itis'' Presenting as a Submucosal Mass With Gastric
Outlet Obstruction}

\emph{Gastroenterology 2018 08;155(2):e1-e2}

PubMed: \url{https://www.ncbi.nlm.nih.gov/pubmed/?term=29409828}

{}

{}

\begin{center}\rule{0.5\linewidth}{\linethickness}\end{center}

\begin{itemize}
\tightlist
\item
  \textbf{The Use of Biomarkers in the Risk Stratification of Cystic
  Neoplasms}
\end{itemize}

\url{https://www.giendo.theclinics.com/article/S1052-5157(18)30725-6/abstract}

\begin{center}\rule{0.5\linewidth}{\linethickness}\end{center}

\begin{itemize}
\tightlist
\item
  \textbf{Current Guideline Controversies in the Management of
  Pancreatic Cystic Neoplasms}
\end{itemize}

\url{https://www.giendo.theclinics.com/article/S1052-5157(18)30724-4/abstract}

\begin{center}\rule{0.5\linewidth}{\linethickness}\end{center}

\begin{itemize}
\tightlist
\item
  \textbf{To resect or not to resect: a review of pancreatic cyst
  disease management}
\end{itemize}

\url{https://journals.lww.com/co-gastroenterology/Abstract/2018/09000/To_resect_or_not_to_resect___a_review_of.13.aspx}

\begin{center}\rule{0.5\linewidth}{\linethickness}\end{center}

 - \textbf{Epithelial-Mesenchymal Transition in Pancreatic Cancer: A
Review}

\emph{BioMed research international 2017 ;2017():2646148}

Pancreatic ductal adenocarcinoma (PDAC) is one of the most aggressive
solid malignancies and is characterized by its insensitivity to current
therapy. The invasion and metastasis of solid tumors such as PDAC are
complex processes involving many factors. Recent insights into the role
of cancer stem cells (CSCs) and the epithelial-mesenchymal transition
(EMT) in tumorigenesis have increased the knowledge base and highlighted
new therapeutic targets of this disease. The process of EMT is regulated
by a complex network of cytokines, transcription factors, growth
factors, signaling pathways, and the tumor microenvironment, exhibiting
CSC-like properties. The transition of solid cancer cells from an
epithelial to a mesenchymal phenotype increases their migratory and
invasive properties, thus promoting metastasis. In PDAC, the exact
influence of EMT on the biological behaviors of cancer cells and its
impact on clinical therapy remain controversial, but the therapeutic
strategy of combining EMT inhibition with chemotherapy deserves
attention. Alternatively, anti-inflammatory therapy that targets the
interaction between inflammation and EMT is a valid strategy for
treating the premalignant stage of tumor progression. In this review, we
summarize the latest research on EMT and the potential relationship
between EMT and PDAC.

PubMed: \url{https://www.ncbi.nlm.nih.gov/pubmed/?term=29379795}

{}

{}

\begin{center}\rule{0.5\linewidth}{\linethickness}\end{center}

 - \textbf{A Phase II Clinical Trial of Molecular Profiled Neoadjuvant
Therapy for Localized Pancreatic Ductal Adenocarcinoma}

\emph{Annals of surgery 2018 Aug;():}

OBJECTIVES: One facet of precision medicine is the use of tumor
molecular profiling to guide chemotherapeutic selection. We conducted
the first prospective clinical trial of molecular profiling to guide
neoadjuvant therapy in patients with operable pancreatic ductal
adenocarcinoma (PDAC). We hypothesized that more effective systemic
therapy would prevent disease progression during neoadjuvant therapy
and, therefore, allow more patients to undergo surgery. METHODS: In
patients with resectable and borderline resectable (BLR) PDAC, molecular
profiling consisted of immunocytochemical staining of pretreatment
endoscopic ultrasound-guided fine needle aspiration tumor biopsies using
6 biomarkers. Neoadjuvant systemic therapy was selected based on the
molecular profiling results. The primary endpoint was the completion of
all intended neoadjuvant therapy and surgery. RESULTS: The trial
enrolled 130 patients; 61 (47\%) resectable and 69 (53\%) BLR. Molecular
profiling was reported within a median of 5 business days (IQR: 3). Of
the 130 patient samples, 95 (73\%) had adequate cellularity for
molecular profiling and 92 (71\%) patients received molecular
profile-directed therapy. Of the 92 patients who had predictive
profiling, 74 (80\%) received fluoropyrimidine-based therapy and 18
(20\%) received gemcitabine-based therapies. Of the 130 patients, 107
(82\%) completed all intended neoadjuvant therapy and surgery; 56 (92\%)
of the 61 with resectable PDAC and 51 (74\%) of 69 with BLR PDAC.
CONCLUSIONS: We report the first prospective clinical trial that
utilized molecular profiling to select neoadjuvant therapy in patients
with operable PDAC. Such high resectability rates have not been observed
in prior neoadjuvant trials, suggesting that molecular profiling may
improve the efficacy of chemotherapy in these patients.

PubMed: \url{https://www.ncbi.nlm.nih.gov/pubmed/?term=30080723}

{}

{}

\begin{center}\rule{0.5\linewidth}{\linethickness}\end{center}

 - \textbf{Functions of the CXC ligand family in the pancreatic tumor
microenvironment}

\emph{Pancreatology : official journal of the International Association
of Pancreatology (IAP) \ldots{} {[}et al.{]} 2018 Aug;():}

Therapeutic resistance is the major contributor to the poor prognosis of
and low survival from pancreatic cancer (PC). Cancer progression is a
complex process reliant on interactions between the tumor and the tumor
microenvironment (TME). Members of the CXCL family of chemokines are
present in the pancreatic TME and seem to play a vital role in
regulating PC progression. As pancreatic tumors interact with the TME
and with PC stem cells (CSCs), determining the roles of specific members
of the CXCL family is vital to the development of improved therapies.
This review highlights the roles of selected CXCLs in the interactions
between pancreatic tumor and its stroma, and in CSC phenotypes, which
can be used to identify potential treatment targets.

PubMed: \url{https://www.ncbi.nlm.nih.gov/pubmed/?term=30078614}

{}

{}

\begin{center}\rule{0.5\linewidth}{\linethickness}\end{center}

 - \textbf{Possible Autocrine Function of~Galectin-3 in Pancreatic
Stellate Cells}

\emph{Gastroenterology 2018 Aug;():}

PubMed: \url{https://www.ncbi.nlm.nih.gov/pubmed/?term=30092185}

{}

{}

\begin{center}\rule{0.5\linewidth}{\linethickness}\end{center}

 - \textbf{Retrospective evaluation of patients diagnosed solid
pseudopapillary neoplasms of the pancreas}

\emph{Current problems in cancer 2018 Jul;():}

PURPOSE: Solid pseudopapillary neoplasm (SPN) is a rare, low-grade
neoplasm with excellent prognosis. In this study, we evaluated
clinicopathological characteristics of patients diagnosed with SPN
retrospectively. METHODS: This is a retrospective study intended to
characterize patients with the diagnosis of SPN between 2005 and 2015.
Clinicopathological features, recurrence rate, and overall survival of
28 patients were recorded. Malignant SPN criteria were defined as the
presence of distant metastasis (developed at diagnosis or during follow
up) or lymph node involvement. RESULTS: The mean age at diagnosis was 42
(range: 17-41). Among patients, 82\% (n\,=\,23) were female and 17.9\%
(n\,=\,5) were male. The mean size of tumor was 5.81 cm (range: 2-15).
The mean follow up period was 55.6 months, 1-year survival was 96.5\%
and 5-year survival rate was 88\%. A total of 25 patients were alive at
the end of follow-up period and 3 of the patients became exitus due to
disease. Two patients had a metastatic presentation in livers at the
diagnosis and metastasis developed in 3 patients during follow-up (liver
of 1 patient, peritoneum in 1 patient and liver and peritoneum in 1
patient). The reason of admission was headache in 68\% patients. The
type of operation was frequently subtotal pancreatectomy (n\,=\,11,
39.3\%) and distal pancreatectomy (n\,=\,10, 35.7\%). Tumors were
located frequently in body and tail regions (n\,=\,18, 64.3\%) and the
number of patients with malignant criteria was 6 (21.4\%). Although the
mean age of malignant patients was significantly higher than benign
patients (P\,=\,0.046), there was no significant difference between 2
groups in terms of gender, tumor size, capsule invasion, perineural
invasion, vascular invasion, and margin status. CONCLUSION: SPN is a
rarely seen tumor with low malignity potential. Surgical resection
provides long-term survival rate even in local invasion or metastasis
conditions.

PubMed: \url{https://www.ncbi.nlm.nih.gov/pubmed/?term=30104029}

{}

{}

\begin{center}\rule{0.5\linewidth}{\linethickness}\end{center}

\begin{itemize}
\tightlist
\item
  \textbf{Immune Checkpoint Inhibition for Pancreatic Ductal
  Adenocarcinoma: Current Limitations and Future Options}
\end{itemize}

\url{https://www.frontiersin.org/articles/10.3389/fimmu.2018.01878/full}

\begin{center}\rule{0.5\linewidth}{\linethickness}\end{center}

 - \textbf{Gut Microbiota Promotes Tumor Growth in Mice by Modulating
Immune Response}

\emph{Gastroenterology 2018 07;155(1):33-37.e6}

We studied the effects of gut microbiome depletion by oral antibiotics
on tumor growth in subcutaneous and liver metastases models of
pancreatic cancer, colon cancer, and melanoma. Gut microbiome depletion
significantly reduced tumor burden in all the models tested. However,
depletion of gut microbiome did not reduce tumor growth in Rag1-knockout
mice, which lack mature T and B cells. Flow cytometry analyses
demonstrated that gut microbiome depletion led to significant increase
in interferon gamma-producing T cells with corresponding decrease in
interleukin 17A and interleukin 10-producing T cells. Our results
suggest that gut microbiome modulation could emerge as a novel
immunotherapeutic strategy.

PubMed: \url{https://www.ncbi.nlm.nih.gov/pubmed/?term=29630898}

{}

{}

\begin{center}\rule{0.5\linewidth}{\linethickness}\end{center}

 - \textbf{Gut microbiome-immune crosstalk affects progression of
cancer}

\emph{Translational gastroenterology and hepatology 2018 ;3():34}

PubMed: \url{https://www.ncbi.nlm.nih.gov/pubmed/?term=30050994}

{}

{}

\begin{center}\rule{0.5\linewidth}{\linethickness}\end{center}

\begin{itemize}
\tightlist
\item
  \textbf{Intrapancreatic recurrence of intraductal tubulopapillary
  neoplasm (ITPN) 16 years after the initial surgery for noninvasive
  ITPN: a case report}
\end{itemize}

\url{https://surgicalcasereports.springeropen.com/articles/10.1186/s40792-018-0497-1}

\begin{center}\rule{0.5\linewidth}{\linethickness}\end{center}

\hypertarget{pancreas-techniques-research-methods}{%
\subsubsection{Pancreas Techniques \& Research
Methods}\label{pancreas-techniques-research-methods}}

\begin{center}\rule{0.5\linewidth}{\linethickness}\end{center}

 - \textbf{Organoidomics - falling star or new galaxy in pancreatic
cancer?}

\emph{Nature reviews. Gastroenterology \& hepatology 2018 Jul;():}

PubMed: \url{https://www.ncbi.nlm.nih.gov/pubmed/?term=30046146}

{}

{}

\begin{center}\rule{0.5\linewidth}{\linethickness}\end{center}

 - \textbf{The Research of Acellular Pancreatic Bioscaffold as a Natural
3-Dimensional Platform In Vitro}

\emph{Pancreas 2018 Sep;47(8):1040-1049}

OBJECTIVE: The aim of the study was to investigate the biochemical and
functional properties of a rat acellular pancreatic bioscaffolds (APBs).
METHODS: Fresh pancreata from 10 rats were soaked and perfused through
portal veins using Easy-Load Digital Drive peristaltic pumps. The
histological structure, extracellular matrix composition, and the DNA
content of the APBs were evaluated. Biocompatibility studies had also
been performed. The proliferation and differentiation of AR42J
pancreatic acinar cells were assessed. RESULTS: The pancreatic tissue
became translucent after decellularization. There were no visible
vascular endothelial cells, cellular components, or cracked cellular
debris. The extracellular matrix components were not decreased after
decellularization (P \textgreater{} 0.05); however, the DNA content was
decreased significantly (P \textless{} 0.05). The subcutaneous
implantation sites showed low immunological response and low
cytotoxicity around the APB. The proliferation rate was higher and the
apoptosis rate was lower when AR42J cells were cultured on APB (P
\textless{} 0.05). The gene expression and the protein expression were
higher for the APB group (P \textless{} 0.001). CONCLUSIONS: Our
findings support the biological utility of whole pancreas APBs as
biomaterial scaffolds, which provides an improved approach for
regenerative medicine.

PubMed: \url{https://www.ncbi.nlm.nih.gov/pubmed/?term=30086100}

{}

{}

\begin{center}\rule{0.5\linewidth}{\linethickness}\end{center}

 - \textbf{Transforming growth factor-β modulates pancreatic cancer
associated fibroblasts cell shape, stiffness and invasion}

\emph{Biochimica et biophysica acta 2018 07;1862(7):1537-1546}

BACKGROUND: Tumor microenvironment consists of the extracellular matrix
(ECM), stromal cells, such as fibroblasts (FBs) and cancer associated
fibroblasts (CAFs), and a myriad of soluble factors. In many tumor
types, including pancreatic tumors, the interplay between stromal cells
and the other tumor microenvironment components leads to desmoplasia, a
cancer-specific type of fibrosis that hinders treatment. Transforming
growth factor beta (TGF-β) and CAFs are thought to play a crucial role
in this tumor desmoplastic reaction, although the involved mechanisms
are unknown. METHODS: Optical/fluorescence microscopy, atomic force
microscopy, image processing techniques, invasion assay in 3D collagen I
gels and real-time PCR were employed to investigate the effect of TGF-β
on normal pancreatic FBs and CAFs with regard to crucial cellular
morphodynamic characteristics and relevant gene expression involved in
tumor progression and metastasis. RESULTS: CAFs present specific
myofibroblast-like characteristics, such as α-smooth muscle actin
expression and cell elongation, they also form more lamellipodia and are
softer than FBs. TGF-β treatment increases cell stiffness (Young's
modulus) of both FBs and CAFs and increases CAF's (but not FB's)
elongation, cell spreading, lamellipodia formation and spheroid
invasion. Gene expression analysis shows that these morphodynamic
characteristics are mediated by Rac, RhoA and ROCK expression in CAFs
treated with TGF-β. CONCLUSIONS: TGF-β modulates CAFs', but not FBs',
cell shape, stiffness and invasion. GENERAL SIGNIFICANCE: Our findings
elucidate on the effects of TGF-β on CAFs' behavior and stiffness
providing new insights into the mechanisms involved.

PubMed: \url{https://www.ncbi.nlm.nih.gov/pubmed/?term=29477748}

{}

{}

\begin{center}\rule{0.5\linewidth}{\linethickness}\end{center}

\hypertarget{pancreas-analogue-tumors}{%
\subsubsection{Pancreas Analogue
Tumors}\label{pancreas-analogue-tumors}}

\begin{center}\rule{0.5\linewidth}{\linethickness}\end{center}

 - \textbf{Intraductal Papillary Mucinous Neoplasms of Minor Salivary
Glands With AKT1 p.Glu17Lys Mutation}

\emph{The American journal of surgical pathology 2018
Aug;42(8):1076-1082}

The spectrum of low-grade intraductal papillary proliferations of the
salivary glands is heterogenous, and reproducible morphologic diagnostic
criteria have not yet been established. Recognized types are sialadenoma
papilliferum, inverted ductal papilloma, and intraductal papilloma, but
some lesions have been possibly included in the morphologic spectrum of
cystadenoma or low-grade intraductal carcinomas. We herein present
detailed morphologic, immunophenotypic, and genotypic features of 3
minor salivary gland neoplasms affecting 2 men (aged 65 and 71 y) and 1
woman (aged 78 y). They ranged in size from 1 to 2.5 cm. All tumors
showed atypical papillary intraductal growth that presented either as
uninodular/unicystic lesions (intraductal papilloma-like; n=2) or as a
discontinuous growth along the ductal system in a manner similar to
pancreatic intraductal papillary mucinous neoplasm (n=1). Variable
cytologic and architectural atypia was observed, ranging from bland
intraductal papilloma-like features, to areas mimicking atypical ductal
hyperplasia and low-grade ductal carcinoma in situ of the breast.
Amplicon-based massive parallel sequencing revealed an identical AKT1
p.Glu17Lys mutation in all 3 cases, but absence of concurring mutations
in other genes of the RAS or PI3K pathway. This small series represents
the first genetic study on salivary intraductal papillary neoplasms. Our
cases showed significant variation in the degree of cytologic and
architectural atypia, which overlaps with intraductal papillomas at the
one end and with low-grade intraductal carcinoma at the other end of the
spectrum, suggesting a disease continuum. As the full biological and
morphologic characteristics of these ductal papillary lesions remain to
be defined, the noncommitted term ``intraductal papillary neoplasms''
might be more appropriate. Our novel genetic findings mirror similar
activating mutations of AKT1 and other PI3K pathway members in
intraductal papillary lesions of the breast and anogenital glands.

PubMed: \url{https://www.ncbi.nlm.nih.gov/pubmed/?term=29738361}

{}

{}

\begin{center}\rule{0.5\linewidth}{\linethickness}\end{center}

 - \textbf{Low-grade intraductal carcinoma of the salivary gland with
prominent oncocytic change: a newly described variant}

\emph{Histopathology 2018 Aug;73(2):314-320}

AIMS: Low-grade intraductal carcinoma (LG-IDC) is a clinically indolent
malignant tumour of the salivary glands. Because of its rarity, the
histological variants of LG-IDC have not been well characterised.
Herein, we describe five LG-IDC cases with prominent oncocytic change in
the major salivary glands. METHODS AND RESULTS: We examined five cases,
three males and two females (mean age~=~63~years), of LG-IDC with
oncocytic change. The sites affected by LG-IDC were the parotid and
submandibular glands. The lesions were macroscopically unilocular or
multilocular cysts with a solid tumour arising from the cyst wall.
Smaller tumour cell nests were also observed. As with classic LG-IDC,
the cyst wall was surrounded by myoepithelial cells with no invasive
component. The tumour cells had abundant oncocytic cytoplasm and
proliferated in a low-papillary, tubular or cribriform pattern.
Immunohistochemically, the tumour cells were diffusely positive for
pan-cytokeratin, S100, mammaglobin and antimitochondria antibody, and
were negative for androgen receptor and gross cystic disease fluid
protein-15. Unlike classic LG-IDC, some of these cases demonstrated
focal invagination of myoepithelial cells in the intraductal tumour.
CONCLUSION: Oncocytic LG-IDC should be recognised as a histologically
unique variant of LG-IDC. Awareness of this entity is important to avoid
erroneous diagnosis and inappropriate treatment for histological mimics.

PubMed: \url{https://www.ncbi.nlm.nih.gov/pubmed/?term=29574881}

{}

{}

\begin{center}\rule{0.5\linewidth}{\linethickness}\end{center}

 - \textbf{Solid pseudopapillary neoplasm (SPN) of the testis:
Comprehensive mutational analysis of 6 testicular and 8 pancreatic SPNs}

\emph{Annals of diagnostic pathology 2018 Aug;35():42-47}

BACKGROUND: Recently, we came with the theory of a possible relationship
between a group of testicular and pancreatic tumors. We used one case of
a pancreatic analogue solid pseudopapillary neoplasm of the testis
composed partially of areas reminiscent of solid pseudopapillary
neoplasm (SPN) of the pancreas and partially of structures identical to
primary signet ring stromal tumor of the testis (PSRSTT) as a connecting
link between these two entities. After demonstrating that PSRSTT and
pancreatic analogue SPN of the testis share the same immunoprofile and
genetic features characteristic for pancreatic SPN, we came to the
conclusion that pancreatic analogue SPN of the testis and PSRSTT
represent a morphological spectrum of a single entity and that both are
related to the pancreatic SPN. DESIGN: The aim of this study is to
present a series of 6 cases of testicular tumors, which lacked the
signet ring cell component and were thus morphologically very similar to
the SPN of the pancreas. The goal of this study is to compare the
genetic background of these testicular tumors that are obviously related
to the PSRSTT/pancreatic analogue SPN of the testis with the series of 8
pancreatic SPN. RESULTS: The mutational analysis revealed an oncogenic
somatic mutation in the exon 3 of the CTNNB1 (β-catenin) gene in all
analyzable (5/6) testicular and all pancreatic (8/8) tumors. The
immunoprofile (positivity with β-catenin, CD10, vimentin, NSE, CD56, and
negativity with inhibin, calretinin, chromogranin) was identical in all
testicular and pancreatic tumors. CONCLUSION: This study expanded the
morphological spectrum of the PSRSTT/pancreatic analogue SPN of the
testis by adding 6 cases without the signet ring cell component.
Considering the obvious analogy of PSRSTT/pancreatic analogue SPN of the
testis/SPN of the testis and their relationship to the pancreatic SPN we
propose the collective term ``solid pseudopapillary neoplasm of the
testis'' for these tumors. The mutational profile of the SPN of the
testis and pancreas was the same in both groups of tumors which we
consider as a final proof that SPN of the testis is identical to the SPN
of the pancreas.

PubMed: \url{https://www.ncbi.nlm.nih.gov/pubmed/?term=29705715}

{}

{}

\begin{center}\rule{0.5\linewidth}{\linethickness}\end{center}

 - \textbf{Pseudo-``solid pseudopapillary neoplasms'' of the testis: in
reality Sertoli cell tumors}

\emph{Human pathology 2018 Aug;():}

PubMed: \url{https://www.ncbi.nlm.nih.gov/pubmed/?term=30110596}

{}

{}

\begin{center}\rule{0.5\linewidth}{\linethickness}\end{center}

 - \textbf{Pseudo-``solid pseudopapillary neoplasms'' of the testis: in
reality Sertoli cell tumors-reply}

\emph{Human pathology 2018 Aug;():}

PubMed: \url{https://www.ncbi.nlm.nih.gov/pubmed/?term=30120970}

{}

{}

\begin{center}\rule{0.5\linewidth}{\linethickness}\end{center}

\hypertarget{animal-studies}{%
\subsubsection{Animal Studies}\label{animal-studies}}

\begin{center}\rule{0.5\linewidth}{\linethickness}\end{center}

 - \textbf{Characterization of Peribiliary Gland-Constituting Cells
Based on Differential Expression of Trophoblast Cell Surface Protein 2
in Biliary Tract}

\emph{The American journal of pathology 2018 Sep;188(9):2059-2073}

Peribiliary glands (PBGs) are accessory glands with mucinous and serous
acini in the biliary tree. The PBG is composed of a heterogeneous cell
population, such as mucus- and pancreatic enzyme-producing epithelial
cells, whereas it constitutes niches for multipotential stem/progenitor
cells in the human extrahepatic bile duct (EHBD). By contrast, the
nature of PBGs in the mouse EHBD remains unclear. Our aim was to
establish a method for isolating and characterizing PBG-constituting
cells in the mouse EHBD. We found that trophoblast cell surface protein
2 (Trop2) was expressed in the luminal epithelium of mouse EHBD
exclusively, but not in the PBG. On the basis of the differential
expression profile of Trop2, lumen-forming biliary epithelial cells
(LBECs) and PBG-constituting epithelial cells (PBECs) were separately
isolated for further characterization. Gene expression analysis revealed
that the isolated mouse PBECs expressed several marker genes related to
human PBGs. In the colony formation assay, PBECs showed significantly
higher colony formation capacity than LBECs. In the organoid formation
assay, PBECs formed cystic organoid with LBEC-like phenotype.
Interestingly, PBECs proliferated, accompanied by reexpression of Trop2
in~vivo after bile duct ligation. Furthermore, the unique expression
profile of Trop2 was conserved in human EHBD. Our findings indicate that
Trop2 is a useful marker in investigating the pathophysiological roles
and characteristics of mouse and human PBGs in biliary diseases.

PubMed: \url{https://www.ncbi.nlm.nih.gov/pubmed/?term=30126547}

{}

{}

\begin{center}\rule{0.5\linewidth}{\linethickness}\end{center}

 - \textbf{Disruption of C1galt1 Gene Promotes Development and
Metastasis of Pancreatic Adenocarcinomas in Mice}

\emph{Gastroenterology 2018 Aug;():}

BACKGROUND \& AIMS: Pancreatic ductal adenocarcinomas (PDACs) produce
higher levels of truncated O-glycan structures (such as Tn and sTn) than
normal pancreata. Dysregulated activity of core 1 synthase,
glycoprotein-N-acetylgalactosamine 3-beta-galactosyltransferase 1
(C1GALT1) leads to increased expression of these truncated O-glycans. We
investigated whether and how truncated O-glycans contributes to
development and progression of PDAC using mice with disruption of
C1galt1. METHODS: We crossed C1galt1 floxed mice (C1galt1loxP/loxP) with
KrasG12D/+; Trp53R172H/+; Pdx1-Cre mice (KPC mice) to create KPCC mice.
Growth and progression of pancreatic tumors were compared between KPC
and KPCC mice; pancreatic tissues were collected and analyzed by
immunohistochemistry; immunofluorescence; and Sirius red, alcian blue,
and lectin staining. We used the CRISPR/Cas9 system to disrupt C1GALT1
in human PDAC cells (T3M4 and CD18/HPAF) and levels of O-glycans were
analyzed by lectin blotting, mass spectrometry and lectin-pull down
assay. Orthotopic studies and RNA sequencing analyses are performed with
control and C1GALT1 knockout PDAC cells. C1GALT1 expression was analyzed
in well differentiated (n=36) and poorly differentiated (n=23) PDAC
samples by immunohistochemistry. RESULTS: KPCC mice had significantly
shorter survival times (median, 102 days) than KPC mice (median, 200
days), and developed early pancreatic intraepithelial neoplasias at 3
weeks, PDAC at 5 weeks, and metastases at 10 weeks compared to KPC.
Pancreatic tumors that developed in KPCC mice were more aggressive than
those of KPC mice (more invasive and metastases), had a decreased amount
of stroma, and had increased production of Tn. Poorly differentiated
PDAC specimens had significantly lower levels of C1GALT1 than
well-differentiated PDACs. Human PDAC cells with knockout of C1GALT1 had
aberrant glycosylation of MUC16 compared with control cells, and
increased expression of genes that regulate tumorigenesis and
metastasis. CONCLUSIONS: In studies of KPC mice with disruption of
C1galt1, we found that loss of C1galt1 promotes development of
aggressive PDACs and increased metastasis. Knockout of C1GALT1 leads to
increased tumorigenicity and truncation of O-glycosylation on MUC16,
which could contribute to increased aggressiveness.

PubMed: \url{https://www.ncbi.nlm.nih.gov/pubmed/?term=30086262}

{}

{}

\begin{center}\rule{0.5\linewidth}{\linethickness}\end{center}

 - \textbf{Local phototherapy synergizes with immunoadjuvant for
treatment of pancreatic cancer through induced immunogenic tumor
vaccine}

\emph{Clinical cancer research : an official journal of the American
Association for Cancer Research 2018 Aug;():}

PURPOSE: To develop a synergistic combination therapy for advanced
pancreatic cancer, using local phototherapy and immunotherapy, and to
determine the efficacy and mechanism of the novel combination therapy
using a highly metastatic pancreatic tumor model in mice. EXPERIMENTAL
DESIGN: Mice bearing Panc02-H7 pancreatic tumors (both subcutaneous and
orthotopic) were treated with non-invasive or interventional
photothermal therapy, followed by local application of an
immunoadjuvant. Tumor growth and animal survival were assessed. Immune
cell populations within spleen and tumors were evaluated by FACS and
IHC, and cytokine levels were determined by ELISA. RESULTS: Up to 75\%
of mice bearing subcutaneous tumors treated with combination therapy had
complete tumor regression. Local photothermal therapy exposed/released
damage-associated molecular patterns, which initiated an immunogenic
tumor cell death, resulting in infiltration of antigen presenting cells
and a T helper 1 (Th1) immunity. Concomitant application of
immunoadjuvant amplified Th1 immunity, especially the tumor-specific
cytotoxic T lymphocytes response, with increased quantity and quality of
T cells. Combination therapy also induced tumor-specific immune memory,
as demonstrated by resistance to tumor rechallenge and production of
memory T cells. For the treatment of orthotopic tumor, the combination
therapy significantly reduced the primary tumors and metastases, and
prolonged the animal survival time. CONCLUSIONS: This study indicated
that combination of local phototherapy and immunotherapy induced a
systemic immunity against established tumors and metastases in an
aggressive, preclinical pancreatic tumor model, leading to a potential
clinical method for patients with advanced pancreatic cancer.

PubMed: \url{https://www.ncbi.nlm.nih.gov/pubmed/?term=30068705}

{}

{}

\begin{center}\rule{0.5\linewidth}{\linethickness}\end{center}

 - \textbf{Systemic Depletion of Nerve Growth Factor Inhibits Disease
Progression in a Genetically Engineered Model of Pancreatic Ductal
Adenocarcinoma}

\emph{Pancreas 2018 Aug;47(7):856-863}

OBJECTIVES: In patients with pancreatic ductal adenocarcinoma (PDAC),
increased expression of proinflammatory neurotrophic growth factors (eg,
nerve growth factor {[}NGF{]}) correlates with a poorer prognosis,
perineural invasion, and, with regard to NGF, pain severity. We
hypothesized that NGF sequestration would reduce inflammation and
disease in the KPC mouse model of PDAC. METHODS: Following biweekly
injections of NGF antibody or control immunoglobulin G, beginning at 4
or 8 weeks of age, inflammation and disease stage were assessed using
histological, protein expression, and quantitative polymerase chain
reaction analyses. RESULTS: In the 8-week anti-NGF group, indicators of
neurogenic inflammation in the dorsal root ganglia (substance P and
calcitonin gene-related peptide) and spinal cord (glial fibrillary
acidic protein) were significantly reduced. In the 4-week anti-NGF
group, TRPA1 mRNA in dorsal root ganglia and spinal phosphorylated ERK
protein were elevated, but glial fibrillary acidic protein expression
was unaffected. In the 8-week anti-NGF group, there was a 40\% reduction
in the proportion of mice with microscopic perineural invasion, and no
macrometastases were observed. CONCLUSIONS: Anti-NGF treatment beginning
at 4 weeks may increase inflammation and negatively impact disease.
Treatment starting at 8 weeks (after disease onset), however, reduces
neural inflammation, neural invasion, and metastasis. These data
indicate that NGF impacts PDAC progression and metastasis in a
temporally dependent manner.

PubMed: \url{https://www.ncbi.nlm.nih.gov/pubmed/?term=29975347}

{}

{}

\begin{center}\rule{0.5\linewidth}{\linethickness}\end{center}

\hypertarget{biliary-tract-3}{%
\subsubsection{Biliary Tract}\label{biliary-tract-3}}

\begin{center}\rule{0.5\linewidth}{\linethickness}\end{center}

 - \textbf{Combined CDK4/6 and pan-mTOR inhibition is synergistic
against intrahepatic cholangiocarcinoma}

\emph{Clinical cancer research : an official journal of the American
Association for Cancer Research 2018 Jul;():}

PURPOSE: Intrahepatic cholangiocarcinoma (ICC) is an aggressive cancer
type, lacking effective therapies and associated with a dismal
prognosis. Palbociclib is a selective CDK4/6 inhibitor, which has been
shown to suppress cell proliferation in many experimental cancer models.
Recently, we demonstrated that pan-mTOR inhibitors, such as MLN0128,
effectively induce apoptosis, while having limited efficacy in
restraining proliferation of ICC cells. Here, we tested the hypothesis
that Palbociclib, due to its ant-proliferative properties in many cancer
types, might synergize with MLN0128 to impair ICC growth. EXPERIMENTAL
DESIGN: Human ICC cell lines and the AKT/YapS127A ICC mouse model were
used to test the therapeutic efficacy of Palbociclib and MLN0128, either
alone or in combination. RESULTS: Administration of Palbociclib
suppressed in vitro ICC cell growth by inhibiting cell cycle
progression. Concomitant administration of Palbociclib and MLN0128 led
to a pronounced, synergistic growth constraint of ICC cell lines.
Furthermore, while treatment with Palbociclib or MLN0128 alone resulted
in tumor growth reduction in AKT/YapS127A mice, a remarkable tumor
regression was achieved when the two drugs were administered
simultaneously. Mechanistically, Palbociclib was found to potentiate
MLN0128 mTOR inhibition activity, whereas MLN0128 prevented the
upregulation of cyclin D1 induced by Pa1bociclib treatment. CONCLUSIONS:
Our study indicates the synergistic activity of Palbociclib and MLN0128
in inhibiting ICC cell proliferation. Thus, combination of CDK4/6 and
mTOR inhibitors might represent a novel, promising, and effective
therapeutic approach against human ICC.

PubMed: \url{https://www.ncbi.nlm.nih.gov/pubmed/?term=30084835}

{}

{}

\begin{center}\rule{0.5\linewidth}{\linethickness}\end{center}

 - \textbf{Preoperative CEA levels are supplementary to CA19-9 levels in
predicting prognosis in patients with resectable intrahepatic
cholangiocarcinoma.}

\url{https://www.researchgate.net/publication/326829300_Preoperative_CEA_levels_are_supplementary_to_CA19-9_levels_in_predicting_prognosis_in_patients_with_resectable_intrahepatic_cholangiocarcinoma}

\begin{center}\rule{0.5\linewidth}{\linethickness}\end{center}

 - \textbf{Improved Survival in Surgically Resected Distal
Cholangiocarcinoma Treated with Adjuvant Therapy: a Propensity Score
Matched Analysis} \emph{Journal of gastrointestinal surgery : official
journal of the Society for Surgery of the Alimentary Tract 2018 Jul;():}

BACKGROUND: Data on the efficacy of adjuvant therapy (AT) in distal
cholangiocarcinoma (dCCA) is limited. This study aimed to determine the
role of AT in resected dCCA and identify subgroups that benefit from AT.
METHODS: We conducted a retrospective review of surgically resected dCCA
in the NCDB from 2004 to 2013. Patients who received AT or observation
(OB) were matched by propensity score. Log-rank test was used to compare
OS. RESULTS: Of 1782 patients with resected dCCA, 840 (47\%) were in the
OB group and 942 (53\%) in the AT group. AT was younger (64.0
vs.~68.7~years, p \textless{} 0.001), had less comorbidities (Charlson
Deyo score 0) (74.6 vs.~68.0\%, p \textless{} 0.001), and more likely to
have private insurance (p \textless{} 0.001). AT was more likely to
present with T3/T4 stage (72 vs.~57\%, p \textless{} 0.001), N1/N2
disease (58 vs.~37\%, p \textless{} 0.001), and positive surgical
margins (26 vs.~16\%, p \textless{} 0.001). After 1:1 propensity score
matching, 500 OB and 500 AT patients were compared. AT was associated
with better OS (HR 0.79; 95\% CI 0.67-0.93). Median OS was 31 and
25~months for the AT and OB (p = 0.006). The 1-, 3-, and 5-year survival
rates were 87, 46, and 31\% for AT; 79, 39, and 24\% for OB. Subgroup
analysis revealed an associated survival advantage for AT in T3/T4
tumors (HR = 0.72; 95\% CI 0.59-0.89), node positive disease (HR 0.70;
95\% CI 0.56-0.87), and positive margins (HR 0.58; 95\% CI 0.42-0.81).
CONCLUSION: AT is associated with improved OS in resected dCCA,
especially in T3/T4 tumors, node positive disease, and positive margins.

PubMed: \url{https://www.ncbi.nlm.nih.gov/pubmed/?term=30030718}

{}

{}

\begin{center}\rule{0.5\linewidth}{\linethickness}\end{center}

 - \textbf{Prognosis and Adherence with the National Comprehensive
Cancer Network Guidelines of Patients with Biliary Tract Cancers: an
Analysis of the National Cancer Database}

\emph{Journal of gastrointestinal surgery : official journal of the
Society for Surgery of the Alimentary Tract 2018 Aug;():}

BACKGROUND: The National Comprehensive Cancer Network (NCCN) guidelines
recommend chemotherapy for patients with inoperable biliary tract
cancers (BTC), as well as patients following resection of BTC with lymph
node metastasis (N1)/positive margins (R1). We sought to define overall
adherence, as well as long-term outcomes, with the NCCN guidelines for
BTC using the National Cancer Database (NCDB). METHODS: A total of
176,536 patients diagnosed with BTC at a hospital participating in the
NCDB between 2004 and 2015 were identified. RESULTS: Among all patients,
63\% of patients received medical therapy (chemotherapy or best
supportive care), 11\% underwent surgical palliation, and 26\% underwent
curative-intent surgery. According to the NCCN guidelines, 86\%
(n = 152,245) of patients were eligible for chemotherapy, yet, only
42.2\% (n = 64,615) received chemotherapy. Factors associated with a
lower adherence with NCCN guidelines included patient age
(\textgreater{} 65~years: OR = 1.02), ethnicity (Black: OR = 1.14,
Hispanic: OR = 1.21, Asian: OR = 1.24), and insurance status
(non-private: OR = 1.45, all p \textless{} 0.001). A smaller subset of
patients was either recommended chemotherapy but refused (n = 9269,
10.6\%) or had medical factors that contraindicated chemotherapy
(n = 8275, 9.4\%). On multivariable analysis, adjusting for clinical and
tumor-specific factors, adherence with NCCN guidelines was associated
with a survival benefit for patients receiving medical therapies
(HR = 0.74) or undergoing curative-intent surgery (HR = 0.73, both
p \textless{} 0.001). CONCLUSION: Less than half of patients with BTC
received systemic chemotherapy in adherence with NCCN guidelines. While
a subset of patients had contraindications or refused chemotherapy,
other factors such as insurance status and ethnicity were associated
with adherence. Adherence with chemotherapy guidelines may influence
long-term outcomes.

PubMed: \url{https://www.ncbi.nlm.nih.gov/pubmed/?term=30112703}

{}

{}

\begin{center}\rule{0.5\linewidth}{\linethickness}\end{center}

\hypertarget{gallbladder-3}{%
\subsubsection{Gallbladder}\label{gallbladder-3}}

\begin{center}\rule{0.5\linewidth}{\linethickness}\end{center}

\begin{itemize}
\tightlist
\item
  \textbf{Incidental Gallbladder Cancer: How Residual Disease Affects
  Outcome in Two Referral HPB Centers from South America}
\end{itemize}

\url{https://link.springer.com/article/10.1007/s00268-018-4762-z}

\begin{center}\rule{0.5\linewidth}{\linethickness}\end{center}

 - \textbf{Gallbladder adenocarcinoma diagnosed from cutaneous
metastases occurring along the tract of a ventriculoperitoneal shunt}

\emph{Journal of cutaneous pathology 2018 Jul;():}

PubMed: \url{https://www.ncbi.nlm.nih.gov/pubmed/?term=30054926}

{}

{}

\begin{center}\rule{0.5\linewidth}{\linethickness}\end{center}

\hypertarget{ampulla-of-vater-3}{%
\subsubsection{Ampulla of Vater}\label{ampulla-of-vater-3}}

\begin{center}\rule{0.5\linewidth}{\linethickness}\end{center}

 - \textbf{Using an endoscopic distal cap to collect pancreatic fluid
from~the ampulla (with video)}

\emph{Gastrointestinal endoscopy 2017 Dec;86(6):1152-1156.e2}

BACKGROUND AND AIMS: Duodenal collections of pancreatic fluid can be
used as a source of mutations and other markers of pancreatic ductal
neoplasia, but admixing pancreatic juice with duodenal contents lowers
the concentrations of mutations. Collecting pancreatic fluid directly
from the ampulla could yield a purer sample of pancreatic fluid.
METHODS: We used an endoscopic distal cap attachment to ``cap'' the
ampulla and collect secretin-stimulated pancreatic fluid samples for 5
minutes from 81 patients undergoing pancreatic evaluation as part of the
Cancer of the Pancreas Screening studies. We compared mutation
concentrations (K-ras and GNAS) measured by droplet-digital PCR (ddPCR)
in ``cap-collected juice'' samples to those found in juice samples
obtained from 77 patients collected by aspiration from the duodenal
lumen without capping the ampulla. RESULTS: Among all subjects, mutation
concentrations were higher in pancreatic juice samples collected using
the endoscopic cap method (median, .028\%; IQR, 0-.077) compared with
the noncap-collected (median, .019\%; IQR, 0-.044; P~= .055). Among
pancreatic juice samples with detectable mutations, mutation
concentrations were higher in the cap-collected juice samples than in
those collected without the cap (.055\%; IQR, .026-.092 vs .032\%; IQR,
.020-.066; P~= .031). CONCLUSIONS: Collecting pancreatic juice directly
from the ampulla using an endoscopic distal cap yields higher
concentrations of pancreatic fluid mutations.

PubMed: \url{https://www.ncbi.nlm.nih.gov/pubmed/?term=28259593}

{}

{}

\begin{center}\rule{0.5\linewidth}{\linethickness}\end{center}

 - \textbf{Outcomes and Treatment Options for Duodenal Adenocarcinoma: A
Systematic Review and Meta-Analysis}

\emph{Annals of surgical oncology 2018 Sep;25(9):2681-2692}

BACKGROUND: Duodenal adenocarcinoma (DA) is a rare tumor for which
survival data per treatment modality and disease stage are unclear. This
systematic review and meta-analysis aims to summarize the current
literature on patient outcome after surgical, (neo)adjuvant, and
palliative treatment in patients with DA. METHODS: A systematic search
was performed according to the preferred reporting items for systematic
reviews and meta-analyses guidelines, to 25 April 2017. Primary outcome
was overall survival (OS), specified for treatment strategy or disease
stage. Random-effects models were used for the calculation of pooled
odds ratios per treatment modality. Included papers were also screened
for prognostic factors. RESULTS: A total of 26 observational studies,
comprising 6438 patients with DA, were included. Of these, resection
with curative intent was performed in 71\% (range 53-100\%) of patients,
and 29\% received palliative treatment (range 0-61\%). The pooled 5-year
OS rate was 46\% after curative resection, compared with 1\% in
palliative-treated patients (OR 0.04, 95\% confidence interval {[}CI{]}
0.02-0.09, p \textless{} 0.0001). Both segmental resection and
pancreaticoduodenectomy allowed adequate assessment of lymph node
involvement and resulted in similar OS. Lymph node involvement
correlated with worse OS (pooled 5-year survival rate 21\% for nodal
metastases vs.~65\% for node-negative disease; OR 0.17, 95\% CI
0.11-0.27, p \textless{} 0.0001). In the current literature, no survival
benefit for adjuvant therapy after curative resection was found.
CONCLUSION: Resection with curative intent, either
pancreaticoduodenectomy or segmental resection, and lack of nodal
metastases, favors survival for DA. Further studies exploring
multimodality (neo)adjuvant therapy are warranted to investigate their
benefit.

PubMed: \url{https://www.ncbi.nlm.nih.gov/pubmed/?term=29946997}

{}

{}

\begin{center}\rule{0.5\linewidth}{\linethickness}\end{center}

 - \textbf{Mixed mucinous adenocarcinoma and somatostatinoma of the
ampulla of Vater associated with neurofibromatosis type 1}

\emph{Pathology 2017 Aug;49(5):553-555}

PubMed: \url{https://www.ncbi.nlm.nih.gov/pubmed/?term=28693748}

{}

{}

\begin{center}\rule{0.5\linewidth}{\linethickness}\end{center}

\hypertarget{pannet-neuroendocrine-3}{%
\subsubsection{PanNET Neuroendocrine}\label{pannet-neuroendocrine-3}}

\begin{center}\rule{0.5\linewidth}{\linethickness}\end{center}

\begin{itemize}
\tightlist
\item
  \textbf{Periampullary neuroendocrine tumor with large intracellular
  mucin globules, an amphicrine tumor?}
\end{itemize}

\url{https://www.sciencedirect.com/science/article/pii/S2214330018300063\#}

\begin{center}\rule{0.5\linewidth}{\linethickness}\end{center}

\begin{itemize}
\tightlist
\item
  \textbf{The Evolving Treatment Algorithm for Advanced Neuroendocrine
  Neoplasms: Diversity and Commonalities Across Tumor Types}
\end{itemize}

\url{http://theoncologist.alphamedpress.org/content/early/2018/08/13/theoncologist.2018-0187.abstract}

\begin{center}\rule{0.5\linewidth}{\linethickness}\end{center}

 - \textbf{Pancreatic islets communicate with lymphoid tissues via
exocytosis of insulin peptides}

\emph{Nature 2018 Aug;560(7716):107-111}

Tissue-specific autoimmunity occurs when selected antigens presented by
susceptible alleles of the major histocompatibility complex are
recognized by T cells. However, the reason why certain specific
self-antigens dominate the response and are indispensable for triggering
autoreactivity is unclear. Spontaneous presentation of insulin is
essential for initiating autoimmune type 1 diabetes in non-obese
diabetic mice1,2. A major set of pathogenic CD4 T cells specifically
recognizes the 12-20 segment of the insulin B-chain (B:12-20), an
epitope that is generated from direct presentation of insulin peptides
by antigen-presenting cells3,4. These T cells do not respond to
antigen-presenting cells that have taken up insulin that, after
processing, leads to presentation of a different segment representing a
one-residue shift, B:13-214. CD4 T cells that recognize B:12-20 escape
negative selection in the thymus and cause diabetes, whereas those that
recognize B:13-21 have only a minor role in autoimmunity3-5. Although
presentation of B:12-20 is evident in the islets3,6, insulin-specific
germinal centres can be formed in various lymphoid tissues, suggesting
that insulin presentation is widespread7,8. Here we use live imaging to
document the distribution of insulin recognition by CD4 T cells
throughout various lymph nodes. Furthermore, we identify catabolized
insulin peptide fragments containing defined pathogenic epitopes in
β-cell granules from mice and humans. Upon glucose challenge, these
fragments are released into the circulation and are recognized by CD4 T
cells, leading to an activation state that results in transcriptional
reprogramming and enhanced diabetogenicity. Therefore, a tissue such as
pancreatic islets, by releasing catabolized products, imposes a constant
threat to self-tolerance. These findings reveal a self-recognition
pathway underlying a primary autoantigen and provide a foundation for
assessing antigenic targets that precipitate pathogenic outcomes by
systemically sensitizing lymphoid tissues.

PubMed: \url{https://www.ncbi.nlm.nih.gov/pubmed/?term=30022165}

{}

{}

\begin{center}\rule{0.5\linewidth}{\linethickness}\end{center}

 - \textbf{Comparison Between Modified Extracellular-Type
Trehalose-Containing Kyoto Solution and University of Wisconsin Solution
in 18-Hour Pancreas Preservation for Islet Transplantation}

\emph{Pancreas 2018 Aug;47(7):e46-e47}

PubMed: \url{https://www.ncbi.nlm.nih.gov/pubmed/?term=29985851}

{}

{}

\begin{center}\rule{0.5\linewidth}{\linethickness}\end{center}

 - \textbf{β-Catenin Expression in Glucagon-Producing α Cells of Human
Fetal Pancreatic Islets Suggests Wnt Signaling-Dependent Development}

\emph{Pancreas 2018 Sep;47(8):e54-e55}

PubMed: \url{https://www.ncbi.nlm.nih.gov/pubmed/?term=30113433}

{}

{}

\begin{center}\rule{0.5\linewidth}{\linethickness}\end{center}

 - \textbf{ASO Author Reflections: Serum Elastase 1 Level as a Risk
Factor for Postoperative Recurrence in Patients with Well-Differentiated
Pancreatic Neuroendocrine Neoplasms}

\emph{Annals of surgical oncology 2018 Aug;():}

PubMed: \url{https://www.ncbi.nlm.nih.gov/pubmed/?term=30136123}

{}

{}

\begin{center}\rule{0.5\linewidth}{\linethickness}\end{center}

\hypertarget{database-seer-ncdb-tcga-oncomine-studies}{%
\subsubsection{Database SEER, NCDB, TCGA, Oncomine
Studies}\label{database-seer-ncdb-tcga-oncomine-studies}}

\begin{center}\rule{0.5\linewidth}{\linethickness}\end{center}

\begin{itemize}
\tightlist
\item
  \textbf{Pancreatic cancer survival analysis defines a signature that
  predicts outcome}
\end{itemize}

\url{http://journals.plos.org/plosone/article?id=10.1371/journal.pone.0201751}

\begin{center}\rule{0.5\linewidth}{\linethickness}\end{center}

 - \textbf{Pancreatic cancer survival analysis defines a signature that
predicts outcome}

\emph{PloS one 2018 ;13(8):e0201751}

Pancreatic ductal adenocarcinoma (PDAC) is the third leading cause of
cancer death in the US. Despite multiple large-scale genetic sequencing
studies, identification of predictors of patient survival remains
challenging. We performed a comprehensive assessment and integrative
analysis of large-scale gene expression datasets, across multiple
platforms, to enable discovery of a prognostic gene signature for
patient survival in pancreatic cancer. PDAC RNA-Sequencing data from The
Cancer Genome Atlas was stratified into Survival+ (\textgreater{}2-year
survival) and Survival-(\textless{}1-year survival) cohorts (n = 47).
Comparisons of RNA expression profiles between survival groups and
normal pancreatic tissue expression data from the Gene Expression
Omnibus generated an initial PDAC specific prognostic differential
expression gene list. The candidate prognostic gene list was then
trained on the Australian pancreatic cancer dataset from the ICGC
database (n = 103), using iterative sampling based algorithms, to derive
a gene signature predictive of patient survival. The gene signature was
validated in 2 independent patient cohorts and against existing PDAC
subtype classifications. We identified 707 candidate prognostic genes
exhibiting differential expression in tumor versus normal tissue. A
substantial fraction of these genes was also found to be differentially
methylated between survival groups. From the candidate gene list, a
5-gene signature (ADM, ASPM, DCBLD2, E2F7, and KRT6A) was identified.
Our signature demonstrated significant power to predict patient survival
in two distinct patient cohorts and was independent of AJCC TNM staging.
Cross-validation of our gene signature reported a better ROC AUC (≥ 0.8)
when compared to existing PDAC survival signatures. Furthermore,
validation of our signature through immunohistochemical analysis of
patient tumor tissue and existing gene expression subtyping data in
PDAC, demonstrated a correlation to the presence of vascular invasion
and the aggressive squamous tumor subtype. Assessment of these genes in
patient biopsies could help further inform risk-stratification and
treatment decisions in pancreatic cancer.

PubMed: \url{https://www.ncbi.nlm.nih.gov/pubmed/?term=30092011}

{}

{}

\begin{center}\rule{0.5\linewidth}{\linethickness}\end{center}

 - \textbf{A simple matrix to predict treatment success and long-term
survival among patients undergoing pancreatectomy}

\emph{HPB : the official journal of the International Hepato Pancreato
Biliary Association 2018 Aug;():}

BACKGROUND: A more accurate measure of long-term survival among patients
who have undergone a successful resection for pancreatic adenocarcinoma
may be computed by accounting for time already survived during the
initial treatment window. METHODS: Patients diagnosed with pancreatic
adenocarcinoma, from 2004 through 2013, were identified from the
American College of Surgeons National Cancer Database (NCDB). A
risk-stratification matrix was constructed including age,
histopathologic factors and the use of adjuvant therapy, given
successful treatment and survival at 3-month following diagnosis.
RESULTS: A total of 25,897 patients (50\% male, 53\%~\textgreater{}65
years of age) presented with stage I-III pancreatic cancer. The majority
of patients had tumors \textgreater{}2~cm size (82\%), grade I/II
(65\%), lymphatic invasion (LI) (66\%), and negative margins (76\%). A
survival advantage for adjuvant therapy was observed among all patients,
independent of their risk-profile. For example, a patient ≤65 years of
age, with early stage cancer (size ≤2~cm, grade I/II,~-ve LI,~-ve
margins) who received adjuvant therapy had a 62\% probability of being
alive beyond three years (95\%CI~=~59\%-66\%). In contrast, the survival
probability decreased to 53\% (95\%CI~=~59\%-66\%) without adjuvant
therapy. CONCLUSIONS: These results provide surgeons and patients with
more accurate information regarding long-term survival, as well as the
benefit of opting for adjuvant therapy after successful pancreatic
surgery.

PubMed: \url{https://www.ncbi.nlm.nih.gov/pubmed/?term=30087052}

{}

{}

\begin{center}\rule{0.5\linewidth}{\linethickness}\end{center}

 - \textbf{Underutilization of Surgery in Periampullary Cancer
Treatment}

\emph{Journal of gastrointestinal surgery : official journal of the
Society for Surgery of the Alimentary Tract 2018 Aug;():}

BACKGROUND: Site-specific outcomes of resection for periampullary cancer
have not been analyzed on a large, registry-based scale. METHODS: We
assessed data on periampullary cancers from the SEER database. Site- and
stage-specific outcomes were analyzed. Resection was compared to no
resection. RESULTS: Resection was the main therapy in stages 1 and 2
(resection vs.~no resection, 8644 vs.~7208 patients), was less frequent
in stage 3 (1248 vs.~2783 patients) and was rarely-but still-used in
stage 4 disease (541 vs.~11,212 patients). Pancreatic head (75.7\%),
11.4\% distal bile duct, 7.7\% ampullary, and 5.3\% duodenal cancers.
Cancer subtype-independent median survival was 22.0 (resection)
vs.~7.0~months (no resection) in stages 1 and 2, 21.0 vs.~8.0~months in
stage 3, and 10.0 vs.~3.0~months in stage 4. Subtype-dependent median
survival (resection vs.~no resection) was 18.0 vs.~5.0~months in
pancreatic head, 19.0 vs 4.0~months in distal bile duct, 41.0 vs
7.0~months in ampullary, and 38.0 vs 4.0~months in duodenal
adenocarcinoma. On multivariable analysis, patient comorbidities,
marital and insurance status, and income all influenced the decision to
undergo resection. CONCLUSIONS: Surgery is still underutilized in the
treatment of periampullary cancers. Patients with cancers originating
from the duodenum or the ampulla of Vater benefit most from resectional
surgery.

PubMed: \url{https://www.ncbi.nlm.nih.gov/pubmed/?term=30088190}

{}

{}

\begin{center}\rule{0.5\linewidth}{\linethickness}\end{center}

 - \textbf{Long-Term Survivors of Pancreatic Cancer: A California
Population-Based Study}

\emph{Pancreas 2018 Sep;47(8):958-966}

OBJECTIVES: Pancreatic cancer continues to carry a poor prognosis with
survival rates that have had minimal improvement over the past 4
decades. We report a population-based, comprehensive analysis of
long-term survivors of pancreatic adenocarcinoma diagnosed in the
diverse population of California. METHODS: Data from the California
Cancer Registry were used to evaluate long-term survival. A total of
70,442 patients diagnosed with pancreatic adenocarcinoma between 1988
and 2009 were identified. Logistic regression was used to identify
factors associated with achieving 5-year survival. RESULTS: The overall
5-year survival was 2.5\%, with minimal incremental improvements
throughout the 3 decades. Age, stage, degree of differentiation, and
surgical resection were associated with 5-year survival. Furthermore,
younger age and receiving care at a National Cancer Institute-designated
cancer center were similarly correlated with 5-year survival regardless
of surgical intervention. In addition, we identified stage,
differentiation, and adjuvant chemotherapy as significant factors for
long-term survival in surgically resected patients. In the unresectable
patients, Asian/Pacific islanders and Hispanics were significantly more
likely to reach the 5-year milestone than non-Hispanic whites.
CONCLUSIONS: Although pancreatic cancer mortality remains high, our
study highlights baseline characteristics, treatment, biological
factors, and ethnicity that are associated with long-term survival.
These findings may serve as a springboard for further investigation.

PubMed: \url{https://www.ncbi.nlm.nih.gov/pubmed/?term=30074526}

{}

{}

\begin{center}\rule{0.5\linewidth}{\linethickness}\end{center}

 - \textbf{Overall survival and cancer-specific survival in patients
with surgically resected pancreatic head adenocarcinoma: A competing
risk nomogram analysis}

\url{https://www.researchgate.net/publication/326829548_Overall_survival_and_cancer-specific_survival_in_patients_with_surgically_resected_pancreatic_head_adenocarcinoma_A_competing_risk_nomogram_analysis}

\begin{center}\rule{0.5\linewidth}{\linethickness}\end{center}

 - \textbf{Race and Health Disparities in Patient Refusal of Surgery for
Early-Stage Pancreatic Cancer: An NCDB Cohort Study}

\emph{Annals of surgical oncology 2018 Jul;():}

AIM: To identify factors associated with refusal of surgery in patients
with early-stage pancreatic cancer and estimate the impact of this
decision on survival. METHODS: Using the National Cancer Data Base,
26,358 patients were identified with potentially resectable tumors
(pretreatment clinical stage I: T1 or T2 N0M0). Multivariate models were
employed to identify factors predicting failure to undergo surgery and
assess the impact on survival. RESULTS: Of early-stage patients who were
recommended surgery, 7.8\% (N = 992) refused surgery for resectable
early-stage pancreatic cancer. On multivariable analysis, patients were
more likely to refuse surgery if they were older {[}odds ratio
(OR) = 1.18; 95\% confidence interval (CI) 1.16-1.19{]}, female
(OR = 1.52; 95\% CI 1.33-1.73), African American (vs White, OR = 1.79;
95\% CI 1.37-2.34), on Medicare/Medicaid (vs private, OR = 2.75; 95\% CI
1.54-4.92) or had higher Charlson-Deyo score (2 vs 0, OR = 1.33; 95\% CI
1.03-1.72). Patients were also significantly more likely to refuse
surgery if they were seen at a center that is not an academic/research
program (OR 1.9; 95\% CI 1.6-2.27). Patients who were recommended
surgery but refused had significantly worse survival than those with
stage I who received surgery {[}median survival 6.8 vs 24~months, Cox
hazard ratio (HR) 3.41; 95\% CI 3.12-3.60{]}. CONCLUSIONS: The
percentage of patients refusing surgery for operable early-stage
pancreatic cancer has been decreasing in the last decade but remains a
significant issue that affects survival. Disparities in refusal of
surgery are independently associated with several variables including
gender, race, and insurance. To mitigate national disparities in
surgical care, future studies should focus on exploring potential
reasons for refusal and developing communication interventions.

PubMed: \url{https://www.ncbi.nlm.nih.gov/pubmed/?term=30043318}

{}

{}

\begin{center}\rule{0.5\linewidth}{\linethickness}\end{center}

 - \textbf{Association Between Hepatitis B Infection and Pancreatic
Cancer: A Population-Based Analysis in the United States}

\emph{Pancreas 2018 Aug;47(7):849-855}

OBJECTIVES: The aim of this study was to assess the role of hepatitis B
(HepB) infection in the causation of pancreatic cancer and the
predictors of pancreatic cancer and mortality. METHODS: We identified
pancreatic cancer patients 11 to 70 years of age from the 2013-2014
National Inpatient Sample. Pearson χ test and Student's t-test were used
for categorical and continuous variables, respectively. We assessed the
association of HepB and pancreatic cancer and the independent mortality
predictors by multivariate analyses. RESULTS: Of 69,210 pancreatic
cancer patients, 175 patients with a history of HepB and 69,035 patients
without a history of HepB were identified. Compared with the pancreatic
cancer-non-HepB group, the pancreatic cancer-HepB group consisted more
of younger (mean, 60.4 {[}standard deviation, 7.4{]} years vs 68.2
{[}standard deviation, 12.1{]} years), male, black, and Asian patients
with low household income and nonelective admissions. The odds of
developing pancreatic cancer among the HepB patients were significantly
higher (adjusted odds ratio, 1.24; 95\% confidence interval,
1.056-1.449; P = 0.008). Black race, age ≥ 65 years, and male sex
demonstrated greater odds of mortality. CONCLUSIONS: This study
concluded up to a 24\% increased likelihood of pancreatic cancer among
the HepB patients. Blacks showed greater odds of pancreatic cancer and
related mortality.

PubMed: \url{https://www.ncbi.nlm.nih.gov/pubmed/?term=29939908}

{}

{}

\begin{center}\rule{0.5\linewidth}{\linethickness}\end{center}

 - \textbf{Impact of Prior Malignancy on Survival Outcomes of Stage IV
Pancreatic Adenocarcinoma: SEER-Based Cohort}

\emph{Journal of gastrointestinal cancer 2018 Aug;():}

PURPOSE: Pancreatic cancer is one of the most fatal malignancies and the
fourth leading cause of cancer-related mortality in the USA. Most
clinical trials involving pancreatic adenocarcinoma (PAC) patients
exclude subjects with a prior malignancy because of the possible effect
of prior malignancies on survival. However, no data in the medical
literature support this assumption. In this paper, we aim to study the
impact of having a prior malignancy on the survival outcomes of stage IV
PAC. METHODS: We used the surveillance, epidemiology, and end results
database to review patients with stage IV PAC diagnosed between 1973 and
2014. We calculated overall and pancreatic cancer-specific survival of
these patients using unadjusted Kaplan-Meier test and multivariable
covariate-adjusted Cox models. RESULTS: We reviewed 66,874 stage IV PAC
patients, of which 4942 had a prior malignancy. Kaplan-Meier and Cox
models showed that a history of prior malignancy did not cause
significant difference in overall survival (HR = 0.938,
95\%CI = 0.880-1.000, p = .052). However, a prior malignancy was
associated with a better pancreatic cancer-specific survival
(HR = 0.855, 95\% CI = 0.796-0.918, p \textless{} .001). CONCLUSION: A
prior malignancy before stage IV PAC was not associated with worse
survival outcomes. Researchers should take these results into
consideration when including/excluding patients to improve the
generalizability and accuracy of their results.

PubMed: \url{https://www.ncbi.nlm.nih.gov/pubmed/?term=30105523}

{}

{}

\begin{center}\rule{0.5\linewidth}{\linethickness}\end{center}

\begin{itemize}
\tightlist
\item
  \textbf{Treatment Outcomes in Patients with Metastatic Neuroendocrine
  Tumors: a Retrospective Analysis of a Community Oncology Database}
\end{itemize}

\url{https://link.springer.com/article/10.1007/s12029-018-0160-x}

\begin{center}\rule{0.5\linewidth}{\linethickness}\end{center}

 - \textbf{Genome-scale analysis to identify prognostic microRNA
biomarkers in patients with early stage pancreatic ductal adenocarcinoma
after pancreaticoduodenectomy}

\emph{Cancer management and research 2018 ;10():2537-2551}

Background: The aim of the study was to investigate potential prognostic
microRNA (miRNA) biomarkers for patients with early stage pancreatic
ductal adenocarcinoma (PDAC) after pancreaticoduodenectomy using a
miRNA-sequencing (miRNA-seq) data set from The Cancer Genome Atlas
(TCGA). A miRNA expression-based prognostic signature was generated, and
the potential role of target genes in overall survival (OS) in patients
with PDAC was examined. Methods: A miRNA-seq data set of 112 PDAC
patients who underwent pancreaticoduodenectomy was obtained from TCGA.
Survival analysis was performed to identify potential prognostic
biomarkers. Results: Eleven miRNAs (hsa-mir-501, hsa-mir-4521,
hsa-mir-5091, hsa-mir-24-1, hsa-mir-126, hsa-mir-30e, hsa-mir-3157,
hsa-let-7a-3, hsa-mir-133a-1, hsa-mir-4709, and hsa-mir-421) were used
to construct a prognostic signature using the step function. The
11-miRNA prognostic signature showed good performance for prognosis
prediction (adjusted P\textless{}0.0001, adjusted hazard ratio =4.285,
95\% confidence interval =2.146-8.554), and the time-dependent receiver
operating characteristic analysis showed an area under the curve of
0.864, 0.877, and 0.787 for 1-, 2-, and 3-year PDAC OS predictions,
respectively. Comprehensive survival analysis suggested that the
prognostic signature could serve as an independent prognostic factor for
PDAC OS and performs better in prognosis prediction than other
traditional clinical indicators. Functional assessment of the target
genes of the miRNAs indicated that they were significantly enriched in
multiple biological processes and pathways, including cell
proliferation, cell cycle biological processes, the forkhead box O,
mitogen-activated protein kinase, Janus kinase/signal transducers and
activators of transcription signaling pathways, pathways in cancer, and
the ErbB signaling pathway. Several target genes of these miRNAs were
also associated with PDAC OS. Conclusion: The present study identified a
novel miRNA expression signature that showed potential as a prognostic
biomarker for PDAC after pancreaticoduodenectomy.

PubMed: \url{https://www.ncbi.nlm.nih.gov/pubmed/?term=30127641}

{}

{}

\begin{center}\rule{0.5\linewidth}{\linethickness}\end{center}

Back to top

\pagebreak

\hypertarget{journals-reviewed}{%
\subsection{Journals Reviewed}\label{journals-reviewed}}

\begin{center}\rule{0.5\linewidth}{\linethickness}\end{center}

Advances in Anatomic Pathology

American Journal of Clinical Pathology

The American Journal of Gastroenterology

The American Journal of Pathology

American Journal of Surgical Pathology

Annals of Diagnostic Pathology

Annals of Surgery

Annals of Surgical Oncology

Annual Review of Pathology-Mechanisms of Disease

APMIS

Applied Immunohistochemistry \& Molecular Morphology

Archives of Pathology \& Laboratory Medicine

Cancer

Cancer Cell

Cancer Cytopathology

Cell

Cellular Oncology

Clinical Cancer Research

Cochrane Database Systematic Reviews

Cytojournal

Cytopathology

Diagnostic Cytopathology

Diagnostic Pathology

Endocrine Pathology

Experimental and Molecular Pathology

Expert Review of Molecular Diagnostics

Gastroenterology

Gut

Histology and Histopathology

Histopathology

Human Pathology

International Journal of Surgical Pathology

International Journal of Clinical and Experimental Pathology

Journal of Clinical Pathology

Journal of Molecular Diagnostics

Journal of Pathology

Laboratory investigation

Lancet

Medical Molecular Morphology

Modern Pathology

Nature

Nature Reviews Gastroenterology \& Hepatology

NEJM

Pancreas

Pancreatology

Pathobiology

Pathologie Biologie

Pathology

Pathology \& Oncology Research

Pathology International

Pathology Research and Practice

PNAS

Science

Seminars in Diagnostic Pathology

Seminars in Immunopathology

Surgical pathology clinics

Tissue Antigens

Trends in Cancer

Virchows Archiv

\begin{center}\rule{0.5\linewidth}{\linethickness}\end{center}

Back to top

\pagebreak

\hypertarget{feedback}{%
\subsection{Feedback}\label{feedback}}

\begin{center}\rule{0.5\linewidth}{\linethickness}\end{center}

Please send your feedbacks using the form below:

\href{https://docs.google.com/forms/d/e/1FAIpQLSeD3Z9J6Y7eMmiyM12f_SfAmHUlykb1zxZcwO6lg7cebGYQIQ/viewform}{Click
here for the Feedback Form}

\begin{center}\rule{0.5\linewidth}{\linethickness}\end{center}

Back to top

\pagebreak

\hypertarget{archive}{%
\subsection{Archive}\label{archive}}

\begin{center}\rule{0.5\linewidth}{\linethickness}\end{center}

\textbf{The PBPath Journal Archive}

\begin{itemize}
\item
  \href{http://pbpath.org/journal-watch/}{Current Issue}
\item
  \href{http://pbpath.org/journal-watch-archive/}{Older Issues}

  \begin{itemize}
  \item
    \href{http://pbpath.org/pbpath-journal-watch-june-july-2018/}{June-July-2018}
  \item
    October - November issue is being prepared. Unofficial version is
    \href{http://pbpath.org/journal-watch-upcoming-issue/}{here}.
  \end{itemize}
\end{itemize}

\begin{center}\rule{0.5\linewidth}{\linethickness}\end{center}

Back to top


\end{document}
