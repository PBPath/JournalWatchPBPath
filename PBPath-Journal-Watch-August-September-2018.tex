\documentclass[]{article}
\usepackage{lmodern}
\usepackage{amssymb,amsmath}
\usepackage{ifxetex,ifluatex}
\usepackage{fixltx2e} % provides \textsubscript
\ifnum 0\ifxetex 1\fi\ifluatex 1\fi=0 % if pdftex
  \usepackage[T1]{fontenc}
  \usepackage[utf8]{inputenc}
\else % if luatex or xelatex
  \ifxetex
    \usepackage{mathspec}
  \else
    \usepackage{fontspec}
  \fi
  \defaultfontfeatures{Ligatures=TeX,Scale=MatchLowercase}
\fi
% use upquote if available, for straight quotes in verbatim environments
\IfFileExists{upquote.sty}{\usepackage{upquote}}{}
% use microtype if available
\IfFileExists{microtype.sty}{%
\usepackage{microtype}
\UseMicrotypeSet[protrusion]{basicmath} % disable protrusion for tt fonts
}{}
\usepackage[margin=1in]{geometry}
\usepackage{hyperref}
\hypersetup{unicode=true,
            pdftitle={PBPath Journal Watch},
            pdfborder={0 0 0},
            breaklinks=true}
\urlstyle{same}  % don't use monospace font for urls
\usepackage{graphicx,grffile}
\makeatletter
\def\maxwidth{\ifdim\Gin@nat@width>\linewidth\linewidth\else\Gin@nat@width\fi}
\def\maxheight{\ifdim\Gin@nat@height>\textheight\textheight\else\Gin@nat@height\fi}
\makeatother
% Scale images if necessary, so that they will not overflow the page
% margins by default, and it is still possible to overwrite the defaults
% using explicit options in \includegraphics[width, height, ...]{}
\setkeys{Gin}{width=\maxwidth,height=\maxheight,keepaspectratio}
\IfFileExists{parskip.sty}{%
\usepackage{parskip}
}{% else
\setlength{\parindent}{0pt}
\setlength{\parskip}{6pt plus 2pt minus 1pt}
}
\setlength{\emergencystretch}{3em}  % prevent overfull lines
\providecommand{\tightlist}{%
  \setlength{\itemsep}{0pt}\setlength{\parskip}{0pt}}
\setcounter{secnumdepth}{0}
% Redefines (sub)paragraphs to behave more like sections
\ifx\paragraph\undefined\else
\let\oldparagraph\paragraph
\renewcommand{\paragraph}[1]{\oldparagraph{#1}\mbox{}}
\fi
\ifx\subparagraph\undefined\else
\let\oldsubparagraph\subparagraph
\renewcommand{\subparagraph}[1]{\oldsubparagraph{#1}\mbox{}}
\fi

%%% Use protect on footnotes to avoid problems with footnotes in titles
\let\rmarkdownfootnote\footnote%
\def\footnote{\protect\rmarkdownfootnote}

%%% Change title format to be more compact
\usepackage{titling}

% Create subtitle command for use in maketitle
\newcommand{\subtitle}[1]{
  \posttitle{
    \begin{center}\large#1\end{center}
    }
}

\setlength{\droptitle}{-2em}

  \title{PBPath Journal Watch}
    \pretitle{\vspace{\droptitle}\centering\huge}
  \posttitle{\par}
  \subtitle{August September 2018}
  \author{}
    \preauthor{}\postauthor{}
      \predate{\centering\large\emph}
  \postdate{\par}
    \date{Last Update on 2018-07-31}


\begin{document}
\maketitle

{
\setcounter{tocdepth}{2}
\tableofcontents
}
\href{http://pbpath.org/}{\includegraphics{figures/PBP_header_logo.jpg}}

\begin{center}\rule{0.5\linewidth}{\linethickness}\end{center}

\hypertarget{surgical-pathology}{%
\subsection{Surgical Pathology}\label{surgical-pathology}}

\begin{center}\rule{0.5\linewidth}{\linethickness}\end{center}

\begin{center}\rule{0.5\linewidth}{\linethickness}\end{center}

\begin{center}\rule{0.5\linewidth}{\linethickness}\end{center}

\begin{itemize}
\tightlist
\item
  \textbf{Interleukin-33 overexpression reflects less aggressive tumour
  features in large-duct type cholangiocarcinomas}
\end{itemize}

\emph{Histopathology 2018 Aug;73(2):259-272}

AIMS: The aim of the present study was to elucidate the
clinicopathological significance of interleukin (IL)-6 and IL-33
expression in intrahepatic cholangiocarcinomas (iCCAs) and perihilar
cholangiocarcinomas (pCCAs). METHODS AND RESULTS: IL-6 and IL-33 mRNA
expression levels were examined in iCCAs (n = 55) and pCCAs (n = 32) by
the use of quantitative real-time polymerase chain reaction and a highly
sensitive in-situ hybridisation protocol (RNAscope), and expression
levels were correlated with clinicopathological features. According to a
recently proposed classification scheme, iCCAs were separated into
small-duct (n = 33) and large-duct (n = 22) types. IL-6 and IL-33
expression levels were higher in large-duct iCCAs and pCCAs than in
small-duct iCCAs, and there was a positive correlation between the
expression levels of these cytokines. Double in-situ
hybridisation/immunostaining showed that IL-6 mRNA was expressed in
actin-positive (myo)fibroblasts, whereas IL-33 mRNA was mainly produced
by CD31-positive endothelial cells. With the average expression level as
a cut-off point, cases were classified as IL-6high and IL-6low or
IL-33high and IL-33low . In the combined cohort of large-duct iCCAs and
pCCAs, IL-6high and IL-6low cholangiocarcinomas shared many features,
whereas IL-33high cases had less aggressive characteristics than
IL-33low cases, as shown by lower tumour marker concentrations, smaller
tumour sizes, less common vascular invasion, lower pT stages, and higher
lymphocyte/monocyte ratios in blood. KRAS mutations were slightly less
common in IL-33high cases than in IL-33low cases (9\% versus 29\%; P =
0.061). The strong expression of IL-33 in tissue appeared to be an
independent favourable prognostic factor. CONCLUSIONS: IL-33high
cholangiocarcinomas may represent a unique, less aggressive
carcinogenetic process of the large bile ducts.

PubMed: \url{https://www.ncbi.nlm.nih.gov/pubmed/?term=29675965}

doi: \url{https://doi.org/10.1111/his.13633}

{}

{}

\begin{center}\rule{0.5\linewidth}{\linethickness}\end{center}

\begin{itemize}
\tightlist
\item
  \textbf{Postradiation Synovial Sarcoma of the Common Bile Duct: A
  Previously Unreported Anatomic Site}
\end{itemize}

\emph{International journal of surgical pathology 2018
Aug;26(5):469-474}

Synovial sarcoma is a ubiquitous neoplasm predominantly affecting soft
tissues of young adults of any gender; few cases have been described in
the digestive system, mostly in the stomach. The (X;18)(p11.2; q11.2)
translocation yields unique SS18-SSX fusion genes. Synovial sarcoma has
been related to radiotherapy, but no synovial sarcoma has been
associated with the digestive system. This article describes the case of
a synovial sarcoma arising along the extrahepatic biliary tree, 10 years
after the application of an abdominal radiotherapy schedule due to a
retroperitoneal metastatic seminoma in a male who developed progressive
obstructive jaundice. Ninety percent of the analyzed cells carried the
SS18 gene with separation of sequences, thus denoting a translocation.
There are only 8 post-radiotherapy synovial sarcomas that have been
reported previously, and this is the first report of a
radiotherapy-related synovial sarcoma arising from the extrahepatic
biliary tree, and the second case described in this anatomic region.

PubMed: \url{https://www.ncbi.nlm.nih.gov/pubmed/?term=29336183}

doi: \url{https://doi.org/10.1177/1066896917752863}

{}

{}

\begin{center}\rule{0.5\linewidth}{\linethickness}\end{center}

\begin{center}\rule{0.5\linewidth}{\linethickness}\end{center}

\begin{center}\rule{0.5\linewidth}{\linethickness}\end{center}

\begin{center}\rule{0.5\linewidth}{\linethickness}\end{center}

\begin{itemize}
\tightlist
\item
  \textbf{Intraobserver and Interobserver Variability in the Assessment
  of Dysplasia in Ampullary Mucosal Biopsies}
\end{itemize}

\emph{The American journal of surgical pathology 2018
Aug;42(8):1095-1100}

Endoscopic mucosal biopsies of the ampulla of Vater (AmpBx) are obtained
to histologically assess for dysplasia or carcinoma. However, biopsy
material is often scant and a host of factors can induce histologic
changes that pose diagnostic challenges. We sought to investigate
observer variability in interpretation of AmpBx and the impact clinical
data may have on diagnostic interpretation. Thirty-one cases from
institutional archives were selected, including 12 cases of reactive
atypia (RA), 8 indefinite for dysplasia (ID), and 11 showing low-grade
dysplasia (LGD). Slides were independently reviewed at 3 time points
with and without clinical information by 6 pathologists who categorized
the biopsies RA, ID, or LGD. Following the reviews, intraobserver and
interobserver agreement was assessed. Review of AmpBx without clinical
data showed fair (κ, 0.27), poor (κ, 0.07), and good (κ, 0.42)
interobserver agreement for diagnoses of RA, ID, and LGD, respectively.
Interobserver agreement improved for LGD (κ, 0.66 and 0.73) when
clinical information was provided; however, agreement remained fair for
RA (κ, 0.4 and 0.42) and poor-to-fair for ID (κ, 0.17 and 0.25). When
follow-up data were reviewed, all cases that reached unanimous agreement
had that diagnosis substantiated by subsequent endoscopic or histologic
findings. The same was true of 13 of 19 cases that reached majority
consensus. Given the potential clinical consequences of these diagnoses
combined with the significant intraobserver and interobserver
variability found in this study, we conclude that better-defined
diagnostic criteria and consensus reads on difficult cases would assist
in the histologic assessment of these challenging cases.

PubMed: \url{https://www.ncbi.nlm.nih.gov/pubmed/?term=29738360}

doi: \url{https://doi.org/10.1097/PAS.0000000000001079}

{}

{}

\begin{center}\rule{0.5\linewidth}{\linethickness}\end{center}

\hypertarget{cytopathology}{%
\subsection{Cytopathology}\label{cytopathology}}

\begin{center}\rule{0.5\linewidth}{\linethickness}\end{center}

\hypertarget{molecular-pathology}{%
\subsection{Molecular Pathology}\label{molecular-pathology}}

\begin{center}\rule{0.5\linewidth}{\linethickness}\end{center}

\begin{itemize}
\tightlist
\item
  \textbf{Liquid Biopsies for Management of Pancreatic Cancer}
\end{itemize}

\url{https://www.sciencedirect.com/science/article/pii/S1931524418301099}

\begin{center}\rule{0.5\linewidth}{\linethickness}\end{center}

\hypertarget{journals-that-are-followed}{%
\subsection{Journals That Are
Followed}\label{journals-that-are-followed}}

Advances in Anatomic Pathology

Am J Clin Path

Am J Gastroenterology

Am J Pathol

American Journal Of Clinical Pathology

American Journal Of Pathology

American Journal Of Surgical Pathology

Annals Of Diagnostic Pathology

Annals Of Surgery

Annals Of Surgical Oncology

Annual Review Of Pathology-Mechanisms Of Disease

APMIS

Applied immunohistochemistry \& Molecular Morphology

Archives of Pathology \& Laboratory Medicine

Cancer

Cancer Cell

Cancer Cytopathology

Cellular Oncology

Clinical Cancer Research

Cochrane Reviews

Cytojournal

Cytopathology

Diagnostic Cytopathology

Diagnostic Pathology

Endocrine Pathology

Experimental And Molecular Pathology

Expert Review Of Molecular Diagnostics

Gastroenterology

Gut

Histology And Histopathology

Histopathology

Human Pathology

International Journal Of Surgical Pathology

International Journal Of Clinical And Experimental Pathology

Journal Of Clinical Pathology

Journal Of Molecular Diagnostics

Journal Of Pathology

Laboratory investigation

Lancet

Medical Molecular Morphology

Modern Pathology

Nature

NEJM

Pancreas

Pancreatology

Pathobiology

Pathologie Biologie

Pathology

Pathology \& Oncology Research

Pathology International

Pathology Research And Practice

PNAS

Science

Seminars in Diagnostic Pathology

Seminars in Immunopathology

The Journal Of Pathology

Tissue Antigens

Virchows Archiv


\end{document}
