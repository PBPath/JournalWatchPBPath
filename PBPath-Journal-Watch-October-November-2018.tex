\documentclass[]{article}
\usepackage{lmodern}
\usepackage{amssymb,amsmath}
\usepackage{ifxetex,ifluatex}
\usepackage{fixltx2e} % provides \textsubscript
\ifnum 0\ifxetex 1\fi\ifluatex 1\fi=0 % if pdftex
  \usepackage[T1]{fontenc}
  \usepackage[utf8]{inputenc}
\else % if luatex or xelatex
  \ifxetex
    \usepackage{mathspec}
  \else
    \usepackage{fontspec}
  \fi
  \defaultfontfeatures{Ligatures=TeX,Scale=MatchLowercase}
\fi
% use upquote if available, for straight quotes in verbatim environments
\IfFileExists{upquote.sty}{\usepackage{upquote}}{}
% use microtype if available
\IfFileExists{microtype.sty}{%
\usepackage{microtype}
\UseMicrotypeSet[protrusion]{basicmath} % disable protrusion for tt fonts
}{}
\usepackage[margin=1in]{geometry}
\usepackage{hyperref}
\hypersetup{unicode=true,
            pdftitle={Pancreatobiliary Pathology Society Journal Watch},
            pdfborder={0 0 0},
            breaklinks=true}
\urlstyle{same}  % don't use monospace font for urls
\usepackage{graphicx,grffile}
\makeatletter
\def\maxwidth{\ifdim\Gin@nat@width>\linewidth\linewidth\else\Gin@nat@width\fi}
\def\maxheight{\ifdim\Gin@nat@height>\textheight\textheight\else\Gin@nat@height\fi}
\makeatother
% Scale images if necessary, so that they will not overflow the page
% margins by default, and it is still possible to overwrite the defaults
% using explicit options in \includegraphics[width, height, ...]{}
\setkeys{Gin}{width=\maxwidth,height=\maxheight,keepaspectratio}
\IfFileExists{parskip.sty}{%
\usepackage{parskip}
}{% else
\setlength{\parindent}{0pt}
\setlength{\parskip}{6pt plus 2pt minus 1pt}
}
\setlength{\emergencystretch}{3em}  % prevent overfull lines
\providecommand{\tightlist}{%
  \setlength{\itemsep}{0pt}\setlength{\parskip}{0pt}}
\setcounter{secnumdepth}{0}
% Redefines (sub)paragraphs to behave more like sections
\ifx\paragraph\undefined\else
\let\oldparagraph\paragraph
\renewcommand{\paragraph}[1]{\oldparagraph{#1}\mbox{}}
\fi
\ifx\subparagraph\undefined\else
\let\oldsubparagraph\subparagraph
\renewcommand{\subparagraph}[1]{\oldsubparagraph{#1}\mbox{}}
\fi

%%% Use protect on footnotes to avoid problems with footnotes in titles
\let\rmarkdownfootnote\footnote%
\def\footnote{\protect\rmarkdownfootnote}

%%% Change title format to be more compact
\usepackage{titling}

% Create subtitle command for use in maketitle
\newcommand{\subtitle}[1]{
  \posttitle{
    \begin{center}\large#1\end{center}
    }
}

\setlength{\droptitle}{-2em}

  \title{Pancreatobiliary Pathology Society Journal Watch}
    \pretitle{\vspace{\droptitle}\centering\huge}
  \posttitle{\par}
  \subtitle{October November 2018}
  \author{}
    \preauthor{}\postauthor{}
      \predate{\centering\large\emph}
  \postdate{\par}
    \date{Last Update on 2018-10-05}


\begin{document}
\maketitle

{
\setcounter{tocdepth}{5}
\tableofcontents
}
\begin{center}\rule{0.5\linewidth}{\linethickness}\end{center}

\hypertarget{the-current-pbpath-journal-watch-articles}{%
\section{The Current PBPath Journal Watch
Articles}\label{the-current-pbpath-journal-watch-articles}}

\textbf{Wellcome to our journal watch for pancreatobiliary pathology
articles, which is released every other month. You may find the previous
issues in the \href{http://pbpath.org/journal-watch-archive/}{archive}.}

\emph{We have created several categories for convenience; however,
articles in each category are in no particular order.}

\emph{Please feel free to
\href{https://docs.google.com/forms/d/e/1FAIpQLSeD3Z9J6Y7eMmiyM12f_SfAmHUlykb1zxZcwO6lg7cebGYQIQ/viewform}{fill
out our feedback form}. You may also recommend articles to be included.}

\pagebreak

\begin{center}\rule{0.5\linewidth}{\linethickness}\end{center}

\hypertarget{surgical-pathology}{%
\subsection{Surgical Pathology}\label{surgical-pathology}}

\begin{center}\rule{0.5\linewidth}{\linethickness}\end{center}

\hypertarget{pancreas}{%
\subsubsection{Pancreas}\label{pancreas}}

\begin{center}\rule{0.5\linewidth}{\linethickness}\end{center}

 \textbf{Rab14 overexpression regulates gemcitabine sensitivity through
regulation of Bcl-2 and mitochondrial function in pancreatic cancer}

\href{https://www.ncbi.nlm.nih.gov/pubmed/?term=30267303}{Link to full
abstract}

\emph{Virchows Archiv : an international journal of pathology 2018
Sep;():}

Rab family protein Rab14 has been implicated in the development of human
cancers. To date, its expression pattern, biological function, and
potential mechanism in pancreatic cancer have not been explored. In this
study, we analyzed Rab14 expression in 103 cases of pancreatic cancer
tissues using immunohistochemistry (IHC) and found that Rab14 was
overexpressed in 41/103 cases (39.8\%). Rab14 overexpression correlated
with the advanced stage. Moreover, elevated Rab14 levels indicated poor
prognosis of patients with pancreatic cancers. We used BxPC-3 and
Capan-2 respectively for plasmid and siRNA transfection. MTT and colony
formation assays showed that Rab14 transfection increased cell
proliferation and colony formation in BxPC-3 cells. Rab14 siRNA
knockdown inhibits proliferation and colony formation ability in Capan-2
cell line. Cell cycle analysis showed that Rab14 facilitated cell cycle
progression. Matrigel invasion assay showed that Rab14 promoted BxPC-3
cell invasion while its depletion inhibited Capan-2 cell invasion. In
addition, MTT and AnnexinV/PI analysis demonstrated that overexpression
of Rab14 reduced gemcitabine sensitivity which conversely was increased
by Rab14 knockdown. We also demonstrated that Rab14 upregulated
mitochondrial membrane potential (MMP) while its depletion downregulated
MMP during gemcitabine treatment. In addition, western blotting revealed
that Rab14 overexpression upregulated cyclin D1, cyclin A, cyclin E,
p-Rb, and Bcl-2 and downregulated p21. Rab14 also downregulated
caspase3, PARP cleavage, and cytochrome c release. In conclusion, our
data indicated that Rab14 was overexpressed in pancreatic cancer and
promotes growth and gemcitabine resistance, possibly through regulation
of mitochondrial function and Bcl-2.

{}

{}

\begin{center}\rule{0.5\linewidth}{\linethickness}\end{center}

 2 \textbf{Rab14 overexpression regulates gemcitabine sensitivity
through regulation of Bcl-2 and mitochondrial function in pancreatic
cancer}

\emph{Virchows Archiv : an international journal of pathology 2018
Sep;():}

Rab family protein Rab14 has been implicated in the development of human
cancers. To date, its expression pattern, biological function, and
potential mechanism in pancreatic cancer have not been explored. In this
study, we analyzed Rab14 expression in 103 cases of pancreatic cancer
tissues using immunohistochemistry (IHC) and found that Rab14 was
overexpressed in 41/103 cases (39.8\%). Rab14 overexpression correlated
with the advanced stage. Moreover, elevated Rab14 levels indicated poor
prognosis of patients with pancreatic cancers. We used BxPC-3 and
Capan-2 respectively for plasmid and siRNA transfection. MTT and colony
formation assays showed that Rab14 transfection increased cell
proliferation and colony formation in BxPC-3 cells. Rab14 siRNA
knockdown inhibits proliferation and colony formation ability in Capan-2
cell line. Cell cycle analysis showed that Rab14 facilitated cell cycle
progression. Matrigel invasion assay showed that Rab14 promoted BxPC-3
cell invasion while its depletion inhibited Capan-2 cell invasion. In
addition, MTT and AnnexinV/PI analysis demonstrated that overexpression
of Rab14 reduced gemcitabine sensitivity which conversely was increased
by Rab14 knockdown. We also demonstrated that Rab14 upregulated
mitochondrial membrane potential (MMP) while its depletion downregulated
MMP during gemcitabine treatment. In addition, western blotting revealed
that Rab14 overexpression upregulated cyclin D1, cyclin A, cyclin E,
p-Rb, and Bcl-2 and downregulated p21. Rab14 also downregulated
caspase3, PARP cleavage, and cytochrome c release. In conclusion, our
data indicated that Rab14 was overexpressed in pancreatic cancer and
promotes growth and gemcitabine resistance, possibly through regulation
of mitochondrial function and Bcl-2.

PubMed: \url{https://www.ncbi.nlm.nih.gov/pubmed/?term=30267303}

{}

{}

\begin{center}\rule{0.5\linewidth}{\linethickness}\end{center}

3 \textbf{Rab14 overexpression regulates gemcitabine sensitivity through
regulation of Bcl-2 and mitochondrial function in pancreatic cancer}

\emph{Virchows Archiv : an international journal of pathology 2018
Sep;():}

Rab family protein Rab14 has been implicated in the development of human
cancers. To date, its expression pattern, biological function, and
potential mechanism in pancreatic cancer have not been explored. In this
study, we analyzed Rab14 expression in 103 cases of pancreatic cancer
tissues using immunohistochemistry (IHC) and found that Rab14 was
overexpressed in 41/103 cases (39.8\%). Rab14 overexpression correlated
with the advanced stage. Moreover, elevated Rab14 levels indicated poor
prognosis of patients with pancreatic cancers. We used BxPC-3 and
Capan-2 respectively for plasmid and siRNA transfection. MTT and colony
formation assays showed that Rab14 transfection increased cell
proliferation and colony formation in BxPC-3 cells. Rab14 siRNA
knockdown inhibits proliferation and colony formation ability in Capan-2
cell line. Cell cycle analysis showed that Rab14 facilitated cell cycle
progression. Matrigel invasion assay showed that Rab14 promoted BxPC-3
cell invasion while its depletion inhibited Capan-2 cell invasion. In
addition, MTT and AnnexinV/PI analysis demonstrated that overexpression
of Rab14 reduced gemcitabine sensitivity which conversely was increased
by Rab14 knockdown. We also demonstrated that Rab14 upregulated
mitochondrial membrane potential (MMP) while its depletion downregulated
MMP during gemcitabine treatment. In addition, western blotting revealed
that Rab14 overexpression upregulated cyclin D1, cyclin A, cyclin E,
p-Rb, and Bcl-2 and downregulated p21. Rab14 also downregulated
caspase3, PARP cleavage, and cytochrome c release. In conclusion, our
data indicated that Rab14 was overexpressed in pancreatic cancer and
promotes growth and gemcitabine resistance, possibly through regulation
of mitochondrial function and Bcl-2.

PubMed: \url{https://www.ncbi.nlm.nih.gov/pubmed/?term=30267303}

{}

{}

\begin{center}\rule{0.5\linewidth}{\linethickness}\end{center}

 - \textbf{CD138/syndecan-1 in pancreatic solid and pseudopapillary
neoplasms}

\emph{Journal of clinical pathology 2018 Oct;():}

PubMed: \url{https://www.ncbi.nlm.nih.gov/pubmed/?term=30275097}

{}

{}

\begin{center}\rule{0.5\linewidth}{\linethickness}\end{center}

\begin{center}\rule{0.5\linewidth}{\linethickness}\end{center}

\begin{center}\rule{0.5\linewidth}{\linethickness}\end{center}

\hypertarget{neuroendocrine}{%
\subsubsection{Neuroendocrine}\label{neuroendocrine}}

\begin{center}\rule{0.5\linewidth}{\linethickness}\end{center}

\begin{itemize}
\tightlist
\item
  \textbf{The use of Ki-67 labeling index to grade pulmonary
  well-differentiated neuroendocrine neoplasms: current best evidence}
\end{itemize}

\url{https://www.nature.com/articles/s41379-018-0076-9}

\begin{center}\rule{0.5\linewidth}{\linethickness}\end{center}

Back to top

\pagebreak

\hypertarget{cytopathology}{%
\subsection{Cytopathology}\label{cytopathology}}

\begin{center}\rule{0.5\linewidth}{\linethickness}\end{center}

\hypertarget{pancreas-1}{%
\subsubsection{Pancreas}\label{pancreas-1}}

\begin{center}\rule{0.5\linewidth}{\linethickness}\end{center}

 - \textbf{Rationale and feasibility of mucin expression profiling by
qRT-PCR as diagnostic biomarkers in cytology specimens of pancreatic
cancer}

\emph{Pancreatology : official journal of the International Association
of Pancreatology (IAP) \ldots{} {[}et al.{]} 2018 Sep;():}

BACKGROUND: Aberrantly expressed mucin glycoproteins (MUC) play
important roles in pancreatic ductal adenocarcinoma (PDAC), yet their
use as a diagnostic aid in fine-needle aspiration biopsy (FNAB) is
poorly documented. The aim of this study was to investigate the
rationale and feasibility of mucin (MUC1, MUC2, MUC3, MUC4, MUC5AC, and
MUC6) expression profiling by RT-PCR for diagnostic applications in
cytology. METHODS: Mucin expression was examined by RT-PCR and
immunohistochemistry in specimens resected from patients with pancreatic
(n\,=\,101), ampullary (n\,=\,23), and common bile duct (n\,=\,10)
cancers and 33 with chronic pancreatitis. Furthermore, mucin profiling
by RT-PCR was prospectively compared in surgical and biopsy specimens of
40 patients with pancreatic solid tumours qualified for FNAB prior to
surgery. RESULTS: A logistic regression model to distinguish PDAC from
chronic pancreatitis using RT-PCR profiling included MUC3, MUC5AC, and
MUC6. The same set of mucins differentiated ampullary and bile duct
cancers from chronic pancreatitis. AUCs for the ROC curves derived from
the two models were 0.95 (95\%CI 0.87-0.99) and 0.92 (95\%CI 0.81-0.98),
respectively. The corresponding positive likelihood ratios were 6.02 and
5.97, while the negative likelihood ratios were 0.10 and 0.12. AUCs of
ROC curves obtained by RT-PCR and immunohistochemistry demonstrated that
both analytical methods were comparable. Surgical and cytological
samples showed significantly correlated values of ΔCt for individual
mucins with the overall Pearson's correlation coefficient r\,=\,0.841
(P\,=\,0.001). CONCLUSIONS: Mucin expression profiling of pancreatic
cancer with RT-PCR is feasible and may be a valuable help in
discriminating malignant lesions from chronic pancreatitis in FNAB
cytology.

PubMed: \url{https://www.ncbi.nlm.nih.gov/pubmed/?term=30268674}

{}

{}

\begin{center}\rule{0.5\linewidth}{\linethickness}\end{center}

\begin{itemize}
\tightlist
\item
  \textbf{Rationale and feasibility of mucin expression profiling by
  qRT-PCR as diagnostic biomarkers in cytology specimens of pancreatic
  cancer}
\end{itemize}

\url{https://www.sciencedirect.com/science/article/pii/S1424390318306859}

\begin{center}\rule{0.5\linewidth}{\linethickness}\end{center}

\begin{center}\rule{0.5\linewidth}{\linethickness}\end{center}

\begin{center}\rule{0.5\linewidth}{\linethickness}\end{center}

\begin{center}\rule{0.5\linewidth}{\linethickness}\end{center}

Back to top

\pagebreak

\hypertarget{molecular-pathology}{%
\subsection{Molecular Pathology}\label{molecular-pathology}}

\begin{center}\rule{0.5\linewidth}{\linethickness}\end{center}

\hypertarget{pancreas-2}{%
\subsubsection{Pancreas}\label{pancreas-2}}

\begin{itemize}
\tightlist
\item
  \textbf{Referral frequency, attrition rate, and outcomes of germline
  testing in patients with pancreatic adenocarcinoma}
\end{itemize}

\url{https://link.springer.com/article/10.1007/s10689-018-0106-2}

\begin{center}\rule{0.5\linewidth}{\linethickness}\end{center}

\begin{center}\rule{0.5\linewidth}{\linethickness}\end{center}

\begin{center}\rule{0.5\linewidth}{\linethickness}\end{center}

\hypertarget{ampulla-of-vater}{%
\subsubsection{Ampulla of Vater}\label{ampulla-of-vater}}

\begin{center}\rule{0.5\linewidth}{\linethickness}\end{center}

 - \textbf{MiR-21 up-regulation in ampullary adenocarcinoma and its
pre-invasive lesions}

\emph{Pathology, research and practice 2018 Jun;214(6):835-839}

Poor information is available on the molecular landscape characterizing
the carcinogenetic process leading to ampullary carcinoma. MiR-21 is one
of the most frequently up-regulated miRNAs in pancreatic adenocarcinoma,
a tumor sharing similar molecular features with ampullary
adenocarcinomas (AVCs), above all with the pancreatic-biliary type. We
profiled, by in situ hybridization (ISH), miR-21 expression in a series
of 26 AVCs, 50 ampullary dysplastic lesions (35 low-grade {[}LG-IEN{]}
and 15 high-grade {[}HG-IEN{]}) and 10 normal duodenal mucosa samples.
The same series was investigated by immunohistochemistry for β-catenin,
p53 and HER2 expression. HER2 gene amplification was evaluated by
chromogenic in situ hybridization. To validate miR-21 ISH results we
performed miR-21 qRT-PCR analysis in a series of 10 AVCs and their
matched normal samples. All the normal control samples showed a negative
or faint miR-21 expression, whereas a significant miR-21 up-regulation
was observed during the carcinogenetic cascade (p\,\textless{}\,0.001),
with 21/26 (80.8\%) of cancer samples showing a miR-21 overexpression.
In comparison to control samples, a significant overexpression was found
in samples of LG-IEN (p\,=\,.0003), HG-IEN (p\,=\,.0001), and AVCs
(p\,\textless{}\,0.0001). No significant difference in miR-21
overexpression was observed between LG-IEN, HG-IEN and AVCs. By qRT-PCR
analysis, AVCs showed a 1.7-fold increase over the controls
(p\,=\,.003). P53 was frequently dysregulated in both dysplastic and
carcinoma samples (44 out of 76; 57.9\%). A 20\% (10/50) of dysplastic
lesions and 11\% (3/26) of carcinomas were characterized by a nuclear
localization of β-catenin. Only 2 AVCs (7.7\%; both intestinal-type)
showed a HER2 overexpression (both 2+), which corresponded to a HER2
gene amplification at CISH analysis. This is the first study
demonstrating a miRNA dysregulation in the whole spectrum of ampullary
carcinogenesis. MiR-21 overexpression is an early molecular event during
ampullary carcinogenesis and its levels increase with the neoplastic
progression.

PubMed: \url{https://www.ncbi.nlm.nih.gov/pubmed/?term=29731265}

{}

{}

\begin{center}\rule{0.5\linewidth}{\linethickness}\end{center}

\begin{center}\rule{0.5\linewidth}{\linethickness}\end{center}

Back to top

\pagebreak

\hypertarget{others}{%
\subsection{Others}\label{others}}

\begin{center}\rule{0.5\linewidth}{\linethickness}\end{center}

\hypertarget{pancreas-3}{%
\subsubsection{Pancreas}\label{pancreas-3}}

\begin{center}\rule{0.5\linewidth}{\linethickness}\end{center}

 - \textbf{Management of pregnancy-associated pancreatic cystic tumors:
Review of the literature and results of a Pancreas Club Inc.~Survey}

\emph{Pancreatology : official journal of the International Association
of Pancreatology (IAP) \ldots{} {[}et al.{]} 2018 Sep;():}

BACKGROUND/OBJECTIVES: Management of patients with pregnancy-associated
cyst pancreatic cystic tumors (PA-PCT) is complicated by lack of large
series. METHODS: A systematic literature review was conducted to
extrapolate data on management of PA-PCT, and make a questionnaire on
pending issues to be administered to the members of the Pancreas Club
Inc.~RESULTS: The literature review demonstrated a total of 35\,PA-PCT
in 34 women, described exclusively in the form of case reports, and
permitted the identification of eleven key questions to be addressed in
the survey. The combined analysis of literature review and survery
responses provided several information. First, PA-PCT are predominantly
located in the body-tail of the pancreas, cause non-specific symptoms,
are of large size (mean size: 11.2\,±\,4.5\,cm), and are nearly always
malignant or premalignant, making timing of surgery, and not indication
for surgery, the main issue in the management of these tumors. Second,
there is a risk of PA-PCT rupture during pregnancy. Ruptured PA-PCT had
a mean size 13.5\,±\,4.9\,cm, but no prognostic factor could be
identified. Survey opinions suggested that this occurrence is quite
rare, even for large tumors. Third, most pregnancies were conducted to
term (mean gestational age: 40.5\,±\,0.7 weeks), with a vaginal
delivery. Fourth, all procedures were carried out through an open
approach and the spleen was rarely preserved. Survey indicated instead
that laparoscopy could play a role, and that the spleen should be
preserved when feasible. CONCLUSIONS: PA-PCT require individualized
treatment. The definition of a management algorithm requires the
implementation of an International Registry.

PubMed: \url{https://www.ncbi.nlm.nih.gov/pubmed/?term=30274883}

{}

{}

\begin{center}\rule{0.5\linewidth}{\linethickness}\end{center}

\hypertarget{pancreas-microenvironment}{%
\subsubsection{Pancreas
Microenvironment}\label{pancreas-microenvironment}}

 - \textbf{Elucidating the link between collagen and pancreatic cancer:
what's next?}

\emph{Expert review of gastroenterology \& hepatology 2018
04;12(4):315-317}

PubMed: \url{https://www.ncbi.nlm.nih.gov/pubmed/?term=29495889}

{}

{}

\begin{center}\rule{0.5\linewidth}{\linethickness}\end{center}

\begin{center}\rule{0.5\linewidth}{\linethickness}\end{center}

\hypertarget{pancreas-reviews-and-meta-analysis}{%
\subsubsection{Pancreas Reviews and Meta
Analysis}\label{pancreas-reviews-and-meta-analysis}}

\begin{verbatim}
30269131
\end{verbatim}

\begin{center}\rule{0.5\linewidth}{\linethickness}\end{center}

 - \textbf{Precursor Lesions of Pancreatic Cancer}

\emph{Oncology research and treatment 2018 ;41(10):603-610}

Pancreatic ductal adenocarcinoma (PDAC) is one of the leading causes of
cancer death. Although the treatment modalities are improving, the
prognosis of PDAC continues to be poor. Therefore, early detection of
PDAC or its precursor lesions may be the best way to improve patient
survival. PDACs have several different precursor lesions, including
pancreatic intraepithelial neoplasias (PanINs), intraductal papillary
mucinous neoplasms (IPMNs), intraductal tubulopapillary neoplasms
(ITPNs), intraductal oncocytic papillary neoplasms (IOPNs), and mucinous
cystic neoplasms (MCNs). PanINs cannot be identified using imaging
modalities, while the other lesions are radiologically detectable. These
precursor lesions are categorized based on structural and cytological
atypia as low-grade and high-grade lesions. We discuss recent updates
regarding histopathological and molecular pathological overviews of PDAC
precursor lesions. Better understanding of such lesions may contribute
to earlier detection of PDAC or its precursor lesions and improve PDAC
patient survival.

PubMed: \url{https://www.ncbi.nlm.nih.gov/pubmed/?term=30269131}

{}

{}

\begin{center}\rule{0.5\linewidth}{\linethickness}\end{center}

\begin{center}\rule{0.5\linewidth}{\linethickness}\end{center}

\hypertarget{pancreas-techniques-research-methods}{%
\subsubsection{Pancreas Techniques \& Research
Methods}\label{pancreas-techniques-research-methods}}

\begin{itemize}
\tightlist
\item
  ** A ``Clearer'' View of Pancreatic Pathology: A Review of Tissue
  Clearing and Advanced Microscopy Techniques. **
\end{itemize}

\url{https://europepmc.org/abstract/med/30256228}

\begin{center}\rule{0.5\linewidth}{\linethickness}\end{center}

\hypertarget{animal-studies}{%
\subsubsection{Animal Studies}\label{animal-studies}}

\begin{center}\rule{0.5\linewidth}{\linethickness}\end{center}

 - \textbf{Host IDO2 gene status influences tumor progression and
radiotherapy response in KRAS-driven sporadic pancreatic cancers}

\emph{Clinical cancer research : an official journal of the American
Association for Cancer Research 2018 Sep;():}

PURPOSE: Heritable genetic variations can affect the inflammatory tumor
microenvironment, which can ultimately impact cancer susceptibility and
clinical outcomes. Recent evidence indicates that IDO2, a positive
modifier in inflammatory disease models, is frequently upregulated in
pancreatic ductal adenocarcinoma (PDAC). A unique feature of IDO2in
humans is the high prevalence of two inactivating single nucleotide
polymorphisms (SNPs) which affords the opportunity to carry out
loss-of-function studies directly in humans. In this study we sought to
address whether genetic loss of IDO2 may influence PDAC development and
responsiveness to treatment. EXPERIMENTAL DESIGN: Transgenic Ido2+/+and
Ido2-/-micein which oncogenic KRAS is activated in pancreatic epithelial
cells were evaluated for PDAC. Two patient datasets (N=200) were
evaluated for the two IDO2-inactivating SNPs together with histologic,
RNA expression and clinical survival data. RESULTS: PDAC development was
notably decreased in the Ido2-/- mice (30\% vs 10\%, P\textless{}0.05),
with a female predominance similar to the association observed for one
of the human SNPs. In patients, the biallelic occurrence of either of
the two IDO2-inactivating SNPs was significantly associated with
markedly improved disease-free survival in response to adjuvant
radiotherapy (P\textless{}0.01), a treatment modality that has been
highly debated due to its variable efficacy. CONCLUSIONS: The results of
this study provide genetic support for IDO2 as a contributing factor in
PDAC development and argue that IDO2genotype analysis has the immediate
potential to influence the PDAC care decision-making process through
stratification of those patients who stand to benefit from adjuvant
radiotherapy.

PubMed: \url{https://www.ncbi.nlm.nih.gov/pubmed/?term=30266763}

{}

{}

\begin{center}\rule{0.5\linewidth}{\linethickness}\end{center}

\begin{center}\rule{0.5\linewidth}{\linethickness}\end{center}

\begin{center}\rule{0.5\linewidth}{\linethickness}\end{center}

\hypertarget{gallbladder}{%
\subsubsection{Gallbladder}\label{gallbladder}}

\begin{itemize}
\tightlist
\item
  \textbf{A case report of intracholecystic papillary neoplasm of the
  gallbladder resembling a submucosal tumor}
\end{itemize}

\url{https://link.springer.com/article/10.1186/s40792-018-0524-2}

\begin{center}\rule{0.5\linewidth}{\linethickness}\end{center}

 - \textbf{Intravascular Large B-Cell Lymphoma of the Gallbladder}

\emph{Turkish journal of haematology : official journal of Turkish
Society of Haematology 2018 May;35(2):145-146}

PubMed: \url{https://www.ncbi.nlm.nih.gov/pubmed/?term=29391332}

{}

{}

\begin{center}\rule{0.5\linewidth}{\linethickness}\end{center}

\hypertarget{gallbladder-reviews-and-meta-analysis}{%
\subsubsection{Gallbladder Reviews and Meta
Analysis}\label{gallbladder-reviews-and-meta-analysis}}

\begin{itemize}
\tightlist
\item
  \textbf{Outcomes of surgical resection of gallbladder cancer in
  patients presenting with jaundice: A systematic review and
  meta‐analysis}
\end{itemize}

\url{https://onlinelibrary.wiley.com/doi/abs/10.1002/jso.25186}

\begin{center}\rule{0.5\linewidth}{\linethickness}\end{center}

\hypertarget{ampulla-of-vater-1}{%
\subsubsection{Ampulla of Vater}\label{ampulla-of-vater-1}}

\begin{center}\rule{0.5\linewidth}{\linethickness}\end{center}

 - ****\\
{[}Prognostic factors in adenocarcinoma of the ampulla of Vater{]}.\\

\url{https://www.ncbi.nlm.nih.gov/pubmed/?term=30269771}

\begin{itemize}
\tightlist
\item
  ;():*
\end{itemize}

PubMed: \url{https://www.ncbi.nlm.nih.gov/pubmed/?term=}

{}

{}

\begin{center}\rule{0.5\linewidth}{\linethickness}\end{center}

\begin{center}\rule{0.5\linewidth}{\linethickness}\end{center}

\hypertarget{database-seer-ncdb-tcga-oncomine-studies}{%
\subsubsection{Database SEER, NCDB, TCGA, Oncomine
Studies}\label{database-seer-ncdb-tcga-oncomine-studies}}

\begin{itemize}
\tightlist
\item
  \textbf{The incidence and survival of pancreatic cancer by histology,
  including rare subtypes: a nation‐wide cancer registry‐based study
  from Taiwan}
\end{itemize}

\url{https://onlinelibrary.wiley.com/doi/pdf/10.1002/cam4.1795}

\begin{center}\rule{0.5\linewidth}{\linethickness}\end{center}

\begin{itemize}
\tightlist
\item
  \textbf{Surgical resection of lymph node positive intrahepatic
  cholangiocarcinoma may not improve survival}
\end{itemize}

\url{https://www.sciencedirect.com/science/article/pii/S1365182X18339352}

\begin{center}\rule{0.5\linewidth}{\linethickness}\end{center}

 - \textbf{Global surveillance of trends in cancer survival 2000-14
(CONCORD-3): analysis of individual records for 37 513 025 patients
diagnosed with one of 18 cancers from 322 population-based registries in
71 countries}

\emph{Lancet (London, England) 2018 03;391(10125):1023-1075}

BACKGROUND: In 2015, the second cycle of the CONCORD programme
established global surveillance of cancer survival as a metric of the
effectiveness of health systems and to inform global policy on cancer
control. CONCORD-3 updates the worldwide surveillance of cancer survival
to 2014. METHODS: CONCORD-3 includes individual records for 37·5 million
patients diagnosed with cancer during the 15-year period 2000-14. Data
were provided by 322 population-based cancer registries in 71 countries
and territories, 47 of which provided data with 100\% population
coverage. The study includes 18 cancers or groups of cancers:
oesophagus, stomach, colon, rectum, liver, pancreas, lung, breast
(women), cervix, ovary, prostate, and melanoma of the skin in adults,
and brain tumours, leukaemias, and lymphomas in both adults and
children. Standardised quality control procedures were applied; errors
were rectified by the registry concerned. We estimated 5-year net
survival. Estimates were age-standardised with the International Cancer
Survival Standard weights. FINDINGS: For most cancers, 5-year net
survival remains among the highest in the world in the USA and Canada,
in Australia and New Zealand, and in Finland, Iceland, Norway, and
Sweden. For many cancers, Denmark is closing the survival gap with the
other Nordic countries. Survival trends are generally increasing, even
for some of the more lethal cancers: in some countries, survival has
increased by up to 5\% for cancers of the liver, pancreas, and lung. For
women diagnosed during 2010-14, 5-year survival for breast cancer is now
89·5\% in Australia and 90·2\% in the USA, but international differences
remain very wide, with levels as low as 66·1\% in India. For
gastrointestinal cancers, the highest levels of 5-year survival are seen
in southeast Asia: in South Korea for cancers of the stomach (68·9\%),
colon (71·8\%), and rectum (71·1\%); in Japan for oesophageal cancer
(36·0\%); and in Taiwan for liver cancer (27·9\%). By contrast, in the
same world region, survival is generally lower than elsewhere for
melanoma of the skin (59·9\% in South Korea, 52·1\% in Taiwan, and
49·6\% in China), and for both lymphoid malignancies (52·5\%, 50·5\%,
and 38·3\%) and myeloid malignancies (45·9\%, 33·4\%, and 24·8\%). For
children diagnosed during 2010-14, 5-year survival for acute
lymphoblastic leukaemia ranged from 49·8\% in Ecuador to 95·2\% in
Finland. 5-year survival from brain tumours in children is higher than
for adults but the global range is very wide (from 28·9\% in Brazil to
nearly 80\% in Sweden and Denmark). INTERPRETATION: The CONCORD
programme enables timely comparisons of the overall effectiveness of
health systems in providing care for 18 cancers that collectively
represent 75\% of all cancers diagnosed worldwide every year. It
contributes to the evidence base for global policy on cancer control.
Since 2017, the Organisation for Economic Co-operation and Development
has used findings from the CONCORD programme as the official benchmark
of cancer survival, among their indicators of the quality of health care
in 48 countries worldwide. Governments must recognise population-based
cancer registries as key policy tools that can be used to evaluate both
the impact of cancer prevention strategies and the effectiveness of
health systems for all patients diagnosed with cancer. FUNDING: American
Cancer Society; Centers for Disease Control and Prevention; Swiss Re;
Swiss Cancer Research foundation; Swiss Cancer League; Institut National
du Cancer; La Ligue Contre le Cancer; Rossy Family Foundation; US
National Cancer Institute; and the Susan G Komen Foundation.

PubMed: \url{https://www.ncbi.nlm.nih.gov/pubmed/?term=29395269}

{}

{}

\begin{center}\rule{0.5\linewidth}{\linethickness}\end{center}

 - \textbf{Achieving better cancer intelligence for global cancer
control}

\emph{Lancet (London, England) 2018 03;391(10125):1003-1004}

PubMed: \url{https://www.ncbi.nlm.nih.gov/pubmed/?term=29395271}

{}

{}

\begin{center}\rule{0.5\linewidth}{\linethickness}\end{center}

 - \textbf{Pattern of distant metastases in primary extrahepatic
bile-duct cancer: A SEER-based study}

\emph{Cancer medicine 2018 Sep;():}

Extrahepatic bile duct cancer (EBDC) is a combined type of malignancy
mainly consisting of extrahepatic cholangiocarcinoma and gallbladder
cancer. Clinically, it is featured with latent symptoms and early
metastasis, leading to a poor prognosis. Therefore, this cohort study
aimed to depict the possible metastatic patterns of EBDC of diverse
sub-types and evaluate the prognostic significance of diverse metastatic
destinations with data from the clinical database. Relevant data of
total 4061 confirmed EBDC patients diagnosed between 2010 and 2013 from
the Surveillance, Epidemiology and End Results (SEER) database was
obtained. We applied t test to describe the baseline data of patients
included and used chi-square test to compare the distribution of distant
metastatic sites. We further adopted odds ratio assess the combined
metastatic patterns and compared survival difference of patients with
different distal metastasis organ by Kaplan-Meier analysis. We
identified totally 4061 patients over 18~years old diagnosed with
extrahepatic bile tract malignancies between 2010 and 2013, with clear
metastatic status and follow-up data, without primary malignancies.
Liver and distant lymph (DL) are the two most common sites as a single
metastasis organ. In combined metastasis patterns, bi-organ is more
frequent than the other types. Lung is the organ preferentially for
bi-organ metastasis, while bone and distant lymph similarly intend to
co-metastasize with brain. Distal metastasis in EBDC patients indicates
an extremely poor prognosis. According to the final analysis results,
malignancies in extrahepatic bile duct exhibit similar metastatic
patterns, suggesting that we can regard them as a unity to assess its
development. Profound differences exist in distribution of distant
extrahepatic metastatic sites and their combinations. Results from our
studies would provide some information for follow-up strategies and
future studies.

PubMed: \url{https://www.ncbi.nlm.nih.gov/pubmed/?term=30277653}

{}

{}

\begin{center}\rule{0.5\linewidth}{\linethickness}\end{center}

\begin{itemize}
\tightlist
\item
  \textbf{Pancreatic cancer and autoimmune diseases: An association
  sustained by computational and epidemiological case-control
  approaches: Autoimmune diseases and pancreatic cancer risk}
\end{itemize}

\url{https://www.researchgate.net/publication/327754246_Pancreatic_cancer_and_autoimmune_diseases_An_association_sustained_by_computational_and_epidemiological_case-control_approaches_Autoimmune_diseases_and_pancreatic_cancer_risk}

\begin{center}\rule{0.5\linewidth}{\linethickness}\end{center}

Back to top

\pagebreak

\hypertarget{journals-reviewed}{%
\subsection{Journals Reviewed}\label{journals-reviewed}}

\begin{center}\rule{0.5\linewidth}{\linethickness}\end{center}

Advances in Anatomic Pathology\\
American Journal of Clinical Pathology\\
The American Journal of Gastroenterology\\
The American Journal of Pathology\\
American Journal of Surgical Pathology\\
Annals of Diagnostic Pathology\\
Annals of Surgery\\
Annals of Surgical Oncology\\
Annual Review of Pathology-Mechanisms of Disease\\
APMIS\\
Applied Immunohistochemistry \& Molecular Morphology\\
Archives of Pathology \& Laboratory Medicine\\
Cancer\\
Cancer Cell\\
Cancer Cytopathology\\
Cell\\
Cellular Oncology\\
Clinical Cancer Research\\
Cochrane Database Systematic Reviews\\
Cytojournal\\
Cytopathology\\
Diagnostic Cytopathology\\
Diagnostic Pathology\\
Endocrine Pathology\\
Experimental and Molecular Pathology\\
Expert Review of Molecular Diagnostics\\
Gastroenterology\\
Gut\\
Histology and Histopathology\\
Histopathology\\
Human Pathology\\
International Journal of Surgical Pathology\\
International Journal of Clinical and Experimental Pathology\\
Journal of Clinical Pathology\\
Journal of Molecular Diagnostics\\
Journal of Pathology\\
Laboratory investigation\\
Lancet\\
Medical Molecular Morphology\\
Modern Pathology\\
Nature\\
Nature Reviews Gastroenterology \& Hepatology\\
NEJM\\
Pancreas\\
Pancreatology\\
Pathobiology\\
Pathologie Biologie\\
Pathology\\
Pathology \& Oncology Research\\
Pathology International\\
Pathology Research and Practice\\
PNAS\\
Science\\
Seminars in Diagnostic Pathology\\
Seminars in Immunopathology\\
Surgical pathology clinics\\
Tissue Antigens\\
Trends in Cancer\\
Virchows Archiv

\begin{center}\rule{0.5\linewidth}{\linethickness}\end{center}

Back to top

\pagebreak

\hypertarget{feedback}{%
\subsection{Feedback}\label{feedback}}

\begin{center}\rule{0.5\linewidth}{\linethickness}\end{center}

Please send your feedbacks using the form below:

\href{https://docs.google.com/forms/d/e/1FAIpQLSeD3Z9J6Y7eMmiyM12f_SfAmHUlykb1zxZcwO6lg7cebGYQIQ/viewform}{Click
here for the Feedback Form}

\begin{center}\rule{0.5\linewidth}{\linethickness}\end{center}

Back to top

\pagebreak

\hypertarget{archive}{%
\subsection{Archive}\label{archive}}

\begin{center}\rule{0.5\linewidth}{\linethickness}\end{center}

\textbf{The PBPath Journal Archive}

\begin{itemize}
\item
  \href{http://pbpath.org/journal-watch/}{Current Issue}
\item
  \href{http://pbpath.org/journal-watch-archive/}{Older Issues}

  \begin{itemize}
  \item
    \href{http://pbpath.org/pbpath-journal-watch-june-july-2018/}{June-July-2018}
  \item
    \href{http://pbpath.org/pbpath-journal-watch-august-september-2018/}{August-September-2018}
  \end{itemize}
\end{itemize}

\begin{center}\rule{0.5\linewidth}{\linethickness}\end{center}

Back to top

\pagebreak


\end{document}
