\documentclass[]{article}
\usepackage{lmodern}
\usepackage{amssymb,amsmath}
\usepackage{ifxetex,ifluatex}
\usepackage{fixltx2e} % provides \textsubscript
\ifnum 0\ifxetex 1\fi\ifluatex 1\fi=0 % if pdftex
  \usepackage[T1]{fontenc}
  \usepackage[utf8]{inputenc}
\else % if luatex or xelatex
  \ifxetex
    \usepackage{mathspec}
  \else
    \usepackage{fontspec}
  \fi
  \defaultfontfeatures{Ligatures=TeX,Scale=MatchLowercase}
\fi
% use upquote if available, for straight quotes in verbatim environments
\IfFileExists{upquote.sty}{\usepackage{upquote}}{}
% use microtype if available
\IfFileExists{microtype.sty}{%
\usepackage{microtype}
\UseMicrotypeSet[protrusion]{basicmath} % disable protrusion for tt fonts
}{}
\usepackage[margin=1in]{geometry}
\usepackage{hyperref}
\hypersetup{unicode=true,
            pdftitle={Pancreatobiliary Pathology Society Journal Watch},
            pdfborder={0 0 0},
            breaklinks=true}
\urlstyle{same}  % don't use monospace font for urls
\usepackage{graphicx,grffile}
\makeatletter
\def\maxwidth{\ifdim\Gin@nat@width>\linewidth\linewidth\else\Gin@nat@width\fi}
\def\maxheight{\ifdim\Gin@nat@height>\textheight\textheight\else\Gin@nat@height\fi}
\makeatother
% Scale images if necessary, so that they will not overflow the page
% margins by default, and it is still possible to overwrite the defaults
% using explicit options in \includegraphics[width, height, ...]{}
\setkeys{Gin}{width=\maxwidth,height=\maxheight,keepaspectratio}
\IfFileExists{parskip.sty}{%
\usepackage{parskip}
}{% else
\setlength{\parindent}{0pt}
\setlength{\parskip}{6pt plus 2pt minus 1pt}
}
\setlength{\emergencystretch}{3em}  % prevent overfull lines
\providecommand{\tightlist}{%
  \setlength{\itemsep}{0pt}\setlength{\parskip}{0pt}}
\setcounter{secnumdepth}{0}
% Redefines (sub)paragraphs to behave more like sections
\ifx\paragraph\undefined\else
\let\oldparagraph\paragraph
\renewcommand{\paragraph}[1]{\oldparagraph{#1}\mbox{}}
\fi
\ifx\subparagraph\undefined\else
\let\oldsubparagraph\subparagraph
\renewcommand{\subparagraph}[1]{\oldsubparagraph{#1}\mbox{}}
\fi

%%% Use protect on footnotes to avoid problems with footnotes in titles
\let\rmarkdownfootnote\footnote%
\def\footnote{\protect\rmarkdownfootnote}

%%% Change title format to be more compact
\usepackage{titling}

% Create subtitle command for use in maketitle
\newcommand{\subtitle}[1]{
  \posttitle{
    \begin{center}\large#1\end{center}
    }
}

\setlength{\droptitle}{-2em}

  \title{Pancreatobiliary Pathology Society Journal Watch}
    \pretitle{\vspace{\droptitle}\centering\huge}
  \posttitle{\par}
  \subtitle{October November 2018}
  \author{}
    \preauthor{}\postauthor{}
      \predate{\centering\large\emph}
  \postdate{\par}
    \date{Last Update on 2018-12-25}


\begin{document}
\maketitle

{
\setcounter{tocdepth}{5}
\tableofcontents
}
\href{http://pbpath.org/}{\includegraphics{figures/PBP_header_logo.jpg}}

\begin{center}\rule{0.5\linewidth}{\linethickness}\end{center}

\hypertarget{pbpath-journal-watch-articles}{%
\section{PBPath Journal Watch
Articles}\label{pbpath-journal-watch-articles}}

\begin{center}\rule{0.5\linewidth}{\linethickness}\end{center}

\textbf{Wellcome to the PBPath Journal Watch!}

This bi-monthly journal watch features exciting recently published
pancreas and biliary pathology articles that will provide up to date
medical knowledge in our field. These articles will be showcased in
several convenient categories, including surgical pathology,
cytopathology, and molecular pathology among others. The articles in
each category are in no particular order.

Previous months' issues may be found in our
\emph{\href{http://pbpath.org/journal-watch-archive/}{archive}}.

We encourage members to actively participate by recommending new
articles and providing feedback using
\emph{\href{https://docs.google.com/forms/d/e/1FAIpQLSeD3Z9J6Y7eMmiyM12f_SfAmHUlykb1zxZcwO6lg7cebGYQIQ/viewform}{the
forms provided}}.

\textbf{We hope that you will enjoy the new PBPath Journal Watch!}

\pagebreak

\begin{center}\rule{0.5\linewidth}{\linethickness}\end{center}

\hypertarget{surgical-pathology}{%
\subsection{Surgical Pathology}\label{surgical-pathology}}

\begin{center}\rule{0.5\linewidth}{\linethickness}\end{center}

\hypertarget{pancreas}{%
\subsubsection{Pancreas}\label{pancreas}}

\hypertarget{morphology-diagnostics-ihc}{%
\paragraph{Morphology, Diagnostics,
IHC}\label{morphology-diagnostics-ihc}}

Morphology, Diagnostics, IHC

\begin{center}\rule{0.5\linewidth}{\linethickness}\end{center}

 - \textbf{Pancreatic cancer arising in the remnant pancreas is not
always a relapse of the preceding primary}

\emph{Modern pathology : an official journal of the United States and
Canadian Academy of Pathology, Inc 2018 Nov;():}

PubMed: \url{https://www.ncbi.nlm.nih.gov/pubmed/?term=30467323}

This study aimed to understand the biology of pancreatic ductal
adenocarcinoma that arises in the remnant pancreas after surgical
resection of a primary pancreatic ductal adenocarcinoma, using
integrated histological and molecular analysis. Patients who underwent a
completion pancreatectomy for local recurrence following resection of a
primary pancreatic ductal adenocarcinoma were studied with histological
analysis and next-generation sequencing of the primary and the recurrent
cancer. Of six patients that met the inclusion criteria, three cases
were classified as ``true'' recurrences, i.e., the primary and the
cancer in the remnant pancreas shared both morphological features and
molecular alterations. Two cases were identified as having independent
cancers that exhibited different histological and molecular profiles. In
the remaining case, the relationship could not be determined. Pancreatic
ductal adenocarcinoma that arises in the remnant pancreas can be either
a second primary or a ``true'' relapse of the preceding primary. The
differentiation of second primaries from local recurrences may have
important implications for patient management.

{}

{}

\begin{center}\rule{0.5\linewidth}{\linethickness}\end{center}

 - \textbf{Epidermoid cyst in intrapancreatic accessory spleen: A
systematic review}

\emph{Pancreatology : official journal of the International Association
of Pancreatology (IAP) \ldots{} {[}et al.{]} 2018 Oct;():}

PubMed: \url{https://www.ncbi.nlm.nih.gov/pubmed/?term=30366677}

BACKGROUND/OBJECTIVES: Due to its rarity, epidermoid cyst in
intrapancreatic accessory spleen (ECIPAS) is still a diagnostic dilemma
during clinical practice. The aim of this review was to summarize the
epidemiologic features and management of ECIPAS. METHODS: MEDLINE and
EMBASE were searched for English articles reporting on ECIPAS up to
April 30th, 2018 following the methodology suggested by the PRISMA
guidelines. Categorical variables were reported as frequency and
percentage. Continuous variables were reported as median (range).
RESULTS: A total of 56 patients from 47 full articles were included for
the final data synthesis. More than half of the ECIPASs (59\%) were
found incidentally. The female/male ratio was 1.33. ECIPAS is typically
a single mono-/multi-lobular cystic lesions in the pancreatic tail with
thickened cystic wall or various amount of solid component which had
identical density/signal to the spleen on imaging examinations. The cyst
is filled with serous or non-serous fluid. Recognition of the
surrounding ectopic splenic tissue is the key point to diagnose ECIPAS.
However, no preoperative examination was able to make a definite
diagnosis. Almost all the patients (96\%) received surgical treatment,
due to the suspicion of pancreatic malignant or potentially malignant
cystic tumor, especially mucinous cystic neoplasm (MCN). CONCLUSIONS:
Although seldom encountered, ECIPAS should be considered as a
differential diagnosis for pancreatic cystic lesions, especially when
solid component was detected. As a benign disease, unnecessary surgery
should be avoided. Because it is difficult to make a definite diagnosis
preoperatively by one single examination, multiple modalities may be
required.

{}

{}

\begin{center}\rule{0.5\linewidth}{\linethickness}\end{center}

 - \textbf{The expression of death receptor systems TRAIL-R1/-R2/-R4,
CD95 and TNF-R1 and their cognate ligands in pancreatic ductal
adenocarcinoma}

\emph{Histology and histopathology 2018 Oct;():18054}

PubMed: \url{https://www.ncbi.nlm.nih.gov/pubmed/?term=30375637}

The expression of five members of the TNF receptor superfamily and two
of their ligands in human pancreatic ductal adenocarcinoma were
investigated in parallel by immunohistochemistry. 41 patients with
histologically confirmed ductal carcinoma of the pancreas were enrolled
in this study in order (i) to compare the individual TNFR-SF expression
and their ligands in PDAC-cells and (ii) to investigate their
correlation with survival data. All patients had undergone
pancreaticoduodenectomy and were staged as pT3N1M0. Immunostaining was
done on FFPE tissue sections of the tumor tissue, using antibodies
directed against TRAIL-Receptor-1, -2 and -4, TRAIL, CD95,
TNF-Receptor-1 and TNF-α. The intensity and quantity of immunostaining
were evaluated separately for tumor cell cytoplasm and tumor cell
nucleus. Immunostaining results were correlated with each other and with
patient survival. All proteins were found to be expressed in the
majority of the tumor cells. The expression (i) of the following members
of TNFR-SF and their ligands correlated with each other: TNF-Receptor-1
and TNFα (cytoplasmatic scores, p=0.001), TNF-Receptor 1 and TRAIL
(nuclear antigen expression p=0.005 and the main score p=0.001, which
contains the overall intracellular antigen expression), TNF-Receptor 1
and CD95 (main score, p=0.001), TRAIL-Receptor-1 and TRAIL-Receptor-2
(nuclear parameters, p=0.023), TRAIL-Receptor-4 and TRAIL (main score
p=0.041). In addition (ii), high cytoplasmatic expression of
TNF-Receptor-1 and a strong cytoplasmatic and nuclear expression of CD95
correlated significantly with a better prognosis of the PDAC patients.

{}

{}

\begin{center}\rule{0.5\linewidth}{\linethickness}\end{center}

 - \textbf{High nuclear Survivin expression as a poor prognostic marker
in pancreatic ductal adenocarcinoma}

\emph{Journal of surgical oncology 2018 Dec;118(7):1115-1121}

PubMed: \url{https://www.ncbi.nlm.nih.gov/pubmed/?term=30261114}

BACKGROUND: Survivin, one of the key regulators of mitosis and
apoptosis, has long been well recognized to play important biological
roles in many neoplasms, including pancreatic ductal adenocarcinoma
(PDAC). However, its prognostic value in PDAC remains controversial.
PATIENTS AND METHODS: Nuclear expression of Survivin was detected, using
tissue microarray-based immunohistochemistry, in paired-tumor and
nontumor samples from 306 patients with radically resected PDAC. The
staining H scores were further correlated with clinicopathologic
features and disease-specific survival (DSS). RESULTS: Nuclear Survivin
expression was much higher in tumor than in nontumor tissues
(P \textless{} 0.001). No significant association between tumoral
Survivin expression and clinicopathologic variables was found. For
prognosis, high Survivin expression was associated with shortened DSS in
all eligible patients and four subgroups, that is, male and nondiabetic
patients as well as those with head-located and G1-2 tumors, shown by
univariate analyses. In addition, a statistically marginal significance
was revealed in eight subgroups. For the entire cohort and two
subgroups, nuclear Survivin expression was also multivariate identified
as an independent predictor for DSS. For patients with G1-2 tumors, it
was the single prognostic marker. CONCLUSION: Our data suggest an
association between high nuclear Survivin expression and poor prognosis
in PDAC. However, further confirmation might be necessary.

{}

{}

\begin{center}\rule{0.5\linewidth}{\linethickness}\end{center}

 - \textbf{Residual Tumor Index: A Prognostically Significant Pathologic
Parameter in Neoadjuvant-treated Pancreatic Ductal Adenocarcinoma}

\emph{The American journal of surgical pathology 2018
Nov;42(11):1480-1487}

PubMed: \url{https://www.ncbi.nlm.nih.gov/pubmed/?term=30179901}

In the setting of neoadjuvant therapy (NAT) for pancreatic ductual
adenocarcinoma (PDAC), accurate measurement of tumor size, and
consequently, staging based on AJCC eighth edition, is difficult.
Attempts to address the limitations of tumor size in the NAT setting
have included correlation of residual tumor percent with survival.
However, only cases with complete pathologic response or minimal
residual disease have shown better prognosis compared with all other
groups. To date, no studies have simultaneously evaluated the prognostic
value of tumor size and tumor regression in the setting of PDAC status
post NAT (NAT-PDAC). Our aim was to study the prognostic value of
residual tumor index (RTI), a metric combining residual tumor percent
and tumor bed size as an interaction term (\% residual tumor×tumor bed
size {[}cm{]}). In a cohort of 105 cases of NAT-PDAC, we show that RTI
supersedes the prognostic value of AJCC eighth edition T staging via
multivariate cox regression. At a binary cutoff of 0.35 for RTI, the
hazard ratio for recurrence-free survival is 3.26 (95\% confidence
interval, 1.51-7.04), P\textless{}0.01. We further identified cutoffs of
≤0.2, 0.2 to 2 and \textgreater{}2 that stratified our cases into 3
groups via RTI, which were statistically significant in Kaplan-Meier
curve analysis of recurrence-free survival (P\textless{}0.01) and
overall survival (P\textless{}0.01). RTI represents a novel metric for
combining the prognostic value of tumor size and residual tumor in
NAT-PDAC.

{}

{}

\begin{center}\rule{0.5\linewidth}{\linethickness}\end{center}

 - \textbf{PD-L1 expression in pancreatic adenosquamous carcinoma: PD-L1
expression is limited to the squamous component}

\emph{Pathology, research and practice 2018 Dec;214(12):2069-2074}

PubMed: \url{https://www.ncbi.nlm.nih.gov/pubmed/?term=30477643}

AIM: We examined the programmed death-ligand 1 (PD-L1) expression in
surgically resected pancreatic adenosquamous carcinoma (PASC) samples.
Furthermore, the detection rate was also assessed using biopsy cases
obtained from endoscopic ultrasound-guided fine needle aspiration
(EUS-FNA). METHODS: Fifteen cases of PASC (six resected and nine EUS-FNA
biopsied) from the Kurume University Hospital between 2009 and 2016 were
used for the evaluation of PD-L1 expression. As a control group, 34
cases of pancreatic ductal adenocarcinomas (PDACs) were selected. To
compare the positivity and intensity of PD-L1, two types of clones
(SP263, E1L3N) were examined for immunostaining. Only the membrane
expression of PD-L1 was regarded as positive. The PD-L1 expressions in
the squamous cell carcinoma component (SCc), adenocarcinoma component
(ACc), and immune cells were assessed separately. The ratio of PD-L1
expression was calculated by counting the positive tumor cells, and
tumor proportion score (TPS) was applied (TPS; Null \textless{} 1\%, low
expression; 1 ≤ TPS ≤ 49\% and high expression; ≥ 50\%). RESULTS: PD-L1
expression was observed in five surgical PASC samples (83\%). This shows
that SCc presented a high expression in these cases. However, the
overall TPS indicated a low expression. In contrast, only one case (3\%)
was positive for PD-L1 in PDACs, and the TPS indicated a low expression.
No differences in PD-L1 expression were observed between the two clones,
SP263 and E1L3N. High PD-L1 expression in the EUS-FNA sample was found
in only one case (11\%). DISCUSSION: Although assessment using the tumor
cells of PASC samples obtained from EUS-FNA was difficult, this study
suggests the selective expression of PD-L1 in the SCc of PASC.
Furthermore, it was considered that immune checkpoint inhibitors could
provide therapeutic effects selectively on the SCc for the entire range
of TPSs, though the PD-L1 expression was low.

{}

{}

\begin{center}\rule{0.5\linewidth}{\linethickness}\end{center}

 - \textbf{Cancerization of the Pancreatic Ducts: Demonstration of a
Common and Under-recognized Process Using Immunolabeling of Paired Duct
Lesions and Invasive Pancreatic Ductal Adenocarcinoma for p53 and Smad4
Expression}

\emph{The American journal of surgical pathology 2018
Nov;42(11):1556-1561}

PubMed: \url{https://www.ncbi.nlm.nih.gov/pubmed/?term=30212393}

Invasive pancreatic ductal adenocarcinoma (PDAC) can infiltrate back
into and spread along preexisting pancreatic ducts and ductules in a
process known as cancerization of ducts (COD). Histologically COD can
mimic high-grade pancreatic intraepithelial neoplasia (HG-PanIN). We
reviewed pancreatic resections from 100 patients with PDAC for the
presence or absence of ducts with histologic features of COD. Features
supporting COD included adjacent histologically similar invasive PDAC
and an abrupt transition between markedly atypical intraductal
epithelium and normal duct epithelium or circumferential involvement of
a duct. As the TP53 and SMAD4 genes are frequently targeted in invasive
PDAC but not HG-PanIN, paired PDAC and histologically suspected COD
lesions were immunolabeled with antibodies to the p53 and Smad4
proteins. Suspected COD was identified on hematoxylin and eosin sections
in 89 (89\%) of the cases. Immunolabeling for p53 and Smad4 was
performed in 68 (76\%) of 89 cases. p53 was interpretable in 55 cases
and all 55 (100\%) cases showed concordant labeling between COD and
invasive PDAC. There was matched aberrant p53 immunolabeling in 37
(67\%) cases including overexpression in 30 (55\%) cases and lack of
expression in 7 (13\%) cases. Smad4 immunolabeling was interpretable in
61 cases and 59 (97\%) cases showed concordant labeling between COD and
invasive PDAC. Matched loss of Smad4 was seen in 28 (46\%) cases. The
immunolabeling of invasive PDAC and COD for p53 and Smad4 supports the
high prevalence of COD observed on hematoxylin and eosin and highlights
the utility of p53 and Smad4 immunolabeling in differentiating COD and
HG-PanIN.

{}

{}

\begin{center}\rule{0.5\linewidth}{\linethickness}\end{center}

 - \textbf{A ``Clearer'' View of Pancreatic Pathology: A Review of
Tissue Clearing and Advanced Microscopy Techniques}

\emph{Advances in anatomic pathology 2019 Jan;26(1):31-39}

PubMed: \url{https://www.ncbi.nlm.nih.gov/pubmed/?term=30256228}

Although pathologic lesions in the pancreas are 3-dimensional (3D)
complex structures, we currently use thin 2D hematoxylin and eosin
stained slides to study and diagnose pancreatic pathology. Two
technologies, tissue clearing and advanced microscopy, have recently
converged, and when used together they open the remarkable world of 3D
anatomy and pathology to pathologists. Advances in tissue clearing and
antibody penetration now make even dense fibrotic tissues amenable to
clearing, and light sheet and confocal microscopies allow labeled cells
deep within these cleared tissues to be visualized. Clearing techniques
can be categorized as solvent-based or aqueous-based techniques, but
both clearing methods consist of 4 fundamental steps, including
pretreatment of specimens, permeabilization and/or removal of lipid,
immunolabeling with antibody penetration, and clearing by refractive
index matching. Specialized microscopes, including the light sheet
microscope, the 2-photon microscope, and the confocal microscope, can
then be used to visualize and evaluate the 3D histology. Both endocrine
and exocrine pancreas pathology can then be visualized. The application
of labeling and clearing to surgically resected human pancreatic
parenchyma can provide detailed visualization of the complexities of
normal pancreatic anatomy. It also can be used to characterize the 3D
architecture of disease processes ranging from precursor lesions, such
as pancreatic intraepithelial neoplasia lesions and intraductal
papillary mucinous neoplasms, to infiltrating pancreatic ductal
adenocarcinomas. The evaluation of 3D histopathology, including
pathology of the pancreatic lesions, will provide new insights into
lesions that previously were seen, and thought of, only in 2 dimensions.

{}

{}

\begin{center}\rule{0.5\linewidth}{\linethickness}\end{center}

 - \textbf{Tenascin C, Fibronectin, and Tumor-Stroma Ratio in Pancreatic
Ductal Adenocarcinoma}

\emph{Pancreas 2019 Jan;48(1):43-48}

PubMed: \url{https://www.ncbi.nlm.nih.gov/pubmed/?term=30451798}

OBJECTIVES: Pancreatic ductal adenocarcinoma (PDAC) is characterized by
abundant stroma with increased expression of tenascin C and fibronectin.
Their role and tumor-stroma ratio in PDAC are not well known. The aim of
this study was to evaluate tenascin C and fibronectin expression and
tumor-stroma ratio and their prognostic relevance in PDAC. METHODS:
Ninety-five resected PDACs were immunohistochemically stained for
tenascin C and fibronectin, and the expression was separately assessed
in tumor bulk and front. Tumor-stroma ratio was determined with sections
stained with hematoxylin-eosin. RESULTS: Tenascin C and fibronectin were
abundantly expressed in the stroma of PDAC, but absent in adjacent
normal pancreatic tissue. Fibronectin expression of the bulk was
associated with high T class (P = 0.045). In the main analysis, tenascin
C and fibronectin expression and tumor-stroma ratio were not associated
with patient survival. In a subgroup analysis of early-stage PDAC (T1-T2
tumors), high tenascin C expression in the tumor bulk was associated
with poor prognosis (hazard ratio, 8.23; 95\% confidence interval,
2.71-24.96). CONCLUSIONS: Tenascin C and fibronectin are abundantly
expressed in PDAC, but they seem to have no major association with
patient survival. However, in early-stage PDAC, tenascin C expression of
the tumor bulk may have prognostic impact. Tumor-stroma ratio has no
prognostic value in PDAC.

{}

{}

\begin{center}\rule{0.5\linewidth}{\linethickness}\end{center}

Back to top

\begin{center}\rule{0.5\linewidth}{\linethickness}\end{center}

\pagebreak

\hypertarget{staging}{%
\paragraph{Staging}\label{staging}}

Pancreas TNM staging, Margins, Survival

\begin{center}\rule{0.5\linewidth}{\linethickness}\end{center}

 - \textbf{International Validation of the Eighth Edition of the
American Joint Committee on Cancer (AJCC) TNM Staging System in Patients
With Resected Pancreatic Cancer}

\emph{JAMA surgery 2018 Dec;153(12):e183617}

PubMed: \url{https://www.ncbi.nlm.nih.gov/pubmed/?term=30285076}

Importance: The recently released eighth edition of the American Joint
Committee on Cancer TNM staging system for pancreatic cancer seeks to
improve prognostic accuracy but lacks international validation.
Objective: To validate the eighth edition of the American Joint
Committee on Cancer TNM staging system in an international cohort of
patients with resected pancreatic ductal adenocarcinoma. Design,
Setting, and Participants: This international multicenter cohort study
took place in 5 tertiary centers in Europe and the United States from
2000 to 2015. Patients who underwent pancreatoduodenectomy for
nonmetastatic pancreatic ductal adenocarcinoma were eligible. Data
analysis took place from December 2017 to April 2018. Exposures:
Patients were retrospectively staged according to the seventh and eighth
editions of the TNM staging system. Main Outcomes and Measures:
Prognostic accuracy on survival rates, assessed by Kaplan-Meier and
multivariate Cox proportional hazards analyses and concordance
statistics. Results: A total of 1525 consecutive patients were included
(median {[}IQR{]} age, 66 (58-72) years; 802 (52.6\%) male).
Distribution among stages via the seventh edition was stage IA in 41
patients (2.7\%), stage IB in 42 (2.8\%), stage IIA in 200 (13.1\%),
stage IIB in 1229 (80.6\%), and stage III in 12 (0.8\%); this changed
with use of the eighth edition to stage IA in 118 patients (7.7\%),
stage IB in 144 (9.4\%), stage IIA in 22 (1.4\%), stage IIB in 643
(42.2\%), and stage III in 598 (39.2\%). With the eighth edition, 774
patients (50.8\%) migrated to a different stage; 183 (12.0\%) were
reclassified to a lower stage and 591 (38.8\%) to a higher stage. Median
overall survival for the entire cohort was 24.4 months (95\% CI,
23.4-26.2 months). On Kaplan-Meier analysis, 5-year survival rates
changed from 38.2\% for patients in stage IA, 34.7\% in IB, 35.3\% in
IIA, 16.5\% in IIB, and 0\% in stage III (log-rank P \textless{} .001)
via classification with the seventh edition to 39.2\% for patients in
stage IA, 33.9\% in IB, 27.6\% in IIA, 21.0\% in IIB, and 10.8\% in
stage III (log-rank P \textless{} .001) with the eighth edition. For
patients who were node negative, the T stage was not associated with
prognostication of survival in either edition. In the eighth edition,
the N stage was associated with 5-year survival rates of 35.6\% in N0,
20.8\% in N1, and 10.9\% in N2 (log-rank P \textless{} .001). The C
statistic improved from 0.55 (95\% CI, 0.53-0.57) for the seventh
edition to 0.57 (95\% CI, 0.55-0.60) for the eighth edition. Conclusions
and Relevance: The eighth edition of the TNM staging system demonstrated
a more equal distribution among stages and a modestly increased
prognostic accuracy in patients with resected pancreatic ductal
adenocarcinoma compared with the seventh edition. The revised T stage
remains poorly associated with survival, whereas the revised N stage is
highly prognostic.

{}

{}

\begin{center}\rule{0.5\linewidth}{\linethickness}\end{center}

 - \textbf{A Refined Staging Model for Resectable Pancreatic Ductal
Adenocarcinoma Incorporating Examined Lymph Nodes, Location of Tumor and
Positive Lymph Nodes Ratio}

\emph{Journal of Cancer 2018 09;9(19):3507-3514}

PubMed: \url{https://www.ncbi.nlm.nih.gov/pubmed/?term=30310507}

Background: Nodal status and tumor site are prognostic factors for
resectable pancreatic ductal adenocarcinoma (PDAC). Parameters for nodal
status are diverse, and the number of examined lymph nodes (eNs) needed
for good prognosis are uncertain. We try to modify staging system of
resectable PDAC with parameters mentioned above by recursive
partitioning analysis. Methods: Patients from the Surveillance,
Epidemiology, and End Results (SEER) database were divided into training
cohort and internal validation cohort, randomly. PDAC patients from Sun
Yat-sen University Cancer Center were regarded as external validation
cohort. The training cohort was used to refine staging model by
recursive partitioning analysis, while the internal validation cohort
and the external validation cohort were applied to assess discriminatory
capacity of staging model. For parameters included in the modified
model, their effects were studied. Results: The number of eNs, tumor
site and tumor size were risk factors for positive nodal status. Lymph
nodes ratio (LNR), tumor site, eNs and T stages of 8th the American
Joint Committee on Cancer (AJCC) were selected to develop a refined
model, dividing patients into 5 groups of different outcomes, preceding
8th AJCC classification. Besides, we found that (1) for small PDAC
(diameter \textless{} 1cm), lymph node metastasis was rarely found; (2)
enough eNs were needed to ensure better prognosis of node-negative
patients; (3) tumors in the head of pancreas were prone to lymph nodes
metastasis; (4) for node-positive patients, LNR was a better nodal
parameter compared to positive lymph nodes (pNs). Conclusion: Our
improved staging system helps to illuminate the interactions among tumor
site, size and eNs.

{}

{}

\begin{center}\rule{0.5\linewidth}{\linethickness}\end{center}

 - \textbf{Evaluation of the prognostic value of the new AJCC 8th
edition staging system for patients with pancreatic adenocarcinoma; a
need to subclassify stage III?}

\emph{European journal of cancer (Oxford, England : 1990) 2018
Nov;104():62-69}

PubMed: \url{https://www.ncbi.nlm.nih.gov/pubmed/?term=30326370}

BACKGROUND: There have been several proposed changes for the 8th edition
of the American Joint Commission on Cancer (AJCC) for pancreatic
adenocarcinoma. The aim of this study was to evaluate the prognostic
value of the new staging system for patients with pancreatic
adenocarcinoma, especially in stage III patients. METHODS: We analysed
the data of patients newly diagnosed with pancreatic adenocarcinoma
between 2008 and 2016 at our hospital. Patients were staged according to
7th edition AJCC criteria, as well as the new 8th edition staging
system. The pathologic stage was used in the surgical cases, and the
clinical stage, determined by radiographic findings, was used in the
unresectable cases. RESULTS: Five hundred two patients were identified
who met the inclusion criteria. In node-negative patients, there were no
significant differences in survival among T 1, 2~and 3 groups according
to the 8th edition. The survival rates of patients with N1 (1-3 positive
nodes) and N2 (≥4 positive nodes) disease, according to 8th edition,
were significantly different (p~\textless{}~0.001). Although N2 and T4
patients are both stage III according to the new staging system, N2
patients had a better survival rate than T4 patients (p~=~0.038). The
new staging system stratifies patients more evenly across stages without
sacrificing the prognostic accuracy. CONCLUSIONS: The AJCC 8th edition
has some advantages over the previous version. However, patients with N2
and T4, who have been integrated into stage III, showed different
treatment modalities and prognoses, and we proposed dividing stage III
into IIIA (T1-3N2M0) and IIIB (T4NanyM0).

{}

{}

\begin{center}\rule{0.5\linewidth}{\linethickness}\end{center}

 - \textbf{Intra-Operative Frozen Section Histology of the Pancreatic
Resection Margins and Clinical Outcome of Patients with Adenocarcinoma
of the Head of the Pancreas Undergoing Pancreaticoduodenectomy}

\emph{Medical science monitor : international medical journal of
experimental and clinical research 2018 Jul;24():4905-4913}

PubMed: \url{https://www.ncbi.nlm.nih.gov/pubmed/?term=30007990}

BACKGROUND The aim of this study was to compare the clinical outcome in
patients with pancreatic ductal adenocarcinoma who underwent frozen
section and paraffin section histology of the surgical resection margins
during pancreaticoduodenectomy. MATERIAL AND METHODS Frozen section and
routine paraffin section histopathology were performed using the
following categories: R0 (no tumor cells at the surgical resection
margin), R1 (tumor cells at, or within 1 mm, of the surgical resection
margin), and R2 (tumor seen macroscopically at the surgical resection
margin). R1 and R2 patients underwent additional resection to achieve
R0. RESULTS Of 346 patients who underwent pancreaticoduodenectomy,
frozen section histology showed positive resection margins in 22
patients (9.2\%) and paraffin section histology was positive in 20
patients (8.4\%). The OS was nine months in frozen section-positive
patients and 20 months in frozen section-negative patients (p=0.001).
The OS rates were significantly different between the paraffin
section-positive and paraffin section-negative patients (11 months
vs.~21 months) (p=0.001). Univariate and multivariate analysis showed
that increased tumor size, high tumor grade, lymph node metastases, a
positive superior mesenteric artery and retroperitoneal margin, and a
positive resection margin on frozen section were significantly
correlated with reduced OS (p\textless{}0.05). Twenty-two patients with
positive resection margins on frozen section histology underwent further
resection; R0 was achieved in 14 patients, with no significant
difference in OS. CONCLUSIONS For patients who underwent
pancreaticoduodenectomy for pancreatic carcinoma with positive resection
margins on frozen section, further surgical resection to achieve R0 had
no significant positive impact on OS.

{}

{}

\begin{center}\rule{0.5\linewidth}{\linethickness}\end{center}

 - \textbf{Nomogram to Predict Cancer-Specific Survival in Patients with
Pancreatic Acinar Cell Carcinoma: A Competing Risk Analysis}

\emph{Journal of Cancer 2018 10;9(22):4117-4127}

PubMed: \url{https://www.ncbi.nlm.nih.gov/pubmed/?term=30519311}

Background: The objective of this study was to evaluate the probability
of cancer-specific death of patients with acinar cell carcinoma (ACC)
and build nomograms to predict overall survival (OS) and cancer-specific
survival (CSS) of these patients. Methods: Data were extracted from the
Surveillance, Epidemiology, and End Results (SEER) database. Patients
diagnosed with ACC between 2004 and 2014 were retrospectively collected.
Cancer-specific mortality and competing risk mortality were evaluated.
Nomograms for estimating 1-, 2- and 3-year OS and CSS were established
based on Cox regression model and Fine and Grey's model. The precision
of the 1-, 2- and 3-year survival of the nomograms was evaluated and
compared using the area under receiver operating characteristic (ROC)
curve (AUC). Results: The study cohort included 227 patients with ACC.
The established nomograms were well calibrated, and had good
discriminative ability, with a concordance index (C-index) of 0.742 for
OS prediction and 0.766 for CSS prediction. The nomograms displayed
better discrimination power than 7th or 8th edition
Tumor-Node-Metastasis (TNM) stage systems in training set and validation
set for predicting both OS and CSS. The AUC values of the nomogram
predicting 1-, 2-, and 3-year OS rates were 0.784, 0.797 and 0.805,
respectively, which were higher than those of 7th or 8th edition TNM
stage systems. Regard to the prediction of CSS rates, the AUC values of
the nomogram were also higher than those of 7th or 8th edition TNM stage
systems. Conclusion: We evaluated the 1-, 2- and 3-year OS and CSS in
patients with ACC for the first time. Our nomograms showed relatively
good performance and could be considered as convenient individualized
predictive tools for prognosis.

{}

{}

\begin{center}\rule{0.5\linewidth}{\linethickness}\end{center}

 - \textbf{Pancreatic cancer and autoimmune diseases: An association
sustained by computational and epidemiological case-control approaches}

\emph{International journal of cancer 2018 Sep;():}

PubMed: \url{https://www.ncbi.nlm.nih.gov/pubmed/?term=30229903}

Deciphering the underlying genetic basis behind pancreatic cancer (PC)
and its associated multimorbidities will enhance our knowledge toward PC
control. The study investigated the common genetic background of PC and
different morbidities through a computational approach and further
evaluated the less explored association between PC and autoimmune
diseases (AIDs) through an epidemiological analysis. Gene-disease
associations (GDAs) of 26 morbidities of interest and PC were obtained
using the DisGeNET public discovery platform. The association between
AIDs and PC pointed by the computational analysis was confirmed through
multivariable logistic regression models in the PanGen European
case-control study population of 1,705 PC cases and 1,084 controls.
Fifteen morbidities shared at least one gene with PC in the DisGeNET
database. Based on common genes, several AIDs were genetically
associated with PC pointing to a potential link between them. An
epidemiologic analysis confirmed that having any of the nine AIDs
studied was significantly associated with a reduced risk of PC (Odds
Ratio (OR) = 0.74, 95\% confidence interval (CI) 0.58-0.93) which
decreased in subjects having ≥2 AIDs (OR = 0.39, 95\%CI 0.21-0.73). In
independent analyses, polymyalgia rheumatica, and rheumatoid arthritis
were significantly associated with low PC risk (OR = 0.40, 95\%CI
0.19-0.89, and OR = 0.73, 95\%CI 0.53-1.00, respectively). Several
inflammatory-related morbidities shared a common genetic component with
PC based on public databases. These molecular links could shed light
into the molecular mechanisms underlying PC development and
simultaneously generate novel hypotheses. In our study, we report sound
findings pointing to an association between AIDs and a reduced risk of
PC.

{}

{}

\begin{center}\rule{0.5\linewidth}{\linethickness}\end{center}

 - \textbf{Impact of tumor size on survival of patients with resected
pancreatic ductal adenocarcinoma: a systematic review and meta-analysis}

\emph{BMC cancer 2018 Oct;18(1):985}

PubMed: \url{https://www.ncbi.nlm.nih.gov/pubmed/?term=30326871}

BACKGROUND: The impact of tumor size on prognosis for surgically treated
patients with pancreatic ductal adenocarcinoma (PDAC) remains
controversial. A systematic review and meta-analysis was performed to
evaluate this issue. METHODS: Relevant studies published from January
2000 to June 2017 were identified through EMBASE and PUBMED. Data were
pooled for meta-analysis using Review Manager 5.3. RESULTS: Twenty eight
observational studies involving a total of 23,945 patients were
included. Tumors \textgreater{} 2~cm was associated with poor prognosis:
the pooled hazard ratio (HR) estimate for overall survival was 1.52
(95\% confidence interval {[}CI{]}: 1.41-1.64; P \textless{} 0.0001) by
univariate analysis and 1.61 (95\% CI: 1.35-1.91; P \textless{} 0.0001)
by multivariate analysis; the pooled HR estimate for disease-free
survival was 1.74 (95\% CI: 1.46-2.07; P \textless{} 0.0001) by
univariate analysis and 1.38 (95\% CI: 1.12-1.68; P = 0.002) by
multivariate analysis. When compared with patients with tumors ≤2~cm,
those with the tumors \textgreater{} 2~cm had higher incidences of lymph
node metastasis, poor tumor differentiation, lymph vessel invasion,
vascular invasion, perineural invasion, and positive intraoperative
peritoneal cytology. CONCLUSION: These data demonstrate that PDAC size
\textgreater{} 2~cm is an independent predictive factor for poor
prognosis after surgical resection and associated with more aggressive
tumor biology.

{}

{}

\begin{center}\rule{0.5\linewidth}{\linethickness}\end{center}

 - \textbf{The New American Joint Committee on Cancer TNM Staging System
for Pancreatic Cancer-Balancing Usefulness With Prognostication}

\emph{JAMA surgery 2018 Dec;153(12):e183629}

PubMed: \url{https://www.ncbi.nlm.nih.gov/pubmed/?term=30285059}

{}

{}

\begin{center}\rule{0.5\linewidth}{\linethickness}\end{center}

 - \textbf{Role of adjuvant therapy in resected stage IA subcentimeter
(T1a/T1b) pancreatic cancer}

\emph{Cancer 2019 Jan;125(1):57-67}

PubMed: \url{https://www.ncbi.nlm.nih.gov/pubmed/?term=30457666}

BACKGROUND: The standard of care for patients with resected stage I to
stage III pancreatic ductal adenocarcinoma (PDAC) is adjuvant
gemcitabine-based chemotherapy. The role of adjuvant treatment in
patients with subcentimeter, stage IA PDAC is unknown. The current study
evaluated the effect of adjuvant treatment on survival outcomes among
patients with American Joint Committee on Cancer/International Union
Against Cancer stage IA (T1N0) resected PDAC using the National Cancer
Data Base (NCDB). METHODS: A retrospective review of the NCDB was
conducted for patients diagnosed with T1 (tumor limited to the pancreas
and measuring ≤2 cm in greatest dimension), lymph node-negative (N0),
resected PDAC between 2004 and 2013. Patient demographics, histology,
adjuvant treatment, and survival trends were examined. Kaplan-Meier
analysis and log-rank tests were performed to determine the unadjusted
association between overall survival (OS), tumor size, and treatment.
RESULTS: A total of 876 patients met the inclusion criteria. The
patients had a mean age of 66.2 years (range, 32-90 years);
approximately 83.3\% were white (730 patients) and 53.1\% were female
(465 patients). Approximately 45.9\% of the patients had moderately
differentiated tumor histology (402 patients); 70.0\% (613 patients) had
tumors measuring 1 to 2 cm (T1c) and 30.0\% (263 patients) had tumors
measuring \textless{}1 cm (T1a/T1b). Approximately 94.2\% of patients
had negative surgical margins (815 patients) and 46.9\% (410 patients)
received adjuvant therapy. The median OS was significantly different for
patients who received adjuvant therapy compared with patients who did
not (70.7 months vs 46.9 months; P = .0001). For patients with tumors
measuring \textless{}1 cm, survival was not found to be significantly
different between patients who received adjuvant treatment compared with
those who did not (not reached vs 85.3 months; P = .54). In the
multivariable analysis, none of the covariates (treatment group,
Charlson-Deyo Score, age, insurance, and facility status) demonstrated
significant differences for patients with tumors measuring \textless{}1
cm. CONCLUSIONS: The current study is the first to demonstrate no
survival benefit for adjuvant therapy in patients with resected
subcentimeter PDAC.

{}

{}

\begin{center}\rule{0.5\linewidth}{\linethickness}\end{center}

 - \textbf{Reappraising the Concept of Conditional Survival After
Pancreatectomy for Ductal Adenocarcinoma: A Bi-institutional Analysis}

\emph{Annals of surgery 2018 Oct;():}

PubMed: \url{https://www.ncbi.nlm.nih.gov/pubmed/?term=30339622}

OBJECTIVE: To reappraise the concept of conditional survival (CS)
following pancreatectomy for pancreatic ductal adenocarcinoma (PDAC),
accounting for the patient's present disease status relative to
recurrence. BACKGROUND: CS, defined as the probability of surviving an
additional time frame based on accrued lifespan, offers dynamic survival
projections as compared with baseline overall survival. METHODS:
Patients undergoing pancreatectomy for PDAC at 2 institutions from 2000
to 2013 were retrospectively analyzed. The 12-month CS was estimated
separately for patients who were disease-free or with recurrence at the
given time points. Next, the conditional probability of reaching
60-months of survival was examined in each conditioning set across
strata of prognostic covariates, including American Joint Committee on
Cancer stage, tumor grade, R-status, and adjuvant treatment. RESULTS:
The study population consisted of 1005 patients. In disease-free
patients, the 12-month CS increased as a function of time already
survived, showing an opposite trend compared with overall survival. In
patients who recurred, the 12-month CS was lower than the disease-free
counterpart, especially within 24 months postoperatively. When
stratifying by the levels of prognostic covariates, the 60-months CS
estimates for disease-free patients tended to level off progressively,
indicating that factors independently associated with survival at the
time of pancreatectomy lost power over time. This concept did not apply
to the conditioning set of patients with recurrence, where CS estimates
across variables strata diverged with accrued lifespan. CONCLUSION: This
paper provides new information on how prognosis following pancreatectomy
for PDAC evolves over time, adjusting for the time the patient already
survived, and for the patient's present disease status relative to
recurrence.

{}

{}

\begin{center}\rule{0.5\linewidth}{\linethickness}\end{center}

 - \textbf{Association Between Very Small Tumor Size and Decreased
Overall Survival in Node-Positive Pancreatic Cancer}

\emph{Annals of surgical oncology 2018 Dec;25(13):4027-4034}

PubMed: \url{https://www.ncbi.nlm.nih.gov/pubmed/?term=30298331}

BACKGROUND: In pancreatic adenocarcinoma (PDAC), increasing tumor size
usually correlates with a worse prognosis. However, patients with a very
small primary tumor who experience lymph node involvement may have a
different disease biology. This study sought to determine the
interaction between tumor size and lymph node involvement in terms of
overall survival (OS). METHODS: The study identified 17,073 patients
with a diagnosis of M0 resected PDAC between 1983 and 2013 using the
Surveillance, Epidemiology, and End Results database. The patients were
stratified by lymph node involvement (N0 vs N+) and T stage (T1a-T1b vs
T1c vs T2 vs T3 vs T4). The Kaplan-Meier method was used to estimate OS,
and Cox regression analysis was used to compare survival between
subgroups after adjustment for patient-specific factors. RESULTS: Lymph
node involvement and T stage significantly interacted
(p \textless{} 0.001). Among the patients with node-negative disease,
5-year OS decreased monotonically with increasing T stage (59.1\%,
30.6\%, 22.9\%, 16.6\%, and 8.0\%, respectively; p \textless{} 0.001).
In contrast, among the patients with node-positive disease, those with
T1a-T1b tumors (\textless{} 10~mm) had worse 5-year OS than those with
T1c tumors (7.4\% vs 17.6\%; adjusted hazard ratio, 0.70; 95\%
confidence interval, 0.50-0.97; p = 0.034) and similar survival compared
with those who had T2, T3, or T4 tumors (9.7\%, 8.2\%, and 4.8\%,
respectively; p \textgreater{} 0.2 in all cases). CONCLUSIONS: Among
patients with lymph node-positive PDAC, very small primary tumors are
associated with decreased OS. This finding raises the possibility that
small tumors capable of lymph node metastasis might represent more
biologically aggressive cancers.

{}

{}

\begin{center}\rule{0.5\linewidth}{\linethickness}\end{center}

 - \textbf{Stratified survival of resected and overall pancreatic cancer
patients in Europe and the USA in the early twenty-first century: a
large, international population-based study}

\emph{BMC medicine 2018 08;16(1):125}

PubMed: \url{https://www.ncbi.nlm.nih.gov/pubmed/?term=30126408}

BACKGROUND: The prognosis of pancreatic cancer (PaC) strongly varies
across different stages and age groups, which has unfortunately not been
well recorded in the literature. This international population-based
study aimed to provide tumor-node-metastasis (TNM) stage- and
age-specific survival estimates and trends in resected and overall
(resected and unresected) PaC in the early twenty-first century.
METHODS: Using data from the US Surveillance, Epidemiology, and End
Results-18 Program and the national cancer registries of the
Netherlands, Belgium, Norway, and Slovenia, short-term and long-term
overall survival results stratified by TNM stage and age in resected and
overall primary PaC, irrespective of being microscopically confirmed or
not, in 2003-2014 were computed using the Kaplan-Meier method. The
temporal survival trends over three predefined periods (2003-2005,
2006-2008, and 2009-2011) were further examined using the log-rank test.
RESULTS: In total, data for 125,183 patients were analyzed. Overall,
age-stratified 3-year survival was 20-34\% (\textless{} 60~years),
14-25\% (60-69~years), and 9-13\% (≥ 70~years) in stages I-II PaC; and
2-5\% (\textless{} 60~years), 1-2\% (60-69~years), and \textless{} 1-1\%
(≥ 70~years) in stages III-IV cancer. Patients who underwent operation
had higher 3-year survival in each stage and age group (stages I-II:
23-39\% (\textless{} 60~years), 16-31\% (60-69~years), and 17-30\%
(≥ 70~years); stages III-IV: 5-19\% (\textless{} 70~years) and 2-14\%
(≥ 70~years)). Perioperative survival also decreased with advancing
stage and older age (stages I-II: 98-100\% (\textless{} 60~years),
97-99\% (60-69~years), and 94-99\% (≥ 70~years); stages III-IV: 94-99\%
(\textless{} 70~years) and 81-96\% (≥ 70~years)). Between 2003 and 2005
and 2009-2011, for overall PaC, both short-term and long-term survival
improvements were observed in all countries except Belgium; for resected
disease, short-term improvements were present only in the USA and
Slovenia, but long-term improvements were observed in all countries
except Slovenia, with stage-specific variations. CONCLUSIONS: Our large
international study provides TNM stage- and age-specific
population-based survival in overall and resected PaC that will
facilitate clinical counseling. While the survival expectations for
patients with resected PaC are substantially higher than the widely
available and known dismal survival predictions for overall patients,
conclusions on the benefits of resection cannot be made from this
observational study. Patients with advanced-stage disease and/or older
age should undergo careful risk assessment before treatment. Limited but
inspiring improvement in survival is observed.

{}

{}

\begin{center}\rule{0.5\linewidth}{\linethickness}\end{center}

Back to top

\begin{center}\rule{0.5\linewidth}{\linethickness}\end{center}

\pagebreak

\hypertarget{preneoplastic-and-preinvasive-lesions}{%
\paragraph{Preneoplastic and Preinvasive
Lesions}\label{preneoplastic-and-preinvasive-lesions}}

Preneoplastic and Preinvasive Lesions, PanIN, IPMN, MCN, ICPN

\begin{center}\rule{0.5\linewidth}{\linethickness}\end{center}

 - \textbf{Toll-like receptors 2, 4 and 9 and hypoxia markers HIF-1alpha
and CAIX in pancreatic intraepithelial neoplasia}

\emph{APMIS : acta pathologica, microbiologica, et immunologica
Scandinavica 2018 Nov;126(11):852-863}

PubMed: \url{https://www.ncbi.nlm.nih.gov/pubmed/?term=30357962}

Pancreatic cancer arises from precursor lesions called pancreatic
intraepithelial neoplasia (PanIN) characterized by inflammatory
microenvironment. In pancreatic cancer, strong innate immunity and
hypoxia responses are typical. Occurrence and relationship of these
responses in human PanINs is unknown. We have studied the expression of
toll-like receptors (TLR) TLR2, TLR4 and TLR9, and hypoxia markers
HIF-1alpha and Carbonic anhydrase IX (CAIX) in normal and inflamed
pancreatic ducts, in PanINs and in cancers. The samples of 69 surgically
resected pancreatic ductal adenocarcinoma patients were stained using
immunohistochemistry. We found TLR2, TLR9, HIF-1alpha and CAIX to be
prominently expressed in pancreatic intraepithelial neoplasia.
Expression of TLR2 showed a linear increase from PanIN1 to PanIN3, while
the highest TLR4 expression was detected in inflamed ducts, and TLR9
expression in PanIN1 lesions. Within the PanIN1-group, nuclear
HIF-1alpha correlated with membranous and cytoplasmic TLR2 expression
(ρ~=~0.982 and 0.815; p~\textless{}~0.001 and p~=~0.025, respectively),
and in the PanIN2-group nuclear HIF-1alpha correlated with nuclear TLR9
expression 0.636, p~=~0.026). Our findings show that the expression of
TLRs 2, 4 and 9, and hypoxia markers HIF-1alpha and CAIX is abnormal in
pancreatic intraepithelial neoplasia suggesting that both the innate
immunity activation and hypoxia response are involved in early
pancreatic carcinogenesis. However, these processes might be
independent.

{}

{}

\begin{center}\rule{0.5\linewidth}{\linethickness}\end{center}

 - \textbf{Pathways of Progression From Intraductal Papillary Mucinous
Neoplasm to Pancreatic Ductal Adenocarcinoma Based on Molecular
Features}

\emph{Gastroenterology 2018 Oct;():}

PubMed: \url{https://www.ncbi.nlm.nih.gov/pubmed/?term=30342036}

BACKGROUND \& AIMS: Intraductal papillary mucinous neoplasms (IPMNs) are
regarded as precursors of pancreatic ductal adenocarcinomas (PDAs), but
little is known about mechanism of progression. This makes it a
challenge to assess cancer risk in patients with IPMNs. We investigated
associations of IPMNs with concurrent PDAs by genetic and histologic
analyses. METHODS: We obtained 30 pancreatic tissues with concurrent
PDAs and IPMNs; 168 lesions, including incipient foci, were mapped,
microdissected, and analyzed for mutations in 18 pancreatic
cancer-associated genes and expression of tumor suppressors. RESULTS: We
determined the clonal relatedness of lesions, based on driver mutations
shared by PDAs and concurrent IPMNs, and classified the lesions into 3
subtypes. Twelve PDAs contained driver mutations shared by all
concurrent IPMNs, which we called the sequential subtype. This subset
was characterized by less diversity in incipient foci with frequent GNAS
mutations. Eleven PDAs contained some driver mutations that were shared
with concurrent IPMNs, which we called the branch-off subtype. In this
subtype, PDAs and IPMNs had identical KRAS mutations but different GNAS
mutations, although the lesions were adjacent. Whole-exome sequencing
and methylation analysis of these lesions indicated clonal origin with
later divergence. Ten PDAs had driver mutations not found in concurrent
IPMNs, called the de novo subtype. Expression profiles of TP53 and SMAD4
increased our ability to differentiate these subtypes compared with
sequencing data alone. The branch-off and de novo subtypes had
substantial heterogeneity among early clones, such as differences in
KRAS mutations. Patients with PDAs of the branch-off subtype had a
longer times of disease-free survival than patients with PDAs of the de
novo or the sequential subtypes. CONCLUSIONS: Detailed histologic and
genetic analysis of PDAs and concurrent IPMNs identified 3 different
pathways by which IPMNs progress to PDAs-we call these the sequential,
branch-off, and de novo subtypes. Subtypes might associate with clinical
and pathology features and be used to select surveillance programs for
patients with IPMNs.

{}

{}

\begin{center}\rule{0.5\linewidth}{\linethickness}\end{center}

 - \textbf{Increased SOX9 Expression in Premalignant and Malignant
Pancreatic Neoplasms}

\emph{Annals of surgical oncology 2018 Oct;():}

PubMed: \url{https://www.ncbi.nlm.nih.gov/pubmed/?term=30357576}

BACKGROUND: SOX9, a progenitor cell marker, is important for pancreatic
ductal development. Our goal was to examine SOX9 expression differences
in intraductal papillary mucinous neoplasms (IPMNs) and ductal
adenocarcinoma (PDAC) compared with benign pancreatic duct (BP).
METHODS: SOX9 expression was evaluated by immunohistochemistry performed
on 93 specimens: 37 BP, 24 low grade (LG) IPMN, 12 high grade (HG) IPMN,
and 20 PDAC. A linear mixed-effects model was used to compare the
percentage of cells expressing SOX9 by specimen type. A separate linear
mixed-effects model evaluated differences in SOX9 expression by staining
intensity in pancreatic epithelial cells. RESULTS: Nuclear SOX9
expression was detected in the epithelial cells of 98\% HG IPMN, 93\% LG
IPMN, 83\% PDAC, and 60\% BP. Compared with BP, SOX9 was expressed from
a significantly greater percentage of cells in LG IMPN, HG IMPN, and
PDAC (p \textless{} 0.001 for each). BP and PDAC showed greater
variability in SOX9 expression in epithelial cells compared with IPMNs
which showed strong, homogenous SOX9 expression in almost all cells.
Compared with BP, both LG and HG IPMN showed significantly greater SOX9
expression (p \textless{} 0.001 for each), but there was no significant
difference in SOX9 expression between LG and HG IPMN
(p \textgreater{} 0.05). PDAC had significantly higher expression of
SOX9 compared with BP but significantly lower SOX9 expression compared
with LG or HG IPMN (p \textless{} 0.001 for each). CONCLUSIONS: IPMNs
demonstrated the highest expression levels of SOX9. SOX9 expression in
BP and PDAC demonstrated much more heterogeneity compared with the
strong, uniform expression in IPMN.

{}

{}

\begin{center}\rule{0.5\linewidth}{\linethickness}\end{center}

 - \textbf{Adjuvant chemotherapy is associated with improved
postoperative survival in specific subtypes of invasive intraductal
papillary mucinous neoplasms (IPMN) of the pancreas: it is time for
randomized controlled data}

\emph{HPB : the official journal of the International Hepato Pancreato
Biliary Association 2018 Oct;():}

PubMed: \url{https://www.ncbi.nlm.nih.gov/pubmed/?term=30366881}

BACKGROUND: Very little is known about adjuvant chemotherapy for
invasive Intraductal Papillary Mucinous Neoplasms (IPMNs) of the
pancreas. The aim was to assess whether adjuvant chemotherapy affects
survival. METHODS: Retrospective evaluation of invasive IPMNs. Patients
treated with surgery alone or followed by adjuvant chemotherapy were
compared in terms of survival. RESULTS: A total of 102 invasive IPMNs
were analyzed. Median follow-up was 72 (5-318) months and 18.6\%
received adjuvant chemotherapy. Overall, recurrence rate was 40.2\%,
while 5-year overall survival and disease specific survival (DSS) were
65.3\% and 69.4\%, respectively. N1 disease (HR5.58, CI95\% 2.49-12.51,
p \textless{} 0.01), tubular type (HR2.35, CI95\% 1.71-4.82, p = 0.05)
and G3 tumors (HR4.54, CI95\% 2.12-15.49, \textless{}0.01) were
predictors of reduced DSS. Overall, there was no difference in the
5-year DSS comparing patients treated with adjuvant chemotherapy to
surgery alone (61.8 vs.~69.4\%, p = 0.8). Adjuvant chemotherapy
significantly improved DSS only in N1 (5-years-DSS 76 vs.~35.8\%, p =
0.01) and tubular carcinomas (5-years-DSS 88.9 vs.~53\%, p = 0.03).
CONCLUSIONS: Adjuvant therapy improves survival only in invasive IPMNs
with nodal disease or tubular differentiation. Future trials are needed
to improve the level of evidence about adjuvant chemotherapy.

{}

{}

\begin{center}\rule{0.5\linewidth}{\linethickness}\end{center}

 - \textbf{High-grade Dysplasia in Resected Main-duct Intraductal
Papillary Mucinous Neoplasm (MD-IPMN) is Associated with an Increased
Risk of Subsequent Pancreatic Cancer}

\emph{The American journal of gastroenterology 2018 Nov;():}

PubMed: \url{https://www.ncbi.nlm.nih.gov/pubmed/?term=30413822}

BACKGROUND: There is lack of consensus on post-operative surveillance
for resected non-invasive intraductal papillary neoplasms (IPMNs). In
this study we explored risk factors for subsequent PC in patients with
MD-IPMN undergoing partial pancreatectomy. METHODS: We searched the Mayo
Clinic surgical pathology database for all cases of resected MD-IPMN
between 1997 and 2014. Cases with histologically confirmed main
pancreatic duct involvement either isolated or in a mixed pattern with
branch-duct involvement were included. Outcomes of PC in the remnant
pancreas, and death related to MD-IPMN were assessed with survival
analyses (Kaplan-Meier and Cox regression). RESULTS: Among the 179
patients with resected MD-IPMN the incidence of concomitant PC and
high-grade dysplasia (HGD) in the resected specimen was 23 and 14\%,
respectively. The mean duration of follow-up was 4.31 years (range
0.12-13.5 years). Excluding 28 subjects who either underwent initial
total pancreatectomy or partial pancreatectomy with surgical margins
positive for PC/HGD, the 5-year incidence of subsequent PC was 12\%,
including 60.6\% and15.6\% in those with initial PC and HGD,
respectively. The 10-year incidence of PC was 21.2\% overall, 60.6\% for
PC, 38.3\% for HGD, and 3.0\% for LGD. Risk of subsequent PC was
significantly higher for those with initial PC compared with HGD
(HR = 4.95, 95\% CI: 1.63-15.03, p = 0.005 and for HGD compared with LGD
(HR = 11.30, 95\% CI: 1.55-82.26, p = 0.017). CONCLUSION: Patients with
MD-IPMN with PC or HGD undergoing segmental pancreatectomy are at higher
risk of subsequent PC and may benefit from post-operative surveillance.
The post-operative surveillance intervals in resected MD- IPMNs need to
be tailored based on dysplasia grade.

{}

{}

\begin{center}\rule{0.5\linewidth}{\linethickness}\end{center}

 - \textbf{Cyst location and presence of high grade dysplasia or
invasive cancer in intraductal papillary mucinous neoplasms of the
pancreas: a seven institution study from the central pancreas
consortium}

\emph{HPB : the official journal of the International Hepato Pancreato
Biliary Association 2018 Oct;():}

PubMed: \url{https://www.ncbi.nlm.nih.gov/pubmed/?term=30361110}

BACKGROUND: Traditionally, intraductal papillary mucinous neoplasms
(IPMNs) of the pancreas with ``high risk stigmata'' (HRS) or ``worrisome
features'' (WF) are referred for resection. We aim to assess if IPMN
location is predictive of harboring either high grade dysplasia (HGD) or
invasive cancer (IC). METHODS: Patients undergoing resection for IPMN
from seven institutions between 2000 and 2015 (n~=~275) were analyzed.
HRS and WF were defined by the 2012 Fukuoka international consensus
guidelines. RESULTS: 168 (61\%) patients had head/uncinate cysts, while
107 (39\%) had neck/body/tail cysts. No differences were noted between
groups with regard to age, duct type, cyst size, or presence of at least
one WF. Patients with cysts in the head/uncinate were more often male
(55\% vs.~40\%), had at least one HRS (24\% vs.~11\%), and more often
harbored HGD or IC(49\% vs.~27\%){[}all p~\textless{}~0.05{]}. On
multivariate analysis, only cyst location in the head/uncinate remained
associated with presence of HGD or IC(odds ratio 4.76, p~=~0.02).
DISCUSSION: Cyst location is predictive of HGD or IC in patients with
IPMNs. Head/uncinated cysts are more likely to harbor malignancy
compared to those of the neck/body/tail. Additional studies are needed
to confirm these findings, however, cyst location should be considered
part of the decision making process for surveillance vs.~resection for
IPMNs.

{}

{}

\begin{center}\rule{0.5\linewidth}{\linethickness}\end{center}

 - \textbf{Concomitant Intraductal Papillary Mucinous Neoplasm in
Pancreatic Ductal Adenocarcinoma Is an Independent Predictive Factor for
the Occurrence of New Cancer in the Remnant Pancreas}

\emph{Annals of surgery 2018 Oct;():}

PubMed: \url{https://www.ncbi.nlm.nih.gov/pubmed/?term=30308608}

OBJECTIVE: To determine the factors predicting the subsequent
development of pancreatic ductal adenocarcinoma in remnant pancreas
(PDAC-RP) after partial pancreatectomy for PDAC. SUMMARY BACKGROUND
DATA: PDAC-RP after partial pancreatectomy for PDAC is currently not so
rare because of improved prognosis of PDAC patients due to recent
advances in surgical techniques and adjuvant therapy. However, the
predictive factors related to PDAC-RP remain unknown. METHODS: We
retrospectively reviewed the clinicopathological data of a consecutive
series of 379 patients with PDAC treated by partial pancreatectomy
between 1992 and 2015; 14 patients (3.69\%) had PDAC-RP.
Clinicopathological variables were compared between PDAC-RP and
non-PDAC-RP. RESULTS: In univariate analysis, concomitant intraductal
papillary mucinous neoplasm (IPMN) (P = 0.0005), cancer location
(body/tail) (P = 0.0060), and lower T factor in UICC (P = 0.0039) were
correlated with PDAC-RP development. Multivariate analysis revealed
concomitant IPMN (P = 0.0135) to be an independent predictive factor for
PDAC-RP. PDAC concomitant with IPMN had higher cumulative incidence of
PDAC-RP (47.5\%/10 yrs) than PDAC without IPMN (9.96\%/10 yrs) (P =
0.0071). Moreover, the density of pancreatic intraepithelial neoplasia
lesions in the background pancreas of cases of PDAC concomitant with
IPMN (1.86/cm) was higher than that of cases of PDAC without IPMN
(0.91/cm) (P = 0.0007). CONCLUSIONS: Concomitant IPMN in PDAC is an
independent predictive factor for the development of new PDAC in remnant
pancreas. Cancer susceptibility of remnant pancreas after resection for
PDAC concomitant with IPMN is probably due to an increased density of
pancreatic intraepithelial neoplasia lesions.

{}

{}

\begin{center}\rule{0.5\linewidth}{\linethickness}\end{center}

Back to top

\begin{center}\rule{0.5\linewidth}{\linethickness}\end{center}

\pagebreak

\hypertarget{pancreatitis-other-diseases}{%
\paragraph{Pancreatitis \& Other
Diseases}\label{pancreatitis-other-diseases}}

\begin{center}\rule{0.5\linewidth}{\linethickness}\end{center}

 - \textbf{Prior History of Pancreatitis Accelerates the Development of
Pancreatic Adenocarcinoma}

\emph{Pancreas 2019 11;47(10):1262-1266}

PubMed: \url{https://www.ncbi.nlm.nih.gov/pubmed/?term=30286010}

OBJECTIVES: Presentation of pancreatic adenocarcinoma (PC) as acute
pancreatitis (AP), association of chronic pancreatitis (CP) with PC, and
role of inflammation in PC carcinogenesis are well recognized. We
hypothesized that inflammatory changes associated with remote history of
AP (≥2 years before PC diagnosis) would result in earlier age of PC
diagnosis. METHODS: We evaluated PC patients prospectively enrolled in
the Pancreatic Adenocarcinoma Gene Environment Risk (PAGER) study at the
University of Pittsburgh for history of pancreatitis and reviewed
relevant medical records and imaging studies. Univariate and
multivariable linear regression analyses evaluated the relationship
between PC and remote history of AP. RESULTS: Among 790 patients with
histologically confirmed PC, 114 (14.4\%) had a history of pancreatitis
(AP within 2 years of PC diagnosis in 69 {[}8.7\%{]}, remote history of
AP in 28 {[}3.5\%{]}, CP in 4 {[}0.5\%{]}, and unknown duration of
pancreatitis in 13 {[}1.6\%{]}). After controlling for age, sex, body
mass index, smoking, alcohol history, and diabetic status at diagnosis,
patients with a remote history of AP were diagnosed on average 4.7 years
earlier with PC when compared with PC patients without history of AP (P
\textless{} 0.035). CONCLUSIONS: Remote history of AP may accelerate
carcinogenesis in PC.

{}

{}

\begin{center}\rule{0.5\linewidth}{\linethickness}\end{center}

 - \textbf{Pancreatic Cancer Following Acute Pancreatitis: A
Population-based Matched Cohort Study}

\emph{The American journal of gastroenterology 2018
Nov;113(11):1711-1719}

PubMed: \url{https://www.ncbi.nlm.nih.gov/pubmed/?term=30315287}

BACKGROUND: Acute pancreatitis is linked to pancreatic cancer, but the
direction of this association is not fully elaborated. METHODS: This was
a population-based cohort study including all Swedish residents
diagnosed with a first-time episode of acute pancreatitis between 1997
and 2013 and corresponding matched pancreatitis-free individuals from
the general population. Hazard ratios for the association between acute
pancreatitis and pancreatic cancer were estimated using multivariable
Cox regression models. RESULTS: Overall, 49,749 individuals with acute
pancreatitis and 138,750 matched individuals without acute pancreatitis
were followed up for 1,192,134 person-years (median 5.3 years). A total
of 769 individuals developed pancreatic cancer, of whom 536 (69.7\%) had
a history of acute pancreatitis. The risk of pancreatic cancer was
substantially increased during the first few years after a diagnosis of
acute pancreatitis but declined gradually over time, reaching a level
comparable to the pancreatitis-free population after \textgreater{}10
years of follow-up. In those with non-gallstone-related acute
pancreatitis, the risk of pancreatic cancer declined to a level
comparable to the pancreatitis-free population only when follow-up time
was censored for a second episode of acute pancreatitis or a diagnosis
of chronic pancreatitis. Increasing number of recurrent episodes of
acute pancreatitis was associated with increased risk of pancreatic
cancer. CONCLUSION: These findings imply a delay in the diagnosis of
pre-existing pancreatic cancer, if clinically presented as acute
pancreatitis. Any association between non-gallstone-related acute
pancreatitis and pancreatic cancer in the long-term (\textgreater{}10
years) could be mediated through recurrent acute pancreatitis or chronic
pancreatitis.

{}

{}

\begin{center}\rule{0.5\linewidth}{\linethickness}\end{center}

 - \textbf{Neonatal Diabetes: Two Cases with Isolated Pancreas Agenesis
due to Homozygous PTF1A Enhancer Mutations and One with Developmental
Delay, Epilepsy, and Neonatal Diabetes Syndrome due to KCNJ11 Mutation}

\emph{Journal of clinical research in pediatric endocrinology 2018
06;10(2):168-174}

PubMed: \url{https://www.ncbi.nlm.nih.gov/pubmed/?term=28943513}

Neonatal diabetes mellitus is a rare form of monogenic diabetes which is
diagnosed in the first six months of life. Here we report three patients
with neonatal diabetes; two with isolated pancreas agenesis due to
mutations in the pancreas-specific transcription factor 1A (PTF1A)
enhancer and one with developmental delay, epilepsy, and neonatal
diabetes (DEND) syndrome, due to a KCNJ11 mutation. The two cases with
mutations in the distal enhancer of PTF1A had a homozygous
g.23508363A\textgreater{}G and a homozygous g.23508437A\textgreater{}G
mutation respectively. Previous functional analyses showed that these
mutations can decrease expression of PTF1A which is involved in pancreas
development. Both patients were born small for gestational age to
consanguineous parents. Both were treated with insulin and pancreatic
enzymes. One of these patients' fathers was also homozygous for the
PTF1A mutation, whilst his partner and the parents of the other patient
were heterozygous carriers. In the case with DEND sydrome, a previosly
reported heterozygous KCNJ11 mutation, p.Cys166Tyr
(c.497G\textgreater{}A), was identified. This patient was born to
nonconsanguineous parents with normal birth weight. The majority of
neonatal diabetes patients with KCNJ11 mutations will respond to
sulphonylurea treatment. Therefore Glibenclamide, an oral antidiabetic
of the sulphonylurea group, was started. This treatment regimen
relatively improved blood glucose levels and neurological symptoms in
the short term. Because we could not follow the patient in the long
term, we are not able to draw conclusions about the efficacy of the
treatment. Although neonatal diabetes mellitus can be diagnosed
clinically, genetic analysis is important since it is a guide for the
treatment and for prognosis.

{}

{}

\begin{center}\rule{0.5\linewidth}{\linethickness}\end{center}

 - \textbf{Serum and histological IgG4-negative type 1 autoimmune
pancreatitis}

\emph{Clinical journal of gastroenterology 2018 Nov;():}

PubMed: \url{https://www.ncbi.nlm.nih.gov/pubmed/?term=30414073}

A 66-year-old man who was on oral medication for type 2 diabetes
experienced a rapid decline in glycemic control (increase in
glycosylated hemoglobin level from 7.7 to 10.2\% over 3~months).
Abdominal ultrasonography revealed a 20-mm hypoechoic mass in the
pancreatic tail. Serum tumor marker carbohydrate antigen 19-9 and DUPAN2
levels were within the respective normal ranges; serum IgG4 level was
also normal at 21.8~mg/dL. Abdominal contrast computed tomography
revealed a 26-mm tumor in the pancreatic tail. Magnetic resonance
cholangiopancreatography revealed disruption of the main pancreatic duct
and dilation of the caudal pancreatic duct. Endoscopic ultrasonography
revealed a near-round-shaped hypoechoic mass with interspersed
hyperechoic areas. Endoscopic ultrasonography-guided fine needle
aspiration was performed using a 22-G needle, but no malignant findings
were observed. There were no signs of sialadenitis, retroperitoneal
fibrosis, nephropathy, or other conditions associated with IgG4-related
diseases. Distal pancreatectomy was performed; a 23-mm white mass was
resected from the pancreatic tail. A histopathological examination
showed advanced inflammatory cell infiltration mainly involving
lymphocytes/plasma cells along with storiform fibrosis and obliterative
phlebitis. No more than five IgG4-positive cells were observed per
high-power field. These were level 1 pathological findings, and a
definitive diagnosis of type 1 autoimmune pancreatitis (AIP) was made
according to the International Consensus Diagnostic Criteria. Type 1 AIP
associated with normal serum IgG4 levels and absence of IgG4-positive
cells on histological examination is a rare clinical entity, which is
very difficult to distinguish from pancreatic cancer. Here we report
such a case and present a review of the relevant literature.

{}

{}

\begin{center}\rule{0.5\linewidth}{\linethickness}\end{center}

 - \textbf{The histologic diagnosis of IgG4-related disease on small
biopsies: Challenges and pitfalls}

\emph{Histopathology 2018 Nov;():}

PubMed: \url{https://www.ncbi.nlm.nih.gov/pubmed/?term=30408214}

INTRODUCTION: The pathologic diagnosis of IgG4-related disease (IgG4-RD)
relies on histology, IgG4 positive cells and increased IgG4:IgG. Small
biopsies from patients with a presumptive diagnosis of IgG4-RD often
fail to meet consensus histologic guidelines. We evaluate consecutive
small biopsies from patients with a presumptive diagnosis of IgG4-RD and
assess the significance of the pathologic findings. METHODS: We
evaluated 55 small biopsies from patients with a presumptive diagnosis
of IgG4-RD. The retrospective cohort comprised of 71 patients with
IgG4-RD and 57 mimics. We performed immunohistochemistry (IHC) and in
situ hybridization (ISH) for IgG4 and IgG. RESULTS: 26 patients from the
prospective cohort met histologic criteria for IgG4-RD (definite);
twenty-nine patients lacked one or more pathologic features
(borderline). Twenty biopsies (36\%) lacked both storiform fibrosis and
obliterative phlebitis, nine (16\%) lacked increase in IgG4 positive
plasma cells. 93\% of patients showed IgG4: total IgG of
\textgreater{}40\% (\textgreater{}30\% by ISH). There was no difference
in the incidence of multi-organ disease (p=0.9), serum IgG4 (p=0.6) and
response to therapy between the definite and borderline groups. A strong
correlation (Pearson 0.77) between the IHC and ISH platforms was noted
with regard to IgG4:total IgG. CONCLUSION: Patients with a presumptive
diagnosis of IgG4-RD but lacking characteristic pathologic features of
this disease appear clinically similar to those that meet current
pathologic guidelines. An elevated IgG4:total IgG is the most sensitive
pathologic feature and ISH provides a robust quantitation platform. We
recommend evaluating tumefactive lymphoplasmacytic infiltrates with
increased IgG4:IgG, regardless of histological features, for IgG4-RD.
This article is protected by copyright. All rights reserved.

{}

{}

\begin{center}\rule{0.5\linewidth}{\linethickness}\end{center}

 - \textbf{Autoimmune pancreatitis in children: A single centre
experience in diagnosis, management and long term follow up}

\emph{Pancreatology : official journal of the International Association
of Pancreatology (IAP) \ldots{} {[}et al.{]} 2018 Nov;():}

PubMed: \url{https://www.ncbi.nlm.nih.gov/pubmed/?term=30455055}

OBJECTIVES: Autoimmune pancreatitis (AIP) is a rare form of chronic
pancreatitis and data is limited in the paediatric population. We aim to
describe in detail a cohort of paediatric patients with AIP including
their presentation, investigations that led to their diagnosis,
management and long-term follow up. METHODS: We retrospectively reviewed
the data of 6 patients diagnosed with AIP over an 10-year period. Data
including demographics, clinical information, laboratory parameters,
serological markers, radiological and histological findings as well as
longitudinal follow up were collected. RESULTS: Out of the six patients,
one was diagnosed with definitive Type 1 AIP, two with definitive Type 2
AIP, two with probable Type 2 AIP and one with suspected Type 2 AIP.
Median time of follow up was 3.9 years (range 2.6-10.1). 4 patients had
pancreatic biopsies with 2 of these patients showing granulocytic
epithelial lesions (GELs). 4 patients received steroids and two of them
developed ulcerative colitis. Azathioprine was commenced on the patient
with Type 1 AIP to help her wean off steroids that caused significant
side effects on her. Only two patients developed exocrine insufficiency.
CONCLUSIONS: The long term follow up of our cohort of paediatric AIP
shows good prognosis. More follow up data on patients with AIP is needed
to help further characterize and define the disease.

{}

{}

\begin{center}\rule{0.5\linewidth}{\linethickness}\end{center}

Back to top

\begin{center}\rule{0.5\linewidth}{\linethickness}\end{center}

\pagebreak

\hypertarget{epidemiology-screening-cancer-risk-factors}{%
\paragraph{Epidemiology, Screening, Cancer Risk
Factors}\label{epidemiology-screening-cancer-risk-factors}}

\begin{center}\rule{0.5\linewidth}{\linethickness}\end{center}

 - \textbf{Prevalence of Asymptomatic Intraductal Papillary Mucinous
Neoplasms in Healthy and Ill Populations Detected by Ultrasonography: A
Single-Center Study of 6353 Outpatients}

\emph{Pancreas 2019 Jan;48(1):113-120}

PubMed: \url{https://www.ncbi.nlm.nih.gov/pubmed/?term=30451793}

OBJECTIVES: The aim of this study was to establish the prevalence of
intraductal papillary mucinous neoplasms (IPMNs) without and with
high-risk stigmata (HRS)/worrisome features (WF) and the epidemiologic
association between IPMNs and other diseases. METHODS: Ultrasound
examinations of outpatients were evaluated. The IPMN was confirmed by
magnetic resonance imaging. The prevalence of IPMNs and HRS/WF IPMNs was
calculated. The association between IPMNs and other diseases was
studied. RESULTS: The prevalence rate of IPMNs was 3.4\%. A total of
1,531,264 IPMNs were expected in Italian population (2.5\%), whereas
2257 per 100,000 citizens (2.3\%) were expected in the European standard
population (ESP2013). The prevalence rates of HRS/WF IPMNs were 0.5\%,
0.7\%, and 0.6\%, in our, the Italian, and the ESP2013 populations,
respectively. A total of 432,881 and 620 HRS/WF IPMNs per 100,000
residents were expected in the Italian and the ESP2013 populations,
respectively. The IPMN prevalence increased over 50 years of age (odds
ratio {[}OR{]}, 3.2; P \textless{} 0.001) and over 70 years of age (OR,
1.9; P \textless{} 0.001). Female sex was related to the presence of
IPMNs (OR, 1.9; P \textless{} 0.001). CONCLUSIONS: Intraductal papillary
mucinous neoplasms had a high prevalence in asymptomatic nonhospitalized
populations. Age older than 50 years identified a possible risk
category.

{}

{}

\begin{center}\rule{0.5\linewidth}{\linethickness}\end{center}

Back to top

\begin{center}\rule{0.5\linewidth}{\linethickness}\end{center}

\pagebreak

\hypertarget{bile-ducts}{%
\subsubsection{Bile Ducts}\label{bile-ducts}}

\begin{center}\rule{0.5\linewidth}{\linethickness}\end{center}

 - \textbf{Prognostic role of BAP-1 and PBRM-1 expression in
intrahepatic cholangiocarcinoma}

\emph{Virchows Archiv : an international journal of pathology 2018
Oct;():}

PubMed: \url{https://www.ncbi.nlm.nih.gov/pubmed/?term=30377796}

Intrahepatic cholangiocarcinoma (ICC) has universally poor outcome,
mainly due to its late clinical presentation. Identification of specific
biomarkers and development of effective treatment are still urgently
required. Mutations in PBRM-1 and BAP-1 genes, and the expression of
S100P have been related to survival in ICC. miR-31 seems also to play
important regulatory functions in ICC and it directly regulates BAP-1
expression in lung cancer. In this study, tissue expression of BAP-1,
PBRM-1, S100P, and miR-31 was investigated in ICC and correlated with
clinical-pathological features. Sixty-one consecutive patients who
underwent curative hepatic resection for ICC were enrolled. None
received any therapy prior to surgery. Immunostaining for BAP-1, PBRM-1,
and S100P, and in situ hybridization for miR-31 were performed, using
tissue microarray slides. A strong retained expression of BAP-1 and
PBRM-1 was associated with a reduced overall (p = 0.04 and p = 0.002,
respectively) and disease-free survival (p = 0.05 and p = 0.02,
respectively). An overexpression of S100P was related to a reduced
overall survival (p = 0.005). The multivariate analyses identified the
presence of perineural invasion and the retained PBRM-1 expression as
independent predictors of worse overall {[}p = 0.02, hazard ratio
(HR) = 2.25 (1.16-4.39) and p = 0.001, HR = 3.13 (1.56-6.28),
respectively{]} and disease-free survivals {[}p = 0.03, HR = 2.43
(1.09-5.4) and p = 0.03, HR = 2.51 (1.11-5.67), respectively{]}. An
overexpression of S100P was predictive of a worse overall survival
{[}p = 0.02, HR = 1.66 (1.08-2.55){]}. High levels of miR-31 were
significantly associated to a low expression of BAP-1 protein
(p = 0.03). In ICC, a retained expression of BAP-1 and PBRM-1, and an
overexpression of S100P are related to a poor prognosis.

{}

{}

\begin{center}\rule{0.5\linewidth}{\linethickness}\end{center}

 - \textbf{Mismatch repair deficiency is a rare but putative
therapeutically relevant finding in non-liver fluke associated
cholangiocarcinoma}

\emph{British journal of cancer 2018 Oct;():}

PubMed: \url{https://www.ncbi.nlm.nih.gov/pubmed/?term=30377340}

BACKGROUND: A major molecular pathway of genetic instability in cancer
is DNA mismatch repair deficiency. High-level microsatellite instability
(MSI-H) is currently the best predictor of responsiveness towards immune
checkpoint blockade. Data about the prevalence of high-level
microsatellite instability in cholangiocarcinoma (CCA) has been
conflicting. METHODS: We employed a cohort comprising 308 Western-world,
non-liver fluke-associated CCAs (159 intrahepatic, 106 perihilar, and 43
distal). We analysed the mononucleotide microsatellite instability
marker panel consisting of BAT25, BAT26, and CAT25 and detected MSI-H in
4/308 CCAs (1.3\%). RESULTS: Patients affected by MSI-H CCA had mostly
an atypical histomorphology (p = 0.004), showed a longer overall
survival, although having a high tumour stage, and were of younger age.
Correlation analysis of microsatellite instability status with
tumour-infiltrating immune cells, MHC I, and PD-L1 expression in the
same cholangiocarcinoma cohort showed higher numbers of CD8 + T cells,
FOXP3 + regulatory T cells, CD20 + B cells and high or at least moderate
MHC I expression levels in MSI-H CCAs. CONCLUSIONS: Even though the
overall number of MSI-H CCAs is low, the dismal prognosis of the disease
and the therapeutic option of immune checkpoint blockade in the
respective patients justify MSI testing of cholangiocarcinoma,
particularly in younger patients showing an atypical histomorphology.

{}

{}

\begin{center}\rule{0.5\linewidth}{\linethickness}\end{center}

 - \textbf{Outcomes of surgery for 2010 WHO classification-based
intraductal papillary neoplasm of the bile duct: Case-control study of a
single Japanese institution's experience with special attention to mucin
expression patterns}

\emph{European journal of surgical oncology : the journal of the
European Society of Surgical Oncology and the British Association of
Surgical Oncology 2018 Oct;():}

PubMed: \url{https://www.ncbi.nlm.nih.gov/pubmed/?term=30389302}

INTRODUCTION: The World Health Organization (WHO) proposed an integrated
classification for intraductal papillary neoplasm of the bile duct
(IPNB) in 2010. However, IPNB reportedly shows considerable geographic
variation. This Japanese single-institution study examined outcomes of
surgery for IPNB and the prognostic impact of immunohistochemical mucin
expression patterns. MATERIALS AND METHODS: Patients with IPNB were
identified from 413 patients who underwent curative-intent surgery for
biliary tract (excluding gallbladder) neoplasms from 1992 to 2016 by
retrospective macro- and microscopic reevaluation of resected specimens.
Their clinicopathological variables were analyzed. RESULTS: Twenty-two
(5\%) 2010 WHO classification-based patients with IPNB were identified.
The other 391 patients had common-type cholangiocarcinoma. The
histopathological grade was low/intermediate in 2 patients (9\%), high
in 8 (36\%), and invasive carcinoma (ICa) in 12 (55\%). The 10-year
overall survival rate was 100\% in 10 patients with low-high grade IPNB
and 69\% in 12 patients with ICa. These rates were significantly
(p\,=\,0.018) or marginally (p\,=\,0.089) better than that (38\%) of 391
other-cholangiocarcinoma patients. In the 12 patients with ICa, R0 or R1
resection, MUC5AC, and MUC6 expression significantly affected survival.
Notably, all seven patients with ICa exhibiting MUC5AC expression
survived throughout the study period, while four of five patients with
ICa who did not exhibit MUC5AC expression died of recurrence (with
vs.~without MUC5AC: 10-year overall survival, 100\% vs.~60\%,
respectively; p\,=\,0.018). CONCLUSION: Our 24-year, single
institution's experience suggests that Japanese patients with IPNB
favorably respond to surgery, even with ICa. MUC5AC and MUC6 expression
may be predictive of favorable outcomes.

{}

{}

\begin{center}\rule{0.5\linewidth}{\linethickness}\end{center}

 - \textbf{Comparison of the Clinicopathologic Characteristics of
Intraductal Papillary Neoplasm of the Bile Duct according to
Morphological and Anatomical Classifications}

\emph{Journal of Korean medical science 2018 Oct;33(42):e266}

PubMed: \url{https://www.ncbi.nlm.nih.gov/pubmed/?term=30310366}

Background: Intraductal papillary neoplasm of the bile duct (IPNB) is a
recently defined entity and its clinical characteristics and
classifications have yet to be established. We aimed to clarify the
clinical features of IPNB and determine the optimal morphological
classification criteria. Methods: From 2003 to 2016, 112 patients with
IPNB who underwent surgery were included in the analysis. After
pathologic reexamination by a specialized biliary-pancreas pathologist,
previously suggested morphological and anatomical classifications were
compared using the clinicopathologic characteristics of IPNB. Results:
In terms of histologic subtypes, most patients had the intestinal type
(n = 53; 48.6\%) or pancreatobiliary type (n = 33; 30.3\%). The simple
``modified anatomical classification'' showed that extrahepatic IPNB
comprised more of the intestinal type and tended to be removed by bile
duct resection or pancreatoduodenectomy. Intrahepatic IPNB had an
equally high proportion of intestinal and pancreatobiliary types and
tended to be removed by hepatobiliary resection. Morphologic
classifications and histologic subtypes had no effect on survival,
whereas a positive resection margin (75.9\% vs.~25.7\%; P = 0.004) and
lymph node metastasis (75.3\% vs.~30.0\%; P = 0.091) were associated
with a poor five-year overall survival rate. In the multivariate
analysis, a positive resection margin and perineural invasion were
important risk factors for survival. Conclusion: IPNB showed better
long-term outcomes after optimal surgical resection. The ``modified
anatomical classification'' is simple and intuitive and can help to
select a treatment strategy and establish the proper scope of the
operation.

{}

{}

\begin{center}\rule{0.5\linewidth}{\linethickness}\end{center}

 - \textbf{Cholangiocarcinoma: Classification, Histopathology and
Molecular Carcinogenesis}

\emph{Pathology oncology research : POR 2018 Nov;():}

PubMed: \url{https://www.ncbi.nlm.nih.gov/pubmed/?term=30448973}

Cholangiocarcinoma (CC) is the second most common tumor of the liver,
originating from the biliary system with increasing incidence and
mortality worldwide. Several new classifications review the significance
of tumor localization, site of origin, proliferation and biomarkers in
the intrahepatic, perihilar and distal forms of the lesion. Based on
growth pattern mass-forming, periductal-infiltrating, intraductal,
undefined and mixed types are differentiated. There are further
subclassifications which are applied for the histological features, in
particular for intrahepatic CC. Recognition of the precursors and early
lesions of CC including biliary intraepithelial neoplasia (BilIN),
intraductal papillary neoplasm of the bile ducts (IPNB), biliary
mucinous cystic neoplasm (MCNB) and the candidate precursors, such as
bile duct adenoma and von Meyenburg complex is of increasing
significance. In addition to the previously used biliary markers
detected by immunohistochemistry, several new markers have been added to
the differentiation of both the benign and malignant lesions, which can
be used to aid in the subclassification in association with the outcome
of CC. Major aspects of biliary carcinogenesis have been revealed, yet,
the exact way of this diverse process is still unclear. The factors
contributing to molecular cholangiocarcinogenesis include various risk
factors, different anatomical localizations, multiple cellular origins,
genetic and epigenetic alterations, tumor microenvironment,
heterogeneity and clonal evolution. Driver mutations have been
identified, implying that they are optimal candidates for targeted
therapy. The most promising therapeutic candidates have entered clinical
trials.

{}

{}

\begin{center}\rule{0.5\linewidth}{\linethickness}\end{center}

 - \textbf{Frequency of bile duct confluence variations in subjects with
pancreas divisum: an analysis of MRCP findings}

\emph{Diagnostic and interventional radiology (Ankara, Turkey) 2018
5;24(2):72-76}

PubMed: \url{https://www.ncbi.nlm.nih.gov/pubmed/?term=29757145}

PURPOSE: We aimed to evaluate the frequency of bile duct branching
pattern variations at the hepatic confluence in patients with pancreas
divisum (PD). METHODS: A search was performed through the hospital
database using the keyword ``pancreas divisum'' to identify patients.
The magnetic resonance cholangiopancreatography (MRCP) images of 137
patients who were diagnosed with PD between August 2011 and November
2016 were retrospectively analyzed for the presence of bile duct
variations. A control group of 137 patients without PD was established
among patients investigated during the same period. Variations of the
biliary tract were grouped into seven types according to the McSweeney
et al.~classification. RESULTS: Biliary tract variations were detected
in 103 of a total of 274 patients. Fifty-eight PD patients (42.3\%) and
45 control patients (32.8\%) had bile duct variation at the hepatic
confluence level. The patients with PD were more likely to have biliary
tract variation compared with the control group; however, it was not
statistically significant (P = 0.105). The most common variation in PD
patients was type 3a variation (16.8\%). CONCLUSION: MRCP studies showed
atypical bile duct confluence pattern in nearly half of both PD patients
and controls. There was no statistically significant difference in the
frequency of anatomic variations at bile duct confluence in patients
with PD versus those without PD. Derivation of these structures from
different outpouchings in early embryological life may explain this
insignificant difference.

{}

{}

\begin{center}\rule{0.5\linewidth}{\linethickness}\end{center}

 - \textbf{EVI1 expression is associated with aggressive behavior in
intrahepatic cholangiocarcinoma}

\emph{Virchows Archiv : an international journal of pathology 2018
Oct;():}

PubMed: \url{https://www.ncbi.nlm.nih.gov/pubmed/?term=30349952}

Ecotropic virus integration site 1 protein homolog (EVI1), a well-known
oncogenic transcriptional factor of hematopoietic cells, contributes to
pancreatic cancer oncogenicity through increased expression of KRAS.
Because EVI1 was upregulated in cholangiocarcinoma by referring The
Cancer Genome Atlas, we investigated the importance of EVI1 in
intrahepatic cholangiocarcinoma (ICC) which has been regarded as a
heterogeneous group of cancers. Immunohistochemical analysis results
demonstrated that EVI1 was overexpressed in about half of ICC (53/101,
52.5\%). Moreover, all intraductal papillary neoplasms of the bile duct
cases expressed EVI1 regardless of histological grading and subtypes
such as gastric, intestinal, pancreatobiliary, or oncocytic (20/20,
100\%). EVI1-positive ICC showed higher frequencies of aggressive
pathological indicators such as periductal infiltrative growth
(p = 0.022), hilar invasion (p = 0.041), advanced UICC stage
(p = 0.026), major vascular invasion (p = 0.026), and perineural
invasion (p = 0.007) than EVI1-negative ICC. Patients with EVI1-positive
ICC showed worse overall survival and recurrence-free survival in all
resected cases and in curative resected cases. Recently, we proposed
type 1/2 (large/small duct types) classification of ICC based on mucin
productivity and immunophenotypes (S100P, N-cadherin, and NCAM). Type 1
predominantly consisted of EVI1-positive ICC (33/42 cases, 79\%), and
the frequency was significantly higher than type 2 (18/55 cases, 32.7\%)
(p \textless{} 0.0001). EVI1-positive ICC was likely to express
stomach-specific claudin CLDN18 (correlation coefficient r = 0.55373)
and mucin MUC5AC (r = 0.42718). EVI1-positive ICC is an aggressive ICC
showing both large-duct and/or gastric phenotypes. Consequently, a
transcriptional factor EVI1 is associated with aggressive behavior in
ICC and can be a therapeutic target molecule, while EVI1 might be a key
molecule for the development of intraductal papillary neoplasms of the
bile duct.

{}

{}

\begin{center}\rule{0.5\linewidth}{\linethickness}\end{center}

Back to top

\begin{center}\rule{0.5\linewidth}{\linethickness}\end{center}

\pagebreak

\hypertarget{staging-1}{%
\paragraph{Staging}\label{staging-1}}

Bile Duct TNM staging, Margins, Survival

\begin{center}\rule{0.5\linewidth}{\linethickness}\end{center}

 - \textbf{Prognostic comparison of the 7th and 8th editions of the
American Joint Committee on Cancer staging system for intrahepatic
cholangiocarcinoma}

\emph{Journal of hepato-biliary-pancreatic sciences 2018
Apr;25(4):240-248}

PubMed: \url{https://www.ncbi.nlm.nih.gov/pubmed/?term=29450978}

BACKGROUND: Several important changes were made to the 8th edition of
the American Joint Committee on Cancer (AJCC) tumor staging system for
intrahepatic cholangiocarcinoma (ICC). We assessed the prognostic impact
of this new tumor staging system compared to the 7th edition. METHODS: A
retrospective single-institution study was performed with 626 patients
who underwent R0 resection for ICC over 20-year period. RESULTS:
Anatomical resection and concurrent bile duct resection were performed
in 571 (91.2\%) and 62 (9.9\%) patients, respectively. Cumulative tumor
recurrence and patient survival rates were 40.6\% and 73.3\% at 1~year;
66.7\% and 43.8\% at 3~years; 73.6\% and 30.4\% at 5~years; and 74.4\%
and 20.3\% at 10~years, respectively. Independent prognostic factors for
tumor recurrence and patient survival were multiple tumors, carbohydrate
antigen 19-9~\textgreater{}200~U/ml, tumor size \textgreater{}5~cm,
direct invasion to extrahepatic structure, and lymph node metastasis.
For tumor-node-metastasis stages in the 7th versus the 8th editions,
concordance index was 0.615 and 0.625 for tumor recurrence and 0.626 and
0.628 for patient survival, respectively. CONCLUSIONS: The 8th edition
of the AJCC staging system appears to provide high prognostic contrast
for T stage categories, except for T3. However, overall prognostic
performance of the 8th edition was not markedly improved over the 7th
edition.

{}

{}

\begin{center}\rule{0.5\linewidth}{\linethickness}\end{center}

Back to top

\begin{center}\rule{0.5\linewidth}{\linethickness}\end{center}

\pagebreak

\hypertarget{epidemiology-screening-cancer-risk-factors-1}{%
\paragraph{Epidemiology, Screening, Cancer Risk
Factors}\label{epidemiology-screening-cancer-risk-factors-1}}

\begin{center}\rule{0.5\linewidth}{\linethickness}\end{center}

 - \textbf{Pattern of distant metastases in primary extrahepatic
bile-duct cancer: A SEER-based study}

\emph{Cancer medicine 2018 Oct;7(10):5006-5014}

PubMed: \url{https://www.ncbi.nlm.nih.gov/pubmed/?term=30277653}

Extrahepatic bile duct cancer (EBDC) is a combined type of malignancy
mainly consisting of extrahepatic cholangiocarcinoma and gallbladder
cancer. Clinically, it is featured with latent symptoms and early
metastasis, leading to a poor prognosis. Therefore, this cohort study
aimed to depict the possible metastatic patterns of EBDC of diverse
sub-types and evaluate the prognostic significance of diverse metastatic
destinations with data from the clinical database. Relevant data of
total 4061 confirmed EBDC patients diagnosed between 2010 and 2013 from
the Surveillance, Epidemiology and End Results (SEER) database was
obtained. We applied t test to describe the baseline data of patients
included and used chi-square test to compare the distribution of distant
metastatic sites. We further adopted odds ratio assess the combined
metastatic patterns and compared survival difference of patients with
different distal metastasis organ by Kaplan-Meier analysis. We
identified totally 4061 patients over 18~years old diagnosed with
extrahepatic bile tract malignancies between 2010 and 2013, with clear
metastatic status and follow-up data, without primary malignancies.
Liver and distant lymph (DL) are the two most common sites as a single
metastasis organ. In combined metastasis patterns, bi-organ is more
frequent than the other types. Lung is the organ preferentially for
bi-organ metastasis, while bone and distant lymph similarly intend to
co-metastasize with brain. Distal metastasis in EBDC patients indicates
an extremely poor prognosis. According to the final analysis results,
malignancies in extrahepatic bile duct exhibit similar metastatic
patterns, suggesting that we can regard them as a unity to assess its
development. Profound differences exist in distribution of distant
extrahepatic metastatic sites and their combinations. Results from our
studies would provide some information for follow-up strategies and
future studies.

{}

{}

\begin{center}\rule{0.5\linewidth}{\linethickness}\end{center}

Back to top

\begin{center}\rule{0.5\linewidth}{\linethickness}\end{center}

\pagebreak

\hypertarget{gallbladder}{%
\subsubsection{Gallbladder}\label{gallbladder}}

\begin{center}\rule{0.5\linewidth}{\linethickness}\end{center}

\hypertarget{morphology-diagnostics-ihc-1}{%
\paragraph{Morphology, Diagnostics,
IHC}\label{morphology-diagnostics-ihc-1}}

Morphology, Diagnostics, IHC

\begin{center}\rule{0.5\linewidth}{\linethickness}\end{center}

 - \textbf{Intracholecystic papillary-tubular neoplasm of gallbladder: A
5-year retrospective pathological study}

\emph{Indian journal of pathology \& microbiology 2018 10;61(4):516-519}

PubMed: \url{https://www.ncbi.nlm.nih.gov/pubmed/?term=30303140}

Background: Intracholecystic papillary-tubular neoplasm (ICPN) is a
relatively new entity which includes neoplastic polyps, adenomas, and
papillary neoplasms that are ≥1.0 cm. This study is done to evaluate the
pathological features of ICPN and to find out the factors associated
with invasion. Materials and Methods: This is a 5-year retrospective
study in a referral pathology center. A total of 19 cases of ICPN are
found. The cases are analyzed for age and sex distribution, clinical
suspicion, stages, histological architecture, differentiation, and grade
of dysplasia. Descriptive statistics and test of significance by
Chi-square and t-test are used in the study. Results: ICPN comprises
23.5\% of all gallbladder neoplasms. Two-thirds of the cases were
suspected radiologically. Age range is 26-65 years with mean age of 50
years. They are 2.8 times more common in female. Approximately one-third
of the cases show invasion. The most common histological pattern is
papillary, followed by papillary-tubular and finally by tubular pattern.
Pyloric and biliary are the most common differentiation pattern followed
by oncocytic and intestinal pattern. About three-fourths of the cases
are associated with high-grade dysplasia mostly diffuse high-grade
dysplasia. Conclusion: We have found the younger age of presentation,
less proportion of invasive tumors, fewer tumors with biliary
phenotypes, and fewer tumors with high-grade dysplasia as compared to
previous studies. Factors significantly associated with invasion are
grade and extent of dysplasia particularly diffuse high-grade dysplasia.

{}

{}

\begin{center}\rule{0.5\linewidth}{\linethickness}\end{center}

Back to top

\begin{center}\rule{0.5\linewidth}{\linethickness}\end{center}

\pagebreak

\hypertarget{staging-2}{%
\paragraph{Staging}\label{staging-2}}

Gallbladder TNM staging, Margins, Survival

\begin{center}\rule{0.5\linewidth}{\linethickness}\end{center}

 - \textbf{Gallbladder polypoid lesions \textgreater{}15mm as indicators
of T1b gallbladder cancer risk}

\emph{Arab journal of gastroenterology : the official publication of the
Pan-Arab Association of Gastroenterology 2017 Sep;18(3):156-158}

PubMed: \url{https://www.ncbi.nlm.nih.gov/pubmed/?term=28958638}

BACKGROUND AND STUDY AIMS: Gallbladder polyps (GBPs) are found in 5-7\%
of the adult population. However, it is very important to differentiate
between benign and malignant polyps to establish an appropriate
treatment. The present study aimed to determine the relevance of the
10-mm size criterion and attempted to determine the cut-off diameter of
T1b tumours, which requires additional surgical intervention. PATIENTS
AND METHODS: Cases with GBPs were collected between January 2005 and
January 2015. A total of 109 patients were enroled retrospectively.
Information on age, sex, ultrasound findings, and blood laboratory tests
was reviewed. The 10-mm criterion and T1b tumours were examined.
RESULTS: Sixty-nine females and 40 males were included in the study.
Patient age was 45±10.7years (range 27-70years). The 10-mm cut-off
sensitivity and specificity for predicting malignant polyps was 93.6\%
and 85.2\%, respectively. Fifteen patients had malignant pathologic
results, and one patient had GBP \textless{}10mm (intraepithelial, 8mm).
Two patients had intraepithelial tumours of 12 and 13mm. Twelve
malignant patients had T1b tumours with polyp sizes \textgreater{}15mm.
CONCLUSION: Gallbladder cancer may occur in polyps of \textless{}10mm.
Larger size and older age were predictors of neoplastic GBPs. We suggest
15mm as the optimal cut-off point to predict T1b cancer.

{}

{}

\begin{center}\rule{0.5\linewidth}{\linethickness}\end{center}

 - \textbf{Impact of the number of examined lymph nodes on outcomes in
patients with lymph node-negative gallbladder carcinoma}

\emph{World journal of gastroenterology 2018 Jul;24(26):2886-2892}

PubMed: \url{https://www.ncbi.nlm.nih.gov/pubmed/?term=30018483}

AIM: To determine whether the number of examined lymph nodes (LNs) is
correlated with the overall survival of gallbladder carcinoma (GBC)
patients. METHODS: Patients were collected from the Surveillance
Epidemiology and End Results database (2004-2013) and categorized by the
number of LNs into six groups: 1 LN, 2 LNs, 3 LNs, 4 LNs, 5 LNs, and ≥ 6
LNs. Survival curves for overall survival were plotted with a
Kaplan-Meier analysis. The log-rank test was used for univariate
comparisons. RESULTS: In a cohort of 893 patients, the median number of
examined LNs was two for the entire cohort. The survival for the 1 LN
group was significantly poorer than those of the stage I and II disease
groups and for the entire cohort. By dichotomizing the number of LNs
from 1 to 6, we found that the minimum number of LNs that should be
examined was four for stage I, four or five for stage II, and six for
stage IIIA disease. Therefore, for the entire cohort, the number of
examined LNs should be at least six, which is exactly consistent with
the American Joint Committee on Cancer criteria. CONCLUSION: The
examination of higher numbers of LNs is associated with improved
survival after resection surgery for N0 GBC. The guidelines for GBC
surgery, which recommend that six LNs be examined at least, are
statistically valid and should be applied in clinical practice widely.

{}

{}

\begin{center}\rule{0.5\linewidth}{\linethickness}\end{center}

Back to top

\begin{center}\rule{0.5\linewidth}{\linethickness}\end{center}

\pagebreak

\hypertarget{ampulla-of-vater}{%
\subsubsection{Ampulla of Vater}\label{ampulla-of-vater}}

\hypertarget{morphology-diagnostics-ihc-2}{%
\paragraph{Morphology, Diagnostics,
IHC}\label{morphology-diagnostics-ihc-2}}

Morphology, Diagnostics, IHC

\begin{center}\rule{0.5\linewidth}{\linethickness}\end{center}

 - \textbf{Duodenal Epithelial Polyps: A Clinicopathologic Review}

\emph{Archives of pathology \& laboratory medicine 2018 Oct;():}

PubMed: \url{https://www.ncbi.nlm.nih.gov/pubmed/?term=30354274}

CONTEXT.---: Duodenal epithelial polyps are reported in 1.5\% to 3\% of
individuals referred for upper endoscopy. Most duodenal epithelial
polyps are asymptomatic and nonneoplastic; however, a small subset is
neoplastic and may progress to adenocarcinoma. Recent advances in
immunohistochemical and molecular techniques have helped further
characterize these polyps, shedding light on their origin,
classification, and risk of progression to adenocarcinoma.
OBJECTIVE.---: To provide a comprehensive clinicopathologic review of
nonneoplastic and neoplastic duodenal epithelial polyps, with particular
emphasis on recent developments in classification schemes and risk
stratification based upon immunohistochemical and molecular profiles.
DATA SOURCES.---: This review is based on peer-reviewed literature and
the authors' experiences. CONCLUSIONS.---: In this review we provide an
update on the clinicopathologic, immunohistochemical, and molecular
features of duodenal epithelial polyps; and discuss the surveillance
recommendations and treatment options available. Particular attention
should be placed in recognizing duodenal adenomas with intestinal,
gastric, and serrated phenotype, as they have an increased risk of
malignant transformation if not completely excised.

{}

{}

\begin{center}\rule{0.5\linewidth}{\linethickness}\end{center}

 - \textbf{Microanatomical profiles on the lymphatic system in the human
ampulla of Vater (immunohistochemistry and scanning electron
microscopy)}

\emph{Journal of hepato-biliary-pancreatic sciences 2017
Oct;24(10):570-575}

PubMed: \url{https://www.ncbi.nlm.nih.gov/pubmed/?term=28846834}

BACKGROUND: Little information is available regarding microanatomy of
lymphatic system in the ampulla of Vater, though it is of critical
importance for an understanding of tumor progression via the lymphatics
and determination of surgical strategy. The present study, therefore,
aimed to demonstrate the distribution and microanatomical profiles on
the lymphatic system in the ampulla. METHODS: The fine distribution and
structure of the lymphatic vessels were investigated in the ampulla and
the stomach by immunohistochemistry for lymphatic- (D2-40) and blood
vascular- (CD31) specific markers and scanning electron microscopy. The
densities of lymphatic and blood vessels were also compared. RESULTS:
The duodenal papilla densely developed the lymphatics with distinct
aspects of lymphatic capillaries, together with blood vessels. The
density of lymphatic capillaries in the extramuscular layer in the
ampulla was higher than those of both the other ampullary layers and the
gastric extramuscular (subserosal) layer. CONCLUSIONS: The ampulla of
Vater showed widespread lymphatic capillaries throughout the entire
wall. The specific vascular system is suited to produce lymph everywhere
and drain without via such a large vessel as lymphatic collector. This
suggests that tumor cells invade the lymphatics and metastasize more
easily in the ampulla than in the other gastrointestinal regions.

{}

{}

\begin{center}\rule{0.5\linewidth}{\linethickness}\end{center}

 - \textbf{A Rare Case of Ampullary Goblet Cell Carcinoid}

\emph{Internal medicine (Tokyo, Japan) 2018 Sep;57(17):2489-2496}

PubMed: \url{https://www.ncbi.nlm.nih.gov/pubmed/?term=29607953}

An asymptomatic 70-year-old woman was referred to our hospital because
of liver enzyme elevation. Enhanced abdominal computed tomography
demonstrated a small, round-shaped tumor with dilation of the common
bile duct and main pancreatic duct. A biopsy specimen from the papilla
showed mucin-containing cells that were positive for endocrine markers
on immunohistochemical staining. Endoscopic snare resection was done,
and there was a positive vertical margin on pathology.
Pancreaticoduodenectomy was then performed later. The final diagnosis
was goblet cell carcinoid, pT2N0M0, pStage IIA {[}Union for
International Cancer Control (UICC) 7th edition{]}. Ampullary goblet
cell carcinoid is an extremely rare disease of which there have been no
recent reports.

{}

{}

\begin{center}\rule{0.5\linewidth}{\linethickness}\end{center}

 - \textbf{Duodenal tumor risk in Lynch syndrome}

\emph{Digestive and liver disease : official journal of the Italian
Society of Gastroenterology and the Italian Association for the Study of
the Liver 2018 Oct;():}

PubMed: \url{https://www.ncbi.nlm.nih.gov/pubmed/?term=30448460}

BACKGROUND AND AIMS: Lynch syndrome (LS) is associated with an increased
risk of small bowel tumors but routine screening is not recommended in
international guidelines. The aim of our study was to determinate the
prevalence of duodenal tumors in a French cohort of LS patients.
METHODS: Patients carrying a germline pathogenic variant in a MMR gene,
supported by our local network, in which at least one upper endoscopy
had been performed, were included. We registered the occurrence of
duodenal lesions in those patients. RESULTS: 154 LS patients were
identified including respectively 85 MSH2 and 41 MLH1 mutated patients
respectively. Seven out of 154 (4.5\%) had at least one duodenal lesion.
Median age at diagnosis was 58 years (range: 49-73). The twelve lesions
locations were: descending duodenum (n\,=\,7), genu inferius (n\,=\,2),
duodenal bulb (n\,=\,1), ampulla (n\,=\,1), fourth duodenum (n\,=\,1).
Three lesions were invasive adenocarcinomas. The incidence rate of
duodenal lesions in patients with MSH2 or MLH1 pathogenic variants was
respectively 7.1\% (6 out of 85) and 2.4\% (1 out of 41) emphasizing a
trend toward increased risk of developing duodenal lesion in MSH2
mutated patients: OR: 5.17, IC95\% (0.8-60.07), p\,=\,0.1307.
CONCLUSION: Regarding this high prevalence rate, especially in MSH2
patients, regular duodenal screening during upper endoscopy should be
considered in routine in LS patients.

{}

{}

\begin{center}\rule{0.5\linewidth}{\linethickness}\end{center}

Back to top

\begin{center}\rule{0.5\linewidth}{\linethickness}\end{center}

\pagebreak

\hypertarget{staging-3}{%
\paragraph{Staging}\label{staging-3}}

Ampulla of Vater TNM staging, Margins, Survival

\begin{center}\rule{0.5\linewidth}{\linethickness}\end{center}

 - \textbf{Perineural Invasion is a Strong Prognostic Moderator in
Ampulla of Vater Carcinoma: A Meta-analysis}

\emph{Pancreas 2019 Jan;48(1):70-76}

PubMed: \url{https://www.ncbi.nlm.nih.gov/pubmed/?term=30451797}

OBJECTIVE: Ampulla of Vater carcinoma (AVC) has a broad spectrum of
different prognoses. As such, new moderators of survival are urgently
needed. We aimed at clarifying the prognostic role of perineural
invasion in AVC. METHODS: Using PubMed and SCOPUS databases, we
conducted the first systematic review and meta-analysis on this topic.
RESULTS: Analyzing 29 articles for a total of 2379 patients, we found
that the presence of perineural invasion increased the risk of all-cause
mortality more than 2 times (relative risk {[}RR{]}, 2.07; 95\%
confidence interval {[}CI{]}, 1.78-2.42 {[}P \textless{} 0.0001{]};
hazard ratio {[}HR{]}, 2.72; 95\% CI, 1.86-3.97 {[}P \textless{}
0.0001{]}), of cancer-specific mortality more than 6 times (RR, 6.12;
95\% CI, 3.25-11.54 {[}P \textless{} 0.0001{]}; HR, 6.59; 95\% CI,
2.29-3.49 {[}P \textless{} 0.0001{]}), and of recurrence more than 2
times (RR, 2.63; 95\% CI, 1.89-3.67 {[}P \textless{} 0.0001{]}; HR,
2.54; 95\% CI, 1.24-5.21 {[}P = 0.01{]}). CONCLUSIONS: Perineural
invasion is strongly associated with a poorer prognosis in AVC,
influencing both survival and risk of recurrence. It should be reported
in the final pathology report and should be taken into account by future
oncologic staging systems, identifying a group of AVC with a more
malignant biological behavior.

{}

{}

\begin{center}\rule{0.5\linewidth}{\linethickness}\end{center}

 - \textbf{Validation of the eighth edition of the American Joint
Committee on Cancer staging system for ampulla of Vater cancer}

\emph{Surgery 2018 05;163(5):1071-1079}

PubMed: \url{https://www.ncbi.nlm.nih.gov/pubmed/?term=29452703}

BACKGROUND: The American Joint Committee on Cancer recently proposed the
eighth edition of cancer staging system. Validation studies are required
to evaluate the prognostic stratification of ampulla of Vater cancer
patients. METHODS: In the study, 369 operatively resected patients with
ampullary cancers were grouped based on the eighth T (T1a, limited to
sphincter of Oddi; T1b, invasion to duodenal submucosa; T2, invasion to
duodenal proper muscle; T3a, invasion to pancreas ≤0.5 cm; T3b, invasion
to pancreas \textgreater{}0.5 cm; and T4, involvement of celiac axis or
superior mesenteric artery) and N (N0, no nodal metastasis; N1, 1-3
nodal metastasis; and N2, ≥4 nodal metastasis) category of ampullary
cancer staging. RESULTS: Overall 5-year survival rates for T and N
categories were as followed: T1a, 83\%; T1b, 71\%; T2, 46\%; T3a, 48\%;
T3b, 28.5\%, T4, 7\% (P\textless{} .001); N0, 44.8\%; N1, 20\%; N2, 4\%
(P \textless{} .001). Pair-wise comparisons demonstrated significant
differences between T1a-b (P = .005), T3a-T3b (P = .03), N0-N1 (P
\textless{} .001), and N1-N2 (P = .007) tumors, but not between T1b-T2
(P = .20), T2-T3a (P = .84), and T3b-T4 (P = .17) lesions. CONCLUSION:
The eighth edition T category for ampullary cancer does not stratify
patients accurately with regard to prognosis. Modification of the
current T category with eliminating subcategories (T1, invasion to
duodenal submucosa; T2, invasion to duodenal proper muscle; T3, invasion
to pancreas or duodenal subserosa) is a better way for determining
prognosis of ampullary cancer. The current N category segregates patient
survival well.

{}

{}

\begin{center}\rule{0.5\linewidth}{\linethickness}\end{center}

 - \textbf{Pancreaticoduodenectomy for periampullary tumours: a review
article based on Surveillance, End Results and Epidemiology (SEER)
database}

\emph{Clinical \& translational oncology : official publication of the
Federation of Spanish Oncology Societies and of the National Cancer
Institute of Mexico 2018 Sep;20(9):1153-1160}

PubMed: \url{https://www.ncbi.nlm.nih.gov/pubmed/?term=29335829}

INTRODUCTION: This study set to examine relative survival of patients
with periampullary cancers undergoing pancreaticoduodenectomy (PD).
METHODS: Using the Surveillance, End Results and Epidemiology (SEER)
database, this study identified 9877 patients with non-metastatic
pancreatic adenocarcinoma who underwent PD between 2004 and 2013.
RESULTS: Ampullary carcinomas have the best survival among periampullary
malignancies. Lymph node ratio is a significant prognostic factor, even
when stratified by tumour types. Patients receiving adjuvant
radiotherapy following PD have superior survival than patients without
radiotherapy (median 25 vs 20~months, p~\textless{}~0.001), particularly
ductal adenocarcinoma (HR: 0.74, CI95\% 0.69-0.78; p~\textless{}~0.001),
cholangiocarcinoma (HR: 0.75, CI95\% 0.59-0.97; p~=~0.027), and
ampullary carcinoma (HR: 0.79, CI95\% 0.64-0.98; p~=~0.029) with
greatest survival benefit at 1-year postresection. CONCLUSION: Future
studies aiming to further define genetic signatures of individual
periampullary cancers would allow a personalised therapeutic approach in
improving survival.

{}

{}

\begin{center}\rule{0.5\linewidth}{\linethickness}\end{center}

 - \textbf{Definition of an extended minimum level of lymphadenectomy in
non-pancreatic periampullary cancer resections}

\emph{HPB : the official journal of the International Hepato Pancreato
Biliary Association 2018 Nov;20(11):1028-1033}

PubMed: \url{https://www.ncbi.nlm.nih.gov/pubmed/?term=29929786}

BACKGROUND: The number of lymph nodes to be resected in surgery for
non-pancreatic periampullary cancer remains unclear. METHODS: The
Surveillance, Epidemiology, and End Results (SEER) database was used to
gather information from a large retrospective cohort. To define a novel,
reasonable cut-off associated with survival, we stratified patients into
subgroups depending on the number of resected lymph nodes. RESULTS: 1481
nodal-negative patients resected for periampullary cancer (excluding
pancreatic ductal adenocarcinoma) were included. The median number of
resected lymph nodes was ten. Median overall survival in the subgroup
with less than 10 removed lymph nodes was 40 months, while median
survival for patients with ≥10 lymph nodes was 97 months
(p~\textless{}~0.001). A significant survival benefit was seen if~≥~16
lymph nodes were harvested (median survival, 117 months), while no
further benefit was observed if more than 21 nodes were removed (median
survival, \textgreater{}120 months). CONCLUSION: Sixteen or more
resected lymph nodes are associated with improved survival in
node-negative periampullary carcinoma. We propose to aim at harvesting
and analyzing at least 16 lymph nodes.

{}

{}

\begin{center}\rule{0.5\linewidth}{\linethickness}\end{center}

Back to top

\begin{center}\rule{0.5\linewidth}{\linethickness}\end{center}

\pagebreak

\hypertarget{neuroendocrine}{%
\subsubsection{Neuroendocrine}\label{neuroendocrine}}

\hypertarget{pannet}{%
\paragraph{PanNet}\label{pannet}}

PanNET, Pancreatic Neuroendocrine Tumors and related neuroendocrine
neoplasms

\begin{center}\rule{0.5\linewidth}{\linethickness}\end{center}

 - \textbf{Comparison of Monitor-Image and Printout-Image Methods in
Ki-67 Scoring of Gastroenteropancreatic Neuroendocrine Tumors}

\emph{Endocrine pathology 2018 Oct;():}

PubMed: \url{https://www.ncbi.nlm.nih.gov/pubmed/?term=30367334}

Gastroenteropancreatic neuroendocrine tumors (GEP-NET) are classified
according to tumor grade. Ki-67 and mitotic count are the two
determinants of this classification. Therefore, Ki-67 scoring becomes
very important in classifying the patients accurately. Eye-balling,
counting of cells through the microscope, automated image analysis
systems, and manual counting of printed image are the four major scoring
methods in use. The aim of this study is to show the agreement between
monitor-image method (MIM) and printout-image method (PIM) of Ki-67
scoring. In our study, 120 GEP-NETs from 85 patients diagnosed between
January 2005 and July 2017 were evaluated. Thirty-seven cases with
either polypectomy or resection material were selected. Seven different
scoring methods using either a monitor-image or a printout-image were
applied for Ki-67 scoring. They are as follows: whole-PIM, 1/9-PIM,
whole-MIM, 1/4-MIM, 1/6-MIM, 1/9-MIM, and 1/12-MIM. In the comparison of
Ki-67 scoring methods, intraclass correlation coefficients ranging from
0.951 to 0.999 were found. The Bland-Altman analysis showed near-perfect
agreement between whole-MIM and whole-PIM as well as 1/9-MIM and
1/9-PIM. The level of agreements among the other methods were sufficient
too, but there was a relative decrease in the level of agreement as the
area of counting becomes smaller. The average application time decreased
from 373.7 to 41.7~s gradually as the scoring area becomes smaller. Our
study shows that there is a remarkable agreement between the MIM and PIM
used in Ki-67 scoring.

{}

{}

\begin{center}\rule{0.5\linewidth}{\linethickness}\end{center}

 - \textbf{Natural History of Small Pancreatic Lesions Suspected to Be
Nonfunctioning Pancreatic Neuroendocrine Tumors}

\emph{Pancreas 2018 10;47(10):1357-1363}

PubMed: \url{https://www.ncbi.nlm.nih.gov/pubmed/?term=30308537}

OBJECTIVES: Nonfunctioning pancreatic neuroendocrine tumors (NF-PNETs)
are a rare disease but have been diagnosed more frequently than before.
The aim of this study was to evaluate the natural course of small
NF-PNETs. METHODS: We performed a retrospective analysis of patients
with incidentally found small NF-PNETs (\textless{}20 mm) from 1999 to
2015. The patients who were recommended surveillance were included.
RESULTS: There were 69 patients with small NF-PNETs with a mean size of
10.9 (standard deviation {[}SD{]}, 3.1) mm. The average follow-up period
was 52.2 (SD, 38.7) months. The changes in tumor size were as follows:
increased (13.0\%), sustained (84.1\%), and decreased (2.9\%). Eighteen
were evaluated with grade 1 NF-PNETs and 1 with grade 2 among the
obtained tissues. Thirteen patients underwent surgery after an average
32.9 (SD, 42.6) months later. There were 7 patients of Ia, 1 of Ib, 2 of
IIa, and 1 of IIb according to the pathologic stages. Two patients
received reoperation for recurrent tumors, and 2 patients showed distant
metastasis after surgery, but no disease-related death occurred.
CONCLUSIONS: Most of the small NF-PNETs did not increase in size and
seldom showed metastasis. The wait-and-see strategy can be used for
NF-PNETs less than 2 cm.

{}

{}

\begin{center}\rule{0.5\linewidth}{\linethickness}\end{center}

 - \textbf{Immunohistochemical Biomarkers of Gastrointestinal,
Pancreatic, Pulmonary, and Thymic Neuroendocrine Neoplasms}

\emph{Endocrine pathology 2018 Jun;29(2):150-168}

PubMed: \url{https://www.ncbi.nlm.nih.gov/pubmed/?term=29520563}

Neuroendocrine neoplasms (NENs) are a heterogeneous group of epithelial
neoplastic proliferations that irrespective of their primary site share
features of neural and endocrine differentiation including the presence
of secretory granules, synaptic-like vesicles, and the ability to
produce amine and/or peptide hormones. NENs encompass a wide spectrum of
neoplasms ranging from well-differentiated indolent tumors to highly
aggressive poorly differentiated neuroendocrine carcinomas. Most cases
arise in the digestive system and in thoracic organs, i.e., the lung and
thymus. A correct diagnostic approach is crucial for the management of
patients with both digestive and thoracic NENs, because their high
clinical and biological heterogeneity is related to their prognosis and
response to therapy. In this context, immunohistochemistry represents an
indispensable diagnostic tool that pathologists need to use for the
correct diagnosis and classification of such neoplasms. In addition,
immunohistochemistry is also useful in identifying prognostic and
theranostic markers. In the present article, the authors will review the
role of immunohistochemistry in the routine workup of digestive and
thoracic NENs.

{}

{}

\begin{center}\rule{0.5\linewidth}{\linethickness}\end{center}

 - \textbf{COMPETITIVE TESTING THE WHO 2010 VS THE WHO 2017 GRADING OF
PANCREAS NEUROENDOCRINE NEOPLASIA: DATA FROM A LARGE INTERNATIONAL
COHORT STUDY}

\emph{Neuroendocrinology 2018 Oct;():}

PubMed: \url{https://www.ncbi.nlm.nih.gov/pubmed/?term=30300897}

Background: the World Health Organization (WHO) and the American Joint
Cancer Committee (AJCC) modified the grading of pancreatic
neuroendocrine neoplasms from a three-tiers (WHO-AJCC 2010) to a
four-tiers system by introducing the novel category of NET G3 (WHO-AJCC
2017). This study aims at validating the WHO-AJCC 2017 and identifying
the most effective grading system. 2102 patients were enrolled; entry
criteria were i) performed surgery; ii) at least two years of follow-up;
iii) observation time up to 2015. Data from 34 variables were collected;
grading was assessed and compared for efficacy by statistical means
including Kaplan Meier method, Cox regression analysis, Harrell's C
statistics and Royston's explained variation in univariable and
multivariable analyses. At descriptive analysis, the two grading systems
demonstrated statistically significant differences for the major
category sex but not for age groups. At Cox regression analysis, both
grading systems showed statistically significant differences between
grades for OS and EFS, however no statistically significant difference
was observed between the two G3 classes of WHO-AJCC 2017. At
multivariable analysis for the two models fitted to compare efficacy,
the two grading systems performed equally well with substantially
similar optimal discrimination and well-explained variation for both OS
and EFS. The WHO-AJCC 2017 grading system retained statistically
significant difference between the two G3 classes for OS but not for
EFS. the WHO-AJCC 2017 grading is at least equally performing as the
WHO-AJCC 2010 but allows the successful identification of the most
aggressive PanNET subgroup. Grading is confirmed as probably the most
powerful tool for patient survival prediction. .

{}

{}

\begin{center}\rule{0.5\linewidth}{\linethickness}\end{center}

 - \textbf{Ki-67 and presence of liver metastases identify different
progression-risk classes in pancreatic neuroendocrine neoplasms (pNEN)
undergoing resection}

\emph{European journal of surgical oncology : the journal of the
European Society of Surgical Oncology and the British Association of
Surgical Oncology 2018 Oct;():}

PubMed: \url{https://www.ncbi.nlm.nih.gov/pubmed/?term=30366875}

In pancreatic neuroendocrine neoplasms (pNEN), size ≤2\,cm and
Ki-67\,\textless{}\,3\% suggest indolent behavior, but no factor alone
predicts prognosis. We investigated factors predictive of tumor
progression in 80 pNENs surgically resected in a single Institution from
1995 to 2015. At multivariable analysis the only two independent
variables related to PFS were Ki-67 (HR 2.97; 95\%CI 1.26-7.02) and
presence of synchronous liver metastases (HR 3.60; 95\%CI 1.70-7.61).
Using Ki-67\,\textless{}\,3\% and M0 as reference, the HR for tumor
progression was 3.21 (95\%CI 1.18-8.74) for M0 patients with Ki-67
3-20\%, 5.06 (2.29-11.2) for M1 patients with Ki-67\,≤\,20\% and 24.3
(6.64-89.2) for those with Ki-67\,\textgreater{}\,20\%. Tumor size (≤2
vs. \textgreater{}2\,cm) was not a predictive factor at any analysis.
Intra-class correlation of Ki-67 values on pre-surgical biopsies
vs.~surgical specimens was 0.99 and Ki-67 classes were correctly
identified in 97\% of biopsies. Ki-67 and presence of liver metastases
are the major prognostic factors in pNEN and identify different
progression risks regardless of tumor size. Pre-surgical pNEN biopsy for
Ki-67 assessment should be included in the evaluation of patients with
1-2\,cm tumors to help in the decision on whether to perform surgical
resection.

{}

{}

\begin{center}\rule{0.5\linewidth}{\linethickness}\end{center}

 - \textbf{Limited role of Chromogranin A as clinical biomarker for
pancreatic neuroendocrine tumors}

\emph{HPB : the official journal of the International Hepato Pancreato
Biliary Association 2018 Oct;():}

PubMed: \url{https://www.ncbi.nlm.nih.gov/pubmed/?term=30366884}

BACKGROUND: Serum Chromogranin A (CgA) is widely used as a biomarker for
pancreatic neuroendocrine tumors (PanNETs). The aim of this study was to
investigate the value of CgA as a diagnostic and prognostic marker for
well-differentiated PanNETs. METHODS: Patients with well-differentiated
PanNET and a baseline CgA measurement, between 2011 and 2016 were
reviewed. The diagnostic value was determined by comparing CgA values
from patients with PanNETs to those with other pancreatic neoplasms and
healthy controls. The Kaplan-Meier method was used to investigate the
CgA prognostic significance. RESULTS: Ninety-nine patients met inclusion
criteria. As a diagnostic marker, CgA had a sensitivity of 66\%,
specificity of 95\%, and overall accuracy of 71\%. The use of PPIs was
associated with a higher CgA level (p~=~0.015). When excluding patients
on PPIs, CgA accuracy in distinguishing PanNETs from other pancreatic
neoplasms was 66\%, the sensitivity and specificity were 60\% and 75\%
respectively. Elevated CgA (p~=~0.004), Ki67\% (p~\textless{}~0.001),
tumor grade (p~\textless{}~0.001) and stage of disease (p~=~0.036) were
associated with disease-specific survival. CONCLUSION: CgA has a limited
role as a diagnostic biomarker for well-differentiated PanNETs. An
elevated CgA level may have prognostic value but its role should be
further investigated with respect to other known pathological factors.

{}

{}

\begin{center}\rule{0.5\linewidth}{\linethickness}\end{center}

 - \textbf{High Minichromosome maintenance protein 7 proliferation
indices: a powerful predictor of progression in pancreatic
neuroendocrine neoplasms without distant metastasis at the time of
surgery}

\emph{Human pathology 2018 Nov;():}

PubMed: \url{https://www.ncbi.nlm.nih.gov/pubmed/?term=30447299}

Pancreatic neuroendocrine neoplasms (PanNENs) have an unpredictable
clinical course that varies from indolent to highly malignant. No
immunohistochemical markers are available for reliable prediction of the
biological behavior of early-stage PanNENs. Minichromosome maintenance
protein 7 (MCM7) is a putative powerful marker of cell proliferation.
Whether the expression of MCM7 is related to the risk of PanNENs
progression remains unclear. We assessed the clinical behavior of 156
PanNENs with respect to stage, grade, Ki-67 index, MCM7 index, and other
pathologic features. A high MCM7 index was significantly associated with
larger tumor size (P\textless{}.001), nonfunctioning tumor
(P\textless{}.001), increased grade (P\textless{}.0001), and later TNM
stage (P\textless{}.001). In multivariate analysis, G2/G3 (hazard ratio
(HR), 2.21; 95\% confidence interval (CI), 1.35-3.62; P\textless{}.001),
stage III/IV (HR, 2.11; 95\% CI, 1.31-3.41; P\textless{}.001), and MCM7
labeling index (LI)\textgreater{}5\% (HR, 3.81; 95\% CI, 1.30-11.17;
P=.02) were independent negative prognostic factors related to the risk
of tumor progression in stage I-IV disease. MCM7 LI\textgreater{}5\% was
associated with an increased risk of progression in stages I-V, I-III,
and I-II. Our study confirms that MCM7 is a valuable marker for
assessing the progression of PanNENs, especially in patients with
early-stage disease and without distant metastasis.

{}

{}

\begin{center}\rule{0.5\linewidth}{\linethickness}\end{center}

 - \textbf{Endoscopic Resection of Duodenal Carcinoid Tumors: A
Single-Center Comparison Between Simple Polypectomy and Endoscopic
Mucosal Resection}

\emph{Pancreas 2019 Jan;48(1):60-65}

PubMed: \url{https://www.ncbi.nlm.nih.gov/pubmed/?term=30451799}

OBJECTIVES: Endoscopic resection is preferred for duodenal carcinoids
less than 20 mm; however, the efficacy of simple polypectomy has not
been compared with advanced endoscopic resection techniques. METHODS: We
performed a retrospective review of 33 patients who underwent endoscopic
duodenal carcinoid resection (10 simple, 23 endoscopic mucosal
resection) at the Hospital of the University of Pennsylvania between
January 1, 2006, and June 15, 2017. The primary outcomes were resection
margin positivity and local tumor recurrence. RESULTS: There were no
significant differences in demographics or tumor functionality. Lesions
managed with simple polypectomy had smaller median gross specimen size
(6.0 mm vs 8.0 mm, P = 0.043). There was no significant difference in
pathology resection margins between simple polypectomy and endoscopic
mucosal resection (86\% vs 68\% positive, P = 0.64). Local recurrence on
surveillance endoscopy was also similar (14.3\% vs 17.7\%, respectively;
P = 1.000), with median time to recurrence 2.3 months (interquartile
range, 1.2-5.4 months). The median follow-up time in patients without
local recurrence was 21.4 months (interquartile range, 7.1-39.6 months).
CONCLUSIONS: Simple polypectomy may be adequate treatment of small
duodenal carcinoids, although further studies are needed for validation
and to define the upper limits of tumor size that can be managed with
this technique.

{}

{}

\begin{center}\rule{0.5\linewidth}{\linethickness}\end{center}

 - \textbf{Proinsulin Expressing Neuroendocrine Tumors of the Pancreas:
An Underrecognized Entity}

\emph{Pancreas 2019 Jan;48(1):55-59}

PubMed: \url{https://www.ncbi.nlm.nih.gov/pubmed/?term=30451800}

OBJECTIVES: Rare cases of pancreatic neuroendocrine tumors (PNETs) that
produce only proinsulin (PI) and manifest with hypoglycemia have been
reported. Proinsulin expression in PNET has not been systematically
studied, and the clinicopathologic features of such tumors remain
unknown. METHODS: We studied expression of PI by immunohistochemistry
(IHC) in 136 PNETs from 2 high-volume surgical oncology centers and
assessed all available clinicopathologic data. RESULTS: Thirty-six
(26\%) of PNETs were positive for PI by IHC, most (89\%) of which
coexpressed insulin IHC. Nine PI-positive tumors represented functional
insulinomas. Patients with PI IHC-positive tumors demonstrated
significantly lower mean preoperative serum glucose compared with
PI-negative PNET patients, even when insulinomas were excluded. No
differences in survival between PI IHC-positive and PI IHC-negative
tumors were observed. We identified 2 PI-positive PNETs from
hypoglycemic patients, which were not insulinomas or other functional
variants and in which serum PI was never tested. These may have been
undetected proinsulinomas. CONCLUSIONS: Proinsulin-expressing PNETs
(functional or non) are not uncommon. Patients who present with
hypoglycemia and normal insulin levels should be screened for
proinsulinoma. Proinsulin IHC could also be used to screen for
proinsulinoma. To further elucidate the clinical significance of PI
expressing PNETs, prospective studies are required.

{}

{}

\begin{center}\rule{0.5\linewidth}{\linethickness}\end{center}

 - \textbf{Neuroendocrine Liver Metastasis-a Specific Set of Markers to
Detect Primary Tumor Sites}

\emph{Endocrine pathology 2018 Nov;():}

PubMed: \url{https://www.ncbi.nlm.nih.gov/pubmed/?term=30456697}

The diagnosis of neuroendocrine neoplasia (NEN) is often made at an
advanced stage of disease, including hepatic metastasis. At this point,
the primary may still be unknown and sometimes cannot even be detected
by functional imaging, especially in very small tumors of the pancreas
(pan) and small intestinal (si) entities. The site of the primary may be
based on biopsy specimens of the liver applying a specific set of
markers. Specimens of liver metastases from 87 patients with NENs were
studied. In retrospect, 50 patients had si and 37 pan NENs. Tissue
samples were evaluated by immunohistochemistry. The markers applied were
insulin gene enhancer protein Islet-1 (ISL-1), homeobox protein CDX-2
(CDX2), thyroid transcription factor 1 (TTF-1), and serotonin. Positive
stains for CDX2 were documented in 43 (86\%) and for serotonin in 45
(90\%) of 50 siNENs. Three panNENs were positive for CDX2 and one for
serotonin, respectively. ISL-1 was negative throughout in siNENs and
also negative in 8 of 50 panNENs (21.6\%). TTF-1 was negative in more
than 90\% of the specimens of either entity. Immunohistochemical markers
in liver metastasis can lead the way to the site of the primary NEN.
They should always be used in combined clusters.

{}

{}

\begin{center}\rule{0.5\linewidth}{\linethickness}\end{center}

 - \textbf{Association between preoperative Vasostatin-1 and
pathological features of aggressiveness in localized nonfunctioning
pancreatic neuroendocrine tumors (NF-PanNET)}

\emph{Pancreatology : official journal of the International Association
of Pancreatology (IAP) \ldots{} {[}et al.{]} 2018 Nov;():}

PubMed: \url{https://www.ncbi.nlm.nih.gov/pubmed/?term=30470614}

BACKGROUND: A reliable and accessible biomarker for nonfunctioning
pancreatic neuroendocrine tumors (NF-PanNET) is currently unavailable.
Chromogranin A (CgA) represents the best-described neuroendocrine
biomarker, but its accuracy is low. Vasostatin-1 (VS-1), a fragment
derived from the cleavage of CgA, was recently investigated and found to
be more accurate as tumor biomarker in a cohort of patients affected by
mainly metastatic small intestinal NET. METHODS: Patients submitted to
surgery for sporadic localized NF-PanNET at San Raffaele Hospital were
included. Preoperative plasma samples were prospectively collected.
Circulating levels of total-CgA and VS-1 were retrospectively
investigated by sandwich Enzyme-Linked ImmunoSorbent Assays. RESULTS:
Overall, 50 patients were included. VS-1 value (P=0.0001) was the only
preoperatively retrievable factor independently associated with
NF-PanNET size. No significant correlation between CgA and tumor
diameter was found (P\,=\,0.057). A VS-1 value of 0.39\,nM was
identified as the optimal VS-1 cut-off accurately associated with
NF-PanNET larger than 4\,cm. Patients with VS-1 \textgreater{} 0.39\,nM
had a significantly higher frequency of microvascular invasion
(P\,=\,0.005) and nodal metastasis (P\,=\,0.027). Median VS-1 plasma
level was significantly higher in the presence of microvascular invasion
(P\,=\,0.001) and nodal metastasis (P\,=\,0.012). PPI assumption
significantly increased total-CgA levels, but not those of VS-1
(P\,=\,0.111). CONCLUSIONS: In localized, non-metastatic NF-PanNET, VS-1
is strongly associated to tumor dimension and its plasma levels are
significantly higher in the presence of microvascular invasion and nodal
metastases; moreover, VS-1 value is not affected by the PPI use.

{}

{}

\begin{center}\rule{0.5\linewidth}{\linethickness}\end{center}

 - \textbf{Unmet Needs in Functional and Nonfunctional pancreatic
neuroendocrine neoplasms(PanNENs)}

\emph{Neuroendocrinology 2018 Oct;():}

PubMed: \url{https://www.ncbi.nlm.nih.gov/pubmed/?term=30282083}

Recently, European Neuroendocrine Tumor Society (ENETS )held working
sessions composed of members of the advisory board and other
Neuroendocrine neoplasms (NEN) experts to attempt to identify unmet
needs in NENs in different locations or with advanced/poorly
differentiated NENs. This section briefly summarizes the main proposed
areas of unmet needs in patients with functional and non-functional
pancreatic neuroendocrine neoplasms (PanNENs).

{}

{}

\begin{center}\rule{0.5\linewidth}{\linethickness}\end{center}

Back to top

\begin{center}\rule{0.5\linewidth}{\linethickness}\end{center}

\pagebreak

\hypertarget{cytopathology}{%
\subsection{Cytopathology}\label{cytopathology}}

\begin{center}\rule{0.5\linewidth}{\linethickness}\end{center}

\hypertarget{pancreas-1}{%
\subsubsection{Pancreas}\label{pancreas-1}}

\begin{center}\rule{0.5\linewidth}{\linethickness}\end{center}

 - \textbf{Imaging and Cytopathological Criteria Indicating Malignancy
in Mucin-Producing Pancreatic Neoplasms: A Series of 68
Histopathologically Confirmed Cases}

\emph{Pancreas 2018 10;47(10):1283-1289}

PubMed: \url{https://www.ncbi.nlm.nih.gov/pubmed/?term=30308535}

OBJECTIVES: This study aims to evaluate the performance of clinical,
imaging, and cytopathological criteria in the identification of
high-grade dysplasia/carcinoma (HGD/Ca) in pancreatic mucin-producing
cystic neoplasms. METHODS: Sixty-eight consecutive, histopathologically
confirmed mucin-producing cystic neoplasms, evaluated by endoscopic
ultrasound-guided fine-needle aspiration, were enrolled; specifically,
39 branch duct intraductal papillary mucinous neoplasms (BD-IPMNs), 21
main duct IPMNs, and 8 mucinous cystic neoplasms. The associations
between HGD/Ca in histopathology and findings of endoscopic ultrasound
and cytology, demographic, lifestyle, and clinical parameters were
evaluated, separately in IPMNs and mucinous cystic neoplasms. RESULTS:
Age 65 years or more was associated with HGD/Ca in IPMNs. In BD-IPMNs,
cyst diameter 3 cm or greater (sensitivity, 68.8\%; specificity,
65.2\%), a mural nodule (sensitivity, 56.3\%; specificity, 78.3\%), main
pancreatic duct diameter 5 to 9 mm (sensitivity, 50.0\%; specificity,
87.0\%), and suspicious cytology (sensitivity, 81.3\%; specificity,
100\%) signaled the presence of HGD/Ca. Similarly, in main duct IPMNs,
suspicious cytology predicted HGD/Ca with high sensitivity (88.9\%) and
excellent specificity (100\%). Regarding cytopathological criteria, in
BD-IPMNs, HGD/Ca was associated with a high nuclear/cytoplasmic ratio,
background necrosis, presence of papillary structures, hypochromatic
nuclei, hyperchromatic nuclei, and major nuclear membrane irregularities
(thickening and/or indentations). CONCLUSIONS: Clinical, imaging, and
cytopathological criteria are useful in the identification of HGD/Ca in
IPMNs.

{}

{}

\begin{center}\rule{0.5\linewidth}{\linethickness}\end{center}

 - \textbf{Reclassification of lesions in biopsies by fine-needle
aspiration of pancreas and biliary tree using Papanicolaou
classification}

\emph{Journal of gastrointestinal oncology 2018 Oct;9(5):847-852}

PubMed: \url{https://www.ncbi.nlm.nih.gov/pubmed/?term=30505584}

Background: Our aim was to evaluate the application of the
classification of the Papanicolaou Cytopathology Society for the report
of biopsies by fine-needle aspiration (FNA) of pancreas and bile duct.
Methods: The FNAs obtained consecutively during 1 year were analyzed.
Descriptive statistics were performed and sensitivity, specificity,
positive predictive value, negative predictive value, and
cytohistological correlation were determined. The reference standard
test was the histopathological study. Results: A total of 134 cases of
FNA were reclassified with ultrasound guidance according to the
classification of the Papanicolaou Society, the median age was 59 years
(range, 25-80 years). A case interpreted as non-diagnostic was
reclassified to category 4 and 3 cases with atypical cells were
reclassified to category 5. All malignant cases remained unchanged.
Surgical follow-up was performed in 35 patients (26.1\%), with a
cytohistological concordance in 21 cases (91.3\%) and 2 discordant cases
(8.7\%), the reasons for the discrepancy were due to sampling error, one
of them with scarce material to make a diagnosis of higher category, the
other case with partial agreement, because cytology was observed
atypical cylindrical epithelium, with histology of grade 2
neuroendocrine neoplasia and low grade mucinous intraepithelial
neoplasia. In general, the sensitivity and specificity were 100\% and
75\% respectively, the positive predictive value 88\% and the negative
predictive value 100\%. Conclusions: The FNA guided with endoscopic
ultrasound (EUS) and interpreted according to the Papanicolaou
Cytopathology Society Classification is an accurate method to evaluate
pancreatic and biliary tract lesions with a high positive predictive
value of 88\%.

{}

{}

\begin{center}\rule{0.5\linewidth}{\linethickness}\end{center}

 - \textbf{Abnormal immunolabelling of SMAD4 in cell block specimens to
distinguish malignant and benign pancreatic cells}

\emph{Cytopathology : official journal of the British Society for
Clinical Cytology 2018 Nov;():}

PubMed: \url{https://www.ncbi.nlm.nih.gov/pubmed/?term=30421464}

BACKGROUND: Accurate diagnosis of malignant and benign pancreatic
lesions can be challenging, especially with endoscopic ultrasound-guided
fine needle aspiration (EUS-FNA) samples that are small and/or degraded.
In the present study, we determined how to best evaluate abnormal SMAD4
expression by immunohistochemical staining on cell block specimens from
EUS-FNA samples. RESULTS: In surgically resected pancreas, when abnormal
SMAD4 immunolabelling was evaluated as negative SMAD4 expression, the
sensitivity was low (33\%), but when it was evaluated as decreased SMAD4
expression, the sensitivity improved (53\%). Specificity and positive
predictive value were high for both evaluations. There were no
false-positive cases. In cell block specimens, decreased SMAD4
expression showed 47\% sensitivity and 72\% specificity, while negative
SMAD4 expression showed lower sensitivity (20\%) and higher specificity
(100\%). Both evaluations in cell block specimens showed lower
sensitivity and specificity compared to resected specimens.
False-positive and -negative rates were higher for cell blocks than for
resected specimens. CONCLUSIONS: Decreased SMAD4 immunolabelling
provided improved sensitivity as compared to negative SMAD4
immunolabelling; therefore, it is important to compare SMAD4 expression
in a sample to its expression in normal cells. Abnormal SMAD4 labelling
showed low sensitivity and high specificity; therefore, SMAD4 staining
using EUS-FNA samples might be helpful to detect malignancies that
possess SMAD4 gene abnormalities.

{}

{}

\begin{center}\rule{0.5\linewidth}{\linethickness}\end{center}

 - \textbf{Cytomorphology of ciliated foregut cyst of the pancreas}

\emph{Diagnostic cytopathology 2018 Nov;():}

PubMed: \url{https://www.ncbi.nlm.nih.gov/pubmed/?term=30478999}

Ciliated foregut cysts are benign congenital lesions that are commonly
found in the mediastinum but are rare in the retroperitoneum. So far
only very few cases of ciliated foregut cyst found in the pancreas have
been reported, and less with cytologic findings described. We report a
case of ciliated foregut cyst in pancreas in an asymptomatic patient
diagnosed on fine needle aspiration cytology. We also discuss the
cytology features that would help with the diagnosis, and the
differential diagnosis that should be considered.

{}

{}

\begin{center}\rule{0.5\linewidth}{\linethickness}\end{center}

 - \textbf{Rationale and feasibility of mucin expression profiling by
qRT-PCR as diagnostic biomarkers in cytology specimens of pancreatic
cancer}

\emph{Pancreatology : official journal of the International Association
of Pancreatology (IAP) \ldots{} {[}et al.{]} 2018 Dec;18(8):977-982}

PubMed: \url{https://www.ncbi.nlm.nih.gov/pubmed/?term=30268674}

BACKGROUND: Aberrantly expressed mucin glycoproteins (MUC) play
important roles in pancreatic ductal adenocarcinoma (PDAC), yet their
use as a diagnostic aid in fine-needle aspiration biopsy (FNAB) is
poorly documented. The aim of this study was to investigate the
rationale and feasibility of mucin (MUC1, MUC2, MUC3, MUC4, MUC5AC, and
MUC6) expression profiling by RT-PCR for diagnostic applications in
cytology. METHODS: Mucin expression was examined by RT-PCR and
immunohistochemistry in specimens resected from patients with pancreatic
(n\,=\,101), ampullary (n\,=\,23), and common bile duct (n\,=\,10)
cancers and 33 with chronic pancreatitis. Furthermore, mucin profiling
by RT-PCR was prospectively compared in surgical and biopsy specimens of
40 patients with pancreatic solid tumours qualified for FNAB prior to
surgery. RESULTS: A logistic regression model to distinguish PDAC from
chronic pancreatitis using RT-PCR profiling included MUC3, MUC5AC, and
MUC6. The same set of mucins differentiated ampullary and bile duct
cancers from chronic pancreatitis. AUCs for the ROC curves derived from
the two models were 0.95 (95\%CI 0.87-0.99) and 0.92 (95\%CI 0.81-0.98),
respectively. The corresponding positive likelihood ratios were 6.02 and
5.97, while the negative likelihood ratios were 0.10 and 0.12. AUCs of
ROC curves obtained by RT-PCR and immunohistochemistry demonstrated that
both analytical methods were comparable. Surgical and cytological
samples showed significantly correlated values of ΔCt for individual
mucins with the overall Pearson's correlation coefficient r\,=\,0.841
(P\,=\,0.001). CONCLUSIONS: Mucin expression profiling of pancreatic
cancer with RT-PCR is feasible and may be a valuable help in
discriminating malignant lesions from chronic pancreatitis in FNAB
cytology.

{}

{}

\begin{center}\rule{0.5\linewidth}{\linethickness}\end{center}

 - \textbf{Insulinoma-associated protein 1 expression in pancreatic
neuroendocrine tumours in endoscopic ultrasound-guided fine-needle
aspiration cytology: An analysis of 14 patients}

\emph{Cytopathology : official journal of the British Society for
Clinical Cytology 2018 Oct;():}

PubMed: \url{https://www.ncbi.nlm.nih.gov/pubmed/?term=30290028}

BACKGROUND: Insulinoma-associated protein 1 (INSM1) has been reported to
be a useful marker for diagnosing pancreatic neuroendocrine tumours
(PNETs). However, INSM1 expression in endoscopic ultrasound-guided fine
needle aspiration cytology has not been examined. We evaluated INSM1
expression in the cytology of cases diagnosed with PNETs. METHODS: We
immunocytochemically stained INSM1 and Ki-67 in 14 PNET cases, and
according to the 2017 World Health Organisation criteria, seven PNET
Grade 1 cases, four Grade 2 cases and three Grade 3/neuroendocrine
carcinoma cases were identified. As a control for INSM1 and Ki-67
expression, we used cytological specimens from 15 cases of pancreatic
ductal adenocarcinoma. RESULTS: In PNET cases, INSM1-expressing tumour
cells were identified in all cytological specimens of PNET. The median
INSM1 expression rate in Grade 1 cases was 49.8\% (mean~±~standard
deviation: 55.1~±~12.5\%, min: 39.3\%, max: 74.1\%), and in Grade 2 and
Grade 3/neuroendocrine carcinoma cases was 81.1\% (mean~±~standard
deviation: 77.6~±~18.6\%, min: 50.3\%, max: 100\%). However, there was
no correlation between INSM1 and Ki-67 expression (r~=~-0.15). The
median expression rate in PNET cases was 64.3\%, which was significantly
higher than that in pancreatic ductal adenocarcinoma cases
(P~\textless{}~0.0001). CONCLUSION: INSM1 immunocytochemistry of
cytological specimens obtained from endoscopic ultrasound-guided fine
needle aspiration cytology can accurately diagnose PNETs; therefore,
INSM1 could be an important diagnostic tool in assessing therapeutic
strategies, including molecular-targeted therapy.

{}

{}

\begin{center}\rule{0.5\linewidth}{\linethickness}\end{center}

 - \textbf{Endoscopic ultrasound-guided fine needle aspiration with
liquid-based cytology preparation in the diagnosis of metastatic
small-cell carcinoma in the pancreas}

\emph{Diagnostic cytopathology 2018 Nov;46(11):977-980}

PubMed: \url{https://www.ncbi.nlm.nih.gov/pubmed/?term=30353700}

Pancreatic metastasis is extremely rare, particularly from small-cell
lung cancer (SCLC). Studies on the role of endoscopic ultrasound-guided
fine needle aspiration (EUS-FNA) with liquid-based cytology (LBC) in the
diagnosis of metastatic small-cell carcinoma in the pancreas have been
rarely conducted. We report herein a case of pancreatic metastasis from
SCLC diagnosed using EUS-FNA with LBC (ThinPrep). A 71-year-old man
presented with chief complaints of hemoptysis and jaundice over the past
1 month. Lung \& pancreas tumors with multiple liver nodules were
detected on computed tomography. The aspirated material from the
pancreas using EUS-FNA was prepared as a cytologic specimen with
ThinPrep method, which revealed scattered and clustered ``small blue
cells'' with scant cytoplasm and stippled chromatin with frequent
apoptotic bodies. Immunocytochemical staining of the cellblock material
revealed strong positivity for CD56 and thyroid transcription factor-1.
Endobronchial biopsy for lung mass revealed nests of small, round, blue
tumor cells with hyperchromatic nuclei showing salt and pepper
chromatin, scant cytoplasm, and brisk mitotic activity. Therefore, a
diagnosis of metastatic small-cell carcinoma to the pancreas with an
extensive stage was finally made.

{}

{}

\begin{center}\rule{0.5\linewidth}{\linethickness}\end{center}

 - \textbf{Comparison of Native Aspirates and Cytological Smears
Obtained by EUS-Guided Biopsies for Effective DNA/RNA Marker Testing in
Pancreatic Cancer}

\emph{Pathology oncology research : POR 2018 Oct;():}

PubMed: \url{https://www.ncbi.nlm.nih.gov/pubmed/?term=30361898}

We compare two types of pancreatic carcinoma samples obtained by
EUS-guided fine needle biopsy (EUS-FNB) in terms of the success rates
and clinical validity of analysis of two most commonly investigated
DNA/RNA pancreatic cancer markers, KRAS mutations and miR-21 expression.
118 patients with pancreatic ductal adenocarcinoma underwent EUS-FNB.
The collected sample was divided, one part was stored in a stabilizing
solution as native aspirate (EUS-FNA) and second part was processed into
the cytological smear (EUS-FNC). DNA/RNA extraction was followed by
analysis of KRAS mutations and miR-21 expression. For both sample types,
the yields of DNA/RNA extraction and success rates of KRAS mutation and
miRNA expression were evaluated. Finally, the resulting KRAS mutation
frequency and miR-21 prognostic role were compared to literature data
from tissue resections. The overall amount of isolated DNA/RNA from
EUS-FNC was lower compared to the EUS-FNA, average yield 10~ng vs 147~ng
for DNA and average yield 164 vs.~642~ng for RNA, but the success rates
for KRAS and miR-21 analysis was 100\% for both sample types. The
KRAS-mutant detection frequency in EUS-FNC was 12\% higher than in
EUS-FNA (90 vs 78\%). The prognostic role of miR-21 was confirmed in
EUS-FNC (p~= 0.02), but did not reach statistical significance in
EUS-FNA (p~= 0.06). Although both types of EUS-FNB samples are suitable
for DNA/RNA extraction and subsequent DNA mutation and miRNA expression
analysis, reliable results with clinical validity were only obtained for
EUS-FNC.

{}

{}

\begin{center}\rule{0.5\linewidth}{\linethickness}\end{center}

 - \textbf{Needle-based confocal laser endomicroscopy of pancreatic
cystic lesions: a prospective multicenter validation study in patients
with definite diagnosis}

\emph{Endoscopy 2018 Oct;():}

PubMed: \url{https://www.ncbi.nlm.nih.gov/pubmed/?term=30347425}

BACKGROUND:  Needle-based confocal laser endomicroscopy (nCLE) enables
observation of the inner wall of pancreatic cystic lesions (PCLs) during
an endoscopic ultrasound-guided fine-needle aspiration (EUS-FNA). This
study prospectively evaluated the diagnostic performance of nCLE for
large, single, noncommunicating PCLs using surgical histopathology or
EUS-FNA cytohistopathology as a reference diagnosis. METHODS:  From
April 2013 to March 2016, consecutive patients referred for EUS-FNA of
indeterminate PCLs without evidence of malignancy or chronic
pancreatitis were prospectively enrolled at five centers. EUS-FNA and
nCLE were performed and cystic fluid was aspirated for
cytohistopathological and carcinoembryonic antigen (CEA) analysis. The
diagnostic performance of nCLE was assessed against the reference
standard and compared with that of EUS and CEA. This study was
registered on ClinicalTrials.gov (NCT01563133). RESULTS:  206 patients
underwent nCLE and 78 PCLs (mean size 40 mm, range 20 - 110 mm) had
reference diagnoses (53 premalignant and 25 benign PCLs). Post-procedure
pancreatitis occurred in 1.3 \% of the patients. nCLE was conclusive in
71 of the 78 cases (91 \%). The sensitivies and specifities of nCLE for
the diagnosis of serous cystadenoma, mucinous PCL, and premalignant PCL
were all ≥ 0.95 (with 95 \% confidence interval from 0.85 to 1.0). The
AUROC was significantly larger for nCLE than for CEA or EUS.
CONCLUSIONS:  nCLE had excellent diagnostic performance that surpassed
that of CEA and EUS for the diagnosis of large, single, noncommunicating
PCLs. The nCLE procedure should be considered in patients with
indeterminate PCLs to ensure a more specific diagnosis.

{}

{}

\begin{center}\rule{0.5\linewidth}{\linethickness}\end{center}

Back to top

\begin{center}\rule{0.5\linewidth}{\linethickness}\end{center}

\pagebreak

\hypertarget{gallbladder-1}{%
\subsubsection{Gallbladder}\label{gallbladder-1}}

\begin{center}\rule{0.5\linewidth}{\linethickness}\end{center}

 - \textbf{Mucinous adenocarcinoma of gallbladder: Subcategorisation on
fine-needle aspiration cytology}

\emph{Diagnostic cytopathology 2018 Oct;():}

PubMed: \url{https://www.ncbi.nlm.nih.gov/pubmed/?term=30375181}

BACKGROUND: Mucinous adenocarcinoma (MC) of gallbladder is a rare
histological subtype of gallbladder carcinoma (CaGB) which presents at
an advanced stage and is associated with a poor prognosis compared to
the conventional CaGB. This variant has been described mostly as reports
or series, except for a single detailed histological and
immunohistochemical analysis. Till date, there are no studies describing
the cytomorphology of MC in detail. Hence, we undertook this study to
analyse the cytomorphological features of MC. METHODS: A retrospective
cytomorphological analysis was performed on MC identified out of all
CaGB diagnosed on cytology over a period of last 4 years. The
architectural and cellular features were recorded in a structured
proforma. RESULTS: Thirty-three cases (33/987, 3.3\%) were identified as
MC. Extracellular mucin \textgreater{}90\% was seen only in 3 cases
whereas the remaining 30 had 50\%-90\% mucin. The predominant
architectural pattern was tight epithelial fragments (14/33). The tumour
cells were mostly of intermediate size (31/33) and had moderate amount
of cytoplasm (31/33). Majority of the cases showed moderate nuclear
pleomorphism (28/33) and nuclear chromatin was fine granular (17/33) or
vesicular (14/33). Most of the cases had single and small nucleoli
(26/33). Presence of inflammation composed predominantly of polymorphs
was noted in 25 cases. Majority of the cases showed no (15/33) or scant
necrosis (13/33). CONCLUSION: The morphological features of MC can very
well be demonstrated on cytology. As they are associated with poor
prognosis compared to conventional CaGB, cytopathologists should try to
document the subtype.

{}

{}

\begin{center}\rule{0.5\linewidth}{\linethickness}\end{center}

Back to top

\begin{center}\rule{0.5\linewidth}{\linethickness}\end{center}

\pagebreak

\hypertarget{neuroendocrine-1}{%
\subsubsection{Neuroendocrine}\label{neuroendocrine-1}}

\begin{center}\rule{0.5\linewidth}{\linethickness}\end{center}

 - \textbf{Immunohistochemical analysis of OTP and NKX6.1 in
neuroendocrine tumors of the lung and pancreas}

\emph{Diagnostic cytopathology 2018 Dec;46(12):1010-1014}

PubMed: \url{https://www.ncbi.nlm.nih.gov/pubmed/?term=30284410}

BACKGROUND: Homeobox transcription factors have demonstrated utility in
diagnosing neuroendocrine tumors. Orthopedia homeobox protein (OTP) has
a well-defined role in embryonic neurodevelopment and has also been
described as a prognostic marker in lung neuroendocrine tumors (NET).
Additionally, NK6 homeobox-1 (NKX6.1) has been described to be necessary
for the development of neuroendocrine cells in the pancreas. We
evaluated immunohistochemical (IHC) expression of OTP and NKX6.1 to
determine their utility in the diagnosis of NETs from lung and pancreas
fine-needle aspirations (FNA). METHODS: Our study examined 50 FNA
specimens, including 30 primary pulmonary NETs (8 carcinoid tumors (CT),
6 atypical carcinoids (AC), 11 small-cell neuroendocrine carcinomas
(SCNEC), 5 large-cell neuroendocrine carcinomas (LCNEC)) and 20 primary
pancreatic NETs (17 well-differentiated pancreatic neuroendocrine tumors
(PanNET) and 3 poorly differentiated pancreatic neuroendocrine
carcinomas (PanNEC)). IHC expression of OTP, NKX6.1, and Ki-67 was
evaluated on FNA cell blocks. RESULTS: Half of the pulmonary TC tumors
expressed OTP, while only 17\% of AC and 20\% of LCNEC expressed OTP.
Neither SCNECs nor any pancreatic NET expressed OTP. In contrast,
intermediate and high-grade tumors expressed NKX6.1 (LCNEC-80\%,
SCNEC-82\%, and AC-83\%) more often than low-grade tumors (TC-63\%,
PanNET-71\%). All three PanNECs expressed NKX6.1. CONCLUSIONS: OTP may
be useful in diagnosing well-differentiated NETs of pulmonary origin.
NKX6.1 may have utility in segregating high from low-grade NETs of both
pulmonary and pancreatic origin, although other methods will be required
to determine site of origin.

{}

{}

\begin{center}\rule{0.5\linewidth}{\linethickness}\end{center}

 - \textbf{Advances in the cytologic diagnosis of gastroenteropancreatic
neuroendocrine neoplasms}

\emph{Cancer cytopathology 2018 Dec;126(12):980-991}

PubMed: \url{https://www.ncbi.nlm.nih.gov/pubmed/?term=30485690}

Two-thirds of neuroendocrine neoplasms arising in the human body
originate from the gastrointestinal system or pancreas.
Gastroenteropancreatic neuroendocrine neoplasms are heterogeneous,
comprising both well differentiated neuroendocrine tumors (NETs) and
poorly differentiated neuroendocrine carcinomas (NECs). The clinical
presentation, molecular characteristics, and behavior are distinct for
NETs and NECs. Fine-needle aspiration is an important modality for the
primary diagnosis and staging of these neoplasms and can provide
information of prognostic and therapeutic significance. Our evolving
understanding of neuroendocrine neoplasm biology has led to several
iterations of classification. In this review, new concepts and issues
most relevant to cytology diagnosis of gastroenteropancreatic
neuroendocrine neoplasms are discussed, such as newer detection methods
that aid in diagnosis and staging, recent changes in World Health
Organization classification, practical issues related to grading these
neoplasms on cytology, guidelines for diagnostic reporting, and panels
of immunohistochemical stains for the diagnosis of metastasis. The
current understanding of genetic and epigenetic events related to tumor
development and potential applications for cytology also are presented
as they relate to prognostication and recent therapeutic advances.

{}

{}

\begin{center}\rule{0.5\linewidth}{\linethickness}\end{center}

 - \textbf{Grading by the Ki-67 Labeling Index of Endoscopic
Ultrasound-Guided Fine Needle Aspiration Biopsy Specimens of Pancreatic
Neuroendocrine Tumors Can Be Underestimated}

\emph{Pancreas 2018 9;47(10):1296-1303}

PubMed: \url{https://www.ncbi.nlm.nih.gov/pubmed/?term=30211805}

OBJECTIVES: There is an increasing need for grading with small
endoscopic ultrasound-guided fine needle aspiration biopsy (EUS-FNAB)
specimens for the proper diagnosis and therapy selection of patients
with unresectable pancreatic neuroendocrine tumors (PanNET). However,
our understanding of EUS-FNAB specimen grading is limited compared with
surgically resected specimens. METHODS: We retrospectively determined
Ki-67 labeling index (LI) of 33 matched EUS-FNAB and surgically resected
PanNETs with digital image analyzer. Pairwise grades between the matched
biopsy and surgically resected PanNET specimens were compared. RESULTS:
The mean Ki-67 LI was higher in surgically resected PanNET specimens
(5.5\%) than in biopsy specimens (3.2\%; P = 0.022). There was moderate
agreement between the Ki-67 LI grades when individually evaluated
matched biopsy and resected specimen pairs were compared (κ value =
0.62; P \textless{} 0.0001). However, discordance was noted in 6 cases
(18\%), and all of them were either grade 2 or 3 in resected PanNETs.
CONCLUSIONS: Although Ki-67 LI grading of EUS-FNAB specimens may be
concordant with that of matched surgically resected specimens in a large
proportion of the PanNET cases, Ki-67 LI grading of EUS-FNAB specimens
should be carefully applied in clinical practice because of the
possibility of grading underestimation with grade 2 to 3 PanNET cases.

{}

{}

\begin{center}\rule{0.5\linewidth}{\linethickness}\end{center}

 - \textbf{Accuracy of grading pancreatic neuroendocrine neoplasms with
Ki-67 index in fine-needle aspiration cellblock material}

\emph{Cytopathology : official journal of the British Society for
Clinical Cytology 2018 Oct;():}

PubMed: \url{https://www.ncbi.nlm.nih.gov/pubmed/?term=30303569}

OBJECTIVE: The aim of this study was to assess the preoperative tumour
grade of pancreatic neuroendocrine neoplasms (panNENs) by determining
the Ki-67 index in endoscopic ultrasound-guided fine needle aspiration
(EUS-FNA) material and to correlate the preoperative tumour grade with
the postoperative tumour grade in surgical specimens. METHODS: We
performed a retrospective review of the institutional pathology database
over a 10-year period (2007-2017) to identify all cases of panNENs with
corresponding preoperative EUS-FNA cytological material and surgical
specimens. Fifteen cases with adequate EUS-FNA material (more than 400
tumour cells on cellblock) were identified. The cytological and
histological samples were graded based on the mitotic rate and the Ki-67
index in accordance with the 2017 World Health Organisation grading
system for panNENs. The tumour grades determined on EUS-FNA cellblock
material were compared with the histological tumour grades. RESULTS:
Mean age at diagnosis was 64.8~±~12.7~years (range, 38-85~years). The
grading scores assigned to the cytological and histological samples were
concordant in all 15 (100\%) cases. Of those, two (13\%) cases were
scored as grade 1, nine (60\%) cases as grade 2 and four (27\%) cases as
grade 3 tumours. CONCLUSION: Our study shows that tumour grade in
patients with PanNENs can be reliably determined by assessing the Ki-67
index in EUS-FNA specimens based on the 2017 World Health Organisation
classification and grading system.

{}

{}

\begin{center}\rule{0.5\linewidth}{\linethickness}\end{center}

Back to top

\begin{center}\rule{0.5\linewidth}{\linethickness}\end{center}

\pagebreak

\hypertarget{molecular-pathology}{%
\subsection{Molecular Pathology}\label{molecular-pathology}}

\begin{center}\rule{0.5\linewidth}{\linethickness}\end{center}

\hypertarget{pancreas-2}{%
\subsubsection{Pancreas}\label{pancreas-2}}

\begin{center}\rule{0.5\linewidth}{\linethickness}\end{center}

 - \textbf{Integrated whole genome microarray analysis and
immunohistochemical assay identifies COL11A1, GJB2 and CTRL as
predictive biomarkers for pancreatic cancer}

\emph{Cancer cell international 2018 11;18():174}

PubMed: \url{https://www.ncbi.nlm.nih.gov/pubmed/?term=30410422}

Background: Pancreatic cancer is characterized by its unsatisfying early
detection rate, rapid disease progression and poor prognosis. Further
studies on molecular mechanism and novel predictive biomarkers for
pancreatic cancer based on a large sample volume are required. Methods:
Multiple bioinformatic analysis tools were utilized for identification
and characterization of differentially expressed genes (DEGs) from a
merged microarray data (100 pancreatic cancer samples and 62 normal
samples). Data from the GEO and TCGA database was utilized to validate
the diagnostic and prognostic value of the top 5
upregulated/downregulated DEGs. Immunohistochemical assay (46 paired
pancreatic and para- cancerous samples) was utilized to validate the
expression and prognostic value of COL11A1, GJB2 and CTRL from the
identified DEGs. Results: A total number of 300 DEGs were identified
from the merged microarray data of 100 pancreatic cancer samples and 62
normal samples. These DEGs were closely correlated with the biological
characteristics of pancreatic cancer. The top 5
upregulated/downregulated DEGs showed good individual
diagnostic/prognostic value and better combined diagnostic/prognostic
value. Validation of COL11A1, GJB2 and CTRL with immunohistochemical
assay showed consistent expression level with bioinformatics analysis
and promising prognostic value. Conclusions: Merged microarray data with
bigger sample volume could reflect the biological characteristics of
pancreatic cancer more effectively and accurately. COL11A1, GJB2 and
CTRL are novel predictive biomarkers for pancreatic cancer.

{}

{}

\begin{center}\rule{0.5\linewidth}{\linethickness}\end{center}

 - \textbf{Identification of a 5‑microRNA signature and hub miRNA‑mRNA
interactions associated with pancreatic cancer}

\emph{Oncology reports 2019 Jan;41(1):292-300}

PubMed: \url{https://www.ncbi.nlm.nih.gov/pubmed/?term=30365134}

miRNA‑gene axes have been reported to serve an important role in the
carcinogenesis of pancreatic cancer (PC). The aim of the present study
was to systematically identity the microRNA signature and hub molecules,
as well as hub miRNA‑gene axes, and to explore the potential biomarkers
and mechanisms associated with the carcinogenesis of PC. Eleven microRNA
profile datasets were obtained from the National Center for
Biotechnology Information (NCBI) Gene Expression Omnibus (GEO) and
ArrayExpress databases, and a meta‑analysis was performed to identify
the differentially expressed miRNAs (DEMs) between tumor tissue and
normal tissue. Subsequently, a diagnostic regression model was
constructed to identify PC based on The Cancer Genome Atlas (TCGA) miRNA
sequence data by using the least absolute shrinkage and selection
operator (LASSO) method. In addition, GSE41368 was downloaded, and a
weighted gene co‑expression network analysis (WGCNA) was performed to
obtain the gene module associated with carcinogenesis by using the
TCGAbiolinks and WGCNA packages, respectively. Finally, miRNA‑gene
networks were constructed and visualized using Cytoscape software,
followed by Gene Ontology (GO) and Kyoto Encyclopedia of Genes and
Genomes (KEGG) analyses based on the Database for Annotation,
Visualization, and Integrated Discovery (DAVID). A total of 14 DEMs were
identified, and a 5‑microRNA‑based score generated by the LASSO
regression model provided a high accuracy for identifying PC {[}area
under the curve (AUC)=0.918{]}. In addition, 44 miRNA‑mRNA interactions
were constructed, and 4~hub genes were screened on the basis of the
above bioinformatic tools and databases. Furthermore, 14 biological
process (BP) functions and 6 KEGG pathways were identified according to
gene set enrichment analysis (GSEA). In summary, the present study
applied integrated bioinformatics approaches to generate a holistic view
of PC, thereby providing a basis for further clinical application of the
5‑miRNA signature and the identified hub molecules, as well as the
miRNA‑gene axes, which could serve as diagnostic markers and potential
treatment targets.

{}

{}

\begin{center}\rule{0.5\linewidth}{\linethickness}\end{center}

 - \textbf{Prospective Evaluation of Germline Alterations in Patients
With Exocrine Pancreatic Neoplasms}

\emph{Journal of the National Cancer Institute 2018
Oct;110(10):1067-1074}

PubMed: \url{https://www.ncbi.nlm.nih.gov/pubmed/?term=29506128}

Background: Identification of pathogenic germline alterations (PGAs) has
important clinical and therapeutic implications in pancreas cancer. We
performed comprehensive germline testing (GT) in an unselected
prospective cohort of patients with exocrine pancreatic neoplasms with
genotype and phenotype association to facilitate identification of
prognostic and/or predictive biomarkers and examine potential
therapeutic implications. Methods: Six hundred fifteen unselected
patients with exocrine pancreatic neoplasms were prospectively consented
for somatic tumor and matched sample profiling for 410-468 genes. GT for
PGAs in 76 genes associated with cancer susceptibility was performed in
an ``identified'' manner in 356 (57.9\%) patients and in an
``anonymized'' manner in 259 (42.1\%) patients, using an institutional
review board-approved protocol. Detailed clinical and pathological
features, response to platinum, and overall survival (OS) were collected
for the identified cohort. OS was analyzed with Kaplan-Meier curves.
Results: PGAs were present in 122 (19.8\%) of 615 patients involving 24
different genes, including BRCA1/2, ATM, PALB2, and multiple additional
genes associated with the DNA damage response pathway. Of 122 patients
with germline alterations, 41.8\% did not meet current guidelines for
GT. The difference in median OS was not statistically significant
between patients with and without PGA (50.8 months, 95\% confidence
interval = 34.5 to not reached, two-sided P = .94). Loss of
heterozygosity was found in 60.0\% of BRCA1/2. Conclusions: PGAs
frequently occur in pancreas exocrine neoplasms and involve multiple
genes beyond those previously associated with hereditary pancreatic
cancer. These PGAs are therapeutically actionable in about 5\% to 10\%
of patients. These data support routinely offering GT in all pancreatic
ductal adenocarcimona patients with a broad panel of known hereditary
cancer predisposition genes.

{}

{}

\begin{center}\rule{0.5\linewidth}{\linethickness}\end{center}

 - \textbf{Identification of hub genes with diagnostic values in
pancreatic cancer by bioinformatics analyses and supervised learning
methods}

\emph{World journal of surgical oncology 2018 Nov;16(1):223}

PubMed: \url{https://www.ncbi.nlm.nih.gov/pubmed/?term=30428899}

BACKGROUND: Pancreatic cancer is one of the most lethal tumors with poor
prognosis, and lacks of effective biomarkers in diagnosis and treatment.
The aim of this investigation was to identify hub genes in pancreatic
cancer, which would serve as potential biomarkers for cancer diagnosis
and therapy in the future. METHODS: Combination of two expression
profiles of GSE16515 and GSE22780 from Gene Expression Omnibus (GEO)
database was served as training set. Differentially expressed genes
(DEGs) with top 25\% variance followed by protein-protein interaction
(PPI) network were performed to find candidate genes. Then, hub genes
were further screened by survival and cox analyses in The Cancer Genome
Atlas (TCGA) database. Finally, hub genes were validated in GSE15471
dataset from GEO by supervised learning methods k-nearest neighbor (kNN)
and random forest algorithms. RESULTS: After quality control and batch
effect elimination of training set, 181 DEGs bearing top 25\% variance
were identified as candidate genes. Then, two hub genes, MMP7 and ITGA2,
correlating with diagnosis and prognosis of pancreatic cancer were
screened as hub genes according to above-mentioned bioinformatics
methods. Finally, hub genes were demonstrated to successfully differ
tumor samples from normal tissues with predictive accuracies reached to
93.59 and 81.31\% by using kNN and random forest algorithms,
respectively. CONCLUSIONS: All the hub genes were associated with the
regulation of tumor microenvironment, which implicated in tumor
proliferation, progression, migration, and metastasis. Our results
provide a novel prospect for diagnosis and treatment of pancreatic
cancer, which may have a further application in clinical.

{}

{}

\begin{center}\rule{0.5\linewidth}{\linethickness}\end{center}

 - \textbf{Higher notch expression implies poor survival in pancreatic
ductal adenocarcinoma: A systematic review and meta-analysis}

\emph{Pancreatology : official journal of the International Association
of Pancreatology (IAP) \ldots{} {[}et al.{]} 2018 Dec;18(8):954-961}

PubMed: \url{https://www.ncbi.nlm.nih.gov/pubmed/?term=30297095}

BACKGROUND: At present, pancreatic ductal adenocarcinoma (PDAC) is a
fetal disease lack of effective prognostic and therapeutic methods
resulting in high mortality. The Notch signaling has been demonstrated
being up- or down-regulated in many cancers, but the effects in
pancreatic ductal adenocarcinoma are still controversial. Moreover, the
available cases in an individual study are of small samples. Therefore,
it is essential to define the effect of Notch signaling in pancreatic
ductal adenocarcinoma with larger samples. METHODS: Conducted from 6
eligible studies and 463 pancreatic ductal adenocarcinoma patients, this
was the first meta-analysis to analyze the correlation between the Notch
signal pathway and pancreatic ductal adenocarcinoma. All data were
sourced from The National Center for Biotechnology Information, Web of
Science and Cochrane. The articles which matched the inclusion criteria
were included. All included data were analyzed and performed by Review
Manager 5.3. RESULTS: The results indicated that high expression of
Notch signaling proteins was associated with poor overall survival of
pancreatic ductal adenocarcinoma patients (pooled hazard
ratio\textgreater{}2.00; P\,\textless{}\,0.001). Moreover, poor survival
was related to high expression of Notch3 (pooled hazard ratio: 2.05;
confidence interval: 1.49-2.82; P\,\textless{}\,0.001) and DLL4 (pooled
hazard ratio: 2.13; confidence interval: 1.37-3.32;
P\,\textless{}\,0.001). CONCLUSIONS: This meta-analysis supports that
Notch signaling proteins may be available as prognostic factors for
pancreatic ductal adenocarcinoma progression and patient survival.
Higher expression of Notch signaling proteins indicated poor survival of
pancreatic ductal adenocarcinoma patients. Targeting Notch signaling
components, especially Notch3 protein, would be beneficial for
therapies.

{}

{}

\begin{center}\rule{0.5\linewidth}{\linethickness}\end{center}

 - \textbf{SNX6 predicts poor prognosis and contributes to the
metastasis of pancreatic cancer cells via activating
epithelial-mesenchymal transition}

\emph{Acta biochimica et biophysica Sinica 2018 Nov;50(11):1075-1084}

PubMed: \url{https://www.ncbi.nlm.nih.gov/pubmed/?term=30307473}

Pancreatic cancer remains a challenging disease with an overall
cumulative 5-year survival rate around 6\%. Though significant progress
has been made in the availability of diagnostic techniques and treatment
strategies, pancreatic cancer remains a disease of high mortality rate.
Therefore, there is an urgent need for a better understanding of the
molecular mechanisms that governs the oncogenesis and metastasis process
of pancreatic cancer. In the present study, by using the Cancer Genome
Atlas (TCGA) dataset analysis, we demonstrated that sorting nexin 6
(SNX6) serves as a biomarker for predicting prognosis of pancreatic
cancer. In vitro studies demonstrated that silencing of SNX6 expression
reduced cell proliferation, colony formation, invasion, and metastasis.
Higher level of SNX6 helps maintain the mesenchymal properties, which
renders migration and invasive capacities to pancreatic cancer cells.
Moreover, in the process of TGF-β-induced epithelial to mesenchymal
transition (EMT), the expression level of SNX6 was increased, and
silencing of SNX6 expression could inhibit the TGF-β-induced EMT
program. These results collectively uncovered a novel predictive marker
for pancreatic cancer and provided the possible underlying molecular
mechanism.

{}

{}

\begin{center}\rule{0.5\linewidth}{\linethickness}\end{center}

 - \textbf{Identification of key microRNAs and their targets in exosomes
of pancreatic cancer using bioinformatics analysis}

\emph{Medicine 2018 Sep;97(39):e12632}

PubMed: \url{https://www.ncbi.nlm.nih.gov/pubmed/?term=30278585}

Pancreatic cancer (PC) is one of the most lethal tumors, due to late
diagnosis and limited surgical strategies. It has been reported that
serum exosomal microRNAs (S-Exo-miRNAs) play a pivotal role as signaling
molecules and serve as noninvasive diagnosis methods for PC. The
combination of S-Exo-miRNAs with the corresponding target also plays an
important role in the tumor microenvironment.Here we investigated
S-Exo-miRNAs involved in PC. The gene expression profile was downloaded
from the Gene Expression Omnibus (GEO) database. The analysis was
carried out using GEO2R. The targets of differentially expressed serum
exosomal miRNAs (DE-S-Exo-miRNAs) were predicted by 4 bioinformatic
algorithms (miRanda, miRDB, miRWalk, and Targetscan). Further analysis
with gene ontology (GO) and Kyoto Encyclopedia of Genomes pathway (KEGG)
enrichment analyses were performed with Cytoscape software version
3.4.0. Subsequently, the interaction regulatory network of target genes
was performed with the Search Tool for the Retrieval of Interacting
Genes (STRING) database (\url{http://www.string-db.org/}) and visualized
using Cytoscape software.We downloaded the gene expression profile
GSE50632, which was based on an Agilent microarray GPL17660 platform
containing 4 eligible samples. In total 467 DE-S-Exo-miRNAs were
obtained, including 7 overexpressed miRNAs (1.50\%), and 460 remaining
underexpressed miRNAs (98.50\%). The databases miRWalk, miRDB, miRanda,
and TargetScan were used to predict their potential targets, which were
subsequently submitted to Cytoscape software version 3.4.0
(www.cytoscape.org). Next the functional and pathway enrichment analysis
were used for the KEGG pathway and GO categories analysis. The
enrichment analysis identified the genes involved in such processes as
developmental and negative regulation of multicellular organismal
processes, regulation of anatomical structure morphogenesis, regulation
of cell death, apoptotic processes and mitogen-activated protein kinase
(MAPK) signaling pathway, transforming growth factor - beta (TGF -β)
signaling pathway, cyclic adenosine monophosphate (cAMP) signaling
pathway, and the phosphatidylinositol-3 kinases/Akt (PI3K-Akt) signaling
pathway. Subsequently according to the protein-protein interaction (PPI)
network, the top 10 genes were obtained. The enrichment analyses of the
genes involved in a significant module revealed that these genes were
related to the TGF-β signaling pathway. After reviewing the literature,
we identified the apoptosis genes, and their corresponding miRNAs that
have a relationship with apoptosis of the tumor.This analysis provides a
comprehensive understanding of the roles of S-Exo-miRNAs and the related
targets in the development of PC. Additionally, the present study
provides promising candidate targets for early diagnosis and therapeutic
intervention. However, these predictions require further experimental
validation in future studies.

{}

{}

\begin{center}\rule{0.5\linewidth}{\linethickness}\end{center}

 - \textbf{Transcriptome-wide association study identifies multiple
genes and pathways associated with pancreatic cancer}

\emph{Cancer medicine 2018 Nov;7(11):5727-5732}

PubMed: \url{https://www.ncbi.nlm.nih.gov/pubmed/?term=30334361}

AIM: To identify novel candidate genes for pancreatic cancer. METHODS:
We performed a transcriptome-wide association study (TWAS) analysis of
pancreatic cancer (PC). GWAS summary data were driven from the published
studies of PC, totally involving 558~542 SNPs in 1896 individuals with
pancreatic cancer and 1939 healthy controls. FUSION software was applied
to the PC GWAS summary data for tissue-related TWAS analysis, including
whole blood, peripheral blood, adipose, and pancreas. The functional
relevance of identified genes with PC was further validated by Oncomine,
STRING, and CluePedia tool. RESULTS: Transcriptome-wide association
study analysis identified 19 genes significantly associated with PC,
such as LRP5L (P value~=~5.21~×~10-5 ), SOX4 (P value~=~3.2~×~10-4 ),
and EGLN3 (P value~=~6.2~×~10-3 ). KEGG pathway enrichment analysis
detected several PC-associated pathways, such as One carbon pool by
folate (P value~=~1.60~×~10-16 ), Cell cycle (P value~=~1.27~×~10-7 ),
TGF-beta signaling pathway (P value~=~4.64~×~10-6 ). Further comparing
the 19 genes with previously identified overexpressed genes in PC
patients found one overlapped gene SOX4. CONCLUSION: We identified some
novel candidate genes and pathways associated with PC. Our results
provide novel clues for the genetic mechanism studies of pancreatic
cancer.

{}

{}

\begin{center}\rule{0.5\linewidth}{\linethickness}\end{center}

 - \textbf{Gene expression analysis of embryonic pancreas development
master regulators and terminal cell fate markers in resected pancreatic
cancer: A correlation with clinical outcome}

\emph{Pancreatology : official journal of the International Association
of Pancreatology (IAP) \ldots{} {[}et al.{]} 2018 Dec;18(8):945-953}

PubMed: \url{https://www.ncbi.nlm.nih.gov/pubmed/?term=30293872}

BACKGROUND: Despite the recent introduction of new drugs and the
development of innovative multi-target treatments, the prognosis of
pancreatic ductal adenocarcinoma (PDAC) remains very poor. Even when
PDAC is resectable, the rate of local or widespread disease recurrence
remains particularly high. Currently, reliable prognostic biomarkers of
recurrence are lacking. We decided to explore the potential usefulness
of pancreatic developmental regulators as biomarkers of PDAC relapse.
METHODS: We analyzed by quantitative real-time PCR the mRNA of selected
factors involved either in pancreatic organogenesis (ISL1, NEUROD1,
NGN3, NKX2.2, NKX6.1, PAX4, PAX6, PDX1 and PTF1α) or associated with
terminally committed pancreatic cells (CHGA, CHGB, GAD2, GCG, HNF6α,
INS, KRT19, SYP) in 17 PDAC cell lines and in frozen tumor samples from
41 PDAC patients. RESULTS: High baseline levels of the ISL1, KRT19, PAX6
and PDX1 mRNAs in PDAC cell lines, were risk factors for time-dependent
xenograft appearance after subcutaneous injection in CD1-Nude mice.
Consistently, in human PDAC samples, high levels of KRT19 mRNA were
associated with reduced overall survival and earlier recurrence. Higher
levels of PDX1 or PAX6 mRNAs were instead associated with a higher
frequency of local recurrence. CONCLUSIONS: Our findings suggest that
selected factors associated with pancreas development or its terminal
differentiation might be implicated in mechanisms of PDAC progression
and/or metastatic spread and that the measurement of their mRNA in
tumors might be potentially used to improve patient prognostic
stratification and prediction of the relapse site.

{}

{}

\begin{center}\rule{0.5\linewidth}{\linethickness}\end{center}

 - \textbf{The receptor for advanced glycation end products: A fuel to
pancreatic cancer}

\emph{Seminars in cancer biology 2018 04;49():37-43}

PubMed: \url{https://www.ncbi.nlm.nih.gov/pubmed/?term=28811077}

The receptor for advanced glycation end products (RAGEs) was first
illustrated in the year 1992. RAGE is a single-transmembrane and
multi-ligand component of the immunoglobulin protein super family. The
engagement of RAGE turns out to an establishment of numerous
intracellular signalling mechanisms resulting in the progression and
perpetuation of many types of cancer including, the pancreatic cancer.
The present review primarily focuses on the multi-ligand activation of
RAGEs leading to the downstream signalling cascade activation. The kick
start of the RAGEs activation leads to the several anomalies and
includes multiple types of cancers. The RAGE expression correlates well
with the survival of pancreatic cancer cells leading to the myeloid
response. RAGEs assist in the tumourogenesis which enhance and thrive to
its fullest in the stressed tumour microenvironment. An improved
perceptive of its involvement in pancreatic cancer may offer novel
targets for tumour supervision and risk measurement.

{}

{}

\begin{center}\rule{0.5\linewidth}{\linethickness}\end{center}

Back to top

\begin{center}\rule{0.5\linewidth}{\linethickness}\end{center}

\pagebreak

\hypertarget{molecular-techniques-research-methods-liquid-biopsy}{%
\paragraph{Molecular Techniques \& Research Methods, Liquid
Biopsy}\label{molecular-techniques-research-methods-liquid-biopsy}}

Molecular Techniques \& Research Methods, Liquid Biopsy

\begin{center}\rule{0.5\linewidth}{\linethickness}\end{center}

 - \textbf{Plasma miR-181a-5p Downregulation Predicts Response and
Improved Survival After FOLFIRINOX in Pancreatic Ductal Adenocarcinoma}

\emph{Annals of surgery 2018 Nov;():}

PubMed: \url{https://www.ncbi.nlm.nih.gov/pubmed/?term=30394883}

OBJECTIVE: The aim of the study was to identify plasma microRNA (miRNA)
biomarkers for stratifying and monitoring patients with locally advanced
or metastatic pancreatic ductal adenocarcinoma (PDAC) treated with
FOLFIRINOX, and to investigate their functional roles. SUMMARY
BACKGROUND DATA: FOLFIRINOX has become a standard therapy for patients
with advanced PDAC and can be used to potentially downstage disease.
However, only a subset of patients respond, and biomarkers to guide
decision-making are urgently needed. METHODS: We used microarray-based
profiling to discover deregulated miRNAs in pre- and postchemotherapy
plasma samples from patients based on their progression-free survival
(PFS) after FOLFIRINOX. Nine candidate plasma miRNAs were validated in
an independent cohort (n = 43). The most discriminative plasma miRNA was
correlated with clinicopathological factors and survival, and also
investigated in an additional cohort treated with gemcitabine plus
nab-paclitaxel. Expression patterns were further evaluated in matched
tumor tissues. In vitro studies explored its function, key downstream
gene-targets, and interaction with 5-fluorouracil, irinotecan, and
oxaliplatin. RESULTS: Plasma miR-181a-5p was significantly downregulated
in non-progressive patients after FOLFIRINOX. In multivariate analysis,
this decline correlated with improved PFS and overall survival,
especially when combined with CA19-9 decline {[}hazard ratio (HR) =
0.153, 95\% confidence interval (CI), 0.067-0.347 and HR = 0.201, 95\%
CI, 0.070-0.576, respectively{]}. This combination did not correlate
with survival in patients treated with gemcitabine plus nab-paclitaxel.
Tissue expression of miR-181a-5p reflected plasma levels. Inhibition of
miR-181a-5p coupled with oxaliplatin exposure in pancreatic cell lines
decreased cell viability. CONCLUSIONS: Plasma miR-181a-5p is a specific
biomarker for monitoring FOLFIRINOX response. Decline in plasma
miR-181a-5p and CA19-9 levels is associated with better prognosis after
FOLFIRINOX and may be useful for guiding therapeutic choices and
surgical exploration.This is an open access article distributed under
the terms of the Creative Commons Attribution-Non Commercial-No
Derivatives License 4.0 (CCBY-NC-ND), where it is permissible to
download and share the work provided it is properly cited. The work
cannot be changed in any way or used commercially without permission
from the journal.
\url{http://creativecommons.org/licenses/by-nc-nd/4.0}.

{}

{}

\begin{center}\rule{0.5\linewidth}{\linethickness}\end{center}

Back to top

\begin{center}\rule{0.5\linewidth}{\linethickness}\end{center}

\pagebreak

\hypertarget{microenvironment}{%
\paragraph{Microenvironment}\label{microenvironment}}

Tumor Stroma Interactions, Microenvironment, Inflammatory Response,
Epithelial Mesenchymal Transition

\begin{center}\rule{0.5\linewidth}{\linethickness}\end{center}

 - \textbf{Typing of pancreatic cancer-associated fibroblasts identifies
different subpopulations}

\emph{World journal of gastroenterology 2018 Nov;24(41):4663-4678}

PubMed: \url{https://www.ncbi.nlm.nih.gov/pubmed/?term=30416314}

AIM: To determine whether it is possible to identify different immune
phenotypic subpopulations of cancer-associated fibroblasts (CAFs) in
pancreatic cancer (PC). METHODS: We defined four different stromal
compartments in surgical specimens with PC: The juxtatumoural,
peripheral, lobular and septal stroma. Tissue microarrays were produced
containing all pre-defined PC compartments, and the expression of 37
fibroblast (FB) and 8 extracellular matrix (ECM) markers was evaluated
by immunohistochemistry, immunofluorescence (IF), double-IF, and/or in
situ hybridization. The compartment-specific mean labelling score was
determined for each marker using a four-tiered scoring system. DOG1 gene
expression was examined by quantitative reverse transcription PCR
(qPCR). RESULTS: CD10, CD271, cytoglobin, DOG1, miR-21, nestin, and
tenascin C exhibited significant differences in expression profiles
between the juxtatumoural and peripheral compartments. The expression of
CD10, cytoglobin, DOG1, nestin, and miR-21 was moderate/strong in
juxtatumoural CAFs (j-CAFs) and barely perceptible/weak in peripheral
CAFs (p-CAFs). The upregulation of DOG1 gene expression in PC compared
to normal pancreas was verified by qPCR. Tenascin C expression was
strong in the juxtatumoural ECM and barely perceptible/weak in the
peripheral ECM. CD271 expression was barely perceptible in j-CAFs but
moderate in the other compartments. Galectin-1 was stronger expressed in
j-CAFs vs septal fibroblasts, PDGF-Rβ, tissue transglutaminase 2, and
hyaluronic acid were stronger expressed in lobular fibroblasts vs
p-CAFs, and plectin-1 was stronger expressed in j-CAFs vs l-FBs. The
expression of the remaining 33 markers did not differ significantly when
related to the quantity of CAFs/FBs or the amount of ECM in the
respective compartments. CONCLUSION: Different immune phenotypic CAF
subpopulations can be identified in PC, using markers such as
cytoglobin, CD271, and miR-21. Future studies should determine whether
CAF subpopulations have different functional properties.

{}

{}

\begin{center}\rule{0.5\linewidth}{\linethickness}\end{center}

 - \textbf{Mast cells and angiogenesis in pancreatic ductal
adenocarcinoma}

\emph{Clinical and experimental medicine 2018 Aug;18(3):319-323}

PubMed: \url{https://www.ncbi.nlm.nih.gov/pubmed/?term=29492715}

Mast cells are recognized as critical components of the tumor stromal
microenvironment in several solid and hematological malignancies,
promoting angiogenesis and tumor growth. A correlation between mast
cells infiltration, angiogenesis and tumor progression has been reported
for pancreatic ductal adenocarcinoma as well. Mast cells contribute to
the aggressiveness of the pancreatic ductal carcinoma enhancing the
expression of several pro-angiogenic factors such as vascular
endothelial growth factor, fibroblast growth factor-2, platelet-derived
growth factor and angiopoietin-1 as well as stimulating the pancreatic
cancer cells proliferation by IL-13 and tryptase. The disruption of this
pro-angiogenic and proliferative stimulation by inhibiting the mast
cells migration and degranulation is under investigation as a potential
therapeutic approach in pancreatic ductal adenocarcinoma patients. This
review will summarize the literature concerning the mast cells
infiltration in the pancreatic ductal adenocarcinoma analyzing its role
in angiogenesis and tumor progression.

{}

{}

\begin{center}\rule{0.5\linewidth}{\linethickness}\end{center}

 - \textbf{Rab14 overexpression regulates gemcitabine sensitivity
through regulation of Bcl-2 and mitochondrial function in pancreatic
cancer}

\emph{Virchows Archiv : an international journal of pathology 2018
Sep;():}

PubMed: \url{https://www.ncbi.nlm.nih.gov/pubmed/?term=30267303}

Rab family protein Rab14 has been implicated in the development of human
cancers. To date, its expression pattern, biological function, and
potential mechanism in pancreatic cancer have not been explored. In this
study, we analyzed Rab14 expression in 103 cases of pancreatic cancer
tissues using immunohistochemistry (IHC) and found that Rab14 was
overexpressed in 41/103 cases (39.8\%). Rab14 overexpression correlated
with the advanced stage. Moreover, elevated Rab14 levels indicated poor
prognosis of patients with pancreatic cancers. We used BxPC-3 and
Capan-2 respectively for plasmid and siRNA transfection. MTT and colony
formation assays showed that Rab14 transfection increased cell
proliferation and colony formation in BxPC-3 cells. Rab14 siRNA
knockdown inhibits proliferation and colony formation ability in Capan-2
cell line. Cell cycle analysis showed that Rab14 facilitated cell cycle
progression. Matrigel invasion assay showed that Rab14 promoted BxPC-3
cell invasion while its depletion inhibited Capan-2 cell invasion. In
addition, MTT and AnnexinV/PI analysis demonstrated that overexpression
of Rab14 reduced gemcitabine sensitivity which conversely was increased
by Rab14 knockdown. We also demonstrated that Rab14 upregulated
mitochondrial membrane potential (MMP) while its depletion downregulated
MMP during gemcitabine treatment. In addition, western blotting revealed
that Rab14 overexpression upregulated cyclin D1, cyclin A, cyclin E,
p-Rb, and Bcl-2 and downregulated p21. Rab14 also downregulated
caspase3, PARP cleavage, and cytochrome c release. In conclusion, our
data indicated that Rab14 was overexpressed in pancreatic cancer and
promotes growth and gemcitabine resistance, possibly through regulation
of mitochondrial function and Bcl-2.

{}

{}

\begin{center}\rule{0.5\linewidth}{\linethickness}\end{center}

 - \textbf{Tumor-Infiltrating NETs Predict Postsurgical Survival in
Patients with Pancreatic Ductal Adenocarcinoma}

\emph{Annals of surgical oncology 2018 Oct;():}

PubMed: \url{https://www.ncbi.nlm.nih.gov/pubmed/?term=30374923}

BACKGROUND: Tumor-infiltrating neutrophils (TINs) indicate poor
prognosis for patients with pancreatic ductal adenocarcinoma (PDAC).
Activated neutrophils can generate neutrophil extracellular traps
(NETs). Little is known about the presence and prognostic significance
of tumor-infiltrating NETs in PDAC. METHODS: This study enrolled 317
patients, in two independent sets (training and validation), who
underwent curative pancreatectomy for PDAC in Shanghai Cancer Center.
TINs and NETs were identified by immunohistochemical staining for CD15
and citrullinated histone H3, respectively. The relationship between
clinicopathological features and outcomes was analyzed. Accuracy of
prognostic prediction models was evaluated using concordance index
(C-index) and Akaike information criterion (AIC). RESULTS: NETs were
associated with OS (both, P \textless{} 0.001) and RFS (both,
P \textless{} 0.001) in the training and validation sets.
Tumor-infiltrating NETs predicted poor postsurgical survival of patients
with PDAC. Moreover, multivariate analysis identified NETs and AJCC TNM
stage as two independent prognostic factors for OS and RFS. Combination
of NETs with the 8th edition TNM staging system (C-index, 0.6994 and
0.6669, respectively; AIC, 1067 and 1126, respectively) generated a
novel model that improved the predictive accuracy for survival in both
sets (C-index, 0.7254 and 0.7117, respectively; AIC, 1047 and 1102,
respectively). The model combining presence of NETs with the 7th edition
AJCC TNM staging system also had improved predictive accuracy.
CONCLUSIONS: NETs were an independent prognostic factor in PDAC and
incorporation of NETs along with the standard TNM stating system refined
risk-stratification and predicted survival in PDAC with improved
accuracy.

{}

{}

\begin{center}\rule{0.5\linewidth}{\linethickness}\end{center}

 - \textbf{Mesenchymal-epithelial transition of pancreatic cancer cells
at perineural invasion sites is induced by Schwann cells}

\emph{Pathology international 2018 Apr;68(4):214-223}

PubMed: \url{https://www.ncbi.nlm.nih.gov/pubmed/?term=29457853}

Epithelial-mesenchymal transition (EMT) promotes invasion and metastasis
of pancreatic ductal adenocarcinoma (PDAC). However, the importance of
its reverse process, mesenchymal-epithelial transition (MET), for PDAC
remains unclear. We aimed to characterize the histological finding
``focal differentiation'' in PDAC at perineural invasion sites in the
context of MET and to investigate the role of Schwann cells in inducing
tumor MET. Tumor differentiation and immunohistochemical expressions of
E-cadherin, SMAD3, and vimentin at perineural invasion sites were
examined in 168 PDAC tissues. Four PDAC cell lines were co-cultured with
Schwann cells to investigate cell morphology, motility, or EMT-related
markers using immunocytochemistry and quantitative PCR. Of 168 tumors,
124 (74\%) showed focal differentiation with enhanced E-cadherin
membrane expression (P \textless{} 0.001) and decreased nuclear
accumulation of SMAD3 (P \textless{} 0.001). Among 115 PDACs harboring
grade 1/2 tumor, tumors with focal differentiation showed worse survival
compared to those without focal differentiation (P = 0.019). PDAC cells
co-cultured with Schwann cells demonstrated a sheet-like appearance,
increased E-cadherin expression, decreased expressions of SMAD3 and
vimentin, and reduced cell motility. In conclusion, MET-like change is
induced by Schwann cells, suggesting that Schwann cells contribute to
PDAC colonization in pancreatic nerves through activating the MET
machinery inside tumor cells in the pancreatic tumor microenvironment.

{}

{}

\begin{center}\rule{0.5\linewidth}{\linethickness}\end{center}

 - \textbf{Expression of Epithelial-Mesenchymal Transition Proteins in
Pancreatic Anaplastic (Undifferentiated) Carcinoma}

\emph{Pancreas 2019 Jan;48(1):36-42}

PubMed: \url{https://www.ncbi.nlm.nih.gov/pubmed/?term=30451796}

OBJECTIVES: The aim of this study was to identify an association of
pancreatic anaplastic carcinoma (APC) with the epithelial-mesenchymal
transition (EMT). METHODS: Resected APCs (n = 24) were examined to
assess components of APCs, including carcinomatous, transitional, and
sarcomatous regions. Analysis was performed based on the
immunoreactivity of E-cadherin and 3 EMT-related proteins: Slug (zinc
finger protein SNAI2), Twist (Twist-related protein 1), and Zeb1 (zinc
finger E-box-binding homeobox 1). Expression score was determined based
on staining intensity and stained area of the target cells. Finally, we
performed a hierarchical clustering based on the expression pattern of
E-cadherin and EMT-related proteins of the sarcomatous component.
RESULTS: The expression score of E-cadherin decreased in the order of
sarcomatous \textgreater{} transitional \textgreater{} carcinomatous
components (P \textless{} 0.01). Although there were significant
differences in the immunohistochemical scores of Slug, Twist, and Zeb1
between carcinomatous and transitional components (P \textless{} 0.01),
the significant difference in immunohistochemical score of Zeb1 between
transitional and sarcomatous components was found (P \textless{} 0.05).
Furthermore, APCs were divided into 2 subgroups based on the expression
patterns of E-cadherin and EMT-related proteins (hierarchical clustering
analysis). Consequently, these subgroups were distinguished by Twist
expression. CONCLUSIONS: Epithelial-mesenchymal transition plays an
essential role in the pathogenesis of APC.

{}

{}

\begin{center}\rule{0.5\linewidth}{\linethickness}\end{center}

 - \textbf{Hyperglycemia aggravates microenvironment hypoxia and
promotes the metastatic ability of pancreatic cancer}

\emph{Computational and structural biotechnology journal 2018
10;16():479-487}

PubMed: \url{https://www.ncbi.nlm.nih.gov/pubmed/?term=30455857}

Background: Diabetes mellitus and pancreatic cancer are intimately
related. Our previous studies showed that high levels of blood glucose
promote epithelial-mesenchymal transition of pancreatic cancer. In this
study, we evaluated the relationship between hyperglycemia and hypoxic
tumor microenvironments. Methods: HIF-1α expression was evaluated by
immunohistochemistry in clinical pancreatic cancer tissues with or
without diabetes mellitus. Statistcal analysis was performed to explore
the relationship between HIF-1α expression and pathological features of
patients with pancreatic cancer. In vivo and in vitro models was
established to detect whether a hyperglycemia environment could cause
hypoxia in the pancreatic parenchyma and promote pancreatic cancer. In
addition, we also tested the effect of HIF-1α siRNA on the high
glucose-induced invasive and migratory abilities of BxPC-3 cells in
culture. Result: Our data showed that pancreatic cancer patients with
diabetes had a higher level of HIF-1α expression as well as biliary duct
invasion and larger tumor volumes than individuals in the euglycemic
group. Diabetic nude mice treated with streptozotocin (STZ) exhibited
larger tumors and were more likely to develop liver metastasis than
control mice. Acinar cells of the pancreas in diabetic mice showed an
obvious expansion of the endoplasmic reticulum and increased nuclear
gaps as well as chromatin close to the cellular membrane in some acinar
cells. The expression area for Hypoxyprobe-1 and HIF-1α in the diabetic
orthotopic xenograft group was larger than that in the control group.
The expression level of HIF-1α in the BxPC-3 cancer cell line increased
in response to high glucose and CoCl2 concentrations. The high
glucose-induced invasive ability, migratory capacity and MMP-9
expression were counter-balanced by siRNA specific to HIF-1α.
Conclusion: Our results demonstrate that the association between
hyperglycemia and poor prognosis can be attributed to microenvironment
hypoxia in pancreatic cancer.

{}

{}

\begin{center}\rule{0.5\linewidth}{\linethickness}\end{center}

 - \textbf{Elucidating the link between collagen and pancreatic cancer:
what's next?}

\emph{Expert review of gastroenterology \& hepatology 2018
04;12(4):315-317}

PubMed: \url{https://www.ncbi.nlm.nih.gov/pubmed/?term=29495889}

{}

{}

\begin{center}\rule{0.5\linewidth}{\linethickness}\end{center}

Back to top

\begin{center}\rule{0.5\linewidth}{\linethickness}\end{center}

\pagebreak

\hypertarget{molecular-research-on-microenvironment}{%
\paragraph{Molecular Research on
Microenvironment}\label{molecular-research-on-microenvironment}}

Tumor Stroma Interactions, Microenvironment, Inflammatory Response,
Microbiome

\begin{center}\rule{0.5\linewidth}{\linethickness}\end{center}

 - \textbf{The microbiome of pancreatic cancer: from molecular
diagnostics to new therapeutic approaches to overcome chemoresistance
caused by metabolic inactivation of gemcitabine}

\emph{Expert review of molecular diagnostics 2018 Nov;():1-5}

PubMed: \url{https://www.ncbi.nlm.nih.gov/pubmed/?term=30392417}

Pancreatic cancer is a complex disease, with an extremely poor response
to chemotherapy. Emerging evidence indicates that the tumor
microenvironment (TME) might play an important role in mediating
chemoresistance. Areas covered: The evaluated study by Geller and
collaborators describes several bacterial species within pancreatic
tumor tissues and TME and investigated their roles in gemcitabine
chemoresistance. Intratumor bacteria express the enzyme cytidine
deaminase (CDD), whose long form (CDDL) was shown to metabolize
gemcitabine into its inactive metabolite. CDDL is mostly expressed by
Gammaproteobacteria and this was among the most common species in
pancreatic cancer tissues. Interestingly, mouse models of colorectal
cancer injected with bacterial CDDL displayed a reduced response to
gemcitabine, but this resistance was neutralized by the antibiotic
ciprofloxacin. Expert Commentary: The increased knowledge on the
microbiome in pancreatic tissues, as well as its role in
chemoresistance, will provide innovative prognostic and therapeutic
strategies.

{}

{}

\begin{center}\rule{0.5\linewidth}{\linethickness}\end{center}

 - \textbf{GPR68, a proton-sensing GPCR, mediates interaction of
cancer-associated fibroblasts and cancer cells}

\emph{FASEB journal : official publication of the Federation of American
Societies for Experimental Biology 2018 03;32(3):1170-1183}

PubMed: \url{https://www.ncbi.nlm.nih.gov/pubmed/?term=29092903}

The microenvironment of pancreatic ductal adenocarcinoma (PDAC) is
characterized by a dense fibrotic stroma (desmoplasia) generated by
pancreatic cancer-associated fibroblasts (CAFs) derived from pancreatic
stellate cells (PSCs) and pancreatic fibroblasts (PFs). Using an
unbiased GPCRomic array approach, we identified 82 G-protein-coupled
receptors (GPCRs) commonly expressed by CAFs derived from 5 primary PDAC
tumors. Compared with PSCs and PFs, CAFs have increased expression of
GPR68 (a proton-sensing GPCR), with the results confirmed by
immunoblotting, The Cancer Genome Atlas data, and immunohistochemistry
of PDAC tumors. Co-culture of PSCs with PDAC cells, or incubation with
TNF-α, induced GPR68 expression. GPR68 activation (by decreasing the
extracellular pH) enhanced IL-6 expression via a cAMP/PKA/cAMP response
element binding protein signaling pathway. Knockdown of GPR68 by short
interfering RNA diminished low pH-induced production of IL-6 and
enhancement of PDAC cell proliferation by CAF conditioned media. CAFs
from other gastrointestinal cancers also express GPR68. PDAC cells thus
induce expression by CAFs of GPR68, which senses the acidic
microenvironment, thereby increasing production of fibrotic markers and
IL-6 and promoting PDAC cell proliferation. CAF-expressed GPR68 is a
mediator of low-pH-promoted regulation of the tumor microenvironments,
in particular to PDAC cell-CAF interaction and may be a novel
therapeutic target for pancreatic and perhaps other types of
cancers.-Wiley, S. Z., Sriram, K., Liang, W., Chang, S. E., French, R.,
McCann, T., Sicklick, J., Nishihara, H., Lowy, A. M., Insel, P. A.
GPR68, a proton-sensing GPCR, mediates interaction of cancer-associated
fibroblasts and cancer cells.

{}

{}

\begin{center}\rule{0.5\linewidth}{\linethickness}\end{center}

 - \textbf{E‑cadherin is downregulated by microenvironmental changes in
pancreatic cancer and induces EMT}

\emph{Oncology reports 2018 Sep;40(3):1641-1649}

PubMed: \url{https://www.ncbi.nlm.nih.gov/pubmed/?term=29956814}

The aim of the present study was to research the effect of
microenvironmental change on epithelial‑mesenchymal transition (EMT) in
pancreatic cancer cells and to determine the correlation between
E‑cadherin expression and the prognosis of pancreatic cancer patients.
We established hypoxic, serum‑deficient and TGF‑β‑induced
microenvironment models of pancreatic cancer cells and studied the
changes in the mRNA and protein expression of EMT‑related molecules,
E‑cadherin and vimentin, using western blot analysis and real‑time PCR.
Furthermore, immunohistochemistry was used to investigate E‑cadherin
expression in pancreatic cancer tissues, and survival analysis and COX
regression analysis were conducted. In pancreatic cancer cells under
hypoxic, serum‑starved and TGF‑β‑induced microenvironments, E‑cadherin
protein and mRNA levels were significantly decreased (P\textless{}0.05),
while vimentin protein and mRNA expression levels were significantly
increased (P\textless{}0.05). The results of immunohistochemistry showed
that the protein level of E‑cadherin in pancreatic cancer tissues was
positively correlated with overall survival (P\textless{}0.01). The
results of Cox regression analysis showed that E‑cadherin was an
independent prognostic factor in pancreatic cancer. In conclusion,
E‑cadherin expression was significantly decreased by microenvironment
changes, and this decrease induced EMT in pancreatic cancer cells.
E‑cadherin is an independent prognostic marker in pancreatic cancer
patients.

{}

{}

\begin{center}\rule{0.5\linewidth}{\linethickness}\end{center}

 - \textbf{From Friend to Enemy: Dissecting the Functional Alteration of
Immunoregulatory Components during Pancreatic Tumorigenesis}

\emph{International journal of molecular sciences 2018 Nov;19(11):}

PubMed: \url{https://www.ncbi.nlm.nih.gov/pubmed/?term=30428588}

Pancreatic ductal adenocarcinoma (PDAC) is a lethal disease with a
5-year survival rate of approximately 8\%. More than 80\% of patients
are diagnosed at an unresectable stage due to metastases or local
extension. Immune system reactivation in patients by immunotherapy may
eliminate tumor cells and is a new strategy for cancer treatment. The
anti-CTLA-4 antibody ipilimumab and anti-PD-1 antibodies pembrolizumab
and nivolumab have been approved for cancer therapy in different
countries. However, the results of immunotherapy on PDAC are
unsatisfactory. The low response rate may be due to poor immunogenicity
with low tumor mutational burden in pancreatic cancer cells and
desmoplasia that prevents the accumulation of immune cells in tumors.
The immunosuppressive tumor microenvironment in PDAC is important in
tumor progression and treatment resistance. Switching from an immune
tolerance to immune activation status is crucial to overcome the
inability of self-defense in cancer. Therefore, thoroughly elucidation
of the roles of various immune-related factors, tumor microenvironment,
and tumor cells in the development of PDAC may provide appropriate
direction to target inflammatory pathway activation as a new therapeutic
strategy for preventing and treating this cancer.

{}

{}

\begin{center}\rule{0.5\linewidth}{\linethickness}\end{center}

Back to top

\begin{center}\rule{0.5\linewidth}{\linethickness}\end{center}

\pagebreak

\hypertarget{preneoplastic-and-preinvasive-lesions-1}{%
\paragraph{Preneoplastic and Preinvasive
Lesions}\label{preneoplastic-and-preinvasive-lesions-1}}

Molecular Pathology Preneoplastic and Preinvasive Lesions, PanIN, IPMN,
MCN, ICPN

\begin{center}\rule{0.5\linewidth}{\linethickness}\end{center}

 - \textbf{Single-cell sequencing defines genetic heterogeneity in
pancreatic cancer precursor lesions}

\emph{The Journal of pathology 2018 Nov;():}

PubMed: \url{https://www.ncbi.nlm.nih.gov/pubmed/?term=30430578}

Intraductal papillary mucinous neoplasms (IPMNs) are precursors to
pancreatic cancer; however, little is known about genetic heterogeneity
in these lesions. The objective of this study was to characterize
genetic heterogeneity in IPMNs at the single-cell level. We isolated
single cells from fresh tissue from ten IPMNs, followed by whole genome
amplification and targeted next generation sequencing of pancreatic
driver genes. We then determined single-cell genotypes using a novel
multi-sample mutation calling algorithm. Our analyses revealed that
different mutations in the same driver gene frequently occur in the same
IPMN. Two IPMNs had multiple mutations in the initiating driver gene
KRAS that occurred in unique tumor clones, suggesting the possibility of
polyclonal origin or an unidentified initiating event preceding this
critical mutation. Multiple mutations in later-occurring driver genes
were also common and were frequently localized to unique tumor clones,
raising the possibility of convergent evolution of these genetic events
in pancreatic tumorigenesis. Single-cell sequencing of IPMNs
demonstrated genetic heterogeneity with respect to early and late
occurring driver gene mutations, suggesting a more complex pattern of
tumor evolution than previously appreciated in these lesions.

{}

{}

\begin{center}\rule{0.5\linewidth}{\linethickness}\end{center}

 - \textbf{Single Cell Transcriptomics of Pancreatic Cancer Precursors
Demonstrates Epithelial and Microenvironmental Heterogeneity as an Early
Event in Neoplastic Progression}

\emph{Clinical cancer research : an official journal of the American
Association for Cancer Research 2018 Nov;():}

PubMed: \url{https://www.ncbi.nlm.nih.gov/pubmed/?term=30385653}

PURPOSE: Early detection of pancreatic ductal adenocarcinoma (PDAC)
remains elusive. Precursor lesions of PDAC, specifically, intraductal
papillary mucinous neoplasms (IPMNs) represent a bona fide pathway to
invasive neoplasia, although the molecular correlates of progression
remain to be fully elucidated. Single cell transcriptomics provides a
unique avenue for dissecting both the epithelial and microenvironmental
heterogeneity that accompany multistep progression from non-invasive
IPMNs to PDAC. METHODS: Single cell RNA-sequencing was performed through
droplet-based sequencing on 5,403 cells from two low-grade IPMNs
(LGD-IPMN), two high-grade IPMNs (HGD-IPMN), and two PDACs (all
surgically resected). RESULTS: Analysis of single cell transcriptomes
revealed heterogeneous alterations within the epithelium and the tumor
microenvironment during the progression of non-invasive dysplasia to
invasive cancer. While HGD-IPMNs expressed many core-signaling pathways
described in PDAC, LGD-IPMNs harbored subsets of single cells with a
transcriptomic profile that overlapped with invasive cancer. Notably, a
pro-inflammatory immune component was readily seen in low-grade IPMNs,
comprised of cytotoxic T-cells, activated T-helper cells, and dendritic
cells, which was progressively depleted during neoplastic progression,
accompanied by infiltration of myeloid-derived suppressor cells.
Finally, stromal myofibroblast populations were heterogeneous, and
acquired a previously described tumor-promoting and immune-evading
phenotype during invasive carcinogenesis. CONCLUSIONS: This study
demonstrates the ability to perform high resolution profiling of the
transcriptomic changes that occur during multistep progression of cystic
PDAC precursors to cancer. Notably, single cell analysis provides an
unparalleled insight into both the epithelial and microenvironmental
heterogeneity that accompany early cancer pathogenesis, and might be a
useful substrate to identify targets for cancer interception.

{}

{}

\begin{center}\rule{0.5\linewidth}{\linethickness}\end{center}

Back to top

\begin{center}\rule{0.5\linewidth}{\linethickness}\end{center}

\pagebreak

\hypertarget{spn}{%
\paragraph{SPN}\label{spn}}

Solid Pseudopapillary Neoplasm

\begin{center}\rule{0.5\linewidth}{\linethickness}\end{center}

 - \textbf{Molecular alterations associated with metastases of solid
pseudopapillary neoplasms of the pancreas}

\emph{The Journal of pathology 2019 Jan;247(1):123-134}

PubMed: \url{https://www.ncbi.nlm.nih.gov/pubmed/?term=30306561}

Solid pseudopapillary neoplasms (SPN) of the pancreas are rare,
low-grade malignant neoplasms that metastasise to the liver or
peritoneum in 10-15\% of cases. They almost invariably present somatic
activating mutations of CTNNB1. No comprehensive molecular
characterisation of metastatic disease has been conducted to date. We
performed whole-exome sequencing and copy-number variation (CNV)
analysis of 10 primary SPN and comparative sequencing of five matched
primary/metastatic tumour specimens by high-coverage targeted sequencing
of 409 genes. In addition to CTNNB1-activating mutations, we found
inactivating mutations of epigenetic regulators (KDM6A, TET1, BAP1)
associated with metastatic disease. Most of these alterations were
shared between primary and metastatic lesions, suggesting that they
occurred before dissemination. Differently from mutations, the majority
of CNVs were not shared among lesions from the same patients and
affected genes involved in metabolic and pro-proliferative pathways.
Immunostaining of 27 SPNs showed that loss or reduction of KDM6A and
BAP1 expression was significantly enriched in metastatic SPNs.
Consistent with an increased transcriptional response to hypoxia in
pancreatic adenocarcinomas bearing KDM6A inactivation, we showed that
mutation or reduced KDM6A expression in SPNs is associated with
increased expression of the HIF1α-regulated protein GLUT1 at both
primary and metastatic sites. Our results suggest that BAP1 and KDM6A
function is a barrier to the development of metastasis in a subset of
SPNs, which might open novel avenues for the treatment of this disease.
© 2018 The Authors. The Journal of Pathology published by John Wiley \&
Sons Ltd on behalf of Pathological Society of Great Britain and Ireland.

{}

{}

\begin{center}\rule{0.5\linewidth}{\linethickness}\end{center}

Back to top

\begin{center}\rule{0.5\linewidth}{\linethickness}\end{center}

\pagebreak

\hypertarget{bile-ducts-1}{%
\subsubsection{Bile Ducts}\label{bile-ducts-1}}

\begin{center}\rule{0.5\linewidth}{\linethickness}\end{center}

 - \textbf{Recurrent Mutations in APC and CTNNB1 and Activated
Wnt/β-catenin Signaling in Intraductal Papillary Neoplasms of the Bile
Duct: A Whole Exome Sequencing Study}

\emph{The American journal of surgical pathology 2018
Dec;42(12):1674-1685}

PubMed: \url{https://www.ncbi.nlm.nih.gov/pubmed/?term=30212390}

This study aimed to elucidate the genetic landscape of biliary papillary
neoplasms. Of 28 cases examined, 7 underwent whole exome sequencing,
while the remaining 21 were used for validation studies with targeted
sequencing. In the whole exome sequencing study, 4/7 cases had mutations
in either APC or CTNNB1, both of which belong to the Wnt/β-catenin
pathway. Somatic mutations were also identified in genes involved in RAS
signaling (KRAS, BRAF), a cell cycle regulator (CDC27), histone
methyltransferase (KMT2C, KMT2D), and DNA mismatch repair (MSH3, MSH6,
PMS1). Combined with discovery and validation cohorts, mutations in APC
or CTNNB1 were observed in 6/28 subjects (21\%) and were mutually
exclusive. When the cases were classified into intraductal papillary
neoplasms of the bile duct (IPNBs, n=14) and papillary
cholangiocarcinomas (n=14) based on the recently proposed classification
criteria, mutations in APC and CTNNB1 appeared to be entirely restricted
to IPNBs with 6/14 cases (43\%) harboring mutations in either gene.
These genetic alterations were detected across the 3 nonintestinal
histologic types. In immunohistochemistry, the aberrant cytoplasmic
and/or nuclear expression of β-catenin was found in not only 5/6 IPNBs
with APC or CTNNB1 mutations, but also 6/8 cases with wild-type APC and
CTNNB1 (total 79\%). In addition, APC and CTNNB1 alterations were
exceptional in nonpapillary cholangiocarcinomas (n=29) with a single
case harboring CTNNB1 mutation (3\%). This study demonstrated recurrent
mutations in APC and CTNNB1 in nonintestinal-type IPNBs, suggesting that
activation of the Wnt/β-catenin signaling pathway is relevant to the
development and progression of IPNBs.

{}

{}

\begin{center}\rule{0.5\linewidth}{\linethickness}\end{center}

 - \textbf{Comprehensive analysis of long noncoding RNA-associated
competing endogenous RNA network in cholangiocarcinoma}

\emph{Biochemical and biophysical research communications 2018
Dec;506(4):1004-1012}

PubMed: \url{https://www.ncbi.nlm.nih.gov/pubmed/?term=30404735}

BACKGROUND: Long non-coding RNAs (lncRNAs) can interact with microRNAs
(miRNAs) as a competing endogenous RNA (ceRNA) to regulate the
expression of target genes, which can largely influence on tumorigenesis
and tumor progression. However, the role of lncRNA-mediated ceRNAs in
cholangiocarcinoma (CCA) remains unknown. This study aimed to develop
novel lncRNAs as well as their action mechanisms in CCA. METHODS: The
expression profiles of lncRNAs, miRNAs, and mRNAs of 36 CCA tissues and
9 non-tumor bile duct tissues were obtained from The Cancer Genome Atlas
(TCGA) database. The differentially expressed RNAs werre identified
using the DESeq package in R. The ceRNA network was constructed in CCA
based on bioinformatics generated from miRcode, miRTarBase, miRDB, and
TargetScan. Gene Ontology (GO) and Kyoto Encyclopedia of Genes and
Genomes (KEGG) analyses were performed using ``DAVID 6.8'' and R
packages ``Clusterprofile''. Survival analysis was performed based on
Kaplan-Meier curve analysis. RESULTS: We identified a total of 1411
differentially expressed lncRNAs, 3494 mRNAs, and 64 miRNAs between CCA
and matched normal tissues. By combining the data predicted by databases
with intersection RNAs, a lncRNA-miRNA-mRNA ceRNA network consisting of
116 lncRNAs, 14 miRNAs and 59 mRNAs was established. According to the
survival analysis, we detected 11 DElncRNA to have a significant impact
on the overall survival in patients with CCA (P\,\textless{}\,0.05).
CONCLUSIONS: Our study identified novel lncRNAs associated with CCA
progression and prognosis and provided data to further understand
lncRNA-mediated ceRNA regulatory mechanisms in the pathogenesis of CCA.

{}

{}

\begin{center}\rule{0.5\linewidth}{\linethickness}\end{center}

 - \textbf{Over-expression of TNNI3K is associated with early-stage
carcinogenesis of cholangiocarcinoma}

\emph{Molecular carcinogenesis 2018 Oct;():}

PubMed: \url{https://www.ncbi.nlm.nih.gov/pubmed/?term=30334579}

Cholangiocarcinoma (CCA) is a devastating disease with very poor
prognosis due to late diagnosis and resistance to traditional
chemotherapies and radiotherapies. Herein, thioacetamide (TAA)-induced
rat CCA model and CGCCA cell line were used; we aim to study the
cytogenetic features during tumoral development of CCA and uncover the
mystery regarding carcinogenesis of CCA. The Array comparative genomic
hybridization analysis, in silico method, gene knockdown, Western blot,
cell count proliferation assay, clonogenecity assay, and IHC staining
were applied in this study. Array comparative genomic hybridization
analysis was performed on all different TAA-induced phases of rat
tissues to reveal the certain pattern, +2q45, +Xq22, -12p12, have been
identified for the tumor early stage, where involve the gene TNNI3K. In
addition, 16 genes and 3 loci were associated with rapid tumor
progression; JAK-STAT signaling pathway was highly correlated to late
stage of CCA. In silico database was used to observe TNNI3K was highly
express at tumor part compared with normal adjacent tissue in CCA
patients from TCGA dataset. Furthermore, the growth of TNNI3K-knockdown
SNU308 and HuCCT1 cells decreased when compared with cells transfected
with an empty vector cell demonstrated by proliferation and
colonogenecity assay. Besides, over expression of TNNI3K was especially
confirmed on human CCA tumors and compared with the intrahepatic duct
stone bile duct tissues and normal bile duct tissues
(P \textless{} 0.001). Our findings might uncover the mystery regarding
carcinogenesis of CCA, and provide the potential genetic mechanism to
the clinicians some ideas for the patients' treatment.

{}

{}

\begin{center}\rule{0.5\linewidth}{\linethickness}\end{center}

Back to top

\begin{center}\rule{0.5\linewidth}{\linethickness}\end{center}

\pagebreak

\hypertarget{gallbladder-2}{%
\subsubsection{Gallbladder}\label{gallbladder-2}}

\begin{center}\rule{0.5\linewidth}{\linethickness}\end{center}

 - \textbf{Regional differences in gallbladder cancer pathogenesis:
Insights from a multi-institutional comparison of tumor mutations}

\emph{Cancer 2018 Nov;():}

PubMed: \url{https://www.ncbi.nlm.nih.gov/pubmed/?term=30427539}

BACKGROUND: Although rare in the United States, gallbladder cancer
(GBCA) is a common cause of cancer death in some parts of the world. To
investigate regional differences in pathogenesis and outcomes for GBCA,
tumor mutations were analyzed from a sampling of specimens. METHODS:
Primary tumors from patients with GBCA who were treated in Chile, Japan,
and the United States between 1999 and 2016 underwent targeted
sequencing of known cancer-associated genes. Fisher exact and
Kruskal-Wallis tests assessed differences in clinicopathologic and
genetic factors. Kaplan-Meier methods evaluated differences in overall
survival from the time of surgery between mutations. RESULTS: A total of
81 patients were included. Japanese patients (11 patients) were older
(median age, 72 years {[}range, 54-81 years{]}) compared with patients
from Chile (21 patients; median age, 59 years {[}range, 32-73 years{]})
and the United States (49 patients; median age, 66 years {[}range, 46-87
years{]}) (P~=~.002) and had more well-differentiated tumors (46\% vs
0\% for Chile/United States; P~\textless{}~.001) and fewer
gallstone-associated cancers (36\% vs 67\% for Chile and 69\% for the
United States; P~=~.13). Japanese patients had a median mutation burden
of 6 (range, 1-23) compared with Chile (median mutation burden, 7
{[}range, 3-20{]}) and the United States (median mutation burden, 4
{[}range, 0-27{]}) (P~=~.006). Tumors from Japanese patients lacked
AT-rich interaction domain 1A (ARID1A) and
phosphatidylinositol-4,5-bisphosphate 3-kinase catalytic subunit alpha
(PIK3CA) mutations, whereas Chilean tumors lacked Erb-B2 receptor
tyrosine kinase 3 (ERBB3) and AT-rich interaction domain 2 (ARID2)
mutations. SMAD family member 4 (SMAD4) was found to be mutated
similarly across centers (38\% in Chile, 36\% in Japan, and 27\% in the
United States; P~=~.68) and was univariately associated with worse
overall survival (median, 10 months vs 25 months; P~=~.039). At least
one potentially actionable gene was found to be altered in 80\% of
tumors. CONCLUSIONS: Differences in clinicopathologic variables suggest
the possibility of distinct GBCA pathogenesis in Japanese patients,
which may be supported by differences in mutation pattern. Among all
centers, SMAD4 mutations were detected in approximately one-third of
patients and may represent a converging factor associated with worse
survival. The majority of patients carried mutations in actionable gene
targets, which may inform the design of future trials.

{}

{}

\begin{center}\rule{0.5\linewidth}{\linethickness}\end{center}

Back to top

\begin{center}\rule{0.5\linewidth}{\linethickness}\end{center}

\pagebreak

\hypertarget{ampulla-of-vater-1}{%
\subsubsection{Ampulla of Vater}\label{ampulla-of-vater-1}}

\begin{center}\rule{0.5\linewidth}{\linethickness}\end{center}

 - \textbf{Ampullary carcinoma-A genetic perspective}

\emph{Mutation research 2018 03;776():10-22}

PubMed: \url{https://www.ncbi.nlm.nih.gov/pubmed/?term=29807574}

Ampulla of vater carcinoma (AVC) is a rare gastrointestinal tumour that
is associated with a high mortality rate and it's often diagnosed at
later stages due to lack of clinical symptoms. Early diagnosis of this
condition is essential to effectively treat patients for better
prognosis. A significant amount of advancement has been made in
understanding the molecular nature of cancer in the past decade. A
substantial number of mutations and alterations have been detected in
various tumors. Despite the occurrence of AVC across the globe, the
number of studies conducted on this tumor type remains low; this is
largely due to its rare occurrence. Moreover, AVC tissues are complex
and contain mutations in oncogenes, tumour suppressors, apoptotic
proteins, cell proliferation proteins, cell signaling proteins,
transcription factors, chromosomal abnormalities and cellular adhesion
proteins. The frequently mutated genes included KRAS, TP53 and SMAD4 and
are associated with prognosis. Several molecules of the PI3K, Wnt
signaling, TGF-beta pathway and cell cycle have also been altered in
AVCs. This review comprises of all the genetic mutations, associated
pathways and related prognosis that are involved in AVCs from the year
1989 to 2017. This report can be used as a stepping-stone to establish
biomarkers for early diagnosis of AVC and to discover molecular targets
for drug therapy.

{}

{}

\begin{center}\rule{0.5\linewidth}{\linethickness}\end{center}

 - \textbf{Ampulla of Vater carcinoma: Molecular landscape and clinical
implications}

\emph{World journal of gastrointestinal oncology 2018
Nov;10(11):370-380}

PubMed: \url{https://www.ncbi.nlm.nih.gov/pubmed/?term=30487949}

Ampulla of Vater is a peculiar anatomical structure, characterized by
the crossroad of three distinct epithelia: Intestinal, ductal pancreatic
and biliary. Adenocarcinomas arising in this area represent an
opportunity to understand the comparative biology of all periampullary
malignancies. These neoplasms can exhibit intestinal, pancreaticobiliary
or mixed features, whereas the subclassification based on morphology and
immunohistochemical features failed in demonstrating a robust prognostic
reliability. In the last few years, the molecular landscape of this
tumor entity has been uncovered, identifying alterations that may serve
as prognostic and predictive biomarkers. In this review, the
histological and genetic characteristics of ampullary carcinomas are
discussed, taking into account the main clinical and therapeutic
implications related to this tumor type as well.

{}

{}

\begin{center}\rule{0.5\linewidth}{\linethickness}\end{center}

 - \textbf{MiR-21 up-regulation in ampullary adenocarcinoma and its
pre-invasive lesions}

\emph{Pathology, research and practice 2018 Jun;214(6):835-839}

PubMed: \url{https://www.ncbi.nlm.nih.gov/pubmed/?term=29731265}

Poor information is available on the molecular landscape characterizing
the carcinogenetic process leading to ampullary carcinoma. MiR-21 is one
of the most frequently up-regulated miRNAs in pancreatic adenocarcinoma,
a tumor sharing similar molecular features with ampullary
adenocarcinomas (AVCs), above all with the pancreatic-biliary type. We
profiled, by in situ hybridization (ISH), miR-21 expression in a series
of 26 AVCs, 50 ampullary dysplastic lesions (35 low-grade {[}LG-IEN{]}
and 15 high-grade {[}HG-IEN{]}) and 10 normal duodenal mucosa samples.
The same series was investigated by immunohistochemistry for β-catenin,
p53 and HER2 expression. HER2 gene amplification was evaluated by
chromogenic in situ hybridization. To validate miR-21 ISH results we
performed miR-21 qRT-PCR analysis in a series of 10 AVCs and their
matched normal samples. All the normal control samples showed a negative
or faint miR-21 expression, whereas a significant miR-21 up-regulation
was observed during the carcinogenetic cascade (p\,\textless{}\,0.001),
with 21/26 (80.8\%) of cancer samples showing a miR-21 overexpression.
In comparison to control samples, a significant overexpression was found
in samples of LG-IEN (p\,=\,.0003), HG-IEN (p\,=\,.0001), and AVCs
(p\,\textless{}\,0.0001). No significant difference in miR-21
overexpression was observed between LG-IEN, HG-IEN and AVCs. By qRT-PCR
analysis, AVCs showed a 1.7-fold increase over the controls
(p\,=\,.003). P53 was frequently dysregulated in both dysplastic and
carcinoma samples (44 out of 76; 57.9\%). A 20\% (10/50) of dysplastic
lesions and 11\% (3/26) of carcinomas were characterized by a nuclear
localization of β-catenin. Only 2 AVCs (7.7\%; both intestinal-type)
showed a HER2 overexpression (both 2+), which corresponded to a HER2
gene amplification at CISH analysis. This is the first study
demonstrating a miRNA dysregulation in the whole spectrum of ampullary
carcinogenesis. MiR-21 overexpression is an early molecular event during
ampullary carcinogenesis and its levels increase with the neoplastic
progression.

{}

{}

\begin{center}\rule{0.5\linewidth}{\linethickness}\end{center}

 - \textbf{Adenosquamous carcinoma of the papilla of Vater: A phenotypic
heterogeneity characterized by a common molecular landscape}

\emph{Pathology international 2018 Dec;68(12):715-716}

PubMed: \url{https://www.ncbi.nlm.nih.gov/pubmed/?term=30417956}

{}

{}

\begin{center}\rule{0.5\linewidth}{\linethickness}\end{center}

Back to top

\begin{center}\rule{0.5\linewidth}{\linethickness}\end{center}

\pagebreak

\hypertarget{neuroendocrine-2}{%
\subsubsection{Neuroendocrine}\label{neuroendocrine-2}}

\begin{center}\rule{0.5\linewidth}{\linethickness}\end{center}

 - \textbf{Clinical and in vitro studies of the correlation between MGMT
and the effect of streptozocin in pancreatic NET}

\emph{Cancer chemotherapy and pharmacology 2018 Oct;():}

PubMed: \url{https://www.ncbi.nlm.nih.gov/pubmed/?term=30310970}

PURPOSE: This study aimed to determine the correlation between DNA
repair enzyme O6-methylguanine DNA methyltransferase (MGMT) status and
the response to streptozocin in advanced well-differentiated pancreatic
neuroendocrine tumors (WD panNETs). METHODS: To test the hypothesis that
MGMT deficiency was required for an alkylating drug response, we
retrospectively reviewed the response of 13 patients with WD panNETs to
alkylating agents in relation to MGMT status. We also studied MGMT
expression in streptozocin resistance using panNET cell lines. RESULTS:
The cohort included 54\% of patients with and 46\% without MGMT
expression. Among these, 83.3\% (5/6) of MGMT-negative cases showed a
partial response to streptozocin. In contrast, only 14.2\% (1/7) of
MGMT-positive cases showed a partial response (P = 0.013). Induced
expression of MGMT in BON1 cells (a panNET cell line with undetectable
endogenous MGMT) produced streptozocin resistance. Knockdown of MGMT in
QGP1 cells, which express MGMT endogenously, did not alter the response
to streptozocin. CONCLUSIONS: We observed a relationship between MGMT
status and streptozocin response in both patients and cell culture.
Despite limited cases examined, high concordance of negative expression
of MGMT and response to streptozocin treatment suggest that MGMT
expression can be a potential biomarker for this treatment.

{}

{}

\begin{center}\rule{0.5\linewidth}{\linethickness}\end{center}

 - \textbf{Evaluating gastroenteropancreatic neuroendocrine tumors
through microRNA sequencing}

\emph{Endocrine-related cancer 2019 Jan;26(1):47-57}

PubMed: \url{https://www.ncbi.nlm.nih.gov/pubmed/?term=30021866}

Gastroenteropancreatic neuroendocrine tumors (GEP-NETs) can be
challenging to evaluate histologically. MicroRNAs (miRNAs) are small RNA
molecules that often are excellent biomarkers due to their abundance,
cell-type and disease stage specificity and stability. To evaluate
miRNAs as adjunct tissue markers for classifying and grading
well-differentiated GEP-NETs, we generated and compared miRNA expression
profiles from four pathological types of GEP-NETs. Using quantitative
barcoded small RNA sequencing and state-of-the-art sequence annotation,
we generated comprehensive miRNA expression profiles from archived
pancreatic, ileal, appendiceal and rectal NETs. Following data
preprocessing, we randomly assigned sample profiles to discovery (80\%)
and validation (20\%) sets prior to data mining using machine-learning
techniques. High expression analyses indicated that miR-375 was the most
abundant individual miRNA and miRNA cistron in all samples. Leveraging
prior knowledge that GEP-NET behavior is influenced by embryonic
derivation, we developed a dual-layer hierarchical classifier for
differentiating GEP-NET types. In the first layer, our classifier
discriminated midgut (ileum, appendix) from non-midgut (rectum,
pancreas) NETs based on miR-615 and -92b expression. In the second
layer, our classifier discriminated ileal from appendiceal NETs based on
miR-125b, -192 and -149 expression, and rectal from pancreatic NETs
based on miR-429 and -487b expression. Our classifier achieved overall
accuracies of 98.5\% and 94.4\% in discovery and validation sets,
respectively. We also found provisional evidence that low- and
intermediate-grade pancreatic NETs can be discriminated based on miR-328
expression. GEP-NETs can be reliably classified and potentially graded
using a limited panel of miRNA markers, complementing morphological and
immunohistochemistry-based approaches to histologic evaluation.

{}

{}

\begin{center}\rule{0.5\linewidth}{\linethickness}\end{center}

 - \textbf{Neuroendocrine neoplasia goes molecular - time for a change}

\emph{Nature reviews. Clinical oncology 2018 Nov;():}

PubMed: \url{https://www.ncbi.nlm.nih.gov/pubmed/?term=30390038}

{}

{}

\begin{center}\rule{0.5\linewidth}{\linethickness}\end{center}

 - \textbf{Blood mRNA Measurement (NETest) for Neuroendocrine Tumors
diagnosis of Image-negative liver metastatic disease}

\emph{The Journal of clinical endocrinology and metabolism 2018 Oct;():}

PubMed: \url{https://www.ncbi.nlm.nih.gov/pubmed/?term=30358858}

Context: Early cancer detection is critical to optimize treatment. This
is particularly problematic in neuroendocrine tumors (NETs) which
exhibit a ∼5-year diagnostic delay due to covert symptomatology,
limitations in imaging and circulating biomarkers. Despite development
of continuous monitoring strategies utilizing advanced modalities
(CT/MRI or 68Gallium-PET/CT), or a repertoire of monoanalyte biomarkers
(e.g.~chromogranin A {[}CgA{]}, pancreastatin, serotonin), detection of
minimal residual disease or micro-recurrence, remains elusive. Emerging
molecular liquid biopsies (e.g.~NETest) provide a significant improved
threshold for disease detection. Case Description: We describe the
utility of a blood-based multigene PCR neuroendocrine measurement
(NETest), representative of core molecular drivers of neuroendocrine
tumorigenesis, to detect hepatic micrometastases in a patient with
negative blood biomarkers, and negative anatomical/functional imaging. A
52-year old female who had undergone margin-negative resection for a NET
of the ileocecal valve, 8 months later developed persistently elevated
NETest levels. CT/MRI/68Gallium-PET and biomarkers remained negative.
Blood multi-gene analysis identified disease and peptide receptor
radionuclide therapy (PRRT) was undertaken. Over 9 months, NETest levels
increased (conventional biomarkers/imaging remained normal). Liver
biopsy was undertaken, and foci of 3mm NET in segment VI were
histologically documented. 3.3 years after PRRT, the disease remains as
a microscopic burden and stable, biomarker/68Gallium PET/MRI occult
despite elevated blood levels of NET genes. Conclusions: Blood
measurement of neuroendocrine tumor transcripts can identify image- and
CgA-negative disease. A NET liquid biopsy strategy has clinical utility
for the early identification of residual or metastatic disease and
optimizes the consideration of adjuvant therapeutic intervention.

{}

{}

\begin{center}\rule{0.5\linewidth}{\linethickness}\end{center}

 - \textbf{Neuropilin-1 (NRP-1) upregulated by IL-6/STAT3 signaling
contributes to invasion in pancreatic neuroendocrine neoplasms}

\emph{Human pathology 2018 Nov;81():192-200}

PubMed: \url{https://www.ncbi.nlm.nih.gov/pubmed/?term=30420046}

Although the upregulation of Neuropilin-1 (NRP-1) is associated with
many solid tumors, its role in pancreatic neuroendocrine neoplasms
(pNEN) has not been well elucidated. The aim of this study was to
investigate the role of NRP-1 in improving treatment and determining the
prognosis of pNEN. In this study, the expression of NRP-1 in pNEN tissue
samples and pNEN cell line BON1 was analyzed by Western blot, polymerase
chain reaction (PCR) and immunocytochemistry upon exposure to
interleukin-6 (IL-6). Additionally, pNEN cell line BON1 was transfected
with small interfering RNAs against NRP-1 or signal transducer and
activator of transcription 3 (STAT3) and assessed by in vitro invasion
assays. The expression of NRP-1 in pNEN tissues was markedly increased
compared with adjacent normal pancreatic tissues. High NRP-1 expression
was strongly correlated with tumor grades (P = .026), lymph node
metastasis (P = .025), and tumor-node-metastasis stages (P = .012).
Furthermore, NRP-1 downregulation notably inhibited the metastatic
capacity of pNEN cells, and STAT3 knockdown was found to downregulate
the expression of NRP-1. BON1 cells upregulated NRP-1 expression upon
stimulation with IL-6. This was accompanied by
activation/phosphorylation of the AKT and STAT3 signaling pathways.
Western blot of extracts of human pNENs confirmed increased NRP-1
expression, as well as AKT/STAT3 phosphorylation in tissue of pNENs with
elevated expression levels of IL-6. In conclusion, our findings suggest
that NRP-1 is upregulated in pNEN and is correlated with the metastatic
capacity of pNEN cells, potentially via interaction with the IL-6/STAT3
signaling pathway.

{}

{}

\begin{center}\rule{0.5\linewidth}{\linethickness}\end{center}

 - \textbf{Cell-Free DNA From Metastatic Pancreatic Neuroendocrine Tumor
Patients Contains Tumor-Specific Mutations and Copy Number Variations}

\emph{Frontiers in oncology 2018 11;8():467}

PubMed: \url{https://www.ncbi.nlm.nih.gov/pubmed/?term=30443491}

Background: Detection of tumor-specific alterations in cell-free DNA
(cfDNA) has proven valuable as a liquid biopsy for several types of
cancer. So far, use of cfDNA remains unexplored for pancreatic
neuroendocrine tumor (PNET) patients. Methods: From 10 PNET patients,
fresh frozen tumor tissue, buffy coat and plasma samples were collected.
Whole-exome sequencing of primary tumor and germline DNA was performed
to identify tumor-specific variants and copy number variations (CNVs).
Subsequently, tumor-specific variants were quantified in plasma cfDNA
with droplet digital PCR. In addition, CNV analysis of cfDNA was
performed using shallow whole-genome sequencing. Results: Tumor-specific
variants were detected in perioperative plasma samples of two PNET
patients, at variant allele fractions (VAFs) of respectively 19 and
21\%. Both patients had metastatic disease at time of surgery, while the
other patients presented with localized disease. In the metastatic
patients, CNV profiles of tumor tissue and cfDNA were significantly
correlated. A follow-up plasma sample of a metastatic patient
demonstrated an increased VAF (57\%) and an increased chromosomal
instability, in parallel with an increase in tumor burden. Conclusions:
We are the first to report the presence of tumor-specific genetic
alterations in cfDNA of metastatic PNET patients and their evolution
during disease progression. Additionally, CNV analysis in cfDNA shows
potential as a liquid biopsy.

{}

{}

\begin{center}\rule{0.5\linewidth}{\linethickness}\end{center}

 - \textbf{Genetic Analysis of Small Well-differentiated Pancreatic
Neuroendocrine Tumors Identifies Subgroups With Differing Risks of Liver
Metastases}

\emph{Annals of surgery 2018 Oct;():}

PubMed: \url{https://www.ncbi.nlm.nih.gov/pubmed/?term=30339629}

OBJECTIVE:: The aim of this study was to investigate the key molecular
alterations in small primary pancreatic neuroendocrine tumors (PanNETs)
associated with the development of liver metastases. BACKGROUND:
Well-differentiated PanNETs with small size are typically indolent;
however, a limited subset metastasize to the liver. METHODS: A total of
87 small primary PanNETs (\textless{}3 cm), including 32 metastatic
cases and 55 nonmetastatic cases after a 5-year follow-up, were
immunolabeled for DAXX/ATRX and analyzed for alternative lengthening of
telomeres (ALT) by Fluorescence In Situ Hybridization. A subset of these
cases, 24 that metastasized and 24 that did not metastasize, were
assessed by targeted next-generation sequencing and whole-genome copy
number variation. RESULTS: In the entire cohort, high Ki-67 (OR 1.369;
95\% CI 1.121-1.673; P = 0.002), N-stage (OR 4.568; 95\% CI
1.458-14.312; P = 0.009), and ALT-positivity (OR 3.486; 95\% CI
1.093-11.115; P = 0.035) were independently associated with liver
metastases. In the subset assessed by next-generation sequencing and
copy number variation analysis, 3 molecular subtypes with differing
risks of liver metastases were identified. Group 1 (n = 15; 73\%
metastasized) was characterized by recurrent chromosomal gains, CN-LOH,
DAXX mutations, and ALT-positivity. Group 2 (n = 19; 42\% metastasized,
including 5 G1 tumors) was characterized by limited copy number
alterations and mutations. Group 3 (n = 14; 35\% metastasized) were
defined by chromosome 11 loss. CONCLUSIONS: We identified genomic
patterns of small PanNETs associated with a different risk for liver
metastases. Molecular alterations, such as DAXX mutations, chromosomal
gains, and ALT, are associated with an increased risk of metastasis in
small PanNETs. Therefore, targeted sequencing and/or ALT analysis may
help in the clinical decisions for these small PanNETs.

{}

{}

\begin{center}\rule{0.5\linewidth}{\linethickness}\end{center}

 - \textbf{DAXX Mutation Status of Embolization-Treated~Neuroendocrine
Tumors Predicts Shorter~Time to Hepatic Progression}

\emph{Journal of vascular and interventional radiology : JVIR 2018
Nov;29(11):1519-1526}

PubMed: \url{https://www.ncbi.nlm.nih.gov/pubmed/?term=30342802}

PURPOSE: To identify common gene mutations in patients with
neuroendocrine liver metastases (NLM) undergoing transarterial
embolization (TAE) and establish relationship between these mutations
and response to TAE. MATERIALS AND METHODS: Patients (n~= 51; mean age
61 y; 29 men, 22 women) with NLMs who underwent TAE and had available
mutation analysis were identified. Mutation status and clinical
variables were recorded and evaluated in relation to hepatic
progression-free survival (HPFS) (Cox proportional hazards) and time to
hepatic progression (TTHP) (competing risk proportional hazards).
Subgroup analysis of patients with pancreatic NLM was performed using
Fisher exact test to identify correlation between mutation and event
(hepatic progression or death) by 6 months. Changes in mutation status
over time and across specimens in a subset of patients were recorded.
RESULTS: Technical success of TAE was 100\%. Common mutations identified
were MEN1 (16/51; 31\%) and DAXX (13/51; 25\%). Median overall survival
was 48.7 months. DAXX mutation status (hazard ratio~= 6.21; 95\%
confidence interval {[}CI{]}, 2.67-14.48; P~\textless{}~.001) and tumor
grade (hazard ratio~= 3.05; 95\% CI, 1.80-5.17; P \textless{} .001) were
associated with shorter HPFS and TTHP on univariate and multivariate
analysis. Median HPFS was 3.6 months (95\% CI, 1.7-5.3) for patients
with DAXX mutation compared with 8.9 months (95\% CI, 6.6-11.4) for
patients with DAXX wild-type status. In patients with pancreatic NLMs,
DAXX mutation status was associated with hepatic progression or death by
6 months (P~= .024). DAXX mutation status was concordant between primary
and metastatic sites. CONCLUSIONS: DAXX mutation is common in patients
with pancreatic NLMs. DAXX mutation status is associated with shorter
HPFS and TTHP after TAE.

{}

{}

\begin{center}\rule{0.5\linewidth}{\linethickness}\end{center}

Back to top

\begin{center}\rule{0.5\linewidth}{\linethickness}\end{center}

\pagebreak

\hypertarget{experimental-others}{%
\subsection{Experimental / Others}\label{experimental-others}}

\begin{center}\rule{0.5\linewidth}{\linethickness}\end{center}

\hypertarget{experimental-studies}{%
\subsubsection{Experimental Studies}\label{experimental-studies}}

\begin{center}\rule{0.5\linewidth}{\linethickness}\end{center}

 - \textbf{Direct therapeutic targeting of immune checkpoint PD-1 in
pancreatic cancer}

\emph{British journal of cancer 2018 Oct;():}

PubMed: \url{https://www.ncbi.nlm.nih.gov/pubmed/?term=30377341}

BACKGROUND: Pancreatic cancer (PC) hijacks innate cellular processes to
promote cancer growth. We hypothesized that PC exploits PD-1/PD-L1 not
only to avoid immune responses, but to directly enhance growth. We also
hypothesized that immune checkpoint inhibitors (ICIs) have direct
cytotoxicity in PC. We sought to elucidate therapeutic targeting of
PD-1/PD-L1. METHODS: PD-1 was assessed in PC cells, patient-derived
organoids (PDOs), and clinical tissues. Then, PC cells were exposed to
PD-L1 to evaluate proliferation. To test PD-1/PD-L1 signaling, cells
were exposed to PD-L1 and MAPK was examined. Radio-immunoconjugates with
anti-PD-1 drugs were developed to test uptake in patient-derived tumor
xenografts (PDTXs). Next, PD-1 function was assessed by xenografting
PD-1-knockdown cells. Finally, PC models were exposed to ICIs. RESULTS:
PD-1 expression was demonstrated in PCs. PD-L1 exposure increased
proliferation and activated MAPK. Imaging PDTXs revealed uptake of
radio-immunoconjugates. PD-1 knockdown in vivo revealed 67\% smaller
volumes than controls. Finally, ICI treatment of both PDOs/PDTXs
demonstrated cytotoxicity and anti-MEK1/2 combined with anti-PD-1 drugs
produced highest cytotoxicity in PDOs/PDTXs. CONCLUSIONS: Our data
reveal PCs innately express PD-1 and activate druggable oncogenic
pathways supporting PDAC growth. Strategies directly targeting PC with
novel ICI regimens may work with adaptive immune responses for optimal
cytotoxicity.

{}

{}

\begin{center}\rule{0.5\linewidth}{\linethickness}\end{center}

 - \textbf{Silencing of TRPM8 inhibits aggressive tumor phenotypes and
enhances gemcitabine sensitivity in pancreatic cancer}

\emph{Pancreatology : official journal of the International Association
of Pancreatology (IAP) \ldots{} {[}et al.{]} 2018 Dec;18(8):935-944}

PubMed: \url{https://www.ncbi.nlm.nih.gov/pubmed/?term=30316690}

The transient receptor potential TRPM8 ion channel is required for
cellular proliferation in pancreatic epithelia and adenocarcinoma. To
elucidate the mechanism that mediates the function of TRPM8, we examined
its role in the proliferation and invasion of pancreatic cancer (PC)
cells. TRPM8 expression increased in both the PC tissues and cell lines;
a high TRPM8 expression was correlated with poorer prognosis in patients
with PC. In PC cell lines, PACN-1 and BxPC-3, Ca2+ influxes could be
evoked by TRPM8; the sensitivity of PC cells to gemcitabine was
increased, while the proliferation and invasion of PC cells were
suppressed after RNA interference-mediated silencing of TRPM8. The
mechanism of TRPM8 in gemcitabine-based chemotherapy was then
investigated. The expression and activity of multidrug
resistance-associated proteins, P-gp, MRP-2, LRP, was significantly
reduced in response to TRPM8 silence. Moreover, TRPM8 knockdown
significantly increased hENT1 protein levels and the ratio of Bax/Bcl-2
while decreased the protein levels of RRM1. Thus, TRPM8 is required for
PC cell proliferation and invasion and was closely related to the
gemcitabine sensitivity of PC. The modulation of TRPM8 expression may
help improve treatment response of PC by combining with traditional
chemotherapy.

{}

{}

\begin{center}\rule{0.5\linewidth}{\linethickness}\end{center}

 - \textbf{Abrogation of glutathione peroxidase-1 drives EMT and
chemoresistance in pancreatic cancer by activating ROS-mediated
Akt/GSK3β/Snail signaling}

\emph{Oncogene 2018 Nov;37(44):5843-5857}

PubMed: \url{https://www.ncbi.nlm.nih.gov/pubmed/?term=29980787}

The devastating prognosis of pancreatic ductal adenocarcinoma (PDAC) is
partially attributed to chemotherapy resistance. Glutathione
peroxidase-1 (GPx1) plays various roles in the development and
progression of multiple tumors, with the exception of pancreatic cancer.
Here, we tentatively explored the role of GPx1 in the malignant
biological behavior and gemcitabine (GEM) resistance of PDAC. GPx1
levels were detected using tissue microarrays and were negatively
correlated with the overall survival of patients with PDAC. GPx1
silencing induced a mesenchymal transition phenotype and increased GEM
resistance in vitro and in vivo. Additionally, the activation of
reactive oxygen species (ROS)-mediated Akt/glycogen synthase kinase 3β
(GSK3β)/Snail signaling was involved in this process, as determined by
RNA sequencing. Moreover, low GPx1 expression correlated with a worse
survival rate in patients with PDAC who received GEM adjuvant
chemotherapy, whereas this correlation was not detected in patients
receiving fluoropyrimidine. Based on our results, GPx1 inhibits the
epithelial-mesenchymal transition (EMT) and chemoresistance by
regulating the Akt/GSK3β/Snail signaling axis in PDAC. Furthermore, GPx1
may be a potential predictive biomarker in GEM-treated PDAC patients.

{}

{}

\begin{center}\rule{0.5\linewidth}{\linethickness}\end{center}

 - \textbf{Targeting Purinergic Receptor P2Y2 prevents the growth of
pancreatic ductal adenocarcinoma by inhibiting cancer cell glycolysis}

\emph{Clinical cancer research : an official journal of the American
Association for Cancer Research 2018 Nov;():}

PubMed: \url{https://www.ncbi.nlm.nih.gov/pubmed/?term=30420446}

PURPOSE: Extensive research has reported that the tumor microenvironment
components play crucial roles in tumor progression. Thus, blocking the
supports of tumor microenvironment is a promising approach to prevent
cancer progression. We aimed to determine whether blocking extracellular
ATP-P2RY2 axis could be a potential therapeutic approach for PDAC
treatment. EXPERIMENTAL DESIGN: Expression of P2RY2 was determined in
264 human PDAC samples, and correlated to patient survival. P2RY2 was
inhibited in human PDAC cell lines by antagonist and shRNA,
respectively, and cell viability, clonogenicity and glycolysis were
determined. RNA sequencing of PDAC cell line was applied to reveal
underlying molecular mechanisms. Multiple PDAC mouse models were used to
assess the effects of the P2RY2 inhibition on PDAC progression. RESULTS:
P2RY2 was upregulated and associated with poor prognosis in PDAC.
Activated P2RY2 by increased extracellular ATP in tumor microenvironment
promoted PDAC growth and glycolysis. Further studies showed that the
agonist-activated P2RY2 triggered PI3K/AKT-mTOR signaling by crosstalk
with PDGFR mediated by Yes1, resulting in elevating expression of c-Myc
and HIF1a, which subsequently enhanced cancer cell glycolysis. Genetic
and pharmacological inhibition of P2RY2 impaired tumor cell growth in
subcutaneous and orthotopic xenograft model, as well as delayed tumor
progression in inflammation-driven PDAC model. Additionally, synergy was
observed when AR-C118925XX, the selective antagonist of P2RY2 receptor,
and gemcitabine were combined, resulting in prolonged survival of
xenografted PDAC mice. CONCLUSIONS: These findings revealed the roles of
the P2RY2 in PDAC metabolic reprogramming, suggesting that P2RY2 might
be a potential metabolic therapeutic target for PDAC.

{}

{}

\begin{center}\rule{0.5\linewidth}{\linethickness}\end{center}

 - \textbf{Suppression of stromal-derived Dickkopf-3 (DKK3) inhibits
tumor progression and prolongs survival in pancreatic ductal
adenocarcinoma}

\emph{Science translational medicine 2018 Oct;10(464):}

PubMed: \url{https://www.ncbi.nlm.nih.gov/pubmed/?term=30355799}

Pancreatic ductal adenocarcinoma (PDAC) has a dismal prognosis, and it
is unclear whether its stromal infiltrate contributes to its
aggressiveness. Here, we demonstrate that Dickkopf-3 (DKK3) is produced
by pancreatic stellate cells and is present in most human PDAC. DKK3
stimulates PDAC growth, metastasis, and resistance to chemotherapy with
both paracrine and autocrine mechanisms through NF-κB activation.
Genetic ablation of DKK3 in an autochthonous model of PDAC inhibited
tumor growth, induced a peritumoral infiltration of CD8+ T cells, and
more than doubled survival. Treatment with a DKK3-blocking monoclonal
antibody inhibited PDAC progression and chemoresistance and prolonged
survival. The combination of DKK3 inhibition with immune checkpoint
inhibition was more effective in reducing tumor growth than either
treatment alone and resulted in a durable improvement in survival,
suggesting that DKK3 neutralization may be effective as a single
targeted agent or in combination with chemotherapy or immunotherapy for
PDAC.

{}

{}

\begin{center}\rule{0.5\linewidth}{\linethickness}\end{center}

 - \textbf{TP63-Mediated Enhancer Reprogramming Drives the Squamous
Subtype of Pancreatic Ductal Adenocarcinoma}

\emph{Cell reports 2018 Nov;25(7):1741-1755.e7}

PubMed: \url{https://www.ncbi.nlm.nih.gov/pubmed/?term=30428345}

The aberrant expression of squamous lineage markers in pancreatic ductal
adenocarcinoma (PDA) has been correlated with poor clinical outcomes.
However, the functional role of this putative transdifferentiation event
in PDA pathogenesis remains unclear. Here, we show that expression of
the transcription factor TP63 (ΔNp63) is sufficient to install and
sustain the enhancer landscape and transcriptional signature of the
squamous lineage in human PDA cells. We also demonstrate that
TP63-driven enhancer reprogramming promotes aggressive tumor phenotypes,
including enhanced cell motility and invasion, and an accelerated growth
of primary PDA tumors and metastases in~vivo. This process ultimately
leads to a powerful addiction of squamous PDA cells to continuous TP63
expression. Our study demonstrates the functional significance of
squamous transdifferentiation in PDA and reveals TP63-based
reprogramming as an experimental tool for investigating mechanisms and
vulnerabilities linked to this aberrant cell fate transition.

{}

{}

\begin{center}\rule{0.5\linewidth}{\linethickness}\end{center}

 - \textbf{Desumoylating Isopeptidase 2 (DESI2) Inhibits Proliferation
and Promotes Apoptosis of Pancreatic Cancer Cells through Regulating
PI3K/AKT/mTOR Signaling Pathway}

\emph{Pathology oncology research : POR 2018 Nov;():}

PubMed: \url{https://www.ncbi.nlm.nih.gov/pubmed/?term=30411297}

This study aimed to investigate the effects of desumoylating
isopeptidase 2 (DESI2) on tumor cell proliferation, apoptosis and
invasion of pancreatic cancer, and to assess the signaling pathway
involved. Overexpression and silence of DESI2 were designed and the
experiments were divided into 5 groups: a normal control group, an
interference control group (shRNA-NC); an interference group (sh-DESI2);
an overexpression control group (NC), an overexpression group (DESI2).
Quantitative real time polymerase chain reaction (qRT-PCR) was used to
screen the appropriate interference sequence. The silencing and
overexpression of DESI2 were confirmed by qRT-PCR and western blotting.
Cell cycling, apoptosis, invasion, and the expression of
phosphatidylinositol-3-kinase (PI3K)-protein kinase B (AKT)-mammalian
target of rapamycin (mTOR) pathway and caspase 3 were measured.
Overexpression and silence of DESI2 were successfully designed in two
pancreatic cancer cells, and the interference effect of sh-DESI2-3
showed the best silencing effects. The biological activities of DESI2
were detected in both ASPC-1 and PANCE-1 cells. Our results showed that
cell proliferation was significantly increased in the sh-DESI2 group,
while decreased in DESI2 group compared with the control group in both
cell lines. In ASPC-1 cells, the events in G1 phase decreased and in S
phase increased obviously in the sh-DESI2 group, compared with control
group. An opposite result was found when DESI2 was overexpressed. In
PANCE-1 cells, the events in G2 phase were higher in the sh-DESI2 group,
while in the DESI2 group was significantly lower than that in control
group. In ASPC-1 and PANCE-1 cells, sh-DESI2 group showed decreased
apoptosis, increased cell invasion and increased expression of AKT,
p-Akt, PI3K, p-PI3K, p-mTOR and mTOR and decreased caspase 3 expression
compared with the control group, while overexpression of DESI2 leaded to
increased apoptosis, decreased cell invasion and reduced expression of
AKT, p-Akt, PI3K, p-PI3K, p-mTOR and mTOR and increased expression of
caspase 3. DESI2 regulates the proliferation and apoptosis of pancreatic
cancer cells through PI3K/AKT/mTOR signaling pathway.

{}

{}

\begin{center}\rule{0.5\linewidth}{\linethickness}\end{center}

 - \textbf{Disruption of stromal hedgehog signaling initiates
RNF5-mediated proteasomal degradation of PTEN and accelerates pancreatic
tumor growth}

\emph{Life science alliance 2018 Oct;1(5):e201800190}

PubMed: \url{https://www.ncbi.nlm.nih.gov/pubmed/?term=30456390}

The contribution of the tumor microenvironment to pancreatic ductal
adenocarcinoma (PDAC) development is currently unclear. We therefore
examined the consequences of disrupting paracrine Hedgehog (HH)
signaling in PDAC stroma. Herein, we show that ablation of the key HH
signaling gene Smoothened (Smo) in stromal fibroblasts led to increased
proliferation of pancreatic tumor cells. Furthermore, Smo deletion
resulted in proteasomal degradation of the tumor suppressor PTEN and
activation of oncogenic protein kinase B (AKT) in fibroblasts. An
unbiased proteomic screen identified RNF5 as a novel E3 ubiquitin ligase
responsible for degradation of phosphatase and tensin homolog (PTEN) in
Smo-null fibroblasts. Ring Finger Protein 5 (Rnf5) knockdown or
pharmacological inhibition of glycogen synthase kinase 3β (GSKβ), the
kinase that marks PTEN for ubiquitination, rescued PTEN levels and
reversed the oncogenic phenotype, identifying a new node of PTEN
regulation. In PDAC patients, low stromal PTEN correlated with reduced
overall survival. Mechanistically, PTEN loss decreased hydraulic
permeability of the extracellular matrix, which was reversed by
hyaluronidase treatment. These results define non-cell autonomous
tumor-promoting mechanisms activated by disruption of the HH/PTEN axis
and identifies new targets for restoring stromal tumor-suppressive
functions.

{}

{}

\begin{center}\rule{0.5\linewidth}{\linethickness}\end{center}

 - \textbf{The Long Noncoding RNA HOST2 Promotes Gemcitabine Resistance
in Human Pancreatic Cancer Cells}

\emph{Pathology oncology research : POR 2018 Nov;():}

PubMed: \url{https://www.ncbi.nlm.nih.gov/pubmed/?term=30406400}

Our study was aimed to identify the fundamental role of lncRNA HOST2 in
gemcitabine resistance regulation in human pancreatic cancer cells. The
levels of HOST2 in pancreatic cancer cell lines were measured by
quantitative real-time PCR (qRT-PCR). Due to high expression and strong
gemcitabine resistance, Hs766T and AsPC-1 cell lines were selected to be
knockdown the expression of HOST2 by transfection sh-HOST2. After
manipulation of HOST2, the cell proliferation induced by gemcitabine was
examined by CCK-8 assay. Next, colony formation ability of Hs766T and
AsPC-1 cell lines was determined by clone-forming assay. At last, the
relationship between HOST2 and cell apoptosis in Hs766T and AsPC-1 cell
lines was evaluated by flow cytometry. QRT-PCR revealed that HOST2 was
overexpressed in six pancreas neoplasm cell lines compared with normal
cell lines HPDE6-C7. HOST2 expression levels in group resistant to
gemcitabine were higher than the group sensitive to gemcitabine.
Additionally, CCK-8 assay verified that cell proliferation was inhibited
by sh-HOST2 with or without gemcitabine treatment. Furthermore,
clone-forming assay revealed that colony formation ability was weakened
by down-regulated HOST2 with or without gemcitabine treatment. Flow
cytometry revealed that cell apoptosis induced by gemcitabine was
promoted by sh-HOST2. In conclusion, down-regulated HOST2 inhibited
proliferation and promoted apoptosis of pancreas cancer cells with or
without gemcitabine treatment. Thus, HOST2 is a potential therapeutic
target for gemcitabine chemoresistance in pancreatic neoplasms.

{}

{}

\begin{center}\rule{0.5\linewidth}{\linethickness}\end{center}

 - \textbf{Preprogramming therapeutic response of PI3K/mTOR dual
inhibitor via the regulation of EHMT2 and p27 in pancreatic cancer}

\emph{American journal of cancer research 2018 09;8(9):1812-1822}

PubMed: \url{https://www.ncbi.nlm.nih.gov/pubmed/?term=30323973}

Pancreatic ductal adenocarcinoma (PDAC) is an aggressive disease, which
is characterized by its high invasiveness, rapid progression, and
profound resistance to therapy. Gemcitabine is the first-line treatment
option for pancreatic cancer patients, but the overall survival is quite
low. Therefore, it is an urgent issue to identify new molecules for
improved therapies, with better efficacy and less toxicity. Our previous
data indicated that Euchromatic histone-lysine N-methyltransferase 2
(EHMT2) functions as a therapeutic target to override GEM resistance and
promote metastasis in the treatment of pancreatic cancer. Here, we
screened a small-molecule library of 143 protein kinase inhibitors, to
verify cytotoxicity of different inhibitors in EHMT2-depleted cells. We
determined that the EHMT2 plays a promising modulating role for targeted
PI3K/mTOR inhibition. Our data revealed that EHMT2 down-regulates p27
expression, and this contributes to tumor growth. The depletion of
EHMT2, ectopic expression of methyltransferase-dead EHMT2, or treatment
with an EHMT2 inhibitor decreases H3K9 methylation of p27 promoter and
induces G1 arrest in PANC-1 pancreatic cancer cells. Consistent with
these findings, in vivo tumor xenograft models, primary tumors, and the
Oncomine database utilizing bioinformatics approaches, also show a
negative correlation between EHMT2 and p27. We further demonstrated that
low EHMT2 elevated BEZ235 sensitivity through up-regulation of p27 in
PDAC cells; high levels of SKP2 decrease BEZ235 responsiveness in PDAC
cells. Altogether, our results suggest the EHMT2-p27 axis as a potential
marker to modulate cell response to dual PI3K/mTOR inhibition, which
might provide a strategy in personalized therapeutics for PDAC patients.

{}

{}

\begin{center}\rule{0.5\linewidth}{\linethickness}\end{center}

 - \textbf{MicroRNA let-7d targets thrombospondin-1 and inhibits the
activation of human pancreatic stellate cells}

\emph{Pancreatology : official journal of the International Association
of Pancreatology (IAP) \ldots{} {[}et al.{]} 2018 Oct;():}

PubMed: \url{https://www.ncbi.nlm.nih.gov/pubmed/?term=30393009}

OBJECTIVES: The microRNA (miRNA) let-7d is linked to the formation of
pancreatic cancer-related fibrosis. In this study, the mechanism by
which let-7d regulates the activation of the human pancreatic stellate
cell (hPSC) was evaluated. METHODS: The transient transfection of a
let-7d mimic in the hPSCs was performed, and the altered thrombospondin
1 (THBS1) expression was confirmed by western blotting and real-time
qPCR. Targeting of the 3'-untranslated region (UTR) of THBS1 by let-7d
was investigated by the luciferase assays. After hPSC transfection using
THBS1 siRNA, the fibrosis markers (α-SMA and collagen 1A1) were
evaluated by western blotting and real-time qPCR. The correlation
between tumor fibrosis and let-7d or THBS1 was estimated using the data
from The Cancer Genome Atlas project. Finally, the effects of genistein
on the hPSCs were evaluated. RESULTS: We found that a let-7d mimic
inhibits THBS1 expression by targeting its 3'-UTR. THBS1 inhibition by
siRNA inhibited hPSC activation. An in silico analysis revealed that
let-7d and THBS1 expression are negatively correlated. Additionally,
let-7d was negatively correlated with the stromal score, while THBS1 was
positively correlated with this score. Genistein substantially induced
let-7d and decreased the expression of fibrosis marker along with the
inhibition of THBS1. CONCLUSIONS: Let-7d inhibited hPSC activation by
targeting THBS1. Genistein induced the expression of let-7d and might
modulate pancreatic fibrosis.

{}

{}

\begin{center}\rule{0.5\linewidth}{\linethickness}\end{center}

 - \textbf{Evaluation of NCAM and c-Kit as hepatic progenitor cell
markers for intrahepatic cholangiocarcinomas}

\emph{Pathology, research and practice 2018 Dec;214(12):2011-2017}

PubMed: \url{https://www.ncbi.nlm.nih.gov/pubmed/?term=30301635}

BACKGROUND: Intrahepatic cholangiocarcinomas (ICCs) are primary liver
malignancies and are the second most common type of malignancy after
hepatocellular carcinoma. ICCs are heterogeneous in clinical features,
genotype, and biological behavior, suggesting that ICCs can initiate in
different cell lineages. AIM: We investigated intrahepatic
cholangiocarcinoma RBE cell lines for the markers neural cell adhesion
molecule (NCAM) and c-Kit, which possess hepatic progenitor cells
properties. METHODS: NCAM + c-Kit + cells were tested for hepatic
progenitor cell properties including proliferation ability, colony
formation, spheroid formation, and invasiveness in NOD/SCID mice. The
Agilent Whole Human Genome Microarray Kit was used to evaluate
differences in gene expression related to stem cell signaling pathways
between NCAM + c-Kit + and NCAM-c-Kit- subset cells. Microarray results
were further confirmed by real-time RT-PCR. RESULTS:
NCAM + c-Kit + cells showed hepatic progenitor cell-like traits
including the abilities to self-renew and differentiate and
tumorigenicity in NOD/SCID mice. Differences were observed in the
expression of 421 genes related to stem cell signaling pathways (fc ≥ 2
or fc ≤ 0.5), among which 231 genes were upregulated and 190 genes were
downregulated. CONCLUSION: NCAM + c-Kit + subset cells in RBE may have
properties of hepatic progenitor cells. NCAM combined with c-Kit may be
a valuable marker for isolating and purifying ICC stem/progenitor cells.

{}

{}

\begin{center}\rule{0.5\linewidth}{\linethickness}\end{center}

Back to top

\begin{center}\rule{0.5\linewidth}{\linethickness}\end{center}

\pagebreak

\hypertarget{techniques-research-methods}{%
\subsubsection{Techniques \& Research
Methods}\label{techniques-research-methods}}

\begin{center}\rule{0.5\linewidth}{\linethickness}\end{center}

 - \textbf{Understanding Pancreatic Diseases Using Animated Pancreas
Patient: Informing Patients for Better Health Outcomes With Visual
Formats of Learning}

\emph{Pancreas 2018 10;47(10):1256-1261}

PubMed: \url{https://www.ncbi.nlm.nih.gov/pubmed/?term=30286013}

OBJECTIVES: The aim of this study was to evaluate the impact of Animated
Pancreas Patient (APP) educational modules (APP website and YouTube) on
pancreas education, awareness, and health outcomes. METHODS: This was a
retrospective study of APP metrics data from September 2013 to October
2017. We evaluated audience reach (number of visit sessions, unique
visitors, page views) and calculated top views by media type (animation,
expert video, patient video, and slide show) and top retention videos
from the modules. We also assessed the educational impact through
learner feedback survey. RESULTS: The APP had 1,475,252 views (547,693
unique visitors, 63.1\% in United States) during the study period. Most
popular topic viewed among the animations was ``Role and Anatomy of the
Pancreas'' (n = 361,116), and most common expert video viewed was
``Chronic Pancreatitis: What Foods and Beverages Should I Avoid?'' (n =
31,667). Participants who completed the online feedback survey reported
knowledge gains and commitments to change. CONCLUSIONS: Pancreas
education in visual formats of learning provided by APP demonstrated
wide reach and has substantial potential to inform and impact behaviors
of patients and caregivers. Continued efforts should be made to provide
patient resources that address health literacy and patient education and
respond to patient needs for better quality of life and improved health
outcomes in pancreatic diseases.

{}

{}

\begin{center}\rule{0.5\linewidth}{\linethickness}\end{center}

 - \textbf{Standard Operating Procedures for Biospecimen Collection,
Processing, and Storage: From the Consortium for the Study of Chronic
Pancreatitis, Diabetes, and Pancreatic Cancer}

\emph{Pancreas 2019 11;47(10):1213-1221}

PubMed: \url{https://www.ncbi.nlm.nih.gov/pubmed/?term=30325860}

High-quality and well-annotated biorepositories are needed to better
understand the pathophysiology and biologic mechanisms of chronic
pancreatitis (CP) and its consequences. We report a methodology for the
development of a robust standard operating procedure (SOP) for a
biorepository based on the experience of the clinical centers within the
consortium to study Chronic Pancreatitis, Diabetes and Pancreas Cancer
Clinical Centers (CPDPC), supported by the National Cancer Institute and
the National Institute for Diabetes and Digestive and Kidney Diseases as
a unique multidisciplinary model to study CP, diabetes, and pancreatic
cancer in both children and adults. Standard operating procedures from
the CPDPC centers were evaluated and consolidated. The literature was
reviewed for standard biorepository operating procedures that
facilitated downstream molecular analysis. The existing literature on
biobanking practices was harmonized with the SOPs from the clinical
centers to produce a biorepository for pancreatic research. This article
reports the methods and basic principles behind the creation of SOPs to
develop a biorepository for the CPDPC. These will serve as a guide for
investigators developing biorepositories in pancreas research. Rigorous
and meticulous adherence to standardized biospecimen collection will
facilitate investigations to better understand the pathophysiology and
biologic mechanisms of CP, diabetes, and pancreatic cancer.

{}

{}

\begin{center}\rule{0.5\linewidth}{\linethickness}\end{center}

Back to top

\begin{center}\rule{0.5\linewidth}{\linethickness}\end{center}

\pagebreak

\hypertarget{journals-reviewed}{%
\subsection{Journals Reviewed}\label{journals-reviewed}}

\begin{center}\rule{0.5\linewidth}{\linethickness}\end{center}

Advances in Anatomic Pathology\\
American Journal of Clinical Pathology\\
The American Journal of Gastroenterology\\
The American Journal of Pathology\\
American Journal of Surgical Pathology\\
Annals of Diagnostic Pathology\\
Annals of Surgery\\
Annals of Surgical Oncology\\
Annual Review of Pathology-Mechanisms of Disease\\
APMIS\\
Applied Immunohistochemistry \& Molecular Morphology\\
Archives of Pathology \& Laboratory Medicine\\
Cancer\\
Cancer Cell\\
Cancer Cytopathology\\
Cell\\
Clinical Cancer Research\\
Cytojournal\\
Cytopathology\\
Diagnostic Cytopathology\\
Diagnostic Pathology\\
Endocrine Pathology\\
Experimental and Molecular Pathology\\
Expert Review of Molecular Diagnostics\\
Gastroenterology\\
Gut\\
Histology and Histopathology\\
Histopathology\\
Human Pathology\\
International Journal of Surgical Pathology\\
International Journal of Clinical and Experimental Pathology\\
Journal of Clinical Pathology\\
Journal of Molecular Diagnostics\\
Journal of Pathology\\
Laboratory investigation\\
Lancet\\
Medical Molecular Morphology\\
Modern Pathology\\
Nature\\
Nature Reviews Gastroenterology \& Hepatology\\
NEJM\\
Pancreas\\
Pancreatology\\
Pathobiology\\
Pathologie Biologie\\
Pathology\\
Pathology \& Oncology Research\\
Pathology International\\
Pathology Research and Practice\\
PNAS\\
Seminars in Diagnostic Pathology\\
Seminars in Immunopathology\\
Surgical pathology clinics\\
Virchows Archiv

\begin{center}\rule{0.5\linewidth}{\linethickness}\end{center}

Back to top

\begin{center}\rule{0.5\linewidth}{\linethickness}\end{center}

\pagebreak

\hypertarget{feedback}{%
\subsection{Feedback}\label{feedback}}

\begin{center}\rule{0.5\linewidth}{\linethickness}\end{center}

Please send your feedbacks using the form below:

\href{https://docs.google.com/forms/d/e/1FAIpQLSeD3Z9J6Y7eMmiyM12f_SfAmHUlykb1zxZcwO6lg7cebGYQIQ/viewform}{Click
here for the Feedback Form}

\begin{center}\rule{0.5\linewidth}{\linethickness}\end{center}

Back to top

\begin{center}\rule{0.5\linewidth}{\linethickness}\end{center}

\pagebreak

\hypertarget{archive}{%
\subsection{Archive}\label{archive}}

\begin{center}\rule{0.5\linewidth}{\linethickness}\end{center}

\textbf{The PBPath Journal Archive}

\begin{itemize}
\item
  \href{http://pbpath.org/journal-watch/}{Current Issue}
\item
  \href{http://pbpath.org/journal-watch-archive/}{Older Issues}

  \begin{itemize}
  \item
    \href{http://pbpath.org/pbpath-journal-watch-june-july-2018/}{June-July-2018}
  \item
    \href{http://pbpath.org/pbpath-journal-watch-august-september-2018/}{August-September-2018}
  \item
    December issue is being prepared. Unofficial version is
    \href{http://pbpath.org/journal-watch-upcoming-issue/}{here}.
  \end{itemize}
\end{itemize}

\begin{center}\rule{0.5\linewidth}{\linethickness}\end{center}

Back to top

\begin{center}\rule{0.5\linewidth}{\linethickness}\end{center}

\pagebreak


\end{document}
