% Options for packages loaded elsewhere
\PassOptionsToPackage{unicode}{hyperref}
\PassOptionsToPackage{hyphens}{url}
%
\documentclass[
]{article}
\usepackage{lmodern}
\usepackage{amssymb,amsmath}
\usepackage{ifxetex,ifluatex}
\ifnum 0\ifxetex 1\fi\ifluatex 1\fi=0 % if pdftex
  \usepackage[T1]{fontenc}
  \usepackage[utf8]{inputenc}
  \usepackage{textcomp} % provide euro and other symbols
\else % if luatex or xetex
  \usepackage{unicode-math}
  \defaultfontfeatures{Scale=MatchLowercase}
  \defaultfontfeatures[\rmfamily]{Ligatures=TeX,Scale=1}
\fi
% Use upquote if available, for straight quotes in verbatim environments
\IfFileExists{upquote.sty}{\usepackage{upquote}}{}
\IfFileExists{microtype.sty}{% use microtype if available
  \usepackage[]{microtype}
  \UseMicrotypeSet[protrusion]{basicmath} % disable protrusion for tt fonts
}{}
\makeatletter
\@ifundefined{KOMAClassName}{% if non-KOMA class
  \IfFileExists{parskip.sty}{%
    \usepackage{parskip}
  }{% else
    \setlength{\parindent}{0pt}
    \setlength{\parskip}{6pt plus 2pt minus 1pt}}
}{% if KOMA class
  \KOMAoptions{parskip=half}}
\makeatother
\usepackage{xcolor}
\IfFileExists{xurl.sty}{\usepackage{xurl}}{} % add URL line breaks if available
\IfFileExists{bookmark.sty}{\usepackage{bookmark}}{\usepackage{hyperref}}
\hypersetup{
  pdftitle={Pancreatobiliary Pathology Society Journal Watch},
  hidelinks,
  pdfcreator={LaTeX via pandoc}}
\urlstyle{same} % disable monospaced font for URLs
\usepackage[margin=1in]{geometry}
\usepackage{graphicx,grffile}
\makeatletter
\def\maxwidth{\ifdim\Gin@nat@width>\linewidth\linewidth\else\Gin@nat@width\fi}
\def\maxheight{\ifdim\Gin@nat@height>\textheight\textheight\else\Gin@nat@height\fi}
\makeatother
% Scale images if necessary, so that they will not overflow the page
% margins by default, and it is still possible to overwrite the defaults
% using explicit options in \includegraphics[width, height, ...]{}
\setkeys{Gin}{width=\maxwidth,height=\maxheight,keepaspectratio}
% Set default figure placement to htbp
\makeatletter
\def\fps@figure{htbp}
\makeatother
\setlength{\emergencystretch}{3em} % prevent overfull lines
\providecommand{\tightlist}{%
  \setlength{\itemsep}{0pt}\setlength{\parskip}{0pt}}
\setcounter{secnumdepth}{5}
\renewcommand{\linethickness}{0.05em}

\title{Pancreatobiliary Pathology Society Journal Watch}
\usepackage{etoolbox}
\makeatletter
\providecommand{\subtitle}[1]{% add subtitle to \maketitle
  \apptocmd{\@title}{\par {\large #1 \par}}{}{}
}
\makeatother
\subtitle{October November December 2019}
\author{}
\date{\vspace{-2.5em}Last Update on 2019-12-26}

\begin{document}
\maketitle

{
\setcounter{tocdepth}{5}
\tableofcontents
}
\begin{center}\rule{0.5\linewidth}{\linethickness}\end{center}

\hypertarget{pbpath_jw}{%
\section{PBPath Journal Watch Articles}\label{pbpath_jw}}

\begin{center}\rule{0.5\linewidth}{\linethickness}\end{center}

\textbf{Wellcome to the PBPath Journal Watch!}

This bi-monthly journal watch features exciting recently published
pancreas and biliary pathology articles that will provide up to date
medical knowledge in our field. These articles will be showcased in
several convenient categories, including surgical pathology,
cytopathology, and molecular pathology among others. The articles in
each category are in no particular order. See the list of journals we
search regularly \href{http://pbpath.org/pbpath-journal-watch/}{here}.
Previous months' issues may be found in our
\emph{\href{http://pbpath.org/journal-watch-archive/}{archive}} and you
may see \href{http://pbpath.org/journal-watch-upcoming-issue/}{drafts of
the upcoming issue here}.

We encourage members to actively participate by recommending new
articles and providing feedback using the
\protect\hyperlink{feedback}{forms provided below.}

\textbf{We hope that you will enjoy the new PBPath Journal Watch!}

\begin{center}\rule{0.5\linewidth}{\linethickness}\end{center}

Back to top

\begin{center}\rule{0.5\linewidth}{\linethickness}\end{center}

\pagebreak

\hypertarget{surgical_pathology}{%
\subsection{Surgical Pathology}\label{surgical_pathology}}

\begin{center}\rule{0.5\linewidth}{\linethickness}\end{center}

\hypertarget{pancreas}{%
\subsubsection{Pancreas}\label{pancreas}}

\begin{center}\rule{0.5\linewidth}{\linethickness}\end{center}

\hypertarget{pancreas_morphology}{%
\paragraph{Morphology, Diagnostics, IHC}\label{pancreas_morphology}}

Morphology, Diagnostics, IHC

\begin{center}\rule{0.5\linewidth}{\linethickness}\end{center}

- \textbf{Comparison of pathology sampling protocols for
pancreatoduodenectomy specimens}

\emph{Virchows Archiv : an international journal of pathology 2019
Dec;():}

PubMed: \url{https://www.ncbi.nlm.nih.gov/pubmed/?term=31802231}

Pancreatoduodenectomy is one of the most challenging surgical specimens
for pathologists. Recently, two different, standardized protocols have
been proposed: the axial slicing Leeds protocol (LP) and the bi-valving
Adsay protocol (AP). Comparison between standardized and
non-standardized protocols (NSP) was performed with emphasis on margin
involvement and lymph node yield. Pancreatoduodenectomy cases were
retrospectively recruited: 46 sampled with LP, 52 cases with AP and 46
cases with NSP. Clinico-pathologic data and rates of margin/surface
involvement were collected and their prognostic influence on survival
was assessed. Statistical differences between NSP and AP and LP were
seen for nodal yield (p = 0.0001), N+ (p = 0.0001) and lymph node ratio
- LNR (p \textless{} 0.0008) but not between AP and LP. Differences in
R1/R0 status were statistically significant between NSP group (R1-15\%)
and both the LP (R1-73.9\%) and AP (R1-70\%) groups (p = 0.0001) but not
between LP and AP groups. At univariate survival analysis, grade
(p = 0.0023) and number of involved margins (p = 0.0096) in AP and
``N-category'' (p = 0.0057) ``resection margin status'' (p = 0.0094),
``stage'' (p = 0.0143), and ``number of involved margins'' (p = 0.00398)
in LP were statistically significant, while no variable was significant
in the NSP group. At multivariate analysis ``N category,'' ``resection
margin status,'' ``stage,'' ``number of involved margins,'' and ``LNR''
retained significance for the LP group. These results show that both LP
and AP perform better than non-standardized sampling making
standardization mandatory in pancreatoduodenectomy cut up. Both AP and
LP show strengths and weaknesses, and these may impact on the choice of
protocol in different institutions.

doi: \url{https://doi.org/10.1007/s00428-019-02687-6}

\begin{center}\rule{0.5\linewidth}{\linethickness}\end{center}

- \textbf{Tumor budding as a prognostic factor in pancreatic ductal
adenocarcinoma}

\emph{Virchows Archiv : an international journal of pathology 2019
Nov;():}

PubMed: \url{https://www.ncbi.nlm.nih.gov/pubmed/?term=31786688}

In this retrospective study, we analyzed the association between tumor
budding and perineural invasion as well as their prognostic role in
pancreatic ductal adenocarcinoma. A total of N = 119 patients resected
for pancreatic ductal carcinoma from 1996 to 2015 were included.
Clinical and standard histopathological parameters were retrieved from
the patient's records. One representative hematoxylin and eosin section
from the tumor region was examined for perineural invasion and tumor
budding using light microscopy. Tumor budding was assessed independently
using two different methods: in the first approach, the number of buds
was counted over three fields of 0.237~mm2 at 40-fold magnification; in
the second approach, tumor budding was quantified according to the
recommendation of the International Tumor Budding Consensus Conference
(ITBCC) over a field of 0.785~mm2 at 20-fold magnification. Linear and
logistic regression was applied to delineate association between
perineural invasion, tumor budding, and other parameters; Kaplan-Meier
and Cox regression were used in the survival analysis. Regardless of the
quantification approach, high tumor budding was a significant negative
prognostic factor in the univariable Cox regression (\textgreater{} 5
buds/0.237~mm2, hazard ratio (HR) 1.66, 95\% confidence interval (CI)
1.06-2.61, p = 0.027; ≥ 10 buds/0.785~mm2, HR 1.68, 95\% CI 1.07-2.64,
p = 0.024). In the multivariable model adjusting for stage and standard
histopathological parameters, lymph vessel invasion (HR = 2.43, 95\% CI
1.47-4.03, p = 0.001) and tumor budding \textgreater{} 5 buds/0.237~mm2
(HR = 1.70, 95\% CI 1.07-2.7, p = 0.026) were independent negative
prognostic factors, while adjuvant therapy was a positive prognostic
factor (HR = 0.54, 95\% CI 0.33-0.86, p = 0.009). No significant
prognostic value could be delineated for perineural invasion. In
conclusion, tumor budding is an independent negative prognostic factor
in pancreatic ductal adenocarcinoma associated with lymph node
metastasis. The prognostic role of perineural invasion remains
uncertain.

doi: \url{https://doi.org/10.1007/s00428-019-02719-1}

\begin{center}\rule{0.5\linewidth}{\linethickness}\end{center}

- \textbf{DNAJB1-PRKACA fusions occur in oncocytic pancreatic and
biliary neoplasms and are not specific for fibrolamellar hepatocellular
carcinoma}

\emph{Modern pathology : an official journal of the United States and
Canadian Academy of Pathology, Inc 2019 Nov;():}

PubMed: \url{https://www.ncbi.nlm.nih.gov/pubmed/?term=31676785}

Recently discovered DNAJB1-PRKACA oncogenic fusions have been considered
diagnostic for fibrolamellar hepatocellular carcinoma. In this study, we
describe six pancreatobiliary neoplasms with PRKACA fusions, five of
which harbor the DNAJB1-PRKACA fusion. All neoplasms were subjected to a
hybridization capture-based next-generation sequencing assay
(MSK-IMPACT), which enables the identification of sequence mutations,
copy number alterations, and selected structural rearrangements
involving ≥410 genes (n = 6) and/or to a custom targeted, RNA-based
panel (MSK-Fusion) that utilizes Archer Anchored Multiplex PCR
technology and next-generation sequencing to detect gene fusions in 62
genes (n = 2). Selected neoplasms also underwent FISH analysis, albumin
mRNA in-situ hybridization, and arginase-1 immunohistochemical labeling
(n = 3). Five neoplasms were pancreatic, and one arose in the
intrahepatic bile ducts. All revealed at least focal oncocytic
morphology: three cases were diagnosed as intraductal oncocytic
papillary neoplasms, and three as intraductal papillary mucinous
neoplasms with mixed oncocytic and pancreatobiliary or gastric features.
Four cases had an invasive carcinoma component composed of oncocytic
cells. Five cases revealed DNAJB1-PRKACA fusions and one revealed an
ATP1B1-PRKACA fusion. None of the cases tested were positive for albumin
or arginase-1. Our data prove that DNAJB1-PRKACA fusion is neither
exclusive nor diagnostic for fibrolamellar hepatocellular carcinoma, and
caution should be exercised in diagnosing liver tumors with
DNAJB1-PRKACA fusions as fibrolamellar hepatocellular carcinoma,
particularly if a pancreatic lesion is present. Moreover, considering
DNAJB1-PRKACA fusions lead to upregulated protein kinase activity and
that this upregulated protein kinase activity has a significant role in
tumorigenesis of fibrolamellar hepatocellular carcinoma, protein kinase
inhibition could have therapeutic potential in the treatment of these
pancreatobiliary neoplasms as well, once a suitable drug is developed.

doi: \url{https://doi.org/10.1038/s41379-019-0398-2}

\begin{center}\rule{0.5\linewidth}{\linethickness}\end{center}

- \textbf{Acinar cell carcinoma of the pancreas with thyroid-like
follicular features: first description of a new diagnostic challenging
subtype}

\emph{Virchows Archiv : an international journal of pathology 2019
Dec;475(6):789-794}

PubMed: \url{https://www.ncbi.nlm.nih.gov/pubmed/?term=31338587}

Acinar cell carcinomas (ACCs) of the pancreas are a heterogeneous group
of neoplasms showing a wide spectrum of morphological features including
acinar, solid, glandular, and trabecular architecture. In addition,
uncommon cytological aspects have recently been described and include
oncocytic, spindle, clear, and pleomorphic cell types. This wide
histological spectrum represents a challenge in the diagnostic task for
pathologists. Molecular mechanisms involved in the onset and progression
of ACCs are not completely known, but, in general, they differ from
those observed in ductal adenocarcinomas or neuroendocrine neoplasms of
the pancreas and frequently include alterations in the APC/β-catenin
pathway. In the present paper, we describe a new variant of ACC showing
thyroid-like follicular features and CTNNB1 mutation. This phenotype
needs to be included in the spectrum of morphological presentation of
ACC.

doi: \url{https://doi.org/10.1007/s00428-019-02628-3}

\begin{center}\rule{0.5\linewidth}{\linethickness}\end{center}

- \textbf{Histomorphology of pancreatic cancer in patients with
inherited ATM serine/threonine kinase pathogenic variants}

\emph{Modern pathology : an official journal of the United States and
Canadian Academy of Pathology, Inc 2019 Dec;32(12):1806-1813}

PubMed: \url{https://www.ncbi.nlm.nih.gov/pubmed/?term=31285527}

Germline pathogenic variants in the ATM serine/threonine kinase (ATM)
gene are associated with an increased risk of pancreatic ductal
adenocarcinoma. It is important to identify germline ATM pathogenic
variants in pancreatic cancer patients because these alterations are
potentially targetable with chemotherapeutic drugs and/or radiation and
have implications for other family members. As germline pathogenic
variants in other genes have been associated with distinct histologic
subtypes of pancreatic cancer, we studied the histomorphology of
pancreatic cancer in 23 patients with germline ATM pathogenic variants.
The histologic subtype was ductal adenocarcinoma in 19/23 (83\%) of the
patients, adenosquamous carcinoma in 1/23 (4\%), and colloid (mucinous
non-cystic) carcinoma in 3/23 (13\%). The percentage of colloid
(mucinous non-cystic) carcinomas is higher than we have previously
observed in patients with familial and sporadic pancreatic cancer (1 and
2\% in prior reports, p \textless{} 0.01 and p \textless{} 0.01,
respectively). Three carcinomas (2 colloid carcinomas, 1 ductal
adenocarcinoma) arose in association with intraductal papillary mucinous
neoplasms. Among the resected pancreata, non-invasive precursor lesions,
including pancreatic intraepithelial neoplasia and incipient intraductal
papillary mucinous neoplasms, were identified in 83\%. We conclude that
pancreatic cancers in patients with germline ATM pathogenic variants are
more frequently of colloid (mucinous non-cystic) morphology but are
overall morphologically diverse supporting the utility of universal
germline genetic testing for patients with pancreatic cancer.

doi: \url{https://doi.org/10.1038/s41379-019-0317-6}

\begin{center}\rule{0.5\linewidth}{\linethickness}\end{center}

- \textbf{Immunohistochemical profiling of liver metastases and
matched-pair analysis in patients with metastatic pancreatic ductal
adenocarcinoma}

\emph{Pancreatology : official journal of the International Association
of Pancreatology (IAP) \ldots{} {[}et al.{]} 2019 Oct;19(7):963-970}

PubMed: \url{https://www.ncbi.nlm.nih.gov/pubmed/?term=31542399}

BACKGROUND: The purpose of the current study was to investigate the
immunohistochemical (IHC) profile of liver metastases (LM) in patients
with pancreatic ductal adenocarcinoma (PDAC). METHODS: Expression of 15
IHC markers in liver biopsies from 77 patients with PDAC, who were
diagnosed between 2010 and 2014, were evaluated. In a separate subgroup
analysis (n\,=\,12), paired samples (LM and primary tumor) from the same
patient were investigated for IHC profile differences. RESULTS: LM
samples were classified as pancreatobiliary-type (PB-type) in 72
patients (93.5\%), intestinal-type (INT-type) in four patients (5.2\%),
and squamous in one patient (1.3\%). There was no significant difference
in overall survival (OS) between LM of the PB-type or INT-type
(p\,=\,0.097). In a multivariate analysis, age \textless70 years
(p\,=\,0.047), absence of SMAD4 mutation (p\,=\,0.026), absence of CDX2
expression (p\,=\,0.003), and well to moderate differentiation were
significant prognostic factors for better OS in patients with LM
(p\,=\,0.031). Analysis of paired tissue samples from LM and the primary
tumor revealed a difference in CDX2 (50\% increase, p\,=\,0.125) and
SMAD4 (33\% loss of SMAD4, p\,=\,0.375). CONCLUSIONS: CDX2 expression
and SMAD4 mutation indicate a poor outcome in patients with LM of PDAC.
Matched-pair analysis revealed differences in distinct IHC marker
expression.

doi: \url{https://doi.org/10.1016/j.pan.2019.09.005}

\begin{center}\rule{0.5\linewidth}{\linethickness}\end{center}

- \textbf{Sclerosing epithelioid mesenchymal neoplasm of the
pancreas~-~a proposed new entity}

\emph{Modern pathology : an official journal of the United States and
Canadian Academy of Pathology, Inc 2019 Aug;():}

PubMed: \url{https://www.ncbi.nlm.nih.gov/pubmed/?term=31383964}

We have encountered pancreatic tumors with unique histologic features,
which do not conform to any of the known tumors of the pancreas or other
anatomical sites. We aimed to define their clinicopathologic features
and whether they are characterized by recurrent molecular signatures.
Eight cases were identified; studied histologically and by
immunohistochemistry. Selected cases were also subjected to whole-exome
sequencing (WES; n = 4), RNA-sequencing (n = 6), Archer FusionPlex assay
(n = 5), methylation profiling using the Illumina MethylationEPIC (850k)
array platform (n = 6), and TERT promoter sequencing (n = 5). Six
neoplasms occurred in females. The mean age was 43 years (range: 26-75).
Five occurred in the head/neck of the pancreas. All patients were
treated surgically; none received neoadjuvant/adjuvant therapy. All
patients are free of disease after 53 months of median follow-up (range:
8-94). The tumors were well-circumscribed, and the median size was
1.8 cm (range: 1.3-5.8). Microscopically, the unencapsulated tumors had
a geographic pattern of epithelioid cell nests alternating with spindle
cell fascicles. Some areas showed dense fibrosis, in which enmeshed
tumor cells imparted a slit-like pattern. The predominant epithelioid
cells had scant cytoplasm and round-oval nuclei with open chromatin. The
spindle cells displayed irregular, hyperchromatic nuclei. Mitoses were
rare. No lymph node metastases were identified. All tumors were positive
for vimentin, CD99 and cytokeratin (patchy), while negative for markers
of solid pseudopapillary neoplasm, neuroendocrine, acinar,
myogenic/rhabdoid, vascular, melanocytic, or lymphoid differentiation,
gastrointestinal stromal tumor as well as MUC4. Whole-exome sequencing
revealed no recurrent somatic mutations or amplifications/homozygous
deletions in any known oncogenes or tumor suppressor genes.
RNA-sequencing and the Archer FusionPlex assay did not detect any
recurrent likely pathogenic gene fusions. Single sample gene set
enrichment analysis revealed that these tumors display a likely
mesenchymal transcriptomic program. Unsupervised analysis (t-SNE) of
their methylation profiles against a set of different mesenchymal
neoplasms demonstrated a distinct methylation pattern. Here, we describe
pancreatic neoplasms with unique morphologic/immunophenotypic features
and a distinct methylation pattern, along with a lack of abnormalities
in any of key genetic drivers, supporting that these neoplasms represent
a novel entity with an indolent clinical course. Given their mesenchymal
transcriptomic features, we propose the designation of ``sclerosing
epithelioid mesenchymal neoplasm'' of the pancreas.

doi: \url{https://doi.org/10.1038/s41379-019-0334-5}

\begin{center}\rule{0.5\linewidth}{\linethickness}\end{center}

Back to top

\begin{center}\rule{0.5\linewidth}{\linethickness}\end{center}

\pagebreak

\hypertarget{pancreas_staging}{%
\paragraph{Staging}\label{pancreas_staging}}

Pancreas TNM staging, Margins, Survival

\begin{center}\rule{0.5\linewidth}{\linethickness}\end{center}

- \textbf{Number of Examined Lymph Nodes and Nodal Status Assessment in
Distal Pancreatectomy for Body/Tail Ductal Adenocarcinoma}

\emph{Annals of surgery 2019 Dec;270(6):1138-1146}

PubMed: \url{https://www.ncbi.nlm.nih.gov/pubmed/?term=29672406}

OBJECTIVE: First, to assess the impact of the number of examined lymph
nodes (ELNs) on staging and survival after distal pancreatectomy (DP)
for pancreatic adenocarcinoma (PDAC). Second, to identify the minimum
number of ELNs (MNELNs) ensuring an accurate detection of nodal
involvement. Third, to reappraise the role of lymph node (LN)
parameters, including N-status and lymph node ratio (LNR). BACKGROUND:
In contrast with pancreatoduodenectomy, information on LN staging and
the MNELN required in DP is lacking. METHODS: Patients undergoing DP for
PDAC at 2 academic hospitals from 2000 through 2013 were retrospectively
analyzed. The eighth edition of the American Joint Committee on Cancer
staging system was used. The MNELN was estimated using the binomial
probability law. Survival analyses were performed separately for
node-negative and node-positive patients using univariable and
multivariable models. RESULTS: The study population consisted of 240
patients. The median number of ELN was 21, significantly lower in
node-negative patients as compared with node-positive patients (18.5 vs
24.0; P = 0.001). The proportion of node-positive patients increased
with increasing numbers of ELNs, whereas LNR showed an inverse trend.
The estimated MNELN was 20. The number of ELN (≥ or \textless20) was an
independent prognostic factor only in node-negative patients {[}odds
ratio (OR) 3.23 for ELN \textless20), suggesting a stage migration
effect. In node-positive patients, N2-class, but not LNR, was a
significant predictor of survival at multivariable analysis (OR 1.68).
CONCLUSION: The number of ELN affects nodal staging in body/tail PDAC.
At least 20 LNs are required for correct staging. N-status is superior
to LNR in predicting survival of node-positive patients.

doi: \url{https://doi.org/10.1097/SLA.0000000000002781}

\begin{center}\rule{0.5\linewidth}{\linethickness}\end{center}

- \textbf{Impact of Changes in the American Joint Committee on Cancer
Staging Manual, Eighth Edition, for Pancreatic Ductal Adenocarcinoma}

\emph{Pancreas 2019 08;48(7):876-882}

PubMed: \url{https://www.ncbi.nlm.nih.gov/pubmed/?term=31268985}

OBJECTIVE: Consistent and reliable tumor staging is a critical factor in
determining treatment strategy, selection of patients for adjuvant
therapy, and for therapeutic clinical trials. The aim of this study was
to evaluate the number and extent of pancreatic ductal adenocarcinoma
(PDAC) cases that would have a different pT, pN, and overall stages
based on the new eighth edition American Joint Committee on Cancer
staging system when compared with the seventh edition. METHODS: Patients
diagnosed with PDAC who underwent pancreaticoduodenectomy, total
pancreatectomy, or distal pancreatectomy from 2007 to 2017 were
retrospectively reviewed. A total of 340 cases were included. RESULTS:
According to the seventh edition, the vast majority of tumors in our
cohort were staged as pT3 tumors (88.2\%). Restaging these cases with
the new size-based pT system resulted in a more equal distribution among
the 3 pT categories, with higher percentage of pT2 cases (55\%).
CONCLUSIONS: The newly adopted pT stage protocol for PDAC is clinically
relevant, ensures a more equal distribution among different stages, and
allows for a significant prognostic stratification. In contrast, the new
pN classification (pN1 and pN2) based on the number of positive lymph
nodes failed to show survival differences and remains controversial.

doi: \url{https://doi.org/10.1097/MPA.0000000000001349}

\begin{center}\rule{0.5\linewidth}{\linethickness}\end{center}

Back to top

\begin{center}\rule{0.5\linewidth}{\linethickness}\end{center}

\pagebreak

\hypertarget{pancreas_preinvasive}{%
\paragraph{Preneoplastic and Preinvasive
Lesions}\label{pancreas_preinvasive}}

Preneoplastic and Preinvasive Lesions, PanIN, IPMN, MCN, ICPN

\begin{center}\rule{0.5\linewidth}{\linethickness}\end{center}

- \textbf{Intraductal Papillary Mucinous Neoplasms: Have IAP Consensus
Guidelines Changed our Approach?: Results from a Multi-institutional
Study}

\emph{Annals of surgery 2019 Dec;():}

PubMed: \url{https://www.ncbi.nlm.nih.gov/pubmed/?term=31804389}

OBJECTIVE: To evaluate the influence of consensus guidelines on the
management of intraductal papillary mucinous neoplasms (IPMN) and the
subsequent changes in pathologic outcomes. BACKGROUND: Over time,
multiple guidelines have been developed to identify high-risk IPMN. We
hypothesized that the development and implementation of guidelines
should have increased the percentage of resected IPMN with high-risk
disease. METHODS: Memorial Sloan-Kettering (MSK), Johns Hopkins (JH),
and Massachusetts General Hospital (MGH) databases were queried for
resected IPMN (2000-2015). Patients were categorized into main-duct
(MD-IPMN) versus branch-duct (BD-IPMN). Guideline-specific
radiographic/endoscopic features were recorded. High-risk disease was
defined as high-grade dysplasia/carcinoma. Fisher's exact test was used
to detect differences between institutions. Logistic regression
evaluated differences between time-points {[}preguidelines (pre-GL,
before 2006), Sendai (SCG, 2006-2012), Fukuoka (FCG, after 2012){]}.
RESULTS: The study included 1210 patients. The percentage of BD-IPMN
with ≥1 high-risk radiographic feature differed between centers (MSK
69\%, JH 60\%, MGH 45\%; P \textless{} 0.001). In MD-IPMN cohort, the
presence of radiographic features such as solid component and main
pancreatic duct diameter ≥10 mm also differed (solid component: MSK
38\%, JH 30\%, MGH 18\%; P \textless{} 0.001; duct ≥10 mm: MSK 49\%, JH
32\%, MGH 44\%; P \textless{} 0.001). The percentage of high-risk
disease on pathology, however, was similar between institutions
(BD-IPMN: P = 0.36, MD-IPMN: P = 0.48). During the study period, the
percentage of BD-IPMN resected with ≥1 high-risk feature increased (52\%
pre-GL vs 67\% FCG; P = 0.005), whereas the percentage of high-risk
disease decreased (pre-GL vs FCG: 30\% vs 20\%). For MD-IPMN, there was
not a clear trend towards guideline adherence, and the rate of high-risk
disease was similar over the time (pre-GL vs FCG: 69\% vs 67\%; P =
0.63). CONCLUSION: Surgical management of IPMN based on radiographic
criteria is variable between institutions, with similar percentages of
high-risk disease. Over the 15-year study period, the rate of BD-IPMN
resected with high-risk radiographic features increased; however, the
rate of high-risk disease decreased. Better predictors are needed.

doi: \url{https://doi.org/10.1097/SLA.0000000000003703}

\begin{center}\rule{0.5\linewidth}{\linethickness}\end{center}

Back to top

\begin{center}\rule{0.5\linewidth}{\linethickness}\end{center}

\pagebreak

\hypertarget{bile-ducts}{%
\subsubsection{Bile Ducts}\label{bile-ducts}}

\begin{center}\rule{0.5\linewidth}{\linethickness}\end{center}

\hypertarget{morphology-diagnostics-ihc}{%
\paragraph{Morphology, Diagnostics,
IHC}\label{morphology-diagnostics-ihc}}

Morphology, Diagnostics, IHC

\begin{center}\rule{0.5\linewidth}{\linethickness}\end{center}

- \textbf{Utility and limitations of Albumin RNA in situ hybridization
in the diagnosis of hepatobiliary lesions and metastatic carcinomas to
the liver}

\emph{Histopathology 2019 Dec;():}

PubMed: \url{https://www.ncbi.nlm.nih.gov/pubmed/?term=31834956}

BACKGROUND: Albumin messenger RNA in situ hybridization (RISH) is a
sensitive and specific marker for hepatocellular carcinoma (HCCs).
Intrahepatic cholangiocarcinoma (ICC) shows variable sensitivity,
whereas extrahepatic cholangiocarcinomas (ECCs) and metastatic
carcinomas are usually negative. We studied the clinical utility and
limitations of albumin RISH in a cohort of HCCs, ICCs, ECCs, bile duct
adenomas (BDAs), bile duct hamartomas (BDHs) and metastatic carcinomas
to the liver; and investigated the variability in sensitivity observed
for this marker in ICCs. DESIGN: We identified 122 cases (40 resections
and 82 biopsies) of hepatobiliary lesions and metastatic carcinomas.
Albumin RISH was performed using the RNAscope (Leica Biosystems, Buffalo
Grove, IL), the Bond III autostainer and probe Hs-ALB-01 (ACD, Newark,
CA) with negative (DapB) and positive probes (PPIB) for RNA. ICCs were
categorized according to the classification proposed by Hayashi et
al.~in small duct (SD), large duct (LD) and indeterminate (IND)
subtypes. RESULTS: Albumin RISH was positive in all 17 HCCs, and focally
in 75\% of BDAs. All 28 non-hepatic carcinomas, 13 BDHs and 9 ECCs were
negative. 35/47 (74.4\%) ICCs expressed albumin with 35/37 (94.6\%)
being of SD subtype, 2/3 (66.6\%) of the IND subtype and 1/7 (14.2\%) of
the LD subtype, P\textless0.003. CONCLUSION: Albumin RISH performed on
resection specimens or on small core biopsies is a sensitive and
specific marker for HCCs. It is highly sensitive and moderately specific
in the diagnosis of ICC with small gland morphology, but not in ICCs
with large duct morphology and in metastatic carcinoma. The variability
in sensitivity of albumin RISH in ICCs may depend on the subtypes of
ICCs.

doi: \url{https://doi.org/10.1111/his.14046}

\begin{center}\rule{0.5\linewidth}{\linethickness}\end{center}

- \textbf{Tumor Budding in Intrahepatic Cholangiocarcinoma: A Predictor
of Postsurgery Outcomes}

\emph{The American journal of surgical pathology 2019
Sep;43(9):1180-1190}

PubMed: \url{https://www.ncbi.nlm.nih.gov/pubmed/?term=31335353}

Intrahepatic cholangiocarcinoma (ICC) is an extremely aggressive
carcinoma. Useful predictors for the patients' prognosis after surgery
have not been fully established. From the University of Tokyo Hospital
pathology archives, we reviewed 107 cases of ICC, 54 cases of perihilar
cholangiocarcinoma, and 40 cases of extrahepatic cholangiocarcinoma
(ECC); we also investigated the significance of tumor budding in ICC, in
comparison with perihilar cholangiocarcinoma and ECC. The tumor-budding
frequencies were different by tumor location: 40.2\% (43/107) in ICC,
70.4\% (38/54) in perihilar cholangiocarcinoma, and 60.0\% (24/40) in
ECC. Tumor budding in ICC was associated with many pathologic indicators
associated with invasion, such as major vascular invasion (P=0.012) and
Union for International Cancer Control stage (P=0.007). Univariate and
multivariate Cox regression analyses revealed tumor budding as a
powerful prognostic factor for both recurrence-free survival (RFS) and
overall survival (OS) in ICC by univariate (RFS: hazard ratio {[}HR{]}:
2.666; 95\% confidence interval {[}CI{]}: 1.517-4.683, OS: HR: 4.206;
95\% CI: 2.447-7.230) and by multivariate analyses (RFS: HR: 3.038; 95\%
CI: 1.591-5.973, OS: HR: 4.547, 95\% CI: 2.348-8.805). Tumor budding was
also a significant prognostic factor of perihilar cholangiocarcinoma,
but not of ECC. When ICC was divided into 2 subtypes, type 1 (hilar) and
type 2 (peripheral), tumor budding was the strong prognostic factor in
type 2 ICC, but not in type 1 ICC, suggesting that some differences in
biological behavior exist between type 1 ICC and perihilar
cholangiocarcinoma. Tumor budding is prognostically important in ICC,
and its pathogenetic role in biliary tract carcinomas might be different
by anatomic location.

doi: \url{https://doi.org/10.1097/PAS.0000000000001332}

\begin{center}\rule{0.5\linewidth}{\linethickness}\end{center}

- \textbf{The Pathologic and Genetic Characteristics of the Intestinal
Subtype of Intraductal Papillary Neoplasms of the Bile Duct}

\emph{The American journal of surgical pathology 2019
Sep;43(9):1212-1220}

PubMed: \url{https://www.ncbi.nlm.nih.gov/pubmed/?term=31166202}

The present study aimed to identify the pathologic and genetic
characteristics of intestinal subtype of intraductal papillary neoplasm
of the bile duct (iIPNB) showing columnar cells with pseudostratified,
cigar-shaped nuclei, and basophilic or amphophilic cytoplasm with the
diffuse immunohistochemical expression of CK20 and/or CDX2. A total of
34 cases of iIPNB were pathologically examined according to their
anatomic location (the bile duct) and were then compared with the
intestinal subtype of intraductal papillary mucinous neoplasm (iIPMN) of
the pancreas (n=22). Mutations of 26 somatic genes were examined in
formalin-fixed paraffin-embedded tissue specimens from 21 cases of iIPNB
using the TruSight Tumor 26 gene panel and next-generation sequencing.
iIPNB cases were divided into intrahepatic (n=6) and extrahepatic (n=28)
categories. Intrahepatic IPNBs showed a less-complicated
villous-papillary pattern, while extrahepatic IPNBs showed a papillary
pattern with tubular and/or villous components and predominant
high-grade dysplasia with complicated architectures. MUC5AC was
frequently and extensively expressed in intrahepatic iIPNBs and iIPMNs
but not in extrahepatic iIPNBs. CD10 was frequently expressed in
extrahepatic IPNBs but not in intrahepatic iIPNBs or iIPMN. Genetic
mutations of TP53 and PIK3CA, which were infrequent or absent in iIPMNs,
were frequently detected in extrahepatic iIPNBs, while KRAS and GNAS,
which were commonly observed in iIPMNs, were frequently detected in
intrahepatic iIPNBs. Intrahepatic iIPNBs showed villous-papillary growth
with features reminiscent of iIPMNs, while extrahepatic iIPNBs showed
papillary growth with tubular and/or villous components, complicated
histology and variable differences from iIPMNs, suggesting differences
in the tumorigenesis of iIPNBs along the biliary tree.

doi: \url{https://doi.org/10.1097/PAS.0000000000001295}

\begin{center}\rule{0.5\linewidth}{\linethickness}\end{center}

- \textbf{Programmed cell death ligand-1 (PD-L1) expression in
extrahepatic biliary tract cancers: a comparative study using 22C3,
SP263 and E1L3N anti-PD-L1 antibodies}

\emph{Histopathology 2019 Oct;75(4):526-536}

PubMed: \url{https://www.ncbi.nlm.nih.gov/pubmed/?term=31081949}

AIMS: Pembrolizumab has shown promising results for patients with
programmed cell death ligand-1 (PD-L1)-positive advanced biliary tract
cancer in an ongoing clinical trial. However, data on PD-L1 expression
in bile duct cancers is limited, and the frequency of PD-L1 positivity
varies, which may be partly due to the assay used. The aim of this study
was to evaluate PD-L1 expression status in bile duct cancers by using
22C3, SP263 and E1L3N antibodies. METHODS AND RESULTS: We evaluated
PD-L1 expression in tissue microarrays of 183 extrahepatic bile duct
cancers, including 89 perihilar and 94 distal bile duct cancers, by
using 22C3, SP263 and E1L3N. When the 22C3 assay was used, tumoral PD-L1
was shown to be expressed in 16.9\% of cases at a 1\% threshold. When
the SP263 and E1L3N assays were used, tumoral PD-L1 was shown to be
expressed in 26\% and 7.1\% of cases, respectively. When whole tissue
sections were examined, 59.6\% of PD-L1-positive cases showed a low
percentage (\textless10\%) of positive tumour cells. Tumoral PD-L1
positivity was associated with poor histological differentiation
(P~=~0.017) and the biliary epithelial phenotype (P~=~0.041). High
tumoral PD-L1 expression (≥10\%) was associated with worse overall
survival (OS) and disease-free survival (DFS) (OS, P~=~0.012; DFS,
P~=~0.042). CONCLUSIONS: PD-L1 was expressed in a small subset of
patients with bile duct cancer, and the percentage of positive tumour
cells was low in PD-L1-positive cases. The SP263 assay showed the
highest PD-L1 positivity in both tumour cells and immune cells, followed
by the 22C3 and E1L3N assays. High PD-L1 expression was associated with
a poor prognosis in extrahepatic bile duct cancer patients.

doi: \url{https://doi.org/10.1111/his.13901}

\begin{center}\rule{0.5\linewidth}{\linethickness}\end{center}

Back to top

\begin{center}\rule{0.5\linewidth}{\linethickness}\end{center}

\pagebreak

\hypertarget{staging}{%
\paragraph{Staging}\label{staging}}

Bile Ducts TNM staging, Margins, Survival

\begin{center}\rule{0.5\linewidth}{\linethickness}\end{center}

- \textbf{Clinicopathologic and Prognostic Significance of Gallbladder
and Cystic Duct Invasion in Distal Bile Duct Carcinoma}

\emph{Archives of pathology \& laboratory medicine 2019 Nov;():}

PubMed: \url{https://www.ncbi.nlm.nih.gov/pubmed/?term=31755778}

CONTEXT.---: The roles of the gallbladder and cystic duct (CD) invasions
in distal bile duct carcinoma (DBDC) have not been well elucidated.
OBJECTIVE.---: To define the characteristics and prognostic significance
of gallbladder or CD invasions in patients with DBDC. DESIGN.---: Organ
invasion patterns with clinicopathologic features were assessed in 258
resected DBDCs. RESULTS.---: CD invasions (N = 31) were associated with
frequent concomitant pancreatic and/or duodenal invasions (23 of 31,
74\%) and showed stromal infiltration (16 of 31, 52\%) and intraductal
cancerization (15 of 31, 48\%) patterns. In only 2 cases, invasions with
intraductal cancerization were observed in the gallbladder neck.
Conversely, all pancreatic (N = 175) and duodenal (83) invasions
developed through stromal infiltration. CD invasions were associated
with larger tumor size (P = .001), bile duct margin positivity (P =
.001), perineural invasions (P = .04), and higher N categories (P =
.007). Patients with pancreatic or duodenal invasions had significantly
lower survival rates than those without pancreatic (median, 31.0 versus
93.9 months) or duodenal (27.5 versus 56.8 months, P \textless{} .001,
both) invasions. However, those with gallbladder or CD invasions did not
have different survival times (P = .13). Patients with concomitant
gallbladder/CD and pancreatic/duodenal invasions demonstrated
significantly lower survival rates than those without organ invasions (P
\textless{} .001). CONCLUSIONS.---: Gallbladder invasions were rare in
DBDCs as neck invasions with intraductal cancerization. CD invasions
occurred by stromal infiltrations and intraductal cancerization, whereas
all pancreatic and duodenal invasions had stromal infiltration patterns.
Gallbladder and/or CD invasions did not affect survival rates of
patients with DBDC, while pancreatic and duodenal invasions affected
survival rates. Therefore, these differences in survival rates may
originate from the different invasive patterns of DBDCs.

doi: \url{https://doi.org/10.5858/arpa.2019-0218-OA}

\begin{center}\rule{0.5\linewidth}{\linethickness}\end{center}

Back to top

\begin{center}\rule{0.5\linewidth}{\linethickness}\end{center}

\pagebreak

\hypertarget{gallbladder}{%
\subsubsection{Gallbladder}\label{gallbladder}}

\begin{center}\rule{0.5\linewidth}{\linethickness}\end{center}

\hypertarget{morphology-diagnostics-ihc-1}{%
\paragraph{Morphology, Diagnostics,
IHC}\label{morphology-diagnostics-ihc-1}}

Morphology, Diagnostics, IHC

\begin{center}\rule{0.5\linewidth}{\linethickness}\end{center}

- \textbf{Non-neoplastic Polyps of the Gallbladder: A Clinicopathologic
Analysis of 447 Cases}

\emph{The American journal of surgical pathology 2019 Nov;():}

PubMed: \url{https://www.ncbi.nlm.nih.gov/pubmed/?term=31725469}

There is no systematic histopathologic analysis of non-neoplastic polyps
in the gallbladder. In this study, in addition to a computer search for
cases designated as ``polyp,'' a systematic review of 2533 consecutive
routinely sampled archival and 203 totally submitted prospective
cholecystectomies were analyzed for \textgreater2 mm polyps (cut-off was
based on radiologic sensitivity). A total of 447 non-neoplastic polyps
were identified. The frequency was 3\% in archival cases and 5\% in
totally submitted cases. Only 21 (5\%) were ≥1 cm. The average age was
52 years, and the female to male ratio was 3.1. Two distinct categories
were delineated: (1) injury-related polyps (n=273): (a)
Fibro(myo)glandular polyps (n=214) were small (mean=0.4 cm),
broad-based, often multiple (45\%), almost always (98\%)
gallstone-associated, and were composed of a mixture of
(myo)fibroblastic tissue/lobular glandular units with chronic
cholecystitis. Dysplasia seen in 9\% seemed to be secondary involvement.
(b) Metaplastic pyloric glands forming polypoid collections (n=42). (c)
Inflammatory-type polyps associated with acute/subacute injury (11
granulation tissue, 3 xanthogranulomatous, 3 lymphoid). (2) Cholesterol
polyps (n=174) occurred in uninjured gallbladders, revealing a very thin
stalk, edematous cores devoid of glands but with cholesterol-laden
macrophages in 85\%, and cholesterolosis in the uninvolved mucosa in
60\%. Focal low-grade dysplasia was seen in 3\%, always confined to the
polyp, unaccompanied by carcinoma. In conclusion, non-neoplastic polyps
are seen in 3\% of cholecystectomies and are often small. Injury-related
fibromyoglandular polyps are the most common. Cholesterol polyps have
distinctive cauliflower architecture, often in a background of uninjured
gallbladders with cholesterolosis and may lack the cholesterol-laden
macrophages in the polyp itself. Although dysplastic changes can involve
non-neoplastic polyps, they do not seem to be the cause of invasive
carcinoma by themselves.

doi: \url{https://doi.org/10.1097/PAS.0000000000001405}

\begin{center}\rule{0.5\linewidth}{\linethickness}\end{center}

- \textbf{HER2 gene (ERBB2)~amplification is a low-frequency driver with
potential predictive value in gallbladder carcinoma}

\emph{Virchows Archiv : an international journal of pathology 2019
Dec;():}

PubMed: \url{https://www.ncbi.nlm.nih.gov/pubmed/?term=31838585}

Gallbladder carcinoma (GBC) is an aggressive type of cancer with a
dismal prognosis. Recent case reports have highlighted the human
epidermal growth factor receptor 2 (HER2) as a promising target for
individualized therapy~in biliary tract cancer; however, current data on
HER2 positivity in GBC is contradictory. This study aimed to assess the
proportion of HER2 positivity and its clinical implications in a large
and well-characterized European GBC cohort. HER2 status was determined
in 186 cases of surgically resected gallbladder adenocarcinoma and a
subset of coexistent high-grade biliary intraepithelial neoplasia
(BilIN, n = 74) in accordance with the up-to-date consensus for HER2
testing in gastric cancer by immunohistochemistry and dual-color
chromogenic in situ hybridization. Positivity for HER2 was observed in
5.4\% of all cases (n = 10). In those patients with concomitant
high-grade BilIN, two of four positive samples also showed amplification
in the precursor lesion, while in the two remaining cases, positivity
was either confined to invasive tumor or high-grade BilIN, exclusively.
Equivocal staining found in eleven cases was not accompanied by gene
amplification. Staging of the HER2-positive group was significantly
different from the HER2-negative group with most cases presenting at
stage IV, paralleled by a trend towards decreased survival. One patient
who received dual HER2 inhibition almost went into full clinical
remission despite treatment initiation in a metastasized state. Our
results reveal a low prevalence of HER2 positivity and highlight HER2
gene~amplification as an early, potentially driving event in
gallbladder~carcinogenesis. Prospective standardized HER2 testing and
randomized control studies are needed to prove clinical efficacy of
targeted HER2 inhibition in GBC.

doi: \url{https://doi.org/10.1007/s00428-019-02706-6}

\begin{center}\rule{0.5\linewidth}{\linethickness}\end{center}

- \textbf{Ultrastructural Characteristics of Gallbladder Epithelial
Inclusions Mimicking Cystoisospora}

\emph{American journal of clinical pathology 2020 Jan;153(1):88-93}

PubMed: \url{https://www.ncbi.nlm.nih.gov/pubmed/?term=31600399}

OBJECTIVES: There is recently reported increased prevalence of Isospora
organisms in cholecystectomy specimens from immunocompetent patients,
especially in acalculous cholecystectomies. We performed an
ultrastructural and molecular evaluation of these specimens. METHODS:
From 28 gallbladders with intraepithelial inclusions, two specimens with
diffuse involvement of the gallbladder epithelium were analyzed by
electron microscopy. Polymerase chain reaction was performed on five
samples for the ITS2 region of C belli and eukaryotic 18S region. The
18S products were sequenced by next-generation sequencing. RESULTS:
Electron microscopic analysis showed cytoplasmic condensations leading
to vacuole formation. In contrast with true C belli, there were no
identifiable organelles or organization. None of these cases showed
amplified products other than human on molecular analysis. CONCLUSIONS:
Electron microscopic analysis demonstrates that the inclusions are
condensed cytoplasmic material and not true organisms.

doi: \url{https://doi.org/10.1093/ajcp/aqz137}

\begin{center}\rule{0.5\linewidth}{\linethickness}\end{center}

- \textbf{Primary Gallbladder Neuroendocrine Tumors: Insights into a
Rare Histology Using a Large National Database}

\emph{Annals of surgical oncology 2019 Oct;26(11):3577-3585}

PubMed: \url{https://www.ncbi.nlm.nih.gov/pubmed/?term=31102094}

BACKGROUND: Primary gallbladder neuroendocrine tumors (NETs) are rare,
poorly understood cancers infrequently encountered at even the largest
of tertiary referral centers. We therefore sought to identify a large
cohort of patients with gallbladder NETs using a national database, with
the aim of defining treatment modalities employed and survival
associated with these uncommon malignancies. METHODS: Patients with
primary gallbladder NETs were identified in the National Cancer
Database, and clinicopathologic characteristics were recorded. A
univariate log-rank survival analysis was completed for patients who
underwent resection. Parameters found to be significant were entered
into a multivariate accelerated failure time analysis. For context,
survival comparisons were included for patients who underwent resections
for NETs at any gastrointestinal site and for gallbladder
adenocarcinoma. RESULTS: Overall, 754 patients with gallbladder NETs
were identified. Patients were predominantly female (n = 518, 69\%),
White (n = 503, 67\%), presented with stage IV disease (n = 295, 39\%)
and had high-grade lesions (n = 312, 41\%). The majority underwent
resection (n = 480, 64\%), primarily simple cholecystectomy (n = 431,
90\%), whereas a minority received multimodal therapy (n = 145, 21\%).
Among patients who underwent resection, older age (p = 0.001), large
cell histology (p = 0.012), and positive margins (p = 0.030) were
independently associated with worse overall survival. Patients with
gallbladder NETs had improved survival relative to those with
gallbladder adenocarcinoma (p = 0.001), but significantly worse survival
than patients with NETs from other gastrointestinal sites
(p \textless{} 0.001). CONCLUSIONS: Primary gallbladder NETs are
aggressive lesions that carry a worse prognosis than NETs of other
gastrointestinal sites. Older age, positive margins, and large cell
histology are associated with abbreviated survival after resection.

doi: \url{https://doi.org/10.1245/s10434-019-07440-6}

\begin{center}\rule{0.5\linewidth}{\linethickness}\end{center}

- \textbf{Cytoplasmic Fibrillar Aggregates in Gallbladder Epithelium Are
a Frequent Mimic of Cystoisospora in Pediatric Cholecystectomy
Specimens}

\emph{Archives of pathology \& laboratory medicine 2019
Oct;143(10):1259-1264}

PubMed: \url{https://www.ncbi.nlm.nih.gov/pubmed/?term=30969156}

CONTEXT.---: Cystoisospora belli is an intracellular parasite associated
with gastrointestinal disease in immunocompromised hosts. Although
infection has been classically associated with intestinal disease,
studies have identified Cystoisospora in the gallbladder of
immunocompetent patients based on hematoxylin-eosin morphology.
Recently, the identity of this histologic finding as Cystoisospora has
been questioned based on negative results of nucleic acid studies.
OBJECTIVE.---: To determine the prevalence of this histologic feature in
pediatric patients, we retrospectively reviewed all cholecystectomy
specimens from a pediatric hospital during a 24-month period.
DESIGN.---: In 180 cholecystectomy specimens, we identified 11 cases
(6.1\%) with classical histologic features previously described to
represent Cystoisospora organisms. To further investigate these
structures, we retrieved tissue from paraffin-embedded blocks and
performed electron microscopy. RESULTS.---: Ultrastructural examination
identified ovoid perinuclear cytoplasmic structures composed of dense
fibrillar aggregates rather than organisms. Patients with positive cases
were similar in age to controls (positive cases: mean patient age 13.4
years {[}range, 2-23 years{]}; negative cases: mean patient age 14.7
years {[}range, 12 weeks-31 years{]}; P = .35). There was no significant
association of this finding with cholelithiasis (54.5\% versus 65.1\%, P
= .52), cholesterolosis (0\% versus 22.5\%, P = .12), acute
cholecystitis (9.1\% versus 10.1\%, P \textgreater{} .99), or chronic
cholecystitis (45.5\% versus 66.3\%, P = .20). CONCLUSIONS.---: To our
knowledge, this is the first positive identification of these structures
as cytoplasmic fibrillar aggregates rather than parasitic inclusions by
ultrastructural examination, and the first study of this histologic
finding in pediatric cholecystectomies.

doi: \url{https://doi.org/10.5858/arpa.2018-0335-OA}

\begin{center}\rule{0.5\linewidth}{\linethickness}\end{center}

- \textbf{IL-33 overexpression in gallbladder cancers associated with
pancreatobiliary maljunction}

\emph{Histopathology 2019 Sep;75(3):365-375}

PubMed: \url{https://www.ncbi.nlm.nih.gov/pubmed/?term=30882917}

AIMS: To investigate whether genetic or inflammatory pro-oncogenic
factors are relevant to the increased risk of gallbladder cancers in
patients with pancreaticobiliary maljunction (PBM). METHODS AND RESULTS:
Mutations in KRAS exon 2 were examined by a highly sensitive, droplet
digital PCR platform using surgically resected specimens of
PBM-associated (n~=~31) and non-associated gallbladder cancers (n~=~49).
The tissue expression of IL-6 and IL-33, which are suspected to promote
biliary carcinogenesis, was analysed by quantitative real-time PCR and
in-situ hybridisation. The incidence of KRAS mutations was similarly low
in PBM-associated (five of 32 cases; 16\%) and non-associated cancers
(four of 49 cases; 8\%) (P~=~0.272). The tissue expression of IL-33
mRNA, but not IL-6 mRNA, was significantly higher in PBM-associated
gallbladder cancers than in gallbladder cancers without PBM (P~=~0.004).
A similar degree of IL-33 overexpression was also observed in the
background non-cancerous mucosa in cases of PBM-associated gallbladder
cancers, and was significantly greater than that in PBM cases with
cholecystitis alone (P~\textless~0.001). The results of in-situ
hybridisation indicated that the source of IL-33 production in
PBM-associated carcinomas was the endothelium, cancer cells and
non-neoplastic biliary epithelium. In a combined PBM-associated and
non-associated cohort, IL-33 overexpression in gallbladder cancers
correlated with less aggressive features (e.g.~a lower pT stage and
longer overall survival), similar to recently reported findings on
large-duct cholangiocarcinomas. CONCLUSIONS: KRAS mutations do not
appear to be associated with a high risk of malignancy in PBM, while
IL-33 overexpression may provide a pro-oncogenic microenvironment in the
gallbladder mucosa of patients with PBM.

doi: \url{https://doi.org/10.1111/his.13863}

\begin{center}\rule{0.5\linewidth}{\linethickness}\end{center}

Back to top

\begin{center}\rule{0.5\linewidth}{\linethickness}\end{center}

\pagebreak

\hypertarget{ampulla-of-vater}{%
\subsubsection{Ampulla of Vater}\label{ampulla-of-vater}}

\begin{center}\rule{0.5\linewidth}{\linethickness}\end{center}

\hypertarget{morphology-diagnostics-ihc-2}{%
\paragraph{Morphology, Diagnostics,
IHC}\label{morphology-diagnostics-ihc-2}}

Morphology, Diagnostics, IHC

\begin{center}\rule{0.5\linewidth}{\linethickness}\end{center}

- \textbf{Mutational profiling and immunohistochemical analysis of a
surgical series of ampullary carcinomas}

\emph{Journal of clinical pathology 2019 Nov;72(11):762-770}

PubMed: \url{https://www.ncbi.nlm.nih.gov/pubmed/?term=31256008}

AIMS: Knowledge regarding the genetic features of ampullary carcinoma
(AC) in European patients is limited. The utility of tumour markers for
the establishment of a malignant diagnosis in biopsies from the
ampullary region has not been fully elucidated. We aimed to describe the
clinical, pathological, immunohistochemical (IHC) and genetic features
of a Danish series of surgically resected ACs. METHODS: Surgically
resected ACs (n=59) were examined regarding (1) clinicopathological
features, (2) histological subtypes, (3) expression of IMP3, maspin,
MUC5AC and S100P and (4) next-generation sequencing using a hybrid
capture-based platform (Illumina HiSeq2500), including 315
cancer-related genes plus introns from 28 genes often rearranged or
altered in cancer. Tumour mutational burden (TMB) and microsatellite
instability (MSI) were also evaluated. RESULTS: Pancreatobiliary
adenocarcinomas (PB-AC), intestinal adenocarcinomas (INT-AC), other
ampullary tumours and mixed adenocarcinomas represented 45.8\%, 23.7\%,
16.9\% and 13.6\%. The proportion of IHC-positive ACs (score ≥2) was:
Maspin (94.9\%), IMP3 (67.8\%), S100P (39.0\%) and MUC5AC (18.6\%). Most
frequently altered genes were TP53 (59.3\%), KRAS (40.7\%), APC
(27.8\%), SMAD4 (20.4\%), CDKN2A (16.7\%) and ARID2/PIK3CA (each
11.1\%). MUC5AC and S100P were frequently expressed in PB-AC, APC
alterations frequent in INT-AC, SOX9 alterations were exclusive in
INT-AC and MDM2 and FRS2 alterations in PB-AC. Four of 49 ACs (8.2\%)
were TMB-high/MSI-high and showed loss of MLH1 and PMS2. CONCLUSIONS:
PB-AC was the most frequent histological subtype of AC. Maspin and IMP3
were the IHC tumour markers with the highest sensitivity. Adenocarcinoma
subtypes differed regarding several genetic alterations, whose
predictive value remains to be evaluated.

doi: \url{https://doi.org/10.1136/jclinpath-2019-205912}

\begin{center}\rule{0.5\linewidth}{\linethickness}\end{center}

- \textbf{Poorly Cohesive (Signet Ring Cell) Carcinoma of the Ampulla of
Vater}

\emph{International journal of surgical pathology 2019
Oct;():1066896919880968}

PubMed: \url{https://www.ncbi.nlm.nih.gov/pubmed/?term=31612756}

In the ampulla of Vater, carcinomas with
``diffuse-infiltrative''/``signet ring cell'' morphology, designated as
``poorly cohesive carcinoma'' (PCC) in the WHO classification, are very
rare and poorly characterized. Nine cases with a classical PCC
morphology constituting \textgreater50\% of the tumor were identified.
Mean age was 64.8 years (vs 64.6 in ampullary carcinomas {[}ACs{]}) and
6 were males, 3 females. The mean invasive tumor size was 2.5 cm (vs 1.9
in ACs). Other morphologic patterns displayed included cord-like
infiltration (n=2), plasmacytoid cells (n=2), and microglandular
component (n=4), including goblet cell adenocarcinoma-like foci. None of
the cases were associated with dysplasia. By immunohistochemistry, the
carcinomas did not show intestinal differentiation (CDX2 0/9, CK20 1/9,
MUC2 3/9), MUC1 was positive in 4/9, MUC5AC was positive in 7/8.
E-cadherin loss was noted in 4/9. All cases were advanced stage
(6/9-pT3, 3/9-pT4) (vs 43\% in ACs). Lymph node metastases were
identified in 44\% (vs 45\% in AC). Six patients (67\%) died of disease
at a median of 25 months, 3 were alive at 13, 15, and 60 months. Overall
median survival was significantly worse than that of intestinal-type ACs
(26 vs 122 months, P = .006) and trended toward worse than
pancreatobiliary type (26 vs 42 months, P = .1). In conclusion, PCCs
constitute 2.45\% of all ACs. These present as advanced tumors and
express upper-gastrointestinal immunoprofile with frequent MUC5AC
labeling, which may be helpful in identifying subtle infiltration in the
surface mucosa since MUC5AC is not expressed in the ampullary mucosa.
Patients have poor prognosis.

doi: \url{https://doi.org/10.1177/1066896919880968}

\begin{center}\rule{0.5\linewidth}{\linethickness}\end{center}

Back to top

\begin{center}\rule{0.5\linewidth}{\linethickness}\end{center}

\pagebreak

\hypertarget{neuroendocrine}{%
\subsubsection{Neuroendocrine}\label{neuroendocrine}}

\begin{center}\rule{0.5\linewidth}{\linethickness}\end{center}

\hypertarget{pannet}{%
\paragraph{PanNET}\label{pannet}}

PanNET, Pancreatic Neuroendocrine Tumors and related neuroendocrine
neoplasms

\begin{center}\rule{0.5\linewidth}{\linethickness}\end{center}

- \textbf{Immunohistochemically Detected Expression of ATRX, TSC2, and
PTEN Predicts Clinical Outcomes in Patients With Grade 1 and 2
Pancreatic Neuroendocrine Tumors}

\emph{Annals of surgery 2019 Oct;():}

PubMed: \url{https://www.ncbi.nlm.nih.gov/pubmed/?term=31599805}

OBJECTIVE: The goal of this retrospective study was to clarify the
clinical implications of immunohistochemically detected protein
expression for genes that are frequently mutated in pancreatic
neuroendocrine tumors (PNETs). BACKGROUND: The clinical management of
PNETs is hindered by their heterogenous biological behavior. Whole-exome
sequencing recently showed that 5 genes (DAXX/ATRX, MEN1, TSC2, and
PTEN) are frequently mutated in PNETs. However, the clinical
implications of the associated alterations in protein expression remain
unclear. METHODS: We collected Grade 1 and 2 (World Health Organization
2017 Classification) primary PNETs samples from 100 patients who
underwent surgical resection. ATRX, DAXX, MEN1, TSC2, and PTEN
expression were determined immunohistochemically to clarify their
relationships with prognosis and clinicopathological findings. RESULTS:
Kaplan-Meier analysis indicated that loss of TSC2 (n = 58) or PTEN (n =
37) was associated with significantly shorter overall survival, and that
loss of TSC2 or ATRX (n = 41) was associated with significantly shorter
recurrence-free survival. Additionally, loss of ATRX or TSC2 was
significantly associated with nodal metastasis. In a multivariate
analysis, combined loss of TSC2 and ATRX (n = 31) was an independent
prognostic factor for shorter recurrence-free survival (hazard ratio
10.1, 95\% confidence interval 2.1-66.9, P = 0.003) in G2 PNETs.
CONCLUSIONS: Loss of ATRX, TSC2, and PTEN expression might be useful as
a method of clarifying the behavior and clinical outcomes of Grade 1 and
2 PNETs in routine clinical practice. Combined loss of TSC2 and ATRX had
an especially strong, independent association with shorter
recurrence-free survival in patients with G2 PNETs. Loss of pairs in
ATRX, TSC2, or PTEN would be useful for selecting the candidate for
postoperative adjuvant therapy.

doi: \url{https://doi.org/10.1097/SLA.0000000000003624}

\begin{center}\rule{0.5\linewidth}{\linethickness}\end{center}

Back to top

\begin{center}\rule{0.5\linewidth}{\linethickness}\end{center}

\pagebreak

\hypertarget{staging-1}{%
\paragraph{Staging}\label{staging-1}}

PanNET TNM staging, Margins, Survival

\begin{center}\rule{0.5\linewidth}{\linethickness}\end{center}

- \textbf{Trends in the Number of Lymph Nodes Evaluated Among Patients
with Pancreatic Neuroendocrine Tumors in the United States: A
Multi-Institutional and National Database Analysis}

\emph{Annals of surgical oncology 2019 Dec;():}

PubMed: \url{https://www.ncbi.nlm.nih.gov/pubmed/?term=31838609}

BACKGROUND: The role of routine lymphadenectomy in the surgical
treatment of pancreatic neuroendocrine tumors (pNET) remains poorly
defined. The objective of the current study was to investigate trends in
the number of lymph nodes (LN) evaluated for pNET treatment at a
nationwide level. METHODS: Patients undergoing surgery for pNET between
2000 and 2016 were identified in the U.S. Neuroendocrine Tumor Study
Group (US-NETSG) database as well as the Surveillance, Epidemiology, and
End Results (SEER) database. The number of LNs examined was evaluated
over time. RESULTS: The median number of evaluated LNs increased roughly
fourfold over the study period (US-NETSG, 2000: 3 LNs vs.~2016: 13 LNs;
SEER, 2000: 3 LNs vs.~2016: 11 LNs, both p \textless{} 0.001). While no
difference in 5-year OS and RFS was noted among patients who had 1-3
lymph node metastases (LNM) vs. ≥ 4 LNM between 2000-2007 (OS 73.5\%
vs.~69.9\%, p = 0.12; RFS: 64.9\% vs.~40.1\%, p = 0.39), patients who
underwent resection and LN evaluation during the period 2008-2016 had an
incrementally worse survival if the patient had node negative disease,
1-3 LNM and ≥ 4 LNM (OS 86.8\% vs.~82.7\% vs.~74.9\%,
p \textless{} 0.001; RFS: 86.3\% vs.~64.7\% vs.~50.4\%,
p \textless{} 0.001). On multivariable analysis, a more recent year of
diagnosis, pancreatic head tumor location, and tumor
size \textgreater{} 2~cm were associated with 12 or more LNs evaluated
in both US-NETSG and SEER databases. CONCLUSION: The number of LNs
examined nearly quadrupled over the last decade. The increased number of
LNs examined suggested a growing adoption of the AJCC staging manual
recommendations regarding LN evaluation in the treatment of pNET.

doi: \url{https://doi.org/10.1245/s10434-019-08120-1}

\begin{center}\rule{0.5\linewidth}{\linethickness}\end{center}

Back to top

\begin{center}\rule{0.5\linewidth}{\linethickness}\end{center}

\pagebreak

\hypertarget{cytopathology}{%
\subsection{Cytopathology}\label{cytopathology}}

\begin{center}\rule{0.5\linewidth}{\linethickness}\end{center}

\hypertarget{pancreas-1}{%
\subsubsection{Pancreas}\label{pancreas-1}}

\begin{center}\rule{0.5\linewidth}{\linethickness}\end{center}

- \textbf{Assessment of preoperative pancreatic biopsy,
cytological/histological review of cell-block-specimens obtained by
endoscopic ultrasound-guided fine-needle aspiration: Laboratory-based
study}

\emph{Diagnostic cytopathology 2019 Dec;():}

PubMed: \url{https://www.ncbi.nlm.nih.gov/pubmed/?term=31825182}

BACKGROUND: Pancreatic cancer is among the most lethal cancers worldwide
due to the limited availability of techniques for early detection of
signs and symptoms. Reportedly, it is the fourth-leading cause of
cancer-related mortality among Japanese adults. With the advent of
endoscopic ultrasound-guided fine-needle aspiration (EUS-FNA) for
diagnosing pancreatic cancer, the rate of the cytological and
histological diagnoses of cell-block-specimens has significantly
increased in Japan. METHODS: The cytological specimens of 165 patients
with pancreatic lesions obtained using EUS-FNA between January 2010 and
July 2016 at the Kyorin University Hospital were investigated. The
clinical course of 153 patients was assessed from their clinical
records, which included information on their imaging diagnosis,
laboratory data, final clinical diagnosis and treatment; moreover, the
accuracy of the cytological/histological examination and clinical
diagnosis at our hospital were analysed. RESULTS: The number of cells in
cell-block-specimens was too small to estimate data. However,
cytological specimens were sufficient to observe the findings of
suspected malignancy such as necrosis. Biopsy was deemed necessary for
diagnosis using both histological and cytological specimens. CONCLUSION:
EUS-FNA can be used not only to diagnose benign or malignant types of
pancreatic cancers but also to assess the sensitivity of molecular
target drugs and chemotherapy methods. Therefore, both histological and
cytological diagnoses are required to enhance diagnostic precision both
in our hospital and at other institutions.

doi: \url{https://doi.org/10.1002/dc.24358}

\begin{center}\rule{0.5\linewidth}{\linethickness}\end{center}

- \textbf{EUS-FNA diagnosis of pancreatic serous cystadenoma with the
aid of cell blocks and α-inhibin immunochemistry: A case series}

\emph{Diagnostic cytopathology 2019 Nov;():}

PubMed: \url{https://www.ncbi.nlm.nih.gov/pubmed/?term=31785091}

Serous cystadenoma (SCA) is an uncommon benign pancreatic neoplasm that
is most often managed conservatively with follow-up rather than surgical
excision. Therefore, to avoid the serious complications of pancreatic
surgery, SCA should be diagnosed accurately at the preoperative level.
Preoperative SCA diagnosis requires a multimodal diagnostic approach
that includes imaging, cystic fluid biochemical analysis and/or
endoscopic ultrasound fine-needle aspiration (EUS-FNA). In this brief
report, we describe six EUS-FNA cases from five patients that were
reported as ``benign, consistent with serous cystadenoma''. Samples were
hypocellular, composed of loose clusters and single cuboidal,
bland-looking cells among epithelial sheets representing
gastrointestinal contamination. Cell blocks were prepared and all six
FNA cases revealed cuboidal cells with a positive α-inhibin
immunophenotype, consistent with a diagnosis of SCA. As EUS-FNAs of SCA
commonly result in non-diagnostic interpretations, cell block
preparations with subsequent immunochemistry can increase their
diagnostic accuracy and guide patient management.

doi: \url{https://doi.org/10.1002/dc.24348}

\begin{center}\rule{0.5\linewidth}{\linethickness}\end{center}

- \textbf{Dissecting the presence of malignant squamous cells in
pancreatic cytopathology: A case series}

\emph{Diagnostic cytopathology 2019 Dec;47(12):1287-1292}

PubMed: \url{https://www.ncbi.nlm.nih.gov/pubmed/?term=31407529}

The presence of malignant squamous cells in pancreatic cytopathology is
a rare phenomenon that results either from a primary or a metastatic
process. Pancreatic adenosquamous carcinoma (PASC) represents the most
common variant of pancreatic ductal adenocarcinoma and is associated
with a dismal prognosis. Within the period of 2013-2018, the archives of
``Hygeia and Mitera Hospital'' were searched for pancreatic
cytopathology-related diagnoses that included the interpretation of
``malignant squamous cells present.'' All fine needle aspirations (FNAs)
of pancreatic lesions, including liver metastases in patients with known
pancreatic primaries, were retrieved along with their relevant clinical
information. Five pancreatic and two liver FNAs acquired from a total of
six patients were reexamined. None of these patients had any documented
history of primary squamous malignancy elsewhere. All pancreatic and one
of the two liver FNAs showed malignant squamous cells, identified based
on either morphology or immunochemistry. The other liver FNA represented
a metastatic deposit which comprised of only a glandular component,
whereas the associated pancreatic FNA exhibited both squamous and
glandular counterparts. Most cases characteristically showed necrosis
and keratinization. Of interest, two cases revealed the presence of
tumor-associated giant cells. In conclusion, the presence of malignant
squamous cells in pancreatic FNAs could mean the presence of PASC,
especially when there is no documented history of a primary malignancy
and a complete clinical and imaging workup has been performed.
Immunochemistry on cell block material could help to confirm squamous
differentiation in the absence of overt keratinization.

doi: \url{https://doi.org/10.1002/dc.24302}

\begin{center}\rule{0.5\linewidth}{\linethickness}\end{center}

- \textbf{Pancreatoblastoma: Cytologic and histologic analysis of 12
adult~cases reveals helpful criteria in their diagnosis and distinction
from common mimics}

\emph{Cancer cytopathology 2019 Nov;127(11):708-719}

PubMed: \url{https://www.ncbi.nlm.nih.gov/pubmed/?term=31581358}

BACKGROUND: Pancreatoblastoma (PBL) is a rare malignant pancreatic tumor
seen predominantly in childhood, and its cytologic diagnosis remains
challenging. METHODS: Twelve fine-needle-aspirations from 11 adults were
analyzed. RESULTS: In total, 6 men and 5 women (median age, 45~years;
age range, 32-60~years) had tumors measuring a median 5.6~cm (range,
2.5-12~cm) located in the pancreatic head (n~=~7) or tail (n~=~4),
including 3 with familial adenomatous polyposis (FAP)/FAP-related
syndromes and 4 with metastasis at diagnosis. The median follow-up was
39.8~months (range, 0.8-348~months), and 5 patients died of disease. The
original cytology diagnoses were: PBL (n~=~2), neuroendocrine neoplasm
(n~=~2), poorly differentiated neuroendocrine carcinoma (n~=~2), well
differentiated neuroendocrine tumor (n~=~1), poorly differentiated
carcinoma (n~=~2), ``positive for malignancy'' (n~=~1), acinar cell
carcinoma (n~=~1), and epithelioid neoplasm with endocrine and acinar
differentiation versus PBL (n~=~1). Universal cytopathologic findings
included hypercellularity; 3-dimensional clusters; and single,
monotonous, blast-like cells that were from 1.5 to 2.0 times the size of
red blood cells with high nuclear-to-cytoplasmic ratio, fine chromatin,
small, distinct nucleoli, and a resemblance to well differentiated
neuroendocrine tumor and poorly differentiated neuroendocrine carcinoma.
Branching pseudopapillae (n~=~7) and grooved nuclei (n~=~3) raised the
differential diagnosis of solid-pseudopapillary neoplasm, but with more
atypia. Uncommon features included pleomorphism (n~=~4) and numerous
mitoses (n~=~1). Squamoid morules were seen on smears (n~=~5) or cell
blocks (n~=~6) in 70\% of patients and were characterized by epithelioid
cells with elongated, streaming nuclei, fine chromatin, absent nucleoli,
and positive nuclear β-catenin (n~=~6 of 8). The median Ki-67 index was
21\% (range, 2\%-70\%), and neuroendocrine marker expression was common
(100\%), but acinar markers were variable (63\%). CONCLUSIONS: A
combination of cytologic findings in PBL, including a predominant
population of primitive blast-like cells, subtle squamoid morules,
frequent neuroendocrine and variable acinar phenotype, should facilitate
accurate cytologic diagnosis and distinction from common mimics.

doi: \url{https://doi.org/10.1002/cncy.22187}

\begin{center}\rule{0.5\linewidth}{\linethickness}\end{center}

Back to top

\begin{center}\rule{0.5\linewidth}{\linethickness}\end{center}

\pagebreak

\hypertarget{molecular-pathology}{%
\subsection{Molecular Pathology}\label{molecular-pathology}}

\begin{center}\rule{0.5\linewidth}{\linethickness}\end{center}

\hypertarget{pancreas-2}{%
\subsubsection{Pancreas}\label{pancreas-2}}

\begin{center}\rule{0.5\linewidth}{\linethickness}\end{center}

\hypertarget{pancreas-3}{%
\paragraph{Pancreas}\label{pancreas-3}}

\begin{center}\rule{0.5\linewidth}{\linethickness}\end{center}

- \textbf{The Proteomic Landscape of Pancreatic Ductal Adenocarcinoma
Liver Metastases Identifies Molecular Subtypes and Associations with
Clinical Response}

\emph{Clinical cancer research : an official journal of the American
Association for Cancer Research 2019 Dec;():}

PubMed: \url{https://www.ncbi.nlm.nih.gov/pubmed/?term=31848187}

PURPOSE: Pancreatic ductal adenocarcinoma (PDAC) is a highly metastatic
disease that can be separated into distinct subtypes based on molecular
signatures. Identifying PDAC subtype-specific therapeutic
vulnerabilities is necessary to develop precision medicine approaches to
treat PDAC. EXPERIMENTAL DESIGN: 56 PDAC liver metastases were obtained
from the UNMC Rapid Autopsy Program and analyzed with quantitative
proteomics. PDAC subtypes were identified by Principal Component
Analysis based on protein expression profiling. Proteomic subtypes were
further characterized by the associated clinical information, including
but not limited to survival analysis, drug treatment response, smoking
and drinking status. RESULTS: Over 3960 proteins were identified and
used to delineated 4 distinct PDAC microenvironment subtypes: (1)
Metabolic; (2) Progenitor-like; (3) Proliferative; and (4) Inflammatory.
PDAC risk factors of alcohol and tobacco consumption correlate with
subtype classifications. Enhanced survival is observed in FOLFIRINOX
treated Metabolic and Progenitor-like subtypes compared to the
Proliferative and Inflammatory subtypes. In addition, TYMP, PDCD6IP,
ERAP1, and STMN showed significant association with patient survival in
a subtype-specific manner. Gemcitabine-induced alterations in the
proteome identify proteins, such as SHMT1, associated with drug
resistance. CONCLUSIONS: These data demonstrate that proteomic analysis
of clinical PDAC liver metastases can identify molecular signatures
unique to disease subtypes and point to opportunities for therapeutic
development to improve the treatment of PDAC.

doi: \url{https://doi.org/10.1158/1078-0432.CCR-19-1496}

\begin{center}\rule{0.5\linewidth}{\linethickness}\end{center}

- \textbf{Germline DNA Sequencing Reveals Novel Mutations Predictive of
Overall Survival in a Cohort of Pancreatic Cancer Patients}

\emph{Clinical cancer research : an official journal of the American
Association for Cancer Research 2019 Dec;():}

PubMed: \url{https://www.ncbi.nlm.nih.gov/pubmed/?term=31871297}

Background and Aims Family history of BRCA-related tumors may correlate
with response to chemotherapy and overall survival (OS) in pancreatic
cancer. The frequency of germline mutations has been reported in
patients predominantly under the age of 60 or with strong family
history. We examine the incidence of deleterious germline mutations and
compare the chemotherapy responses and OS in an unselected group of
metastatic pancreatic cancer patients. Methods Metastatic pancreatic
cancer patients, who were seen at a single cancer center between 2010
and 2016, were included. Germline DNA was sequenced using a 263-gene
panel to identify novel mutations (N = 133 MD Anderson cohort, N = 127
TCGA cohort). Chemotherapy response and OS were determined by review of
medical records. Results Deleterious germline mutations were identified
in 26 of 133 patients (19.5\%). Patients with DNA damage repair (DDR)
gene mutations (ATM, BRCA1/2, CDKN2A, CHEK2, ERCC4, PALB2, n = 15) had
an improved OS as compared to patients without (16.8 versus 9.1 months,
P = 0.03). Conversely, patients with other deleterious mutations had a
trend towards worse OS. Although, survival in the later group was longer
(P= NS) in those mutants initially treated with
gemcitabine/nab-paclitaxel. A family history of multiple breast,
ovarian, and pancreatic cancers was associated with DDR gene mutations
and better survival. Conclusion We have identified novel germline
mutations that are prognostic for survival in pancreatic cancer
patients. We observe improved survival in patients with DDR gene
mutations and worsened survival in patients with deleterious mutations
in non-DDR genes.

doi: \url{https://doi.org/10.1158/1078-0432.CCR-19-0224}

\begin{center}\rule{0.5\linewidth}{\linethickness}\end{center}

- \textbf{RET gene rearrangements occur in a subset of pancreatic acinar
cell carcinomas}

\emph{Modern pathology : an official journal of the United States and
Canadian Academy of Pathology, Inc 2019 Sep;():}

PubMed: \url{https://www.ncbi.nlm.nih.gov/pubmed/?term=31558784}

Pancreatic acinar cell carcinoma is relatively rare (1 to 2\% of
pancreatic malignancies) but may be under-recognized. In contrast to
pancreatic ductal adenocarcinoma, most acinar cell carcinomas lack
mutations in KRAS, DPC, CDKN2A or TP53, but appear to have a high
incidence of gene rearrangements, with up to 20\% reported to be driven
by BRAF fusions. With the development of a new class of RET-specific
tyrosine kinase inhibitors, which appear to have particularly strong
activity against RET gene rearranged tumours, there is now considerable
interest in identifying RET gene rearrangements across a wide range of
cancers. RET rearrangements have been reported to occur at a very low
incidence (\textless1\%) in all pancreatic carcinomas. We postulated
that given its unique molecular profile, RET gene rearrangements may be
common in acinar cell carcinomas. We performed fluorescent in-situ
hybridization (FISH) studies on a cohort of 40 acinar cell spectrum
tumours comprising 36 pure acinar cell carcinomas, three
pancreatoblastomas and one mixed acinar-pancreatic neuroendocrine
tumour. RET gene rearrangements were identified in 3 (7.5\%) cases and
BRAF gene rearrangements in 5 (12.5\%). All gene rearranged tumours were
pure acinar cell carcinomas. Our findings indicate that amongst all
pancreatic carcinomas, acinar carcinomas are highly enriched for
potentially actionable gene rearrangements in RET or BRAF. FISH testing
is inexpensive and readily available in the routine clinical setting and
may have a role in the assessment of all acinar cell carcinomas-at this
stage to recruit patients for clinical trials of new targeted therapies,
but perhaps in the near future as part of routine care.

doi: \url{https://doi.org/10.1038/s41379-019-0373-y}

\begin{center}\rule{0.5\linewidth}{\linethickness}\end{center}

Back to top

\begin{center}\rule{0.5\linewidth}{\linethickness}\end{center}

\pagebreak

\hypertarget{spn}{%
\paragraph{SPN}\label{spn}}

Solid Pseudopapillary Neoplasm

\begin{center}\rule{0.5\linewidth}{\linethickness}\end{center}

\begin{itemize}
\tightlist
\item
  Targeted next generation sequencing of pancreatic solid
  pseudopapillary neoplasms show mutations in Wnt signaling pathway
  genes.
\end{itemize}

\url{https://www.ncbi.nlm.nih.gov/pubmed/30811747}

\begin{center}\rule{0.5\linewidth}{\linethickness}\end{center}

Back to top

\begin{center}\rule{0.5\linewidth}{\linethickness}\end{center}

\pagebreak

\hypertarget{preneoplastic-and-preinvasive-lesions}{%
\paragraph{Preneoplastic and Preinvasive
Lesions}\label{preneoplastic-and-preinvasive-lesions}}

Molecular Pathology Preneoplastic and Preinvasive Lesions, PanIN, IPMN,
MCN, ICPN, IOPN

\begin{center}\rule{0.5\linewidth}{\linethickness}\end{center}

- \textbf{Recurrent Rearrangements in PRKACA and PRKACB in Intraductal
Oncocytic Papillary Neoplasms of the Pancreas and Bile Duct}

\emph{Gastroenterology 2019 Oct;():}

PubMed: \url{https://www.ncbi.nlm.nih.gov/pubmed/?term=31678302}

BACKGROUND \& AIMS: Intraductal oncocytic papillary neoplasms (IOPNs) of
the pancreas and bile duct contain epithelial cells with numerous, large
mitochondria and are cystic precursors to pancreatic ductal
adenocarcinoma (PDAC) and cholangiocarcinoma (CCA), respectively.
However, IOPNs do not have the genomic alterations found in other
pancreatobiliary neoplasms. In fact, no recurrent genomic alterations
have been described in IOPNs. PDACs without activating mutations in KRAS
contain gene rearrangements, so we investigated whether IOPNs have
recurrent fusions in genes. METHODS: We analyzed 20 resected pancreatic
IOPNs and 3 resected biliary IOPNs using a broad RNA-based targeted
sequencing panel to detect cancer-related fusion genes. Four invasive
PDACs and 2 intrahepatic cholangiocarcinomas from the same patients as
the IOPNs, were also available for analysis. Samples of pancreatic cyst
fluid (n=5, collected before surgery) and bile duct brushings (n=2) were
analyzed for translocations. For comparison, we analyzed
pancreatobiliary lesions from 126 patients without IOPN (controls).
RESULTS: All IOPNs evaluated were found to have recurring fusions of
ATP1B1-PRKACB (n = 13), DNAJB1-PRKACA (n = 6), or ATP1B1-PRKACA (n = 4).
These fusions were also found in corresponding invasive PDACs and
intrahepatic cholangiocarcinomas, as well as in matched pancreatic cyst
fluid and bile duct brushings. These gene rearrangements were absent
from all 126 control pancreatobiliary lesions. CONCLUSIONS: We
identified fusions in PRKACA and PRKACB genes in pancreatic and biliary
IOPNs, as well as in PDACs and pancreatic cyst fluid and bile duct cells
from the same patients. We did not identify these gene fusions in 126
control pancreatobiliary lesions. These fusions might be used to
identify patients at risk for IOPNs and their associated invasive
carcinomas.

doi: \url{https://doi.org/10.1053/j.gastro.2019.10.028}

\begin{center}\rule{0.5\linewidth}{\linethickness}\end{center}

- \textbf{Multiple KRAS mutations in the non-mucinous epithelial lining
in the majority of mucinous cystic neoplasms of the pancreas}

\emph{Histopathology 2019 Oct;75(4):559-567}

PubMed: \url{https://www.ncbi.nlm.nih.gov/pubmed/?term=31077597}

AIMS: Mucinous cystic neoplasms (MCNs) of the pancreas are cystic
neoplasms lined by mucinous lining epithelium (MLE) with associated
ovarian-type stroma. Although a non-MLE (NMLE) can be observed in some
MCNs, whether cystic neoplasms with ovarian-type stroma and NMLE should
be classified as MCNs or separately designated is debated. METHODS AND
RESULTS: To test this, NMLEs were defined as flat or cuboidal epithelial
cells without intracytoplasmic mucin. A total of 112 MCNs were reviewed,
and the epithelium was classified as NMLE or MLE. A total of 110 females
and two males with a mean age of 46.5~±~12.3~years were included in this
study. At least focal NMLE was noted in 76.8\% (86/112) of MCNs. The
mean percentage of the neoplastic epithelium that was NMLE in these 86
cases was 46\%. NMLE was predominant (\textgreater50\%) in 38.4\%
(43/112) of cases. MCNs with NMLE were smaller (42~±~21~mm) than those
with MLE (60~±~36~mm, P~\textless~0.001), and all NMLEs had low-grade
dysplasia. Twelve MCNs with NMLE or MLE were selected for KRAS mutation
analysis with droplet digital polymerase chain reaction after laser
capture microdissection. All 12 MCNs showed multiple types of KRAS
mutation, which were detected in 92\% (11/12) of NMLE foci and 89\%
(8/9) of MLE foci. Predominant NMLE was common in small MCNs with
low-grade dysplasia. CONCLUSIONS: Clonal KRAS mutations were observed in
both NMLE and MLE, supporting the hypothesis that MCNs with NMLE should
be classified as MCNs.

doi: \url{https://doi.org/10.1111/his.13897}

\begin{center}\rule{0.5\linewidth}{\linethickness}\end{center}

Back to top

\begin{center}\rule{0.5\linewidth}{\linethickness}\end{center}

\pagebreak

\hypertarget{bile-ducts-1}{%
\subsubsection{Bile Ducts}\label{bile-ducts-1}}

\begin{center}\rule{0.5\linewidth}{\linethickness}\end{center}

- \textbf{Targeted next-generation sequencing identifies distinct
clinicopathologic and molecular entities of intraductal papillary
neoplasms of the bile duct}

\emph{Modern pathology : an official journal of the United States and
Canadian Academy of Pathology, Inc 2019 Nov;32(11):1637-1645}

PubMed: \url{https://www.ncbi.nlm.nih.gov/pubmed/?term=31231124}

Intraductal papillary neoplasm of the bile duct (IPNB) is a mass-forming
neoplasm in the bile duct considered to be the biliary counterpart of
pancreatic intraductal papillary mucinous neoplasm (IPMN). By its cell
lineage, IPNB can be classified into gastric, intestinal,
pancreatobiliary, and oncocytic types. Recently, a group of Japanese and
Korean pathologists proposed that IPNB be classified into two types,
with type 1, being the histological counterpart of IPMN and type 2,
having a more complex histological architecture. We used targeted
next-generation sequencing to study the molecular change of 37 IPNBs and
identified frequent mutations of KRAS (49\%), GNAS (32\%), RNF43 (24\%),
APC (24\%), TP53 (24\%), and CTNNB1 (11\%) in IPNBs. Intestinal-type
IPNB was associated with KRAS, GNAS, and RNF43 mutations. Japan-Korea
consensus type 1 was associated with KRAS and GNAS mutations. All four
IPNBs with CTNNB1 mutations were of pancreatobiliary type and located in
the extrahepatic bile duct. A hierarchical analysis identified three
distinct groups within IPNB: group 1 was Japan-Korea consensus type 1
tumors with macroscopic mucin, old age, and frequent KRAS, GNAS, and
RNF43 mutations. Group 2 was Japan-Korea consensus type 2 with
intestinal differentiation and frequent KRAS mutation but rare GNAS
mutation, MUC2 expression, and macroscopic mucin. Group 3 was
characterized by CTNNB1 mutation, extrahepatic location, lack of
expression of intestinal markers, Japan-Korea consensus type 2, and lack
of mutations in KRAS, APC, RNF43, and GNAS. Our results indicated that
IPNB is a heterogeneous disease and that the activation of
Ras-mitogen-activated protein kinase (MAPK), Wnt/β-catenin, and
G-protein-coupled receptor (GPCR)-cAMP signaling is the main oncogenic
mechanism of IPNB.

doi: \url{https://doi.org/10.1038/s41379-019-0306-9}

\begin{center}\rule{0.5\linewidth}{\linethickness}\end{center}

Back to top

\begin{center}\rule{0.5\linewidth}{\linethickness}\end{center}

\pagebreak

\hypertarget{feedback}{%
\section{Feedback}\label{feedback}}

\href{https://docs.google.com/forms/d/e/1FAIpQLSeD3Z9J6Y7eMmiyM12f_SfAmHUlykb1zxZcwO6lg7cebGYQIQ/viewform}{Google
Feedback Form}

\hypertarget{disqus_thread}{}

Please enable JavaScript to view the comments powered by Disqus.

\begin{center}\rule{0.5\linewidth}{\linethickness}\end{center}

Back to top

\begin{center}\rule{0.5\linewidth}{\linethickness}\end{center}

\pagebreak

\end{document}
