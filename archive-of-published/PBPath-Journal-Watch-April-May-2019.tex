\documentclass[]{article}
\usepackage{lmodern}
\usepackage{amssymb,amsmath}
\usepackage{ifxetex,ifluatex}
\usepackage{fixltx2e} % provides \textsubscript
\ifnum 0\ifxetex 1\fi\ifluatex 1\fi=0 % if pdftex
  \usepackage[T1]{fontenc}
  \usepackage[utf8]{inputenc}
\else % if luatex or xelatex
  \ifxetex
    \usepackage{mathspec}
  \else
    \usepackage{fontspec}
  \fi
  \defaultfontfeatures{Ligatures=TeX,Scale=MatchLowercase}
\fi
% use upquote if available, for straight quotes in verbatim environments
\IfFileExists{upquote.sty}{\usepackage{upquote}}{}
% use microtype if available
\IfFileExists{microtype.sty}{%
\usepackage{microtype}
\UseMicrotypeSet[protrusion]{basicmath} % disable protrusion for tt fonts
}{}
\usepackage[margin=1in]{geometry}
\usepackage{hyperref}
\hypersetup{unicode=true,
            pdftitle={Pancreatobiliary Pathology Society Journal Watch},
            pdfborder={0 0 0},
            breaklinks=true}
\urlstyle{same}  % don't use monospace font for urls
\usepackage{graphicx,grffile}
\makeatletter
\def\maxwidth{\ifdim\Gin@nat@width>\linewidth\linewidth\else\Gin@nat@width\fi}
\def\maxheight{\ifdim\Gin@nat@height>\textheight\textheight\else\Gin@nat@height\fi}
\makeatother
% Scale images if necessary, so that they will not overflow the page
% margins by default, and it is still possible to overwrite the defaults
% using explicit options in \includegraphics[width, height, ...]{}
\setkeys{Gin}{width=\maxwidth,height=\maxheight,keepaspectratio}
\IfFileExists{parskip.sty}{%
\usepackage{parskip}
}{% else
\setlength{\parindent}{0pt}
\setlength{\parskip}{6pt plus 2pt minus 1pt}
}
\setlength{\emergencystretch}{3em}  % prevent overfull lines
\providecommand{\tightlist}{%
  \setlength{\itemsep}{0pt}\setlength{\parskip}{0pt}}
\setcounter{secnumdepth}{0}
% Redefines (sub)paragraphs to behave more like sections
\ifx\paragraph\undefined\else
\let\oldparagraph\paragraph
\renewcommand{\paragraph}[1]{\oldparagraph{#1}\mbox{}}
\fi
\ifx\subparagraph\undefined\else
\let\oldsubparagraph\subparagraph
\renewcommand{\subparagraph}[1]{\oldsubparagraph{#1}\mbox{}}
\fi

%%% Use protect on footnotes to avoid problems with footnotes in titles
\let\rmarkdownfootnote\footnote%
\def\footnote{\protect\rmarkdownfootnote}

%%% Change title format to be more compact
\usepackage{titling}

% Create subtitle command for use in maketitle
\providecommand{\subtitle}[1]{
  \posttitle{
    \begin{center}\large#1\end{center}
    }
}

\setlength{\droptitle}{-2em}

  \title{Pancreatobiliary Pathology Society Journal Watch}
    \pretitle{\vspace{\droptitle}\centering\huge}
  \posttitle{\par}
  \subtitle{April May 2019}
  \author{}
    \preauthor{}\postauthor{}
      \predate{\centering\large\emph}
  \postdate{\par}
    \date{Last Update on 2019-08-05}


\begin{document}
\maketitle

{
\setcounter{tocdepth}{5}
\tableofcontents
}
\begin{center}\rule{0.5\linewidth}{\linethickness}\end{center}

\hypertarget{pbpath-journal-watch-articles}{%
\section{PBPath Journal Watch
Articles}\label{pbpath-journal-watch-articles}}

\begin{center}\rule{0.5\linewidth}{\linethickness}\end{center}

\textbf{Wellcome to the PBPath Journal Watch!}

This bi-monthly journal watch features exciting recently published
pancreas and biliary pathology articles that will provide up to date
medical knowledge in our field. These articles will be showcased in
several convenient categories, including surgical pathology,
cytopathology, and molecular pathology among others. The articles in
each category are in no particular order. See the list of journals we
search regularly \href{http://pbpath.org/pbpath-journal-watch/}{here}.
Previous months' issues may be found in our
\emph{\href{http://pbpath.org/journal-watch-archive/}{archive}} and you
may see
\href{http://pbpath.org/journal-watch-upcoming-issue/}{preparation of
upcoming issue here}.

We encourage members to actively participate by recommending new
articles and providing feedback using the forms provided below.

\textbf{We hope that you will enjoy the new PBPath Journal Watch!}

\begin{center}\rule{0.5\linewidth}{\linethickness}\end{center}

\textbf{Feedback}

Please send your feedbacks using the forms below:

\href{https://docs.google.com/forms/d/e/1FAIpQLSeD3Z9J6Y7eMmiyM12f_SfAmHUlykb1zxZcwO6lg7cebGYQIQ/viewform}{Google
Feedback Form}

\hypertarget{disqus_thread}{}

Please enable JavaScript to view the comments powered by Disqus.

\begin{center}\rule{0.5\linewidth}{\linethickness}\end{center}

Back to top

\begin{center}\rule{0.5\linewidth}{\linethickness}\end{center}

\pagebreak

\hypertarget{surgical-pathology}{%
\subsection{Surgical Pathology}\label{surgical-pathology}}

\begin{center}\rule{0.5\linewidth}{\linethickness}\end{center}

\hypertarget{pancreas}{%
\subsubsection{Pancreas}\label{pancreas}}

\begin{center}\rule{0.5\linewidth}{\linethickness}\end{center}

\hypertarget{morphology-diagnostics-ihc}{%
\paragraph{Morphology, Diagnostics,
IHC}\label{morphology-diagnostics-ihc}}

Morphology, Diagnostics, IHC

\begin{center}\rule{0.5\linewidth}{\linethickness}\end{center}

 - \textbf{Pancreatic Nerve Sheath Tumors: a Single Institutional Series
and Systematic Review of the Literature}

\emph{Journal of gastrointestinal surgery : official journal of the
Society for Surgery of the Alimentary Tract 2019 Apr;():}

PubMed: \url{https://www.ncbi.nlm.nih.gov/pubmed/?term=30941687}

INTRODUCTION: Improvement in imaging has resulted in frequent diagnosis
of benign and premalignant pancreatic tumors. Pancreatic nerve sheath
(PNS) tumors are one of the rarest pancreatic tumors. Literature on PNS
is limited and their biology is poorly understood. Here, we report the
largest series of PNS tumors to date and review the literature to
evaluate the current data available on PNS tumors. METHODS: An
institutional database was used to identify patients who underwent
resection for PNS tumors. Clinicopathological characteristics and
outcomes of these patients were reported. Furthermore, a review of
literature was performed. RESULTS: From January 1994 through December
2016, seven patients underwent resection for PNS tumors. The median age
was 57.7~years (IQR, 44.9-61.9) and the sex was approximately equally
distributed (male = 4; 57.1\%). Three (42.9\%) patients were diagnosed
incidentally and six (85.7\%) were misdiagnosed as having other
pancreatic tumors. The median tumor size was 2.1 (IQR 1.8-3.0) cm and
six (85.7\%) had no nodal disease. At a median follow-up of 15.5 (IQR
13.7-49.3) months, six patients were alive without evidence of disease
and one patient was lost to follow-up. The literature review identified
49 studies reporting 54 patients with PNS tumors. Forty-six were
misdiagnosed as having other pancreatic tumors. The median tumor size
was 3.6 (range 1-20) cm, nodal disease was present in six patients
(22.2\%), and no patient had distant metastatic disease. At the time of
last follow-up, all patients were free of disease. CONCLUSION: This is
the largest single institution series on PNS tumors reported to date.
These tumors are rare and are often misdiagnosed, given their
radiological characteristics. PNS tumors have a benign course of disease
and surgical resection results in favorable long-term outcomes.

{}

{}

\begin{center}\rule{0.5\linewidth}{\linethickness}\end{center}

 - \textbf{Primary Extranodal Rosai-Dorfman Disease (Sinus Histiocytosis
With Massive Lymphadenopathy) in the Pancreatic Tail: A Case Report With
Literature Review}

\emph{Pancreas 2019 04;48(4):e31-e33}

PubMed: \url{https://www.ncbi.nlm.nih.gov/pubmed/?term=30973472}

{}

{}

\begin{center}\rule{0.5\linewidth}{\linethickness}\end{center}

 - \textbf{Challenges in Diagnosis and Management of Pancreatic
Inflammatory Myofibroblastic Tumors in Children}

\emph{Pancreas 2019 04;48(4):e27-e29}

PubMed: \url{https://www.ncbi.nlm.nih.gov/pubmed/?term=30973469}

{}

{}

\begin{center}\rule{0.5\linewidth}{\linethickness}\end{center}

 - \textbf{Incidence and Significance of GATA3 Positivity in Pancreatic
Ductal Adenocarcinoma and Cholangiocarcinoma}

\emph{Applied immunohistochemistry \& molecular morphology : AIMM 2019
Apr;():}

PubMed: \url{https://www.ncbi.nlm.nih.gov/pubmed/?term=31033498}

GATA3 is a transcription factor involved in the development and
differentiation of lymphocytes, breast, and hair follicles. The protein
is a useful immunohistochemical (IHC) marker for supporting diagnoses of
breast or urothelial carcinoma. This can be especially helpful in
metastatic neoplasms to help delineate site of origin. GATA3 is also
reportedly positive in a percentage of pancreatic ductal adenocarcinomas
(PDACs) and cholangiocarcinomas (CCs), but no study has closely
evaluated this relationship with respect to clininopathologic features
or patient outcome. Using tissue microarrays, we analyzed 240 PDACs and
60 CCs with GATA3 IHC and compared expression to various clinical and
pathologic parameters. Overall, GATA3 positivity was seen in 16\% of
PDACs and 5\% of CCs. GATA3 positivity in PDAC cases was more common in
male patients (P=0.013). GATA3-positive PDACs trended toward worse
survival on multivariate analysis (P=0.074). The only 3 GATA3-positive
CCs were poorly differentiated (P=0.069); low case number precluded
multivariate survival analysis for CCs. GATA3 positivity can occur in
carcinomas of the pancreatobiliary system, which should be considered
during IHC workup of neoplasms of unclear origin. This positivity seems
to have minimal relevance to patient outcome.

{}

{}

\begin{center}\rule{0.5\linewidth}{\linethickness}\end{center}

 - \textbf{Loss of SMAD4 Protein Expression in Gastrointestinal and
Extra-Gastrointestinal Carcinomas}

\emph{Histopathology 2019 May;():}

PubMed: \url{https://www.ncbi.nlm.nih.gov/pubmed/?term=31054158}

BACKGROUND: SMAD4 (DPC4) is a tumor suppressor gene that is dysregulated
in various tumor types, particularly pancreaticobiliary and
gastrointestinal carcinomas. Corresponding loss of protein expression
has been reported in approximately 50\% of pancreatic and 25\% of
colonic adenocarcinomas. In the evaluation of carcinoma of unknown
primary site, immunohistochemical loss of SMAD4 expression is often used
to suggest pancreaticobiliary origin, but there is limited data on the
spectrum of SMAD4 expression in carcinomas of other sites. This study
evaluates the frequency of SMAD4 loss in a large cohort of carcinomas
from diverse anatomic sites. DESIGN: Immunohistochemistry for SMAD4 was
performed on tissue microarrays or whole tissue sections of 1210
carcinomas from various organs: gastrointestinal tract, liver,
pancreas/biliary tract, lung, breast, thyroid, kidney, ovary, and
uterus. Expression was considered lost when there was complete absence
of staining in tumor cell nuclei, in the presence of intact staining in
non-neoplastic cells. RESULTS: SMAD4 loss was seen in 58\% of pancreatic
adenocarcinomas, 27\% of appendiceal adenocarcinomas, 19\% of colorectal
adenocarcinomas, 16\% of cholangiocarcinomas, 10\% of lung
adenocarcinomas, and \textless{}5\% of esophageal, breast, gastric and
mucinous ovarian adenocarcinomas. All papillary thyroid, hepatocellular,
non-mucinous ovarian, endometrial, and renal cell carcinomas showed
intact SMAD4 nuclear expression. CONCLUSION: In addition to
pancreaticobiliary, appendiceal and colonic tumors, SMAD4 loss is also
seen in a small subset of other carcinomas, specifically breast, lung,
esophageal and gastric adenocarcinomas, all of which are typically CK7
positive, similar to pancreaticobiliary carcinoma. Awareness of SMAD4
loss in these other carcinoma types is helpful in the evaluation of
carcinomas of unknown primary site. This article is protected by
copyright. All rights reserved.

{}

{}

\begin{center}\rule{0.5\linewidth}{\linethickness}\end{center}

 - \textbf{Expression and prognostic value of NSD1 and SETD2 in
pancreatic ductal adenocarcinoma and its precursor lesions}

\emph{Pathology 2019 Jun;51(4):392-398}

PubMed: \url{https://www.ncbi.nlm.nih.gov/pubmed/?term=31060750}

Epigenetic regulation is emerging as a critical mechanism for pancreatic
ductal adenocarcinoma (PDA) development. Histone methylation is an
important regulatory mechanism, altering chromatin structure and
promoter accessibility and causing aberrant gene expression. NSD1 and
SETD2 genes encoding two histone H3K36 methyltransferases, are mutated
or altered in 8-10\% of PDA cases. However, whether there is altered
protein expression of NSD1 or SETD2 in PDA and its precursors, and
whether they have diagnostic or prognostic utility is unknown. Tissue
microarrays composed of a total of 190 and 192 duplicated cases of PDA
(n=74 and 75), metastatic PDA (n=17 and 18), pancreatic intraepithelial
neoplasia (PanIN; n=19 and 24), intraductal papillary mucinous neoplasm
(IPMN; n=36), mucinous cystic neoplasm (MCN; n=12) and benign pancreatic
tissues (n=27 and 32) were analysed for expression of NSD1 and SETD2 by
immunohistochemistry. We assessed intensity and percentage of positive
cells. NSD1 expression was significantly increased in metastatic PDA
compared to benign ducts, primary PDA, and all other lesions combined
(p=0.03, 0.02, and 0.03 respectively). Additionally, significantly
decreased SETD2 protein expression was found in metastatic PDA and PanIN
lesions compared to benign ducts (p=0.04 and 0.007, respectively). High
NSD1 expression was associated with clinical stage III/IV disease
(p=0.026), tumour grade 2 (p=0.022), use of neoadjuvant therapy
(p=0.037), and overall higher clinical stage (p=0.022). There is no
significant difference in overall and progression-free survival between
NSD1/SETD2 high and low PDA. Expression of NSD1 and SETD2 is
specifically altered in metastatic PDA and some of the PDA precursor
lesions, supporting their important role in PDA development and
metastasis. In addition, increased NSD1 expression is significantly
associated with higher clinical stage and neoadjuvant therapy,
suggesting that NSD1 may be a useful prognostic marker.

{}

{}

\begin{center}\rule{0.5\linewidth}{\linethickness}\end{center}

Back to top

\begin{center}\rule{0.5\linewidth}{\linethickness}\end{center}

\pagebreak

\hypertarget{staging}{%
\paragraph{Staging}\label{staging}}

Pancreas TNM staging, Margins, Survival

\begin{center}\rule{0.5\linewidth}{\linethickness}\end{center}

 - \textbf{Tumor location as an indicator of survival in T1 resectable
pancreatic ductal adenocarcinoma: a propensity score-matched analysis}

\emph{BMC gastroenterology 2019 Apr;19(1):59}

PubMed: \url{https://www.ncbi.nlm.nih.gov/pubmed/?term=31014264}

BACKGROUND: The latest 8th edition of the AJCC staging system emphasizes
the importance of tumor size however, the clinical significance of the
combination of tumor location with tumor size remains unknown. METHODS:
We conducted this study to investigate the prognostic role of tumor
location in T1 resectable pancreatic ductal adenocarcinoma (PDAC).
Resectable PDAC patients from Surveillance, Epidemiology, and End
Results (SEER) database (2004-2014) were selected for the propensity
score matching analysis. We used matched cohort to analyze the
relationship between clinicopathologic features and survival of
patients. RESULT: Eight thousand, four hundred nine patients were
included in the propensity score matching analysis and 4571 patients
were selected for final analysis. In T1 patients, the patients with
pancreatic head tumor had worse prognosis compared to the patients with
body/tail tumors. Multivariate analysis result showed that pancreatic
body/tail location was an independent indicator for better chances of
survival in T1 PDAC patients (hazard ratio, 0.69; 95\%CI, 0.52-0.93;
P = 0.01). The modified staging system was more efficient than the AJCC
8th staging system. CONCLUSION: Modified staging system exhibited a good
assessment of the survival rate. The tumor location is a good prognostic
indicator for T1 resectable PDAC patients. Modification of T1 subgroup
according to tumor location exhibited favorable survival prediction
effects.

{}

{}

\begin{center}\rule{0.5\linewidth}{\linethickness}\end{center}

 - \textbf{A Single-Institution Validation Study of Lymph Node Staging
By the AJCC 8th Edition for Patients with Pancreatic Head Cancer: A
Proposal to Subdivide the N2 Category}

\emph{Annals of surgical oncology 2019 Jul;26(7):2112-2120}

PubMed: \url{https://www.ncbi.nlm.nih.gov/pubmed/?term=31037440}

BACKGROUND: According to the revised staging of the American Joint
Committee on Cancer, 8th edition (AJCC8), the N category in pancreatic
ductal adenocarcinoma is classified as N0 (0), N1 (1-3), and N2 (≥ 4)
based on the number of metastatic lymph nodes (LNs). This study aimed to
validate this classification and analyze cutoff values of metastatic LN
numbers. METHODS: Patients with pancreatic head ductal adenocarcinoma
who underwent pancreaticoduodenectomy at our institution between 2005
and 2016 without preoperative therapy were retrospectively analyzed. The
patients were staged by AJCC8, and prognostic analyses were performed.
The best cutoff value for the metastatic LN number was determined by the
minimum P value approach. RESULTS: In 228 of 309 patients, LN metastases
were found (median number of examined LNs, 41). The median survival time
(MST) was 56~months in the N0 group, 34~months in the N1 group, and
20~months in the N2 group (N0 vs N1: P = 0.023; N1 vs N2:
P \textless{} 0.001). The best cutoff number of metastatic LNs was 4 for
patients with LN metastases and 7 for patients with N2 disease. The MST
for patients with four to six positive nodes (N2a) was significantly
longer than for those with seven or more positive nodes (N2b) (24.0 vs
19.1~months: P = 0.012). For N2b patients, conventional adjuvant
chemotherapy did not show survival benefits (P = 0.133), and overall
survival did not differ significantly from that for patients with
para-aortic LN metastasis (P = 0.562). CONCLUSION: The N staging of
AJCC8 was valid. Clinicians should regard N2b as similar to distant LN
metastasis, and more intensive adjuvant therapy may be indicated for
this group.

{}

{}

\begin{center}\rule{0.5\linewidth}{\linethickness}\end{center}

 - \textbf{Outcome of head compared to body and tail pancreatic cancer:
a systematic review and meta-analysis of 93 studies}

\emph{Journal of gastrointestinal oncology 2019 Apr;10(2):259-269}

PubMed: \url{https://www.ncbi.nlm.nih.gov/pubmed/?term=31032093}

Background: Even when resectable pancreatic cancer (PC) is associated
with a dismal prognosis. Initial presentation varies according with
primary tumor location. Aim of this systematic review and meta-analysis
was to evaluate the prognosis associated with site (head versus
body/tail) in patients with PC. Methods: We searched PubMed, Cochrane
Library, SCOPUS, Web of Science, EMBASE, Google Scholar, LILACS, and
CINAHL databases from inception to March 2018. Studies reporting
information on the independent prognostic role of site in PC and
comparing overall survival (OS) in head versus body/tail tumors were
selected. Data were aggregated using hazard ratios (HRs) for OS of head
versus body/tail PC according to fixed- or random-effect model. Results:
A total of 93 studies including 254,429 patients were identified.
Long-term prognosis of head was better than body/tail cancers (HR =0.96,
95\% CI: 0.92-0.99; P=0.02). A pooled HR of 0.95 (95\% CI: 0.92-0.99,
P=0.02) from multivariate analysis only (n=77 publications) showed that
head site was an independent prognostic factor for survival.
Conclusions: Primary tumor location in the head of the pancreas at the
time of diagnosis is a predictor of better survival. Such indicator
should be acknowledged when designing future studies, in particular in
the operable and neoadjuvant setting.

{}

{}

\begin{center}\rule{0.5\linewidth}{\linethickness}\end{center}

Back to top

\begin{center}\rule{0.5\linewidth}{\linethickness}\end{center}

\pagebreak

\hypertarget{preneoplastic-and-preinvasive-lesions}{%
\paragraph{Preneoplastic and Preinvasive
Lesions}\label{preneoplastic-and-preinvasive-lesions}}

Preneoplastic and Preinvasive Lesions, PanIN, IPMN, MCN, ICPN

\begin{center}\rule{0.5\linewidth}{\linethickness}\end{center}

 - \textbf{A Consensus Study of the Grading and Typing of Intraductal
Papillary Mucinous Neoplasms of the Pancreas}

\emph{Pancreas 2019 04;48(4):480-487}

PubMed: \url{https://www.ncbi.nlm.nih.gov/pubmed/?term=30946243}

OBJECTIVE: The grading and typing of intraductal papillary mucinous
neoplasms (IPMNs) of the pancreas are challenging for pathologists. We
aimed to clarify the points of consistency and disagreement in assessing
the grades and types of IPMNs. METHODS: Digital slide images of 20 IPMNs
were independently assessed by 10 Japanese pathologists, who then held a
consensus meeting to discuss the points of disagreement and develop a
consensus and recommendations. RESULTS: The average agreement rates for
grade and type were 83.5\% (range, 100\%-40\%) and 82.5\% (range,
100\%-50\%) and the Fleiss' κ values were 0.567 and 0.636, respectively.
CONCLUSIONS: The disagreement points and recommendations were as
follows: destructed ductal walls with desquamated neoplastic epithelia
or mucin lakes partially lined with neoplastic cells could be invasion;
intraductal stromal invasion could be dismissed unless vascular or
lymphatic invasion existed; elastica staining may help visualize ducts
in colloidal nodules; high-grade can be distinguished from
low/intermediate grade by marked nuclear disarrangements and complex
architecture in the intestinal papillae; oncocytic papillae are
characterized by eosinophilic cells with round disoriented nuclei;
high-grade gastric papillae can be distinguished from pancreatobiliary
papillae by relatively low but complex architecture; and the most
dysplastic papillae should be used to assess type in mixed papillae
types.

{}

{}

\begin{center}\rule{0.5\linewidth}{\linethickness}\end{center}

 - \textbf{Clinical assessment of the GNAS mutation status in patients
with intraductal papillary mucinous neoplasm of the pancreas}

\emph{Surgery today 2019 Mar;():}

PubMed: \url{https://www.ncbi.nlm.nih.gov/pubmed/?term=30879148}

Intraductal papillary mucinous neoplasm (IPMN) of the pancreas is
characterized by cystic dilation of the pancreatic duct, caused by mucin
hypersecretion, with slow progression via the adenoma-carcinoma sequence
mechanism. Mutation of GNAS at codon 201 is found exclusively in IPMNs,
occurring at a rate of 41-75\%. Recent advances in molecular biological
techniques have demonstrated that GNAS mutation might play a role in the
transformation of IPMNs after the appearance of neoplastic cells, rather
than in the tumorigenesis of IPMNs. GNAS mutation is observed frequently
in the intestinal subtype of IPMNs with MUC2 expression, and less
frequently in IPMNs with concomitant pancreatic ductal adenocarcinoma
(PDAC). Research has focused on assessing GNAS mutation status in
clinical practice using various samples. In this review, we discuss the
clinical application of GNAS mutation assessment to differentiate
invasive IPMNs from concomitant PDAC, examine the clonality of recurrent
IPMNs in the remnant pancreas using resected specimens, and
differentiate pancreatic cystic lesions using cystic fluid collected by
endoscopic ultrasound-guided fine needle aspiration (EUS-FNA), duodenal
fluid, and serum liquid biopsy samples.

{}

{}

\begin{center}\rule{0.5\linewidth}{\linethickness}\end{center}

 - \textbf{Intraductal Oncocytic Papillary Neoplasms:
Clinical-Pathologic Characterization of 24 Cases, With An Emphasis on
Associated Invasive Carcinomas}

\emph{The American journal of surgical pathology 2019 May;43(5):656-661}

PubMed: \url{https://www.ncbi.nlm.nih.gov/pubmed/?term=30986801}

BACKGROUND: Intraductal oncocytic papillary neoplasm (IOPN) of the
pancreas is a rare tumor. Recent molecular data indicate that it is
distinct from other intraductal neoplasms; however, its
clinicopathologic characteristics, especially the frequency/significance
of an invasive carcinoma component, and biologic behavior remain to be
fully defined. DESIGN: Clinicopathologic characteristics and survival of
24 IOPNs were analyzed. By definition, all tumors exhibited intraductal
growth and oncocytic morphology. RESULTS: The female:male ratio was 1.7,
and mean age was 59. In 44\% of the patients, the IOPN was discovered
incidentally; however, the working diagnosis was ``ductal
adenocarcinoma'' in 42\%. Fourteen IOPNs occurred in the head of the
pancreas. The median tumor size was 4.5 cm. The tumors often grew along
adjacent benign ducts, mimicking invasion, but only 29\% exhibited
unequivocal invasive carcinoma, mostly in the form of microscopic foci
(pT1a=4, pT1b=1, pT2=2), and only 6\% had lymph node metastasis.
Invasive carcinoma was predominantly composed of small tubular units
lined by oncocytic cells, or individual oncocytic cells infiltrating the
periductal stroma. Follow-up information was available for 18 patients
(median=6.8 y). No patients died from the disease, and the overall
10-year survival was 94\%. Patients with invasive carcinoma trended
toward a lower 5-year recurrence-free survival than those with
noninvasive IOPNs (66\% vs.~93\%, P=0.066), but overall survival was not
impacted by the presence of invasion (P=0.38). CONCLUSIONS: IOPN is a
distinct tumor type in the pancreas. Despite its morphologic complexity
and often extensive pagetoid spread to adjacent ducts, conventional
invasive carcinoma is seen in only 29\% and usually as microscopic foci.
Thus, it is not surprising that IOPN exhibits indolent behavior even
when invasion is present.

{}

{}

\begin{center}\rule{0.5\linewidth}{\linethickness}\end{center}

 - \textbf{Cyclooxygenase-2 and Cytosolic Phospholipase A2 Are
Overexpressed in Mucinous Pancreatic Cysts}

\emph{Clinical and translational gastroenterology 2019 Apr;10(4):e00028}

PubMed: \url{https://www.ncbi.nlm.nih.gov/pubmed/?term=31009406}

OBJECTIVES: Expression of prostaglandin biosynthetic pathway enzymes in
mucinous pancreatic cysts is unknown. Cyclooxygenase-2 (COX-2)
inhibition is a potential cancer chemoprevention strategy for these
lesions. We evaluated the expression of COX-2, cytosolic phospholipase
A2 (cPLA2), and protein kinase B (AKT) in the epithelium of pancreatic
cysts and correlated enzyme expression with aspirin (ASA) use and cyst
fluid prostaglandin E2 (PGE2) concentration. METHODS: Pathology of 80
resected pancreatic cysts was reviewed. Expression of COX-2, cPLA2, and
AKT was quantified by tissue immunohistochemistry immunoreactivity
scores (IRSs). IRS values were compared between cyst types and (in 30
cases) with matched cyst fluid PGE2 concentrations. RESULTS: The mean
IRS was higher in the epithelium of mucinous vs nonmucinous cysts for
COX-2 (6.1 ± 4.7 vs 3.2 ± 2.8, P = 0.01) and cPLA2 (6.9 ± 3.0 vs 2.9 ±
2.9, P \textless{} 0.001). Cyst epithelial COX-2 expression was higher
in mucinous cysts with low-grade dysplasia vs those with high-grade
dysplasia or invasive carcinoma (IRS 8.0 ± 3.9 vs 1.5 ± 2.9, P
\textless{} 0.001), whereas the opposite was found for cPLA2 (6.2 ± 3.0
vs 8.6 ± 2.3, P = 0.005). Cyst fluid PGE2 concentrations did not
correlate with either the IRS or a history of low- to moderate-dose ASA
use. CONCLUSIONS: COX-2 and cPLA2 are overexpressed in the epithelium of
mucinous pancreatic cysts. COX-2 and/or cPLA2 inhibition might prevent
the emergence or progression of mucinous pancreatic cysts, but higher
doses of ASA or nonsteroidal anti-inflammatory drugs may be necessary to
substantially inhibit cyst epithelial COX-2 activity.

{}

{}

\begin{center}\rule{0.5\linewidth}{\linethickness}\end{center}

Back to top

\begin{center}\rule{0.5\linewidth}{\linethickness}\end{center}

\pagebreak

\hypertarget{spn}{%
\paragraph{SPN}\label{spn}}

Solid Pseudopapillary Neoplasm

\begin{center}\rule{0.5\linewidth}{\linethickness}\end{center}

 - \textbf{Solid Pseudopapillary Neoplasms of the Pancreas: A Large
American Cohort}

\emph{Pancreas 2019 04;48(4):e21-e22}

PubMed: \url{https://www.ncbi.nlm.nih.gov/pubmed/?term=30973464}

{}

{}

\begin{center}\rule{0.5\linewidth}{\linethickness}\end{center}

Back to top

\begin{center}\rule{0.5\linewidth}{\linethickness}\end{center}

\pagebreak

\begin{center}\rule{0.5\linewidth}{\linethickness}\end{center}

\hypertarget{bile-ducts}{%
\subsubsection{Bile Ducts}\label{bile-ducts}}

\begin{center}\rule{0.5\linewidth}{\linethickness}\end{center}

\hypertarget{morphology-diagnostics-ihc-1}{%
\paragraph{Morphology, Diagnostics,
IHC}\label{morphology-diagnostics-ihc-1}}

Morphology, Diagnostics, IHC

\begin{center}\rule{0.5\linewidth}{\linethickness}\end{center}

 - \textbf{Impact of tumour budding grade in 310 patients who underwent
surgical resection for extrahepatic cholangiocarcinoma}

\emph{Histopathology 2019 May;74(6):861-872}

PubMed: \url{https://www.ncbi.nlm.nih.gov/pubmed/?term=30667537}

AIMS: Tumour budding is a risk factor for poor prognosis in various
cancers. Tumour buds may present an epithelial-mesenchymal transition
(EMT) morphological phenotype. This study aimed to elucidate the
prognostic impact of tumour budding grade and its association with
clinicopathological and EMT-related features in perihilar
cholangiocarcinoma (PHCC) or distal cholangiocarcinoma (DCC). METHODS
AND RESULTS: Subjects included 195 PHCC and 115 DCC patients. The
numbers of tumour buds in different patients were stratified for
postoperative survival using the recursive partitioning technique.
Consequently, the numbers of tumour buds in PHCC patients were
classified into three grades; namely, low (0-4 buds); intermediate (5-11
buds); and high (≥12 buds); those of DCC patients were classified into
two grades; namely, low (0-4 buds) and high (≥5 buds). In both PHCC and
DCC patients, high tumour budding grade was associated with poor
histological differentiation, higher pT factor, presence of lymphatic,
venous, perineural invasion and regional lymph node metastasis. In PHCC
patients, residual invasive tumour in the resected margin was also
associated with high tumour budding grade. For both PHCC and DCC
patients, high tumour budding grade was an independent adverse
prognostic factor in multivariate analysis (P~\textless{}~0001 and
P~=~0.046, respectively). Immunohistochemical examination revealed that
the number of tumour buds increased in patients with tumours showing a
mesenchymal profile (negative for E-cadherin and positive for vimentin).
CONCLUSIONS: Higher tumour budding grade is associated with invasive
clinicopathological features, adverse postoperative prognosis and EMT
status in extrahepatic cholangiocarcinoma.

{}

{}

\begin{center}\rule{0.5\linewidth}{\linethickness}\end{center}

 - \textbf{Intrahepatic Cholangiocarcinomas Have Histologically and
Immunophenotypically Distinct Small and Large Duct Patterns}

\emph{The American journal of surgical pathology 2018
Oct;42(10):1334-1345}

PubMed: \url{https://www.ncbi.nlm.nih.gov/pubmed/?term=30001234}

Intrahepatic cholangiocarcinomas are histologically heterogenous. Using
a cohort of 184 clinically defined, resected intrahepatic
cholangiocarcinomas, we retrospectively classified the histology into 4
subtypes: large duct (LD), small duct (SD) (predominantly tubular
{[}SD1{]} or predominantly anastomosing/cholangiolar, {[}SD2{]}), or
indeterminate. Then, we tested the 4 subtypes for associations with risk
factors, patient outcomes, histology, and immunophenotypic
characteristics. SD was the most common (84\%; 24\% SD1 and 60\% SD2)
with lower proportions of LD (8\%), and indeterminate (8\%). Primary
sclerosing cholangitis was rare (2\%), but correlated with LD (P=0.005).
Chronic hepatitis, frequent alcohol use, smoking, and steatosis had no
histologic association. LD was associated with mucin production
(P\textless{}0.001), perineural invasion (P=0.002), CA19-9 staining
(P\textless{}0.001), CK7, CK19, CD56 immunophenotype (P=0.005), and
negative albumin RNA in situ hybridization (P\textless{}0.001). SD was
histologically nodular (P=0.019), sclerotic (P\textless{}0.001),
hepatoid (P=0.042), and infiltrative at the interface with hepatocytes
(P\textless{}0.001). Albumin was positive in 71\% of SD and 18\% of LD
(P=0.0021). Most albumin positive tumors (85\%) lacked extracellular
mucin (P\textless{}0.001). S100P expression did not associate with
subtype (P\textgreater{}0.05). There was no difference in
disease-specific or recurrence-free survival among the subtypes.
Periductal infiltration and American Joint Committee on Cancer eighth
edition pT stage predicted survival by multivariable analysis accounting
for gross configuration, pT stage, and histologic type. pT2 had worse
outcome relative to other pT stages. Significant differences in
histology and albumin expression distinguish LD from SD, but there is
insufficient evidence to support further subclassification of SD.

{}

{}

\begin{center}\rule{0.5\linewidth}{\linethickness}\end{center}

Back to top

\begin{center}\rule{0.5\linewidth}{\linethickness}\end{center}

\pagebreak

\hypertarget{staging-1}{%
\paragraph{Staging}\label{staging-1}}

Bile Ducts TNM staging, Margins, Survival

\begin{center}\rule{0.5\linewidth}{\linethickness}\end{center}

 - \textbf{The Evaluation of the Eighth Edition of the AJCC/UICC Staging
System for Intrahepatic Cholangiocarcinoma: a Proposal of a Modified New
Staging System}

\emph{Journal of gastrointestinal surgery : official journal of the
Society for Surgery of the Alimentary Tract 2019 Apr;():}

PubMed: \url{https://www.ncbi.nlm.nih.gov/pubmed/?term=31012045}

BACKGROUND: The objective was to clarify the prognostic impact of the
8th edition of American Joint Committee on Cancer (AJCC)/International
Union Against Cancer (UICC) of intrahepatic cholangiocarcinoma (ICC).
METHODS: A total of 103 ICC patients who underwent hepatectomy between
2002 and 2016 were enrolled. The survival impact of AJCC/UICC 8th
edition was examined. RESULTS: The 5-year disease-specific survival
(DSS) rate was 75.9\% in T1a (n = 23), 88.9\% in T1b (n = 10), 14.9\% in
T2 (n = 24), 52.5\% in T3 (n = 11), and 15.2\% in T4 (n = 35). The DSS
was comparable among T2, T3, and T4 (T2 vs.~T3; p = 0.345, T3 vs.~T4;
0.295). A multivariate analysis identified multiple tumors (hazard ratio
{[}HR{]} 2.821), periductal infiltrating (HR 2.439), perforation of the
visceral peritoneum (HR 1.850), and vascular invasion (HR 1.872) as
independent prognostic factors that were associated with the DSS. The
optimum tumor size with the greatest difference in the DSS was 2~cm
(p = 0.014). The new T classification was developed as follows: T1, size
≤ 2~cm without other factors; T2, size \textgreater{} 2~cm without other
factors; T3, vascular invasion or perforation of the visceral
peritoneum; and T4, multiple tumors or periductal infiltrating. The
5-year DSS was 100\% in T1 (n = 7), 76.6\% in T2 (n = 28), 45.1\% in T3
(n = 28), and 3.4\% in T4 (n = 40). There were differences in the DSS
between T2 and T3 (p = 0.035) and between T3 and T4 (p = 0.003).
CONCLUSIONS: T2, T3, and T4 of AJCC/UICC overlapped with regard to the
DSS. The new staging can classify ICC patients with sufficient
prognostic differences.

{}

{}

\begin{center}\rule{0.5\linewidth}{\linethickness}\end{center}

 - \textbf{Validation Study of Tumor Invasive Thickness for
Postoperative Prognosis in 110 Patients Who Underwent
Pancreatoduodenectomy for Distal Cholangiocarcinoma at a Single
Institution}

\emph{The American journal of surgical pathology 2019 May;43(5):717-723}

PubMed: \url{https://www.ncbi.nlm.nih.gov/pubmed/?term=30986803}

The pT classification of the 8th American Joint Committee on Cancer
(AJCC) for distal cholangiocarcinoma (DCC) is classified according to
depth of invasion (DOI), which is the distance from the basal lamina to
the most deeply advanced tumor cells. The Nagoya group proposed a new T
classification for DCC based on invasive tumor thickness (ITT), which is
the maximal vertical distance of the invasive cancer component (the ITT
grade). In this study, we aimed to validate the ITT grade for the next
pT classification of DCC in 110 patients. ITT could be measured in all
patients, but DOI could only be measured in 62 (56\%) patients.
According to ITT grade, patients were classified into grades A to D, as
follows: grade A, ITT \textless{}1 mm (n=9); grade B, ITT 1 mm or more
but \textless{}5 mm (n=35); grade C, ITT 5 mm or more but
\textless{}10 mm (n=40); and grade D, ITT 10 mm or greater (n=26). The
median overall survival times in patients with ITT grades A, B, C, and D
were 12.8, 5.7, 3.7, and 2.0 years, respectively. ITT grade could
discriminate postoperative survivals between grades. On multivariate
analysis, ITT grade, regional lymph node metastasis, and distant
metastasis were selected as independent prognostic factors. In summary,
our results showed that ITT grade was a suitable alternative to DOI for
pT classification in the next edition of the AJCC for DCC.

{}

{}

\begin{center}\rule{0.5\linewidth}{\linethickness}\end{center}

 - \textbf{Should Utilization of Lymphadenectomy Vary According to
Morphologic Subtype of Intrahepatic Cholangiocarcinoma?}

\emph{Annals of surgical oncology 2019 Jul;26(7):2242-2250}

PubMed: \url{https://www.ncbi.nlm.nih.gov/pubmed/?term=30927194}

OBJECTIVE: We sought to evaluate the utilization of lymphadenectomy
(LND) and the incidence of lymph node metastasis (LNM) among different
morphologic types of intrahepatic cholangiocarcinoma (ICC). METHODS:
Clinical data of patients undergoing curative-intent resection for ICC
between 1990 and 2017 were collected and analyzed. The preoperative
nodal status was evaluated by imaging studies, and the morphologic and
lymph node (LN) status was collected on final pathology report. RESULTS:
Overall, 1032 patients had a mass-forming (MF) or intraductal growth
(IG) ICC subtype, whereas 150 patients had a periductal infiltrating
(PI) or MF + PI subtype. Among the 924 patients with MF/IG ICC subtype
who had nodal assessment on preoperative imaging, 747 (80.8\%) were
node-negative, whereas 177 (19.2\%) patients were suspicious for
metastatic nodal disease. On final pathological analysis, 71 of 282
(25.2\%) patients who had preoperative node-negative disease ultimately
had LNM. In contrast, 79 of 135 (58.5\%) patients with preoperative
suspicious/metastatic LNs had pathologically confirmed LNM (odds ratio
{[}OR{]} 4.2, p \textless{} 0.001). Among the 129 patients with
PI/MF + PI ICC subtype and preoperative nodal information, 72 (55.8\%)
were node-negative on preoperative imaging. In contrast, 57 (44.2\%)
patients had suspicious/metastatic LNs. On final pathologic examination,
45.3\% (n = 24) of patients believed to be node-negative on preoperative
imaging had LNM; 68.0\% (n = 34) of patients who had suspicious/positive
nodal disease on imaging ultimately had LNM (OR 2.6, p = 0.009).
CONCLUSION: Given the low accuracy of preoperative imaging evaluation of
nodal status, routine LND should be performed at the time of resection
for both MF/IG and PI/MF + PI ICC subtypes.

{}

{}

\begin{center}\rule{0.5\linewidth}{\linethickness}\end{center}

 - \textbf{Assessment of the Lymph Node Status in Patients Undergoing
Liver Resection for Intrahepatic Cholangiocarcinoma: the New Eighth
Edition AJCC Staging System}

\emph{Journal of gastrointestinal surgery : official journal of the
Society for Surgery of the Alimentary Tract 2018 01;22(1):52-59}

PubMed: \url{https://www.ncbi.nlm.nih.gov/pubmed/?term=28424987}

INTRODUCTION: The role of routine lymphadenectomy for intrahepatic
cholangiocarcinoma (ICC) is still controversial. The AJCC eighth edition
recommends a minimum of six harvested lymph nodes (HLNs) for adequate
nodal staging. We sought to define outcome and risk of death among
patients who were staged with ≥6 HLNs versus \textless{}6 HLNs.
MATERIALS AND METHODS: Patients undergoing hepatectomy for ICC between
1990 and 2015 at 1 of the 14 major hepatobiliary centers were
identified. RESULTS: Among 1154 patients undergoing hepatectomy for ICC,
515 (44.6\%) had lymphadenectomy. On final pathology, 200 (17.3\%)
patients had metastatic lymph node (MLN), while 315 (27.3\%) had
negative lymph node (NLN). Among NLN patients, HLN was associated with
5-year OS (p~=~0.098). While HLN did not impact 5-year OS among MLN
patients (p~=~0.71), the number of MLN was associated with 5-year OS
(p~=~0.02). Among the 317 (27.5\%) patients staged according the AJCC
eighth edition staging system, N1 patients had a 3-fold increased risk
of death compared with N0 patients (hazard ratio 3.03;
p~\textless{}~0.001). CONCLUSION: Only one fourth of patients undergoing
hepatectomy for ICC had adequate nodal staging according to the AJCC
eighth edition. While the six HLN cutoff value impacted prognosis of N0
patients, the number of MLN rather than HLN was associated with
long-term survival of N1 patients.

{}

{}

\begin{center}\rule{0.5\linewidth}{\linethickness}\end{center}

Back to top

\begin{center}\rule{0.5\linewidth}{\linethickness}\end{center}

\pagebreak

\hypertarget{gallbladder}{%
\subsubsection{Gallbladder}\label{gallbladder}}

\begin{center}\rule{0.5\linewidth}{\linethickness}\end{center}

\hypertarget{morphology-diagnostics-ihc-2}{%
\paragraph{Morphology, Diagnostics,
IHC}\label{morphology-diagnostics-ihc-2}}

Morphology, Diagnostics, IHC

\begin{center}\rule{0.5\linewidth}{\linethickness}\end{center}

 - \textbf{Follicular cholecystitis: clinicopathologic associations}

\emph{Human pathology 2019 Jun;88():1-6}

PubMed: \url{https://www.ncbi.nlm.nih.gov/pubmed/?term=30946935}

Follicular cholecystitis (FC) is a relatively rare entity with uncertain
causal associations. In this study, we aimed to explore different
clinicopathologic associations of FC, and to better characterize the
entity. A retrospective review of archival hematoxylin and eosin slides
and pertinent clinical information was undertaken for all
cholecystectomy cases with a rendered diagnosis of ``follicular
cholecystitis,'' from 1991 to 2017. Concurrent conventional chronic
cholecystitis (CC) and lymphocytic cholecystitis (LC) were documented.
Forty-three consecutive patients were confirmed to have FC. The majority
of the patients (88.4\%) had at least one other histologic association
in the gallbladder (LC, CC, or both). Remarkably, functional distal
biliary obstruction (from choledocholithiasis, sclerosing cholangitis,
distal biliary strictures, or malignancies of the pancreatic head or
ampulla) was found in 76.7\% of the patients, irrespective of the
presence of other concurrent histologic findings. FC associated with CC
was relatively more common in females (61\%) and strongly associated
with cholelithiasis (70\%). However, those without CC were predominantly
males (70\%) and had a significant association with LC (75\%). All four
cases of FC without any other histologic associations who had clinical
information available showed some form of distal biliary obstruction. FC
cases without concurrent LC were often associated with CC (74\%). FC is
strongly associated with extrahepatic biliary obstruction distal to the
gallbladder. Therefore, this finding at routine cholecystectomy may
warrant further evaluation to rule out a cause for distal biliary tract
obstruction. Additionally, it is commonly associated with other
concomitant histologic abnormalities in the gallbladder such as CC
and/or LC.

{}

{}

\begin{center}\rule{0.5\linewidth}{\linethickness}\end{center}

 - \textbf{Primary Gallbladder Neuroendocrine Tumors: Insights into a
Rare Histology Using a Large National Database}

\emph{Annals of surgical oncology 2019 May;():}

PubMed: \url{https://www.ncbi.nlm.nih.gov/pubmed/?term=31102094}

BACKGROUND: Primary gallbladder neuroendocrine tumors (NETs) are rare,
poorly understood cancers infrequently encountered at even the largest
of tertiary referral centers. We therefore sought to identify a large
cohort of patients with gallbladder NETs using a national database, with
the aim of defining treatment modalities employed and survival
associated with these uncommon malignancies. METHODS: Patients with
primary gallbladder NETs were identified in the National Cancer
Database, and clinicopathologic characteristics were recorded. A
univariate log-rank survival analysis was completed for patients who
underwent resection. Parameters found to be significant were entered
into a multivariate accelerated failure time analysis. For context,
survival comparisons were included for patients who underwent resections
for NETs at any gastrointestinal site and for gallbladder
adenocarcinoma. RESULTS: Overall, 754 patients with gallbladder NETs
were identified. Patients were predominantly female (n = 518, 69\%),
White (n = 503, 67\%), presented with stage IV disease (n = 295, 39\%)
and had high-grade lesions (n = 312, 41\%). The majority underwent
resection (n = 480, 64\%), primarily simple cholecystectomy (n = 431,
90\%), whereas a minority received multimodal therapy (n = 145, 21\%).
Among patients who underwent resection, older age (p = 0.001), large
cell histology (p = 0.012), and positive margins (p = 0.030) were
independently associated with worse overall survival. Patients with
gallbladder NETs had improved survival relative to those with
gallbladder adenocarcinoma (p = 0.001), but significantly worse survival
than patients with NETs from other gastrointestinal sites
(p \textless{} 0.001). CONCLUSIONS: Primary gallbladder NETs are
aggressive lesions that carry a worse prognosis than NETs of other
gastrointestinal sites. Older age, positive margins, and large cell
histology are associated with abbreviated survival after resection.

{}

{}

\begin{center}\rule{0.5\linewidth}{\linethickness}\end{center}

 - \textbf{Low frequency of mismatch repair deficiency in gallbladder
cancer}

\emph{Diagnostic pathology 2019 May;14(1):36}

PubMed: \url{https://www.ncbi.nlm.nih.gov/pubmed/?term=31068195}

BACKGROUND: DNA mismatch repair (MMR) deficiency is a major pathway of
genomic instability in cancer. It leads to the accumulation of numerous
mutations predominantly at microsatellite sequences, a phenotype known
as microsatellite instability (MSI). MSI tumors have a distinct clinical
behavior and commonly respond well to immune checkpoint blockade,
irrespective of their origin. Data about the prevalence of MSI among
gallbladder cancer (GBC) have been conflicting. We here analyzed a
well-characterized cohort of 69 Western-world GBCs. METHODS: We analyzed
the mononucleotide MSI marker panel consisting of BAT25, BAT26, and
CAT25 to determine the prevalence of MMR deficiency-induced MSI.
RESULTS: MSI was detected in 1/69 (1.4\%)~of analyzed GBCs. The detected
MSI GBC had a classical histomorphology, i.e.~of
acinar/tubular/glandular pancreatobiliary phenotype, and showed nuclear
expression of all four MMR proteins MLH1, MSH2, MSH6, and PMS2. The MSI
GBC patient showed a prolonged overall survival, despite having a high
tumor stage at diagnosis. The patient had no known background or family
history indicative of Lynch syndrome. CONCLUSIONS: Even though the
overall number of MSI tumors is low in GBC, the potentially therapeutic
benefit of checkpoint blockade in the respective patients may justify
MSI analysis of GBC.

{}

{}

\begin{center}\rule{0.5\linewidth}{\linethickness}\end{center}

Back to top

\begin{center}\rule{0.5\linewidth}{\linethickness}\end{center}

\pagebreak

\hypertarget{staging-2}{%
\paragraph{Staging}\label{staging-2}}

Gallbladder TNM staging, Margins, Survival

\begin{center}\rule{0.5\linewidth}{\linethickness}\end{center}

 - \textbf{Systematic review of management of incidental gallbladder
cancer after cholecystectomy}

\emph{The British journal of surgery 2019 01;106(1):32-45}

PubMed: \url{https://www.ncbi.nlm.nih.gov/pubmed/?term=30582640}

BACKGROUND: Gallbladder cancer is rare, but cancers detected
incidentally after cholecystectomy are increasing. The aim of this study
was to review the available data for current best practice for optimal
management of incidental gallbladder cancer. METHODS: A systematic
PubMed search of the English literature to May 2018 was conducted.
RESULTS: The search identified 12 systematic reviews and meta-analyses,
in addition to several consensus reports, multi-institutional series and
national audits. Some 0·25-0·89 per cent of all cholecystectomy
specimens had incidental gallbladder cancer on pathological examination.
Most patients were staged with pT2 (about half) or pT1 (about one-third)
cancers. Patients with cancers confined to the mucosa (T1a or less) had
5-year survival rates of up to 100 per cent after cholecystectomy alone.
For cancers invading the muscle layer of the gallbladder wall (T1b or
above), reresection is recommended. The type, extent and timing of
reresection remain controversial. Observation time may be used for new
cross-sectional imaging with CT and MRI. Perforation at initial surgery
had a higher risk of disease dissemination. Gallbladder cancers are
PET-avid, and PET may detect residual disease and thus prevent
unnecessary surgery. Routine laparoscopic staging before reresection is
not warranted for all stages. Risk of peritoneal carcinomatosis
increases with each T category. The incidence of port-site metastases is
about 10 per cent. Routine resection of port sites has no effect on
survival. Adjuvant chemotherapy is poorly documented and probably
underused. CONCLUSION: Management of incidental gallbladder cancer
continues to evolve, with more refined suggestions for subgroups at risk
and a selective approach to reresection.

{}

{}

\begin{center}\rule{0.5\linewidth}{\linethickness}\end{center}

 - \textbf{Staging gallbladder cancer with lymphadenectomy: the
practical application of new AHPBA and AJCC guidelines}

\emph{HPB : the official journal of the International Hepato Pancreato
Biliary Association 2019 Apr;():}

PubMed: \url{https://www.ncbi.nlm.nih.gov/pubmed/?term=31010632}

BACKGROUND: Current guidelines recommend harvesting a total lymph node
count (TLNC) ≥6 from portal lymphadenectomy in ≥pT1b gallbladder cancers
(GBC) for accurate staging and prognostication. This study aimed to
determine nodal yields from portal lymphadenectomy and identify measures
to maximize TLNC. METHODS: We retrospectively reviewed all ≥pT1b GBC
which underwent resection with curative intent including portal
lymphadenectomy at our specialized HPB center from 2007 to 2017. We
compared outcomes of TLNC~\textless{}~6 and TLNC~≥~6 cohorts and
determined factors predictive of TLNC. RESULTS: Of 92 patients, 20\% had
a TLNC~≥~6 (IQR 7-11) and 9\% had no nodes found on pathology. Malignant
lymphadenopathy was twice as common in TLNC~≥~6 as TLNC~\textless{}~6
(p~=~0.003) most frequently from portal, cystic and pericholedochal
stations. On logistic regression analysis, concomitant liver resection
was an independent predictor of higher TLNC {[}4b/5 wedge resection (OR
0.166, CI 0.057-0.486, p~=~0.001) extended hepatectomy (OR 0.065, CI
0.012-0.340, p~=~0.001){]}; biliary resection and en bloc adjacent organ
resection were not. CONCLUSION: At our center, prior to current
guidelines, a TLNC≥6 was not met in 80\% undergoing portal
lymphadenectomy for~≥~pT1b GBC. To increase nodal yield, future
guidelines should consider including additional lymph node stations and
incorporation of frozen section analysis.

{}

{}

\begin{center}\rule{0.5\linewidth}{\linethickness}\end{center}

 - \textbf{The incidence rates and survival of gallbladder cancer in the
USA}

\emph{European journal of cancer prevention : the official journal of
the European Cancer Prevention Organisation (ECP) 2019 01;28(1):1-9}

PubMed: \url{https://www.ncbi.nlm.nih.gov/pubmed/?term=28683010}

Gallbladder cancer is a rare malignancy in most countries. The racial
and sociodemographic factors associated with its incidence and survival
are poorly defined. We aimed to investigate population-based gallbladder
cancer incidence and survival trends on the basis of clinical
characteristics and sociodemographic factors in the USA. Gallbladder
cancer incidence and survival data from 2001 to 2012 were obtained from
18 registries of the Surveillance, Epidemiology, and End Results
database. Incidence rates and Joinpoint trends were calculated by
demographic subgroup. Survival trends were assessed using Cox
proportional hazard models. A total of 7769 patients were identified.
The overall gallbladder cancer incidence rates did not significantly
change during the 2001-2012 period. Incidence rates were three times
higher in Hispanics and 1.6 times higher in Blacks compared with Whites.
Over the time period, incidence rates significantly increased among
Blacks and decreased among Hispanics. Male sex {[}hazard ratio (HR):
1.10, 95\% confidence interval (CI): 1.03-1.17{]}, older age (HR: 1.73,
95\% CI: 1.53-1.96), and single and divorced statuses (HR: 1.19, 95\%
CI: 1.09-1.30 and 1.12, 95\% CI: 1.01-1.24) were independently
associated with shorter overall survival, whereas higher education (HR:
0.89, 95\% CI: 0.82-0.97) and higher income (HR: 0.89, 95\% CI:
0.82-0.96) were associated with longer survival. Furthermore, overall
survival has improved in all races/ethnicities except for Hispanics and
Blacks. The overall incidence rates for gallbladder cancer were stable
during 2001-2012. Hispanics have the highest incidence rates, but the
incidence rates in Blacks are on the rise.

{}

{}

\begin{center}\rule{0.5\linewidth}{\linethickness}\end{center}

 - \textbf{Optimal surgical treatment in patients with T1b gallbladder
cancer: An international multicenter study}

\emph{Journal of hepato-biliary-pancreatic sciences 2018
Dec;25(12):533-543}

PubMed: \url{https://www.ncbi.nlm.nih.gov/pubmed/?term=30562839}

BACKGROUND: There is no consensus on the optimal treatment of T1b
gallbladder cancer (GBC) due to the lack of evidence and the difficulty
of anatomy and pathological standardization. METHODS: A total of 272
patients with T1b GBC who underwent surgical resection at 14 centers
with specialized hepatobiliary-pancreatic surgeons and pathologists in
Korea, Japan, Chile, and the United States were studied. Clinical
outcomes including disease-specific survival (DSS) rates according to
the types of surgery were analyzed. RESULTS: After excluding patients,
the 237 qualifying patients consisted of 90 men and 147 women. Simple
cholecystectomy (SC) was performed in 116 patients (48.9\%) and extended
cholecystectomy (EC) in 121 patients (51.1\%). The overall 5-year DSS
was 94.6\%, and it was similar between SC and EC patients (93.7\%
vs.~95.5\%, P~=~0.496). The 5-year DSS was similar between SC and EC
patients in America (82.3\% vs.~100.0\%, P~=~0.249) as well as in Asia
(98.6\% vs.~95.2\%, P~=~0.690). The 5-year DSS also did not differ
according to lymph node metastasis (P~=~0.688) or tumor location
(P~=~0.474). CONCLUSIONS: SC showed similar clinical outcomes (including
recurrence) and survival outcomes as EC; therefore, EC is not needed for
the treatment of T1b GBC.

{}

{}

\begin{center}\rule{0.5\linewidth}{\linethickness}\end{center}

Back to top

\begin{center}\rule{0.5\linewidth}{\linethickness}\end{center}

\pagebreak

\hypertarget{ampulla-of-vater}{%
\subsubsection{Ampulla of Vater}\label{ampulla-of-vater}}

\begin{center}\rule{0.5\linewidth}{\linethickness}\end{center}

\hypertarget{morphology-diagnostics-ihc-3}{%
\paragraph{Morphology, Diagnostics,
IHC}\label{morphology-diagnostics-ihc-3}}

Morphology, Diagnostics, IHC

\begin{center}\rule{0.5\linewidth}{\linethickness}\end{center}

 - \textbf{Distinct immunological properties of the two histological
subtypes of adenocarcinoma of the ampulla of Vater}

\emph{Cancer immunology, immunotherapy : CII 2019 Mar;68(3):443-454}

PubMed: \url{https://www.ncbi.nlm.nih.gov/pubmed/?term=30604042}

Adenocarcinoma of the ampulla of Vater (AOV) is classified into
intestinal type (IT) and pancreatobiliary type (PB); however, the
immunological properties of these subtypes remain to be characterized.
Here, we evaluated the clinical implications of PD-L1 expression and
CD8+ T lymphocyte density in adenocarcinomas of the AOV and their
potential association with Yes-associated protein (YAP). We analyzed 123
adenocarcinoma-of-the-AOV patients who underwent surgical resection, and
tumors were classified into IT or PB type. Tumor or inflammatory cell
PD-L1 expression, CD8+ T lymphocyte density in the cancer cell nest
(intratumoral) or in the adjacent stroma, and YAP localization and
intensity were analyzed using immunohistochemical staining. PB-type
tumors showed higher tumoral PD-L1 expression than IT-type tumors, and
tumoral PD-L1 expression was associated with a shorter disease-free
survival (DFS) {[}hazard ratio (HR), 1.77; p = 0.045{]} and overall
survival (OS) (HR 1.99; p = 0.030). Intratumoral CD8+ T lymphocyte
density was higher in IT type than in PB type and was associated with a
favorable DFS (HR 0.47; p = 0.022). The nuclear staining pattern of YAP
in tumor cells, compared to non-nuclear staining patterns, was more
frequently associated with PB type and increased tumoral PD-L1
expression. Nuclear YAP staining was a significant prognostic factor for
OS (HR 2.21; p = 0.022). These results show that the two subtypes of
adenocarcinoma of the AOV exhibit significant differences in tumoral
PD-L1 expression and intratumoral CD8+ T lymphocyte density, which might
contribute to their distinct clinical features.

{}

{}

\begin{center}\rule{0.5\linewidth}{\linethickness}\end{center}

 - \textbf{Role of Immunohistochemistry in the Subtyping of
Periampullary Adenocarcinoma}

\emph{International journal of surgical pathology 2019
Apr;():1066896919837606}

PubMed: \url{https://www.ncbi.nlm.nih.gov/pubmed/?term=30942099}

CONTEXT: Subtyping of periampullary adenocarcinoma into intestinal and
pancreatobiliary subtypes has emerged as an important prognostic factor
with potential therapeutic implications. This distinction on morphology
alone is often difficult with significant interobserver variability.
OBJECTIVE: To analyze the usefulness of a panel of immunohistochemistry
(IHC) markers as an aid to morphologic subtyping of periampullary
adenocarcinoma. DESIGN: A total of 172 periampullary adenocarcinomas
were classified morphologically by 3 study pathologists. Interobserver
agreement was assessed in each case. Cases were then typed using a
predetermined IHC panel (comprising CK7, CK20, MUC1, and CDX2). RESULTS:
Morphologically, 66 (38.4\%) cases were intestinal, 56 (32.6\%)
pancreatobiliary, 25 (14.5\%) mixed, 16 (9.3\%) poorly differentiated, 6
(3.5\%) mucinous, and 3 (1.7\%) signet ring cell adenocarcinoma.
Concordant diagnosis was reached in 138 cases (80.2\%) with moderate
overall interobserver agreement (κ = 0.47). Concordance was higher in
morphologically distinct mucinous (100\%; κ = 0.94) and signet ring cell
subtypes (100\%; κ = 1.0) than in intestinal (84.6\%; κ = 0.47) and
pancreatobiliary (82.1\%; κ = 0.43) types. Concordance was poor for
mixed (64\%; κ = 0.27) and poorly differentiated (68.8\%; κ = 0.76)
tumors. IHC subtyped 79 cases (46\%) as pancreatobiliary, 73 (42.4\%) as
intestinal, and was inconclusive in 20 cases (11.6\%). IHC helped
classify 21 out of 25 (84\%) mixed and 10 out of 16 poorly
differentiated (62.5\%) adenocarcinomas. Combination of histology and
IHC classified 161 of the total 172 cases (93.6\%). CONCLUSION: Use of
an IHC panel aids in subtyping of periampullary adenocarcinomas,
especially in tumors with mixed morphology and poor differentiation.

{}

{}

\begin{center}\rule{0.5\linewidth}{\linethickness}\end{center}

 - \textbf{Can we classify ampullary tumours better? Clinical,
pathological and molecular features. Results of an AGEO study}

\emph{British journal of cancer 2019 Apr;120(7):697-702}

PubMed: \url{https://www.ncbi.nlm.nih.gov/pubmed/?term=30837681}

BACKGROUND: Ampullary adenocarcinoma (AA) originates from either
intestinal (INT) or pancreaticobiliary (PB) epithelium. Different
prognostic factors of recurrence have been identified in previous
studies. METHODS: In 91 AA patients of the AGEO retrospective
multicentre cohort, we evaluated the centrally reviewed morphological
classification, panel markers of Ang et al.~including CK7, CK20, MUC1,
MUC2 and CDX2, the 50-gene panel mutational analysis, and the
clinicopathological AGEO prognostic score. RESULTS: Forty-three (47\%)
of the 91 tumours were Ang-INT, 29 (32\%) were Ang-PB, 18 (20\%) were
ambiguous (Ang-AMB) and one could not be classified. Among these 90
tumours, 68.7\% of INT tumours were Ang-INT and 78.2\% of PB tumours
were Ang-PB. MUC5AC expression was detected in 32.5\% of the 86
evaluable cases. Among 71 tumours, KRAS, TP53, APC and PIK3CA were the
most frequently mutated genes. The KRAS mutation was significantly more
frequent in the PB subtype. In multivariate analysis, only AGEO
prognostic score and tumour subtype were associated with relapse-free
survival. Only AGEO prognostic score was associated with overall
survival. CONCLUSIONS: Mutational analysis and MUC5AC expression provide
no additional value in the prognostic evaluation of AA patients. Ang et
al.~classification and the AGEO prognostic score were confirmed as a
strong prognosticator for disease recurrence.

{}

{}

\begin{center}\rule{0.5\linewidth}{\linethickness}\end{center}

Back to top

\begin{center}\rule{0.5\linewidth}{\linethickness}\end{center}

\pagebreak

\hypertarget{neuroendocrine}{%
\subsubsection{Neuroendocrine}\label{neuroendocrine}}

\begin{center}\rule{0.5\linewidth}{\linethickness}\end{center}

\hypertarget{pannet}{%
\paragraph{PanNET}\label{pannet}}

PanNET, Pancreatic Neuroendocrine Tumors and related neuroendocrine
neoplasms

\begin{center}\rule{0.5\linewidth}{\linethickness}\end{center}

 - \textbf{Loss of Menin Expression by Immunohistochemistry in
Pancreatic Neuroendocrine Tumors: Comparison Between Primary and
Metastatic Tumors}

\emph{Pancreas 2019 04;48(4):510-513}

PubMed: \url{https://www.ncbi.nlm.nih.gov/pubmed/?term=30946241}

OBJECTIVES: Molecular characterization of sporadic pancreatic
neuroendocrine tumors (PanNETs) demonstrates frequent alterations in
MEN1. As the role of menin immunohistochemistry as a potential biomarker
is being developed, knowledge of whether the pattern of menin expression
is the same in primary tumors and distant metastases may help in patient
care. Therefore, we compared patterns of menin expression in matched
primary tumors and metastases. METHODS: We evaluated loss of menin
nuclear expression by immunohistochemistry in 44 matched samples of
primary and metastatic PanNETs and concordance in staining pattern
between primary and metastatic tumors. RESULTS: Menin nuclear expression
was lost in 18 (41\%) of 44 primary tumors and 17 (39\%) of 44
metastases. Concordant loss of menin expression was observed in 41 cases
(93\%); discordance was observed in 3 cases (7\%; 95\% confidence
interval, 1.4\%-18.7\%), including 2 with loss in the primary tumor but
not the metastasis. CONCLUSIONS: The concordance of menin staining
between primary tumor and metastasis in most cases suggests that menin
loss is an early event in PanNET tumorigenesis. The discordant
expression observed in a small subset may be a source of menin-directed
therapy failure; thus, repeat assessment of metastases may be helpful
for treatment selection.

{}

{}

\begin{center}\rule{0.5\linewidth}{\linethickness}\end{center}

 - \textbf{Predicting Survival of Small Intestine Neuroendocrine Tumors:
Experience From a Major Referral Center}

\emph{Pancreas 2019 04;48(4):514-518}

PubMed: \url{https://www.ncbi.nlm.nih.gov/pubmed/?term=30946234}

OBJECTIVE: Neuroendocrine tumors (NETs) comprise 41.8\% of small
intestine malignancies. The NET nomogram is a 15-item prognostic tool
that includes relevant factors for guiding management decisions. This is
the first external validation of this tool among American patients at a
tertiary treatment center. METHODS: Patients who underwent surgical
intervention from 2005 to 2017 were screened by retrospective chart
review. Nomogram scores were calculated following the methods outlined
by Modlin et al (Neuroendocrinology. 2010;92:143-157). Validation
assessed the association between nomogram scores and survival using
Wilcoxon test and Cox regression. RESULTS: Among the 121 patients
selected, the NET nomogram significantly predicted survival as a
continuous variable (P \textless{} 0.01) and when dichotomized using 83
points to distinguish low-risk versus high-risk groups (P \textless{}
0.01). However, the nomogram was not universally applicable as even at
our specialty center, variables such as chromogranin A and urinary
5-hydroxyindoleacetic acid are not routinely collected, whereas others,
like tumor grade, do not reflect the most recently updated
classifications. CONCLUSION: The NET nomogram accurately identified
patients at low and high risk of death. However, revision to update
prognosticators could improve its usefulness for predicting survival of
small intestine NETs.

{}

{}

\begin{center}\rule{0.5\linewidth}{\linethickness}\end{center}

 - \textbf{Clinicopathological characteristics and risk factors for
recurrence of well-differentiated pancreatic neuroendocrine tumors after
radical surgery: a case-control study}

\emph{World journal of surgical oncology 2019 Apr;17(1):66}

PubMed: \url{https://www.ncbi.nlm.nih.gov/pubmed/?term=30975157}

BACKGROUND: Well-differentiated pancreatic neuroendocrine tumors
(PanNETs) usually have a good prognosis; however, there are patients
that experience recurrence after curative resection. AIM: To explore
recurrence-related risk factors by analyzing clinicopathological data of
PanNETs after radical surgery. METHODS: Clinical and pathological data
from 47 patients with well-differentiated PanNETs at China-Japan
Friendship Hospital from January 2012 to March 2016 were analyzed
retrospectively. Univariate and multivariate analyses of the risk
factors of PanNETs for postoperative recurrence were conducted. RESULTS:
Among the 47 patients with well-differentiated PanNETs, there were 38
cases with non-functioning tumors, 9 cases with functional tumors (6
insulinomas, 1 gastrinoma, 1 glucagonoma, and 1 VIPomas). There are 17
cases (36.2\%) in the pancreatic head, 17 (36.2\%) in the body and tail,
9 (19.1\%) in the tail, and 4 (8.5\%) in the body. The median tumor size
was 3.65 (IQR 2-5.5) cm. Fourteen cases (29.8\%) were NET G1, and 33
cases (70.2\%) were NET G2. In regard to the clinical stage, 9 (19.1\%)
cases were IA, 14 (29.8\%) cases were IB, 7 (14.9\%) cases were IIA, 14
(29.8\%) cases were IIB, and 3 cases unknown. There were 17 patients who
presented with postoperative recurrence. Univariate analysis showed that
AJCC TNM staging, Ki67 index, vascular invasion, margin status, and the
regional stage of the tumors are related to the recurrence of patients
with PanNETs (p \textless{} 0.05). The results of multivariate analysis
showed that Ki67 index ≥ 10\% is an independent risk factor for the
postoperative recurrence of PanNETs (p \textless{} 0.05). CONCLUSION:
The Ki67 index ≥ 10\% is an independent risk factor for recurrence in
well-differentiated PanNETs after radical surgery, and close
surveillance for these patients may be needed.

{}

{}

\begin{center}\rule{0.5\linewidth}{\linethickness}\end{center}

 - \textbf{Pancreatic islet (of Langerhans) revisited}

\emph{Histology and histopathology 2019 Apr;():18118}

PubMed: \url{https://www.ncbi.nlm.nih.gov/pubmed/?term=31020988}

One hundred and fifty years ago, Paul Langerhans described what would
come to be known as pancreatic `islet of Langerhans'. Since then, we
have accumulated knowledge about the pancreatic islet, the cells that
exist there and the hormones secreted by these cells. The increasing
prevalence of obesity, diabetes and Alzheimer's disease in the
population (three conditions that are linked to pancreatic islet
function), the islet has been playing a significant role in
endocrinological and metabolic studies searching how we can protect the
pancreatic islet and its cell content, or how we can regenerate it. This
review will be interested in the most recent and relevant aspects of
knowledge regarding the pancreatic islet, always mentioning the
evolution of knowledge and future perspectives for the treatment of
diabetes and Alzheimer's disease. The most recent research with
microRNAs and islet culture and pseudoislet culture (organoids) allows
predicting advances in knowledge with new drugs to act on the
islet/cells (such as the hormone glucagon-like peptide (GLP) -1) as well
as induction of other islet cells like alpha-cells and delta-cells to
transform into beta-cells.

{}

{}

\begin{center}\rule{0.5\linewidth}{\linethickness}\end{center}

Back to top

\begin{center}\rule{0.5\linewidth}{\linethickness}\end{center}

\pagebreak

\hypertarget{staging-3}{%
\paragraph{Staging}\label{staging-3}}

PanNET TNM staging, Margins, Survival

\begin{center}\rule{0.5\linewidth}{\linethickness}\end{center}

 - \textbf{Validation of the 8th AJCC Cancer Staging System for Pancreas
Neuroendocrine Tumors Using Korean Nationwide Surgery Database}

\emph{Cancer research and treatment : official journal of Korean Cancer
Association 2019 Apr;():}

PubMed: \url{https://www.ncbi.nlm.nih.gov/pubmed/?term=30999719}

Purpose: The 8th edition of the American Joint Committee on Cancer
(AJCC) staging system for pancreatic neuroendocrine tumor (PNET)
included several significant changes. We aim to evaluate this staging
system compared to the 7th edition AJCC staging system and European
Neuroendocrine Tumors Society (ENETS) system. Materials and Methods: We
used Korean nationwide surgery database (2000-2014). Of 972 patients who
had undergone surgery for PNET, excluding patients diagnosed with
ENETS/World Health Organization 2010 grade 3 (G3), only 472 patients
with accurate stage were included. Results: Poor discrimination in
overall survival rate (OSR) was noted between AJCC 8th stage III and IV
(p=0.180). The disease-free survival (DFS) curves of 8th AJCC
classification were well separated between all stages. Compared with
stage I, the hazard ratio of II, III, and IV was 3.808, 13.928, and
30.618, respectively (p=0.007, p \textless{} 0.001, and p \textless{}
0.001). The curves of OSR and DFS of certain prognostic group in AJCC
7th and ENETS overlapped. In ENETS staging system, no significant
difference in DFS between stage IIB versus IIIA (p=0.909) and IIIA
versus IIIB (p=0.291). In multivariable analysis, lymphovascular
invasion (p=0.002), perineural invasion (p=0.003), and grade (p
\textless{} 0.001) were identified as independent prognostic factors for
DFS. Conclusion: This is the first large-scale validation of the AJCC
8th edition staging system for pancreatic neuroendocrine tumor. The
revised 8th system provides better discrimination compared to that of
the 7th edition and ENETS TNM system. This supports the clinical use of
the system.

{}

{}

\begin{center}\rule{0.5\linewidth}{\linethickness}\end{center}

 - \textbf{Defining the Role of Lymphadenectomy for Pancreatic
Neuroendocrine Tumors: An Eight-Institution Study of 695 Patients from
the US Neuroendocrine Tumor Study Group}

\emph{Annals of surgical oncology 2019 Aug;26(8):2517-2524}

PubMed: \url{https://www.ncbi.nlm.nih.gov/pubmed/?term=31004295}

BACKGROUND: Preoperative factors that reliably predict lymph node (LN)
metastases in pancreatic neuroendocrine tumors (PanNETs) are unclear.
The number of LNs needed to accurately stage PanNETs has not been
defined. METHODS: Patients who underwent curative-intent resection of
non-functional PanNETs at eight institutions from 2000 to 2016 were
analyzed. Preoperative factors associated with LN metastases were
identified. A procedure-specific target for LN retrieval to accurately
stage patients was determined. RESULTS: Of 695 patients who underwent
resection, 33\% of tumors were proximal (head/uncinate) and 67\% were
distal (neck/body/tail). Twenty-six percent of patients (n = 158) had
LN-positive disease, which was associated with a worse 5-year
recurrence-free survival (RFS; 60\% vs.~86\%; p \textless{} 0.001). The
increasing number of positive LNs was not associated with worse RFS.
Preoperative factors associated with positive LNs included tumor
size ≥ 2~cm (odds ratio {[}OR{]} 6.6; p \textless{} 0.001), proximal
location (OR 2.5; p \textless{} 0.001), moderate versus
well-differentiation (OR 2.1; p = 0.006), and Ki-67 ≥ 3\% (OR 3.1;
p \textless{} 0.001). LN metastases were also present in tumors without
these risk factors: \textless{} 2~cm (9\%), distal location (19\%),
well-differentiated (23\%), and Ki-67 \textless{} 3\% (16\%). Median LN
retrieval was 13 for pancreatoduodenectomy (PD), but only 9 for distal
pancreatectomy (DP). Given that PD routinely includes a complete
regional lymphadenectomy, a minimum number of LNs to accurately stage
patients was not identified. However, for DP, removal of less than seven
LNs failed to discriminate 5-year RFS between LN-positive and
LN-negative patients (less than seven LNs: 72\% vs.~83\%, p = 0.198;
seven or more LNs: 67\% vs.~86\%; p = 0.002). CONCLUSIONS: Tumor
size ≥ 2~cm, proximal location, moderate differentiation, and
Ki-67 ≥ 3\% are preoperative factors that predict LN positivity in
resected non-functional PanNETs. Given the 9-23\% incidence of LN
metastases in patients without such risk factors, routine regional
lymphadenectomy should be considered. PD inherently includes sufficient
LN retrieval, while DP should aim to remove seven or more LNs for
accurate staging.

{}

{}

\begin{center}\rule{0.5\linewidth}{\linethickness}\end{center}

Back to top

\begin{center}\rule{0.5\linewidth}{\linethickness}\end{center}

\pagebreak

\hypertarget{cytopathology}{%
\subsection{Cytopathology}\label{cytopathology}}

\begin{center}\rule{0.5\linewidth}{\linethickness}\end{center}

\hypertarget{pancreas-1}{%
\subsubsection{Pancreas}\label{pancreas-1}}

\begin{center}\rule{0.5\linewidth}{\linethickness}\end{center}

 - \textbf{Glycosylation of ascites-derived exosomal CD133: a potential
prognostic biomarker in patients with advanced pancreatic cancer}

\emph{Medical molecular morphology 2019 Feb;():}

PubMed: \url{https://www.ncbi.nlm.nih.gov/pubmed/?term=30805710}

Cancer cells surviving in ascites exhibit cancer stem cell (CSC)-like
features. This study analyzed the expression of the CSC marker CD133 in
the ascites-derived exosomes obtained from patients with unresectable
pancreatic cancer. In addition, inverse correlation of CD133 expression
with prognosis was examined. Of the 133 consecutive patients, 19
patients were enrolled in the study. Exosomes derived from the malignant
ascites demonstrated higher density and wider variation in size than
those from non-malignant ascites. Western blot revealed enhanced
expression of CD133 in exosomes obtained from patients with pancreatic
cancer compared to those obtained from patients with gastric cancer or
liver cirrhosis. A xenograft mouse model with malignant ascites was
established by intraperitoneal inoculation of human pancreatic cancer
cells in nude mice. Results obtained from the human study were
reproduced in the mouse model. Statistically significant equilateral
correlation was identified between the band intensity of CD133 in
western blot and overall survival of patients. Lectin microarray
analyses revealed glycosylation of CD133 by sialic acids as the major
glycosylation among diverse others responsible for the glycosylation of
exosomal CD133. These findings suggest that highly glycosylated CD133 in
ascites-derived exosomes as a potential biomarker for better prognosis
of patients with advanced pancreatic cancer.

{}

{}

\begin{center}\rule{0.5\linewidth}{\linethickness}\end{center}

 - \textbf{Assessment of CD133-positive extracellular membrane vesicles
in pancreatic cancer ascites and beyond}

\emph{Medical molecular morphology 2019 Apr;():}

PubMed: \url{https://www.ncbi.nlm.nih.gov/pubmed/?term=30953194}

{}

{}

\begin{center}\rule{0.5\linewidth}{\linethickness}\end{center}

 - \textbf{Intraoperative Peritoneal Washing Cytology on Survival in
Pancreatic Ductal Adenocarcinoma With Resectable, Locally Advanced, and
Metastatic Disease}

\emph{Pancreas 2019 04;48(4):519-525}

PubMed: \url{https://www.ncbi.nlm.nih.gov/pubmed/?term=30946232}

OBJECTIVES: The prognostic implications of intraoperative peritoneal
washing cytology (IPWC) in patients with pancreatic ductal
adenocarcinoma (PDAC) remains incompletely understood. METHODS: A
meta-analysis was conducted to investigate the impact of IPWC status on
the clinicopathologic features and survival outcomes in potentially
resectable, locally advanced, and metastatic PDAC. Hazard ratio (HR) and
95\% confidence interval (CI) were used as the pooled estimates.
RESULTS: A total of 12 studies qualified for inclusion with 3751 PDAC
patients. In resectable PDAC, the postoperative 5-year overall survival
was significantly better in negative IPWC than in positive IPWC
patients, with a pooled HR of 2.47 (95\% CI, 1.90-3.21; P \textless{}
0.001; I = 69\%) in a random-effects model. Likely, combined outcome
showed a significantly longer survival benefit in the negative IPWC
group (HR, 2.80; 95\% CI, 1.94-4.04; P \textless{} 0.001) in terms of
recurrence-free survival. The presence of positive IPWC did not
significantly alter survival outcomes in those PDAC patients with
locally advanced or metastatic disease. CONCLUSIONS: This systematic
review and meta-analysis demonstrated that a positive IPWC status in
patients with clinically resectable PDAC predicts a poor prognosis.
Patients with positive IPWC should be regarded as a specific subgroup,
with intensive adjuvant chemotherapy that seems to be warranted for
further evaluation.

{}

{}

\begin{center}\rule{0.5\linewidth}{\linethickness}\end{center}

 - \textbf{Addition of analysis of KRAS mutation or immunohistochemistry
with MUC1 and carcinoembryonic antigen improves the diagnostic
performance of fine needle aspiration cytology for the diagnosis of
pancreatic carcinoma}

\emph{Cytopathology : official journal of the British Society for
Clinical Cytology 2019 Mar;():}

PubMed: \url{https://www.ncbi.nlm.nih.gov/pubmed/?term=30929285}

BACKGROUND: Pancreatic adenocarcinoma (PAC) is a health problem because
of high lethality, increasing incidence and the absence of an early
diagnosis. Biopsy by fine needle aspiration guided by endoscopic
ultrasound has allowed obtaining tissue for cytopathological analysis,
but there are several problems with their interpretation. We aimed to
compare the diagnostic performance of the cytopathological analysis with
the addition of either an immunohistochemical (IHC) panel or the KRAS
mutation for the diagnosis of PAC. METHODS: We evaluated 62 pancreatic
lesions by fine needle aspiration guided by endoscopic ultrasound,
applying an IHC panel with mucin (MUC)-1, MUC4, carcinoembryonic antigen
(CEA) and p53. All cases also had a KRAS mutation determination. Three
cytopathologists blinded to clinical data and the KRAS status reviewed
the cytology independently. We calculated diagnostic performances for
the cytology alone, cytology+IHC and cytology+KRAS to show the best
method to diagnose PAC. RESULTS: From 62 samples, 50 (80.6\%) were PAC
and 12 benign lesions. The cytopathological analysis correctly
interpreted 26 malignant and 12 non-neoplastic cases (sensitivity 52\%,
specificity 100\% and diagnostic accuracy 61.3\%). The KRAS mutation was
present in 88\% of PAC. The cytology+ KRAS mutation increased the
sensitivity by 10\% and the diagnostic accuracy by 8\%. The sensitivity
increased by 2\% adding either MUC1 or CEA to the cytology, and the
diagnostic accuracy by 10 or 18\%, respectively. CONCLUSION: The
addition of IHC either with CEA or MUC1 improved the diagnostic
performance of the cytology alone to diagnose PAC. The cytology + IHC
evaluation was superior to the cytology + KRAS mutation to diagnose PAC.

{}

{}

\begin{center}\rule{0.5\linewidth}{\linethickness}\end{center}

 - \textbf{Ancillary Techniques in Cytologic Specimens Obtained from
Solid Lesions of the Pancreas: A Review}

\emph{Acta cytologica 2019 Apr;():1-21}

PubMed: \url{https://www.ncbi.nlm.nih.gov/pubmed/?term=30970350}

Advanced methods of molecular characterization have elucidated the
genetic, epigenetic, and proteomic alterations associated with the broad
spectrum of pancreatic disease, particularly neoplasia. Next-generation
sequencing, in particular, has revealed the genomic diversity among
pancreatic ductal adenocarcinoma, neuroendocrine and acinar tumors,
solid pseudopapillary neoplasm, and other pancreatico-biliary neoplasms.
Differentiating these entities from one another by morphologic analysis
alone may be challenging, especially when examining the small quantities
of diagnostic material inherent to cytologic specimens. In order to
enhance the sensitivity and specificity of pancreatic cytomorphology,
multiple diagnostic, prognostic, and predictive ancillary tests have
been and continue to be developed. Although a great number of such tests
have been developed for evaluation of specimens collected from cystic
lesions and strictures, ancillary techniques also play a significant
role in the evaluation of cytologic specimens obtained from solid
lesions of the pancreas. Furthermore, while some tests have been
developed to differentiate diagnostic entities from one another, others
have been developed to simply identify dysplasia and malignancy.
Ancillary studies are particularly important in the subset of cases for
which cytomorphologic analysis provides a result that is equivocal or
insufficient to guide clinical management. Selection of appropriate
ancillary testing modalities requires familiarity with both their
methodology and the molecular basis of the pancreatic diseases for which
testing is being performed.

{}

{}

\begin{center}\rule{0.5\linewidth}{\linethickness}\end{center}

Back to top

\begin{center}\rule{0.5\linewidth}{\linethickness}\end{center}

\pagebreak

\hypertarget{bile-ducts-1}{%
\subsubsection{Bile Ducts}\label{bile-ducts-1}}

\begin{center}\rule{0.5\linewidth}{\linethickness}\end{center}

 - \textbf{Improving the diagnostic accuracy of biliary cytology}

\emph{Diagnostic cytopathology 2019 Jul;47(7):639-640}

PubMed: \url{https://www.ncbi.nlm.nih.gov/pubmed/?term=31041845}

{}

{}

\begin{center}\rule{0.5\linewidth}{\linethickness}\end{center}

 - \textbf{Washing cytology of removed self-expandable metal stent for
biliary stricture: A novel cytology technique}

\emph{Diagnostic cytopathology 2019 Jul;47(7):743-745}

PubMed: \url{https://www.ncbi.nlm.nih.gov/pubmed/?term=31059182}

{}

{}

\begin{center}\rule{0.5\linewidth}{\linethickness}\end{center}

Back to top

\begin{center}\rule{0.5\linewidth}{\linethickness}\end{center}

\pagebreak

\hypertarget{neuroendocrine-1}{%
\subsubsection{Neuroendocrine}\label{neuroendocrine-1}}

\begin{center}\rule{0.5\linewidth}{\linethickness}\end{center}

 - \textbf{Neuroendocrine Tumors (NETs) of the Minor Papilla/Ampulla:
Analysis of 16 Cases Underlines Homology With Major Ampulla NETs and
Differences From Extra-Ampullary Duodenal NETs}

\emph{The American journal of surgical pathology 2019 Jun;43(6):725-736}

PubMed: \url{https://www.ncbi.nlm.nih.gov/pubmed/?term=30913089}

Neuroendocrine tumors (NETs) of the minor papilla/ampulla (MIPA) are
rare and poorly studied. Only individual case reports and no
comprehensive analysis are available from the literature. We collected
16 MIPA NETs and investigated their clinicopathologic and
immunohistochemical features, including markers such as somatostatin,
pancreatic polypeptide, gastrin, serotonin, MUC1, cytokeratin 7, and
somatostatin receptors type 2A and 5. The median age at diagnosis was
57.5 years, and the female-to-male ratio was 2.2:1. The median NET size
was 1.45 cm, and most (94\%) were low-grade (G1) tumors. Similarly to
what was observed in the major ampulla, 3 histotypes were found: (i)
ampullary-type somatostatin-producing tumors (ASTs, 10 cases),
characterized by somatostatin expression in most tumor cells,
focal-to-extensive tubulo-acinar structures, often with psammoma bodies,
MUC1 reactivity, and no or rare membranous reactivity for somatostatin
receptor type 2A; (ii) gangliocytic paragangliomas (3 cases),
characterized by the coexistence of 3 tumor cell types: epithelioid,
often reactive for pancreatic polypeptide, ganglion-like cells, and S100
reactive sustentacular/stromal cells; and (iii) ordinary nonfunctioning
NETs (3 cases), resembling those more commonly observed in the
extra-ampullary duodenum. Comparable histotypes could also be recognized
among the 30 MIPA NETs from the literature. No NET-related patient death
among MIPA cases was observed during a median follow-up of 38 months;
however, MIPA ASTs showed lymph node metastases and invasion of the
duodenal muscularis propria or beyond in 44\% and 40\% of cases,
respectively. In conclusion, MIPA NETs closely resemble tumors arising
in the major ampulla, with predominance of ASTs.

{}

{}

\begin{center}\rule{0.5\linewidth}{\linethickness}\end{center}

Back to top

\begin{center}\rule{0.5\linewidth}{\linethickness}\end{center}

\pagebreak

\hypertarget{molecular-pathology}{%
\subsection{Molecular Pathology}\label{molecular-pathology}}

\begin{center}\rule{0.5\linewidth}{\linethickness}\end{center}

\hypertarget{pancreas-2}{%
\subsubsection{Pancreas}\label{pancreas-2}}

\begin{center}\rule{0.5\linewidth}{\linethickness}\end{center}

 - \textbf{A Multiscale Map of the Stem Cell State in Pancreatic
Adenocarcinoma}

\emph{Cell 2019 Apr;177(3):572-586.e22}

PubMed: \url{https://www.ncbi.nlm.nih.gov/pubmed/?term=30955884}

Drug resistance and relapse remain key challenges in~pancreatic cancer.
Here, we have used RNA sequencing (RNA-seq), chromatin
immunoprecipitation (ChIP)-seq, and genome-wide CRISPR analysis to map
the molecular dependencies of pancreatic cancer stem cells, highly
therapy-resistant cells that preferentially drive tumorigenesis and
progression. This integrated genomic approach revealed an unexpected
utilization of immuno-regulatory signals by pancreatic cancer epithelial
cells. In particular, the nuclear hormone receptor
retinoic-acid-receptor-related orphan receptor gamma (RORγ), known to
drive inflammation and T~cell differentiation, was upregulated during
pancreatic cancer progression, and its genetic or pharmacologic
inhibition led to a striking defect in pancreatic cancer growth and a
marked improvement in survival. Further, a large-scale retrospective
analysis in patients revealed that RORγ expression may predict
pancreatic cancer aggressiveness, as it positively correlated with
advanced disease and metastasis. Collectively, these data identify an
orthogonal co-option of immuno-regulatory signals by pancreatic cancer
stem cells, suggesting that autoimmune drugs should be evaluated as
novel treatment strategies for pancreatic cancer patients.

{}

{}

\begin{center}\rule{0.5\linewidth}{\linethickness}\end{center}

 - \textbf{Syndecan 1 is a critical mediator of macropinocytosis in
pancreatic cancer}

\emph{Nature 2019 Apr;568(7752):410-414}

PubMed: \url{https://www.ncbi.nlm.nih.gov/pubmed/?term=30918400}

Pancreatic ductal adenocarcinoma (PDAC) remains recalcitrant to all
forms of cancer treatment and carries a five-year survival rate of only
8\%1. Inhibition of oncogenic KRAS (hereafter KRAS\emph{), the earliest
lesion in disease development that is present in more than 90\% of
PDACs, and its signalling surrogates has yielded encouraging preclinical
results with experimental agents2-4. However, KRAS}-independent disease
recurrence following genetic extinction of Kras* in mouse models
anticipates the need for co-extinction strategies5,6. Multiple oncogenic
processes are initiated at the cell surface, where KRAS* physically and
functionally interacts to direct signalling~that is essential for
malignant transformation and tumour maintenance. Insights into the
complexity of the functional cell-surface-protein repertoire
(surfaceome) have been technologically limited until recently and-in the
case of PDAC-the genetic control of the function and composition of the
PDAC surfaceome in the context of KRAS* signalling remains largely
unknown. Here we develop an unbiased, functional target-discovery
platform to query KRAS\emph{-dependent changes of the PDAC surfaceome,
which reveals syndecan 1 (SDC1, also known as CD138) as a protein that
is upregulated at the cell surface by KRAS}. Localization of SDC1 at the
cell surface-where it regulates macropinocytosis, an essential metabolic
pathway that fuels PDAC cell growth-is essential for disease maintenance
and progression. Thus, our study forges a mechanistic link between KRAS*
signalling and a targetable molecule driving nutrient salvage pathways
in PDAC and validates oncogene-driven surfaceome annotation as a
strategy to identify cancer-specific vulnerabilities.

{}

{}

\begin{center}\rule{0.5\linewidth}{\linethickness}\end{center}

 - \textbf{Ring1b-dependent epigenetic remodelling is an essential
prerequisite for pancreatic carcinogenesis}

\emph{Gut 2019 Apr;():}

PubMed: \url{https://www.ncbi.nlm.nih.gov/pubmed/?term=30954952}

BACKGROUND AND AIMS: Besides well-defined genetic alterations, the
dedifferentiation of mature acinar cells is an important prerequisite
for pancreatic carcinogenesis. Acinar-specific genes controlling cell
homeostasis are extensively downregulated during cancer development;
however, the underlying mechanisms are poorly understood. Now, we
devised a novel in vitro strategy to determine genome-wide dynamics in
the epigenetic landscape in pancreatic carcinogenesis. DESIGN: With our
in vitro carcinogenic sequence, we performed global gene expression
analysis and ChIP sequencing for the histone modifications H3K4me3,
H3K27me3 and H2AK119ub. Followed by a comprehensive bioinformatic
approach, we captured gene clusters with extensive epigenetic and
transcriptional remodelling. Relevance of Ring1b-catalysed H2AK119ub in
acinar cell reprogramming was studied in an inducible Ring1b knockout
mouse model. CRISPR/Cas9-mediated Ring1b ablation as well as
drug-induced Ring1b inhibition were functionally characterised in
pancreatic cancer cells. RESULTS: The epigenome is vigorously modified
during pancreatic carcinogenesis, defining cellular identity.
Particularly, regulatory acinar cell transcription factors are
epigenetically silenced by the Ring1b-catalysed histone modification
H2AK119ub in acinar-to-ductal metaplasia and pancreatic cancer cells.
Ring1b knockout mice showed greatly impaired acinar cell
dedifferentiation and pancreatic tumour formation due to a retained
expression of acinar differentiation genes. Depletion or drug-induced
inhibition of Ring1b promoted tumour cell reprogramming towards a less
aggressive phenotype. CONCLUSIONS: Our data provide substantial evidence
that the epigenetic silencing of acinar cell fate genes is a mandatory
event in the development and progression of pancreatic cancer. Targeting
the epigenetic repressor Ring1b could offer new therapeutic options.

{}

{}

\begin{center}\rule{0.5\linewidth}{\linethickness}\end{center}

 - \textbf{Evaluation of the prognostic significances of γ-secretase
genes in pancreatic cancer}

\emph{Oncology letters 2019 May;17(5):4614-4620}

PubMed: \url{https://www.ncbi.nlm.nih.gov/pubmed/?term=30944650}

With the growing requirement for novel prognostic biomarkers for
pancreatic cancer, many studies have focused on clinical and/or genomic
variables. Although many studies have been performed, carbohydrate
antigen 19-9 is the only biomarker in clinical use. Therefore, the
present study examined whether γ-secretase genes, including presenilin
(PSEN), nicastrin (NCSTN), presenilin enhancer protein 2 (PSENEN), and
anterior pharynx-defective 1 (APH1-), could serve as prognostic factors
for pancreatic cancer. The cohorts selected included \textgreater{}100
pancreatic cancer patients. Patient data were downloaded from The Cancer
Genome Atlas (TCGA) and Gene Expression Omnibus (GSE21501). The
prognostic roles of the γ-secretase genes were analyzed by several
survival analysis methods. Among the γ-secretase genes, the prognosis
tended to be worse in the 2 cohorts with increasing expression of PSEN1,
APH1A, and PSENEN, while the remaining genes were the opposite in the 2
cohorts. Notably, although the patient characteristics were quite
different, APH1A was statistically significantly associated with
prognosis in the 2 cohorts. The hazard ratio of APH1A for overall
survival was 1.598 (TCGA) and 2.724 (GSE21501). These results contribute
to the study of γ-secretase in pancreatic cancer. We believe that
γ-secretase, particularly APH1A, will be a new prognostic biomarker for
pancreatic cancer.

{}

{}

\begin{center}\rule{0.5\linewidth}{\linethickness}\end{center}

 - \textbf{Hear Pancreatic Cancer Stem Cells ROR}

\emph{Cell 2019 Apr;177(3):516-518}

PubMed: \url{https://www.ncbi.nlm.nih.gov/pubmed/?term=31002791}

In this issue of Cell, Lytle et~al. (2019) integrate functional genomic
approaches to identify molecular dependencies of pancreatic cancer stem
cells that may be exploited therapeutically. The comprehensive analysis
reveals an unexpected role for retinoic acid receptor-related orphan
receptor gamma (RORγ), a T-cell-associated transcription factor, in
defining the stemness and the aggressive behavior of pancreatic cancer.

{}

{}

\begin{center}\rule{0.5\linewidth}{\linethickness}\end{center}

 - \textbf{MiR-539 functions as a tumor suppressor in pancreatic cancer
by targeting TWIST1}

\emph{Experimental and molecular pathology 2019 Jun;108():143-149}

PubMed: \url{https://www.ncbi.nlm.nih.gov/pubmed/?term=31022384}

The dysregulation of microRNA (miRNA) expression has been highlighted in
a variety of human malignant conditions with reports implicating a
critical role in the process of tumor growth. The role of miR-539 in
pancreatic cancer (PC) is yet to be fully elucidated, hence the aim of
the current study was to investigate the effect of miR-539 expression in
relation to a cohort of 52 PC specimens. The application of a real-time
quantitative polymerase chain reaction (qRT-PCR) revealed a
significantly down-regulated miR-539 level, which was accompanied by an
increased TWIST1 expression in PC when compared with the controls. The
in vitro experiment results demonstrated that the endogenic mimic of
miR-539 significantly suppressed the growth of the xenograft tumors in
PANC-1 cells, when compared to the delivery of the control miRNA and
blank control. Meanwhile, the key epithelial-mesenchymal transition
(EMT) inducer, TWIST1 was verified as a direct target gene of miR-539
through the application of a luciferase reporter assay. In conclusion,
the results of the current study present evidence emphasizing the
significance of the interactions between miR-539 and TWIST1 in the
development of and progression of PC, highlighting its potential as a
therapeutic target in the treatment of PC patients.

{}

{}

\begin{center}\rule{0.5\linewidth}{\linethickness}\end{center}

 - \textbf{The ERBB receptor inhibitor dacomitinib suppresses
proliferation and invasion of pancreatic ductal adenocarcinoma cells}

\emph{Cellular oncology (Dordrecht) 2019 Aug;42(4):491-504}

PubMed: \url{https://www.ncbi.nlm.nih.gov/pubmed/?term=31025257}

PURPOSE: Pancreatic ductal adenocarcinoma (PDAC), the most common
malignancy of the pancreas, is the fourth most common cause of
cancer-related death in the USA. Local progression, early tumor
dissemination and low efficacy of current treatments are the major
reasons for its high mortality rate. The ERBB family is over-expressed
in PDAC and plays essential roles in its tumorigenesis; however,
single-targeted ERBB inhibitors have shown limited activity in this
disease. Here, we examined the anti-tumor activity of dacomitinib, a
pan-ERBB receptor inhibitor, on PDAC cells. METHODS: Anti-proliferative
effects of dacomitinib were determined using a cell proliferation assay
and crystal violet staining. Annexin V/PI staining, radiation therapy
and cell migration and invasion assays were carried out to examine the
effects of dacomitinib on apoptosis, radio-sensitivity and cell
motility, respectively. Quantitative reverse transcription-PCR (qRT-PCR)
and Western blot analyses were applied to elucidate the molecular
mechanisms underlying the anti-tumor activity of dacomitinib. RESULTS:
We found that dacomitinib diminished PDAC cell proliferation via
inhibition of FOXM1 and its targets Aurora kinase B and cyclin B1.
Moreover, we found that dacomitinib induced apoptosis and potentiated
radio-sensitivity via inhibition of the anti-apoptotic proteins survivin
and MCL1. Treatment with dacomitinib attenuated cell migration and
invasion through inhibition of the epithelial-to-mesenchymal transition
(EMT) markers ZEB1, Snail and N-cadherin. In contrast, we found that the
anti-tumor activity of single-targeted ERBB agents including cetuximab
(anti-EGFR mAb), trastuzumab (anti-HER2 mAb), H3.105.5 (anti-HER3 mAb)
and erlotinib (EGFR small molecule inhibitor) were marginal.
CONCLUSIONS: Our findings indicate that dacomitinib-mediated blockade of
the ERBB receptors yields advantages over single-targeted ERBB
inhibition and provide a rationale for further investigation of the
therapeutic potential of dacomitinib in the treatment of ERBB-driven
PDAC.

{}

{}

\begin{center}\rule{0.5\linewidth}{\linethickness}\end{center}

 - \textbf{Regulation of pH by Carbonic Anhydrase 9 Mediates Survival of
Pancreatic Cancer Cells With Activated KRAS in Response to Hypoxia}

\emph{Gastroenterology 2019 May;():}

PubMed: \url{https://www.ncbi.nlm.nih.gov/pubmed/?term=31078621}

BACKGROUND \& AIMS: Most pancreatic ductal adenocarcinomas (PDACs)
express an activated form of KRAS, become hypoxic and dysplastic, and
are refractory to chemo and radiation therapies. To survive in the
hypoxic environment, PDAC cells upregulate enzymes and transporters
involved in pH regulation, including the extracellular facing carbonic
anhydrase 9 (CA9). We evaluated the effect of blocking CA9, in
combination with administration of gemcitabine, in mouse models of
pancreatic cancer. METHODS: We knocked down expression of KRAS in human
(PK-8 and PK-1) PDAC cells with small hairpin RNAs. Human and mouse
(KrasG12D/Pdx1-Cre/Tp53/RosaYFP) PDAC cells were incubated with
inhibitors of MEK (trametinib) or extracellular signal-regulated kinase
(ERK), and some cells were cultured under hypoxic conditions. We
measured levels and stability of the hypoxia-inducible factor 1 subunit
alpha (HIF1A), endothelial PAS domain 1 protein (EPAS1, also called
HIF2A), CA9, solute carrier family 16 member 4 (SLC16A4, also called
MCT4), and SLC2A1 (also called GLUT1) by immunoblot analyses. We
analyzed intracellular pH (pHi) and extracellular metabolic flux. We
knocked down expression of CA9 in PDAC cells, or inhibited CA9 with
SLC-0111, incubated them with gemcitabine, and assessed pHi, metabolic
flux, and cytotoxicity under normoxic and hypoxic conditions. Cells were
also injected into either immune-compromised or immune-competent mice
and growth of xenograft tumors was assessed. Tumor fragments derived
from patients with PDAC were surgically ligated to the pancreas of mice
and the growth of tumors was assessed. We performed tissue microarray
analyses of 205 human PDAC samples to measure levels of CA9 and
associated expression of genes that regulate hypoxia with outcomes of
patients using the Cancer Genome Atlas database. RESULTS: Under hypoxic
conditions, PDAC cells had increased levels of HIF1A and endothelial PAS
domain 1 protein (EPAS1, also called HIF2A), upregulated expression of
CA9, and activated glycolysis. Knockdown of KRAS in PDAC cells, or
incubation with trametinib, reduced the posttranscriptional
stabilization of HIF1A and HIF2A, upregulation of CA9, pHi, and
glycolysis in response to hypoxia. CA9 was expressed by 66\% of PDAC
samples analyzed; high expression of genes associated with metabolic
adaptation to hypoxia, including CA9, correlated with significantly
reduced survival times of patients. Knockdown or pharmacologic
inhibition of CA9 in PDAC cells significantly reduced pHi in cells under
hypoxic conditions, decreased gemcitabine-induced glycolysis, and
increased their sensitivity to gemcitabine. PDAC cells with knockdown of
CA9 formed smaller xenograft tumors in mice, and injection of
gemcitabine inhibited tumor growth and significantly increased survival
times of mice. In mice with xenograft tumors grown from human PDAC
cells, oral administration of SLC-0111 and injection of gemcitabine
increased intratumor acidosis and increased cell death. These tumors,
and tumors grown from PDAC patient-derived tumor fragments, grew more
slowly than xenograft tumors in mice given control agents, resulting in
longer survival times. In KrasG12D/Pdx1-Cre/Tp53/RosaYFP genetically
modified mice, oral administration of SLC-0111 and injection of
gemcitabine reduced numbers of B cells in tumors. CONCLUSIONS: In
response to hypoxia, PDAC cells that express activated KRAS increase
expression of CA9, via stabilization of HIF1A and HIF2A, to regulate pH
and glycolysis. Disruption of this pathway slows growth of PDAC
xenograft tumors in mice and might be developed for treatment of
pancreatic cancer.

{}

{}

\begin{center}\rule{0.5\linewidth}{\linethickness}\end{center}

Back to top

\begin{center}\rule{0.5\linewidth}{\linethickness}\end{center}

\pagebreak

\hypertarget{pancreatitis-other-diseases}{%
\paragraph{Pancreatitis \& Other
Diseases}\label{pancreatitis-other-diseases}}

Molecular Studies on Pancreatitis \& Other Diseases

\begin{center}\rule{0.5\linewidth}{\linethickness}\end{center}

 - \textbf{Novel p.K374E variant of CPA1 causes misfolding-induced
hereditary pancreatitis with autosomal dominant inheritance}

\emph{Gut 2019 Apr;():}

PubMed: \url{https://www.ncbi.nlm.nih.gov/pubmed/?term=31005883}

{}

{}

\begin{center}\rule{0.5\linewidth}{\linethickness}\end{center}

Back to top

\begin{center}\rule{0.5\linewidth}{\linethickness}\end{center}

\pagebreak

\hypertarget{molecular-research-on-microenvironment}{%
\paragraph{Molecular Research on
Microenvironment}\label{molecular-research-on-microenvironment}}

Tumor Stroma Interactions, Microenvironment, Inflammatory Response,
Microbiome

\begin{center}\rule{0.5\linewidth}{\linethickness}\end{center}

 - \textbf{Mobilization of CD8+ T Cells via CXCR4 Blockade Facilitates
PD-1 Checkpoint Therapy in Human Pancreatic Cancer}

\emph{Clinical cancer research : an official journal of the American
Association for Cancer Research 2019 Jul;25(13):3934-3945}

PubMed: \url{https://www.ncbi.nlm.nih.gov/pubmed/?term=30940657}

PURPOSE: Pancreatic ductal adenocarcinoma (PDA) is rarely cured, and
single-agent immune checkpoint inhibition has not demonstrated clinical
benefit despite the presence of large numbers of CD8+ T cells. We
hypothesized that tumor-infiltrating CD8+ T cells harbor latent
antitumor activity that can be reactivated using combination
immunotherapy. EXPERIMENTAL DESIGN: Preserved human PDA specimens were
analyzed using multiplex IHC (mIHC) and T-cell receptor (TCR)
sequencing. Fresh tumor was treated in organotypic slice culture to test
the effects of combination PD-1 and CXCR4 blockade. Slices were analyzed
using IHC, flow cytometry, and live fluorescent microscopy to assess
tumor kill, in addition to T-cell expansion and mobilization. RESULTS:
mIHC demonstrated fewer CD8+ T cells in juxtatumoral stroma containing
carcinoma cells than in stroma devoid of them. Using TCR sequencing, we
found clonal expansion in each tumor; high-frequency clones had multiple
DNA rearrangements coding for the same amino acid binding sequence,
which suggests response to common tumor antigens. Treatment of fresh
human PDA slices with combination PD-1 and CXCR4 blockade led to
increased tumor cell death concomitant with lymphocyte expansion. Live
microscopy after combination therapy demonstrated CD8+ T-cell migration
into the juxtatumoral compartment and rapid increase in tumor cell
apoptosis. CONCLUSIONS: Endogenous tumor-reactive T cells are present
within the human PDA tumor microenvironment and can be reactivated by
combined blockade of PD-1 and CXCR4. This provides a new basis for the
rational selection of combination immunotherapy for PDA.See related
commentary by Medina and Miller, p.~3747.

{}

{}

\begin{center}\rule{0.5\linewidth}{\linethickness}\end{center}

 - \textbf{L1CAM induces perineural invasion of pancreas cancer cells by
upregulation of metalloproteinase expression}

\emph{Oncogene 2019 01;38(4):596-608}

PubMed: \url{https://www.ncbi.nlm.nih.gov/pubmed/?term=30171263}

Pancreas cancer cells have a tendency to invade along nerves. Such
cancerous nerve invasion (CNI) is associated with poor outcome; however,
the exact mechanism that drives cancer cells to disseminate along nerves
is unknown. Immunohistochemical analysis of human pancreatic ductal
adenocarcinoma (PDAC) specimens showed overexpression of the L1 cell
adhesion molecule (L1CAM) in cancer cells and in adjacent Schwann cells
(SC) in invaded nerves. By modeling the neural microenvironment, we
found that L1CAM secreted from SCs acts as a strong chemoattractant to
cancer cells, through activation of MAP kinase signaling. L1CAM also
upregulated expression of metalloproteinase-2 (MMP-2) and MMP-9 by PDAC
cells, through STAT3 activation. Using a transgenic Pdx-1-Cre/KrasG12D
/p53R172H (KPC) mouse model, we show that treatment with anti-L1CAM Ab
significantly reduces CNI in vivo. We provide evidence of a paracrine
response between SCs and cancer cells in the neural niche, which
promotes cancer invasion via L1CAM secretion.

{}

{}

\begin{center}\rule{0.5\linewidth}{\linethickness}\end{center}

 - \textbf{Targeting LIF-mediated paracrine interaction for pancreatic
cancer therapy and monitoring}

\emph{Nature 2019 05;569(7754):131-135}

PubMed: \url{https://www.ncbi.nlm.nih.gov/pubmed/?term=30996350}

Pancreatic ductal adenocarcinoma (PDAC) has a dismal prognosis largely
owing to inefficient diagnosis and tenacious drug resistance. Activation
of pancreatic stellate cells (PSCs) and consequent development of dense
stroma are prominent features accounting for this aggressive biology1,2.
The reciprocal interplay between PSCs and pancreatic cancer cells (PCCs)
not only enhances tumour progression and metastasis but also sustains
their own activation, facilitating a vicious cycle to exacerbate
tumorigenesis and drug resistance3-7. Furthermore, PSC activation occurs
very early during PDAC tumorigenesis8-10, and activated PSCs comprise a
substantial fraction of the tumour mass, providing a rich source of
readily detectable factors. Therefore, we hypothesized that the
communication between PSCs and PCCs could be an exploitable target to
develop effective strategies for PDAC therapy and diagnosis. Here,
starting with a~systematic proteomic investigation of secreted disease
mediators and underlying molecular mechanisms, we reveal that leukaemia
inhibitory factor (LIF) is a key paracrine factor from activated PSCs
acting on cancer cells. Both pharmacologic LIF blockade and genetic Lifr
deletion markedly slow tumour progression and augment the~efficacy of
chemotherapy to prolong survival of PDAC mouse models, mainly by
modulating cancer cell differentiation and epithelial-mesenchymal
transition status. Moreover, in both mouse models and human PDAC,
aberrant production of LIF in the pancreas is restricted to pathological
conditions and correlates with PDAC pathogenesis, and changes in the
levels of~circulating LIF correlate well with tumour response to
therapy. Collectively, these findings reveal a function of LIF in PDAC
tumorigenesis, and suggest its translational potential as an attractive
therapeutic target and circulating marker. Our studies underscore how a
better understanding of cell-cell communication within the tumour
microenvironment can suggest novel strategies for cancer therapy.

{}

{}

\begin{center}\rule{0.5\linewidth}{\linethickness}\end{center}

Back to top

\begin{center}\rule{0.5\linewidth}{\linethickness}\end{center}

\pagebreak

\hypertarget{preneoplastic-and-preinvasive-lesions-1}{%
\paragraph{Preneoplastic and Preinvasive
Lesions}\label{preneoplastic-and-preinvasive-lesions-1}}

Molecular Pathology Preneoplastic and Preinvasive Lesions, PanIN, IPMN,
MCN, ICPN, IOPN

\begin{center}\rule{0.5\linewidth}{\linethickness}\end{center}

 - \textbf{GNAS but Not Extended RAS Mutations Spectrum are Associated
with a Better Prognosis in Intraductal Pancreatic Mucinous Neoplasms}

\emph{Annals of surgical oncology 2019 Aug;26(8):2640-2650}

PubMed: \url{https://www.ncbi.nlm.nih.gov/pubmed/?term=31025231}

BACKGROUND: The management of intraductal papillary mucinous neoplasms
(IPMNs) is mainly based on imaging features and clinical symptoms, and
remains challenging. OBJECTIVE: The aim of this study was to assess
GNAS, RAS family (KRAS, NRAS and HRAS), BRAF, and PIK3CA mutation status
in resected IPMNs and correlate it with clinicopathological
characteristics and patient survival. METHODS: Overall, 149 consecutive
unselected patients who underwent pancreatectomy for IPMNs were
included. After dissection from formalin-fixed and paraffin-embedded
tumors, GNAS mutational screening was assessed by allelic discrimination
using Taqman® probes and confirmed by SNaPshot analysis. RAS family,
BRAF, and PIK3CA mutational screening was assessed by high resolution
melt and Sanger sequencing. RESULTS: Gastric- and intestinal-type IPMNs
were the most frequent lesions (52\% and 41\%, respectively).
Intestinal-type IPMNs were more frequently associated high-grade
dysplasia (49\%) and were the only IPMNs associated with colloid-type
carcinoma. All pancreatobiliary IPMNs were invasive lesions, located in
the main pancreatic duct. GNAS-activating mutations were strongly
associated with the intestinal phenotype (p \textless{} 10-4), while RAS
pathway mutations were not associated with any particular phenotype.
Mutations within other members of the epidermal growth factor receptor
(EGFR) pathway were very rare (2\%). GNAS-mutated IPMNs were rarely
invasive (11\%) and almost exclusively (83\%) of the colloid type. For
invasive lesions, multivariate analyses determined that only node
negativity was associated with improved cancer-specific survival, but,
in univariate analysis, GNAS mutation was associated with prolonged
survival. CONCLUSION: In patients selected for surgery, GNAS mutation
analysis and tumor phenotype can help to better predict patient
prognosis. In the near future, a more precise mutational analysis of
IPMNs might help to better tailor their management.

{}

{}

\begin{center}\rule{0.5\linewidth}{\linethickness}\end{center}

 - \textbf{Prevalence of Germline Mutations Associated With Cancer Risk
in Patients With Intraductal Papillary Mucinous Neoplasms}

\emph{Gastroenterology 2019 05;156(6):1905-1913}

PubMed: \url{https://www.ncbi.nlm.nih.gov/pubmed/?term=30716324}

BACKGROUND \& AIMS: Many patients with pancreatic adenocarcinoma carry
germline mutations associated with increased risk of cancer. It is not
clear whether patients with intraductal papillary mucinous neoplasms
(IPMNs), which are precursors to some pancreatic cancers, also carry
these mutations. We assessed the prevalence of germline mutations
associated with cancer risk in patients with histologically confirmed
IPMN. METHODS: We obtained nontumor tissue samples from 315 patients
with surgically resected IPMNs from 1997 through 2017, and we sequenced
94 genes with variants associated with cancer risk. Mutations associated
with increased risk of cancer were identified and compared with
individuals from the Exome Aggregation Consortium. RESULTS: We
identified 23 patients with a germline mutation associated with cancer
risk (7.3\%; 95\% confidence interval, 4.9-10.8). Nine patients had a
germline mutation associated with pancreatic cancer susceptibility
(2.9\%; 95\% confidence interval, 1.4-5.4). More patients with IPMNs
carried germline mutations in ATM (P \textless{} .0001), PTCH1 (P
\textless{} .0001), and SUFU (P \textless{} .0001) compared with
controls. Patients with IPMNs and germline mutations associated with
pancreatic cancer were more like to have concurrent invasive pancreatic
carcinoma compared with patients with IPMNs without these mutations (P
\textless{} .0320). CONCLUSIONS: In sequence analyses of 315 patients
with surgically resected IPMNs, we found that almost 3\% to carry
mutations associated with pancreatic cancer risk. More patients with
IPMNs and germline mutations associated with pancreatic cancer had
concurrent invasive pancreatic carcinoma compared with patients with
IPMNs without these mutations. Genetic analysis of patients with IPMNs
might identify those at greatest risk for cancer.

{}

{}

\begin{center}\rule{0.5\linewidth}{\linethickness}\end{center}

Back to top

\begin{center}\rule{0.5\linewidth}{\linethickness}\end{center}

\pagebreak

\hypertarget{bile-ducts-2}{%
\subsubsection{Bile Ducts}\label{bile-ducts-2}}

\begin{center}\rule{0.5\linewidth}{\linethickness}\end{center}

 - \textbf{Integrating next-generation sequencing to endoscopic
retrograde cholangiopancreatography (ERCP)-obtained biliary specimens
improves the detection and management of patients with malignant bile
duct strictures}

\emph{Gut 2019 Apr;():}

PubMed: \url{https://www.ncbi.nlm.nih.gov/pubmed/?term=30971436}

OBJECTIVE: Despite improvements in imaging, serum CA19-9 and
pathological evaluation, differentiating between benign and malignant
bile duct strictures remains a diagnostic conundrum. Recent developments
in next-generation sequencing (NGS) have opened new opportunities for
early detection and management of cancers but, to date, have not been
rigorously applied to biliary specimens. DESIGN: We prospectively
evaluated a 28-gene NGS panel (BiliSeq) using endoscopic retrograde
cholangiopancreatography-obtained biliary specimens from patients with
bile duct strictures. The diagnostic performance of serum CA19-9,
pathological evaluation and BiliSeq was assessed on 252 patients (57
trainings and 195 validations) with 346 biliary specimens. RESULTS: The
sensitivity and specificity of BiliSeq for malignant strictures was 73\%
and 100\%, respectively. In comparison, an elevated serum CA19-9 and
pathological evaluation had sensitivities of 76\% and 48\%, and
specificities of 69\% and 99\%, respectively. The combination of BiliSeq
and pathological evaluation increased the sensitivity to 83\% and
maintained a specificity of 99\%. BiliSeq improved the sensitivity of
pathological evaluation for malignancy from 35\% to 77\% for biliary
brushings and from 52\% to 83\% for biliary biopsies. Among patients
with primary sclerosing cholangitis (PSC), BiliSeq had an 83\%
sensitivity as compared with pathological evaluation with an 8\%
sensitivity. Therapeutically relevant genomic alterations were
identified in 20 (8\%) patients. Two patients with ERBB2-amplified
cholangiocarcinoma received a trastuzumab-based regimen and had
measurable clinicoradiographic response. CONCLUSIONS: The combination of
BiliSeq and pathological evaluation of biliary specimens increased the
detection of malignant strictures, particularly in patients with PSC.
Additionally, BiliSeq identified alterations that may stratify patients
for specific anticancer therapies.

{}

{}

\begin{center}\rule{0.5\linewidth}{\linethickness}\end{center}

Back to top

\begin{center}\rule{0.5\linewidth}{\linethickness}\end{center}

\pagebreak

\hypertarget{ampulla-of-vater-1}{%
\subsubsection{Ampulla of Vater}\label{ampulla-of-vater-1}}

\begin{center}\rule{0.5\linewidth}{\linethickness}\end{center}

 - \textbf{Diagnosis, risk stratification, and management of ampullary
dysplasia by DNA flow cytometric analysis of paraffin-embedded tissue}

\emph{Modern pathology : an official journal of the United States and
Canadian Academy of Pathology, Inc 2019 Apr;():}

PubMed: \url{https://www.ncbi.nlm.nih.gov/pubmed/?term=30976103}

The limited accuracy of endoscopic biopsy in detecting high-grade
dysplasia or adenocarcinoma within ampullary adenoma or dysplasia has
been reported. The natural history of ampullary dysplasia is also
unclear, and there are no established guidelines to determine which
patients with ampullary dysplasia require resection versus surveillance
endoscopy. DNA flow cytometry was performed on 47 ampullary biopsies
with low-grade dysplasia, 18 high-grade dysplasia, and 23 negative for
dysplasia, as well as 11 cases of ampullary adenocarcinoma. Abnormal DNA
content (aneuploidy or elevated 4N fraction \textgreater{} 6\%) was
identified in 9 (82\%) of adenocarcinoma, 13 (72\%) of high-grade
dysplasia, 7 (15\%) of low-grade dysplasia, and none (0\%) of
non-dysplastic mucosa. One-, 2-, and 7-year detection rates of
high-grade dysplasia or adenocarcinoma in low-grade dysplasia patients
with abnormal DNA content were 57\%, 86\%, and 88\%, respectively,
whereas low-grade dysplasia patients in the setting of normal DNA
content had 1-, 2-, and 7-year detection rates of 10\%, 10\%, and 10\%,
respectively. The univariate and multivariate hazard ratios (HRs) for
subsequent detection of high-grade dysplasia or adenocarcinoma in
low-grade dysplasia patients with DNA content abnormality were 16.8
(p = \textless{}0.01) and 9.8 (p = \textless{}0.01), respectively. Among
the 13 high-grade dysplasia patients with DNA content abnormality, 5
patients (38\%) were subsequently found to have adenocarcinoma within a
mean follow-up time of 3 months, whereas only 1 (20\%) of the remaining
5 patients in the setting of normal DNA content developed adenocarcinoma
in a month (HR = 2.6, p = 0.39). The overall 1- and 2-year detection
rates of adenocarcinoma in all high-grade dysplasia patients (regardless
of flow cytometric results) were 34\% (95\% confidence
interval = 16-63\%) and 47\% (95\% confidence interval = 23-79\%),
respectively. In conclusion, the majority of low-grade dysplasia
patients (86\%) in the setting of abnormal DNA content developed
high-grade dysplasia or adenocarcinoma within 2 years and thus may
benefit from resection, whereas those with normal DNA content may be
followed with surveillance endoscopy. The presence of DNA content
abnormality can also confirm a morphologic suspicion of high-grade
dysplasia, which should be managed with resection, as nearly 50\% of the
high-grade dysplasia patients were found to have adenocarcinoma within 2
years.

{}

{}

\begin{center}\rule{0.5\linewidth}{\linethickness}\end{center}

Back to top

\begin{center}\rule{0.5\linewidth}{\linethickness}\end{center}

\pagebreak

\hypertarget{neuroendocrine-2}{%
\subsubsection{Neuroendocrine}\label{neuroendocrine-2}}

\begin{center}\rule{0.5\linewidth}{\linethickness}\end{center}

 - \textbf{The Pancreas as a Site of Metastasis or Second Primary in
Patients with Small Bowel Neuroendocrine Tumors}

\emph{Annals of surgical oncology 2019 Aug;26(8):2525-2532}

PubMed: \url{https://www.ncbi.nlm.nih.gov/pubmed/?term=31011904}

BACKGROUND: The small bowel and pancreas are the most common primary
sites of neuroendocrine tumors (NETs) giving rise to metastatic disease.
Some patients with small bowel NETs (SBNETs) present with synchronous or
metachronous pancreatic NETs (PNETs), and it is unclear whether these
are separate primaries or metastases from one site to the other.
METHODS: A surgical NET database including patients undergoing
operations for SBNETs or PNETs was reviewed. Patients with synchronous
or metachronous tumors in both the small bowel and pancreas were
identified, and available tissues from primary tumors and metastases
were examined using a 4-gene quantitative polymerase chain reaction
(qPCR) and immunohistochemistry (IHC) panel developed for evaluating
NETs of unknown primary. RESULTS: Of 338 patients undergoing
exploration, 11 had NETs in both the small bowel and pancreas. Tissues
from 11 small bowel tumors, 9 pancreatic tumors, and 10 metastases were
analyzed. qPCR and IHC data revealed that three patients had separate
SBNET and PNET primaries, and five patients had SBNETs that metastasized
to the pancreas. Pancreatic tissue was unavailable in two patients, and
qPCR and IHC gave discrepant results in one patient. CONCLUSIONS: NETs
in both the small bowel and pancreas were found in 3\% of our patients.
In nearly two-thirds of evaluable patients, the pancreatic tumor was a
metastasis from the SBNET primary, while in the remaining one-third of
patients it represented a separate primary. Determining the origin of
these tumors can help guide the choice of systemic therapy and surgical
management.

{}

{}

\begin{center}\rule{0.5\linewidth}{\linethickness}\end{center}

 - \textbf{PLAC8 is overexpressed and regulates cell proliferation in
low-grade human PanNET}

\emph{Neuroendocrinology 2019 Apr;():}

PubMed: \url{https://www.ncbi.nlm.nih.gov/pubmed/?term=31018208}

Background/Aims: Many aspects of the biology of pancreatic
neuroendocrine tumors (PanNETs), including determinants of
proliferative, invasive and metastatic potential, remain poorly
understood. Placenta-specific 8 (PLAC8), a gene with unknown molecular
function, has been reported to have tumor-promoting roles in different
human malignancies, including exocrine pancreatic cancer. Since
preliminary data suggested deregulation of PLAC8 expression in PanNET,
we have performed detailed analyses of PLAC8 expression and function in
human PanNET. Primary tissue from PanNET patients was
immunohistochemically stained for PLAC8 and expression was correlated
with clinicopathological data. In vitro, PLAC8 expression was inhibited
by siRNA transfection in PanNET cell lines and effects were analyzed by
qRT-PCR, Western blot and proliferation assays. We report that PLAC8 is
expressed in the majority of well-differentiated human PanNETs,
predominantly in early stage and low grade tumors. SiRNA-mediated
knockdown of PLAC8 in PanNET cells resulted in decreased proliferation
and viability, while apoptosis was not induced. Mechanistically, these
effects were mediated by attenuation of cell cycle progression, as
Western blot analyses demonstrated upregulation of the tumor suppressor
p21/CDKN2A and downregulation of the cell cycle regulator Cyclin D1 as
well as reduced levels of phosphorylated ribosomal protein s6 (pRPS6)
and retinoblastoma protein (pRb). Our findings establish PLAC8 as a
central mediator of cell growth in a subset of human PanNET, providing
evidence for the existence of distinct molecular subtypes within this
class of tumors..

{}

{}

\begin{center}\rule{0.5\linewidth}{\linethickness}\end{center}

Back to top

\begin{center}\rule{0.5\linewidth}{\linethickness}\end{center}

\pagebreak


\end{document}
