\documentclass[]{article}
\usepackage{lmodern}
\usepackage{amssymb,amsmath}
\usepackage{ifxetex,ifluatex}
\usepackage{fixltx2e} % provides \textsubscript
\ifnum 0\ifxetex 1\fi\ifluatex 1\fi=0 % if pdftex
  \usepackage[T1]{fontenc}
  \usepackage[utf8]{inputenc}
\else % if luatex or xelatex
  \ifxetex
    \usepackage{mathspec}
  \else
    \usepackage{fontspec}
  \fi
  \defaultfontfeatures{Ligatures=TeX,Scale=MatchLowercase}
\fi
% use upquote if available, for straight quotes in verbatim environments
\IfFileExists{upquote.sty}{\usepackage{upquote}}{}
% use microtype if available
\IfFileExists{microtype.sty}{%
\usepackage{microtype}
\UseMicrotypeSet[protrusion]{basicmath} % disable protrusion for tt fonts
}{}
\usepackage[margin=1in]{geometry}
\usepackage{hyperref}
\hypersetup{unicode=true,
            pdftitle={PBPath Journal Watch},
            pdfborder={0 0 0},
            breaklinks=true}
\urlstyle{same}  % don't use monospace font for urls
\usepackage{graphicx,grffile}
\makeatletter
\def\maxwidth{\ifdim\Gin@nat@width>\linewidth\linewidth\else\Gin@nat@width\fi}
\def\maxheight{\ifdim\Gin@nat@height>\textheight\textheight\else\Gin@nat@height\fi}
\makeatother
% Scale images if necessary, so that they will not overflow the page
% margins by default, and it is still possible to overwrite the defaults
% using explicit options in \includegraphics[width, height, ...]{}
\setkeys{Gin}{width=\maxwidth,height=\maxheight,keepaspectratio}
\IfFileExists{parskip.sty}{%
\usepackage{parskip}
}{% else
\setlength{\parindent}{0pt}
\setlength{\parskip}{6pt plus 2pt minus 1pt}
}
\setlength{\emergencystretch}{3em}  % prevent overfull lines
\providecommand{\tightlist}{%
  \setlength{\itemsep}{0pt}\setlength{\parskip}{0pt}}
\setcounter{secnumdepth}{0}
% Redefines (sub)paragraphs to behave more like sections
\ifx\paragraph\undefined\else
\let\oldparagraph\paragraph
\renewcommand{\paragraph}[1]{\oldparagraph{#1}\mbox{}}
\fi
\ifx\subparagraph\undefined\else
\let\oldsubparagraph\subparagraph
\renewcommand{\subparagraph}[1]{\oldsubparagraph{#1}\mbox{}}
\fi

%%% Use protect on footnotes to avoid problems with footnotes in titles
\let\rmarkdownfootnote\footnote%
\def\footnote{\protect\rmarkdownfootnote}

%%% Change title format to be more compact
\usepackage{titling}

% Create subtitle command for use in maketitle
\newcommand{\subtitle}[1]{
  \posttitle{
    \begin{center}\large#1\end{center}
    }
}

\setlength{\droptitle}{-2em}

  \title{PBPath Journal Watch}
    \pretitle{\vspace{\droptitle}\centering\huge}
  \posttitle{\par}
  \subtitle{June-July 2018}
  \author{}
    \preauthor{}\postauthor{}
      \predate{\centering\large\emph}
  \postdate{\par}
    \date{Last Update on 2018-08-07}


\begin{document}
\maketitle

{
\setcounter{tocdepth}{2}
\tableofcontents
}
\begin{center}\rule{0.5\linewidth}{\linethickness}\end{center}

\textbf{Wellcome to our journal watch for pancreatobiliary pathology
articles, which is released every other month.}

\emph{We have created several categories for convenience; however,
articles in each category are in no particular order.}

\emph{Please feel free to fill out our feedback form. You may also
recommend articles to be included.}

\pagebreak

\hypertarget{the-current-pbpath-journal-watch-articles}{%
\section{The Current PBPath Journal Watch
Articles}\label{the-current-pbpath-journal-watch-articles}}

\begin{center}\rule{0.5\linewidth}{\linethickness}\end{center}

\hypertarget{surgical-pathology}{%
\subsection{Surgical Pathology}\label{surgical-pathology}}

 - \textbf{PD-1, PD-L1 and CD163 in pancreatic undifferentiated
carcinoma with osteoclast-like giant cells: expression patterns and
clinical implications}

\emph{Human pathology 2018 Jul;():}

Undifferentiated carcinoma with osteoclast-like giant cells (UCOGC), a
variant of pancreatic ductal adenocarcinoma (PDAC), has striking genetic
similarity to PDAC but a significantly improved overall survival. We
hypothesize that this difference could be due to the immune response to
the tumor, and as such, we investigated the expression of PD-1, PD-L1
and CD163 in a series of UCOGC. To this aim, 27 pancreatic UCOGCs (11
pure and 16 PDAC-associated), 5 extra-pancreatic tumors with
osteoclast-like giant cells and 10 pancreatic anaplastic carcinomas
(ACs) were immunostained using antibodies against PD-1, PD-L1 and CD163.
In pancreatic UCOGCs, PD-L1 was expressed in neoplastic cells of 17/27
(63\%) cases, more often in cases with an associated PDAC (P=.04).
Expression of PD-L1 was associated with poor prognosis, confirmed by
multivariate analysis: patients with PD-L1-positive UCOGCs had a risk of
all-cause mortality that was 3 times higher than patients with
PD-L1-negative UCOGCs (HR: 3.397, 95\%CI: 1.023-18.375, P=.034). PD-L1
expression on tumor cells was also associated with aberrant P53
expression (P=.035). PD-1 was expressed on rare lymphocytes in 12 UCOGCs
(44.4\%), mainly located at the tumor periphery. CD163 was expressed on
histiocytes, with a diffuse and strong staining pattern in all UCOGCs.
Extra-pancreatic tumors with osteoclast-like giant cells showed very
similar staining patterns for the same proteins. ACs have some
similarities to UCOGCs, but PD-L1 has no prognostic roles. Our results
may have important implications for immunotherapeutic strategies in
UCOGCs; these tumors may also represent a model for future therapeutic
approaches against PDAC.

PubMed: \url{https://www.ncbi.nlm.nih.gov/pubmed/?term=30031096}

{}

{}

\begin{center}\rule{0.5\linewidth}{\linethickness}\end{center}

 - \textbf{TFE3 is a diagnostic marker for solid pseudopapillary
neoplasms of the pancreas}

\emph{Human pathology 2018 Jul;():}

Aberrant Wnt signaling is a hallmark of solid pseudopapillary neoplasms
of the pancreas (SPN). Transcription factor E3 (TFE3) plays a critical
role in activation and regulation of the Wnt pathway, and is predicted
to be a candidate gene implicated in SPN by gene regulatory network
analysis. The aim of this study was to evaluate TFE3 as a marker for
SPN. Paraffin embedded tissues of SPN (n = 75) and other primary
pancreatic tumors were analyzed, including pancreatic neuroendocrine
tumors (PanNET) (n = 17), pancreatic ductal adenocarcinomas (PDAC) (n =
14), pancreatic neuroendocrine carcinomas (PanNEC) (n = 4) and acinar
cell carcinomas (ACC) (n = 3). The clinicopathological features were
summarized as well. Differentiation of specific pancreatic duct or
acinus was not found in any SPN tissue. Morphological and
immunohistochemical results indicated that SPN displays certain
characteristics of neuroendocrine cells. Overall, 71 (94.67\%) cases of
SPN showed nuclear accumulation for TFE3, most of which displayed
moderate to intense expression. The TFE3 positive rate in PanNET, PDAC
and PanNEC was 23.53\%, 14.29\%, and 25\%, respectively. All three cases
of ACC were negative for TFE3. We conclude that SPN may originate from
primordial pancreatic cells and is accompanied by some characteristics
of neuroendocrine tumors. TFE3, besides β-catenin, can be an additional
diagnostic marker of SPN in differential diagnosis.

PubMed: \url{https://www.ncbi.nlm.nih.gov/pubmed/?term=30030118}

{}

{}

\begin{center}\rule{0.5\linewidth}{\linethickness}\end{center}

 - \textbf{S100A10, a Novel Biomarker in Pancreatic Ductal
Adenocarcinoma}

\emph{Molecular oncology 2018 Jul;():}

Pancreatic cancer is arguably the deadliest cancer type. The efficacy of
current therapies is often hindered by the inability to predict patient
outcome. As such, the development of tools for early detection and risk
prediction is key for improving outcome and quality of life. Here, we
introduce the plasminogen receptor S100A10 as a novel predictive
biomarker and a driver of pancreatic tumor growth and invasion. We
demonstrated that S100A10 mRNA and protein are overexpressed in human
pancreatic tumors compared to normal ducts and non-ductal stroma.
S100A10 mRNA and methylation status were predictive of overall survival
and recurrence-free survival across multiple patient cohorts. S100A10
expression was driven by promoter methylation and the oncogene KRAS.
S100A10 knockdown reduced surface plasminogen activation, invasiveness
and in vivo growth of pancreatic cancer cell lines. These findings
delineate the clinical and functional contribution of S100A10 as a
biomarker in pancreatic cancer.

PubMed: \url{https://www.ncbi.nlm.nih.gov/pubmed/?term=30009399}

{}

{}

\begin{center}\rule{0.5\linewidth}{\linethickness}\end{center}

 - \textbf{Primary Hepatoid Carcinoma of the Pancreas: A
Clinicopathological Study of 3 Cases With Review of Additional 31 Cases
in the Literature}

\emph{International journal of surgical pathology 2018
Jun;():1066896918783468}

Primary pancreatic hepatoid carcinoma (PHC) is very rare. Here, we
reported 3 such cases with review of additional 31 cases in the
literature. Our 3 patients were male (83, 72, and 54 years old,
respectively). Serum α-fetoprotein (AFP) was elevated in 1 patient (case
3, 8338 ng/mL) and not measured in the other two. The PHC in patient 1
(pathological stage pT2N0M0) and patient 2 (pT3N0M0) showed pure
hepatocellular carcinoma (HCC)-like morphology, whereas in case 3 it was
a PHC with true glandular differentiation (pT4N0M0). The diagnosis of
PHC was confirmed with positive immunohistochemical staining in the
tumor cells for AFP (2/3), Hep Par 1 (3/3), glypican-3 (2/3), arginase-1
(2/3), and Sal-like protein 4 (1/3). CD10 and polyclonal
carcinoembryonic antigen stains show focal canalicular pattern in 2/3
tumors. Patient 1 did not receive further treatment after resection and
was alive with no evidence of disease at 107 months. Patient 2 died of
postoperative complications, whereas patient 3 received postsurgical
chemoradiation and died of disease at 29 months. Our findings and
literature review indicate that PHCs can be divided into 4 histological
subtypes: with pure HCC-like morphology (n = 22), with neuroendocrine
differentiation (n = 8), with true glandular differentiation (n = 3),
and with acinar cell differentiation (n = 1). On univariate analysis,
pure HCC-like morphology was associated with better disease-specific
survival (DSS; P = .04), whereas lymph node and distant metastases were
associated with worse DSS ( P = .002 for both). Age, gender, presenting
symptoms, serum AFP level, and T stage were not associated with DSS. On
multivariate analysis, none of these parameters was significantly
associated with DSS.

PubMed: \url{https://www.ncbi.nlm.nih.gov/pubmed/?term=29961402}

{}

{}

\begin{center}\rule{0.5\linewidth}{\linethickness}\end{center}

 - \textbf{Comparison of 3 Ways of Dissecting the Pancreatoduodenectomy
Specimen and Their Impact in the Lymph Node Count and the Lymph Node
Metastatic Ratio}

\emph{International journal of surgical pathology 2018
Jun;():1066896918780343}

BACKGROUND: Lymph node metastasis (LNM) is a strong prognostic factor in
the cancer of the pancreatobiliary tree, but it is influenced by the
number of lymph nodes (LNs). The lymph node ratio (LNR) is considered a
more reliable factor than the number of LNM. The aim was to examine the
LN retrieval and the LNR of 3 pathologic work-up strategies. METHODS:
Pancreaticoduodenectomies (n = 165) were analyzed comparing 3
pathological dissection techniques, classified as ``control,'' ``Verbeke
method,'' and ``Adsay method'' groups. RESULTS: The mean of the
dissected LNs and the number of cases with \textgreater{}20 LNs were
superior in the Adsay method group, compared with the other groups ( P
\textless{} .001). The LNR was different between the Adsay and Verbeke
groups (0.144 vs 0.069, P = .032). The median of the 3 positive LNs was
associated with decreased survival compared with an absence of LNM
(3-year specific survival of 48\% vs 22\%, P = .011). In the
multivariate analysis, LNM (hazard ratio = 6.148, 95\% confidence
interval = 2.02-8.1, P = .042) and the evaluation of \textgreater{}15
LNs (hazard ratio = 12.52, 95\% confidence interval = 5.51-21.01, P =
.001) were independent predictors of survival. CONCLUSION: The Adsay
technique for LN retrieval was associated with a better LN count, more
cases with LNM, and an LNR \textgreater{}0.1.

PubMed: \url{https://www.ncbi.nlm.nih.gov/pubmed/?term=29873282}

{}

{}

\begin{center}\rule{0.5\linewidth}{\linethickness}\end{center}

 - \textbf{Pancreatic Lipomatous Hamartoma: A Hitherto Unrecognized
Variant}

\emph{The American journal of surgical pathology 2018 Jul;42(7):891-897}

Pancreatic masses consisting of lipomatous components clinically include
lipoma, liposarcoma, lipomatous pseudohypertrophy of the pancreas,
fat-containing neoplasms such as perivascular epithelioid cell tumor,
and malignant neoplasm with lipoid degeneration. We present pancreatic
lipomatous hamartoma, which has not been reported hitherto. A solid
pancreatic mass was detected from a computed tomographic scan check-up
in each of 3 cases of Japanese men. Macroscopically, well-demarcated
solid lipomatous masses were detected at the uncus, body, and tail of
the pancreas, respectively. Microscopically, the masses predominantly
consisted of mature adipocytes with no atypia, but contained
characteristics components of pancreatic hamartoma, such as small ducts,
a well-preserved acinar structure, and/or fibrous stroma. On the basis
of the unique features, lack of islets and absence of periductal elastic
fibers, these tumors are a distinct variant of pancreatic hamartoma.
Furthermore, high-mobility group AT-hook 2 expression in the
fibro-adipocytes of this tumor indicated that these cells are an
integral component of the pancreatic lipomatous hamartoma. Consequently,
the unique tumors described herein are pancreatic lipomatous hamartoma,
which must be discriminated from other lipomatous lesions of the
pancreas.

PubMed: \url{https://www.ncbi.nlm.nih.gov/pubmed/?term=29738363}

{}

{}

\begin{center}\rule{0.5\linewidth}{\linethickness}\end{center}

 - \textbf{Hereditary Pancreatitis in the United States: Survival and
Rates of Pancreatic Cancer}

\emph{The American journal of gastroenterology 2018 Jul;():}

OBJECTIVES: Hereditary pancreatitis (HP), an autosomal dominant disease
typically caused by mutations in PRSS1, has a broad range of clinical
characteristics and high cumulative risk of pancreatic cancer. We
describe survival and pancreatic cancer risk in the largest HP cohort in
the US. METHODS: HP probands and family members prospectively recruited
from 1995 to 2013 completed medical and family history questionnaires,
and provided blood for DNA testing. Overall survival (until 12/31/2015)
was determined from the Social Security Death Index (SSDI), National
Death Index (NDI), and family members. Cause of death was obtained from
the NDI. RESULTS: 217 PRSS1 carriers (181 symptomatic) formed the study
cohort. The most frequently detected mutations were p.R122H (83.9\%) and
p.N29I (11.5\%). Thirty-seven PRSS1 carriers (30 symptomatic, 7
asymptomatic) were deceased at conclusion of the study (5 from
pancreatic cancer). Median overall survival was 79.3 years (IQR
72.2-85.2). Risk of pancreatic cancer was significantly greater than
age- and sex- matched SEER data (SIR 59, 95\% CI 19-138), and cumulative
risk was 7.2\% (95\% CI 0-15.4) at 70 years. DISCUSSION: We confirm
prior observations on survival and pancreatic cancer SIR in PRSS1
subjects. Although risk of pancreatic cancer was significantly high in
these patients, its cumulative risk was much lower than previous
reports.

PubMed: \url{https://www.ncbi.nlm.nih.gov/pubmed/?term=30018304}

{}

{}

\begin{center}\rule{0.5\linewidth}{\linethickness}\end{center}

 - \textbf{Distribution of dysplasia and cancer in the gallbladder: an
analysis from a high cancer-risk population}

\url{https://www.sciencedirect.com/science/article/pii/S004681771830282X}

\emph{Human pathology 2018 Jul;():}

Gallbladder dysplasia can progress to cancer and may be associated with
increased cancer risk at other biliary tract sites. Thus, its accurate
identification is relevant both for etiologic understanding and for
clinical purposes. Data on the frequency and distribution of gallbladder
dysplasia are lacking due to limited gallbladder sampling and inability
to visualize dysplasia grossly. An expert pathology group used consensus
criteria to review 140 totally sampled consecutive cholecystectomy
specimens from Chilean women. Three cases (2\%) revealed incidental
invasive carcinoma, all T2, along with high-grade dysplasia (HGD). The
surface area covered by dysplasia or cancer in these cases was 9\%,
37\%, and 87\%. Although the first longitudinal (``diagnostic'') section
of the whole gallbladder captured HGD or cancer in all three cases, the
deepest focus of invasive carcinoma was not present in this section.
Fourteen additional cases (10\%) had low-grade dysplasia (LGD), which
was typically very focal (covering \textless{}5\% of the surface) and
most often occurred in the fundus. LGD was not present in the diagnostic
section of five cases (38\%) and would have been missed without
additional sampling. None of the cancers or dysplasias were grossly
visible. Although HGD and carcinoma are likely to be identified in
``diagnostic'' sections, accurate staging requires total sampling. LGD
is typically very focal and would often be missed in routine practice.
To identify cancer precursors, additional sampling, particularly of the
fundus, may be warranted. The predominance of LGD in the fundus also
provides etiologic insight, supporting the contribution of gallstones
and chronic inflammation.

PubMed: \url{https://www.ncbi.nlm.nih.gov/pubmed/?term=30036595}

{}

{}

\begin{center}\rule{0.5\linewidth}{\linethickness}\end{center}

 - \textbf{Epithelial Inclusions in Gallbladder Specimens Mimic Parasite
Infection: Histologic and Molecular Examination of Reported Cystoisopora
belli Infection in Gallbladders of Immunocompetent Patients}

\emph{The American journal of surgical pathology 2018 Jul;():}

Recent publications have described epithelial cytoplasmic vacuoles and
inclusions incidentally noted within gallbladder epithelium and
concluded that they represent coccidian parasite infection, in
particular, Cystoisospora belli. We identified 8 gallbladder specimens
from our institution in the past 3 years in which this diagnosis was
suggested or in which similar epithelial alterations were prominent.
Molecular analysis was performed on the 8 gallbladder specimens and on 3
positive control specimens: small bowel biopsies from acquired
immunodeficiency syndrome patients with diarrhea. Polymerase chain
reaction using primers designed to amplify an internal transcribed
spacer (ITS2) in the C. belli ribosomal gene cluster was performed on
the DNA samples. All 8 gallbladder specimens were negative for
amplification, while a product consistent with C. belli was amplified
from all 3 positive controls. Histologically, the gallbladder
cytoplasmic inclusions stained diffusely positive for Grocott-Gomori's
methenamine silver and Periodic acid-Schiff with diastase. In contrast,
sections from a positive control small bowel biopsy demonstrated
organisms that were negative for Grocott-Gomori's methenamine silver and
showed a distinct capsular and punctate internal staining on Periodic
acid-Schiff with diastase in various parasite forms. Together, the lack
of molecular evidence of C. belli and the distinct morphologic and
special staining patterns in these gallbladders compared with positive
control small bowel suggest that these epithelial changes do not
represent true C. belli infection. Our results suggest that gallbladders
of immunocompetent patients may occasionally show epithelial changes
that can morphologically mimic C. belli infection. Pathologists should
be aware of this histologic variant to minimize unnecessary treatment,
testing, and patient anxiety.

\url{https://journals.lww.com/ajsp/Pages/articleviewer.aspx?year=9000\&issue=00000\&article=97799\&type=Abstract}

PubMed: \url{https://www.ncbi.nlm.nih.gov/pubmed/?term=30020094}

{}

{}

\begin{center}\rule{0.5\linewidth}{\linethickness}\end{center}

 - \textbf{Pyloric Gland Adenoma (PGA) of the Gallbladder: A Unique and
Distinct Tumor from PGAs of the Stomach, Duodenum, and Pancreas}

\emph{The American journal of surgical pathology 2018 Jul;():}

Twenty-four surgically resected, gallbladder pyloric gland adenomas
(GB-PGAs) were examined and their features were compared with the
reported features of stomach, duodenum, and pancreatic PGAs to better
understand GB-PGAs. Clinical information on background gallbladder
lesions and histologic data, including tumor grade, existence of
squamoid morules, intratumoral cholesterosis, and intracytoplasmic
mucins were collected. Immunohistochemical staining for MUC2, MUC5AC,
MUC6, CDX2, pepsinogen I, p53, and MIB-1/nuclear β-catenin were
evaluated. Targeted mutational analyses of KRAS exon2, GNAS exon 7, and
CTNNB1 exon 3 were conducted. We found that 29.2\% of the GB-PGAs were
histologically high-grade dysplasias/carcinomas; 70.8\% were low grade;
and 20.8\% and 33.3\% contained squamoid morules and intratumoral
cholesterosis, respectively. In addition, 45.8\% and 54.2\% of GB-PGAs
were mucin-rich and mucin-poor types, respectively.
Immunohistochemically, MUC6 was diffusely positive in all GB-PGAs; MUC2,
MUC5AC, and CDX2 were only focally positive, and no pepsinogen-I
positive cells were observed. Nuclear β-catenin accumulation was
observed in all cases; however, the ratio varied among cases. Mucin-poor
types were significantly associated with high histologic grade
dysplasias/carcinomas and high nuclear β-catenin labeling indices.
Mutational analyses identified CTNNB1 mutations in 100\% of GB-PGAs
(21/21), KRAS in 4.2\% (1/23), and GNAS in 0\% (0/22). The present study
clarified the unique histologic features, phenotypic differentiation,
and molecular statuses frequently associated with GB-PGAs. Altogether,
our data suggest that tumorigenesis of GB-PGA is distinct from that of
stomach, duodenum, and pancreatic PGAs.

PubMed: \url{https://www.ncbi.nlm.nih.gov/pubmed/?term=29975247}

{}

{}

\begin{center}\rule{0.5\linewidth}{\linethickness}\end{center}

 - \textbf{Clinical relevance of PD-L1 expression in gallbladder cancer:
a potential target for therapy}

\emph{Histopathology 2018 Jun;():}

AIMS: Programmed death-ligand 1 (PD-L1), a potential target for immune
checkpoint inhibitors in various solid neoplasms, has been studied in
very few cases of Gall Bladder Carcinoma (GBC). The current study aimed
to evaluate PD-L1 expression at primary and metastatic sites of GBC, and
its associations with standard prognostic clinicopathological
parameters, as well as with overall survival. METHODS AND RESULTS: One
hundred and seventy-four cases of GBC were evaluated for PD-L1
expression by the use of the SP263 clone in tissue microarrays.
Clinicopathological characteristics and survival data were correlated
with PD-L1 expression analysed at different cut-offs of ≥1\%, ≥10\% and
≥50\% in tumour cells and tumour-infiltrating lymphocytes (TILs). The
mean age of patients was 49.9 years, and the male/female ratio was
1:2.9. Of the cases, 73.6\% presented with stage 3/4 disease. Tumour
cells expressed PD-L1 in 23.0\% of cases, and TILs expressed PD-L1 in
24.1\% of cases. At a cut-off of 10\%, 14.9\% of cases expressed PD-L1,
and at a cut-off of 50\%, 7.5\% of cases expressed PD-L1. Significant
associations were seen between tumour proportion score and histological
type (P = 0.004), histological grade (P = 0.004), nuclear grade (P =
0.008), nodal metastasis (P = 0.051), higher stage (P = 0.058), and TILs
(P \textless{} 0.001). Tumour size, growth pattern, the presence of
necrosis and lymphovascular emboli showed no significant associations
with PD-L1 in tumour cells or TILs. In synchronous paired samples from
primary and metastatic lymph nodes, discordantly higher PD-L1 expression
was evident in lymph nodes. Overall survival was not associated with
PD-L1 expression (P = 0.546). CONCLUSION: PD-L1 does not appear to be a
prognostic marker or influence survival in GBC patients. However, PD-L1
expression occurs in one of four GBCs, supporting the future possibility
of immune-modulation therapy to improve the dismal overall survival.

PubMed: \url{https://www.ncbi.nlm.nih.gov/pubmed/?term=29882997}

{}

{}

\begin{center}\rule{0.5\linewidth}{\linethickness}\end{center}

 - \textbf{DJ-1 is a useful biomarker for invasive extrahepatic
cholangiocarcinoma}

\emph{Human pathology 2018 Jun;76():28-36}

We have previously reported that DJ-1 protein is up-regulated in
cholangiocarcinoma compared with non-neoplastic epithelium of the bile
duct in a study using liquid-chromatography mass spectrometry-based
proteomics. The aim of this study was to clarify whether DJ-1 expression
offers a biomarker for patients with invasive extrahepatic
cholangiocarcinoma (EHCC) who undergo surgical resection with curative
intent. Positive immunohistochemical (IHC) staining of DJ-1 was
significantly more frequent in the cytoplasm of 96 invasive EHCCs
(n\,=\,28, 29.2\%) than in that of 66 non-neoplastic epithelial lesions
adjacent to invasive EHCC (n\,=\,7, 10.6\%; P\,=\,.006). No significant
difference in clinicopathological features was evident between invasive
EHCC patients with negative (n\,=\,68) and positive (n\,=\,28) IHC
staining. However, negative IHC staining for DJ-1 in cytoplasm was
selected as an independent risk factor for adverse prognosis on
multivariate analysis (P\,=\,.004, hazard ratio 2.13, 95\% confidence
interval 1.28-3.57). Serum levels of DJ-1 in 16 invasive EHCC patients
with metastasis were compared with 12 invasive EHCC patients without
metastasis. Serum levels of DJ-1 tended to be higher in 16 patients with
metastasis (median, 40.9\,ng/ml) than in 12 patients without
(27.6\,ng/ml, P\,=\,.137). In addition, patients with high serum levels
(≥ 40\,ng/ml) of DJ-1 tended to have metastasis more frequently than
those without (P\,=\,.054, Fisher's exact test). We concluded that IHC
staining pattern and serum level of DJ-1 in patients with invasive EHCC
might be predictive of prognosis and metastasis, respectively.

PubMed: \url{https://www.ncbi.nlm.nih.gov/pubmed/?term=29447925}

{}

{}

\begin{center}\rule{0.5\linewidth}{\linethickness}\end{center}

 - \textbf{Intrahepatic Cholangiocarcinomas Have Histologically and
Immunophenotypically Distinct Small and Large Duct Patterns}

\emph{The American journal of surgical pathology 2018 Jul;():}

Intrahepatic cholangiocarcinomas are histologically heterogenous. Using
a cohort of 184 clinically defined, resected intrahepatic
cholangiocarcinomas, we retrospectively classified the histology into 4
subtypes: large duct (LD), small duct (SD) (predominantly tubular
{[}SD1{]} or predominantly anastomosing/cholangiolar, {[}SD2{]}), or
indeterminate. Then, we tested the 4 subtypes for associations with risk
factors, patient outcomes, histology, and immunophenotypic
characteristics. SD was the most common (84\%; 24\% SD1 and 60\% SD2)
with lower proportions of LD (8\%), and indeterminate (8\%). Primary
sclerosing cholangitis was rare (2\%), but correlated with LD (P=0.005).
Chronic hepatitis, frequent alcohol use, smoking, and steatosis had no
histologic association. LD was associated with mucin production
(P\textless{}0.001), perineural invasion (P=0.002), CA19-9 staining
(P\textless{}0.001), CK7, CK19, CD56 immunophenotype (P=0.005), and
negative albumin RNA in situ hybridization (P\textless{}0.001). SD was
histologically nodular (P=0.019), sclerotic (P\textless{}0.001),
hepatoid (P=0.042), and infiltrative at the interface with hepatocytes
(P\textless{}0.001). Albumin was positive in 71\% of SD and 18\% of LD
(P=0.0021). Most albumin positive tumors (85\%) lacked extracellular
mucin (P\textless{}0.001). S100P expression did not associate with
subtype (P\textgreater{}0.05). There was no difference in
disease-specific or recurrence-free survival among the subtypes.
Periductal infiltration and American Joint Committee on Cancer eighth
edition pT stage predicted survival by multivariable analysis accounting
for gross configuration, pT stage, and histologic type. pT2 had worse
outcome relative to other pT stages. Significant differences in
histology and albumin expression distinguish LD from SD, but there is
insufficient evidence to support further subclassification of SD.

PubMed: \url{https://www.ncbi.nlm.nih.gov/pubmed/?term=30001234}

{}

{}

\begin{center}\rule{0.5\linewidth}{\linethickness}\end{center}

 - \textbf{Ampullary adenocarcinoma: Defining predictors of survival and
the impact of adjuvant therapy following surgical resection for stage I
disease}

\emph{Journal of surgical oncology 2018 Jun;117(7):1500-1508}

BACKGROUND AND OBJECTIVES: Outcomes and recommendations regarding
adjuvant therapy (AT) for stage I ampullary adenocarcinoma (AAC) are
inadequately described. We sought to determine factors associated with
survival and better define the impact of AT. METHODS: The NCDB was
queried for stage I AAC patients undergoing resection. We evaluated
variables influencing the administration of AT and affecting survival,
including the receipt of AT. RESULTS: Five hundred thirty-seven patients
were identified. 1, 3, and 5-year OS were 91.3\%, 78.8\%, and 67.4\%,
respectively. 103 received AT: 101 chemotherapy, 31 radiation, and 29 a
combination of both. AT was more commonly utilized in patients with
poorly differentiated and T2 tumors. Comorbid disease was inversely
associated with use of AT. Age ≥65 was associated with decreased
survival for stage IA and IB, while positive resection margins and
sampling of \textless{}12 LNs were associated with decreased OS for
stage IA and IB, respectively. After propensity matching key covariates,
no significant difference in OS was observed between those receiving and
not receiving AT (P = 0.449). CONCLUSION: This analysis revealed a
modest 5-year OS for stage I AAC. Age, positive resection margins, and
evaluation of \textless{}12 LNs negatively influenced OS and AT did not
convey a survival benefit.

PubMed: \url{https://www.ncbi.nlm.nih.gov/pubmed/?term=29518820}

{}

{}

\begin{center}\rule{0.5\linewidth}{\linethickness}\end{center}

 - \textbf{Prognostic factors of non-ampullary duodenal adenocarcinoma}

\emph{Japanese journal of clinical oncology 2018 Aug;48(8):743-747}

Background: Non-ampullary duodenal adenocarcinoma, excluding carcinoma
in the ampulla of Vater, is a rare disease. Although several prognostic
factors have been reported, they remain controversial due to the rarity
of non-ampullary duodenal adenocarcinoma. The aims of this study were to
investigate prognostic factors in patients with non-ampullary duodenal
adenocarcinoma and to assess chemotherapy in patients with recurrence.
Patients and methods: Records of 25 patients who underwent surgical
treatment for non-ampullary duodenal adenocarcinoma from 2004 to 2016
were retrospectively reviewed. The relationship between the
clinicopathological factors and outcomes was investigated. Results:
Serum level of CA19-9, gross appearance, tumor size, tumor invasion,
lymph node metastases, TNM stage and lymphatic and vascular invasion
were significant risk factors of recurrence. Patients with recurrence
who received chemotherapy according to regimens used to treat colorectal
cancer had a better prognosis than those without chemotherapy (P =
0.016). Conclusion: Advanced non-ampullary duodenal adenocarcinoma has a
poor prognosis, but chemotherapy possibly improves the prognosis in the
patients with recurrent non-ampullary duodenal adenocarcinoma.

PubMed: \url{https://www.ncbi.nlm.nih.gov/pubmed/?term=29931295}

{}

{}

\begin{center}\rule{0.5\linewidth}{\linethickness}\end{center}

 - \textbf{The Problem of High-Grade Gastroenteropancreatic
Neuroendocrine Neoplasms: Well-Differentiated Neuroendocrine Tumors,
Neuroendocrine Carcinomas, and Beyond}

\url{https://www.sciencedirect.com/science/article/pii/S0889852918305279?via\%3Dihub}

\url{https://www.endo.theclinics.com/article/S0889-8529(18)30527-9/fulltext}

\begin{center}\rule{0.5\linewidth}{\linethickness}\end{center}

 - \textbf{Ampullary neuroendocrine neoplasms: surgical experience of a
rare and challenging entity}

\emph{Langenbeck's archives of surgery 2018 Jul;():}

PURPOSE: Ampullary neuroendocrine neoplasms (NENs) account for
\textless{} 0.3\% of gastrointestinal NENs. Surgical options include
transduodenal ampullectomy/tumour excision or pancreaticoduodenectomy
(PD). We report the experience of two high-volume pancreatic surgical
centres of ampullary NENs. METHODS: Clinical records of patients who
underwent surgery for ampullary NENs (January 2007-November 2017) in the
study centres were retrieved retrospectively. We evaluated
clinical-pathological features, post-operative outcome and follow-up
(FU). RESULTS: Eighteen patients (9 M/9 F, averaging 62~years) were
enrolled. All but one were non-functioning NENs; four (22\%) patients
presented with jaundice. Seven (39\%) of the patients underwent
ampullectomy/excision (median tumour size 1.5~cm), and 11 (61\%)
patients underwent PD (median tumour size 2.4~cm). The median operation
time of ampullectomy/excision was 221~min with operative blood loss of
75~ml (vs.~506~min and 425~ml in PD). The median hospital stay was
10~days in both groups. Overall surgical morbidity was 33\%, due to four
biochemical leaks, one pancreatic fistula and one abdominal haemorrhage.
No reoperations were needed. The median tumour size was 1.8 (range
0.5-6.7)~cm. All G2-G3 NENs were N1 (vs.~1/7 in G1 NENs). Three (17\%)
cases were mixed exocrine/G3 NECs. After a median FU of 45 (up to
124)~months, recurrence occurred in four G3 NEC (31\%) patients (median
disease-free survival 14~months) after an R0 PD. Disease-related
survival was 93, 77 and 66\% at 1, 3 and 5~years, respectively.
CONCLUSION: Ampullary NENs are mostly G1-G2 neoplasms. Lymph node
metastases rarely occur in G1 NENs \textless{} 2~cm in size, which may
be treated with ampullectomy/excision. Survival is 66\% 5~years after
surgery.

PubMed: \url{https://www.ncbi.nlm.nih.gov/pubmed/?term=30043166}

{}

{}

\begin{center}\rule{0.5\linewidth}{\linethickness}\end{center}

 - \textbf{Neoplasms of the Neuroendocrine Pancreas: An Update in the
Classification, Definition, and Molecular Genetic Advances}

\emph{Advances in anatomic pathology 2018 Jun;():}

This review focuses on discussing the main modifications of the recently
published 2017 WHO Classification of Neoplasms of the Neuroendocrine
Pancreas (panNEN). Recent updates separate pancreatic neuroendocrine
tumors into 2 broad categories: well-differentiated pancreatic
neuroendocrine tumors (panNET) and poorly differentiated pancreatic
neuroendocrine carcinoma (panNEC), and incorporates a new subcategory of
``well-differentiated high-grade NET (G3)'' to the well-differentiated
NET category. This new classification algorithm aims to improve the
prediction of clinical outcomes and survival and help clinicians select
better therapeutic strategies for patient care and management. In
addition, these neuroendocrine neoplasms are capable of producing large
quantity of hormones leading to clinical hormone hypersecretion
syndromes. These functioning tumors include, insulinomas, glucagonomas,
somatostatinomas, gastrinomas, VIPomas, serotonin-producing tumors, and
ACTH-producing tumors. Although most panNENs arise as sporadic diseases,
a subset of these heterogeneous tumors present as parts on inherited
genetic syndromes, such as multiple endocrine neoplasia type 1, von
Hippel-Lindau, neurofibromatosis type 1, tuberous sclerosis, and
glucagon cell hyperplasia and neoplasia syndromes. Characteristic
clinical and morphologic findings for certain functioning and syndromic
panNENs should alert both pathologists and clinicians as appropriate
patient management and possible genetic counseling may be necessary.

PubMed: \url{https://www.ncbi.nlm.nih.gov/pubmed/?term=29912000}

{}

{}

\begin{center}\rule{0.5\linewidth}{\linethickness}\end{center}

Back to top

\pagebreak

\hypertarget{cytopathology}{%
\subsection{Cytopathology}\label{cytopathology}}

 - \textbf{The Diagnostic Accuracy of Cytology for the Diagnosis of
Hepatobiliary and Pancreatic Cancers}

\emph{Acta cytologica 2018 ;62(4):311-316}

OBJECTIVE: Although cytology testing is considered a valuable method to
diagnose tumors that are difficult to access such as
hepato-biliary-pancreatic (HBP) malignancies, its diagnostic accuracy
remains unclear. We therefore aimed to investigate the diagnostic
accuracy of cytology testing for HBP tumors. STUDY DESIGN: We performed
a retrospective study of all cytology samples that were used to confirm
radiologically detected HBP tumors between 2002 and 2016. The cytology
techniques used in our center included fine needle aspiration (FNA),
brush cytology, and aspiration of bile. Sensitivity, specificity,
positive and negative predictive values, and likelihood ratios were
calculated in comparison to histological confirmation. RESULTS: From a
total of 133 medical records, we calculated an overall sensitivity of
76\%, specificity of 74\%, a negative likelihood ratio of 0.30, and a
positive likelihood ratio of 2.9. Cytology was more accurate in
diagnosing lesions of the liver (sensitivity 79\%, specificity 57\%) and
biliary tree (sensitivity 100\%, specificity 50\%) compared to
pancreatic (sensitivity 60\%, specificity 83\%) and gallbladder lesions
(sensitivity 50\%, specificity 85\%). Cytology was more accurate in
detecting primary cancers (sensitivity 77\%, specificity 73\%) when
compared to metastatic cancers (sensitivity 73\%, specificity 100\%).
FNA was the most frequently used cytological technique to diagnose HBP
lesions (sensitivity 78.8\%). CONCLUSION: Cytological testing is
efficient in diagnosing HBP cancers, especially for hepatobiliary
tumors. Given its relative simplicity, cost-effectiveness, and paucity
of alternative diagnostic methods, cytology should still be considered
as a first-line tool for diagnosing HBP malignancies.

PubMed: \url{https://www.ncbi.nlm.nih.gov/pubmed/?term=29898439}

{}

{}

\begin{center}\rule{0.5\linewidth}{\linethickness}\end{center}

 - \textbf{Immunohistochemical Staining for S100P, SMAD4, and IMP3 on
Cell Block Preparations is Sensitive and Highly Specific for Pancreatic
Ductal Adenocarcinoma}

\url{https://www.sciencedirect.com/science/article/pii/S2213294518301042}

\begin{center}\rule{0.5\linewidth}{\linethickness}\end{center}

 - \textbf{Factors Impacting the Performance Characteristics of Bile
Duct Brushings: A Clinico-Cytopathologic Analysis of 253 Patients}

\emph{Archives of pathology \& laboratory medicine 2018
Jul;142(7):863-870}

CONTEXT: - Literature on factors impacting bile duct brushings (BDBs)
performance characteristics remain limited. OBJECTIVE: - To capture the
current state of daily practice with BDB sign-out. DESIGN: - Two hundred
fifty-three of 444 BDBs signed out by more than 7 cytopathologists, with
histopathologic and/or clinical follow-up of at least 18 months, were
examined. RESULTS: - One hundred thirty-five of 253 BDBs (53\%) had
histologically confirmed malignancies, 22 (9\%) had cancer-related
deaths, and 96 (38\%) were benign. Cytologic diagnoses in the 444 BDBs
were nondiagnostic (11 {[}2.5\%{]}), negative (284 {[}64\%{]}), atypical
(62 {[}13.9\%{]}), suspicious (34 {[}7.7\%{]}), and malignant (53
{[}11.9\%{]}). Sensitivity, specificity, positive predictive value
(PPV), negative predictive value (NPV), and accuracy of malignancy
detection were 35\%, 100\%, 100\%, 58\%, and 66\%, respectively. When
atypical, suspicious, and malignant (ASM) categories were combined,
sensitivity increased (58\%), specificity and PPV dropped (97\%), and
accuracy increased (73\%). Carcinoma type (bile-duct versus
pancreatic-ductal) had no effect on accuracy ( P = .60) or diagnostic
class ( P = .84), nor did time of performance (first 7.5 versus latter
7.5 years, P = .13). Interestingly, ThinPrep + cell block (n = 41) had
higher sensitivity (61\%) and lower specificity (80\%) than ThinPrep
only (versus 51\% and 100\%, respectively). Sensitivity and specificity
were higher (47\% and 100\%) in nonstented than stented specimens (59\%
and 97\%). Relative risk of malignancy for ``suspicious'' (2.30) and
``atypical'' (2.28) categories was lower but not very different from
that of ``malignant'' category (2.41). CONCLUSIONS: - Bile duct
brushings had fairly low sensitivity but high specificity and PPV with
no false positives. Sensitivity almost doubled and specificity dipped
minimally when ASM categories were combined, highlighting the need for
better classification criteria for atypical/suspicious cases. Higher
specificity, PPV, NPV, and accuracy but lower sensitivity in stented
BDBs suggest that they be called malignant only when evidence is
overwhelmingly convincing.

PubMed: \url{https://www.ncbi.nlm.nih.gov/pubmed/?term=29582676}

{}

{}

\begin{center}\rule{0.5\linewidth}{\linethickness}\end{center}

 - \textbf{UroVysion Multi-Target Fluorescence in situ Hybridization
Assay for the Detection of Malignant Bile Duct Brushing Specimens: A
Comparison with Routine Cytology}

\emph{Acta cytologica 2018 ;62(4):295-301}

OBJECTIVE: Routine bile duct brushing cytology is an important
diagnostic tool in the evaluation of bile duct stricture. The purpose of
this study was to evaluate the performance of the UroVysion fluorescence
in situ hybridization (FISH) assay for the detection of malignant bile
duct brushing specimens. STUDY DESIGN: Thirty-five bile duct brushing
specimens were included in the study. The FISH assay utilized the
commercially available UroVysion probes. The indeterminate cytology
results were considered as negative for statistical analysis. RESULTS:
Twenty-two of 35 patients were diagnosed as having malignancy based on
tissue diagnosis or clinical progression of disease by image assessment.
The sensitivity of routine cytology and FISH for the detection of
malignancy was 14\% (3/22) and 55\% (12/22), respectively (p = 0.003).
The specificity of routine cytology and FISH was 100\% (13/13) and 62\%
(8/13), respectively (p = 0.025). The false-positive rate for routine
cytology and FISH was 0\% (0/13) and 38\% (5/13), respectively.
CONCLUSIONS: Our study shows that FISH is significantly more sensitive
than routine cytology for the detection of malignancy in bile duct
brushing specimens. However, in our study, the specificity of FISH was
poor compared to the excellent specificity of routine cytology. The
compromised specificity of FISH may limit its utility in the detection
of malignant bile duct brushing specimens.

PubMed: \url{https://www.ncbi.nlm.nih.gov/pubmed/?term=29734171}

{}

{}

\begin{center}\rule{0.5\linewidth}{\linethickness}\end{center}

 - \textbf{Fine-Needle Aspiration-Based Grading of Pancreatic
Neuroendocrine Neoplasms Using Ki-67: Is Accurate WHO Grading Possible
on Cytologic Material?}

\url{https://www.ncbi.nlm.nih.gov/pubmed/30038894}

\url{https://www.sciencedirect.com/science/article/pii/S2213294517304143}

\begin{center}\rule{0.5\linewidth}{\linethickness}\end{center}

 - \textbf{Fine-needle aspiration of a pancreatic neuroendocrine tumor
with prominent rhabdoid features}

\emph{Diagnostic cytopathology 2018 Jul;46(7):600-603}

Pancreatic neuroendocrine tumors (PanNETs) are uncommon neoplasms that
conventionally possess architectural and cytomorphological features seen
in neuroendocrine neoplasms found at other sites. When present, these
features often allow rapid identification of neuroendocrine
differentiation and an accurate diagnosis. Here, we report the cytologic
findings seen on fine-needle aspiration (FNA) of a PanNET with distinct
rhabdoid features. This morphology is rare in PanNETs and has been
reported in only two case series examining surgical resection specimens
and has not been described on FNA. It is important to recognize this
morphology as this variant appears to portend an aggressive clinical
course. Furthermore, unfamiliarity with this morphologic variant may
lead to a larger initial differential and thus delay final diagnosis.

PubMed: \url{https://www.ncbi.nlm.nih.gov/pubmed/?term=29359517}

{}

{}

\begin{center}\rule{0.5\linewidth}{\linethickness}\end{center}

 - \textbf{Endoscopic ultrasound guided brush/fine-needle aspiration
cytology: A 15-month study}

\emph{Diagnostic cytopathology 2018 Jun;46(6):461-472}

BACKGROUND: Endoscopic ultrasound-guided fine-needle aspiration
(EUS-FNA) has become increasingly popular for the diagnosis and staging
of gastrointestinal diseases and peri-gastrointestinal lesions. The
application of FNA/Brush has dramatically expanded the clinical utility
of EUS. AIMS AND OBJECTIVE: To evaluate the diagnostic accuracy, study
the spectrum of lesions encountered in EUS-FNAC/brush cytology of
gastrointestinal and peri-gastrointestinal lesions. MATERIALS AND
METHODS: Total of 124 patients during the period from August 2015 to
November 2016 was included in the study. Routine staining was done.
RESULTS: A total of 124 cases were studied with 86\% (107 cases) being
satisfactory for evaluation. M:F ratio was 1:1.03, mean age of 50.5
years. The most common site was common bile duct (CBD) (37\%) followed
by lymph node (21\%), pancreas (17.7\%), esophagus (17\%), stomach
(3.5\%), liver (1.8\%), gallbladder (1\%), and spleen (1\%). In total,
53.4\% lesions were benign, in 6.5\% atypical cells were seen, 12.1\%
were suspicious for malignancy, and 28\% cases were positive for
malignancy. Follow-up was available in 102 cases with
cyto-histopathological concordance rate of 90\%. CONCLUSION:
EUS-FNA/Brush is a reliable, sensitive, specific and minimally invasive
way to establish a diagnosis. It can be utilized as a pre-operative
procedure for the management of many intra-abdominal lesions and prevent
unnecessary invasive procedures.

PubMed: \url{https://www.ncbi.nlm.nih.gov/pubmed/?term=29504276}

{}

{}

\begin{center}\rule{0.5\linewidth}{\linethickness}\end{center}

Back to top

\pagebreak

\hypertarget{molecular-pathology}{%
\subsection{Molecular Pathology}\label{molecular-pathology}}

 - \textbf{Simple Detection of Telomere Fusions in Pancreatic Cancer,
Intraductal Papillary Mucinous Neoplasm, and Pancreatic Cyst Fluid}

\emph{The Journal of molecular diagnostics : JMD 2018 Jan;20(1):46-55}

Telomere end-to-end fusions are an important source of chromosomal
instability that arise in cells with critically shortened telomeres. We
developed a nested real-time quantitative PCR method for telomere fusion
detection in pancreatic ductal adenocarcinomas, intraductal papillary
mucinous neoplasms (IPMNs), and IPMN cyst fluids. Ninety-one pancreatic
cancer cell lines and xenograft samples, 93 IPMNs, and 93 surgically
aspirated IPMN cyst fluid samples were analyzed. The association between
telomere shortening, telomerase activity, and telomere fusion detection
was evaluated. Telomere fusions were detected in 56 of 91 pancreatic
cancers (61.5\%). Telomere fusion-positive cell lines had significantly
shorter telomere lengths than fusion-negative lines (P~=~0.003).
Telomere fusions were undetectable in normal pancreas or IPMNs with
low-grade dysplasia (0.0\%) and were detected in IPMN with high-grade
dysplasia (HGD; 48.0\%) (P~\textless{}~0.001). In IPMN cyst fluids,
telomere fusions were more frequent in IPMNs with HGD (26.9\%) or
associated invasive cancer (42.9\%) than IPMN with intermediate-grade
dysplasia (15.4\%) or low-grade dysplasia (0\%) (P~=~0.025). Telomerase
activity levels were higher in cyst fluids with fusions than in those
without (P~=~0.0414). Cyst fluid telomere fusion status was an
independent predictor of HGD/invasive cancer by multivariate analysis
(odds ratio, 6.23; 95\% CI, 1.61-28.0). Telomere fusions are detected in
later stages of IPMN progression and can serve as a marker for
predicting the presence of HGD and/or invasive cancer.

PubMed: \url{https://www.ncbi.nlm.nih.gov/pubmed/?term=29229290}

{}

{}

\begin{center}\rule{0.5\linewidth}{\linethickness}\end{center}

 - \textbf{Genomic testing for pancreatic cancer in clinical practice as
real-world evidence}

\url{https://www.pancreatology.net/article/S1424-3903(18)30633-1/abstract}

\begin{center}\rule{0.5\linewidth}{\linethickness}\end{center}

 - \textbf{MiR-21 up-regulation in ampullary adenocarcinoma and its
pre-invasive lesions}

\emph{Pathology, research and practice 2018 Jun;214(6):835-839}

Poor information is available on the molecular landscape characterizing
the carcinogenetic process leading to ampullary carcinoma. MiR-21 is one
of the most frequently up-regulated miRNAs in pancreatic adenocarcinoma,
a tumor sharing similar molecular features with ampullary
adenocarcinomas (AVCs), above all with the pancreatic-biliary type. We
profiled, by in situ hybridization (ISH), miR-21 expression in a series
of 26 AVCs, 50 ampullary dysplastic lesions (35 low-grade {[}LG-IEN{]}
and 15 high-grade {[}HG-IEN{]}) and 10 normal duodenal mucosa samples.
The same series was investigated by immunohistochemistry for β-catenin,
p53 and HER2 expression. HER2 gene amplification was evaluated by
chromogenic in situ hybridization. To validate miR-21 ISH results we
performed miR-21 qRT-PCR analysis in a series of 10 AVCs and their
matched normal samples. All the normal control samples showed a negative
or faint miR-21 expression, whereas a significant miR-21 up-regulation
was observed during the carcinogenetic cascade (p\,\textless{}\,0.001),
with 21/26 (80.8\%) of cancer samples showing a miR-21 overexpression.
In comparison to control samples, a significant overexpression was found
in samples of LG-IEN (p\,=\,.0003), HG-IEN (p\,=\,.0001), and AVCs
(p\,\textless{}\,0.0001). No significant difference in miR-21
overexpression was observed between LG-IEN, HG-IEN and AVCs. By qRT-PCR
analysis, AVCs showed a 1.7-fold increase over the controls
(p\,=\,.003). P53 was frequently dysregulated in both dysplastic and
carcinoma samples (44 out of 76; 57.9\%). A 20\% (10/50) of dysplastic
lesions and 11\% (3/26) of carcinomas were characterized by a nuclear
localization of β-catenin. Only 2 AVCs (7.7\%; both intestinal-type)
showed a HER2 overexpression (both 2+), which corresponded to a HER2
gene amplification at CISH analysis. This is the first study
demonstrating a miRNA dysregulation in the whole spectrum of ampullary
carcinogenesis. MiR-21 overexpression is an early molecular event during
ampullary carcinogenesis and its levels increase with the neoplastic
progression.

PubMed: \url{https://www.ncbi.nlm.nih.gov/pubmed/?term=29731265}

{}

{}

\begin{center}\rule{0.5\linewidth}{\linethickness}\end{center}

 - \textbf{Reduced RNA-binding protein HuD in pancreatic neuroendocrine
tumors lowers p27Kip1 levels linked to poor prognosis}

\emph{The Journal of pathology 2018 Jul;():}

For the majority of patients diagnosed with pancreatic neuroendocrine
tumors (pancreatic NETs) there is a significant malignant potential with
a poor prognosis, however the molecular abnormalities and pathogenesis
of pancreatic NETs have not been firmly established. Here, we report
that loss of RNA-binding protein HuD expression correlates with low
p27Kip1 (p27) levels and poor prognosis in pancreatic NETs. HuD
expression was frequently lost in many human pancreatic NETs and these
pancreatic NETs showed aggressive clinico-pathological phenotypes with
low p27 levels, increased tumor size, higher WHO grade and pathological
T stage of the tumor, and presence of angioinvasion. Furthermore, loss
of HuD was an independent, progress-free prognostic factor in
multivariate survival analysis. However, level of HuR, the same Hu
protein family member with HuD, was not significantly correlated with
pancreatic NET size and progression. Mechanistically, HuD enhanced p27
mRNA translation by interacting with both 5'- and 3'-untranslated
regions (UTRs) of p27 mRNA and consequently suppressed cell cycle
progression and tumor growth. In addition, HuD competed with miR-30a-3p
for binding to 3'UTR of p27 mRNA, suggesting interplay between HuD and
miR-30a-3p in controlling p27 translation. Our results identify HuD as a
pivotal suppressor of pancreatic NET growth, and propose that HuD has
potential value as a prognostic factor of pancreatic NETs. This article
is protected by copyright. All rights reserved.

PubMed: \url{https://www.ncbi.nlm.nih.gov/pubmed/?term=30014466}

{}

{}

\begin{center}\rule{0.5\linewidth}{\linethickness}\end{center}

 - \textbf{Pancreatic neuroendocrine carcinomas reveal a closer
relationship to ductal adenocarcinomas than to neuroendocrine tumors G3}

\emph{Human pathology 2018 Jul;77():70-79}

Pancreatic neuroendocrine carcinoma is a rare aggressive tumor commonly
harboring TP53 and RB1 alterations and lacking neuroendocrine-related
genetic changes such as mutations in MEN1 and ATRX/DAXX. Little is known
about its genetic profile with regard to that of pancreatic ductal
adenocarcinoma. We therefore conducted a detailed genetic study in 12
pancreatic neuroendocrine carcinomas of large cell (n = 9) and small
cell type (n = 3) using massive parallel sequencing applying a 409-gene
panel on an Ion Torrent system. The genetic data were compared with
known data of pancreatic ductal adenocarcinoma and correlated with
exocrine lineage marker expression. A similar analysis was performed in
11 pancreatic neuroendocrine tumors G3. Neuroendocrine carcinomas
harbored 63 somatic mutations in 45 different genes, affecting most
commonly TP53 (8/12 cases), KRAS (5/12 cases), and RB1 (loss of
expression with or without deletion in 4/12 cases). Five carcinomas had
both TP53 and KRAS mutations. Neuroendocrine tumors G3 only shared
singular mutations in 5 different genes with neuroendocrine carcinomas,
including TP53, CDKN2A, ARID1A, LRP1B, and APC, affecting 5 different
cases. Most KRAS-positive neuroendocrine carcinomas also expressed MUC1
(4/5) and carcinoembryonic antigen (3/5) as markers of ductal
differentiation. Our data indicate that almost half of the pancreatic
neuroendocrine carcinomas are genetically and phenotypically related to
pancreatic ductal adenocarcinoma, and might therefore respond to
chemotherapies targeting the latter carcinomas.

PubMed: \url{https://www.ncbi.nlm.nih.gov/pubmed/?term=29596894}

{}

{}

\begin{center}\rule{0.5\linewidth}{\linethickness}\end{center}

Back to top

\pagebreak

\hypertarget{others}{%
\subsection{Others}\label{others}}

 - \textbf{VEGF receptor-2/neuropilin1 trans-complex formation between
endothelial and tumor cells is an independent predictor of pancreatic
cancer survival}

\emph{The Journal of pathology 2018 Jul;():}

Unstable and dysfunctional tumor vasculature promotes cancer progression
and spread. Signal transduction by the pro-angiogenic vascular
endothelial growth factor (VEGF) receptor-2 (VEGFR2) is modulated by
VEGFA-dependent complex formation with Neuropilin-1 (NRP1). NRP1
expressed on tumor cells can form VEGFR2/NRP1 trans-complexes between
tumor cells and endothelial cells which arrests VEGFR2 on the
endothelial surface, thus interfering with productive VEGFR2 signaling.
In mouse fibrosarcoma, VEGFR2/NRP1 trans-complexes correlated with
reduced tumor vessel branching and reduced tumor cell proliferation.
Pancreatic ductal adenocarcinoma (PDAC) strongly expressed NRP1 on both
tumor cells and endothelial cells in contrast to other common cancer
forms. Using proximity ligation assay, VEGFR2/NRP1 trans-complexes were
identified in human PDAC tumor tissue, and its presence was associated
with reduced tumor vessel branching, reduced tumor cell proliferation
and improved patient survival after adjusting for other known survival
predictors. We conclude that VEGFR2/NRP1 trans-complex formation is an
independent predictor of PDAC patient survival. This article is
protected by copyright. All rights reserved.

PubMed: \url{https://www.ncbi.nlm.nih.gov/pubmed/?term=30027561}

{}

{}

\begin{center}\rule{0.5\linewidth}{\linethickness}\end{center}

 - \textbf{Immunolabeling of Cleared Human Pancreata Provides Insights
into Three-Dimensional Pancreatic Anatomy and Pathology}

\emph{The American journal of pathology 2018 Jul;188(7):1530-1535}

Visualizing pathologies in three dimensions can provide unique insights
into the biology of human diseases. A rapid and easy-to-implement
dibenzyl ether-based technique was used to clear thick sections of
surgically resected human pancreatic parenchyma. Protocols were
applicable to both fresh and formalin-fixed, paraffin-embedded tissue.
The penetration of antibodies into dense pancreatic parenchyma was
optimized using both gradually increasing antibody concentrations and
centrifugal flow. Immunolabeling with antibodies against cytokeratin 19
was visualized using both light sheet and confocal laser scanning
microscopy. The technique was applied successfully to 26 sections of
pancreas, providing three-dimensional (3D) images of normal pancreatic
tissue, pancreatic intraepithelial neoplasia, intraductal papillary
mucinous neoplasms, and infiltrating pancreatic ductal adenocarcinomas.
3D visualization highlighted processes that are hard to conceptualize in
two dimensions, such as invasive carcinoma growing into what appeared to
be pre-existing pancreatic ducts and within venules, and the tracking of
long cords of neoplastic cells parallel to blood vessels. Expanding this
technique to formalin-fixed, paraffin-embedded tissue opens pathology
archives to 3D visualization of unique biosamples and rare diseases. The
application of immunolabeling and clearing to human pancreatic
parenchyma provides detailed visualization of normal pancreatic anatomy,
and can be used to characterize the 3D architecture of diseases
including pancreatic intraepithelial neoplasia, intraductal papillary
mucinous neoplasm, and pancreatic ductal adenocarcinomas.

\url{https://www.sciencedirect.com/science/article/pii/S0002944018300014}

PubMed: \url{https://www.ncbi.nlm.nih.gov/pubmed/?term=29684363}

{}

{}

\begin{center}\rule{0.5\linewidth}{\linethickness}\end{center}

 - \textbf{The BRG1/SOX9 axis is critical for acinar cell-derived
pancreatic tumorigenesis}

\emph{The Journal of clinical investigation 2018 Aug;128(8):3475-3489}

Chromatin remodeler Brahma related gene 1 (BRG1) is silenced in
approximately 10\% of human pancreatic ductal adenocarcinomas (PDAs). We
previously showed that BRG1 inhibits the formation of intraductal
pancreatic mucinous neoplasm (IPMN) and that IPMN-derived PDA originated
from ductal cells. However, the role of BRG1 in pancreatic
intraepithelial neoplasia-derived (PanIN-derived) PDA that originated
from acinar cells remains elusive. Here, we found that exclusive
elimination of Brg1 in acinar cells of Ptf1a-CreER; KrasG12D; Brg1fl/fl
mice impaired the formation of acinar-to-ductal metaplasia (ADM) and
PanIN independently of p53 mutation, while PDA formation was inhibited
in the presence of p53 mutation. BRG1 bound to regions of the Sox9
promoter to regulate its expression and was critical for recruitment of
upstream regulators, including PDX1, to the Sox9 promoter and enhancer
in acinar cells. SOX9 expression was downregulated in BRG1-depleted
ADMs/PanINs. Notably, Sox9 overexpression canceled this PanIN-attenuated
phenotype in KBC mice. Furthermore, Brg1 deletion in established PanIN
by using a dual recombinase system resulted in regression of the lesions
in mice. Finally, BRG1 expression correlated with SOX9 expression in
human PDAs. In summary, BRG1 is critical for PanIN initiation and
progression through positive regulation of SOX9. Thus, the BRG1/SOX9
axis is a potential target for PanIN-derived PDA.

PubMed: \url{https://www.ncbi.nlm.nih.gov/pubmed/?term=30010625}

{}

{}

\begin{center}\rule{0.5\linewidth}{\linethickness}\end{center}

 - \textbf{Pancreatic Effects of a Bruton's Tyrosine Kinase
Small-molecule Inhibitor in Rats Are Strain-dependent}

\emph{Toxicologic pathology 2018 Jun;46(4):460-472}

Inhibitors of Bruton's tyrosine kinase (BTK) are under development as
potential therapies for various autoimmune diseases. In repeat-dose
toxicity studies, small-molecule BTK inhibitors (BTKi) have been
reported to cause a constellation of histologic effects at the
pancreatic endocrine-exocrine interface in male rats; however, similar
findings were not reported in other species. Since the BTKi-induced
pancreatic effect is morphologically similar to well-documented
spontaneous changes (predominantly characterized by insular/peri-insular
hemorrhage, pigment deposition, chronic inflammation, and fibrosis) that
are known to vary by rat strain, we investigated potential
strain-dependent differences in the pancreatic effects of a
small-molecule BTKi, LY3337641. Following 13 weeks of LY3337641
treatment, Crl:CD(SD) rats were most sensitive, Crl:WI(Han) rats were of
intermediate sensitivity, and Hsd:SD rats were least sensitive. These
strain differences appear to be related to differences in rate of weight
gain across strains and sexes; however, a definitive mechanism was not
determined. This study demonstrated that BTKi-induced pancreatic effects
were highly dependent on rat strain and correlated with differences in
the incidence and severity of the spontaneous background change. When
considered with the lack of pancreas effects in nonrat species, these
changes in rats are unlikely predictive of similar changes in humans
administered a BTK inhibitor.

PubMed: \url{https://www.ncbi.nlm.nih.gov/pubmed/?term=29699458}

{}

{}

\begin{center}\rule{0.5\linewidth}{\linethickness}\end{center}

 - \textbf{Orthotopic and heterotopic murine models of pancreatic cancer
and their different responses to FOLFIRINOX chemotherapy}

\emph{Disease models \& mechanisms 2018 Jul;11(7):}

Syngeneic, immunocompetent allograft tumor models recapitulate important
aspects of the tumor microenvironment and have short tumor latency with
predictable growth kinetics, making them useful for trialing novel
therapeutics. Here, we describe surgical techniques for orthotopic and
heterotopic pancreatic ductal adenocarcinoma (PDAC) tumor implantation
and characterize phenotypes based on implantation site.Mice (n=8 per
group) were implanted with 104 cells in the pancreas or flank. Hy15549
and Han4.13 cell lines were derived from primary murine PDAC (Ptf1-Cre;
LSL-KRAS-G12D; Trp53 Lox/+) on C57BL/6 and FVB strains, respectively.
Single-cell suspension and solid tumor implants were compared. Tumors
were treated with two intravenous doses of FOLFIRINOX and responses
evaluated.All mice developed pancreatic tumors within 7 days. Orthotopic
tumors grew faster and larger than heterotopic tumors. By 3 weeks,
orthotopic mice began losing weight, and showed declines in body
condition requiring euthanasia starting at 4 weeks. Single-cell
injection into the pancreas had near 100\% engraftment, but solid tumor
implant engraftment was ∼50\% and was associated with growth
restriction. Orthotopic tumors were significantly more responsive to
intravenous FOLFIRINOX compared with heterotopic tumors, with greater
reductions in size and increased apoptosis. Heterotopic tumors were more
desmoplastic and hypovascular. However, drug uptake into tumor tissue
was equivalent regardless of tumor location or degree of fibrosis,
indicating that microenvironment differences between heterotopic and
orthotopic tumors influenced response to therapy.Our results show that
orthotopic and heterotopic allograft locations confer unique
microenvironments that influence growth kinetics, desmoplastic response
and angiogenesis. Tumor location influences chemosensitivity to
FOLFIRINOX and should inform future preclinical trials.This article has
an associated First Person interview with the first author of the paper.

PubMed: \url{https://www.ncbi.nlm.nih.gov/pubmed/?term=29903803}

{}

{}

\begin{center}\rule{0.5\linewidth}{\linethickness}\end{center}

Back to top

\pagebreak

\hypertarget{journals-reviewed}{%
\subsection{Journals Reviewed}\label{journals-reviewed}}

Advances in Anatomic Pathology

American Journal of Clinical Pathology

The American Journal of Gastroenterology

The American Journal of Pathology

American Journal of Surgical Pathology

Annals of Diagnostic Pathology

Annals of Surgery

Annals of Surgical Oncology

Annual Review of Pathology-Mechanisms of Disease

APMIS : acta pathologica, microbiologica, et immunologica Scandinavica

Applied Immunohistochemistry \& Molecular Morphology

Archives of Pathology \& Laboratory Medicine

Cancer

Cancer Cell

Cancer Cytopathology

Cellular Oncology

Clinical Cancer Research

Cochrane Reviews

Cytojournal

Cytopathology

Diagnostic Cytopathology

Diagnostic Pathology

Endocrine Pathology

Experimental and Molecular Pathology

Expert Review of Molecular Diagnostics

Gastroenterology

Gut

Histology and Histopathology

Histopathology

Human Pathology

International Journal of Surgical Pathology

International Journal of Clinical and Experimental Pathology

Journal of Clinical Pathology

Journal of Molecular Diagnostics

Journal of Pathology

Laboratory investigation

Lancet

Medical Molecular Morphology

Modern Pathology

Nature

NEJM

Pancreas

Pancreatology

Pathobiology

Pathologie Biologie

Pathology

Pathology \& Oncology Research

Pathology International

Pathology Research and Practice

PNAS

Science

Seminars in Diagnostic Pathology

Seminars in Immunopathology

The Journal of Pathology

Tissue Antigens

Virchows Archiv

\begin{center}\rule{0.5\linewidth}{\linethickness}\end{center}

Back to top

\pagebreak

\hypertarget{feedback}{%
\subsection{Feedback}\label{feedback}}

Please send your feedbacks using the form below:

\href{https://docs.google.com/forms/d/e/1FAIpQLSeD3Z9J6Y7eMmiyM12f_SfAmHUlykb1zxZcwO6lg7cebGYQIQ/viewform}{Click
here for the Feedback Form}

\begin{center}\rule{0.5\linewidth}{\linethickness}\end{center}

Back to top

\pagebreak

\hypertarget{archive}{%
\subsection{Archive}\label{archive}}

\textbf{The PBPath Journal Archive}

\begin{itemize}
\item
  \href{http://pbpath.org/journal-watch/}{Current Issue}
\item
  \href{http://pbpath.org/journal-watch-archive/}{Older Issues}

  \begin{itemize}
  \item
    \href{http://pbpath.org/pbpath-journal-watch-june-july-2018/}{June-July-2018}
  \item
    August - Semptember issue is being prepared. Unofficial version is
    \href{http://pbpath.org/journal-watch-upcoming-issue/}{here}.
  \end{itemize}
\end{itemize}

\begin{center}\rule{0.5\linewidth}{\linethickness}\end{center}

Back to top


\end{document}
