\documentclass[]{article}
\usepackage{lmodern}
\usepackage{amssymb,amsmath}
\usepackage{ifxetex,ifluatex}
\usepackage{fixltx2e} % provides \textsubscript
\ifnum 0\ifxetex 1\fi\ifluatex 1\fi=0 % if pdftex
  \usepackage[T1]{fontenc}
  \usepackage[utf8]{inputenc}
\else % if luatex or xelatex
  \ifxetex
    \usepackage{mathspec}
  \else
    \usepackage{fontspec}
  \fi
  \defaultfontfeatures{Ligatures=TeX,Scale=MatchLowercase}
\fi
% use upquote if available, for straight quotes in verbatim environments
\IfFileExists{upquote.sty}{\usepackage{upquote}}{}
% use microtype if available
\IfFileExists{microtype.sty}{%
\usepackage{microtype}
\UseMicrotypeSet[protrusion]{basicmath} % disable protrusion for tt fonts
}{}
\usepackage[margin=1in]{geometry}
\usepackage{hyperref}
\hypersetup{unicode=true,
            pdftitle={Pancreatobiliary Pathology Society Journal Watch},
            pdfborder={0 0 0},
            breaklinks=true}
\urlstyle{same}  % don't use monospace font for urls
\usepackage{graphicx,grffile}
\makeatletter
\def\maxwidth{\ifdim\Gin@nat@width>\linewidth\linewidth\else\Gin@nat@width\fi}
\def\maxheight{\ifdim\Gin@nat@height>\textheight\textheight\else\Gin@nat@height\fi}
\makeatother
% Scale images if necessary, so that they will not overflow the page
% margins by default, and it is still possible to overwrite the defaults
% using explicit options in \includegraphics[width, height, ...]{}
\setkeys{Gin}{width=\maxwidth,height=\maxheight,keepaspectratio}
\IfFileExists{parskip.sty}{%
\usepackage{parskip}
}{% else
\setlength{\parindent}{0pt}
\setlength{\parskip}{6pt plus 2pt minus 1pt}
}
\setlength{\emergencystretch}{3em}  % prevent overfull lines
\providecommand{\tightlist}{%
  \setlength{\itemsep}{0pt}\setlength{\parskip}{0pt}}
\setcounter{secnumdepth}{0}
% Redefines (sub)paragraphs to behave more like sections
\ifx\paragraph\undefined\else
\let\oldparagraph\paragraph
\renewcommand{\paragraph}[1]{\oldparagraph{#1}\mbox{}}
\fi
\ifx\subparagraph\undefined\else
\let\oldsubparagraph\subparagraph
\renewcommand{\subparagraph}[1]{\oldsubparagraph{#1}\mbox{}}
\fi

%%% Use protect on footnotes to avoid problems with footnotes in titles
\let\rmarkdownfootnote\footnote%
\def\footnote{\protect\rmarkdownfootnote}

%%% Change title format to be more compact
\usepackage{titling}

% Create subtitle command for use in maketitle
\newcommand{\subtitle}[1]{
  \posttitle{
    \begin{center}\large#1\end{center}
    }
}

\setlength{\droptitle}{-2em}

  \title{Pancreatobiliary Pathology Society Journal Watch}
    \pretitle{\vspace{\droptitle}\centering\huge}
  \posttitle{\par}
  \subtitle{December 2018 - January 2019}
  \author{}
    \preauthor{}\postauthor{}
      \predate{\centering\large\emph}
  \postdate{\par}
    \date{Last Update on 2019-02-08}


\begin{document}
\maketitle

{
\setcounter{tocdepth}{5}
\tableofcontents
}
\begin{center}\rule{0.5\linewidth}{\linethickness}\end{center}

\hypertarget{pbpath-journal-watch-articles}{%
\section{PBPath Journal Watch
Articles}\label{pbpath-journal-watch-articles}}

\begin{center}\rule{0.5\linewidth}{\linethickness}\end{center}

\textbf{Wellcome to the PBPath Journal Watch!}

This bi-monthly journal watch features exciting recently published
pancreas and biliary pathology articles that will provide up to date
medical knowledge in our field. These articles will be showcased in
several convenient categories, including surgical pathology,
cytopathology, and molecular pathology among others. The articles in
each category are in no particular order.

Previous months' issues may be found in our
\emph{\href{http://pbpath.org/journal-watch-archive/}{archive}}.

We encourage members to actively participate by recommending new
articles and providing feedback using
\emph{\href{https://docs.google.com/forms/d/e/1FAIpQLSeD3Z9J6Y7eMmiyM12f_SfAmHUlykb1zxZcwO6lg7cebGYQIQ/viewform}{the
forms provided}}.

\textbf{We hope that you will enjoy the new PBPath Journal Watch!}

\pagebreak

\begin{center}\rule{0.5\linewidth}{\linethickness}\end{center}

\hypertarget{surgical-pathology}{%
\subsection{Surgical Pathology}\label{surgical-pathology}}

\begin{center}\rule{0.5\linewidth}{\linethickness}\end{center}

\hypertarget{pancreas}{%
\subsubsection{Pancreas}\label{pancreas}}

\begin{center}\rule{0.5\linewidth}{\linethickness}\end{center}

\hypertarget{morphology-diagnostics-ihc}{%
\paragraph{Morphology, Diagnostics,
IHC}\label{morphology-diagnostics-ihc}}

Morphology, Diagnostics, IHC

\begin{center}\rule{0.5\linewidth}{\linethickness}\end{center}

 - \textbf{Malformations, choristomas, and hamartomas of the
gastrointestinal tract and pancreas}

\emph{Seminars in diagnostic pathology 2019 Jan;36(1):24-38}

PubMed: \url{https://www.ncbi.nlm.nih.gov/pubmed/?term=30482417}

Congenital and hamartomatous lesions of the gastrointestinal tract cause
diagnostic challenges for surgical pathologists. Many of these are
merely histologic curiosities, whereas others have substantial clinical
implications because they herald cancer syndromes or associated
anomalies. Although a comprehensive discussion of all developmental
abnormalities that can occur in the gastrointestinal tract is beyond the
scope of a single manuscript, some entities are more likely to be
encountered by surgical pathologists, have important clinical
consequences, or pose diagnostic difficulties. The purpose of this
review is to discuss the more common malformations and choristomas, as
well as hamartomatous lesions that may be clinically important due to
their risk for cancer development, frequent associations with heritable
cancer syndromes and other anomalies, or potential to simulate other
entities.

{}

{}

\begin{center}\rule{0.5\linewidth}{\linethickness}\end{center}

 - \textbf{AQP1 and AQP3 Expression are Associated With Severe Symptoms
and Poor-prognosis of the Pancreatic Ductal Adenocarcinoma}

\emph{Applied immunohistochemistry \& molecular morphology : AIMM 2019
Jan;27(1):40-47}

PubMed: \url{https://www.ncbi.nlm.nih.gov/pubmed/?term=30531392}

BACKGROUND: Approximately 80\% of patients with pancreatic ductal
adenocarcinoma (PDAC) have metastatic disease with poor prognosis, but
clinically available biomarkers for the diagnosis, prediction of
prognosis, and target therapy have not yet been identified. OBJECTIVE:
To investigate the expression of aquaporin-1 (AQP1) and AQP3 protein and
their clinicopathological significances in PDACs. MATERIALS AND METHOD:
AQP1 and AQP3 protein expression in 106 PDAC, 35 peritumoral tissues, 55
benign pancreatic lesions, and 13 normal pancreatic tissues was measured
by immunohistochemistry. RESULTS: Western blot showed that AQP1 and AQP3
protein expression was significantly higher in PDAC tissues than that in
benign pancreatic tissues (P\textless{}0.01). Immunohistochemistry
showed that the percentages of positive AQP1 and AQP3 expressions were
significantly higher in PDAC tumors than that in peritumoral tissues,
benign, and normal pancreatic tissues (P\textless{}0.01). Benign
pancreatic lesions with positive AQP1 and AQP3 expression exhibited a
dysplasia or intraepithelial neoplasia. The percentage of cases with
positive AQP1 and AQP3 expression was significantly lower in PDAC
patients without lymph node metastasis and invasion, and having low
Tumor, Node and Metastasis (TNM) stage disease than in patients with
lymph node metastasis, invasion, and high TNM stage disease
(P\textless{}0.05 or \textless{}0.01). Kaplan-Meier survival analysis
showed that positive AQP1 and AQP3 expression were significantly
associated with survival in PDAC patients (P\textless{}0.001). Cox
multivariate analysis revealed that positive AQP1 and AQP3 expression
was independent poor prognosis factors in PDAC patients. The area under
the curve of receiver operating characteristic curve was 0.669 for AQP1
and 0.707 for AQP3, respectively. CONCLUSIONS: Positive AQP1 and AQP3
expressions are associated with the tumorigenesis and progression of
PDAC. Both AQP1 and AQP3 are a diagnostic marker of PDAC and a
predictive marker of poor prognosis in PDAC patients.

{}

{}

\begin{center}\rule{0.5\linewidth}{\linethickness}\end{center}

 - \textbf{Clinicopathologic and prognostic significance of MKK4 and
MKK7 in resectable pancreatic ductal adenocarcinoma}

\emph{Human pathology 2018 Dec;():}

PubMed: \url{https://www.ncbi.nlm.nih.gov/pubmed/?term=30537492}

Mitogen-activated protein kinase kinase 4 (MKK4) and mitogen-activated
protein kinase kinase 7 (MKK7) were shown to regulate biological
behavior in many malignancies. In pancreatic ductal adenocarcinoma
(PDAC), it remains controversial whether MKK4 and MKK7 have
pro-oncogenic or tumor-suppressive activities. Furthermore, their
clinicopathologic and prognostic implications are unknown. In the
present study, we detected MKK4 and MKK7 expressions in the nucleus and
cytoplasm of resected PDAC tissues from 321 patients by tissue
microarray-based immunohistochemistry. Cytoplasmic MKK4 and MKK7
expressions were significantly down-regulated, while nuclear MKK4
expression was significantly up-regulated in tumor tissues compared with
non-tumor tissues. Tumor cytoplasmic MKK4 and MKK7 expressions were
significantly negatively associated with histological grade. Cytoplasmic
MKK4 expression was also negatively correlated with CA19-9 level. By
univariate analysis, high cytoplasmic MKK4 expression was significantly
associated with longer cancer-specific survival (hazard ratio {[}HR{]}:
0.705; 95\% confidence interval {[}95\%CI{]}: 0.510-0.974), with a
similar trend observed for MKK7 expression. High MKK4 and MKK7 mRNA
expressions were significantly associated with longer overall survival
in the TCGA database. Although MKK4 expression was not significant in a
multivariate Cox regression analysis, combination of MKK4/MKK7 and pN
stage was identified as an independent prognostic indicator and had the
lowest HR (HR: 0.308; 95\%CI: 0.126-0.752). Furthermore, combined
analysis of MKK4 and MKK7 greatly increased the prognostic predictive
power. In addition, down-regulation of MKK4 or MKK7 increased
proliferation of pancreatic cancer cells in vitro. In conclusion, high
MKK4 expression and its combination with high MKK7 expression both
predicted favorable prognosis in resectable PDAC.

{}

{}

\begin{center}\rule{0.5\linewidth}{\linethickness}\end{center}

 - \textbf{PD-L1 expression in pancreatic adenosquamous carcinoma: PD-L1
expression is limited to the squamous component}

\emph{Pathology, research and practice 2018 Dec;214(12):2069-2074}

PubMed: \url{https://www.ncbi.nlm.nih.gov/pubmed/?term=30477643}

AIM: We examined the programmed death-ligand 1 (PD-L1) expression in
surgically resected pancreatic adenosquamous carcinoma (PASC) samples.
Furthermore, the detection rate was also assessed using biopsy cases
obtained from endoscopic ultrasound-guided fine needle aspiration
(EUS-FNA). METHODS: Fifteen cases of PASC (six resected and nine EUS-FNA
biopsied) from the Kurume University Hospital between 2009 and 2016 were
used for the evaluation of PD-L1 expression. As a control group, 34
cases of pancreatic ductal adenocarcinomas (PDACs) were selected. To
compare the positivity and intensity of PD-L1, two types of clones
(SP263, E1L3N) were examined for immunostaining. Only the membrane
expression of PD-L1 was regarded as positive. The PD-L1 expressions in
the squamous cell carcinoma component (SCc), adenocarcinoma component
(ACc), and immune cells were assessed separately. The ratio of PD-L1
expression was calculated by counting the positive tumor cells, and
tumor proportion score (TPS) was applied (TPS; Null \textless{} 1\%, low
expression; 1 ≤ TPS ≤ 49\% and high expression; ≥ 50\%). RESULTS: PD-L1
expression was observed in five surgical PASC samples (83\%). This shows
that SCc presented a high expression in these cases. However, the
overall TPS indicated a low expression. In contrast, only one case (3\%)
was positive for PD-L1 in PDACs, and the TPS indicated a low expression.
No differences in PD-L1 expression were observed between the two clones,
SP263 and E1L3N. High PD-L1 expression in the EUS-FNA sample was found
in only one case (11\%). DISCUSSION: Although assessment using the tumor
cells of PASC samples obtained from EUS-FNA was difficult, this study
suggests the selective expression of PD-L1 in the SCc of PASC.
Furthermore, it was considered that immune checkpoint inhibitors could
provide therapeutic effects selectively on the SCc for the entire range
of TPSs, though the PD-L1 expression was low.

{}

{}

\begin{center}\rule{0.5\linewidth}{\linethickness}\end{center}

 - \textbf{Pancreatic Ductal Adenocarcinoma: Recent Updates}

\emph{The American journal of pathology 2019 Jan;189(1):6-8}

PubMed: \url{https://www.ncbi.nlm.nih.gov/pubmed/?term=30558724}

This Guest Editorial introduces this month's special Pancreatic Cancer
Theme Issue, a series of reviews intended to highlight the pathologic to
molecular profiles and diagnoses of benign and neoplastic pancreatic
lesions.

{}

{}

\begin{center}\rule{0.5\linewidth}{\linethickness}\end{center}

 - \textbf{Pancreatic Ductal Adenocarcinoma and Its Precursor Lesions:
Histopathology, Cytopathology, and Molecular Pathology}

\emph{The American journal of pathology 2019 Jan;189(1):9-21}

PubMed: \url{https://www.ncbi.nlm.nih.gov/pubmed/?term=30558727}

Pancreatic ductal adenocarcinoma is one of the most aggressive malignant
neoplasms with poor outcomes. At the time of diagnosis, the disease is
usually at an advanced stage and only a minority is eligible for
surgical resection. To improve the prognosis, it is essential to
diagnose and treat the disease in an early stage before its progression
into an invasive disease. This article reviews clinical features,
histopathology, cytopathology, and molecular alterations of pancreatic
ductal adenocarcinoma and its precursors. Moreover, we review a recently
updated two-tier classification system for precursor lesions, new
findings in premalignant cystic neoplasms, and recently updated staging
criteria for invasive carcinoma based on the Cancer Staging Manual,
eighth edition, from the American Joint Committee on Cancer. Finally, we
discuss the potential clinical applications of the rapidly growing
molecular and genetic information of pancreatic cancer and its
precursors.

{}

{}

\begin{center}\rule{0.5\linewidth}{\linethickness}\end{center}

 - \textbf{Overexpression of folate receptor alpha is an independent
prognostic factor for outcomes of pancreatic cancer patients}

\emph{Medical molecular morphology 2018 Jun;():}

PubMed: \url{https://www.ncbi.nlm.nih.gov/pubmed/?term=29926190}

Pancreatic cancer has a poor prognosis; hence, novel prognostic markers
and effective therapeutic targets should be identified. We aimed to
evaluate folate receptor alpha (FR-α) expression in pancreatic cancer
and examine its association with clinicopathological features. We
utilized tissue samples from 100 primary pancreatic cancer patients who
underwent surgery. FR-α was expressed in 37 of 100 cases (37\%). The
FR-α-positive group (median, 18.8 months) had a significantly poorer
prognosis than the FR-α-negative group {[}median 21.3 months; HR 1.89
(1.12-3.12); P = 0.017{]}. These groups were not significantly different
regarding progression-free survival (P = 0.196). Furthermore, other
serum tumor markers including CA19-9 (mean, 186 vs.~822~U/ml;
P = 0.001), Dupan-2 (286 vs.~1133~U/ml; P = 0.000), and Span-1 (69.7
vs.~171.9~U/ml; P = 0.006) were significantly downregulated in the
FR-α-positive group. CA19-9 was another prognostic factor, in addition
to FR-α, and patient prognosis showed clear stratification curves with
the expression of these two molecules. Along with CA19-9, FR-α
expression was an independent prognostic factor for the overall
survival. FR-α and CA19-9 helped predict patient prognosis based on
stratification curves.

{}

{}

\begin{center}\rule{0.5\linewidth}{\linethickness}\end{center}

Back to top

\begin{center}\rule{0.5\linewidth}{\linethickness}\end{center}

\pagebreak

\hypertarget{staging}{%
\paragraph{Staging}\label{staging}}

Pancreas TNM staging, Margins, Survival

\begin{center}\rule{0.5\linewidth}{\linethickness}\end{center}

 - \textbf{Application of the Eighth Edition of the American Joint
Committee on Cancer Staging for Pancreatic Adenocarcinoma}

\emph{Pancreas 2018 07;47(6):742-747}

PubMed: \url{https://www.ncbi.nlm.nih.gov/pubmed/?term=29851752}

OBJECTIVES: Notable modifications have been made in the American Joint
Committee on Cancer (AJCC) Staging eighth edition staging for pancreatic
cancer for the consideration of the irreproducible and inapplicable of
the AJCC seventh edition staging. However, the new staging
classification has not been systemically verified. METHODS: A comparison
was performed to evaluate the application of the AJCC seventh and eighth
staging classifications using the Surveillance, Epidemiology, and End
Results registry (18,450 patients) and an institutional series (2040
patients). RESULTS: For the eighth staging classification, patients with
tumor diameter of greater than 4 cm (T3N0M0, IIA) had similar prognosis
to patients with 1 to 3 positive nodes (T1-3N1M0, IIB). For patients who
underwent tumor resection and without lymph node involvement, survival
curves of T1 (≤2 cm), T2 (2-4 cm), and T3 (\textgreater{}4 cm) were well
separated. Statistical difference in survival analyses was demonstrated
in N0 (0 positive node), N1 (1-3 positive nodes), and N2 (≥4 positive
nodes) patients underwent tumor resection. The AJCC eighth edition had
better stage distribution than the AJCC seventh edition for pancreatic
cancer. CONCLUSIONS: The eighth edition of AJCC staging is more
applicable and accurate than the seventh edition for pancreatic
adenocarcinoma.

{}

{}

\begin{center}\rule{0.5\linewidth}{\linethickness}\end{center}

 - \textbf{Prognostic significance of the degree of lymphatic vessel
invasion in locally advanced, surgically resectable pancreatic head
cancer: A single center experience}

\emph{Medicine 2018 Dec;97(49):e13466}

PubMed: \url{https://www.ncbi.nlm.nih.gov/pubmed/?term=30544433}

Little is known concerning the prognostic significance of the degree of
lymphatic vessel invasion in pancreatic head cancer. To address this gap
in knowledge, we retrospectively examined 60 patients with locally
advanced, surgically resectable pancreatic head cancer who underwent
pancreaticoduodenectomy and lymph node (LN) dissection.All cases were
histopathologically diagnosed as ductal adenocarcinoma, stage II (25
pT3N0 cases, 35 pT3N1 cases). The following variables were investigated:
age; sex; neoadjuvant therapy; adjuvant therapy; tumor size; tumor
grade; invasion into the serosa, retropancreatic tissue, duodenum, bile
duct, portal venous system and perineural area; cut margins; LN
metastasis; and the number of invaded lymphatic vessels
(LVI-score).Univariate analysis demonstrated that LN metastasis and an
LVI-score ≥5 were significantly associated with poor disease-free
survival. Multivariate Cox regression analysis confirmed that LN
metastasis and an LVI-score ≥7 were significantly associated with poor
disease-free survival. Additionally, LVI-scores ≥9 and ≥10 were
comparable to or surpassed the significance of LN metastasis based on
the hazard ratio. Univariate analysis demonstrated that tumor size
\textgreater{}30 mm, duodenal invasion, LN metastasis and an LVI-score
≥2 were significantly associated with poor overall survival.
Multivariate Cox regression analysis confirmed that LN metastasis and
LVI-scores ≥9 and ≥10 were significantly associated with poor overall
survival, and an LVI-score ≥10 was comparable to or surpassed the
significance of LN metastasis based on the hazard ratio.Our study
strongly suggests that a high degree of lymphatic vessel invasion is
associated with a poor prognosis in patients with locally advanced,
surgically resectable pancreatic head cancer.

{}

{}

\begin{center}\rule{0.5\linewidth}{\linethickness}\end{center}

 - \textbf{Multinational validation of the American Joint Committee on
Cancer 8th edition pancreatic cancer staging system in a pancreas head
cancer cohort}

\emph{Journal of hepato-biliary-pancreatic sciences 2018
Sep;25(9):418-427}

PubMed: \url{https://www.ncbi.nlm.nih.gov/pubmed/?term=30118171}

BACKGROUND: The aim of the present study was to compare the 7th and 8th
editions of the American Joint Committee on Cancer (AJCC) staging system
for pancreas head cancer and to validate the 8th edition using three
multinational tertiary center data. METHODS: Data of 2,864 patients with
pancreas head cancer were collected from Korea (571), Japan (824), and
the USA (1,469). Survival analysis was performed to compare the 7th and
8th editions. Validation was performed by log-rank tests and test for
trend repeated 1,000 times with random sets. RESULTS: In the 7th
edition, 4.1\%, 3.1\%, 18.6\%, 67.5\%, 3.6\%, and 3.1\% were stage IA,
IB, IIA, IIB, III, and IV. In the 8th edition, 8.8\%, 13.9\%, 3.1\%,
38.2\%, 32.9\%, and 3.1\% were stage IA, IB, IIA, IIB, III, and IV,
respectively. The change in T category downstaged 459 patients from IIA
to the new IA and IB. The new N2 category upstaged 856 patients from the
former IIB to III. The 7th edition reversely stratified IA and IB. The
8th edition corrected this mis-stratification of the 7th edition, but
lacked discriminatory power between IB and IIA (P = 0.271). Validation
using the log-rank showed that the 8th edition provided better
discrimination in 6.387 test sets among 10 tests. The test for trend
validated the 8th edition to stratify stages in correct order more often
(7.815/10). CONCLUSION: The 8th edition provides more even distribution
with more powerful discrimination compared to the 7th edition.

{}

{}

\begin{center}\rule{0.5\linewidth}{\linethickness}\end{center}

 - \textbf{Pancreatic Cancer Lymph Node Resection Revisited: A Novel
Calculation of the Number of Lymph Nodes Required}

\emph{Journal of the American College of Surgeons 2019 Jan;():}

PubMed: \url{https://www.ncbi.nlm.nih.gov/pubmed/?term=30677528}

BACKGROUND: Pancreatic cancer is the third leading cause of cancer
related deaths in the U.S. Though lymph node metastasis is a prognostic
indicator, the extent of lymph node (LN) resection is still debated. Our
goal was to use the distribution of the ratio of positive to negative
lymph nodes to derive a more adequate number of necessary examined lymph
nodes based on the target lymph node threshold (TLNT). STUDY DESIGN:
Using the National Cancer Database (NCDB), we performed a retrospective
study of surgically resected pancreatic adenocarcinoma (2010-2015). We
evaluated the number of positive LN (PNL) and total lymph nodes (LNE)
examined, and the log of the ratio of PLNs to negative LNs (LODDS). The
distribution of LODDS was examined in order to determine a target LN
examined threshold (TLNT) sufficient to detect N1 disease. Using the
LODDS distribution of N1 cases, TLNT were calculated to encompass 90 of
the N1 group distribution. RESULTS: Of the total 24,038 resected
patients included in this study, 26\% underwent surgery only, 18\%
received neoadjuvant therapy and 56\% underwent adjuvant therapy. 8,144
(34\%) of patients had N0 disease while 15,894(66\%) had N1 disease. In
order to capture 90-95\% of the N1 group, the minimum number of LN
examined would be 18 (LODDS -2.74) to 24 (LODDS -3.04) respectively.
CONCLUSIONS: Though previous studies have suggested 11-17 LNs required
for adequate LN sampling in pancreatic cancer, our findings suggest that
in order to capture 90\% of cases with N1 disease, 18 LN is more
appropriate.

{}

{}

\begin{center}\rule{0.5\linewidth}{\linethickness}\end{center}

 - \textbf{Determining the optimal number of examined lymph nodes for
accurate staging of pancreatic cancer: An analysis using the nodal
staging score model}

\emph{European journal of surgical oncology : the journal of the
European Society of Surgical Oncology and the British Association of
Surgical Oncology 2019 Jan;():}

PubMed: \url{https://www.ncbi.nlm.nih.gov/pubmed/?term=30685327}

INTRODUCTION: The aim of this study was to determine the optimal number
of examined lymph nodes (ELNs) for accurate staging of pancreatic cancer
using the nodal staging score model. MATERIALS AND METHODS:
Clinicopathological data for patients with resected pancreatic cancer
were collected from SEER database (development cohort {[}DC{]}) and
Fudan University Shanghai Cancer Center database (validation cohort
{[}VC{]}). Multivariable models were constructed to assess how the
number of ELNs was associated with stage migration and overall survival
(OS). Using the β-binomial distribution, we developed a nodal staging
score model from the DC and tested it with the VC. RESULTS: Both cohorts
exhibited significant proportional increases from node-negative to
node-positive disease (DC: odds ratio {[}OR{]}, 1.047;
P\,\textless{}\,0.001; VC: OR, 1.035; P\,\textless{}\,0.001) and
improved OS (DC: hazard ratio {[}HR{]}, 0.982; P\,\textless{}\,0.001;
VC: HR, 0.979; P\,\textless{}\,0.001) as ELNs increased. Nodal staging
scores escalated separately as ELNs increased for different tumor (T)
stages, with plateaus at 16, 21, and 23 LNs (cut-offs) for T1, T2, and
T3 tumors, respectively. Multivariable analysis indicated that examining
more LNs than the corresponding cut-off value was a significant survival
predictor (DC: HR, 0.813; P\,\textless{}\,0.001; VC: HR, 0.696;
P\,=\,0.028). CONCLUSION: The optimal number of ELNs for adequate
staging of pancreatic cancer was related to T stage. We recommend
examining at least 16, 21, and 23 LNs for T1, T2, and T3 tumors,
respectively, as a nodal staging quality measure for both surgery and
pathological analysis.

{}

{}

\begin{center}\rule{0.5\linewidth}{\linethickness}\end{center}

Back to top

\begin{center}\rule{0.5\linewidth}{\linethickness}\end{center}

\pagebreak

\hypertarget{preneoplastic-and-preinvasive-lesions}{%
\paragraph{Preneoplastic and Preinvasive
Lesions}\label{preneoplastic-and-preinvasive-lesions}}

Preneoplastic and Preinvasive Lesions, PanIN, IPMN, MCN, ICPN

\begin{center}\rule{0.5\linewidth}{\linethickness}\end{center}

 - \textbf{New Model for Predicting Malignancy in Patients With
Intraductal Papillary Mucinous Neoplasm}

\emph{Annals of surgery 2018 Nov;():}

PubMed: \url{https://www.ncbi.nlm.nih.gov/pubmed/?term=30499803}

OBJECTIVE: To create a simple, objective model to predict the presence
of malignancy in patients with intraductal papillary mucinous neoplasm
(IPMN), which can be easily applied in daily practice and, importantly,
adopted for any lesion types. BACKGROUND: No predictive model for
malignant IPMN has been widely applied in clinical practice. METHODS:
The clinical details of 466 patients with IPMN who underwent pancreatic
resection at 3 hospitals were retrospectively analyzed for model
development. Then, the model was validated in 664 surgically resected
patients at 8 hospitals in Japan.In the preoperative examination,
endoscopic ultrasonography (EUS) was considered to be essential to
observe mural nodules in both the model development and external
validation sets. Malignant IPMNs were defined as those with high-grade
dysplasia and associated invasive carcinoma. RESULTS: Of the 466
patients, 258 (55\%) had malignant IPMNs (158 high-grade dysplasia, 100
invasive carcinoma), and 208 (45\%) had benign IPMNs. Logistic
regression analysis resulted in 3 variables (mural nodule size, main
pancreatic duct diameter, and cyst size) being selected to construct the
model. The area under the receiver operating characteristic curve (AUC)
for the model was 0.763. In external validation sets, the pathological
diagnosis was malignant and benign IPMN in 351 (53\%) and 313 (47\%)
cases, respectively. For the external validation, the malignancy
prediction ability of the model corresponded to an AUC of 0.725.
CONCLUSION: This predictive model provides important information for
physicians and patients in assessing an individual's risk for malignancy
and may help to identify patients who need surgery.

{}

{}

\begin{center}\rule{0.5\linewidth}{\linethickness}\end{center}

 - \textbf{Subtyping of intraductal papillary mucinous neoplasms -
pitfalls of MUC1 immunohistochemistry}

\emph{APMIS : acta pathologica, microbiologica, et immunologica
Scandinavica 2019 Jan;127(1):27-32}

PubMed: \url{https://www.ncbi.nlm.nih.gov/pubmed/?term=30549137}

Intraductal papillary mucinous neoplasms (IPMNs) are precursor lesions
of pancreatic ductal adenocarcinoma (PDAC). Current edition of WHO
Classification of Tumors of the Digestive System recognizes four
different subtypes (gastric, intestinal, pancreatobiliary, and
oncocytic) and recommends analysis of mucin expression (MUC1, MUC2,
MUC5AC, MUC6) as well as evaluation of architectural and cell
differentiation patterns for correct classification. However, there is
no consensus on MUC1 expression of IPMN-lesions in the literature.
Current recommendations are based on studies where antibodies against
the core MUC1 protein or sialylated MUC1 (tumor associated MUC1), not
the fully glycosylated MUC1 were used. We have recently reported that
MUC1 is strongly expressed in both gastric and intestinal types IPMN
specimens from the cystic wall, obtained by endoscopic ultrasound guided
microbiopsy procedure. We have used a commercial MUC1 antibody,
validated and recommended for diagnostic use, which recognizes fully
glycosylated MUC1. Based on the above, we propose a revision of the WHO
Classification, specifying that antibodies against tumor associated MUC1
should be used for IPMN subtyping.

{}

{}

\begin{center}\rule{0.5\linewidth}{\linethickness}\end{center}

 - \textbf{Should we regard all main duct type intraductal papillary
mucinous neoplasms of the pancreas (MD-IPMN) as an indication of
surgery? -A retrospective study in 29 patients with MD-IPMN showing
mural nodules}

\emph{Pancreatology : official journal of the International Association
of Pancreatology (IAP) \ldots{} {[}et al.{]} 2019 Jan;():}

PubMed: \url{https://www.ncbi.nlm.nih.gov/pubmed/?term=30679137}

PURPOSE: To elucidate predictive factors for malignant main duct type
IPMN (MD-IPMN). METHODS: All 29 subjects had mural nodules (MNs) in the
main pancreatic duct (MPD) on preoperative endoscopic ultrasonography
and underwent surgery (19, malignant; 10, benign). Possible predictive
factors for malignancy such as background, imaging, and histological
factors including histological subtype (HS), were evaluated. RESULTS:
Multivariate analysis revealed an MPD diameter of ≥12\,mm (p\,=\,0.042)
and non-gastric type (p\,=\,0.001) to be the statistically significant
predictive factors for malignancy. The sensitivity, specificity,
positive predictive value, negative predictive value, and accuracy to
detect malignancy by using ``an MPD diameter of ≥12\,mm and/or
non-gastric type'' were 95\%, 70\%, 86\%, 88\%, and 86\%, respectively.
In 7 subjects in whom HS was preoperatively evaluated using pancreatic
specimens obtained before surgery, the agreement rate of the
preoperative HS with definitive HS evaluated using resected specimens
was 86\%. CONCLUSIONS: For MD-IPMNs with MNs, ``an MPD diameter of
≥12\,mm and/or non-gastric type'' are indicated for surgery. On the
other hand, careful surveillance without immediate pancreatic surgery
may be an option for MD-IPMNs showing both an MPD diameter of
\textless{}12\,mm and gastric type.

{}

{}

\begin{center}\rule{0.5\linewidth}{\linethickness}\end{center}

Back to top

\begin{center}\rule{0.5\linewidth}{\linethickness}\end{center}

\pagebreak

\hypertarget{microenvironment}{%
\paragraph{Microenvironment}\label{microenvironment}}

Tumor Stroma Interactions, Microenvironment, Inflammatory Response

\begin{center}\rule{0.5\linewidth}{\linethickness}\end{center}

 - \textbf{Quasimesenchymal phenotype predicts systemic metastasis in
pancreatic ductal adenocarcinoma}

\emph{Modern pathology : an official journal of the United States and
Canadian Academy of Pathology, Inc 2019 Jan;():}

PubMed: \url{https://www.ncbi.nlm.nih.gov/pubmed/?term=30683911}

Metastasis following surgical resection is a leading cause of mortality
in pancreatic ductal adenocarcinoma. Epithelial-mesenchymal transition
is thought to play an important role in metastasis, although its
clinical relevance in metastasis remains uncertain. We evaluated a panel
of RNA in-situ hybridization probes for epithelial-mesenchymal
transition-related genes expressed in circulating tumor cells. We
assessed the predictive value of this panel for metastasis in pancreatic
ductal adenocarcinoma and, to determine if the phenotype is
generalizable between cancers, in colonic adenocarcinoma. One hundred
fifty-eight pancreatic ductal adenocarcinomas and 205 colonic
adenocarcinomas were classified as epithelial or quasimesenchymal
phenotype using dual colorimetric RNA-in-situ hybridization. SMAD4
expression on pancreatic ductal adenocarcinomas was assessed by
immunohistochemistry. Pancreatic ductal adenocarcinomas with
quasimesenchymal phenotype had a significantly shorter disease-specific
survival (P = 0.031) and metastasis-free survival (P = 0.0001) than
those with an epithelial phenotype. Pancreatic ductal adenocarcinomas
with SMAD4 loss also had lower disease-specific survival (P = 0.041) and
metastasis-free survival (P = 0.001) than those with intact SMAD4.
However, the quasimesenchymal phenotype proved a more robust predictor
of metastases-area under the curve for quasimesenchymal = 0.8;
SMAD4 = 0.6. The quasimesenchymal phenotype also predicted
metastasis-free survival (P = 0.004) in colonic adenocarcinoma.
Epithelial-mesenchymal transition defined two phenotypes with distinct
metastatic capabilities-epithelial phenotype tumors with predominantly
organ-confined disease and quasimesenchymal phenotype with high risk of
metastatic disease in two epithelial malignancies. Collectively, this
work validates the relevance of epithelial-mesenchymal transition in
human gastrointestinal tumors.

{}

{}

\begin{center}\rule{0.5\linewidth}{\linethickness}\end{center}

 - \textbf{The Paradoxical Web of Pancreatic Cancer Tumor
Microenvironment}

\emph{The American journal of pathology 2019 Jan;189(1):44-57}

PubMed: \url{https://www.ncbi.nlm.nih.gov/pubmed/?term=30558722}

Pancreatic ductal adenocarcinoma (PDAC) is increasing in incidence and
is projected to become the second leading cause of cancer death in the
United States. Despite significant advances in understanding the
disease, there has been minimal increase in PDAC patient survival. PDAC
tumors are unique in the fact that there is significant desmoplasia.
This generates a large stromal compartment composed of immune cells,
inflammatory cells, growth factors, extracellular matrix, and
fibroblasts, comprising the tumor microenvironment (TME), which may
represent anywhere from 15\% to 85\% of the tumor. It~has become evident
that the TME, including both the stroma and extracellular component,
plays an important role in tumor progression and chemoresistance of
PDAC. This review will discuss the multiple components of the TME, their
specific impact on tumorigenesis, and the multiple therapeutic targets.

{}

{}

\begin{center}\rule{0.5\linewidth}{\linethickness}\end{center}

 - \textbf{Immune Cell and Stromal Signature Associated With
Progression-Free Survival of Patients With Resected Pancreatic Ductal
Adenocarcinoma}

\emph{Gastroenterology 2018 11;155(5):1625-1639.e2}

PubMed: \url{https://www.ncbi.nlm.nih.gov/pubmed/?term=30092175}

BACKGROUND \& AIMS: Changes to the microenvironment of~pancreatic ductal
adenocarcinomas (PDACs) have been associated with poor outcomes of
patients. We studied the associations between composition of the
pancreatic stroma (fibrogenic, inert, dormant, or fibrolytic stroma) and
infiltration by inflammatory cells and times of progression-free
survival (PFS) of patients with PDACs after resection. METHODS: We
obtained 1824 tissue microarray specimens from 385 patients included in
the European Study Group for Pancreatic Cancer trial 1 and 3 and
performed immunohistochemistry to detect alpha smooth muscle actin, type
1 collagen, CD3, CD4, CD8, CD68, CD206, and neutrophils. Tumors that
expressed high and~low levels of these markers were compared with
patient outcomes using Kaplan-Meier curves and multivariable recursive
partitioning for discrete-time survival tree analysis. Prognostic index
was delineated by a multivariable Cox proportional hazards model of
immune cell and stromal markers and PFS. Findings were validated using
279 tissue microarray specimens from 93 patients in a separate cohort.
RESULTS: Levels of CD3, CD4, CD8, CD68, and CD206 were independently
associated with tumor recurrence. Recursive partitioning for
discrete-time survival tree analysis identified a high level of CD3 as
the strongest independent predictor for longer PFS. Tumors with levels
of CD3 and high levels of CD206 associated with a median PFS time of
16.6 months and a median prognostic index of -0.32 (95\% confidence
interval {[}CI{]} -0.35 to -0.31), whereas tumors with low level of CD3
cell and low level of CD8 and high level of CD68 associated with a
median PFS time of 7.9 months and a prognostic index of 0.32 (95\% CI
0.050-0.32); we called these patterns histologic signatures. Stroma
composition, when unassociated with inflammatory cell~markers, did not
associate significantly with PFS. In the~validation cohort, the
histologic signature resulted in an error matrix accuracy of predicted
response of 0.75 (95\% CI 0.64-0.83; accuracy P \textless{} .001).
CONCLUSIONS: In an analysis of PDAC tissue microarray specimens, we
identified and validated a histologic signature, based on leukocyte and
stromal factors, that associates with PFS times of patients with
resected PDACs. Immune cells might affect the composition of the
pancreatic stroma to affect progression of PDAC. These findings provide
new insights into the immune response to PDAC.

{}

{}

\begin{center}\rule{0.5\linewidth}{\linethickness}\end{center}

 - \textbf{Inter- and intra-tumoral heterogeneity in cancer-associated
fibroblasts of human pancreatic ductal adenocarcinoma}

\emph{The Journal of pathology 2018 Dec;():}

PubMed: \url{https://www.ncbi.nlm.nih.gov/pubmed/?term=30575030}

Cancer-associated fibroblasts (CAF) are orchestrators of the pancreatic
ductal adenocarcinoma (PDAC) microenvironment. Stromal heterogeneity may
explain differential pathophysiological roles of the stroma (pro-
vs.~anti-tumoral) in PDAC. We hypothesized that multiple CAF functional
subtypes exist in PDAC, that contribute to stromal heterogeneity through
interactions with cancer cells. Using molecular and functional analysis
of patient-derived CAF primary cultures, we demonstrated that human
PDAC-derived CAFs display a high level of inter- and intra-tumour
heterogeneity. We identified at least four subtypes of CAFs based on
transcriptomic analysis, and propose a classification for human
PDAC-derived CAFs (pCAFassigner). Multiple CAF subtypes co-existed in
individual patient samples. The presence of these CAF subtypes in bulk
tumours was confirmed using publicly available gene expression profiles,
and immunostainings of CAF subtype markers. Each subtype displayed
specific phenotypic features (matrix- and immune-related signatures,
vimentin and α-smooth muscle actin expression, proliferation rate), and
was associated with an assessable prognostic impact. A prolonged
exposure of non-tumoral pancreatic stellate cells to conditioned media
from cancer cell lines (cancer education experiment) induced a CAF-like
phenotype, including loss of capacity to revert to quiescence and an
increase in the expression of genes related to CAF subtypes B and C.
This classification demonstrates molecular and functional inter- and
intra-tumoral heterogeneity of CAFs in human PDAC. Our subtypes overlap
with those identified from single-cell analyses in other cancers, and
pave the way for the development of therapies targeting specific CAF
sub-populations in PDAC. This article is protected by copyright. All
rights reserved.

{}

{}

\begin{center}\rule{0.5\linewidth}{\linethickness}\end{center}

 - \textbf{Epithelial-Stromal Interactions in Pancreatic Cancer}

\emph{Annual review of physiology 2018 Nov;():}

PubMed: \url{https://www.ncbi.nlm.nih.gov/pubmed/?term=30418798}

Pancreatic cancer is characterized by an extensive fibroinflammatory
reaction that includes immune cells, fibroblasts, extracellular matrix,
vascular and lymphatic vessels, and nerves. Overwhelming evidence
indicates that the pancreatic cancer microenvironment regulates cancer
initiation, progression, and maintenance. Pancreatic cancer treatment
has progressed little over the past several decades, and the prognosis
remains one of the worst for any cancer. The contribution of the
microenvironment to carcinogenesis is a key area of research, offering
new potential targets for treating the disease. Here, we explore the
composition of the pancreatic cancer stroma, discuss the network of
interactions between different components, and describe recent attempts
to target the stroma therapeutically. We also discuss current areas of
active research related to the tumor microenvironment. Expected final
online publication date for the Annual Review of Physiology Volume 81 is
February 10, 2019. Please see
\url{http://www.annualreviews.org/page/journal/pubdates} for revised
estimates.

{}

{}

\begin{center}\rule{0.5\linewidth}{\linethickness}\end{center}

Back to top

\begin{center}\rule{0.5\linewidth}{\linethickness}\end{center}

\pagebreak

\hypertarget{spn}{%
\paragraph{SPN}\label{spn}}

Solid Pseudopapillary Neoplasm

\begin{center}\rule{0.5\linewidth}{\linethickness}\end{center}

 - \textbf{Prognostic value of progesterone receptor in solid
pseudopapillary neoplasm of the pancreas: evaluation of a pooled case
series}

\emph{BMC gastroenterology 2018 Dec;18(1):187}

PubMed: \url{https://www.ncbi.nlm.nih.gov/pubmed/?term=30547767}

BACKGROUND: The role of progesterone receptor (PR) has been reported in
a series of pancreatic cysts. However, the relationship between PR and
prognosis of solid pseudopapillary neoplasm of the pancreas (SPNP) has
not been elucidated so far. The aim of our study was to evaluate the
prognostic value of PR in SPNP. METHODS: A total of 76 patients with
SPNP treated in our institution from January 2012 to December 2017 were
included. Demographic parameters, laboratory data, pathologic
information and clinical outcomes were analyzed by the use of survival
analysis. In addition, a pooled case series was performed to evaluate
the results. RESULTS: The institutional data included 76 patients (17
male and 59 female) ranging from 8 to 90 years (median, 30 years) in
age. Kaplan-Meier survival analysis confirmed negative PR result was
significantly associated with poorer disease-free survival (DFS) and
disease-specific survival (DSS) (both P \textless{} 0.001). In the
pooled analysis, a total of 62 studies comprising 214 patients with SPNP
were included. After multivariable cox analysis, negative PR result
remained an independent prognostic factor for SPNP (DFS HR: 14.50, 95\%
CI: 1.98-106.05, P = 0.008; DSS HR: 9.15, 95\% CI: 1.89-44.17,
P = 0.006). CONCLUSION: Our results indicated the role of PR in
predicting adverse outcome of patients with SPNP and negative PR result
may serve as a potential prognostic factor.

{}

{}

\begin{center}\rule{0.5\linewidth}{\linethickness}\end{center}

 - \textbf{CD200 expression is a feature of solid pseudopapillary
neoplasms of the pancreas}

\emph{Virchows Archiv : an international journal of pathology 2019
Jan;474(1):105-109}

PubMed: \url{https://www.ncbi.nlm.nih.gov/pubmed/?term=30132130}

CD200 has been recently indicated as a robust marker of
well-differentiated neuroendocrine neoplasms. Here, we evaluate its role
in differential diagnosis of solid pancreatic neoplasms. We
immunostained for CD200 22 solid pseudopapillary neoplasms (SPNs), 8
acinar carcinomas (ACs), 2 pancreatoblastomas (PBs), 138 neuroendocrine
tumors (PanNETs), and 48 ductal adenocarcinomas. All SPNs showed strong
cytoplasmic and membranous staining for CD200, while only one case of AC
had focal positivity. The two PBs showed focal CD200 positivity, mainly
located in squamoid nests. The vast majority of PanNETs (96\%) showed
strong cytoplasmic and membranous staining for CD200, whereas all PDACs
were negative. As both PanNETs and SPNs express CD200, it has no role in
the differential diagnosis between these two entities.

{}

{}

\begin{center}\rule{0.5\linewidth}{\linethickness}\end{center}

Back to top

\begin{center}\rule{0.5\linewidth}{\linethickness}\end{center}

\pagebreak

\hypertarget{analogue-tumors-lesions}{%
\paragraph{Analogue Tumors \& Lesions}\label{analogue-tumors-lesions}}

\begin{center}\rule{0.5\linewidth}{\linethickness}\end{center}

 - \textbf{Pancreatic Acinar Metaplasia in Distal Esophageal Biopsies Is
Associated With Chronic Nonsteroidal Anti-inflammatory Drug Use}

\emph{Archives of pathology \& laboratory medicine 2018 Dec;():}

PubMed: \url{https://www.ncbi.nlm.nih.gov/pubmed/?term=30525934}

CONTEXT.---: The cause of pancreatic acinar metaplasia (PAM) at the
distal esophagus/esophagogastric junction is still controversial.
Whereas some authors believe it is congenital, others believe it is
acquired because of inflammation of the gastric cardia, and more
recently it was proposed to be due to chronic proton pump inhibitor use
based on a study in rats. OBJECTIVE.---: To determine whether there is
correlation between chronic proton pump inhibitor use and PAM in humans.
We also investigated the correlation between several clinical and
pathologic factors and PAM. DESIGN.---: Four hundred forty-four
consecutive biopsies from the distal esophagus/esophagogastric junction
were reviewed for the presence of PAM, which was then correlated with
several clinical and pathologic findings. RESULTS.---: Pancreatic acinar
metaplasia was found in 71 patients (16\%). Pancreatic acinar metaplasia
was significantly associated with patient age younger than 51 years ( P
\textless{} .001), chronic carditis ( P = .01), and chronic proton pump
inhibitor use ( P = .008). Surprisingly, we also found significant
association between PAM and chronic nonsteroidal anti-inflammatory drug
use ( P \textless{} .001). These associations, including that with
chronic nonsteroidal anti-inflammatory drug use, remained significant in
multivariate analysis. CONCLUSIONS.---: Our findings confirm the
previous reports of significant association between PAM and chronic
carditis and the findings from animal studies of association with
chronic proton pump inhibitor use. The strong association with chronic
nonsteroidal anti-inflammatory drug use has not been previously reported
and warrants further studies.

{}

{}

\begin{center}\rule{0.5\linewidth}{\linethickness}\end{center}

Back to top

\begin{center}\rule{0.5\linewidth}{\linethickness}\end{center}

\pagebreak

\hypertarget{pancreatitis-other-diseases}{%
\paragraph{Pancreatitis \& Other
Diseases}\label{pancreatitis-other-diseases}}

\begin{center}\rule{0.5\linewidth}{\linethickness}\end{center}

 - \textbf{Tumefactive Inflammatory Diseases of the Pancreas}

\emph{The American journal of pathology 2019 Jan;189(1):82-93}

PubMed: \url{https://www.ncbi.nlm.nih.gov/pubmed/?term=30558726}

Advances in the past two decades have resulted in the recognition of
several tumefactive pancreatic lesions that, on histologic evaluation,
show a varying combination of inflammation and fibrosis. Autoimmune
pancreatitis, the prototypic tumefactive pancreatic fibroinflammatory
lesion, is composed of two distinct diseases, type 1 autoimmune
pancreatitis and the less common type 2 autoimmune pancreatitis.
Although designated as autoimmune pancreatitis, the two diseases show
little morphologic or pathogenic overlap. In type 1 disease, subsets of
T lymphocytes (type 2 helper T cells, regulatory T cells, and T
follicular helper 2 cells) are hypothesized to drive the inflammatory
reaction. The B-cell response is characterized by an oligoclonal
expansion of plasmablasts, with dominant clones that vary among patients
and distinct clones that emerge at the time of relapse. Although the
precise role of IgG4 in this condition remains uncertain, recent studies
suggest that other IgG subclasses (eg, IgG1) may mediate the immune
reactions, whereas IgG4 represents a response to dampen excessive
inflammation. A recent study of type 2 autoimmune pancreatitis
highlights the role of CXCL8 (alias IL-8), with duct epithelium and
infiltrating T lymphocytes expressing this chemokine; the latter may
contribute to the distinct form of neutrophilic inflammation in this
disease. The review also highlights other forms of mass-forming chronic
pancreatitis: follicular pancreatitis, groove pancreatitis, and those
associated with rheumatologic diseases.

{}

{}

\begin{center}\rule{0.5\linewidth}{\linethickness}\end{center}

 - \textbf{Clinicopathological and immunological features of follicular
pancreatitis-a distinct disease entity characterized by Th17 activation}

\emph{Histopathology 2018 Dec;():}

PubMed: \url{https://www.ncbi.nlm.nih.gov/pubmed/?term=30515871}

AIM: Follicular pancreatitis is a recently recognized, distinct
clinicopathological entity characterized by the presence of many
intrapancreatic lymphoid follicles with reactive germinal centres.
However, the clinicopathological and immunological features and causes
have not yet been established. We assessed the clinicopathological and
immunological profiles of patients with follicular pancreatitis who
underwent surgery. METHODS AND RESULTS: This study included three
patients with pancreatic masses (age range: 62-75 years; women:men:
1:2). A histopathological study of the resected pancreatic masses
revealed abundant lymphoid follicles with reactive germinal centres in
both periductal regions and diffusely within the parenchyma. No
storiform fibrosis, obliterative phlebitis, or granulocytic epithelial
lesions were observed. The immunohistochemical examination revealed an
IgG4/IgG-positive plasma cell ratio \textless{}30\% in all patients.
Podoplanin (Th17 marker)-expressing lymphocytes were present in the
lymphoid follicles of those with follicular pancreatitis, whereas these
were absent in normal lymph nodes and in lymphoid follicles of those
with IgG4-related autoimmune pancreatitis (AIP). An RNA digital counting
assay clearly demonstrated that the expression counts of 20 genes,
including dendritic cells and lymphoid follicles markers, and related
cytokines were significantly higher in follicular pancreatitis than in
IgG4-related AIP (p\textless{}0.01). The expressions of CCR6 and IL23A,
which are genes related to Th17, were high. CONCLUSIONS: This study
shows that follicular pancreatitis is a histopathologically and
immunologically distinct disease entity of pancreatitis and is
characterized by upregulated Th17 expression. This article is protected
by copyright. All rights reserved.

{}

{}

\begin{center}\rule{0.5\linewidth}{\linethickness}\end{center}

Back to top

\begin{center}\rule{0.5\linewidth}{\linethickness}\end{center}

\pagebreak

\hypertarget{bile-ducts}{%
\subsubsection{Bile Ducts}\label{bile-ducts}}

\begin{center}\rule{0.5\linewidth}{\linethickness}\end{center}

\hypertarget{morphology-diagnostics-ihc-1}{%
\paragraph{Morphology, Diagnostics,
IHC}\label{morphology-diagnostics-ihc-1}}

Morphology, Diagnostics, IHC

\begin{center}\rule{0.5\linewidth}{\linethickness}\end{center}

 - \textbf{Cholangiolocellular Carcinoma With ``Ductal Plate
Malformation'' Pattern may be Characterized by ARID1A Genetic
Alterations}

\emph{The American journal of surgical pathology 2018 Dec;():}

PubMed: \url{https://www.ncbi.nlm.nih.gov/pubmed/?term=30520820}

Cholangiolocellular carcinoma (CLC) is a unique subtype of primary liver
carcinoma, which sometimes coexists with hepatocellular carcinoma (HCC),
cholangiocarcinoma and combined hepatocellular-cholangiocarcinoma
(cHCC-CCA). ``Ductal plate malformation'' (DPM)-pattern of primary liver
carcinoma, which resembles biliary lesions in Caroli disease and von
Meyenburg complex, is sometimes associated with CLC. We examined genetic
alterations of hTERT promoter (hTERT), IDH1 or 2 (IDH1/2), KRAS, ARID1A,
PBRM1, ARID2, BAP1, p53 and their association with histologic features
such as proportion of CLC and DPM-pattern in 77 patients with primary
liver carcinoma diagnosed as cHCC-CCA or CLC. Primary liver carcinomas
were histologically subdivided into 29 CLC-predominant (CLC component
\textgreater{}80\%), 31 with CLC (5\% to 80\%) and 17 without CLC
(\textless{}5\%). CLC-predominant group was characterized by older age,
male-predominant and smaller tumor size. Genetic alterations were
detected in hTERT (25\%), ARID1A (21\%), PBRM1 (20\%), ARID2 (3\%), BAP1
(1\%), p53 (46\%), KRAS (5\%), and IDH1/2 (8\%). ARID1A alteration was
more frequent in CLC-predominant group, compared with other groups
(P\textless{}0.05) and was correlated with the degree of DPM-pattern
(P\textless{}0.01). Alterations of hTERT and p53 were less frequent in
CLC-predominant group compared with ``with CLC group''
(P\textless{}0.05). hTERT mutation was less frequent in carcinomas with
DPM-pattern (P\textless{}0.01). PBRM1 alteration was more frequent in
CLC with focal HCC subgroup and without CLC group compared with other
groups (P\textless{}0.05). CLC may be a distinct subgroup of primary
liver carcinoma, which is different from cHCC-CCA, based on
clinicopathologic and genetic alterations. ARID1A alterations may
characterize CLC with DPM-pattern and could be a diagnostic
immunohistochemical marker for small CLCs with DPM-pattern.

{}

{}

\begin{center}\rule{0.5\linewidth}{\linethickness}\end{center}

 - \textbf{Data set for the reporting of intrahepatic
cholangiocarcinoma, perihilar cholangiocarcinoma and hepatocellular
carcinoma: recommendations from the International Collaboration on
Cancer Reporting (ICCR)}

\emph{Histopathology 2018 Sep;73(3):369-385}

PubMed: \url{https://www.ncbi.nlm.nih.gov/pubmed/?term=29573451}

Optimal patient management benefits from comprehensive and accurate
pathology reports that contribute to cancer staging and prognostication.
Proforma reports are used in many countries, but these vary in their
structure and implementation. The International Collaboration on Cancer
Reporting (ICCR) is an alliance formed by the Royal College of
Pathologists of Australasia, the Royal College of Pathologists of the
United Kingdom, the College of American Pathologists, the Canadian
Partnership Against Cancer the European Society of Pathology and the
American Society of Clinical Pathology (ASCP), with the aim of
developing an evidence-based reporting data set for each cancer site. It
is argued that this should reduce the global burden of cancer data set
development and reduplication of effort by different international
institutions that commission, publish and maintain standardised cancer
reporting data sets. The resultant standardisation of cancer reporting
will benefit not only those countries directly involved in the
collaboration but also others not in a position to develop their own
data sets. We describe the development of a cancer data set by the ICCR
expert panel for the reporting of the main malignant liver tumours:
intrahepatic cholangiocarcinoma, perihilar cholangiocarcinoma and
hepatocellular carcinoma and present the `required' and `recommended'
elements to be included in the report with an explanatory commentary.
This data set incorporates definitions and classifications in the most
recent World Health Organisation (WHO) publication on hepatic
malignancies (4th edition) and the recently published
tumour-node-metastasis (TNM)8 staging system. Widespread adoption and
implementation of this data set will enable consistent and accurate data
collection, comparison of epidemiological and pathological parameters
between different populations, facilitate research and ultimately result
in better patient outcomes.

{}

{}

\begin{center}\rule{0.5\linewidth}{\linethickness}\end{center}

 - \textbf{Clinical features of isolated proximal-type IgG4-related
sclerosing cholangitis}

\emph{Digestive endoscopy : official journal of the Japan
Gastroenterological Endoscopy Society 2018 Dec;():}

PubMed: \url{https://www.ncbi.nlm.nih.gov/pubmed/?term=30570170}

BACKGROUND AND AIMS: IgG4-related sclerosing cholangitis (IgG4-SC)
presents as isolated proximal-type sclerosing cholangitis (i-SC). This
study sought to clarify the imaging differences between i-SC and
Klatskin tumor. The differences between i-SC and IgG4-SC associated with
autoimmune pancreatitis (AIP-SC) were also studied. METHODS:
Differentiating factors between i-SC and Klatskin tumor were studied.
The serum IgG4 level, CA19-9 level, CT findings, cholangiography
findings (symmetrical smooth long stricture extending into upper bile
duct {[}SSLS{]}), endosonographic features (continuous symmetric mucosal
lesion to the hilar part {[}CSML{]}), endoscopic biopsy results,
treatment, relapse, and survival were also compared between patients
with i-SC and those with AIP-SC. RESULTS: For a differential diagnosis
between i-SC (N = 9) and Klatskin tumor (N = 47), the cutoff value of
serum IgG4 level was 150mg/dL (sensitivity=0.857, specificity=0.966).
Logistic regression analysis indicated that serum IgG4 level, the
presence of SSLS, presence of CSML and the presence of ampulla swollen
are independent factor for identifying i-SC. The relapse rate was
significantly higher in the IgG4-SC with AIP group than in the i-SC
group (log rank, p = 0.046). CONCLUSION: i-SC presents as a nodular
lesion with SSLS and/or CSML mimicking a Klatskin tumor. Those
endoscopic features might provide a diagnostic clue for i-SC. i-SC is
likely to have a more favorable prognosis than IgG4-SC with AIP. This
article is protected by copyright. All rights reserved.

{}

{}

\begin{center}\rule{0.5\linewidth}{\linethickness}\end{center}

 - \textbf{Impact of Tumour Budding Grade in 310 Patients Who Underwent
Surgical Resection for Extrahepatic Cholangiocarcinoma}

\emph{Histopathology 2019 Jan;():}

PubMed: \url{https://www.ncbi.nlm.nih.gov/pubmed/?term=30667537}

AIMS: Tumour budding is a risk factor for poor prognosis in various
cancers. Tumour buds may present an epithelial-mesenchymal transition
(EMT) morphological phenotype. This study aimed to elucidate the
prognostic impact of tumour budding grade and its association with
clinicopathological and EMT-related features in perihilar
cholangiocarcinoma (PHCC) or distal cholangiocarcinoma (DCC). METHODS
AND RESULTS: Subjects included 195 PHCC and 115 DCC patients. The
numbers of tumour buds in different patients were stratified for
postoperative survival using the recursive partitioning technique.
Consequently, the numbers of tumour buds in PHCC patients were
classified into three grades, namely, low (0-4 buds); intermediate (5-11
buds); and high (≥12 buds); those of DCC patients were classified into
two grades, namely, low (0-4 buds) and high (≥5 buds). In both PHCC and
DCC patients, high tumour budding grade was associated with poor
histological differentiation, higher pT factor, presence of lymphatic,
venous, perineural invasion, and regional lymph node metastasis. In PHCC
patients, residual invasive tumour in the resected margin was also
associated with high tumour budding grade. For both PHCC and DCC
patients, high tumour budding grade was an independent adverse
prognostic factor in multivariate analysis (p\textless{}0001 and
p=0.046, respectively). Immunohistochemical examination revealed that
the number of tumour buds increased in patients with tumours showing a
mesenchymal profile (negative for E-cadherin and positive for vimentin).
CONCLUSIONS: Higher tumour budding grade is associated with invasive
clinicopathological features, adverse postoperative prognosis, and EMT
status in extrahepatic cholangiocarcinoma. This article is protected by
copyright. All rights reserved.

{}

{}

\begin{center}\rule{0.5\linewidth}{\linethickness}\end{center}

Back to top

\begin{center}\rule{0.5\linewidth}{\linethickness}\end{center}

\pagebreak

\hypertarget{staging-1}{%
\paragraph{Staging}\label{staging-1}}

Bile Ducts TNM staging, Margins, Survival

\begin{center}\rule{0.5\linewidth}{\linethickness}\end{center}

 - \textbf{Proposal for a new classification for perihilar
cholangiocarcinoma based on tumour depth}

\emph{The British journal of surgery 2019 Jan;():}

PubMed: \url{https://www.ncbi.nlm.nih.gov/pubmed/?term=30675908}

BACKGROUND: The T system for distal cholangiocarcinoma has been revised
from a layer-based to a depth-based approach in the current American
Joint Committee on Cancer (AJCC) classification. In perihilar
cholangiocarcinoma, tumour depth in the staging scheme has not yet been
addressed. The aim of this study was to propose a new T system using
measured tumour depth in perihilar cholangiocarcinoma. METHODS: Patients
who underwent hepatectomy for perihilar cholangiocarcinoma between 2001
and 2014 were reviewed retrospectively. The vertical distance between
the top of the tumour and deepest invasive cells was measured as
invasive tumour thickness (ITT) by two independent pathologists. Log
rank statistics were used to determine cut-off points, and the
concordance (C) index was used to assess survival discrimination of each
T system. RESULTS: ITT was measurable in all 440 patients, with a median
value of 6·0 (range 0-45) mm. The median difference in ITT between
observers was 0·6 (range 0-20) mm. Cut-off points for prognosis were 1,
5 and 8 mm. Five-year survival decreased with increasing ITT (P
\textless{} 0·001): 67 per cent for ITT less than 1 mm (25 patients),
54·9 per cent for ITT 1 mm and over to less than 5 mm (138 patients),
43·4 per cent for ITT 5 mm and over to less than 8 mm (118 patients),
and 32·2 per cent for ITT 8 mm and over (159 patients). The C-index of
this classification was comparable to that of the current AJCC T
classification (0·598 versus 0·589). CONCLUSION: ITT is a reliable
approach for making a depth assessment in perihilar cholangiocarcinoma.
A four-tier ITT classification with cut-off points of 1, 5 and 8 mm is
an adequate alternative to the current layer-based T classification.

{}

{}

\begin{center}\rule{0.5\linewidth}{\linethickness}\end{center}

 - \textbf{Validation of the Eighth American Joint Committee on Cancer
Staging System for Distal Bile Duct Carcinoma}

\emph{Cancer research and treatment : official journal of Korean Cancer
Association 2019 Jan;51(1):98-111}

PubMed: \url{https://www.ncbi.nlm.nih.gov/pubmed/?term=29510611}

PURPOSE: T category of the eighth edition of the American Joint
Committee on Cancer (AJCC) staging system for distal bile duct carcinoma
(DBDC) was changed to include tumor invasion depth measurement, while
the N category adopted a 3-tier classification system based on the
number of metastatic nodes. Materials and Methods: To validate cancer
staging, a total of 200 surgically resected DBDCs were staged and
compared according to the seventh and eighth editions. RESULTS: T
categories included T1 (n=37, 18.5\%), T2 (n=114, 57.0\%), and T3 (n=49,
24.5\%). N categories included N0 (n=133, 66.5\%), N1 (n=50, 25.0\%),
and N2 (n=17, 8.5\%). Stage groupings included I (n=33, 16.5\%), II
(n=150, 75.0\%), and III (n=17, 8.5\%). The overall 5-year survival
rates (5-YSRs) of T1, T2, and T3 were 59.3\%, 42.4\%, and 12.2\%,
respectively. T category could discriminate patient survival by both
pairwise (T1 and T2, p=0.011; T2 and T3, p \textless{} 0.001) and
overall (p \textless{} 0.001) comparisons. The overall 5-YSRs of N0, N1,
and N2 were 47.3\%, 17.0\%, and 14.7\%, respectively. N category could
partly discriminate patient survival by both pairwise (N0 and N1, p
\textless{} 0.001; N1 and N2, p=0.579) and overall (p \textless{} 0.001)
comparisons. The overall 5-YSRs of stages I, II, and III were 59.0\%,
35.4\%, and 14.7\%, respectively. Stages could distinguish patient
survival by both pairwise (I and II, p=0.002; II and III, p=0.015) and
overall (p \textless{} 0.001) comparisons. On multivariate analyses, T
and N categories (p=0.014 and p=0.029) and pancreatic invasion (p=0.006)
remained significant prognostic factors. CONCLUSION: The T
andNcategories of the eighth edition AJCC staging system for DBDC
accurately predict patient prognosis.

{}

{}

\begin{center}\rule{0.5\linewidth}{\linethickness}\end{center}

Back to top

\begin{center}\rule{0.5\linewidth}{\linethickness}\end{center}

\pagebreak

\hypertarget{gallbladder}{%
\subsubsection{Gallbladder}\label{gallbladder}}

\begin{center}\rule{0.5\linewidth}{\linethickness}\end{center}

\hypertarget{morphology-diagnostics-ihc-2}{%
\paragraph{Morphology, Diagnostics,
IHC}\label{morphology-diagnostics-ihc-2}}

Morphology, Diagnostics, IHC

\begin{center}\rule{0.5\linewidth}{\linethickness}\end{center}

 - ****

\begin{itemize}
\tightlist
\item
  ;():*
\end{itemize}

PubMed: \url{https://www.ncbi.nlm.nih.gov/pubmed/?term=}

{}

{}

\begin{center}\rule{0.5\linewidth}{\linethickness}\end{center}

 - \textbf{Systematic review of management of incidental gallbladder
cancer after cholecystectomy}

\emph{The British journal of surgery 2019 Jan;106(1):32-45}

PubMed: \url{https://www.ncbi.nlm.nih.gov/pubmed/?term=30582640}

BACKGROUND: Gallbladder cancer is rare, but cancers detected
incidentally after cholecystectomy are increasing. The aim of this study
was to review the available data for current best practice for optimal
management of incidental gallbladder cancer. METHODS: A systematic
PubMed search of the English literature to May 2018 was conducted.
RESULTS: The search identified 12 systematic reviews and meta-analyses,
in addition to several consensus reports, multi-institutional series and
national audits. Some 0·25-0·89 per cent of all cholecystectomy
specimens had incidental gallbladder cancer on pathological examination.
Most patients were staged with pT2 (about half) or pT1 (about one-third)
cancers. Patients with cancers confined to the mucosa (T1a or less) had
5-year survival rates of up to 100 per cent after cholecystectomy alone.
For cancers invading the muscle layer of the gallbladder wall (T1b or
above), reresection is recommended. The type, extent and timing of
reresection remain controversial. Observation time may be used for new
cross-sectional imaging with CT and MRI. Perforation at initial surgery
had a higher risk of disease dissemination. Gallbladder cancers are
PET-avid, and PET may detect residual disease and thus prevent
unnecessary surgery. Routine laparoscopic staging before reresection is
not warranted for all stages. Risk of peritoneal carcinomatosis
increases with each T category. The incidence of port-site metastases is
about 10 per cent. Routine resection of port sites has no effect on
survival. Adjuvant chemotherapy is poorly documented and probably
underused. CONCLUSION: Management of incidental gallbladder cancer
continues to evolve, with more refined suggestions for subgroups at risk
and a selective approach to reresection.

{}

{}

\begin{center}\rule{0.5\linewidth}{\linethickness}\end{center}

Back to top

\begin{center}\rule{0.5\linewidth}{\linethickness}\end{center}

\pagebreak

\hypertarget{staging-2}{%
\paragraph{Staging}\label{staging-2}}

Gallbladder TNM staging, Margins, Survival

\begin{center}\rule{0.5\linewidth}{\linethickness}\end{center}

 - \textbf{Optimal surgical treatment in patients with T1b gallbladder
cancer: An international multicenter study}

\emph{Journal of hepato-biliary-pancreatic sciences 2018
Dec;25(12):533-543}

PubMed: \url{https://www.ncbi.nlm.nih.gov/pubmed/?term=30562839}

BACKGROUND: There is no consensus on the optimal treatment of T1b
gallbladder cancer (GBC) due to the lack of evidence and the difficulty
of anatomy and pathological standardization. METHODS: A total of 272
patients with T1b GBC who underwent surgical resection at 14 centers
with specialized hepatobiliary-pancreatic surgeons and pathologists in
Korea, Japan, Chile, and the United States were studied. Clinical
outcomes including disease-specific survival (DSS) rates according to
the types of surgery were analyzed. RESULTS: After excluding patients,
the 237 qualifying patients consisted of 90 men and 147 women. Simple
cholecystectomy (SC) was performed in 116 patients (48.9\%) and extended
cholecystectomy (EC) in 121 patients (51.1\%). The overall 5-year DSS
was 94.6\%, and it was similar between SC and EC patients (93.7\%
vs.~95.5\%, P~=~0.496). The 5-year DSS was similar between SC and EC
patients in America (82.3\% vs.~100.0\%, P~=~0.249) as well as in Asia
(98.6\% vs.~95.2\%, P~=~0.690). The 5-year DSS also did not differ
according to lymph node metastasis (P~=~0.688) or tumor location
(P~=~0.474). CONCLUSIONS: SC showed similar clinical outcomes (including
recurrence) and survival outcomes as EC; therefore, EC is not needed for
the treatment of T1b GBC.

{}

{}

\begin{center}\rule{0.5\linewidth}{\linethickness}\end{center}

 - \textbf{Positive relationship between number of negative lymph nodes
and duration of gallbladder cancer cause-specific survival after
surgery}

\emph{Cancer management and research 2018 ;10():6961-6969}

PubMed: \url{https://www.ncbi.nlm.nih.gov/pubmed/?term=30588101}

Background: Although the prognostic implications of negative lymph nodes
(NLNs) has been reported for a variety of tumors, little information has
been published about the NLNs in gallbladder cancer (GBC). Patients and
methods: In this study, clinicopathological characteristics and survival
times of patients who had undergone surgery for GBC were collected from
the Surveillance, Epidemiology, and End Results Program-registered TNM
stage database and analyzed. Univariate and multivariate Cox
proportional hazards models were used to identify the predictors of
survival. Results: It was found that a cutoff of one to two NLNs is
optimal when assessing the association with survival, survival rates
being consistently better with two or more NLNs than with fewer than
two. This optimal cutoff value of 2 was identified as an independent
prognostic factor by univariate and multivariate analyses (all
P\textless{}0.001). Specifically, patients with two or more NLNs had
better 5-year gallbladder cancer cause-specific survival than those with
fewer than NLNs examined for stage I/II, stage III/IV, and all TNM
stages (all P\textless{}0.001). Conclusion: Our findings indicate that
the number of NLNs is an independent prognostic factor after GBC
surgery, and, together with the number of positive lymph nodes, this
will provide better prognostic information than the number of positive
lymph nodes alone.

{}

{}

\begin{center}\rule{0.5\linewidth}{\linethickness}\end{center}

 - \textbf{External validation of the 8th American Joint Committee on
Cancer staging system for gall bladder carcinoma}

\emph{Journal of gastrointestinal oncology 2018 Dec;9(6):1084-1090}

PubMed: \url{https://www.ncbi.nlm.nih.gov/pubmed/?term=30603127}

Background: To validate the changes within the American Joint Committee
on Cancer (AJCC) 8th staging system for gall bladder carcinoma compared
to AJCC 7th staging system. Methods: Surveillance, Epidemiology and End
Results (SEER) database {[}2004-2014{]} was queried. Kaplan-Meier
survival analyses and Log-rank testing were assessed according to both
AJCC 7th and 8th staging systems. Likewise, Cox cancer-specific hazard
ratio was evaluated according to both staging systems. Results: Overall
survival was assessed according to the two staging systems; and P values
for overall trend (log/rank test) were significant (P\textless{}0.001)
for both scenarios. Cox regression cancer-specific hazard adjusted for
age, gender, histology, gender and surgery was evaluated according to
the two staging systems. According to AJCC 7th staging system, the
following pair wise hazard ratio comparisons were significant (II
vs.~IIIA; IIIB vs.~IVA; IVA vs.~IVB). According to AJCC 8th staging
system, the following pair wise hazard ratio comparisons were
significant (II vs.~IIIA; IVA vs.~IVB). C-statistic was assessed using
death from gall bladder carcinoma as the dependent variable; and the
findings for the two staging systems were as follows: AJCC 7th staging
system: 0.684 (SE: 0.008; 95\% CI: 0.667-0.701); AJCC 8th staging
system: 0.682 (SE: 0.009; 95\% CI: 0.665-0.698). Conclusions: There is a
comparable discriminatory performance for AJCC 8th staging system
compared to AJCC 7th staging system. Change form location-based to
number-based N category assessment does not improve the overall
prognostic performance of the staging system.

{}

{}

\begin{center}\rule{0.5\linewidth}{\linethickness}\end{center}

Back to top

\begin{center}\rule{0.5\linewidth}{\linethickness}\end{center}

\pagebreak

\hypertarget{ampulla-of-vater}{%
\subsubsection{Ampulla of Vater}\label{ampulla-of-vater}}

\begin{center}\rule{0.5\linewidth}{\linethickness}\end{center}

\hypertarget{morphology-diagnostics-ihc-3}{%
\paragraph{Morphology, Diagnostics,
IHC}\label{morphology-diagnostics-ihc-3}}

Morphology, Diagnostics, IHC

\begin{center}\rule{0.5\linewidth}{\linethickness}\end{center}

 - \textbf{Identification of ampullary carcinoma mixed subtype using a
panel of six antibodies and its clinical significance}

\emph{Journal of surgical oncology 2019 Mar;119(3):295-302}

PubMed: \url{https://www.ncbi.nlm.nih.gov/pubmed/?term=30548547}

OBJECTIVES: To investigate the function of immunomarkers CK7, CK20,
CK17, CDX2, MUC1, and MUC2 in the identification of primary ampullary
carcinoma mixed subtype. METHODS: Forty-two cases of primary ampullary
carcinoma were performed by immunohistochemical studies. The correlation
between the mixed subtype and the other two subtypes and patient
survival data was analyzed using the SPSS 16.0 statistical software.
RESULTS: Among 42 cases, 12 (28.6\%) cases were classified as mixed
subtype, which showed variable expression patterns: 91.7\% (11/12) for
CK7, 83.3\% (10/12) for CK20; 66.7\% (8/12) for CK17, CDX2, and MUC1;
and 50\% (6/12) for MUC2. Ten (83.3\%) mixed types coexpressed four or
more immunomarkers. Eight (19\%) intestinal subtypes mainly showed a
positive expression of CK20, CDX2, and MUC2. Twenty-two (52.4\%)
pancreaticobiliary subtypes showed a positive expression of CK7, MUC1,
and CK17. Stages III and IV diseases in mixed subtype (25\%) and
intestinal subtype (25\%) were less than pancreaticobiliary
subtype(63.6\%) (p = 0.039). Follow-up data appeared to show a better
survival rate for patients with mixed subtype than those with
pancreaticobiliary subtypes. CONCLUSION: Immunohistochemical staining
provided a more reliable means of diagnosing mixed ampulla carcinoma.
Accurate subtyping of ampullary carcinoma is clinically important to
select effective chemotherapy regimens and to assess disease prognosis.

{}

{}

\begin{center}\rule{0.5\linewidth}{\linethickness}\end{center}

 - \textbf{Distinct immunological properties of the two histological
subtypes of adenocarcinoma of the ampulla of Vater}

\emph{Cancer immunology, immunotherapy : CII 2019 Jan;():}

PubMed: \url{https://www.ncbi.nlm.nih.gov/pubmed/?term=30604042}

Adenocarcinoma of the ampulla of Vater (AOV) is classified into
intestinal type (IT) and pancreatobiliary type (PB); however, the
immunological properties of these subtypes remain to be characterized.
Here, we evaluated the clinical implications of PD-L1 expression and
CD8+ T lymphocyte density in adenocarcinomas of the AOV and their
potential association with Yes-associated protein (YAP). We analyzed 123
adenocarcinoma-of-the-AOV patients who underwent surgical resection, and
tumors were classified into IT or PB type. Tumor or inflammatory cell
PD-L1 expression, CD8+ T lymphocyte density in the cancer cell nest
(intratumoral) or in the adjacent stroma, and YAP localization and
intensity were analyzed using immunohistochemical staining. PB-type
tumors showed higher tumoral PD-L1 expression than IT-type tumors, and
tumoral PD-L1 expression was associated with a shorter disease-free
survival (DFS) {[}hazard ratio (HR), 1.77; p = 0.045{]} and overall
survival (OS) (HR 1.99; p = 0.030). Intratumoral CD8+ T lymphocyte
density was higher in IT type than in PB type and was associated with a
favorable DFS (HR 0.47; p = 0.022). The nuclear staining pattern of YAP
in tumor cells, compared to non-nuclear staining patterns, was more
frequently associated with PB type and increased tumoral PD-L1
expression. Nuclear YAP staining was a significant prognostic factor for
OS (HR 2.21; p = 0.022). These results show that the two subtypes of
adenocarcinoma of the AOV exhibit significant differences in tumoral
PD-L1 expression and intratumoral CD8+ T lymphocyte density, which might
contribute to their distinct clinical features.

{}

{}

\begin{center}\rule{0.5\linewidth}{\linethickness}\end{center}

Back to top

\begin{center}\rule{0.5\linewidth}{\linethickness}\end{center}

\pagebreak

\hypertarget{staging-3}{%
\paragraph{Staging}\label{staging-3}}

Ampulla of Vater TNM staging, Margins, Survival

\begin{center}\rule{0.5\linewidth}{\linethickness}\end{center}

 - \textbf{Histopathologic Predictors of Survival and Recurrence in
Resected Ampullary Adenocarcinoma: International Multicenter Cohort
Study}

\emph{Annals of surgery 2019 Jan;():}

PubMed: \url{https://www.ncbi.nlm.nih.gov/pubmed/?term=30628913}

OBJECTIVE: The aim of the study was to define histopathologic
characteristics that independently predict overall survival (OS) and
disease-free survival (DFS), in patients who underwent resection of an
ampullary adenocarcinoma with curative intent. SUMMARY BACKGROUND DATA:
A broad range of survival rates have been described for adenocarcinoma
of the ampulla of Vater, presumably due to morphological heterogeneity
which is a result of the different epitheliums ampullary adenocarcinoma
can arise from (intestinal or pancreaticobiliary). Large series with
homogenous patient selection are scarce. METHODS: A retrospective
multicenter cohort analysis of patients who underwent
pancreatoduodenectomy for ampullary adenocarcinoma in 9 European
tertiary referral centers between February 2006 and December 2017 was
performed. Collected data included demographics, histopathologic
details, survival, and recurrence. OS and DFS analyses were performed
using Kaplan-Meier curves and Cox proportional hazard models. RESULTS:
Overall, 887 patients were included, with a mean age of 66 ± 10 years.
The median OS was 64 months with 1-, 3-, 5-, and 10-year OS rates of
89\%, 63\%, 52\%, and 37\%, respectively. Histopathologic subtype,
differentiation grade, lymphovascular invasion, perineural invasion,
T-stage, N-stage, resection margin, and adjuvant chemotherapy were
correlated with OS and DFS. N-stage (HR = 3.30 {[}2.09-5.21{]}),
perineural invasion (HR = 1.50 {[}1.01-2.23{]}), and adjuvant
chemotherapy (HR = 0.69 {[}0.48-0.97{]}) were independent predictors of
OS in multivariable analysis, whereas DFS was only adversely predicted
by N-stage (HR = 2.65 {[}1.65-4.27{]}). CONCLUSIONS: Independent
predictors of OS in resected ampullary cancer were N-stage, perineural
invasion, and adjuvant chemotherapy. N-stage was the only predictor of
DFS. These findings improve predicting survival and recurrence after
resection of ampullary adenocarcinoma.

{}

{}

\begin{center}\rule{0.5\linewidth}{\linethickness}\end{center}

 - \textbf{Prognostic importance of lymph node ratio after resection of
ampullary carcinomas}

\emph{Journal of gastrointestinal oncology 2018 Dec;9(6):1144-1149}

PubMed: \url{https://www.ncbi.nlm.nih.gov/pubmed/?term=30603134}

Background: The prognosis of the lymph node ratio (LNR) in Vater's
ampulla carcinomas (VACs) is recently studied. However, there are not
enough data in several populations like Latin American people. Our aim
is to demonstrate the prognosis significance of the LNR in this setting.
Methods: Pancreaticoduodenectomies for VACs were identified (n=128) from
1980 through 2015. Based on a ROC curve, a cut-off point of 0.1 was
assigned for the LNR and the population was divided into two groups for
comparison. Results: The LNR ≥0.1 group was statistically significant
associated with recurrence (38.5\% vs.~19.5\%), pT3-T4 tumors (69.2\%
vs.~29.3\%), poorly differentiated tumors (46.2\% vs.~17.5\%),
lymphovascular invasion (61.5 vs.~17.1\%), perineural invasion (38.5\%
vs.~19.5\%), and positive margins (15.4\% vs.~2.4\%). In the
multivariate analysis, LNR (HR 2.891; CI: 1.987-3.458, P=0.02), LNM (HR
2.945; CI: 2.478-3.245, P=0.002), perineural invasion (HR 3.327; CI:
3.172-4.156, P=0.003), and recurrence (HR 3.490; CI: 2.896-4.122,
P=0.001) were associated with lower survival. Conclusions: The LNR is a
good predictor of survival and worse oncological outcomes for VACs after
resection.

{}

{}

\begin{center}\rule{0.5\linewidth}{\linethickness}\end{center}

 - \textbf{Prognostic Nomogram for Disease-Specific Survival in Patients
with Non-metastatic Ampullary Carcinoma After Surgery}

\emph{Annals of surgical oncology 2019 Jan;():}

PubMed: \url{https://www.ncbi.nlm.nih.gov/pubmed/?term=30659390}

OBJECTIVE: The aim of this study was to establish and validate an
individualized nomogram for predicting disease-specific survival (DSS)
in patients with non-metastatic ampullary carcinoma after surgery.
METHODS: The nomogram was prepared using retrospective data from the
Surveillance, Epidemiology, and End Results database, and included 2022
patients (training dataset: 1276; validation dataset: 746 patients) with
non-metastatic ampullary carcinoma who were surgically treated between
2004 and 2014. Cox multivariate regression was performed to identify
independent risk factors. The predictive accuracy was determined using
the concordance index (C-index) and calibration curves. Results were
validated internally using bootstrap resampling, and externally against
the validation dataset. RESULTS: The median follow-up for the training
dataset was 25.5~months (range 1-143), the median survival time was
52~months {[}95\% confidence interval (CI) 41.67-62.33{]}, and the
postoperative 1-, 3-, and 5-year DSS rates were 86.7\%, 57.3\%, and
47.2\%, respectively. Univariate and multivariate regression analysis
demonstrated that age, grade, tumor size, lymph node ratio, extension
range, and~histology were independent risk factors for DSS. The C-index
of the internal validation dataset for predicting DSS was 0.70 (95\% CI
0.68-0.72), which was superior to that of the American Joint Committee
on Cancer staging, i.e.~0.64 (95\% CI 0.62-0.66; p \textless{} 0.001).
The 5-year DSS and median DSS time for the low-risk group were
significantly greater than those for the high-risk group
(p \textless{} 0.001). CONCLUSION: Our nomogram reliably and accurately
predicted DSS in patients with non-metastatic ampullary carcinoma after
surgery. This model may help clinicians in their decision making.

{}

{}

\begin{center}\rule{0.5\linewidth}{\linethickness}\end{center}

Back to top

\begin{center}\rule{0.5\linewidth}{\linethickness}\end{center}

\pagebreak

\hypertarget{neuroendocrine}{%
\subsubsection{Neuroendocrine}\label{neuroendocrine}}

\begin{center}\rule{0.5\linewidth}{\linethickness}\end{center}

\hypertarget{pannet}{%
\paragraph{PanNET}\label{pannet}}

PanNET, Pancreatic Neuroendocrine Tumors and related neuroendocrine
neoplasms

\begin{center}\rule{0.5\linewidth}{\linethickness}\end{center}

 - \textbf{Clinicopathological characteristics of non-functioning cystic
pancreatic neuroendocrine tumors}

\emph{Pancreatology : official journal of the International Association
of Pancreatology (IAP) \ldots{} {[}et al.{]} 2019 Jan;19(1):50-56}

PubMed: \url{https://www.ncbi.nlm.nih.gov/pubmed/?term=30497875}

BACKGROUND/OBJECTIVES: The biological features of cystic pancreatic
neuroendocrine tumors (PNETs) remain unclear. The aim of this study was
to clarify the clinicopathological characteristics of non-functioning
PNETs (NF-PNETs) with a cystic component. METHODS: The medical records
of 75 patients with NF-PNETs who had undergone resection in our
institution were retrospectively reviewed. Clinicopathological factors
were compared between PNETs with and without a cystic component.
Expression of somatostatin 2 receptor (SSTR-2) was also analyzed.
RESULTS: Cystic PNETs were diagnosed in 14 patients (19\%). The
proportion of men was significantly higher for cystic than solid PNETs
(79\% vs.~44\%, P\,\textless{}\,0.05) and cystic PNETs were
significantly larger than solid PNETs (25\,mm vs.~17\,mm,
P\,\textless{}\,0.01). However, there were no significant differences in
the prevalence of lymph node metastases (14\% vs.~10\%, P\,=\,0.64),
hepatic metastasis (7\% vs.~3\%, P\,=\,0.54), or disease-free survival
rate (both 86\%, P\,=\,0.29) between PNETs with and without a cystic
component. SSTR-2 expression was more frequently observed in PNETs with
a cystic component than in those without (100\% vs.~70\%,
P\,\textless{}\,0.01). CONCLUSIONS: Although cystic PNETs were larger
upon diagnosis than solid PNETs in this study, prognosis after surgical
resection did not differ significantly between these types of PNET.
Somatostatin receptor scintigraphy and somatostatin analogues may be
more useful for diagnosing and treating cystic PNETs, respectively.

{}

{}

\begin{center}\rule{0.5\linewidth}{\linethickness}\end{center}

 - \textbf{In Situ Hybridization Analysis of Long Non-coding RNAs MALAT1
and HOTAIR in Gastroenteropancreatic Neuroendocrine Neoplasms}

\emph{Endocrine pathology 2019 Jan;():}

PubMed: \url{https://www.ncbi.nlm.nih.gov/pubmed/?term=30600442}

Recent studies suggest onco-regulatory roles for two long non-coding
RNAs (lncRNAs), MALAT1 and HOTAIR, in various malignancies; however,
these lncRNAs have not been previously examined in neuroendocrine
neoplasms (NENs) of gastroenteropancreatic origins (GEP-NENs). In this
study, we evaluated the expressions and prognostic significance of
MALAT1 and HOTAIR in 83 cases of GEP-NENs (60 grade 1, 17 grade 2, and 6
grade 3 tumors) diagnosed during the years 2005-2017. Expression levels
of MALAT1 and HOTAIR were digitally quantitated in assembled tissue
microarray slides labeled by chromogenic in situ hybridization (ISH)
using InForm 1.4.0 software. We found diffuse nuclear expression of both
HOTAIR and MALAT1 in all primary tumors of GEP-NENs with variable
intensities. By multivariate model which adjusted for age and histologic
grade, high expression of HOTAIR was associated with lower presenting T
and M stages and subsequent development of metastases
(P \textless{} 0.05). MALAT1 expression was associated with presenting T
stage and development of metastases (P \textless{} 0.05). In summary,
MALAT1 and HOTAIR are commonly expressed in GEP-NENs. High expression of
either lncRNA showed grade-independent associations with clinically less
aggressive disease.

{}

{}

\begin{center}\rule{0.5\linewidth}{\linethickness}\end{center}

 - \textbf{Can we predict recurrence in WHO G1-G2 pancreatic
neuroendocrine neoplasms? Results from a multi-institutional Spanish
study}

\emph{Pancreatology : official journal of the International Association
of Pancreatology (IAP) \ldots{} {[}et al.{]} 2019 Jan;():}

PubMed: \url{https://www.ncbi.nlm.nih.gov/pubmed/?term=30683515}

INTRODUCTION: Pancreatic neuroendocrine neoplasms (PNEN) are rare
tumours and well differentiated PNEN are associated with relatively
indolent physiological behaviour. For this reason, only few studies have
investigated those factors associated with recurrence in this group of
patients. The aim of this study is to analyse whether it is possible to
predict tumour recurrence in World Health Organization (WHO) 2017 G1-G2
PNEN patients. METHODS: This is a retrospective multi-institutional
study. Patients submitted to pancreatic resection from 7 Spanish centres
were reviewed. Only patients with WHO G1-G2 PNEN were included.
Demographic and clinicopathological variables were analysed. RESULTS:
Data from 137 patients were reviewed. Median age was 59.2 (25-84) years.
Recurrence of disease occurred in 19 (13.9\%) patients. Median DFS was
55 months. At multivariate analysis, tumour size \textgreater{}20\,mm,
lymphnode metastasis and a new tumour grade 2 incorporating Ki-67
labelling index (LI)\,\textgreater{}\,5\% and mitotic index
(MI)\,\textgreater{}\,2 were independently associated with recurrence.
We developed a risk score model with these three factors. High-risk
patients had a significantly lower 5-year disease-specific survival
compared to low-risk patients (70\% vs 100\%). CONCLUSION: We propose a
novel risk score for recurrence based on lymphnode metastasis, tumour
size\,\textgreater{}\,20\,mm and a new grade 2 based on Ki-67 LI
\textgreater{}5\% and MI\,\textgreater{}\,2. If 2 factors are present,
patients have a higher risk for recurrence and a significantly poorer
DSS, and therefore they should be closely monitored during follow-up.
The role of adjuvant chemotherapy in these patients needs to be
evaluated in clinical trials.

{}

{}

\begin{center}\rule{0.5\linewidth}{\linethickness}\end{center}

Back to top

\begin{center}\rule{0.5\linewidth}{\linethickness}\end{center}

\pagebreak

\hypertarget{cytopathology}{%
\subsection{Cytopathology}\label{cytopathology}}

\begin{center}\rule{0.5\linewidth}{\linethickness}\end{center}

\hypertarget{pancreas-1}{%
\subsubsection{Pancreas}\label{pancreas-1}}

\begin{center}\rule{0.5\linewidth}{\linethickness}\end{center}

 - \textbf{Touch imprint cytology on endoscopic ultrasound fine-needle
biopsy provides comparable sample quality and diagnostic yield to
standard endoscopic ultrasound fine-needle aspiration specimens in the
evaluation of solid pancreatic lesions}

\emph{Cytopathology : official journal of the British Society for
Clinical Cytology 2018 Nov;():}

PubMed: \url{https://www.ncbi.nlm.nih.gov/pubmed/?term=30484917}

OBJECTIVES: Endoscopic ultrasound-guided fine-needle aspiration
(EUS-FNA) is the gold standard for the diagnosis of solid pancreatic
lesions (SPLs). Cytological samples can also be obtained using touch
imprint cytology (TIC) on EUS fine-needle biopsy (FNB) specimens. We
aimed to compare sample quality and diagnostic yield of EUS-FNA-standard
cytology (EUS-FNA-SC) to that of EUS-FNB-TIC in a series of patients
with SPLs. METHODS: Thirty-two consecutive patients referred for
EUS-tissue acquisition of SPLs who underwent rapid on-site evaluation of
both EUS-FNA-SC and paired EUS-FNB-TIC during the same endoscopic
session were retrospectively identified. Sample quality (evaluated in
terms of blood contamination, presence of clots, tissue casts,
cellularity, and necrosis) and diagnostic yield were compared between
the techniques. RESULTS: The mean number of passes to reach diagnosis at
rapid on-site evaluation was similar between EUS-FNA-SC and EUS-FNB-TIC
(1.09~±~0.3 vs 1.13~±~0.34, P = .711). EUS-FNA-SC scores of sample
quality were comparable to those of EUS-FNB-TIC (blood contamination,
2.47~±~1.11 vs 2.25~±~1.14, P = .109; clots, 1.25~±~0.76 vs 1.19~±~0.69,
P = .624; tissue casts, 3.56~±~0.88 vs 3.59~±~1.09, P = .872;
cellularity, 2.84~±~1.11 vs 3.09~±~1.09, P = .244; necrosis, 2.25~±~1.08
vs 2.53~±~1.02 P = .059; total score, 12.38~±~2.88 vs 17.66~±~2.38, P =
.536). Adequacy, sensitivity and diagnostic accuracy of the two sampling
techniques were equal (93.7\%, 90.6\% and 90.6\%, respectively).
CONCLUSIONS: EUS-FNB-TIC provides comparable samples to those of
EUS-FNA-SC and combines the benefits of cytology and histology for the
evaluation of SPLs by employing a single needle during the same
endoscopic procedure.

{}

{}

\begin{center}\rule{0.5\linewidth}{\linethickness}\end{center}

 - ****

\begin{itemize}
\tightlist
\item
  ;():*
\end{itemize}

PubMed: \url{https://www.ncbi.nlm.nih.gov/pubmed/?term=}

{}

{}

\begin{center}\rule{0.5\linewidth}{\linethickness}\end{center}

 - \textbf{Pleomorphic and atypical multinucleated giant cells in solid
pseudopapillary neoplasm of pancreas: A diagnostic pitfall in cytology
and a review of the literature}

\emph{Diagnostic cytopathology 2018 Dec;():}

PubMed: \url{https://www.ncbi.nlm.nih.gov/pubmed/?term=30552752}

Solid pseudopapillary neoplasm of the pancreas (SPN) is a rare low-grade
malignancy typically occurring in young women. Occasionally, these
neoplasms present with pleomorphic to atypical multinucleated giant
tumor cells which may mimic high-grade malignancy. Our patient is a
25-year-old male who presented with one year of intermittent epigastric
pain. Magnetic resonance imaging showed a 3.1 × 2.5 cm mass in the
pancreas body. Endoscopic ultrasound-guided fine needle aspiration of
the mass showed large pleomorphic cells and atypical multinucleated
giant cells in a background of singly scattered polygonal cells.
Focally, these cells surrounded delicate hyalinized to fibrovascular
cores forming pseudopapillae. Immunohistochemical stains show tumor
cells are positive for beta-catenin, CD10, vimentin, and CD56. Although
rare surgical pathology publications have described the presence of
pleomorphic to atypical multinucleated giant cells occurring in SPN, to
our knowledge, this is the first case reported example focused on
cytomorphologic illustration and description.

{}

{}

\begin{center}\rule{0.5\linewidth}{\linethickness}\end{center}

 - \textbf{Cytopathological results of initial endoscopic
ultrasound-guided fine needle aspiration for primary mass and prognosis
in pancreatic cancer patients}

\emph{Cytopathology : official journal of the British Society for
Clinical Cytology 2018 Dec;():}

PubMed: \url{https://www.ncbi.nlm.nih.gov/pubmed/?term=30570774}

OBJECTIVES: Clinical outcomes remain unclear in patients suspected of
having pancreatic cancer with indeterminate endoscopic ultrasound-guided
fine needle aspiration (EUS-FNA) results. This work aimed to investigate
the prognosis of pancreatic cancer patients with indeterminate findings
at initial EUS-FNA. METHODS: Findings in all patients who underwent
EUS-FNA for suspected pancreatic cancer between 2008 and 2015 at the
National Cancer Center, Korea, were retrospectively reviewed. A final
diagnosis of pancreatic ductal adenocarcinoma was based on pathology
reports. RESULTS: Of the 144 patients evaluated, 113 (78\%) were
diagnosed as being positive/suspicious for malignancy on cytological
evaluation and 31 (22\%) as having atypia/negative/non-diagnostic
findings at initial EUS-FNA but subsequently diagnosed with pancreatic
ductal adenocarcinoma. Tumour size, clinical stage and treatment
modalities did not differ significantly between these two groups. Median
overall survival was significantly shorter in patients diagnosed
(11.3~±~0.74~months; 95\% confidence interval {[}CI{]}, 9.4-12.8~months)
than non-diagnosed (16.9~±~2.34~months; 95\% CI, 12.0-17.4~months) on
initial EUS-FNA (P~=~.024). Multivariate Cox regression analysis showed
that a non-diagnosis on initial EUS-FNA was independently associated
with better overall survival (hazard ratio, 0.58; 95\% CI, 0.38-0.88;
P~=~.011). CONCLUSIONS: Non-diagnostic results on initial EUS-FNA of a
primary mass may be associated with better prognosis in patients with
pancreatic cancer.

{}

{}

\begin{center}\rule{0.5\linewidth}{\linethickness}\end{center}

 - \textbf{Utility of cytomorphology in distinguishing solid
pseudopapillary neoplasm of pancreas from pancreatic neuroendocrine
tumor with emphasis on nuclear folds and nuclear grooves}

\emph{Diagnostic cytopathology 2019 Jan;():}

PubMed: \url{https://www.ncbi.nlm.nih.gov/pubmed/?term=30677247}

BACKGROUND: Pancreatic solid pseudopapillary tumor (SPN) and pancreatic
neuroendocrine tumors (Pan-NET) have close resemblance on imaging and
cytomorphology, though they differ in their prognosis and treatment
strategy. SPNs are low-grade indolent tumors while Pan-NETs harbor
malignant potential with propensity to metastasize. We aim to
differentiate SPN from Pan-NET based on cyto-morphology; to classify
nuclear membrane (NM) irregularities or nuclear folds into four grades
and see whether they bear any difference with respect to the two
entities. METHODS: Eighteen and ten confirmed cases of SPN and Pan-NET
were included in the study. Smears were assessed for architecture,
background changes, cellular, and nuclear features, which were compared
between the two study groups. Nuclear folds were classified into four
grades. Nuclear folds and nuclear grooves were also compared between the
two groups. RESULTS: All SPN patients were females; mean age of
28 years. Pan-NET patients had equal male to female ratio; mean age of
46 years. Both SPN (78\%) and Pan-NET (71\%) showed predilection for
pancreatic head. Mean size of lesion was 4.8 cm and 3.1 cm in SPN and
Pan-NET groups. Papillary pattern, branching capillaries, degenerative
background were significantly more prominent in SPN; sudden
anisonucleosis and cytoplasmic granularity in Pan-NET. Metachromatic
matrix, hyaline globules, and nuclear grooves were noted exclusively in
SPNs. Nuclear fold grades 2 and 3 were more characteristic of SPN than
Pan-NET (P = 0.041 and 0.002, respectively). CONCLUSIONS: Cytomorphology
is vital in differentiating SPN from Pan-NET with nuclear folds being an
important nuclear feature.

{}

{}

\begin{center}\rule{0.5\linewidth}{\linethickness}\end{center}

Back to top

\begin{center}\rule{0.5\linewidth}{\linethickness}\end{center}

\pagebreak

\hypertarget{bile-ducts-1}{%
\subsubsection{Bile Ducts}\label{bile-ducts-1}}

\begin{center}\rule{0.5\linewidth}{\linethickness}\end{center}

 - \textbf{Factors Associated with Malignant Biliary Strictures in
Patients with Atypical or Suspicious Cells on Brush Cytology}

\emph{Clinical endoscopy 2019 Jan;():}

PubMed: \url{https://www.ncbi.nlm.nih.gov/pubmed/?term=30625267}

Background/Aims: Pathological diagnosis of biliary strictures with
atypical or suspicious cells on endoscopic retrograde brush cytology and
indeterminate strictures on imaging is challenging. The aim of this
study was to identify markers for malignant strictures in such cases.
Methods: We retrospectively analyzed data collected from 146 consecutive
patients with indeterminate biliary strictures on imaging who underwent
endoscopic retrograde brush cytology from 2007 to 2013. Factors
associated with malignant strictures in patients with atypical or
suspicious cells on brush cytology were identified. Results: Among the
67 patients with a malignant disease (48 cholangiocarcinoma, 6
gallbladder cancer, 5 pancreatic cancer, 5 ampulla of Vater cancer, and
3 other types), 36 (53.7\%) had atypical or suspicious cells on brush
cytology. Among these, the factors that independently correlated with
malignant strictures were stricture length (odds ratio {[}OR{]}, 5.259;
95\% confidence interval {[}CI{]}, 1.802- 15.294) and elevated
carbohydrate antigen 19-9 (CA19-9) (OR, 3.492; 95\% CI, 1.242-9.815),
carcinoembryonic antigen (CEA) (OR, 4.909; 95\% CI, 1.694-14.224),
alkaline phosphatase (ALP) (OR, 3.362; 95\% CI, 1.207-9.361), and
gamma-glutamyl transpeptidase (rGT) (OR, 4.318; 95\% CI, 1.512-12.262).
Conclusions: Elevated levels of CA19-9, CEA, ALP, and rGT and stricture
length are associated with malignant strictures in patients with
indeterminate biliary strictures on imaging and atypical or suspicious
cells on brush cytology.

{}

{}

\begin{center}\rule{0.5\linewidth}{\linethickness}\end{center}

Back to top

\begin{center}\rule{0.5\linewidth}{\linethickness}\end{center}

\pagebreak

\hypertarget{neuroendocrine-1}{%
\subsubsection{Neuroendocrine}\label{neuroendocrine-1}}

\begin{center}\rule{0.5\linewidth}{\linethickness}\end{center}

 - \textbf{Advances in the cytologic diagnosis of gastroenteropancreatic
neuroendocrine neoplasms}

\emph{Cancer cytopathology 2018 Dec;126(12):980-991}

PubMed: \url{https://www.ncbi.nlm.nih.gov/pubmed/?term=30485690}

Two-thirds of neuroendocrine neoplasms arising in the human body
originate from the gastrointestinal system or pancreas.
Gastroenteropancreatic neuroendocrine neoplasms are heterogeneous,
comprising both well differentiated neuroendocrine tumors (NETs) and
poorly differentiated neuroendocrine carcinomas (NECs). The clinical
presentation, molecular characteristics, and behavior are distinct for
NETs and NECs. Fine-needle aspiration is an important modality for the
primary diagnosis and staging of these neoplasms and can provide
information of prognostic and therapeutic significance. Our evolving
understanding of neuroendocrine neoplasm biology has led to several
iterations of classification. In this review, new concepts and issues
most relevant to cytology diagnosis of gastroenteropancreatic
neuroendocrine neoplasms are discussed, such as newer detection methods
that aid in diagnosis and staging, recent changes in World Health
Organization classification, practical issues related to grading these
neoplasms on cytology, guidelines for diagnostic reporting, and panels
of immunohistochemical stains for the diagnosis of metastasis. The
current understanding of genetic and epigenetic events related to tumor
development and potential applications for cytology also are presented
as they relate to prognostication and recent therapeutic advances.

{}

{}

\begin{center}\rule{0.5\linewidth}{\linethickness}\end{center}

Back to top

\begin{center}\rule{0.5\linewidth}{\linethickness}\end{center}

\pagebreak

\hypertarget{molecular-pathology}{%
\subsection{Molecular Pathology}\label{molecular-pathology}}

\begin{center}\rule{0.5\linewidth}{\linethickness}\end{center}

\hypertarget{pancreas-2}{%
\subsubsection{Pancreas}\label{pancreas-2}}

\begin{center}\rule{0.5\linewidth}{\linethickness}\end{center}

 - \textbf{miRNA and Gene Expression in Pancreatic Ductal
Adenocarcinoma}

\emph{The American journal of pathology 2019 Jan;189(1):58-70}

PubMed: \url{https://www.ncbi.nlm.nih.gov/pubmed/?term=30558723}

Pancreatic ductal adenocarcinoma (PDAC) remains a challenging disease
that is mostly diagnosed late in the course of the illness. Unlike other
cancers in which measurable successes have been achieved with
traditional chemotherapy, targeted therapy, and, recently,
immunotherapy, PDAC has proved to be poorly responsive to these
treatments, with only marginal to modest incremental benefits using
conventional cytotoxic therapy. There is, therefore, a great unmet need
to develop better therapies based on improved understanding of biology
and identification of predictive and prognostic biomarkers that would
guide therapy. miRNAs are small noncoding RNAs that regulate the
expression of some key genes by targeting their 3'-untranslated mRNA
region. Aberrant expression of miRNAs has been linked to the development
of various malignancies, including PDAC. A series of miRNAs have been
identified as potential tools for early diagnosis, prediction of
treatment response, and prognosis of patients with PDAC. In this review,
we present a summary of the miRNAs that have been studied in PDAC in
the~context of disease biology.

{}

{}

\begin{center}\rule{0.5\linewidth}{\linethickness}\end{center}

 - \textbf{Mechanosignalling via integrins directs fate decisions of
pancreatic progenitors}

\emph{Nature 2018 12;564(7734):114-118}

PubMed: \url{https://www.ncbi.nlm.nih.gov/pubmed/?term=30487608}

The pancreas originates from two epithelial evaginations of the foregut,
which consist of multipotent epithelial progenitors that organize into a
complex tubular epithelial network. The trunk domain of each epithelial
branch consists of bipotent pancreatic progenitors (bi-PPs) that give
rise to both duct and endocrine lineages, whereas the tips give rise to
acinar cells1. Here we identify the extrinsic and intrinsic signalling
mechanisms that coordinate the fate-determining transcriptional events
underlying these lineage decisions1,2. Single-cell analysis of
pancreatic bipotent pancreatic progenitors derived from human
embryonic~stem cells reveal that cell confinement is a prerequisite for
endocrine specification, whereas spreading drives the progenitors
towards a ductal fate. Mechanistic studies identify the interaction of
extracellular matrix (ECM) with integrin α5 as the extracellular cue
that cell-autonomously, via the F-actin-YAP1-Notch mechanosignalling
axis, controls the fate of bipotent pancreatic progenitors. Whereas
ECM-integrin α5 signalling promotes differentiation towards the duct
lineage, endocrinogenesis is stimulated when this signalling cascade is
disrupted. This cascade can be disrupted pharmacologically or
genetically to convert bipotent pancreatic progenitors derived from
human embryonic stem cells to hormone-producing islet cells. Our
findings identify the cell-extrinsic and intrinsic mechanotransduction
pathway that acts as gatekeeper in the fate decisions of bipotent
pancreatic progenitors in the developing pancreas.

{}

{}

\begin{center}\rule{0.5\linewidth}{\linethickness}\end{center}

 - \textbf{Identification of Key Potential Targets and Pathway for
Arsenic Trioxide by Systemic Bioinformatics Analysis in Pancreatic
Cancer}

\emph{Pathology oncology research : POR 2018 Nov;():}

PubMed: \url{https://www.ncbi.nlm.nih.gov/pubmed/?term=30506130}

Arsenic trioxide is an approved chemotheraputic agent for the treatment
of acute promyelocytic leukemia (APL). Recently, numerous studies
suggested that arsenic trioxide acts as anti-cancer roles in various
human malignancies. However, the molecular mechanisms are not fully
elucidated. In this study, we explored the critical targets of arsenic
trioxide and their interaction network systematically by searching the
publicly available published database like DrugBank (DB) and STRING.
Seven direct protein targets (DPTs) and 111 DPT-associated genes were
identified. The enrichment analysis of arsenic trioxide associated
genes/proteins revealed 10 Kyoto Encyclopedia of Genes and Genomes
(KEGG) pathways. Among these pathways,
phosphatidylinositol-4,5-bisphosphate-3-kinase -Akt (PI3K-Akt) single
pathway and pancreatic cancer pathway are highly correlated with arsenic
trioxide and have 5 overlapped targets. Then we investigated the gene
alternation of selected critical genes in pancreatic cancer studies
using cBio portal. These results indicated that arsenic trioxide could
act anti-tumor function through PI3K-Akt single pathway and identified
critical genes might be therapeutic targets for pancreatic cancer.

{}

{}

\begin{center}\rule{0.5\linewidth}{\linethickness}\end{center}

 - \textbf{Is mitochondrial DNA copy number a good prognostic marker in
resectable pancreatic cancer?}

\emph{Pancreatology : official journal of the International Association
of Pancreatology (IAP) \ldots{} {[}et al.{]} 2019 Jan;19(1):73-79}

PubMed: \url{https://www.ncbi.nlm.nih.gov/pubmed/?term=30528645}

BACKGROUND: The aim of this prospective study was to investigate
mitochondrial DNA (mtDNA) copy number in a group of resectable
pancreatic cancer (PC) tumor tissues and adjacent normal pancreatic
tissues, and to explore the correlation between the mtDNA content in
tissues and the clinicopathological parameters and the overall survival.
METHODS: Relative mtDNA copy number was measured by the quantitative
PCR-based assay. The tumors specimens (n\,=\,43) originated from the
patients with pathologically confirmed pancreatic ductal adenocarcinoma
who did not receive any neoadjuvant systemic therapy. The adjacent
normal pancreatic tissue samples (n\,=\,31) were obtained from surgical
margins. RESULTS: mtDNA copy number was significantly lower in PC tissue
(P\,\textless{}\,0.001) compared to adjacent normal pancreatic tissue.
Jonckheere-Terpstra trend testing indicated a statistically significant
decrease in median mtDNA copy number across the differentiation
(adjacent normal pancreatic tissue, low-grade, intermediate-grade,
high-grade cancer), P\,\textless{}\,0.001. However, the survival
analyses failed to show a significant difference in survival between
patients with high and low mtDNA copy number. CONCLUSIONS: To the best
of our knowledge, we provided the first evidence that mitochondrial DNA
copy number was significantly lower in pancreatic cancer tissue
(P\,\textless{}\,0.001) compared to adjacent normal pancreatic tissue.
Also, we demonstrated that mitochondrial copy number was not a
significant marker for predicting prognosis in resectable pancreatic
cancer.

{}

{}

\begin{center}\rule{0.5\linewidth}{\linethickness}\end{center}

 - \textbf{Sonic Hedgehog Protein is Frequently Up-Regulated in
Pancreatic Cancer Compared to Colorectal Cancer}

\emph{Pathology oncology research : POR 2018 Dec;():}

PubMed: \url{https://www.ncbi.nlm.nih.gov/pubmed/?term=30539521}

Sonic hedgehog (SHH) is a secreted protein which functions in autocrine
or paracrine fashion on target cells to activate hedgehog (HH)
signalling cascade responsible for growth and proliferation. This study
is an attempt to understand the expression dynamics of SHH protein in
colon, rectal and pancreatic cancers. Protein expression of SHH was
studied by Western Blotting in the histologically confirmed colon,
rectum and pancreatic cancer tissue samples along with their adjacent
normal tissues. Only 31.4\% (11 of 35) and 26.9\% (7 of 26) of colon and
rectal cancer cases respectively showed an increase in SHH expression in
tumours compared to 72.7\% (24 of 33) of the pancreatic cancer cases
when compared with their adjacent normal tissues. Our results suggest
that SHH may have a strong role in the predisposition of Pancreatic
cancer and could possibly be used as a diagnostic or prognostic
biomarker.

{}

{}

\begin{center}\rule{0.5\linewidth}{\linethickness}\end{center}

 - \textbf{Higher notch expression implies poor survival in pancreatic
ductal adenocarcinoma: A systematic review and meta-analysis}

\emph{Pancreatology : official journal of the International Association
of Pancreatology (IAP) \ldots{} {[}et al.{]} 2018 Dec;18(8):954-961}

PubMed: \url{https://www.ncbi.nlm.nih.gov/pubmed/?term=30297095}

BACKGROUND: At present, pancreatic ductal adenocarcinoma (PDAC) is a
fetal disease lack of effective prognostic and therapeutic methods
resulting in high mortality. The Notch signaling has been demonstrated
being up- or down-regulated in many cancers, but the effects in
pancreatic ductal adenocarcinoma are still controversial. Moreover, the
available cases in an individual study are of small samples. Therefore,
it is essential to define the effect of Notch signaling in pancreatic
ductal adenocarcinoma with larger samples. METHODS: Conducted from 6
eligible studies and 463 pancreatic ductal adenocarcinoma patients, this
was the first meta-analysis to analyze the correlation between the Notch
signal pathway and pancreatic ductal adenocarcinoma. All data were
sourced from The National Center for Biotechnology Information, Web of
Science and Cochrane. The articles which matched the inclusion criteria
were included. All included data were analyzed and performed by Review
Manager 5.3. RESULTS: The results indicated that high expression of
Notch signaling proteins was associated with poor overall survival of
pancreatic ductal adenocarcinoma patients (pooled hazard
ratio\textgreater{}2.00; P\,\textless{}\,0.001). Moreover, poor survival
was related to high expression of Notch3 (pooled hazard ratio: 2.05;
confidence interval: 1.49-2.82; P\,\textless{}\,0.001) and DLL4 (pooled
hazard ratio: 2.13; confidence interval: 1.37-3.32;
P\,\textless{}\,0.001). CONCLUSIONS: This meta-analysis supports that
Notch signaling proteins may be available as prognostic factors for
pancreatic ductal adenocarcinoma progression and patient survival.
Higher expression of Notch signaling proteins indicated poor survival of
pancreatic ductal adenocarcinoma patients. Targeting Notch signaling
components, especially Notch3 protein, would be beneficial for
therapies.

{}

{}

\begin{center}\rule{0.5\linewidth}{\linethickness}\end{center}

 - \textbf{ALKBH5 gene is a novel biomarker that predicts the prognosis
of pancreatic cancer: A retrospective multicohort study}

\emph{Annals of hepato-biliary-pancreatic surgery 2018
Nov;22(4):305-309}

PubMed: \url{https://www.ncbi.nlm.nih.gov/pubmed/?term=30588520}

Backgrounds/Aims: Discovery of new prognostic factors for cases in which
the pancreatic cancer scoring and staging system does not result in a
clear definition is imperative. We examined the role of Human AlkB
homolog H5 (ALKBH5) as a prognostic marker for pancreatic cancer.
Methods: Patient data were extracted from the International Cancer
Genome Consortium (ICGC) and The Cancer Genome Atlas (TCGA). The
prognostic value of ALKBH5 was confirmed via analysis of ALKBH5 and
other clinical factors, such as age, sex, and stage, using the
time-dependent area under the curve (AUC) of Uno's C-index, the AUC
value of the receiver operating characteristics (ROC) at three years,
the Kaplan-Meier survival curve, and multivariate analysis. Results:
ALKBH5 showed excellent prognosis prediction in comparison with existing
markers in the two independent cohorts (n=262). Kaplan-Meier survival
analysis showed that ALKBH5 expression was positively associated with
overall survival (log-rank test, ICGC, p=0.001; TCGA, p=0.01). Notably,
comparison of C-index and AUC values in ROC analysis showed that ALKBH5
was associated with high C-index and AUC values compared with other
clinical variables (C-index: ICGC, 0.621; TCGA, 0.614 and AUC at three
years: ICGC, 0.609; TCGA, 0.558). Multivariate analysis demonstrated
thatALKBH5 is an independent prognostic factor (ICGC, p=0.0123; TCGA,
p\textless{}0.001). Conclusions: These findings contribute to the study
of RNA methylation in pancreatic cancer. We believe that ALKBH5 is a new
prognostic marker for pancreatic cancer.

{}

{}

\begin{center}\rule{0.5\linewidth}{\linethickness}\end{center}

 - \textbf{Agnostic Pathway/Gene Set Analysis of Genome-Wide Association
Data Identifies Associations for Pancreatic Cancer}

\emph{Journal of the National Cancer Institute 2018 Dec;():}

PubMed: \url{https://www.ncbi.nlm.nih.gov/pubmed/?term=30541042}

Background: Genome-wide association studies (GWAS) identify associations
of individual single-nucleotide polymorphisms (SNPs) with cancer risk
but usually only explain a fraction of the inherited variability.
Pathway analysis of genetic variants is a powerful tool to identify
networks of susceptibility genes. Methods: We conducted a large agnostic
pathway-based meta-analysis of GWAS data using the summary-based
adaptive rank truncated product method to identify gene sets and
pathways associated with pancreatic ductal adenocarcinoma (PDAC) in 9040
cases and 12~496 controls. We performed expression quantitative trait
loci (eQTL) analysis and functional annotation of the top SNPs in genes
contributing to the top associated pathways and gene sets. All
statistical tests were two-sided. Results: We identified 14 pathways and
gene sets associated with PDAC at a false discovery rate of less than
0.05. After Bonferroni correction (P ≤ 1.3 × 10-5), the strongest
associations were detected in five pathways and gene sets, including
maturity-onset diabetes of the young, regulation of beta-cell
development, role of epidermal growth factor (EGF) receptor
transactivation by G protein-coupled receptors in cardiac hypertrophy
pathways, and the Nikolsky breast cancer chr17q11-q21 amplicon and
Pujana ATM Pearson correlation coefficient (PCC) network gene sets. We
identified and validated rs876493 and three correlating SNPs (PGAP3) and
rs3124737 (CASP7) from the Pujana ATM PCC gene set as eQTLs in two
normal derived pancreas tissue datasets. Conclusion: Our agnostic
pathway and gene set analysis integrated with functional annotation and
eQTL analysis provides insight into genes and pathways that may be
biologically relevant for risk of PDAC, including those not previously
identified.

{}

{}

\begin{center}\rule{0.5\linewidth}{\linethickness}\end{center}

 - \textbf{High Expression of Long Noncoding RNA HOTAIRM1 is Associated
with the Proliferation and Migration in Pancreatic Ductal
Adenocarcinoma}

\emph{Pathology oncology research : POR 2019 Jan;():}

PubMed: \url{https://www.ncbi.nlm.nih.gov/pubmed/?term=30613920}

Pancreatic ductal adenocarcinoma (PDAC) is an incurable malignancy. Long
noncoding RNA (LncRNA) HOTAIRM1 (HOX antisense intergenic RNA myeloid 1)
has been shown to play important roles in the progression of several
type cancers. However, the exact role of HOTAIRM1 in PDAC development
remains largely unknown. This study aims to evaluate the potential
function of HOTAIRM1 in the development and progress of PDAC. HOTAIRM1
expression was measured by RT-qPCR in forty seven paired human PDAC
tissues and five PDAC cell lines. SW1990 and PANC-1 cells were
transfected with siHOTAIRM1 to achieve HOTAIRM1 silence. MTT assay and
colony formation assay were used to detect the effect of HOTAIRM1
knockdown on cell proliferation. The impact of HOTAIRM1 silence on cell
cycle and apoptosis was assessed by flow cytometry assay. Transwell
migration assay was performed to explore the influence of HOTAIRM1
downregulation on the migratory potential of PDAC cells. Western blot
assay was applied to determine the expression changes of cell cycle,
apoptosis, and migration-related genes before and after downregulating
HOTAIRM1. HOTAIRM1 expression was abnormally upregulated in PDAC tissues
and cells when compared with the control samples, and was positively
associated with the expression of KRAS gene mutation. In vitro
functional experiments, HOTAIRM1 expression was significantly
downregulated by transfection with siHOTAIRM1 in SW1990 and PANC cell
lines. HOTAIRM1 knockdown attenuated cell proliferation by inducing cell
cycle arrest at G0/G1 phase, promoted cell apoptosis, and inhibited cell
migration in PDAC cells by regulating related-genes expression. In
conclusion, HOTAIRM1 plays a critical role in PDAC progression, which
may be a novel diagnostic and rational therapeutic target for the
treatment of pancreatic ductal adenocarcinoma.

{}

{}

\begin{center}\rule{0.5\linewidth}{\linethickness}\end{center}

 - \textbf{Genomic Landscape of Pancreatic Adenocarcinoma in Younger vs
Older Patients: Does Age Matter?}

\emph{Clinical cancer research : an official journal of the American
Association for Cancer Research 2019 Jan;():}

PubMed: \url{https://www.ncbi.nlm.nih.gov/pubmed/?term=30617137}

PURPOSE: State-of the art genomic analyses of pancreatic adenocarcinoma
(PDAC) have yielded insight into signaling pathways underlying
carcinogenesis. PDAC is characterized by substantial genomic
heterogeneity. We aimed to determine if early-onset PDAC (EOPC; ≤ 55
years) displays a distinctive molecular landscape from average-age onset
PDAC (AOPC; ≥ 70 years). EXPERIMENTAL DESIGN: Three distinct datasets
for PDAC were analyzed. In the first, patients undergoing treatment at
Memorial Sloan Kettering (MSK) were consented for MSK-IMPACT next
generation sequencing. The second cohort analyzed was The Cancer Genome
Atlas (TCGA) dataset for differences in somatic mutations, gene
expression and protein expression. The third dataset was an Australian
cohort of PDAC. Clinical data were correlated with genomic analyses.
RESULTS: Two hundred and ninety-three samples were analyzed, yielding 90
patients (pts) aged ≤ 55 years and 203 pts aged ≥ 70 years. Among the
genes known to be associated with carcinogenesis SMAD4 displayed higher
mutation rates in younger patients. Comprehensive transcriptomic
analysis of cellular pathways indicated that the TGFb pathway has
increased activation and the expression levels of phospo-GSK3 were
higher in EOPC. Survival outcomes revealed no differences between age
groups. CONCLUSIONS: These exploratory analyses suggest that there may
be somatic gene alterations within the population of early onset PDAC
patients that involve unique cellular pathways compared with average
onset PDAC. Former studies imply these cellular pathways may play a role
in smoking-related PDAC carcinogenesis. Larger genomic datasets are
warranted for future evaluation to extend these observations.

{}

{}

\begin{center}\rule{0.5\linewidth}{\linethickness}\end{center}

 - \textbf{Rab14 overexpression regulates gemcitabine sensitivity
through regulation of Bcl-2 and mitochondrial function in pancreatic
cancer}

\emph{Virchows Archiv : an international journal of pathology 2019
Jan;474(1):59-69}

PubMed: \url{https://www.ncbi.nlm.nih.gov/pubmed/?term=30267303}

Rab family protein Rab14 has been implicated in the development of human
cancers. To date, its expression pattern, biological function, and
potential mechanism in pancreatic cancer have not been explored. In this
study, we analyzed Rab14 expression in 103 cases of pancreatic cancer
tissues using immunohistochemistry (IHC) and found that Rab14 was
overexpressed in 41/103 cases (39.8\%). Rab14 overexpression correlated
with the advanced stage. Moreover, elevated Rab14 levels indicated poor
prognosis of patients with pancreatic cancers. We used BxPC-3 and
Capan-2 respectively for plasmid and siRNA transfection. MTT and colony
formation assays showed that Rab14 transfection increased cell
proliferation and colony formation in BxPC-3 cells. Rab14 siRNA
knockdown inhibits proliferation and colony formation ability in Capan-2
cell line. Cell cycle analysis showed that Rab14 facilitated cell cycle
progression. Matrigel invasion assay showed that Rab14 promoted BxPC-3
cell invasion while its depletion inhibited Capan-2 cell invasion. In
addition, MTT and AnnexinV/PI analysis demonstrated that overexpression
of Rab14 reduced gemcitabine sensitivity which conversely was increased
by Rab14 knockdown. We also demonstrated that Rab14 upregulated
mitochondrial membrane potential (MMP) while its depletion downregulated
MMP during gemcitabine treatment. In addition, western blotting revealed
that Rab14 overexpression upregulated cyclin D1, cyclin A, cyclin E,
p-Rb, and Bcl-2 and downregulated p21. Rab14 also downregulated
caspase3, PARP cleavage, and cytochrome c release. In conclusion, our
data indicated that Rab14 was overexpressed in pancreatic cancer and
promotes growth and gemcitabine resistance, possibly through regulation
of mitochondrial function and Bcl-2.

{}

{}

\begin{center}\rule{0.5\linewidth}{\linethickness}\end{center}

 - \textbf{Identification of candidate diagnostic and prognostic
biomarkers for pancreatic carcinoma}

\emph{EBioMedicine 2019 Jan;():}

PubMed: \url{https://www.ncbi.nlm.nih.gov/pubmed/?term=30639415}

BACKGROUND: Pancreatic carcinoma (PC) is one of the most aggressive
cancers affecting human health. It is essential to identify candidate
biomarkers for the diagnosis and prognosis of PC. The present study
aimed to investigate the diagnosis and prognosis biomarkers of PC.
METHODS: Differentially expressed genes (DEGs) were identified from the
mRNA expression profiles of GSE62452, GSE28735 and GSE16515. Functional
analysis and the protein-protein interaction network analysis was
performed to explore the biological function of the identified DEGs.
Diagnosis markers for PC were identified using ROC curve analysis.
Prognosis markers were identified via survival analysis of TCGA data.
The protein expression pattern of the identified genes was verified in
clinical tissue samples. A retrospective clinical study was performed to
evaluate the correlation between the expression of candidate proteins
and survival time of patients. Moreover, comprehensive analysis of the
combination of multiple genes/proteins for the prognosis prediction of
PC was performed using both TCGA data and clinical data. In vitro
studies were undertaken to elaborate the potential roles of these
biomarkers in clonability and invasion of PC cells. FINDINGS: In total,
389 DEGs were identified. These genes were mainly associated with
pancreatic secretion, protein digestion and absorption, cytochrome P450
drug metabolism, and energy metabolism pathway. The top 10 genes were
filtered out following Fisher's exact test. ROC curve analysis
demonstrated that TMPRSS4, SERPINB5, SLC6A14, SCEL, and TNS4 could be
used as biomarkers for the diagnosis of PC. Survival analysis of TCGA
data and clinical data suggested that TMC7, TMPRSS4, SCEL, SLC2A1,
CENPF, SERPINB5 and SLC6A14 can be potential biomarkers for the
prognosis of PC. Comprehensive analysis show that a combination of
identified genes/proteins can predict the prognosis of PC.
Mechanistically, the identified genes attributes to clonability and
invasiveness of PC cells. INTERPRETATION: We synthesized several sets of
public data and preliminarily clarified pathways and functions of PC.
Candidate molecular markers were identified for diagnosis and prognosis
prediction of PC including a novel gene, TMC7. Moreover, we found that
the combination of TMC7, TMPRSS4, SCEL, SLC2A1, CENPF, SERPINB5 and
SLC6A14 can serve as a promising indicator of the prognosis of PC
patients. The candidate proteins may attribute to clonability and
invasiveness of PC cells. This research provides a novel insight into
molecular mechanisms as well as diagnostic and prognostic markers of PC.
FUND: National Natural Science Foundation of China {[}No.~81602646
\&81802339{]}, Natural Science Foundation of Guangdong Province
{[}No.~2016A030310254{]} and China Postdoctoral Science Foundation
{[}No.~2016M600648{]}.

{}

{}

\begin{center}\rule{0.5\linewidth}{\linethickness}\end{center}

 - \textbf{Experimental microdissection enables functional harmonisation
of pancreatic cancer subtypes}

\emph{Gut 2019 Jan;():}

PubMed: \url{https://www.ncbi.nlm.nih.gov/pubmed/?term=30658994}

OBJECTIVE: Pancreatic ductal adenocarcinoma (PDA) has among the highest
stromal fractions of any cancer and this has complicated attempts at
expression-based molecular classification. The goal of this work is to
profile purified samples of human PDA epithelium and stroma and examine
their respective contributions to gene expression in bulk PDA samples.
DESIGN: We used laser capture microdissection (LCM) and RNA sequencing
to profile the expression of 60 matched pairs of human PDA malignant
epithelium and stroma samples. We then used these data to train a
computational model that allowed us to infer tissue composition and
generate virtual compartment-specific expression profiles from bulk gene
expression cohorts. RESULTS: Our analysis found significant variation in
the tissue composition of pancreatic tumours from different public
cohorts. Computational removal of stromal gene expression resulted in
the reclassification of some tumours, reconciling functional differences
between different cohorts. Furthermore, we established a novel
classification signature from a total of 110 purified human PDA stroma
samples, finding two groups that differ in the extracellular
matrix-associated and immune-associated processes. Lastly, a systematic
evaluation of cross-compartment subtypes spanning four patient cohorts
indicated partial dependence between epithelial and stromal molecular
subtypes. CONCLUSION: Our findings add clarity to the nature and number
of molecular subtypes in PDA, expand our understanding of global
transcriptional programmes in the stroma and harmonise the results of
molecular subtyping efforts across independent cohorts.

{}

{}

\begin{center}\rule{0.5\linewidth}{\linethickness}\end{center}

 - \textbf{Genetics of Familial and Sporadic Pancreatic Cancer}

\emph{Gastroenterology 2019 Jan;():}

PubMed: \url{https://www.ncbi.nlm.nih.gov/pubmed/?term=30660730}

In the previous decade, comprehensive genomic analyses have yielded
important insights about the genetic alterations that underlie
pancreatic tumorigenesis. Whole exome and whole genome sequencing of
pancreatic ductal adenocarcinomas have confirmed the critical driver
genes altered in the majority of pancreatic cancers, as well as
identifying numerous less frequently altered driver genes and has
delineated cancer subgroups with unique biological and clinical
features. It is now appreciated that pancreatic susceptibility gene
alterations are often identified in patients with pancreatic cancer
without family histories suggestive of a familial cancer syndrome,
prompting recent efforts to expand gene testing to all patients with
pancreatic cancer. Studies of pancreatic cancer precursor lesions have
begun to elucidate the evolutionary history of pancreatic tumorigenesis
and help to understand the utility of biomarkers for early detection as
well as targets to develop new therapeutic strategies. In this review,
we discuss the results of comprehensive genomic characterization of
pancreatic ductal adenocarcinoma and its precursor lesions, and we
highlight translational applications in early detection and therapy.

{}

{}

\begin{center}\rule{0.5\linewidth}{\linethickness}\end{center}

 - \textbf{Importance of gene expression signatures in pancreatic cancer
prognosis and the establishment of a prediction model}

\emph{Cancer management and research 2019 ;11():273-283}

PubMed: \url{https://www.ncbi.nlm.nih.gov/pubmed/?term=30643453}

Background and aim: Pancreatic cancer (PC) is one of the most common
tumors with a poor prognosis. The current American Joint Committee on
Cancer (AJCC) staging system, based on the anatomical features of
tumors, is insufficient to predict PC outcomes. The current study is
endeavored to identify important prognosis-related genes and build an
effective predictive model. Methods: Multiple public datasets were used
to identify differentially expressed genes (DEGs) and survival-related
genes (SRGs). Bioinformatics analysis of DEGs was used to identify the
main biological processes and pathways involved in PC. A risk score
based on SRGs was computed through a univariate Cox regression analysis.
The performance of the risk score in predicting PC prognosis was
evaluated with survival analysis, Harrell's concordance index (C-index),
area under the curve (AUC), and calibration plots. A predictive nomogram
was built through integrating the risk score with clinicopathological
information. Results: A total of 945 DEGs were identified in five Gene
Expression Omnibus datasets, and four SRGs (LYRM1, KNTC1, IGF2BP2, and
CDC6) were significantly associated with PC progression and prognosis in
four datasets. The risk score showed relatively good performance in
predicting prognosis in multiple datasets. The predictive nomogram had
greater C-index and AUC values, compared with those of the AJCC stage
and risk score. Conclusion: This study identified four new biomarkers
that are significantly associated with the carcinogenesis, progression,
and prognosis of PC, which may be helpful in studying the underlying
mechanism of PC carcinogenesis. The predictive nomogram showed robust
performance in predicting PC prognosis. Therefore, the current model may
provide an effective and reliable guide for prognosis assessment and
treatment decision-making in the clinic.

{}

{}

\begin{center}\rule{0.5\linewidth}{\linethickness}\end{center}

 - \textbf{LIMS1 Promotes Pancreatic Cancer Cell Survival Under
Oxygen-Glucose Deprivation Conditions by Enhancing HIF1A Protein
Translation}

\emph{Clinical cancer research : an official journal of the American
Association for Cancer Research 2019 Jan;():}

PubMed: \url{https://www.ncbi.nlm.nih.gov/pubmed/?term=30679163}

PURPOSE: Oxygen and glucose deprivation is a common feature of the solid
tumor. Regulatory network underlying the adaptation of cancer cells to
harsh microenvironment remains unclear. We determined the mechanistic
role of LIM and senescent cell antigen-like-containing domain protein 1
(LIMS1) in cancer cell survival under oxygen-glucose deprivation
conditions. EXPERIMENTAL DESIGN: The expression level of LIMS1 was
determined by immunohistochemical staining and analysing the mRNA
expression profiles from The Cancer Genome Atlas of three human solid
tumors. Roles of LIMS1 in cancer cell metabolism and growth were
determined by molecular and cell biology methods. A jetPEI nanocarrier
was used as vehicle for anti-LIMS1 siRNAs in mouse models of cancer
therapeutics. RESULTS: LIMS1 expression was drastically elevated in
PDAC. High LIMS1 level was associated with advanced TNM stage and poor
prognosis of tumour patients. Increased LIMS1 expression was pivotal for
tumour cells to survive in the oxygen-glucose deprivation conditions.
Mechanistically, LIMS1 enhanced GLUT1 expression and membrane
translocation, which facilitated tumor cell adaptation to the glucose
deprivation stress. Furthermore, LIMS1 promoted HIF1A protein
translation by activating AKT/mTOR signalling, while HIF1 transactivated
LIMS1 transcription, thus forming a positive feedback loop in PDAC cell
adaptation to oxygen deprivation stress. Inhibition of LIMS1 with jetPEI
nanocarrier-delivered anti-LIMS1 siRNAs significantly increased cell
death and suppressed tumour growth. CONCLUSIONS: LIMS1 promotes
pancreatic cancer cell survival under oxygen-glucose deprivation
conditions by activating AKT/mTOR signaling and enhancing HIF1A protein
translation. LIMS1 is crucial for tumor adaptation to oxygen-glucose
deprivation conditions and is a promising therapeutic target for cancer
treatment.

{}

{}

\begin{center}\rule{0.5\linewidth}{\linethickness}\end{center}

Back to top

\begin{center}\rule{0.5\linewidth}{\linethickness}\end{center}

\pagebreak

\hypertarget{pancreatitis-other-diseases-1}{%
\paragraph{Pancreatitis \& Other
Diseases}\label{pancreatitis-other-diseases-1}}

Molecular Studies on Pancreatitis \& Other Diseases

\begin{center}\rule{0.5\linewidth}{\linethickness}\end{center}

 - \textbf{Loss of TLR3 and its downstream signaling accelerates acinar
cell damage in the acute phase of pancreatitis}

\emph{Pancreatology : official journal of the International Association
of Pancreatology (IAP) \ldots{} {[}et al.{]} 2019 Jan;19(1):149-157}

PubMed: \url{https://www.ncbi.nlm.nih.gov/pubmed/?term=30583980}

BACKGROUND: Acute pancreatitis is accompanied by acinar cell damage
releasing potential toll-like receptor 3 (TLR3) ligands. So far, TLR3 is
known as a pattern recognition receptor in the immune signaling cascade
triggering a type I interferon response. In addition, TLR3 signaling
contributes to programmed cell death through the activation of caspase
8. However, the functional role of TLR3 and its downstream toll-like
receptor adaptor molecule 1 (TICAM1) in the inflamed pancreas is
unknown. METHODS: To uncover the role of TLR3 signaling in acute
pancreatitis, we induced a cerulein-mediated pancreatitis in Tlr3 and
Ticam1 knockout (KO) mice and in wildtype animals. The exocrine damage
was determined by blood serum analysis and histological examination.
Immunohistochemistry, gene expression and immunoblot analysis were
conducted to study TLR3 function. RESULTS: After the induction of an
acute pancreatitis, wildtype mice showed a high endosomal TLR3
expression in acinar cells. In comparison to wildtype and Ticam1 KO
mice, Tlr3 KO mice exhibited the highest severity of pancreatitis with
an increased NF-κB activation and elevated expression of the
pro-inflammatory cytokines Il6 and Tnf, although the amount of
infiltrating immune cells was unaffected. Additionally, we detected a
strong elevation of acinar cell necrosis and reduced levels of cleaved
caspase 8 in Tlr3 and Ticam1 KO mice. CONCLUSIONS: TLR3 and its
downstream adaptor TICAM1 are important mediators of acinar cell damage
in acute pancreatitis. They possess a critical role in programmed cell
death and our data suggest that TLR3 signaling controls the onset and
severity of acute pancreatitis.

{}

{}

\begin{center}\rule{0.5\linewidth}{\linethickness}\end{center}

Back to top

\begin{center}\rule{0.5\linewidth}{\linethickness}\end{center}

\pagebreak

\hypertarget{molecular-research-on-microenvironment}{%
\paragraph{Molecular Research on
Microenvironment}\label{molecular-research-on-microenvironment}}

Tumor Stroma Interactions, Microenvironment, Inflammatory Response,
Microbiome

\begin{center}\rule{0.5\linewidth}{\linethickness}\end{center}

 - \textbf{Stratification of Pancreatic Ductal Adenocarcinomas Based on
Tumor and Microenvironment Features}

\emph{Gastroenterology 2018 12;155(6):1999-2013.e3}

PubMed: \url{https://www.ncbi.nlm.nih.gov/pubmed/?term=30165049}

BACKGROUND \& AIMS: Genomic studies have revealed subtypes of pancreatic
ductal adenocarcinoma (PDA) based on their molecular features, but
different studies have reported different classification systems. It is
a challenge to obtain high-quality, freshly frozen tissue for clinical
analysis and determination of PDA subtypes. We aimed to redefine
subtypes of PDA using a large number of formalin-fixed and
paraffin-embedded PDA samples, which are more amenable to routine
clinical evaluation. METHODS: We collected PDA samples from 309
consecutive patients who underwent surgery from September 1996 through
December 2010 at 4 academic hospitals in Europe; nontumor tissue samples
were not included. Samples were formalin fixed and paraffin embedded.
DNA and RNA were isolated; gene expression, targeted DNA sequencing, and
immunohistochemical analyses were performed. We used independent
component analysis to deconvolute normal, tumor, and microenvironment
transcriptome patterns in samples. We devised classification systems
from an unsupervised analysis using a consensus clustering approach of
our data set after removing normal contamination components. We
associated subtypes with overall survival and disease-free survival of
patients using Cox proportional hazards regression with estimation of
hazard ratios and 95\% confidence interval. We used The Cancer Genome
Consortium and International Cancer Genome Consortium PDA data sets as
validation cohorts. RESULTS: We validated the previously reported
basal-like and classical tumor-specific subtypes of PDAs. We identified
features of the PDA, including microenvironment gene expression
patterns, that allowed tumors to be categorized into 5 subtypes, called
pure basal like, stroma activated, desmoplastic, pure classical, and
immune classical. These PDA subtypes have features of cancer cells and
immune cells that could be targeted by pharmacologic agents. Tumor
subtypes were associated with patient outcomes, based on analysis of our
data set and the International Cancer Genome Consortium and The Cancer
Genome Consortium PDA data sets. We also observed an exocrine signal
associated with acinar cell contamination (from pancreatic tissue).
CONCLUSIONS: We identified a classification system based on gene
expression analysis of formalin-fixed PDA samples. We identified 5 PDA
subtypes, based on features of cancer cells and the tumor
microenvironment. This system might be used to select therapies and
predict patient outcomes. We found evidence that the previously reported
exocrine-like (called ADEX) tumor subtype resulted from contamination
with pancreatic acinar cells. ArrayExpress accession number:
E-MTAB-6134.

{}

{}

\begin{center}\rule{0.5\linewidth}{\linethickness}\end{center}

 - \textbf{Increase of Tumor Infiltrating γδ T-cells in Pancreatic
Ductal Adenocarcinoma Through Remodeling of the Extracellular Matrix by
a Hyaluronan Synthesis Suppressor, 4-Methylumbelliferone}

\emph{Pancreas 2019 Feb;48(2):292-298}

PubMed: \url{https://www.ncbi.nlm.nih.gov/pubmed/?term=30589828}

OBJECTIVES: Desmoplastic changes of extracellular matrix (ECM)
containing large amounts of hyaluronan (HA) are of interest in chemo-
and immunoresistance of pancreatic ductal adenocarcinoma (PDAC). The
goal of this study was to evaluate the effects of 4-methylumbelliferone
(MU), a selective inhibitor of HA, on ECM and to examine how MU affects
adoptive immunotherapy. METHODS: The effect of MU on cell proliferation,
HA synthesis and formation of ECM were investigated in four PDAC cell
lines. In addition, the cytotoxicity of γδ T-cell-rich peripheral blood
mononuclear cells (PBMCs) collected from healthy donors and stimulated
with zoledronate and interleukin-2 was examined in the presence of MU.
The amount of HA and tumor-infiltrating lymphocytes were also
investigated in mice xenograft models. RESULTS: In vitro, 1.0 mM MU
inhibited cell proliferation by 45-70\% and HA synthesis by 55-80\% in
all four PDAC cell lines, and enhanced γδ T-cell-rich PBMC-mediated
cytotoxicity against PDAC cells. In vivo, MU reduced intratumoral HA and
promoted infiltration of inoculated γδ T-cells into tumor tissue, and
consequently suppressed tumor growth. CONCLUSIONS: 4-methylumbelliferone
may be an effective immunosensitizer against PDAC through induction of
structural changes in the ECM.

{}

{}

\begin{center}\rule{0.5\linewidth}{\linethickness}\end{center}

 - \textbf{Evaluating the Regulatory Immunomodulation Effect of
Irreversible Electroporation (IRE) in Pancreatic Adenocarcinoma}

\emph{Annals of surgical oncology 2019 Mar;26(3):800-806}

PubMed: \url{https://www.ncbi.nlm.nih.gov/pubmed/?term=30610562}

BACKGROUND: Irreversible electroporation (IRE) has been demonstrated as
an effective local method for locally advanced (stage 3) pancreatic
adenocarcinoma. Immune regulatory T cells (Tregs) induce
immunosuppression of tumors by inhibiting patients' anti-tumor adaptive
immune response. This study aimed to evaluate the immunomodulation
effect of IRE to identify an ideal time point for potential adjuvant
immunotherapy. METHODS: This study prospectively evaluated an
institutional review board-approved study of patients undergoing either
in situ IRE or pancreatectomy. Patient blood samples were collected at
different time points (before surgery {[}preOP{]} and on postoperative
day {[}POD{]} 1, POD3, and POD5). Peripheral blood mononuclear cells
(PBMCs) were isolated and evaluated for three different CD4 + Treg
subsets (CD25 + CD4 +, CD4 + CD25 + FoxP3 +, CD4 + CD25 + FoxP3 -) by
flow cytometry and analyzed for median fold change (MFC) between each
two consecutive time points (MFC = log2(T2/T1)). RESULTS: The study
analyzed 15 patients with in situ IRE (n = 11) or pancreatectomy (PAN)
(n = 4). In both groups, CD25 + CD4 + Tregs decreased on POD1 followed
by a steady increase in pancreatectomy, whereas the trend in the IRE
group reversed between D3 and D5 (MFC: IRE {[}- 0.01{]}, PAN
{[}+ 0.39{]}). For each period, CD4 + CD25 + FoxP3 + Tregs showed the
most dramatic inverse effect, with D3 to D5 showing the most change
(MFC: IRE {[}- 0.18{]}, PAN {[}+ 0.39{]}). Also, CD4 + CD25 + FoxP3 -
Tregs showed an inverse effect between D3 and D5 (MFC: IRE {[}- 0.25{]},
PAN {[}+ 0.49{]}). Altogether, the Treg trend was inversely affected by
the in situ IRE procedure, with the greatest cumulative significant
change for all three Treg subsets between D3 and D5 (MFC ± SEM: IRE
{[}- 0.24 ± 0.05{]}, PAN {[}+ 0.37 ± 0.02{]}; p = 0.016). CONCLUSIONS:
The study data suggest that in situ IRE procedure-mediated Treg
attenuation between POD3 and POD5 can provide a clinical window of
opportunity for potentiating clinical efficacy in combination with
immunotherapy.

{}

{}

\begin{center}\rule{0.5\linewidth}{\linethickness}\end{center}

 - \textbf{Functions of the CXC ligand family in the pancreatic tumor
microenvironment}

\emph{Pancreatology : official journal of the International Association
of Pancreatology (IAP) \ldots{} {[}et al.{]} 2018 Oct;18(7):705-716}

PubMed: \url{https://www.ncbi.nlm.nih.gov/pubmed/?term=30078614}

Therapeutic resistance is the major contributor to the poor prognosis of
and low survival from pancreatic cancer (PC). Cancer progression is a
complex process reliant on interactions between the tumor and the tumor
microenvironment (TME). Members of the CXCL family of chemokines are
present in the pancreatic TME and seem to play a vital role in
regulating PC progression. As pancreatic tumors interact with the TME
and with PC stem cells (CSCs), determining the roles of specific members
of the CXCL family is vital to the development of improved therapies.
This review highlights the roles of selected CXCLs in the interactions
between pancreatic tumor and its stroma, and in CSC phenotypes, which
can be used to identify potential treatment targets.

{}

{}

\begin{center}\rule{0.5\linewidth}{\linethickness}\end{center}

 - \textbf{Comparison of immune infiltrates in melanoma and pancreatic
cancer highlights VISTA as a potential target in pancreatic cancer}

\emph{Proceedings of the National Academy of Sciences of the United
States of America 2019 Jan;116(5):1692-1697}

PubMed: \url{https://www.ncbi.nlm.nih.gov/pubmed/?term=30635425}

Immune checkpoint therapy (ICT) has transformed cancer treatment in
recent years; however, treatment response is not uniform across tumor
types. The tumor immune microenvironment plays a critical role in
determining response to ICT; therefore, understanding the differential
immune infiltration between ICT-sensitive and ICT-resistant tumor types
will help to develop effective treatment strategies. We performed a
comprehensive analysis of the immune tumor microenvironment of an
ICT-sensitive tumor (melanoma, n = 44) and an ICT-resistant tumor
(pancreatic cancer, n = 67). We found that a pancreatic tumor has
minimal to moderate infiltration of CD3, CD4, and CD8 T cells; however,
the immune infiltrates are predominantly present in the stromal area of
the tumor and are excluded from tumoral area compared with melanoma,
where the immune infiltrates are primarily present in the tumoral area.
Metastatic pancreatic ductal adenocarcinomas (PDACs) had a lower
infiltration of total T cells compared with resectable primary PDACs,
suggesting that metastatic PDACs have poor immunogenicity. Further, a
significantly higher number of CD68+ macrophages and VISTA+ cells (also
known as V-domain immunoglobulin suppressor of T cell activation) were
found in the pancreatic stromal area compared with melanoma. We
identified VISTA as a potent inhibitory checkpoint that is predominantly
expressed on CD68+ macrophages on PDACs. These data suggest that VISTA
may be a relevant immunotherapy target for effective treatment of
patients with pancreatic cancer.

{}

{}

\begin{center}\rule{0.5\linewidth}{\linethickness}\end{center}

Back to top

\begin{center}\rule{0.5\linewidth}{\linethickness}\end{center}

\pagebreak

\hypertarget{preneoplastic-and-preinvasive-lesions-1}{%
\paragraph{Preneoplastic and Preinvasive
Lesions}\label{preneoplastic-and-preinvasive-lesions-1}}

Molecular Pathology Preneoplastic and Preinvasive Lesions, PanIN, IPMN,
MCN, ICPN

\begin{center}\rule{0.5\linewidth}{\linethickness}\end{center}

 - \textbf{Cancer-associated acinar-to-ductal metaplasia within the
invasive front of pancreatic cancer contributes to local invasion}

\emph{Cancer letters 2019 Mar;444():70-81}

PubMed: \url{https://www.ncbi.nlm.nih.gov/pubmed/?term=30590101}

The pancreas is an organ prone to inflammation, fibrosis, and atrophy
because of an abundance of acinar cells that produce digestive enzymes.
A characteristic of pancreatic cancer is the presence of desmoplasia,
inflammatory cell infiltration, and cancer-associated acinar atrophy
(CAA) within the invasive front. CAA is characterized by a high
frequency of small ducts and resembles acinar-to-ductal metaplasia
(ADM). However, the clinical significance of changes in acinar
morphology, such as ADM with acinar atrophy, within the tumor
microenvironment remains unclear. Here, we find that ADM within the
invasive front of tumors is associated with cell invasion and
desmoplasia in an orthotopic mouse model of pancreatic cancer. An
analysis of resected human tumors revealed that regions of
cancer-associated ADM were positive for TGFα, and that this TGFα
expression was associated with primary tumor size and shorter survival
times. Gene expression analysis identified distinct phenotypic profiles
for cancer-associated ADM, sporadic ADM and chronic pancreatitis ADM.
These findings suggest that the mechanisms driving ADM differ according
to the specific tissue microenvironment and that cancer-associated ADM
and acinar atrophy contribute to tumor cell invasion of the local
pancreatic parenchyma.

{}

{}

\begin{center}\rule{0.5\linewidth}{\linethickness}\end{center}

 - \textbf{No Cell Left Unturned: Intraductal Papillary Mucinous
Neoplasm Heterogeneity}

\emph{Clinical cancer research : an official journal of the American
Association for Cancer Research 2019 Jan;():}

PubMed: \url{https://www.ncbi.nlm.nih.gov/pubmed/?term=30642914}

Intraductal papillary mucinous neoplasm (IPMN) is a pancreatic cancer
precursor lesion with established genetic features, but the cellular
ecosystem of these tumors remains to be fully characterized. This study
utilizes single cell RNA-seq to describe the dynamic landscape of
epithelial, immune, and stromal cells during IPMN progression to
invasive cancer.

{}

{}

\begin{center}\rule{0.5\linewidth}{\linethickness}\end{center}

Back to top

\begin{center}\rule{0.5\linewidth}{\linethickness}\end{center}

\pagebreak

\hypertarget{bile-ducts-2}{%
\subsubsection{Bile Ducts}\label{bile-ducts-2}}

\begin{center}\rule{0.5\linewidth}{\linethickness}\end{center}

 - \textbf{DLEC1 methylation is associated with a better clinical
outcome in patients with intrahepatic cholangiocarcinoma of the small
duct subtype}

\emph{Virchows Archiv : an international journal of pathology 2019
Jan;():}

PubMed: \url{https://www.ncbi.nlm.nih.gov/pubmed/?term=30610381}

Intrahepatic cholangiocarcinoma is a complex disease with three
different histologic subtypes, the large duct, small duct, and bile
ductular types. In the present study, we elucidated whether the three
histological subtypes have differences in their methylation profiles and
developed a DNA methylation marker that might help identify a subset of
ICC with a different prognosis. We screened 113 promoter CpG island loci
against 10 cases of intrahepatic cholangiocarcinoma and normal cystic
ducts using the MethyLight assay and selected 30 CpG island loci with
cancer-associated hypermethylation. Then, we analyzed 172 intrahepatic
cholangiocarcinomas for the methylation state at these 30 loci. Six
loci, including DLEC1, were more frequently methylated in the bile
ductular type and small duct type, whereas six loci were more frequently
methylated in the large duct type. Of these 30 loci, DLEC1 methylation
was found mainly in the bile ductular type and small duct type but
rarely in the large duct type. DLEC1 methylation was significantly
associated with a better clinical outcome in intrahepatic
cholangiocarcinomas of the small duct type but not of the bile ductular
type. DLEC1 methylation was an independent prognostic variable in both
cancer-specific survival and recurrence-free survival. For patients with
intrahepatic cholangiocarcinoma of the small duct type (n = 68), DLEC1
methylation was found in 26 (38.2\%) and was associated with a better
clinical outcome for both cancer-specific survival and recurrence-free
survival. Our findings suggest that DLEC1 methylation can be utilized to
identify a subset with a better prognosis in intrahepatic
cholangiocarcinomas of the small duct type.

{}

{}

\begin{center}\rule{0.5\linewidth}{\linethickness}\end{center}

 - \textbf{EVI1 expression is associated with aggressive behavior in
intrahepatic cholangiocarcinoma}

\emph{Virchows Archiv : an international journal of pathology 2019
Jan;474(1):39-46}

PubMed: \url{https://www.ncbi.nlm.nih.gov/pubmed/?term=30349952}

Ecotropic virus integration site 1 protein homolog (EVI1), a well-known
oncogenic transcriptional factor of hematopoietic cells, contributes to
pancreatic cancer oncogenicity through increased expression of KRAS.
Because EVI1 was upregulated in cholangiocarcinoma by referring The
Cancer Genome Atlas, we investigated the importance of EVI1 in
intrahepatic cholangiocarcinoma (ICC) which has been regarded as a
heterogeneous group of cancers. Immunohistochemical analysis results
demonstrated that EVI1 was overexpressed in about half of ICC (53/101,
52.5\%). Moreover, all intraductal papillary neoplasms of the bile duct
cases expressed EVI1 regardless of histological grading and subtypes
such as gastric, intestinal, pancreatobiliary, or oncocytic (20/20,
100\%). EVI1-positive ICC showed higher frequencies of aggressive
pathological indicators such as periductal infiltrative growth
(p = 0.022), hilar invasion (p = 0.041), advanced UICC stage
(p = 0.026), major vascular invasion (p = 0.026), and perineural
invasion (p = 0.007) than EVI1-negative ICC. Patients with EVI1-positive
ICC showed worse overall survival and recurrence-free survival in all
resected cases and in curative resected cases. Recently, we proposed
type 1/2 (large/small duct types) classification of ICC based on mucin
productivity and immunophenotypes (S100P, N-cadherin, and NCAM). Type 1
predominantly consisted of EVI1-positive ICC (33/42 cases, 79\%), and
the frequency was significantly higher than type 2 (18/55 cases, 32.7\%)
(p \textless{} 0.0001). EVI1-positive ICC was likely to express
stomach-specific claudin CLDN18 (correlation coefficient r = 0.55373)
and mucin MUC5AC (r = 0.42718). EVI1-positive ICC is an aggressive ICC
showing both large-duct and/or gastric phenotypes. Consequently, a
transcriptional factor EVI1 is associated with aggressive behavior in
ICC and can be a therapeutic target molecule, while EVI1 might be a key
molecule for the development of intraductal papillary neoplasms of the
bile duct.

{}

{}

\begin{center}\rule{0.5\linewidth}{\linethickness}\end{center}

 - \textbf{Prognostic role of BAP-1 and PBRM-1 expression in
intrahepatic cholangiocarcinoma}

\emph{Virchows Archiv : an international journal of pathology 2019
Jan;474(1):29-37}

PubMed: \url{https://www.ncbi.nlm.nih.gov/pubmed/?term=30377796}

Intrahepatic cholangiocarcinoma (ICC) has universally poor outcome,
mainly due to its late clinical presentation. Identification of specific
biomarkers and development of effective treatment are still urgently
required. Mutations in PBRM-1 and BAP-1 genes, and the expression of
S100P have been related to survival in ICC. miR-31 seems also to play
important regulatory functions in ICC and it directly regulates BAP-1
expression in lung cancer. In this study, tissue expression of BAP-1,
PBRM-1, S100P, and miR-31 was investigated in ICC and correlated with
clinical-pathological features. Sixty-one consecutive patients who
underwent curative hepatic resection for ICC were enrolled. None
received any therapy prior to surgery. Immunostaining for BAP-1, PBRM-1,
and S100P, and in situ hybridization for miR-31 were performed, using
tissue microarray slides. A strong retained expression of BAP-1 and
PBRM-1 was associated with a reduced overall (p = 0.04 and p = 0.002,
respectively) and disease-free survival (p = 0.05 and p = 0.02,
respectively). An overexpression of S100P was related to a reduced
overall survival (p = 0.005). The multivariate analyses identified the
presence of perineural invasion and the retained PBRM-1 expression as
independent predictors of worse overall {[}p = 0.02, hazard ratio
(HR) = 2.25 (1.16-4.39) and p = 0.001, HR = 3.13 (1.56-6.28),
respectively{]} and disease-free survivals {[}p = 0.03, HR = 2.43
(1.09-5.4) and p = 0.03, HR = 2.51 (1.11-5.67), respectively{]}. An
overexpression of S100P was predictive of a worse overall survival
{[}p = 0.02, HR = 1.66 (1.08-2.55){]}. High levels of miR-31 were
significantly associated to a low expression of BAP-1 protein
(p = 0.03). In ICC, a retained expression of BAP-1 and PBRM-1, and an
overexpression of S100P are related to a poor prognosis.

{}

{}

\begin{center}\rule{0.5\linewidth}{\linethickness}\end{center}

Back to top

\begin{center}\rule{0.5\linewidth}{\linethickness}\end{center}

\pagebreak

\hypertarget{gallbladder-1}{%
\subsubsection{Gallbladder}\label{gallbladder-1}}

\begin{center}\rule{0.5\linewidth}{\linethickness}\end{center}

 - \textbf{MiR-1-5p is down-regulated in gallbladder carcinoma and
suppresses cell proliferation, migration and invasion by targeting
Notch2}

\emph{Pathology, research and practice 2019 Jan;215(1):200-208}

PubMed: \url{https://www.ncbi.nlm.nih.gov/pubmed/?term=30497876}

BACKGROUND: Numerous studies have demonstrated that aberrant microRNAs
(miRNAs) are involved in tumorigenesis and tumor progression.
Nevertheless, the precise role of miR-1-5p in gallbladder carcinoma cell
growth and metastasis remains not fully revealed. MATERIAL AND METHODS:
The levels of miR-1-5p were detected in gallbladder carcinoma tissues
and cell lines using qRT-PCR method. A series of functional assays,
including cell proliferation, colony formation, wound healing and
Transwell invasion were conducted using miR-1-5p or miR-1-5p inhibitor
transfected cells. RESULTS: MiR-1-5p was remarkably down-regulated in
gallbladder carcinoma tissues and cell lines compared to normal. In
addition, over-expression of miR-1-5p markedly suppressed the growth,
migration and invasion of gallbladder carcinoma cell. Conversely,
down-expression of miR-1-5p facilitated gallbladder carcinoma cell
proliferation and aggressiveness. Mechanistic investigations
demonstrated that neurogenic locus notch homolog protein 2 (Notch2) was
the directly target of miR-1-5p and Notch2 mediated the inhibitory
effect of miR-1-5p in gallbladder carcinoma cell growth and
aggressiveness. CONCLUSION: Our findings demonstrated that miR-1-5p
acted as a suppressive miRNA and played vital roles in the growth,
migration and invasion of gallbladder carcinoma cell through targeting
Notch2.

{}

{}

\begin{center}\rule{0.5\linewidth}{\linethickness}\end{center}

 - \textbf{Roles of Pin1 as a Key Molecule for EMT Induction by
Activation of STAT3 and NF-κB in Human Gallbladder Cancer}

\emph{Annals of surgical oncology 2019 Mar;26(3):907-917}

PubMed: \url{https://www.ncbi.nlm.nih.gov/pubmed/?term=30610554}

BACKGROUND: Despite developments in multidisciplinary treatment, the
prognosis for advanced gallbladder cancer (GBC) still is poor because of
its rapid progression. Epithelial-mesenchymal transition (EMT) plays a
central role in promoting tumor invasion and metastasis in malignancies
thorough signal transducer and activator of transcription-3 (STAT3) and
nuclear factor κB (NF-κB) activation. Whereas Pin1 mediates STAT3 and
NF-κB activation, the involvement of Pin1 in GBC progression is unclear.
METHODS: Factors regulating Pin1-related STAT3 and NF-κB activation were
evaluated using surgical specimens collected from 76 GBC patients, GBC
cells, and orthotopic GBC xenograft mice. RESULTS: In the patients with
GBC, high Pin1 expression in GBC was associated with aggressive tumor
invasion and increased tumor metastasis, and was an independent factor
for a poor prognosis. Pin1 expression was correlated with
phosphorylation of STAT3(Ser727) and NF-κB-p65(Ser276), thereby
activating STAT3 and NF-κB in GBC. Pin1-mediated STAT3 and NF-κB
activation induced EMT in GBC. When Pin1 knockdown was performed in GBC
cells, the phosphorylation of STAT3(Ser727) and NF-κB-p65(Ser276) was
inhibited, and STAT3 and NF-κB activation was suppressed. Inactivation
of STAT3 and NF-κB in Pin1-depleted cells decreased snail and zeb-2
expression, thereby reducing the rate of mesenchymal-like cells,
suggesting that EMT was inhibited in GBC cells. PiB, a Pin1-specific
inhibitor, inhibited EMT and reduced tumor cell invasion by inactivating
STAT3 and NF-κB in vitro. Moreover, PiB treatment inhibited lymph node
metastasis and intrahepatic metastasis in orthotopic GBC xenograft tumor
in vivo. CONCLUSIONS: Pin1 accelerates GBC invasion and metastasis by
activating STAT3 and NF-κB. Therefore, Pin1 inhibition by PiB is an
excellent therapy for GBC by safely inhibiting its metastasis.

{}

{}

\begin{center}\rule{0.5\linewidth}{\linethickness}\end{center}

Back to top

\begin{center}\rule{0.5\linewidth}{\linethickness}\end{center}

\pagebreak

\hypertarget{ampulla-of-vater-1}{%
\subsubsection{Ampulla of Vater}\label{ampulla-of-vater-1}}

\begin{center}\rule{0.5\linewidth}{\linethickness}\end{center}

 - \textbf{Ampullary cancer: Evaluation of somatic and germline genetic
alterations and association with clinical outcomes}

\emph{Cancer 2019 Jan;():}

PubMed: \url{https://www.ncbi.nlm.nih.gov/pubmed/?term=30620386}

BACKGROUND: Ampullary carcinoma (AC) is a rare gastrointestinal cancer.
Pathogenic germline alterations (PGAs) in BRCA2 and potentially
targetable somatic alterations (SAs) in ERBB2 and ELF3 have been
previously described in AC. Memorial Sloan Kettering Cancer Center has
implemented an opt-in strategy for germline testing (GT) and somatic
testing (ST) for patients with AC to further evaluate the spectrum of
PGAs and SAs. METHODS: Forty-five patients with pathologically confirmed
AC prospectively consented with the Memorial Sloan Kettering Integrated
Mutation Profiling of Actionable Cancer Targets (MSK-IMPACT) test
(410-468 genes). A subset of the cohort (23 of the 45 patients) also
consented to GT with MSK-IMPACT (76-88 genes). Germline data for 21 of
the remaining 22 patients who had not consented to GT were obtained in a
de-identified fashion without clinical correlation. Clinicopathologic
features, treatment histories, and survival data for consenting patients
were collected and analyzed. RESULTS: Pancreaticobiliary, intestinal,
and mixed features of the 2 types were the primary pathologic subtypes
of AC identified in this cohort. No difference in median overall
survival was found between pathologic subtypes. Eight of 44 patients
(18\%) were identified as harboring pathogenic mutations in BRCA2, ATM,
RAD50, and MUTYH. In addition, this study found a wide spectrum of SAs
in genes such as KRAS, MDM2, ERBB2, ELF3, and PIK3CA. Two patients in
the cohort underwent SA-targeted therapy, and 1 had a partial
radiographic response. CONCLUSIONS: Mutations in multiple somatic and
germline genes were identified in this cohort. Significantly, actionable
targets were identified in the tumors, and broader testing for PGAs and
SAs should be considered for all patients with AC.

{}

{}

\begin{center}\rule{0.5\linewidth}{\linethickness}\end{center}

Back to top

\begin{center}\rule{0.5\linewidth}{\linethickness}\end{center}

\pagebreak

\hypertarget{neuroendocrine-2}{%
\subsubsection{Neuroendocrine}\label{neuroendocrine-2}}

\begin{center}\rule{0.5\linewidth}{\linethickness}\end{center}

 - \textbf{The Molecular and Clinical Landscape of Pancreatic
Neuroendocrine Tumors}

\emph{Pancreas 2019 Jan;48(1):9-21}

PubMed: \url{https://www.ncbi.nlm.nih.gov/pubmed/?term=30531241}

Pancreatic neuroendocrine tumors are rare tumors of the pancreas
originating from the islets of the Langerhans. These tumors comprise 1\%
to 3\% of all newly diagnosed pancreatic cancers every year and have a
unique heterogeneity in clinical presentation. Whole-genome sequencing
has led to an increased understanding of the molecular biology of these
tumors. In this review, we will summarize the current knowledge of the
signaling pathways involved in the tumorigenesis of pancreatic
neuroendocrine tumors as well as the major studies targeting these
pathways at preclinical and clinical levels.

{}

{}

\begin{center}\rule{0.5\linewidth}{\linethickness}\end{center}

 - \textbf{Distinct genome-wide methylation patterns in sporadic and
hereditary nonfunctioning pancreatic neuroendocrine tumors}

\emph{Cancer 2019 Jan;():}

PubMed: \url{https://www.ncbi.nlm.nih.gov/pubmed/?term=30620390}

BACKGROUND: Aberrant methylation is a known cause of cancer initiation
and/or progression. There are scant data on the genome-wide methylation
pattern of nonfunctioning pancreatic neuroendocrine tumors (NFPanNETs)
and sporadic and hereditary NFPanNETs. METHODS: Thirty-three tissue
samples were analyzed: they included samples from sporadic (n~=~9), von
Hippel-Lindau (VHL)-related (n~=~10), and multiple endocrine neoplasia
type 1 (MEN1)-related NFPanNETs (n~=~10) as well as normal islet cells
(n~=~4) for comparison. Genome-wide CpG methylation profiling was
performed with the Infinium MethylationEPIC BeadChip assay and was
analyzed with R-based tools. RESULTS: In unsupervised hierarchical
clustering, sporadic and MEN1-related NFPanNETs clustered together, and
the VHL group was in a separate cluster. MEN1-related NFPanNETs had a
higher rate of hypermethylated CpG sites in comparison with sporadic and
VHL-related tumor groups. Differentially methylated region analysis
confirmed the higher rate of hypermethylation in MEN1-related tumors.
Moreover, in an integrated analysis of gene expression data for the same
tumor samples, downregulated gene expression was found in most genes
that were hypermethylated. In a CpG island methylator phenotype
analysis, 3 genes were identified and confirmed to have downregulated
gene expression: secreted frizzle-related protein 5 (SFRP5) in sporadic
NFPanNETs and cell division cycle-associated 7-like (CDCA7L) and RNA
binding motif 47 (RBM47) in MEN1-related NFPanNETs. CONCLUSIONS: MEN1
NFPanNETs have a higher rate of geno me-wide hypermethylation than other
NFPanNET subtypes. The similarity between the pathways enriched in a
methylation analysis of known genes involved in NFPanNET tumorigenesis
suggests a key role for aberrant methylation in the pathogenesis of
NFPanNETs.

{}

{}

\begin{center}\rule{0.5\linewidth}{\linethickness}\end{center}

Back to top

\begin{center}\rule{0.5\linewidth}{\linethickness}\end{center}

\pagebreak

\hypertarget{journals-reviewed}{%
\subsection{Journals Reviewed}\label{journals-reviewed}}

\begin{center}\rule{0.5\linewidth}{\linethickness}\end{center}

Advances in Anatomic Pathology\\
American Journal of Clinical Pathology\\
The American Journal of Gastroenterology\\
The American Journal of Pathology\\
American Journal of Surgical Pathology\\
Annals of Diagnostic Pathology\\
Annals of Surgery\\
Annals of Surgical Oncology\\
Annual Review of Pathology-Mechanisms of Disease\\
APMIS\\
Applied Immunohistochemistry \& Molecular Morphology\\
Archives of Pathology \& Laboratory Medicine\\
Cancer\\
Cancer Cell\\
Cancer Cytopathology\\
Cell\\
Clinical Cancer Research\\
Cytojournal\\
Cytopathology\\
Diagnostic Cytopathology\\
Diagnostic Pathology\\
Endocrine Pathology\\
Experimental and Molecular Pathology\\
Expert Review of Molecular Diagnostics\\
Gastroenterology\\
Gut\\
Histology and Histopathology\\
Histopathology\\
Human Pathology\\
International Journal of Surgical Pathology\\
International Journal of Clinical and Experimental Pathology\\
Journal of Clinical Pathology\\
Journal of Molecular Diagnostics\\
Journal of Pathology\\
Laboratory investigation\\
Lancet\\
Medical Molecular Morphology\\
Modern Pathology\\
Nature\\
Nature Reviews Gastroenterology \& Hepatology\\
NEJM\\
Pancreas\\
Pancreatology\\
Pathobiology\\
Pathologie Biologie\\
Pathology\\
Pathology \& Oncology Research\\
Pathology International\\
Pathology Research and Practice\\
PNAS\\
Seminars in Diagnostic Pathology\\
Seminars in Immunopathology\\
Surgical pathology clinics\\
Virchows Archiv

\begin{center}\rule{0.5\linewidth}{\linethickness}\end{center}

Back to top

\begin{center}\rule{0.5\linewidth}{\linethickness}\end{center}

\pagebreak

\hypertarget{feedback}{%
\subsection{Feedback}\label{feedback}}

\begin{center}\rule{0.5\linewidth}{\linethickness}\end{center}

Please send your feedbacks using the form below:

\href{https://docs.google.com/forms/d/e/1FAIpQLSeD3Z9J6Y7eMmiyM12f_SfAmHUlykb1zxZcwO6lg7cebGYQIQ/viewform}{Click
here for the Feedback Form}

\begin{center}\rule{0.5\linewidth}{\linethickness}\end{center}

\hypertarget{disqus_thread}{}

Please enable JavaScript to view the comments powered by Disqus.

\begin{center}\rule{0.5\linewidth}{\linethickness}\end{center}

Back to top

\begin{center}\rule{0.5\linewidth}{\linethickness}\end{center}

\pagebreak

\hypertarget{archive}{%
\subsection{Archive}\label{archive}}

\begin{center}\rule{0.5\linewidth}{\linethickness}\end{center}

\textbf{The PBPath Journal Watch Archive}

\begin{itemize}
\item
  \href{http://pbpath.org/journal-watch/}{Current Issue}
\item
  \href{http://pbpath.org/journal-watch-archive/}{Older Issues}
\item
  \href{http://pbpath.org/journal-watch-upcoming-issue/}{Upcoming Issue,
  Unofficial version}
\end{itemize}

\begin{center}\rule{0.5\linewidth}{\linethickness}\end{center}

Back to top

\begin{center}\rule{0.5\linewidth}{\linethickness}\end{center}

\pagebreak


\end{document}
