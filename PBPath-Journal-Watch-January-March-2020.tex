% Options for packages loaded elsewhere
\PassOptionsToPackage{unicode}{hyperref}
\PassOptionsToPackage{hyphens}{url}
%
\documentclass[
]{article}
\usepackage{lmodern}
\usepackage{amssymb,amsmath}
\usepackage{ifxetex,ifluatex}
\ifnum 0\ifxetex 1\fi\ifluatex 1\fi=0 % if pdftex
  \usepackage[T1]{fontenc}
  \usepackage[utf8]{inputenc}
  \usepackage{textcomp} % provide euro and other symbols
\else % if luatex or xetex
  \usepackage{unicode-math}
  \defaultfontfeatures{Scale=MatchLowercase}
  \defaultfontfeatures[\rmfamily]{Ligatures=TeX,Scale=1}
\fi
% Use upquote if available, for straight quotes in verbatim environments
\IfFileExists{upquote.sty}{\usepackage{upquote}}{}
\IfFileExists{microtype.sty}{% use microtype if available
  \usepackage[]{microtype}
  \UseMicrotypeSet[protrusion]{basicmath} % disable protrusion for tt fonts
}{}
\makeatletter
\@ifundefined{KOMAClassName}{% if non-KOMA class
  \IfFileExists{parskip.sty}{%
    \usepackage{parskip}
  }{% else
    \setlength{\parindent}{0pt}
    \setlength{\parskip}{6pt plus 2pt minus 1pt}}
}{% if KOMA class
  \KOMAoptions{parskip=half}}
\makeatother
\usepackage{xcolor}
\IfFileExists{xurl.sty}{\usepackage{xurl}}{} % add URL line breaks if available
\IfFileExists{bookmark.sty}{\usepackage{bookmark}}{\usepackage{hyperref}}
\hypersetup{
  pdftitle={Pancreatobiliary Pathology Society Journal Watch},
  hidelinks,
  pdfcreator={LaTeX via pandoc}}
\urlstyle{same} % disable monospaced font for URLs
\usepackage[margin=1in]{geometry}
\usepackage{graphicx}
\makeatletter
\def\maxwidth{\ifdim\Gin@nat@width>\linewidth\linewidth\else\Gin@nat@width\fi}
\def\maxheight{\ifdim\Gin@nat@height>\textheight\textheight\else\Gin@nat@height\fi}
\makeatother
% Scale images if necessary, so that they will not overflow the page
% margins by default, and it is still possible to overwrite the defaults
% using explicit options in \includegraphics[width, height, ...]{}
\setkeys{Gin}{width=\maxwidth,height=\maxheight,keepaspectratio}
% Set default figure placement to htbp
\makeatletter
\def\fps@figure{htbp}
\makeatother
\setlength{\emergencystretch}{3em} % prevent overfull lines
\providecommand{\tightlist}{%
  \setlength{\itemsep}{0pt}\setlength{\parskip}{0pt}}
\setcounter{secnumdepth}{5}
\renewcommand{\linethickness}{0.05em}
\usepackage{booktabs}
\usepackage{sectsty} \allsectionsfont{\nohang\centering \emph}

\title{Pancreatobiliary Pathology Society Journal Watch}
\usepackage{etoolbox}
\makeatletter
\providecommand{\subtitle}[1]{% add subtitle to \maketitle
  \apptocmd{\@title}{\par {\large #1 \par}}{}{}
}
\makeatother
\subtitle{January-March 2020}
\author{}
\date{\vspace{-2.5em}Last Update on 2020-04-16}

\begin{document}
\maketitle

{
\setcounter{tocdepth}{2}
\tableofcontents
}
\begin{center}\rule{0.5\linewidth}{0.5pt}\end{center}

\hypertarget{pbpath_jw}{%
\section{PBPath Journal Watch Articles}\label{pbpath_jw}}

\begin{center}\rule{0.5\linewidth}{0.5pt}\end{center}

\textbf{Welcome to the PBPath Journal Watch!}

This bi-monthly journal watch features exciting recently published
pancreas and biliary pathology articles that will provide up to date
medical knowledge in our field. These articles will be showcased in
several convenient categories, including surgical pathology,
cytopathology, molecular pathology, pancreas, gallbladder, bile ducts,
and ampulla among others. The articles in each category are in no
particular order. See the list of journals we search regularly
\href{http://pbpath.org/pbpath-journal-watch/}{here}. Previous months'
issues may be found in our
\emph{\href{http://pbpath.org/journal-watch-archive/}{archive}} and you
may see \href{http://pbpath.org/journal-watch-upcoming-issue/}{drafts of
the upcoming issue here}.

We encourage members to actively participate by recommending new
articles and providing feedback using the
\protect\hyperlink{feedback}{forms provided below.}

\textbf{We hope that you will enjoy the new PBPath Journal Watch!}

Click here to see these articles and graphical summaries in other
databases

The \emph{journal watch} articles are collected in
\href{https://doi.org/10.5281/zenodo.3635440}{zenodo} and
\href{https://osf.io/3d67y/}{OSF}.

To see these selected articles in PubMed click here

To see these selected articles in Lens.org click here

Below is the content based groupings via
\href{https://openknowledgemaps.org/map/69587f66f7536830fe8dafa88c1f521b}{openknowledgemaps}.

\begin{center}\rule{0.5\linewidth}{0.5pt}\end{center}

\pagebreak

\hypertarget{pancreas}{%
\subsection{Pancreas}\label{pancreas}}

\begin{center}\rule{0.5\linewidth}{0.5pt}\end{center}

\begin{center}\rule{0.5\linewidth}{0.5pt}\end{center}

- \textbf{A 15-gene immune, stromal and proliferation gene signature
that significantly associates with poor survival in patients with
pancreatic ductal adenocarcinoma}

\emph{Clinical cancer research : an official journal of the American
Association for Cancer Research 2020 Mar;():}

PubMed: \url{https://www.ncbi.nlm.nih.gov/pubmed/?term=32234757}

PURPOSE: Pancreatic ductal adenocarcinoma (PDAC) is a lethal disease
with dismal survival rates. Tumor microenvironment (TME), comprising of
immune cells and cancer-associated fibroblasts, plays a key role in
driving poor prognosis and resistance to chemotherapy. Herein, we aimed
to identify a TME-associated, risk-stratification gene biomarker
signature in PDAC. EXPERIMENTAL DESIGN: The initial biomarker discovery
was performed in The Cancer Genome Atlas (TCGA, n=163) transcriptomic
data. This was followed by independent validation of the gene signature
in The International Cancer Genome Consortium (ICGC, n=95), E-MTAB-6134
(n=288), and GSE71729 (n=123) datasets for predicting overall survival
(OS), and for its ability to detect poor molecular subtypes. Clinical
validation and nomogram establishment was undertaken by performing
multivariate cox regression analysis. RESULTS: Our biomarker discovery
effort identified a 15-gene immune, stromal and proliferation (ISP) gene
signature that significantly associated with poor OS (HR: 3.90, 95\% CI,
2.36-6.41, p\textless0.0001). This signature also robustly predicted
survival in 3 independent validation cohorts ICGC (HR:2.63
{[}1.56-4.41{]}, p\textless0.0001), E-MTAB-6134 (HR:1.53
{[}1.14-2.04{]}, p=0.004), and GSE71729 (HR:2.33 {[}1.49-3.63{]},
p\textless0.0001). Interestingly, the ISP signature also permitted
identification of poor molecular PDAC subtypes with excellent accuracy
in all 4 cohorts; TCGA (AUC=0.94), ICGC (AUC=0.91), E-MTAB-6134
(AUC=0.80), and GSE71729 (AUC=0.83). The ISP-derived high-risk patients
exhibited significantly poor OS in a clinical validation cohort (n=119;
HR:2.62 {[}1.50-4.56{]}, p=0.0004). A nomogram was established which
included the ISP, CA19-9, T and N-stage for eventual clinical
translation. CONCLUSIONS: We report a novel gene signature for
risk-stratification and robust identification of PDAC patients with poor
molecular subtypes.

doi: \url{https://doi.org/10.1158/1078-0432.CCR-19-4044}

\begin{center}\rule{0.5\linewidth}{0.5pt}\end{center}

\begin{center}\rule{0.5\linewidth}{0.5pt}\end{center}

- \textbf{Expression Patterns and Prognostic Value of DNA Damage Repair
Proteins in Resected Pancreatic Neuroendocrine Neoplasms}

\emph{Annals of surgery 2020 Mar;():}

PubMed: \url{https://www.ncbi.nlm.nih.gov/pubmed/?term=32209898}

OBJECTIVE: This study aimed to examine the expression profiles and
prognostic value of multiple DDR proteins in resected PanNENs.
BACKGROUND: DDR proteins play important roles in various cancers,
including pancreatic ductal adenocarcinoma. However, the expression
patterns and prognostic value of DDR proteins in PanNENs remain unclear.
METHODS: This retrospective analysis included PanNEN patients who
underwent resection at the Fudan University Shanghai Cancer Center from
2012 to 2018. Immunohistochemical staining was performed for 12 DDR
proteins in tissue microarrays. The associations of DDR protein
expression and clinicopathological features with recurrence-free
survival (RFS) were examined via a Cox regression model and random
survival forest. A recurrence signature was constructed using recursive
partitioning analysis. RESULTS: In total, 131 PanNEN patients were
included, with 32 (24.4\%) cases of recurrence. Among the 12 DDR
proteins, low checkpoint kinase 2 (CHK2) expression (P = 0.020) and loss
of ataxia-telangiectasia-mutated (ATM) (P = 0.0007) significantly
correlated with recurrence. Multivariable Cox regression analysis
identified tumor size ≥3 cm, lymph node (LN) metastasis, high tumor
grade, low CHK2 expression, and ATM loss as independent risk factors for
recurrence. A recurrence signature was established based on the
importance of recurrence-specific risk factors; patients with the
LNnegTumorSize\textless3cm signature had a 5-year RFS rate of 96.8\%,
whereas patients with the LNposCHK2low signature had the worst 5-year
RFS rate (0\%). Discrimination (concordance index: 0.770) and
calibration plots indicated that the recurrence signature had a good
ability to identify patients at risk for recurrence. CONCLUSIONS: By
analyzing large-scale tissue microarrays of PanNENs, we evaluated 12 DDR
protein expression profiles. We developed a recurrence signature that
can identify distinct subpopulations according to RFS, which may help
refine individual follow-up.

doi: \url{https://doi.org/10.1097/SLA.0000000000003884}

\begin{center}\rule{0.5\linewidth}{0.5pt}\end{center}

\begin{center}\rule{0.5\linewidth}{0.5pt}\end{center}

- \textbf{DETECTION OF CIRCULATING TUMOR DNA IN PATIENTS WITH PANCREATIC
CANCER USING DIGITAL NEXT-GENERATION SEQUENCING}

\emph{The Journal of molecular diagnostics : JMD 2020 Mar;():}

PubMed: \url{https://www.ncbi.nlm.nih.gov/pubmed/?term=32205290}

Circulating tumor DNA (ctDNA) measurements can be used to estimate tumor
burden, but avoiding false-positives is a challenge. We evaluated
digital next-generation sequencing (NGS) as a ctDNA detection method.
Plasma KRAS and GNAS hotspot mutation levels were measured in 140
subjects including 67 with pancreatic ductal adenocarcinoma, and 73
healthy and disease controls. To limit chemical modifications of DNA
that yield false-positive mutation calls, plasma DNA was enzymatically
pre-treated, after which DNA was aliquoted for digital detection of
mutations (up to 384 aliquots/sample) by PCR and NGS. A digital NGS
score of two standard deviations above the mean in controls was
considered positive. 37\% of patients with pancreatic cancer, including
31\% of patients with Stage I/II disease had positive KRAS codon 12
ctDNA scores; only one patient had a positive GNAS mutation score. Two
disease control patients had positive ctDNA scores. Low normal-range
digital NGS scores at mutation hot-spots were found at similar levels in
healthy and disease controls, usually at sites of cytosine deamination,
and were likely the result of chemical modification of plasma DNA and
NGS error, rather than true mutations. Digital NGS detects mutated ctDNA
in patients with pancreatic cancer with similar yield to other methods.
The detection of low-level, true-positive ctDNA is limited by frequent
low-level detection of false-positive mutation cells in plasma DNA from
controls.

doi: \url{https://doi.org/10.1016/j.jmoldx.2020.02.010}

\begin{center}\rule{0.5\linewidth}{0.5pt}\end{center}

\begin{center}\rule{0.5\linewidth}{0.5pt}\end{center}

- \textbf{GATA6 expression distinguishes classical and basal-like
subtypes in advanced pancreatic cancer}

\emph{Clinical cancer research : an official journal of the American
Association for Cancer Research 2020 Mar;():}

PubMed: \url{https://www.ncbi.nlm.nih.gov/pubmed/?term=32156747}

PURPOSE: To determine the impact of basal-like and classical subtypes in
advanced PDAC and to explore GATA6 expression as a surrogate biomarker.
EXPERIMENTAL DESIGN: Within the COMPASS trial patients proceeding to
chemotherapy for advanced PDAC undergo tumour biopsy for RNA sequencing.
Overall response rate (ORR) and overall survival (OS) were stratified by
subtypes and according to chemotherapy received. Correlation of GATA6
with the subtypes using gene expression profiling, in situ hybridization
(ISH) were explored. RESULTS: Between December 2015-May 2019, 195
patients (95\%) had enough tissue for RNA sequencing; 39 (20\%) were
classified as basal-like and 156 (80\%) as classical. RECIST response
data were available for 157 patients; 29 basal-like and 128 classical
where the ORR was 10\% vs.~33\% respectively (p=0.02). In patients with
basal-like tumours treated with modified FOLFIRINOX (mFFX) (n=22) the
progression rate was 60\% compared to 15\% in classical PDAC (p=
0.0002). Median OS in the intention to treat population (n=195) was 9.3
months for classical vs.~5.9 months for basal-like PDAC (HR 0.47 95\% CI
0.32-0.69, p=0.0001). GATA6 expression by RNAseq highly correlated with
the classifier (p\textless0.001) and ISH predicted the subtypes with
sensitivity of 89\% and specificity of 83\%. In a multivariable
analysis, GATA6 expression was prognostic (p=0.02). In exploratory
analyses, basal-like tumours, could be identified by keratin 5, were
more hypoxic and enriched for a T cell inflamed gene expression
signature. CONCLUSIONS: The basal-like subtype is chemoresistant and can
be distinguished from classical PDAC by GATA6 expression.

doi: \url{https://doi.org/10.1158/1078-0432.CCR-19-3724}

\begin{center}\rule{0.5\linewidth}{0.5pt}\end{center}

\begin{center}\rule{0.5\linewidth}{0.5pt}\end{center}

- \textbf{Validation and modification of staging Systems for Poorly
Differentiated Pancreatic Neuroendocrine Carcinoma}

\emph{BMC cancer 2020 Mar;20(1):188}

PubMed: \url{https://www.ncbi.nlm.nih.gov/pubmed/?term=32138704}

BACKGROUND: The American Joint Committee on Cancer (AJCC) and the
European Neuroendocrine Tumor Society (ENETS) staging classifications
are two broadly used systems for pancreatic neuroendocrine tumors. This
study aims to identify the most accurate and useful
tumor-node-metastasis (TNM) staging system for poorly differentiated
pancreatic neuroendocrine carcinomas (pNECs). METHODS: An analysis was
performed to evaluate the application of the ENETS, 7th edition (7th)
AJCC and 8th edition (8th) AJCC staging classifications using the
Surveillance, Epidemiology, and End Results (SEER) registry (N = 568
patients), and a modified system based on the analysis of the 7th AJCC
classification was proposed. RESULTS: In multivariable analyses, only
the 7th AJCC staging system allocated patients into four different risk
groups, although there was no significant difference. We modified the
staging classification by maintaining the T and M definitions of the 7th
AJCC staging and adopting new staging definitions. An increased hazard
ratio (HR) of death was also observed from class I to class IV for the
modified 7th (m7th) staging system (compared with stage I disease; HR
for stage II =1.23, 95\% confidence interval (CI) = 0.73-2.06, P = 0.44;
HR for stage III =2.20, 95\% CI =1.06-4.56, P = 0.03; HR for stage IV
=4.95, 95\% CI =3.20-7.65, P \textless{} 0.001). The concordance index
(C-index) was higher for local disease with the m7th AJCC staging system
than with the 7th AJCC staging system. CONCLUSIONS: The m7th AJCC
staging system for pNECs proposed in this study provides improvements
and may be assessed for potential adoption in the next edition.

doi: \url{https://doi.org/10.1186/s12885-020-6634-9}

\begin{center}\rule{0.5\linewidth}{0.5pt}\end{center}

\begin{center}\rule{0.5\linewidth}{0.5pt}\end{center}

- \textbf{INSM1 Is a Highly Specific Marker of Neuroendocrine
Differentiation in Primary Neoplasms of the Gastrointestinal Tract,
Appendix, and Pancreas}

\emph{American journal of clinical pathology 2020 Mar;():}

PubMed: \url{https://www.ncbi.nlm.nih.gov/pubmed/?term=32128564}

OBJECTIVES: INSM1 has been described as a sensitive and specific
neuroendocrine marker. This study aims to compare INSM1 with traditional
neuroendocrine markers in gastrointestinal neuroendocrine neoplasms.
METHODS: Retrospective review (2008-2018) was used to retrieve
paraffin-embedded tissue from 110 gastrointestinal neuroendocrine
neoplasms and controls that was subsequently stained with INSM1,
synaptophysin, chromogranin, CD56, and Ki-67. RESULTS: INSM1 was
positive in 16 of 17 (94.1\%) gastric, 17 of 18 (94.4\%) pancreatic, 13
of 18 (72.2\%) small bowel, 17 of 21 (81.0\%) colonic, and 26 of 36
(72.2\%) appendiceal tumors. INSM1 was positive in 58 of 70 (82.9\%)
well-differentiated neuroendocrine tumors, 17 of 20 (85.0\%) poorly
differentiated neuroendocrine carcinomas, 8 of 11 (72.7\%) low-grade
goblet cell adenocarcinomas (grade 1), and 6 of 9 (66.7\%) high-grade
goblet cell adenocarcinomas (grade 2/3). INSM1 sensitivity for
neuroendocrine neoplasms (80.9\%) was less than that of synaptophysin
(99.1\%), chromogranin (88\%), and CD56 (95.3\%); specificity was higher
(95.7\% vs 86.0\%, 87.3\%, and 86.0\%, respectively). CONCLUSIONS: INSM1
is a useful marker of neuroendocrine differentiation in gastrointestinal
neuroendocrine and mixed neuroendocrine neoplasms. Compared with
traditional neuroendocrine markers, INSM1 is less sensitive but more
specific.

doi: \url{https://doi.org/10.1093/ajcp/aqaa014}

\begin{center}\rule{0.5\linewidth}{0.5pt}\end{center}

\begin{center}\rule{0.5\linewidth}{0.5pt}\end{center}

- \textbf{Global, regional and national burden of pancreatic cancer,
1990 to 2017: Results from the Global Burden of Disease Study 2017}

\emph{Pancreatology : official journal of the International Association
of Pancreatology (IAP) \ldots{} {[}et al.{]} 2020 Apr;20(3):462-469}

PubMed: \url{https://www.ncbi.nlm.nih.gov/pubmed/?term=32113937}

BACKGROUND: The global burden of pancreatic cancer (PCa) continues to
grow. Detailed data on PCa epidemiology are essential for policy-making
and appropriate healthcare resource allocation. METHODS: Estimates of
incidence, death and disability-adjusted life years (DALYs) of PCa from
1990 to 2017 were collected from the Global Burden of Disease Study
2017. Decomposition analysis was conducted to detect the contributing
factors related to PCa incidence variation. The estimated annual
percentage change (EAPC) was calculated to quantify the PCa epidemiology
trends over a specified interval. RESULTS: Globally, the incidence of
PCa cases increased by 129.1\% to 447 664~664 (95\% uncertainty interval
(UI) 438~597-456~295), death increased by 125.2\% to 441 082~082 (95\%
UI 448~960-432~833), and DALYs increased by 107.3\% to 9~080~004 (95\%
UI 8~894~128-9~256~346) between 1990 and 2017. Relatively higher
sociodemographic index (SDI) regions were observed with greater
incidences, more deaths and a greater number of DALYs of PCa, but
relatively lower SDI regions experienced a sharply increasing trend in
these measures. Decomposition analysis indicated that the global
increase in PCa incidence was driven by the aging population from 2007
to 2017, especially in higher SDI regions. In addition, a significant
negative correlation was found between EAPC and ASIR (in 1990)
(r~=~-0.56, P~\textless~0.001). CONCLUSIONS: PCa remains a major public
health burden globally. The unfavorable trend in PCa suggesting that
further study for prevention should be conducted to forestall the
increase in pancreatic cancer.

doi: \url{https://doi.org/10.1016/j.pan.2020.02.011}

\begin{center}\rule{0.5\linewidth}{0.5pt}\end{center}

\begin{center}\rule{0.5\linewidth}{0.5pt}\end{center}

- \textbf{Pancreatic acinar cell carcinomas and mixed
acinar-neuroendocrine carcinomas are more clinically aggressive than
grade 1 pancreatic neuroendocrine tumours}

\emph{Pathology 2020 Apr;52(3):336-347}

PubMed: \url{https://www.ncbi.nlm.nih.gov/pubmed/?term=32111396}

Acinar cell carcinomas (ACCs) and mixed acinar-neuroendocrine carcinomas
(MAcNECs) of the pancreas are extremely rare carcinomas with a
significant component with acinar differentiation. To date, the
clinicopathological behaviours of these neoplasms remain unclear. In
this study, we evaluated the histopathological and molecular
characteristics of 20 ACCs and 13 MAcNECs and compared them to a cohort
of 269 well-differentiated pancreatic neuroendocrine tumours (PanNETs).
Compared to PanNETs, both ACCs and MAcNECs had an advanced pT
classification (p\textless0.001), as well as more prevalent
lymphovascular and perineural invasion (p=0.002) and lymph node and
distant metastases (p\textless0.001). Patients with MAcNECs had worse
overall (p\textless0.001) and recurrence-free survival (p\textless0.001)
than those with PanNETs, but no significant difference with those with
ACCs. Subgroup analyses revealed that patients with ACCs and MAcNECs had
significantly worse recurrence-free survival than those with grade 1
PanNET (p\textless0.001), and patients with MAcNECs also had worse
overall survival than those with grade 1 and 2 PanNETs (p\textless0.001,
and p=0.001). ACCs presented more commonly with intraductal growth
(p=0.014) than MAcNECs, while MAcNECs more often had lymph node
metastasis (p=0.012) than ACCs. The telomere maintenance mechanism
Alternative Lengthening of Telomeres (ALT) was assessed by
telomere-specific FISH, and ALT was detected in 1 of 20 ACCs and in
three of the 13 MAcNECs. Patients with MAcNECs and ACCs had worse
survival and more aggressive behaviour than those with grade 1 PanNETs;
thus, the clinicopathological behaviour of MAcNECs resembles ACCs rather
than PanNETs. Combined neuroendocrine and acinar cell
immunohistochemical markers are helpful for differentiating these
different tumour types.

doi: \url{https://doi.org/10.1016/j.pathol.2020.01.437}

\begin{center}\rule{0.5\linewidth}{0.5pt}\end{center}

\begin{center}\rule{0.5\linewidth}{0.5pt}\end{center}

- \textbf{MMR Deficiency is Homogeneous in Pancreatic Carcinoma and
Associated with High Density of Cd8-Positive Lymphocytes}

\emph{Annals of surgical oncology 2020 Feb;():}

PubMed: \url{https://www.ncbi.nlm.nih.gov/pubmed/?term=32108923}

BACKGROUND: Microsatellite instability (MSI) has emerged as a predictive
biomarker for immune checkpoint inhibitor therapy. Cancer heterogeneity
represents a potential obstacle for the analysis of predicitive
biomarkers. MSI has been reported in pancreatic cancer, but data on the
possible extent of intratumoral heterogeneity are lacking. METHODS: To
study MSI heterogeneity in pancreatic cancer, a tissue microarray (TMA)
comprising 597 tumors was screened by immunohistochemistry with
antibodies for the mismatch repair (MMR) proteins MLH1, PMS2, MSH2, and
MSH6. RESULTS: In six suspicious cases, large section
immunohistochemistry and microsatellite analysis (Bethesda panel)
resulted in the identification of 4 (0.8\%) validated MSI cases out of
480 interpretable pancreatic ductal adenocarcinomas. MSI was absent in
55 adenocarcinomas of the ampulla of Vater and 7 acinar cell carcinomas.
MMR deficiency always involved MSH6 loss, in three cases with additional
loss of MSH2 expression. Three cancers were MSI-high and one case with
isolated MSH6 loss was MSS in PCR analysis. The analysis of 44
cancer-containing tumor blocks revealed that the loss of MMR protein
expression was always homogeneous in affected tumors. Automated digital
image analysis of CD8 immunostaining demonstrated markedly higher
CD8 + tumor infiltrating lymphocytes in tumors with (mean = 685,
median = 626) than without (mean = 227; median = 124) MMR deficiency
(p \textless{} 0.0001), suggesting a role of MSI for immune response.
CONCLUSIONS: Our data suggest that MSI occurs early in a small subset of
ductal adenocarcinomas of the pancreas and that immunohistochemical MMR
analysis on limited biopsy or cytology material may be sufficient to
estimate MMR status of the entire cancer mass.

doi: \url{https://doi.org/10.1245/s10434-020-08209-y}

\begin{center}\rule{0.5\linewidth}{0.5pt}\end{center}

\begin{center}\rule{0.5\linewidth}{0.5pt}\end{center}

- \textbf{Tumor-Insular Complex in Neoadjuvant Treated Pancreatic Ductal
Adenocarcinoma Is Associated With Higher Residual Tumor}

\emph{The American journal of surgical pathology 2020 Feb;():}

PubMed: \url{https://www.ncbi.nlm.nih.gov/pubmed/?term=32091434}

The tumor microenvironment in pancreatic ductal adenocarcinoma (PDAC)
plays a vital role in treatment response, and therefore, patient
survival. We and others have observed an intimate association of
neoplastic ductal cells with non-neoplastic islet cells, recapitulating
the ductoinsular complex. We define this phenomenon as tumor-insular
complex (TIC). Herein, we describe the clinicopathologic characteristics
of TIC in neoadjuvant treated PDAC cases for the first time. We
retrospectively reviewed the pathology of 105 cases of neoadjuvant
treated PDAC resected at our institution. TIC was noted in 35 cases
(33.3\%), the mean tumor bed size was 2.7±1.0 cm, mean percentage of
residual tumor 40±28\% and mean Residual Tumor Index (RTI) (an index
previously established as a prognostic parameter by our group) was
1.1±1.0. TIC was significantly associated with perineural invasion
(P=0.001), higher tumor bed size (P=0.007), percentage of residual tumor
(P=0.009), RTI (P=0.001), ypT stage (P=0.045), and poor treatment
response, grouped by a previously established criteria (P=0.010). Using
our prior binary reported prognostic cutoff for RTI of ≤0.35 and
\textgreater0.35, TIC was associated with a RTI \textgreater0.35
(P=0.002). Moreover, patients who did not receive neoadjuvant radiation
were associated with a higher frequency of TIC (P=0.003). In this
cohort, RTI but not TIC was also shown to be a significant independent
prognosticator for recurrence-free survival and overall survival on
multivariate analysis. In conclusion, TIC is significantly associated
with a more aggressive neoplasm which shows a poor treatment response.
Further studies will be needed to better understand the tumor biology of
TICs.

doi: \url{https://doi.org/10.1097/PAS.0000000000001454}

\begin{center}\rule{0.5\linewidth}{0.5pt}\end{center}

\begin{center}\rule{0.5\linewidth}{0.5pt}\end{center}

- \textbf{Insulinoma-associated protein 1 (INSM1) is a robust marker for
identifying and grading pancreatic neuroendocrine tumors}

\emph{Cancer cytopathology 2020 Apr;128(4):269-277}

PubMed: \url{https://www.ncbi.nlm.nih.gov/pubmed/?term=31977134}

BACKGROUND: Pancreatic neuroendocrine tumor (PNET) is a diagnostic
challenge with limited samples in not only identification but grading.
Prior studies have shown insulinoma-associated protein 1 (INSM1) to be a
robust marker in identifying PNETs from other solid pancreatic tumors on
resection specimens. In this study, we investigated the utility of INSM1
not only for identifying PNETs but also for grading in cell blocks (CBs)
and surgical resections (SRs). METHODS: A search for PNET cases between
2000 and 2019 identified 55 samples (26 CBs and 29 SRs) that were
further separated into high (2 CBs, 3 SRs), intermediate (4 CBs, 7 SRs),
and low (20 CBs, 19 SRs) grades based on their final pathology report
and Ki-67 level. Immunohistochemical (IHC) staining for INSM1 (C-8,
Santa Cruz Biotechnology {[}1:100{]}) was performed and quantified using
an H score of 0 to 300. Non-PNET solid pancreatic tumors were compared
and included acinar cell carcinoma, solid pseudopapillary neoplasm, and
ductal adenocarcinoma. RESULTS: All 55 cases of PNET demonstrated
nuclear INSM1 staining. The average H scores for INSM1 staining of PNET
were 254 and 252 in CB and SR, respectively. The H scores decreased with
increasing tumor grade, with low-grade (G1), intermediate-grade (G2),
and high-grade (G3) tumors showing average INSM1 H scores of 229 and
253, 266 and 253, and 30 and 33 in both CB and SR, respectively.
CONCLUSION: IHC with INSM1 plays a role in identifying and potentially
grading PNETs.

doi: \url{https://doi.org/10.1002/cncy.22242}

\begin{center}\rule{0.5\linewidth}{0.5pt}\end{center}

\begin{center}\rule{0.5\linewidth}{0.5pt}\end{center}

- \textbf{Microscopic Size Measurements Predict Outcomes in
Post-Neoadjuvant Resections of Pancreatic Ductal Adenocarcinoma (PDAC)}

\emph{Histopathology 2020 Jan;():}

PubMed: \url{https://www.ncbi.nlm.nih.gov/pubmed/?term=31965618}

BACKGROUND: Pancreatic ductal adenocarcinomas (PDACs) are increasingly
treated with neoadjuvant therapy. However, the American Joint Committee
on Cancer (AJCC) 8th Edition T staging based on tumor size does not
reflect treatment effect, which often results in multiple, small foci of
residual tumor in a background of mass-forming fibrosis. Thus, we
evaluated the performance of AJCC 8th Edition T staging in predicting
patient outcomes using a microscopic tumor size measurement method.
METHODS AND RESULTS: 106 post-neoadjuvant therapy pancreatectomies were
reviewed, and all individual tumor foci were measured. T stages based on
gross size (GS) and the largest single microscopic focus size (MFS) were
examined in association with clinicopathologic variables and patient
outcomes. 63/106 (59\%) were locally advanced; 78\% received FOLFIRINOX
treatment. Average GS and MFS were 2.5cm and 1.1cm, respectively; 9
cases each were classified as T0, 35 and 85 cases as T1, 42 and 12 cases
as T2, and 20 and 0 cases as T3, based on the GS and MFS, respectively.
Higher GS- and MFS-based T stages were significantly associated with
higher tumor regression grade, lymphovascular and perineural invasion,
and higher N stage. Furthermore, higher MFS-based T stage was
significantly associated with shorter disease-free survival (DFS)
(p\textless0.001) and shorter overall survival (OS) (p=0.002). GS was
significantly associated with OS (p=0.046), but not with DFS.
CONCLUSIONS: In post-neoadjuvant PDAC resections, MFS-based T staging is
superior to GS-based T staging for predicting patient outcomes,
suggesting that microscopic measurements have clinical utility beyond
the conventional use of GS measurements alone.

doi: \url{https://doi.org/10.1111/his.14067}

\begin{center}\rule{0.5\linewidth}{0.5pt}\end{center}

\begin{center}\rule{0.5\linewidth}{0.5pt}\end{center}

- \textbf{DNAJB1-PRKACA fusions occur in oncocytic pancreatic and
biliary neoplasms and are not specific for fibrolamellar hepatocellular
carcinoma}

\emph{Modern pathology : an official journal of the United States and
Canadian Academy of Pathology, Inc 2020 04;33(4):648-656}

PubMed: \url{https://www.ncbi.nlm.nih.gov/pubmed/?term=31676785}

Recently discovered DNAJB1-PRKACA oncogenic fusions have been considered
diagnostic for fibrolamellar hepatocellular carcinoma. In this study, we
describe six pancreatobiliary neoplasms with PRKACA fusions, five of
which harbor the DNAJB1-PRKACA fusion. All neoplasms were subjected to a
hybridization capture-based next-generation sequencing assay
(MSK-IMPACT), which enables the identification of sequence mutations,
copy number alterations, and selected structural rearrangements
involving ≥410 genes (n = 6) and/or to a custom targeted, RNA-based
panel (MSK-Fusion) that utilizes Archer Anchored Multiplex PCR
technology and next-generation sequencing to detect gene fusions in 62
genes (n = 2). Selected neoplasms also underwent FISH analysis, albumin
mRNA in-situ hybridization, and arginase-1 immunohistochemical labeling
(n = 3). Five neoplasms were pancreatic, and one arose in the
intrahepatic bile ducts. All revealed at least focal oncocytic
morphology: three cases were diagnosed as intraductal oncocytic
papillary neoplasms, and three as intraductal papillary mucinous
neoplasms with mixed oncocytic and pancreatobiliary or gastric features.
Four cases had an invasive carcinoma component composed of oncocytic
cells. Five cases revealed DNAJB1-PRKACA fusions and one revealed an
ATP1B1-PRKACA fusion. None of the cases tested were positive for albumin
or arginase-1. Our data prove that DNAJB1-PRKACA fusion is neither
exclusive nor diagnostic for fibrolamellar hepatocellular carcinoma, and
caution should be exercised in diagnosing liver tumors with
DNAJB1-PRKACA fusions as fibrolamellar hepatocellular carcinoma,
particularly if a pancreatic lesion is present. Moreover, considering
DNAJB1-PRKACA fusions lead to upregulated protein kinase activity and
that this upregulated protein kinase activity has a significant role in
tumorigenesis of fibrolamellar hepatocellular carcinoma, protein kinase
inhibition could have therapeutic potential in the treatment of these
pancreatobiliary neoplasms as well, once a suitable drug is developed.

doi: \url{https://doi.org/10.1038/s41379-019-0398-2}

\begin{center}\rule{0.5\linewidth}{0.5pt}\end{center}

\begin{center}\rule{0.5\linewidth}{0.5pt}\end{center}

- \textbf{RET gene rearrangements occur in a subset of pancreatic acinar
cell carcinomas}

\emph{Modern pathology : an official journal of the United States and
Canadian Academy of Pathology, Inc 2020 04;33(4):657-664}

PubMed: \url{https://www.ncbi.nlm.nih.gov/pubmed/?term=31558784}

Pancreatic acinar cell carcinoma is relatively rare (1 to 2\% of
pancreatic malignancies) but may be under-recognized. In contrast to
pancreatic ductal adenocarcinoma, most acinar cell carcinomas lack
mutations in KRAS, DPC, CDKN2A or TP53, but appear to have a high
incidence of gene rearrangements, with up to 20\% reported to be driven
by BRAF fusions. With the development of a new class of RET-specific
tyrosine kinase inhibitors, which appear to have particularly strong
activity against RET gene rearranged tumours, there is now considerable
interest in identifying RET gene rearrangements across a wide range of
cancers. RET rearrangements have been reported to occur at a very low
incidence (\textless1\%) in all pancreatic carcinomas. We postulated
that given its unique molecular profile, RET gene rearrangements may be
common in acinar cell carcinomas. We performed fluorescent in-situ
hybridization (FISH) studies on a cohort of 40 acinar cell spectrum
tumours comprising 36 pure acinar cell carcinomas, three
pancreatoblastomas and one mixed acinar-pancreatic neuroendocrine
tumour. RET gene rearrangements were identified in 3 (7.5\%) cases and
BRAF gene rearrangements in 5 (12.5\%). All gene rearranged tumours were
pure acinar cell carcinomas. Our findings indicate that amongst all
pancreatic carcinomas, acinar carcinomas are highly enriched for
potentially actionable gene rearrangements in RET or BRAF. FISH testing
is inexpensive and readily available in the routine clinical setting and
may have a role in the assessment of all acinar cell carcinomas-at this
stage to recruit patients for clinical trials of new targeted therapies,
but perhaps in the near future as part of routine care.

doi: \url{https://doi.org/10.1038/s41379-019-0373-y}

\begin{center}\rule{0.5\linewidth}{0.5pt}\end{center}

\begin{center}\rule{0.5\linewidth}{0.5pt}\end{center}

- \textbf{Elucidating the roles of ASPM isoforms reveals a novel
prognostic marker for pancreatic cancer}

\emph{The Journal of pathology 2020 Feb;250(2):123-125}

PubMed: \url{https://www.ncbi.nlm.nih.gov/pubmed/?term=31595972}

Pancreatic ductal adenocarcinoma (PDAC) is one of the deadliest cancers
worldwide. Late diagnosis, desmoplastic tissue and intrinsic resistance
to therapy are among the main reasons for its aggressive phenotype. In
addition, it is now appreciated that cancer stem cells~-~a rare
subpopulation of tumor cells highly resistant to therapy~-~are crucial
players in PDAC initiation, progression and resistance to therapy. In a
recent article in The Journal of Pathology, Hsu et al elucidated the
specific roles of abnormal spindle-like, microcephaly-associated protein
(ASPM) isoforms in PDAC. The authors reported that ASPM isoform I
(ASPM-iI) is mainly expressed in the cytoplasm of PDAC cells. Its
expression is associated with the Wnt signaling pathway, which promotes
stemness and maintains the cancer stem cell niche. Clinically,
expression of ASPM-iI correlates with poor survival in PDAC patients.
Thus, this study revealed a novel prognostic marker as well as a
potential therapeutic target for PDAC. © 2019 Pathological Society of
Great Britain and Ireland. Published by John Wiley \& Sons, Ltd.

doi: \url{https://doi.org/10.1002/path.5355}

\begin{center}\rule{0.5\linewidth}{0.5pt}\end{center}

\begin{center}\rule{0.5\linewidth}{0.5pt}\end{center}

- \textbf{Sclerosing epithelioid mesenchymal neoplasm of the
pancreas~-~a proposed new entity}

\emph{Modern pathology : an official journal of the United States and
Canadian Academy of Pathology, Inc 2020 03;33(3):456-467}

PubMed: \url{https://www.ncbi.nlm.nih.gov/pubmed/?term=31383964}

We have encountered pancreatic tumors with unique histologic features,
which do not conform to any of the known tumors of the pancreas or other
anatomical sites. We aimed to define their clinicopathologic features
and whether they are characterized by recurrent molecular signatures.
Eight cases were identified; studied histologically and by
immunohistochemistry. Selected cases were also subjected to whole-exome
sequencing (WES; n = 4), RNA-sequencing (n = 6), Archer FusionPlex assay
(n = 5), methylation profiling using the Illumina MethylationEPIC (850k)
array platform (n = 6), and TERT promoter sequencing (n = 5). Six
neoplasms occurred in females. The mean age was 43 years (range: 26-75).
Five occurred in the head/neck of the pancreas. All patients were
treated surgically; none received neoadjuvant/adjuvant therapy. All
patients are free of disease after 53 months of median follow-up (range:
8-94). The tumors were well-circumscribed, and the median size was
1.8 cm (range: 1.3-5.8). Microscopically, the unencapsulated tumors had
a geographic pattern of epithelioid cell nests alternating with spindle
cell fascicles. Some areas showed dense fibrosis, in which enmeshed
tumor cells imparted a slit-like pattern. The predominant epithelioid
cells had scant cytoplasm and round-oval nuclei with open chromatin. The
spindle cells displayed irregular, hyperchromatic nuclei. Mitoses were
rare. No lymph node metastases were identified. All tumors were positive
for vimentin, CD99 and cytokeratin (patchy), while negative for markers
of solid pseudopapillary neoplasm, neuroendocrine, acinar,
myogenic/rhabdoid, vascular, melanocytic, or lymphoid differentiation,
gastrointestinal stromal tumor as well as MUC4. Whole-exome sequencing
revealed no recurrent somatic mutations or amplifications/homozygous
deletions in any known oncogenes or tumor suppressor genes.
RNA-sequencing and the Archer FusionPlex assay did not detect any
recurrent likely pathogenic gene fusions. Single sample gene set
enrichment analysis revealed that these tumors display a likely
mesenchymal transcriptomic program. Unsupervised analysis (t-SNE) of
their methylation profiles against a set of different mesenchymal
neoplasms demonstrated a distinct methylation pattern. Here, we describe
pancreatic neoplasms with unique morphologic/immunophenotypic features
and a distinct methylation pattern, along with a lack of abnormalities
in any of key genetic drivers, supporting that these neoplasms represent
a novel entity with an indolent clinical course. Given their mesenchymal
transcriptomic features, we propose the designation of ``sclerosing
epithelioid mesenchymal neoplasm'' of the pancreas.

doi: \url{https://doi.org/10.1038/s41379-019-0334-5}

\begin{center}\rule{0.5\linewidth}{0.5pt}\end{center}

\begin{center}\rule{0.5\linewidth}{0.5pt}\end{center}

Back to top

\begin{center}\rule{0.5\linewidth}{0.5pt}\end{center}

\pagebreak

\hypertarget{gallbladder}{%
\subsection{Gallbladder}\label{gallbladder}}

\begin{center}\rule{0.5\linewidth}{0.5pt}\end{center}

\begin{center}\rule{0.5\linewidth}{0.5pt}\end{center}

- \textbf{Long-term outcomes of surgical resection for T1b gallbladder
cancer: an institutional evaluation}

\emph{BMC cancer 2020 Jan;20(1):20}

PubMed: \url{https://www.ncbi.nlm.nih.gov/pubmed/?term=31907021}

BACKGROUND: There is no comprehensive agreement concerning the overall
performance of radical resection for T1b gallbladder cancer (GBC). This
research focused on addressing whether T1b GBC may spread
loco-regionally and whether radical resection is necessary. METHODS: A
retrospective analysis was conducted of 1032 patients with GBC who
underwent surgical resection at our centre and its affiliated
institutions between January 1982 and December 2018. A total of 47
patients with T1b GBC, 29 (62\%) of whom underwent simple
cholecystectomy and 18 (38\%) of whom underwent radical resection with
regional lymph node dissection, were enrolled in the study. RESULTS: GBC
was diagnosed pre-operatively in 16 patients (34\%), whereas 31 patients
(66\%) had incidental GBC. There was no blood venous or perineural
invasion in any patient on histology evaluation, except for lymphatic
vessel invasion in a single patient. There were no metastases in any
analysed lymph nodes. The open surgical approach was more prevalent
among the 18 patients who underwent radical resection (open in all 18
patients) than among the 29 patients who underwent simple
cholecystectomy (open in 21; laparoscopic in 8) (P = 0.017). The
cumulative 10- and 20-year overall survival rates were 65 and 25\%,
respectively. The outcome following simple cholecystectomy (10-year
overall survival rate of 66\%) was akin to that following radical
resection (64\%, P = 0.618). The cumulative 10- and 20-year
disease-specific survival rates were 93 and 93\%, respectively. The
outcome following simple cholecystectomy (10-year disease-specific
survival rate of 100\%) was equivalent to that following radical
resection (that of 86\%, P = 0.151). While age (\textgreater{} 70 years,
hazard ratio 5.285, P = 0.003) and gender (female, hazard ratio 0.272,
P = 0.007) had a strong effect on patient overall survival, surgical
procedure (simple cholecystectomy vs.~radical resection) and surgical
approach (open vs.~laparoscopic) did not. CONCLUSIONS: Most T1b GBCs
represent local disease. As pre-operative diagnosis, including tumour
penetration of T1b GBC, is difficult, the decision of radical resection
is justified. Additional radical resection is not required following
simple cholecystectomy provided that the penetration depth is restricted
towards the muscular layer and that surgical margins are uninvolved.

doi: \url{https://doi.org/10.1186/s12885-019-6507-2}

\begin{center}\rule{0.5\linewidth}{0.5pt}\end{center}

\begin{center}\rule{0.5\linewidth}{0.5pt}\end{center}

- \textbf{TRAIL receptors are differentially regulated and clinically
significant in gallbladder cancer}

\emph{Pathology 2020 Apr;52(3):348-358}

PubMed: \url{https://www.ncbi.nlm.nih.gov/pubmed/?term=32111400}

Deregulation of the receptors of TNF-related apoptosis inducing ligand
(TRAIL) has been reported in various cancers. In an effort to define the
role of these receptors we profiled their expression in gallbladder
cancer (GBC) and explored their clinical significance. Expression of
TRAIL receptors' mRNA in GBC was analysed through reverse transcriptase
polymerase chain reaction (RT-PCR), and protein through western
blotting, immunohistochemistry and enzyme-linked immunosorbent assay
(ELISA). mRNA data show frequent higher expression of TRAIL receptors in
GBC samples. Death receptors DR4 and DR5 showed significant negative
correlation with tumour stage, T stage and tumour grade; DcR1 transcript
showed positive correlation with tumour stage, N stage, M stage and
tumour grade. Similarly, IHC showed frequent positive staining for DR4,
DR5 and DcR1in GBC samples. Cytoplasmic and nuclear DR4 protein showed
negative correlation with T stage and tumour grade, whereas cytoplasmic
DcR1 protein showed positive correlation with tumour stage and N stage.
Nuclear DcR1 showed positive correlation with N stage. ELISA results
showed significantly higher expression of secretory DcR1 in GBC
patients. Kaplan-Meier analysis demonstrated significantly decreased
mean survival of patients with positive staining of cytoplasmic DcR1.
High level of death receptors identified the patients with early
gallbladder cancer, whereas high DcR1 expression served as a prognostic
factor for poor outcome.

doi: \url{https://doi.org/10.1016/j.pathol.2019.12.001}

\begin{center}\rule{0.5\linewidth}{0.5pt}\end{center}

\begin{center}\rule{0.5\linewidth}{0.5pt}\end{center}

- \textbf{Non-neoplastic Polyps of the Gallbladder: A Clinicopathologic
Analysis of 447 Cases}

\emph{The American journal of surgical pathology 2020 Apr;44(4):467-476}

PubMed: \url{https://www.ncbi.nlm.nih.gov/pubmed/?term=31725469}

There is no systematic histopathologic analysis of non-neoplastic polyps
in the gallbladder. In this study, in addition to a computer search for
cases designated as ``polyp,'' a systematic review of 2533 consecutive
routinely sampled archival and 203 totally submitted prospective
cholecystectomies were analyzed for \textgreater2 mm polyps (cut-off was
based on radiologic sensitivity). A total of 447 non-neoplastic polyps
were identified. The frequency was 3\% in archival cases and 5\% in
totally submitted cases. Only 21 (5\%) were ≥1 cm. The average age was
52 years, and the female to male ratio was 3.1. Two distinct categories
were delineated: (1) injury-related polyps (n=273): (a)
Fibro(myo)glandular polyps (n=214) were small (mean=0.4 cm),
broad-based, often multiple (45\%), almost always (98\%)
gallstone-associated, and were composed of a mixture of
(myo)fibroblastic tissue/lobular glandular units with chronic
cholecystitis. Dysplasia seen in 9\% seemed to be secondary involvement.
(b) Metaplastic pyloric glands forming polypoid collections (n=42). (c)
Inflammatory-type polyps associated with acute/subacute injury (11
granulation tissue, 3 xanthogranulomatous, 3 lymphoid). (2) Cholesterol
polyps (n=174) occurred in uninjured gallbladders, revealing a very thin
stalk, edematous cores devoid of glands but with cholesterol-laden
macrophages in 85\%, and cholesterolosis in the uninvolved mucosa in
60\%. Focal low-grade dysplasia was seen in 3\%, always confined to the
polyp, unaccompanied by carcinoma. In conclusion, non-neoplastic polyps
are seen in 3\% of cholecystectomies and are often small. Injury-related
fibromyoglandular polyps are the most common. Cholesterol polyps have
distinctive cauliflower architecture, often in a background of uninjured
gallbladders with cholesterolosis and may lack the cholesterol-laden
macrophages in the polyp itself. Although dysplastic changes can involve
non-neoplastic polyps, they do not seem to be the cause of invasive
carcinoma by themselves.

doi: \url{https://doi.org/10.1097/PAS.0000000000001405}

\begin{center}\rule{0.5\linewidth}{0.5pt}\end{center}

\begin{center}\rule{0.5\linewidth}{0.5pt}\end{center}

- \textbf{Clinicopathologic and Prognostic Significance of Gallbladder
and Cystic Duct Invasion in Distal Bile Duct Carcinoma}

\emph{Archives of pathology \& laboratory medicine 2019 Nov;():}

PubMed: \url{https://www.ncbi.nlm.nih.gov/pubmed/?term=31755778}

CONTEXT.---: The roles of the gallbladder and cystic duct (CD) invasions
in distal bile duct carcinoma (DBDC) have not been well elucidated.
OBJECTIVE.---: To define the characteristics and prognostic significance
of gallbladder or CD invasions in patients with DBDC. DESIGN.---: Organ
invasion patterns with clinicopathologic features were assessed in 258
resected DBDCs. RESULTS.---: CD invasions (N = 31) were associated with
frequent concomitant pancreatic and/or duodenal invasions (23 of 31,
74\%) and showed stromal infiltration (16 of 31, 52\%) and intraductal
cancerization (15 of 31, 48\%) patterns. In only 2 cases, invasions with
intraductal cancerization were observed in the gallbladder neck.
Conversely, all pancreatic (N = 175) and duodenal (83) invasions
developed through stromal infiltration. CD invasions were associated
with larger tumor size (P = .001), bile duct margin positivity (P =
.001), perineural invasions (P = .04), and higher N categories (P =
.007). Patients with pancreatic or duodenal invasions had significantly
lower survival rates than those without pancreatic (median, 31.0 versus
93.9 months) or duodenal (27.5 versus 56.8 months, P \textless{} .001,
both) invasions. However, those with gallbladder or CD invasions did not
have different survival times (P = .13). Patients with concomitant
gallbladder/CD and pancreatic/duodenal invasions demonstrated
significantly lower survival rates than those without organ invasions (P
\textless{} .001). CONCLUSIONS.---: Gallbladder invasions were rare in
DBDCs as neck invasions with intraductal cancerization. CD invasions
occurred by stromal infiltrations and intraductal cancerization, whereas
all pancreatic and duodenal invasions had stromal infiltration patterns.
Gallbladder and/or CD invasions did not affect survival rates of
patients with DBDC, while pancreatic and duodenal invasions affected
survival rates. Therefore, these differences in survival rates may
originate from the different invasive patterns of DBDCs.

doi: \url{https://doi.org/10.5858/arpa.2019-0218-OA}

\begin{center}\rule{0.5\linewidth}{0.5pt}\end{center}

\begin{center}\rule{0.5\linewidth}{0.5pt}\end{center}

- \textbf{Ultrastructural Characteristics of Gallbladder Epithelial
Inclusions Mimicking Cystoisospora}

\emph{American journal of clinical pathology 2020 Jan;153(1):88-93}

PubMed: \url{https://www.ncbi.nlm.nih.gov/pubmed/?term=31600399}

OBJECTIVES: There is recently reported increased prevalence of Isospora
organisms in cholecystectomy specimens from immunocompetent patients,
especially in acalculous cholecystectomies. We performed an
ultrastructural and molecular evaluation of these specimens. METHODS:
From 28 gallbladders with intraepithelial inclusions, two specimens with
diffuse involvement of the gallbladder epithelium were analyzed by
electron microscopy. Polymerase chain reaction was performed on five
samples for the ITS2 region of C belli and eukaryotic 18S region. The
18S products were sequenced by next-generation sequencing. RESULTS:
Electron microscopic analysis showed cytoplasmic condensations leading
to vacuole formation. In contrast with true C belli, there were no
identifiable organelles or organization. None of these cases showed
amplified products other than human on molecular analysis. CONCLUSIONS:
Electron microscopic analysis demonstrates that the inclusions are
condensed cytoplasmic material and not true organisms.

doi: \url{https://doi.org/10.1093/ajcp/aqz137}

\begin{center}\rule{0.5\linewidth}{0.5pt}\end{center}

\begin{center}\rule{0.5\linewidth}{0.5pt}\end{center}

- \textbf{Gallbladder and extrahepatic bile duct cancers in the
Americas: Incidence and mortality patterns and trends}

\emph{International journal of cancer 2020 Jan;():}

PubMed: \url{https://www.ncbi.nlm.nih.gov/pubmed/?term=31922259}

Trends in gallbladder cancer incidence and mortality in populations
across the Americas can provide insight into shifting epidemiologic
patterns and the current and potential impact of preventative and
curative programs. Estimates of gallbladder and extrahepatic bile duct
cancer incidence and mortality for the year 2018 were extracted from
International Agency for Research on Cancer (IARC) GLOBOCAN database for
185 countries. Recorded registry-based incidence from 13 countries was
extracted from IARCs Cancer Incidence in Five Continents series and
corresponding national deaths from the WHO mortality database. Among
females, the highest estimated incidence for gallbladder and
extrahepatic bile duct cancer in the Americas were found in Bolivia
(21.0 per 100,000), Chile (11.7) and Peru (6.0). In the US, the highest
incidence rates were observed among Hispanics (1.8). In the Chilean
population, gallbladder cancer rates declined in both females and males
between 1998 and 2012. Rates dropped slightly in Canada, Costa Rica, US
Whites and Hispanics in Los Angeles. Gallbladder cancer mortality rates
also decreased across the studied countries, although rising trends were
observed in Colombia and Canada after 2010. Countries within Southern
and Central America tended to have a higher proportion of unspecified
biliary tract cancers. In public health terms, the decline in
gallbladder cancer incidence and mortality rates is encouraging.
However, the slight increase in mortality rates during recent years in
Colombia and Canada warrant further attention. Higher proportions of
unspecified biliary tract cancers (with correspondingly higher mortality
rates) suggest more rigorous pathology procedures may be needed after
surgery.

doi: \url{https://doi.org/10.1002/ijc.32863}

\begin{center}\rule{0.5\linewidth}{0.5pt}\end{center}

\begin{center}\rule{0.5\linewidth}{0.5pt}\end{center}

Back to top

\begin{center}\rule{0.5\linewidth}{0.5pt}\end{center}

\pagebreak

\hypertarget{bileducts}{%
\subsection{Bile Ducts}\label{bileducts}}

\begin{center}\rule{0.5\linewidth}{0.5pt}\end{center}

\begin{center}\rule{0.5\linewidth}{0.5pt}\end{center}

- \textbf{Germline alterations in patients with biliary tract cancers: A
spectrum of significant and previously underappreciated findings}

\emph{Cancer 2020 Jan;126(9):1995-2002}

PubMed: \url{https://www.ncbi.nlm.nih.gov/pubmed/?term=32012241}

BACKGROUND: With limited information on germline mutations in biliary
tract cancers, this study performed somatic and germline testing for
patients at Memorial Sloan Kettering Cancer Center with known biliary
tract carcinoma with the aim of determining the frequency and range of
pathogenic germline alterations (PGAs). METHODS: Patients with biliary
tract carcinoma were consented for somatic tumor and matched blood
testing of up to 468 genes via the Memorial Sloan Kettering Cancer
Center Integrated Mutation Profiling of Actionable Cancer Targets
next-generation sequencing platform. A germline variant analysis was
performed on a panel of up to 88 genes associated with an increased
predisposition for cancer. Demographic and diagnostic details were
collected. RESULTS: Germline mutations were tested in 131 patients.
Intrahepatic cholangiocarcinoma was the most common cancer (63.4\%), and
it was followed by gallbladder adenocarcinoma (16.8\%), extrahepatic
cholangiocarcinoma (16\%), and otherwise unspecified biliary tract
cancer (3.8\%). Known and likely PGAs were present in 21 patients
(16.0\%), with 9.9\% harboring a PGA in a high/moderate-penetrance
cancer predisposition gene. Among high-penetrance cancer susceptibility
genes, PGAs were most commonly observed in BRCA1 and BRCA2 (33.3\%),
which made up 5.3\% of the entire cohort, and they were followed by
PALB2, BAP1, and PMS2. Mutations in ATM, MITF, and NBN,
moderate-penetrance cancer susceptibility genes, were identified in 1
patient each. There was no observed difference in the types of mutations
among the subtypes of biliary tract cancer. CONCLUSIONS: The frequency
of PGAs found was comparable to existing data on the prevalence of
germline mutations in other solid tumor types with matched tumor
analysis. This provides support for the role of the BRCA1/2, ATM, and
BAP1 genes in biliary tract cancer susceptibility.

doi: \url{https://doi.org/10.1002/cncr.32740}

\begin{center}\rule{0.5\linewidth}{0.5pt}\end{center}

\begin{center}\rule{0.5\linewidth}{0.5pt}\end{center}

- \textbf{DNA Flow Cytometric Analysis of Paraffin-Embedded Tissue for
the Diagnosis of Malignancy in Bile Duct Biopsies}

\emph{Human pathology 2020 Apr;():}

PubMed: \url{https://www.ncbi.nlm.nih.gov/pubmed/?term=32272125}

Differentiation of reactive versus neoplastic epithelial changes can be
challenging in bile duct biopsies. The samples are often scant,
distorted, and mixed with significant inflammation, ulceration, and/or
debris. Histological confirmation of malignancy is often required before
the initiation of surgical therapy, while an erroneous diagnosis of
malignancy can lead to unnecessary clinical management. Aneuploidy
assessment by DNA flow cytometry was performed on formalin-fixed
paraffin-embedded (FFPE) tissue from 63 bile duct biopsies: 10 with a
malignant diagnosis (7 adenocarcinoma and 3 at least high-grade
dysplasia {[}HGD{]}); 3 with an ``atypical'' diagnosis showing rare
atypical glands/cells, concerning but not definite for malignancy; 28
likely ``reactive'' biopsies with acute/chronic inflammation,
ulceration, and/or mild nuclear atypia; and 22 additional benign
biopsies without significant inflammation, ulceration, or nuclear
atypia. Aneuploidy was detected in 7 (70\%) of the 10 biopsies with
definite neoplasia (5 of 7 adenocarcinoma and 2 of 3 at least HGD), all
3 (100\%) ``atypical'' biopsies, and none of the 50 benign biopsies. All
3 ``atypical'' cases with aneuploidy were subsequently found to have
adenocarcinoma (n = 2) or HGD (n = 1). Among the 2 cases of at least HGD
with aneuploidy, 1 case developed adenocarcinoma, but no follow-up
information was available in the other case. The remaining 1 case of at
least HGD, despite having normal DNA content, was found to have
adenocarcinoma on follow-up. None of the 50 benign cases (further
supported by normal DNA content) developed adenocarcinoma within a mean
follow-up time of 37 months (range: 0-282 months). The estimated
sensitivity of aneuploidy as a diagnostic marker of malignancy
(adenocarcinoma and HGD) was 70\% with the specificity of 100\%,
positive predictive value of 100\%, and negative predictive value of
94\%. In conclusion, DNA flow cytometry using FFPE tissue from bile duct
biopsies demonstrates a high rate of aneuploidy (70\%) in malignant
cases, and normal DNA content in all benign biopsies. Although the
sample size is small, the results indicate that this assay can be
potentially useful in challenging ``atypical'' cases, where
morphological evaluation is limited by scarcity of atypical
glands/cells, inflammation, and/or ulceration.

doi: \url{https://doi.org/10.1016/j.humpath.2020.04.002}

\begin{center}\rule{0.5\linewidth}{0.5pt}\end{center}

\begin{center}\rule{0.5\linewidth}{0.5pt}\end{center}

- \textbf{Biliary intraductal tubule-forming neoplasm: A whole exome
sequencing study of MUC5AC-positive and -negative cases}

\emph{Histopathology 2020 Mar;():}

PubMed: \url{https://www.ncbi.nlm.nih.gov/pubmed/?term=32181510}

AIMS: Biliary intraductal tubular neoplasms that are non-mucinous and
negative for MUC5AC are called intraductal tubulopapillary neoplasm
(ITPNs). Intraductal tubular neoplasms with mucinous cytoplasm and
MUC5AC positivity also occur, and their nature remains unclear although
some pathologists may classify these as ``IPNBs of gastric type''. This
study aimed to elucidate genetic features of biliary intraductal tubular
neoplasms. METHODS: Six resected cases of biliary intraductal neoplasm
with \textgreater70\% tubular configuration were characterized by the
clinicopathological examination and whole exome sequencing, and obtained
findings were compared between MUC5AC-positive (n=2) and -negative cases
(n=4). RESULTS: The intraductal tumours consisted of the
pancreatobiliary-type epithelium with high-grade dysplasia arranged in
back-to-back tubules. Both two MUC5AC-negative cases were non-invasive
neoplasms developed in the liver, whereas all MUC5AC-positive cases had
invasive carcinoma and were present in the intrahepatic (n=2), perihilar
(n=1), and distal bile ducts (n=1). In an exome-sequencing study,
MUC5AC-negative cases harboured mutations in CTNNB1, SF3B1, BAP1, and
BRCA1 (one case each). KRAS mutations were observed in 3/4
MUC5AC-positive cases (75\%) but none of MUC5AC-negative neoplasms.
Compared to published data, known driver genes of other intraductal
neoplasms of the pancreatobiliary systems (e.g., APC, CTNNB1, STK11,
GNAS, and PIK3CA) were wild-type in all but one MUC5AC-negative case
with CTNNB1 mutation. Chromatin modifiers (ARID1A, BAP1, and KMT2C) were
also altered in MUC5AC-positive cases, similar to usual
cholangiocarcinomas. CONCLUSIONS: This exome-sequencing study suggested
that MUC5AC-negative biliary ITPNs are genetically distinct from
pancreatic ITPNs and IPNBs. They may also biologically differ from
MUC5AC-positive tubular neoplasms despite morphological resemblance.

doi: \url{https://doi.org/10.1111/his.14103}

\begin{center}\rule{0.5\linewidth}{0.5pt}\end{center}

\begin{center}\rule{0.5\linewidth}{0.5pt}\end{center}

- \textbf{Validation of the T category for distal cholangiocarcinoma:
Measuring the depth of invasion is complex but correlates with survival}

\emph{Annals of diagnostic pathology 2020 Mar;46():151489}

PubMed: \url{https://www.ncbi.nlm.nih.gov/pubmed/?term=32169826}

According to the current 8th edition of the American Joint Committee of
Cancer (AJCC), the T category of distal cholangiocarcinomas is
classified based on the depth of invasion (DOI) (T1, \textless{} 5~mm;
T2, between 5 and 12~mm; T3, \textgreater{} 12~mm). In consideration of
the discrepancies between previous studies about the prognostic
significance, we aimed to validate the current AJCC T staging system of
distal cholangiocarcinomas. DOI was measured using three different
methods: DOI1, DOI2, and DOI3. DOI1 was defined and stratified according
to the AJCC 8th edition. DOI2 was measured as the distance from an
imaginary curved line approximated along the distorted mucosal surface
to the deepest invasive tumor cells. DOI3 was defined as the total tumor
thickness. DOI2 and DOI3 were also divided into three categories using
the same cut-off points as in the AJCC 8th edition. We compared these
three DOI methods to the AJCC 7th edition as well. In contrast with the
AJCC 7th edition, all three groups showed a correlation with patients'
overall survival. Above all, the DOI2 group demonstrated the best
significance in multivariate analysis. However, when the C indices were
compared between these groups, differential significance proved to be
negligible (DOI1 vs DOI2, p~=~0.915; DOI2 vs DOI3, p~=~0.057).
Therefore, the measurement of DOI does not need to be rigorously and
stringently performed. In conclusion, we showed that the current T
classification system better correlates with the overall survival of
patients with distal cholangiocarcinomas than the previous system.

doi: \url{https://doi.org/10.1016/j.anndiagpath.2020.151489}

\begin{center}\rule{0.5\linewidth}{0.5pt}\end{center}

\begin{center}\rule{0.5\linewidth}{0.5pt}\end{center}

- \textbf{Global trends in intrahepatic and extrahepatic
cholangiocarcinoma incidence from 1993 to 2012}

\emph{Cancer 2020 Mar;():}

PubMed: \url{https://www.ncbi.nlm.nih.gov/pubmed/?term=32129902}

BACKGROUND: Intrahepatic cholangiocarcinomas (ICCs) and extrahepatic
cholangiocarcinomas (ECCs) are highly lethal bile duct tumors. Their
incidence can be difficult to estimate because of changes in cancer
coding over time. No studies to date have examined their global
incidence and trends with high-quality topography- and
histology-specific cancer registry data. Therefore, this study examined
ICC and ECC incidence with the Cancer Incidence in Five Continents Plus
database. METHODS: Regional and national cancer registry data were used
to estimate age-standardized incidence rates (ASRs) per 100,000
person-years, 95\% confidence intervals, and average annual percent
changes (AAPCs) for ICC in 38 countries and for ECC in 33 countries from
1993 to 2012. ICC and ECC trends were tabulated and plotted by country.
Rates versus birth cohort by age were plotted, and an age-period-cohort
analysis was performed to assess age and cohort incidence rate ratios.
RESULTS: The highest rates of ICC and ECC were in Asia, specifically
South Korea (ASR for ICC, 2.80; ASR for ECC, 2.24), Thailand (ASR for
ICC, 2.19; ASR for ECC, 0.71), and Japan (ASR for ICC, 0.95; ASR for
ECC, 0.83). Between 1993 and 2012, incidence rates of both ICC and ECC
increased in most countries. The largest ASR increases over the study
period occurred in Latvia (AAPC, 20.1\%) and China (AAPC, 11.1\%) for
ICC and in Thailand (AAPC, 8.8\%) and Colombia (AAPC, 8.5\%) for ECC.
CONCLUSIONS: In the 20~years examined, ICC and ECC incidence increased
in the majority of countries worldwide. ICC and ECC incidence may
continue to increase because of metabolic and infectious etiologic
factors. Efforts to further elucidate risk factors contributing to these
increases in incidence are warranted.

doi: \url{https://doi.org/10.1002/cncr.32803}

\begin{center}\rule{0.5\linewidth}{0.5pt}\end{center}

\begin{center}\rule{0.5\linewidth}{0.5pt}\end{center}

- \textbf{Intraductal papillary neoplasms of the bile duct consist of
two distinct types specifically associated with clinicopathological
features and molecular phenotypes}

\emph{The Journal of pathology 2020 Feb;():}

PubMed: \url{https://www.ncbi.nlm.nih.gov/pubmed/?term=32100878}

Intraductal papillary neoplasm of the bile duct (IPNB) is a grossly
visible papillary biliary neoplasm with morphological variations and
occasional invasion. Recently a new classification of IPNB into type 1
and type 2 was proposed in which the type 1 IPNBs consist of fine
papillary neoplastic glands and the type 2 IPNBs consist of complex
branching glands, seldom with foci of solid-tubular components. However,
clinicopathological and molecular characteristics of these types of
IPNBs are yet to be identified. We aimed to uncover clinicopathological
and molecular characteristics of the types of IPNBs. Thirty-six IPNBs
were studied retrospectively. Clinicopathological features as well as
molecular alterations of 31 genes were evaluated by means of targeted
next-generation sequencing and immunohistochemical examination of
expression of mucin and cancer-associated molecules. The 36 IPNBs were
classified into 22 of type 1 and 14 of type 2. The type 1 IPNBs were
associated with a non-invasive phenotype, intestinal and oncocytic
subtypes, development in the intrahepatic bile duct, overt mucin
production, and a relatively good prognosis. The type 2 IPNBs were
associated with an invasive phenotype, the pancreatobiliary subtype,
development within the extrahepatic bile duct, and worse prognosis
compared with the type 1 IPNBs. In the molecular analysis, recurrent
mutations were found in TP53 (34.3\%), KRAS (31.4\%), STK11 (25.7\%),
CTNNB1 (17.1\%), APC (14.3\%), SMAD4 (14.3\%), GNAS (11.4\%), PBRM1
(11.4\%), ELF3 (8.6\%), KMT2C (8.6\%), NF1 (8.6\%), PIK3CA (8.6\%),
ARID1A (5.7\%), ARID2 (5.7\%), BAP1 (5.7\%), BRAF (5.7\%), EPHA6
(5.7\%), ERBB2 (5.7\%), ERBB3 (5.7\%), KMT2D (5.7\%), and RNF43 (5.7\%).
Mutations in KRAS and GNAS were enriched in the type 1 IPNBs, whereas
mutations in TP53, SMAD4, and KMT2C were enriched in the type 2 IPNBs.
These results indicate that IPNBs consist of two distinct types of
neoplasms specifically associated with clinicopathological features and
molecular phenotypes. © 2020 Pathological Society of Great Britain and
Ireland. Published by John Wiley \& Sons, Ltd.

doi: \url{https://doi.org/10.1002/path.5398}

\begin{center}\rule{0.5\linewidth}{0.5pt}\end{center}

\begin{center}\rule{0.5\linewidth}{0.5pt}\end{center}

- \textbf{DNAJB1-PRKACA fusions occur in oncocytic pancreatic and
biliary neoplasms and are not specific for fibrolamellar hepatocellular
carcinoma}

\emph{Modern pathology : an official journal of the United States and
Canadian Academy of Pathology, Inc 2020 04;33(4):648-656}

PubMed: \url{https://www.ncbi.nlm.nih.gov/pubmed/?term=31676785}

Recently discovered DNAJB1-PRKACA oncogenic fusions have been considered
diagnostic for fibrolamellar hepatocellular carcinoma. In this study, we
describe six pancreatobiliary neoplasms with PRKACA fusions, five of
which harbor the DNAJB1-PRKACA fusion. All neoplasms were subjected to a
hybridization capture-based next-generation sequencing assay
(MSK-IMPACT), which enables the identification of sequence mutations,
copy number alterations, and selected structural rearrangements
involving ≥410 genes (n = 6) and/or to a custom targeted, RNA-based
panel (MSK-Fusion) that utilizes Archer Anchored Multiplex PCR
technology and next-generation sequencing to detect gene fusions in 62
genes (n = 2). Selected neoplasms also underwent FISH analysis, albumin
mRNA in-situ hybridization, and arginase-1 immunohistochemical labeling
(n = 3). Five neoplasms were pancreatic, and one arose in the
intrahepatic bile ducts. All revealed at least focal oncocytic
morphology: three cases were diagnosed as intraductal oncocytic
papillary neoplasms, and three as intraductal papillary mucinous
neoplasms with mixed oncocytic and pancreatobiliary or gastric features.
Four cases had an invasive carcinoma component composed of oncocytic
cells. Five cases revealed DNAJB1-PRKACA fusions and one revealed an
ATP1B1-PRKACA fusion. None of the cases tested were positive for albumin
or arginase-1. Our data prove that DNAJB1-PRKACA fusion is neither
exclusive nor diagnostic for fibrolamellar hepatocellular carcinoma, and
caution should be exercised in diagnosing liver tumors with
DNAJB1-PRKACA fusions as fibrolamellar hepatocellular carcinoma,
particularly if a pancreatic lesion is present. Moreover, considering
DNAJB1-PRKACA fusions lead to upregulated protein kinase activity and
that this upregulated protein kinase activity has a significant role in
tumorigenesis of fibrolamellar hepatocellular carcinoma, protein kinase
inhibition could have therapeutic potential in the treatment of these
pancreatobiliary neoplasms as well, once a suitable drug is developed.

doi: \url{https://doi.org/10.1038/s41379-019-0398-2}

\begin{center}\rule{0.5\linewidth}{0.5pt}\end{center}

\begin{center}\rule{0.5\linewidth}{0.5pt}\end{center}

- \textbf{Clinicopathologic and Prognostic Significance of Gallbladder
and Cystic Duct Invasion in Distal Bile Duct Carcinoma}

\emph{Archives of pathology \& laboratory medicine 2019 Nov;():}

PubMed: \url{https://www.ncbi.nlm.nih.gov/pubmed/?term=31755778}

CONTEXT.---: The roles of the gallbladder and cystic duct (CD) invasions
in distal bile duct carcinoma (DBDC) have not been well elucidated.
OBJECTIVE.---: To define the characteristics and prognostic significance
of gallbladder or CD invasions in patients with DBDC. DESIGN.---: Organ
invasion patterns with clinicopathologic features were assessed in 258
resected DBDCs. RESULTS.---: CD invasions (N = 31) were associated with
frequent concomitant pancreatic and/or duodenal invasions (23 of 31,
74\%) and showed stromal infiltration (16 of 31, 52\%) and intraductal
cancerization (15 of 31, 48\%) patterns. In only 2 cases, invasions with
intraductal cancerization were observed in the gallbladder neck.
Conversely, all pancreatic (N = 175) and duodenal (83) invasions
developed through stromal infiltration. CD invasions were associated
with larger tumor size (P = .001), bile duct margin positivity (P =
.001), perineural invasions (P = .04), and higher N categories (P =
.007). Patients with pancreatic or duodenal invasions had significantly
lower survival rates than those without pancreatic (median, 31.0 versus
93.9 months) or duodenal (27.5 versus 56.8 months, P \textless{} .001,
both) invasions. However, those with gallbladder or CD invasions did not
have different survival times (P = .13). Patients with concomitant
gallbladder/CD and pancreatic/duodenal invasions demonstrated
significantly lower survival rates than those without organ invasions (P
\textless{} .001). CONCLUSIONS.---: Gallbladder invasions were rare in
DBDCs as neck invasions with intraductal cancerization. CD invasions
occurred by stromal infiltrations and intraductal cancerization, whereas
all pancreatic and duodenal invasions had stromal infiltration patterns.
Gallbladder and/or CD invasions did not affect survival rates of
patients with DBDC, while pancreatic and duodenal invasions affected
survival rates. Therefore, these differences in survival rates may
originate from the different invasive patterns of DBDCs.

doi: \url{https://doi.org/10.5858/arpa.2019-0218-OA}

\begin{center}\rule{0.5\linewidth}{0.5pt}\end{center}

\begin{center}\rule{0.5\linewidth}{0.5pt}\end{center}

- \textbf{Recurrent Rearrangements in PRKACA and PRKACB in Intraductal
Oncocytic Papillary Neoplasms of the Pancreas and~Bile Duct}

\emph{Gastroenterology 2020 02;158(3):573-582.e2}

PubMed: \url{https://www.ncbi.nlm.nih.gov/pubmed/?term=31678302}

BACKGROUND \& AIMS: Intraductal oncocytic papillary neoplasms (IOPNs) of
the pancreas and bile duct contain epithelial cells with numerous, large
mitochondria and are cystic precursors to pancreatic ductal
adenocarcinoma (PDAC) and cholangiocarcinoma (CCA), respectively.
However, IOPNs do not have the genomic alterations found in other
pancreatobiliary neoplasms. In fact, no recurrent genomic alterations
have been described in IOPNs. PDACs without activating mutations in KRAS
contain gene rearrangements, so we investigated whether IOPNs have
recurrent fusions in genes. METHODS: We analyzed 20 resected pancreatic
IOPNs and 3 resected biliary IOPNs using a broad RNA-based targeted
sequencing panel to detect cancer-related fusion genes. Four invasive
PDACs and 2 intrahepatic CCAs from the same patients as the IOPNs, were
also available for analysis. Samples of pancreatic cyst fluid (n~= 5,
collected before surgery) and bile duct brushings (n~= 2) were analyzed
for translocations. For comparison, we analyzed pancreatobiliary lesions
from 126 patients without IOPN (controls). RESULTS: All IOPNs evaluated
were found to have recurring fusions of ATP1B1-PRKACB (n~= 13),
DNAJB1-PRKACA (n~= 6), or ATP1B1-PRKACA (n~= 4). These fusions also were
found in corresponding invasive PDACs and intrahepatic CCAs, as well as
in matched pancreatic cyst fluid and bile duct brushings. These gene
rearrangements were absent from all 126 control pancreatobiliary
lesions. CONCLUSIONS: We identified fusions in PRKACA and PRKACB genes
in pancreatic and biliary IOPNs, as well as in PDACs and pancreatic cyst
fluid and bile duct cells from the same patients. We did not identify
these gene fusions in 126 control pancreatobiliary lesions. These
fusions might be used to identify patients at~risk~for~IOPNs and their
associated invasive carcinomas.

doi: \url{https://doi.org/10.1053/j.gastro.2019.10.028}

\begin{center}\rule{0.5\linewidth}{0.5pt}\end{center}

\begin{center}\rule{0.5\linewidth}{0.5pt}\end{center}

Back to top

\begin{center}\rule{0.5\linewidth}{0.5pt}\end{center}

\pagebreak

\hypertarget{ampulla}{%
\subsection{Ampulla}\label{ampulla}}

\begin{center}\rule{0.5\linewidth}{0.5pt}\end{center}

\begin{center}\rule{0.5\linewidth}{0.5pt}\end{center}

- \textbf{Prognostic significance of stem cell/ epithelial-mesenchymal
transition markers in periampullary/pancreatic cancers: FGFR1 is a
promising prognostic marker}

\emph{BMC cancer 2020 Mar;20(1):216}

PubMed: \url{https://www.ncbi.nlm.nih.gov/pubmed/?term=32171280}

BACKGROUND: Periampullary cancers (PAC) including pancreatic, ampulla of
Vater (AOV), and common bile duct (CBD) cancers are highly aggressive
with a lack of useful prognostic markers beyond T stage. However, T
staging can be biased due to the anatomic complexity of this region.
Recently, several markers related to cancer stem cells and
epithelial-mesenchymal transition (EMT) such as octamer transcription
factor-4 (Oct4) and fibroblast growth factor receptor 1 (FGFR1)
respectively, have been proposed as new promising markers in other solid
cancers. The aim of this study was to assess the expression and
prognostic significance of stem cell/EMT markers in PACs. METHODS:
Formalin-fixed, paraffin-embedded tissues of surgically excised PACs
from the laboratory archives from 1998 to 2014 were evaluated by
immunohistochemical staining for stem cell/EMT markers using tissue
microarray. The clinicopathologic parameters were documented and
statistically analyzed with the immunohistochemical findings. Survival
and recurrence data were collected and analyzed. RESULTS: A total of 126
PAC cases were evaluated. The average age was 63 years, with 76 male and
50 female patient samples. Age less than 74 years, AOV cancers, lower T
\& N stage, lower tumor size, no lymphatic, vascular, perineural
invasion and histologic well differentiation, intestinal type, no
fibrosis, severe inflammation were significantly associated with the
better overall survival High expression levels of FGFR1 as well as CK20,
CDX2, and VEGF were significantly related to better overall survival,
while other stem cell markers were not related. Similar findings were
observed for tumor recurrence using disease-free survival. CONCLUSIONS:
In addition to other clinicopathologic parameters, severe fibrosis was
related to frequent tumor recurrence, and high FGFR1 expression was
associated with better overall survival. Histologic changes such as
extensive fibrosis need to be investigated further in relation to EMT of
PACs.

doi: \url{https://doi.org/10.1186/s12885-020-6673-2}

\begin{center}\rule{0.5\linewidth}{0.5pt}\end{center}

\begin{center}\rule{0.5\linewidth}{0.5pt}\end{center}

Back to top

\begin{center}\rule{0.5\linewidth}{0.5pt}\end{center}

\pagebreak

\hypertarget{feedback}{%
\section{Feedback}\label{feedback}}

\href{https://docs.google.com/forms/d/e/1FAIpQLSeD3Z9J6Y7eMmiyM12f_SfAmHUlykb1zxZcwO6lg7cebGYQIQ/viewform}{Google
Feedback Form}

\hypertarget{disqus_thread}{}

Please enable JavaScript to view the comments powered by Disqus.

\begin{center}\rule{0.5\linewidth}{0.5pt}\end{center}

Back to top

\begin{center}\rule{0.5\linewidth}{0.5pt}\end{center}

\pagebreak

\end{document}
