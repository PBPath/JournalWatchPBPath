\documentclass[]{article}
\usepackage{lmodern}
\usepackage{amssymb,amsmath}
\usepackage{ifxetex,ifluatex}
\usepackage{fixltx2e} % provides \textsubscript
\ifnum 0\ifxetex 1\fi\ifluatex 1\fi=0 % if pdftex
  \usepackage[T1]{fontenc}
  \usepackage[utf8]{inputenc}
\else % if luatex or xelatex
  \ifxetex
    \usepackage{mathspec}
  \else
    \usepackage{fontspec}
  \fi
  \defaultfontfeatures{Ligatures=TeX,Scale=MatchLowercase}
\fi
% use upquote if available, for straight quotes in verbatim environments
\IfFileExists{upquote.sty}{\usepackage{upquote}}{}
% use microtype if available
\IfFileExists{microtype.sty}{%
\usepackage{microtype}
\UseMicrotypeSet[protrusion]{basicmath} % disable protrusion for tt fonts
}{}
\usepackage[margin=1in]{geometry}
\usepackage{hyperref}
\hypersetup{unicode=true,
            pdftitle={Pancreatobiliary Pathology Society Journal Watch},
            pdfborder={0 0 0},
            breaklinks=true}
\urlstyle{same}  % don't use monospace font for urls
\usepackage{graphicx,grffile}
\makeatletter
\def\maxwidth{\ifdim\Gin@nat@width>\linewidth\linewidth\else\Gin@nat@width\fi}
\def\maxheight{\ifdim\Gin@nat@height>\textheight\textheight\else\Gin@nat@height\fi}
\makeatother
% Scale images if necessary, so that they will not overflow the page
% margins by default, and it is still possible to overwrite the defaults
% using explicit options in \includegraphics[width, height, ...]{}
\setkeys{Gin}{width=\maxwidth,height=\maxheight,keepaspectratio}
\IfFileExists{parskip.sty}{%
\usepackage{parskip}
}{% else
\setlength{\parindent}{0pt}
\setlength{\parskip}{6pt plus 2pt minus 1pt}
}
\setlength{\emergencystretch}{3em}  % prevent overfull lines
\providecommand{\tightlist}{%
  \setlength{\itemsep}{0pt}\setlength{\parskip}{0pt}}
\setcounter{secnumdepth}{0}
% Redefines (sub)paragraphs to behave more like sections
\ifx\paragraph\undefined\else
\let\oldparagraph\paragraph
\renewcommand{\paragraph}[1]{\oldparagraph{#1}\mbox{}}
\fi
\ifx\subparagraph\undefined\else
\let\oldsubparagraph\subparagraph
\renewcommand{\subparagraph}[1]{\oldsubparagraph{#1}\mbox{}}
\fi

%%% Use protect on footnotes to avoid problems with footnotes in titles
\let\rmarkdownfootnote\footnote%
\def\footnote{\protect\rmarkdownfootnote}

%%% Change title format to be more compact
\usepackage{titling}

% Create subtitle command for use in maketitle
\providecommand{\subtitle}[1]{
  \posttitle{
    \begin{center}\large#1\end{center}
    }
}

\setlength{\droptitle}{-2em}

  \title{Pancreatobiliary Pathology Society Journal Watch}
    \pretitle{\vspace{\droptitle}\centering\huge}
  \posttitle{\par}
  \subtitle{February March 2019}
  \author{}
    \preauthor{}\postauthor{}
      \predate{\centering\large\emph}
  \postdate{\par}
    \date{Last Update on 2019-04-30}


\begin{document}
\maketitle

{
\setcounter{tocdepth}{5}
\tableofcontents
}
\begin{center}\rule{0.5\linewidth}{\linethickness}\end{center}

\hypertarget{pbpath-journal-watch-articles}{%
\section{PBPath Journal Watch
Articles}\label{pbpath-journal-watch-articles}}

\begin{center}\rule{0.5\linewidth}{\linethickness}\end{center}

\textbf{Wellcome to the PBPath Journal Watch!}

This bi-monthly journal watch features exciting recently published
pancreas and biliary pathology articles that will provide up to date
medical knowledge in our field. These articles will be showcased in
several convenient categories, including surgical pathology,
cytopathology, and molecular pathology among others. The articles in
each category are in no particular order. See the list of journals we
search regularly \href{http://pbpath.org/pbpath-journal-watch/}{here}.
Previous months' issues may be found in our
\emph{\href{http://pbpath.org/journal-watch-archive/}{archive}} and you
may see
\href{http://pbpath.org/journal-watch-upcoming-issue/}{preparation of
upcoming issue here}.

We encourage members to actively participate by recommending new
articles and providing feedback using the forms provided below.

\textbf{We hope that you will enjoy the new PBPath Journal Watch!}

\begin{center}\rule{0.5\linewidth}{\linethickness}\end{center}

\textbf{Feedback}

Please send your feedbacks using the forms below:

\href{https://docs.google.com/forms/d/e/1FAIpQLSeD3Z9J6Y7eMmiyM12f_SfAmHUlykb1zxZcwO6lg7cebGYQIQ/viewform}{Google
Feedback Form}

\hypertarget{disqus_thread}{}

Please enable JavaScript to view the comments powered by Disqus.

\begin{center}\rule{0.5\linewidth}{\linethickness}\end{center}

Back to top

\begin{center}\rule{0.5\linewidth}{\linethickness}\end{center}

\pagebreak

\hypertarget{surgical-pathology}{%
\subsection{Surgical Pathology}\label{surgical-pathology}}

\begin{center}\rule{0.5\linewidth}{\linethickness}\end{center}

\hypertarget{pancreas}{%
\subsubsection{Pancreas}\label{pancreas}}

\begin{center}\rule{0.5\linewidth}{\linethickness}\end{center}

\hypertarget{morphology-diagnostics-ihc}{%
\paragraph{Morphology, Diagnostics,
IHC}\label{morphology-diagnostics-ihc}}

Morphology, Diagnostics, IHC

\begin{center}\rule{0.5\linewidth}{\linethickness}\end{center}

 - \textbf{Prognostic evaluation of pancreatic ductal adenocarcinoma:
Associations between molecular biomarkers and CT imaging findings}

\emph{Pancreatology : official journal of the International Association
of Pancreatology (IAP) \ldots{} {[}et al.{]} 2019 Mar;19(2):331-339}

PubMed: \url{https://www.ncbi.nlm.nih.gov/pubmed/?term=30718187}

OBJECTIVES: To investigate association between molecular biomarkers and
computed tomography (CT) imaging findings in patients with pancreatic
ductal adenocarcinoma (PDAC). METHODS: Fifty-three consecutive patients
with PDAC (34 men and 19 women; mean age, 70.6\,±\,8.1 years; range,
56-86 years) who underwent dynamic contrast-enhanced CT prior to
pancreatectomy were included. The Ki-67 index and expressions of
E-cadherin, Vimentin, and TWIST were immunohistochemically evaluated.
Qualitative image analysis and histogram analysis of CT numbers were
conducted. Clinical and molecular biomarkers were tested as possible
prognostic factors for overall survival (OS) using Kaplan-Meier method
and Cox proportional hazards regression. In addition, associations
between CT imaging findings and significant molecular biomarkers were
investigated. RESULTS: The TNM stage (P\,=\,0.018) and E-cadherin
expression status (P\,=\,0.018) were independently associated with OS.
E-cadherin-negative PDACs had a worse prognosis than E-cadherin-positive
PDACs (hazard ratio: 2.21). Irregular tumor margin was observed more
frequently in E-cadherin-negative PDACs (54.7\%) than in
E-cadherin-positive PDACs (45.3\%) (P\,=\,0.00054). The kurtosis of CT
number during the pancreatic parenchymal phase was significantly higher
in E-cadherin-negative PDACs than in E-cadherin-positive PDACs
(P\,=\,0.035). CONCLUSIONS: E-cadherin suppression was found to be a
prognostic factor for OS in patients with PDAC, and irregular tumor
margin and kurtosis of CT numbers during the pancreatic parenchymal
phase could be indicators for E-cadherin suppression.

{}

{}

\begin{center}\rule{0.5\linewidth}{\linethickness}\end{center}

 - \textbf{Microvessel Density and Impact of Angiogenesis on Survival of
Resected Pancreatic Cancer Patients: A Systematic Review and
Meta-analysis}

\emph{Pancreas 2019 02;48(2):233-241}

PubMed: \url{https://www.ncbi.nlm.nih.gov/pubmed/?term=30629030}

OBJECTIVES: Angiogenesis plays a major role in tumor progression and
metastasis; however, its role in pancreatic cancer (PC) remains unclear.
The aim of the study was to explore the cumulative evidence concerning
the impact of microvessel density (MVD), an estimator of angiogenesis,
on resected PC patients. METHODS: A systematic review of literature and
a meta-analysis of relevant reports were performed. Overall survival and
disease-free survival were scrutinized. RESULTS: One thousand five
hundred patients were analyzed. Overall survival (hazard ratio, 2.0;
95\% confidence interval, 1.57-2.54; P \textless{} 0.001) and
disease-free survival (hazard ratio, 1.99; 95\% confidence interval,
1.24-3.2; P = 0.004) were in favor of the low-MVD group. Use of CD105
antibody and of a computerized image analysis system was found to
significantly reduce the heterogeneity. Disease staging, tumor location,
and grading showed significant effect on survival. CONCLUSIONS: High-MVD
expression was strongly associated with poorer survival and recurrence
among resected PC patients, demonstrating a negative prognostic value.
Use of CD105 antibody and of a computerized image analysis system is
recommended in future studies because they reduce heterogeneity of
results. The potential role of MVD as a marker to select PC patients who
would benefit from antiangiogenetic treatment should be further explored
in clinical trials.

{}

{}

\begin{center}\rule{0.5\linewidth}{\linethickness}\end{center}

Back to top

\begin{center}\rule{0.5\linewidth}{\linethickness}\end{center}

\pagebreak

\hypertarget{staging}{%
\paragraph{Staging}\label{staging}}

Pancreas TNM staging, Margins, Survival

\begin{center}\rule{0.5\linewidth}{\linethickness}\end{center}

 - \textbf{Association Between Very Small Tumor Size and Decreased
Overall Survival in Node-Positive Pancreatic Cancer}

\emph{Annals of surgical oncology 2018 Dec;25(13):4027-4034}

PubMed: \url{https://www.ncbi.nlm.nih.gov/pubmed/?term=30298331}

BACKGROUND: In pancreatic adenocarcinoma (PDAC), increasing tumor size
usually correlates with a worse prognosis. However, patients with a very
small primary tumor who experience lymph node involvement may have a
different disease biology. This study sought to determine the
interaction between tumor size and lymph node involvement in terms of
overall survival (OS). METHODS: The study identified 17,073 patients
with a diagnosis of M0 resected PDAC between 1983 and 2013 using the
Surveillance, Epidemiology, and End Results database. The patients were
stratified by lymph node involvement (N0 vs N+) and T stage (T1a-T1b vs
T1c vs T2 vs T3 vs T4). The Kaplan-Meier method was used to estimate OS,
and Cox regression analysis was used to compare survival between
subgroups after adjustment for patient-specific factors. RESULTS: Lymph
node involvement and T stage significantly interacted
(p \textless{} 0.001). Among the patients with node-negative disease,
5-year OS decreased monotonically with increasing T stage (59.1\%,
30.6\%, 22.9\%, 16.6\%, and 8.0\%, respectively; p \textless{} 0.001).
In contrast, among the patients with node-positive disease, those with
T1a-T1b tumors (\textless{} 10~mm) had worse 5-year OS than those with
T1c tumors (7.4\% vs 17.6\%; adjusted hazard ratio, 0.70; 95\%
confidence interval, 0.50-0.97; p = 0.034) and similar survival compared
with those who had T2, T3, or T4 tumors (9.7\%, 8.2\%, and 4.8\%,
respectively; p \textgreater{} 0.2 in all cases). CONCLUSIONS: Among
patients with lymph node-positive PDAC, very small primary tumors are
associated with decreased OS. This finding raises the possibility that
small tumors capable of lymph node metastasis might represent more
biologically aggressive cancers.

{}

{}

\begin{center}\rule{0.5\linewidth}{\linethickness}\end{center}

 - \textbf{Validation of the eighth edition of the American Joint
Committee on Cancer staging system and proposal of an improved staging
system for pancreatic ductal adenocarcinoma}

\emph{Annals of hepato-biliary-pancreatic surgery 2019 Feb;23(1):46-55}

PubMed: \url{https://www.ncbi.nlm.nih.gov/pubmed/?term=30863807}

Backgrounds/Aims: This study aimed to validate the eighth edition of the
American Joint Committee on Cancer (AJCC) staging system for pancreatic
adenocarcinoma and to propose an improved staging system for this
disease. Methods: Between 2000 and 2014, 1656 patients underwent
surgical resection for pancreatic ductal adenocarcinoma at Asan Medical
Center, Seoul, South Korea. The 1169 patients included in this study
were recategorized according to the eighth edition of the AJCC staging
system. Patients were also categorized according to a new staging
system, based on tumor size and number of metastatic lymph nodes.
Results: The seventh edition of the AJCC staging system categorized
93.7\% of patients as having stage T3 tumors. Stages were distributed
more evenly with the eighth edition. In the N0 group, classification
according to the seventh edition showed no statistically significant
differences in survival rate between patients with T1 and T2 (p=0.717)
and with IA and IB (p=0.717) tumors. Survival rates classified according
to the eighth edition differed significantly for all pairs of T stages
(p\textless{}0.05). With both editions, N stages showed statistically
significant differences (p\textless{}0.05). Reanalysis showed that a
staging system using a tumor size ≥3 cm and ≥1 metastatic lymph nodes
was more predictive of survival rates. Conclusions: Compared with the
seventh edition, the eighth edition of the AJCC staging system for
pancreatic adenocarcinoma showed a more even distribution in T stage but
marginal differences in other stages. The proposed system, using tumor
size and number of metastatic lymph nodes, was better at predicting
survival.

{}

{}

\begin{center}\rule{0.5\linewidth}{\linethickness}\end{center}

 - \textbf{Should a standard lymphadenectomy include the No.~9 lymph
nodes for body and tail pancreatic ductal adenocarcinoma?}

\emph{Pancreatology : official journal of the International Association
of Pancreatology (IAP) \ldots{} {[}et al.{]} 2019 Apr;19(3):414-418}

PubMed: \url{https://www.ncbi.nlm.nih.gov/pubmed/?term=30902419}

OBJECTIVES: This study aimed to use a retrospective data base to
investigate whether a standard lymphadenectomy during distal
pancreatectomy should include the No.~9 lymph nodes (LNs) for resectable
pancreatic ductal adenocarcinoma (PDAC) located in the body and tail of
the pancreas. METHODS: Data from 169 patients undergoing curative distal
pancreatectomy for PDAC between Jan 1, 2013 and Dec 31, 2016 were
collected. According to the tumor location, patients were divided into
three groups: pancreatic neck tumor, pancreatic body and tail tumor with
margin-to-bifurcation-distance (MTBD)\,≤\,2.5\,cm and pancreatic body
and tail tumor with MTBD\,\textgreater{}\,2.5\,cm. The metastatic rate
of the No.~9 LNs was compared among the 3 groups. The survival outcomes
were analyzed. RESULTS: The involvement rate for No.~9 LNs was 20.7\%
(6/29) for pancreatic neck tumors, 17.6\% (15/85) for body and tail
tumors with MTBD\,≤\,2.5\,cm and 1.8\% (1/55) for
MTBD\,\textgreater{}\,2.5\,cm. The No.~9 LNs were significantly more
frequently involved in neck or body and tail tumors with MTBD ≤2.5\,cm
than with the cases with MTBD \textgreater{}2.5\,cm (OR 0.082,
P\,=\,0.016). No.~9 LN involvement was not associated with worse
survival compared with survival associated with involvement of other LNs
(P\,=\,0.780). CONCLUSIONS: For PDAC located in the neck or in the body
and tail of the pancreas with MTBD\,≤\,2.5\,cm, the involvement rate for
No.~9 LNs is high. Standard lymphadenectomy should include the No.~9
LNs.

{}

{}

\begin{center}\rule{0.5\linewidth}{\linethickness}\end{center}

Back to top

\begin{center}\rule{0.5\linewidth}{\linethickness}\end{center}

\pagebreak

\hypertarget{microenvironment}{%
\paragraph{Microenvironment}\label{microenvironment}}

Tumor Stroma Interactions, Microenvironment, Inflammatory Response

\begin{center}\rule{0.5\linewidth}{\linethickness}\end{center}

 - \textbf{Immune cell score in pancreatic cancer-comparison of hotspot
and whole-section techniques}

\emph{Virchows Archiv : an international journal of pathology 2019
Mar;():}

PubMed: \url{https://www.ncbi.nlm.nih.gov/pubmed/?term=30843106}

An immune cell score (ICS) was introduced for predicting survival in
pancreatic ductal adenocarcinoma (PDAC). Few studies have compared
different methods of evaluating immune infiltrate. This study compared
ICSs determined in whole sections or tissue microarray-like hotspots for
predicting survival after PDAC surgery. We included in 79 consecutive
patients from a single geographical area that underwent surgery for PDAC
(R0/R1, stages I-III). We performed digital image analyses to evaluate
CD3 and CD8 staining. ICSs were classified as low, moderate, or high,
based on the numbers of immune cells in the tumour core and invasive
margin. We compared ICS groups determined with the hotspot and
whole-section techniques. Associations between ICS and survival were
analysed with Cox regression models, adjusted for sex, age, tumour
stage, differentiation grade, perineural invasion, and resection
radicality. In hotspot ICS analysis, 5-year overall survival rates for
low, moderate, and high groups were 12.1\%, 26.3\%, and 26.8\%,
respectively (p = 0.193). In whole-section analyses, overall survival
rates were 5.3\%, 26.4\%, and 43.8\%, respectively (p = 0.030). In the
adjusted Cox model, whole-section ICS groups were inversely associated
with the overall mortality hazard ratio (HR): low, moderate, and high
ICS groups had HRs of 1.00, 0.42 (95\% CI 0.20-0.88), and 0.27 (95\% CI
0.11-0.67), respectively. The number of immune cells per square
millimetre in the tumour core and the invasive margin were significantly
higher and had a wider range in hotspots than in whole-tissue sections.
Accordingly, ICS could predict survival in patients with PDAC after
surgery. Whole tissue section ICSs exhibited better prognostic value
than hotspot ICSs.

{}

{}

\begin{center}\rule{0.5\linewidth}{\linethickness}\end{center}

 - \textbf{Fibroblasts in Pancreatic Ductal Adenocarcinoma: Biological
Mechanisms and Therapeutic Targets}

\emph{Gastroenterology 2019 May;156(7):2085-2096}

PubMed: \url{https://www.ncbi.nlm.nih.gov/pubmed/?term=30721663}

The desmoplastic reaction of pancreas cancer may begin as a wound
healing response to the nascent neoplasm, but it soon creates an
insidious shelter that can sustain the growing tumor and rebuff therapy.
Among the many cell types subverted by transformed epithelial cells,
fibroblasts are recruited and activated to lay a foundation of
extracellular matrix proteins and glycosaminoglycans that alter tumor
biophysics and signaling. Their near-universal presence in pancreas
cancer and ostensible support of disease progression make fibroblasts
attractive therapeutic targets. More recently, however, it has also
become apparent that diverse subpopulations of fibroblasts with distinct
phenotypes and secretomes inhabit the stroma, and that targeted
depletion of particular fibroblast subsets could either provide
substantial therapeutic benefit or accelerate disease progression. An
improved characterization of these fibroblast subtypes, along with their
potential relationships to tumor subtypes and mutational repertoires, is
needed in order to make anti-fibroblast therapies clinically viable.

{}

{}

\begin{center}\rule{0.5\linewidth}{\linethickness}\end{center}

Back to top

\begin{center}\rule{0.5\linewidth}{\linethickness}\end{center}

\pagebreak

\hypertarget{bile-ducts}{%
\subsubsection{Bile Ducts}\label{bile-ducts}}

\begin{center}\rule{0.5\linewidth}{\linethickness}\end{center}

\hypertarget{morphology-diagnostics-ihc-1}{%
\paragraph{Morphology, Diagnostics,
IHC}\label{morphology-diagnostics-ihc-1}}

Morphology, Diagnostics, IHC

\begin{center}\rule{0.5\linewidth}{\linethickness}\end{center}

 - \textbf{Overexpression of matriptase in tumor stroma is a poor
prognostic indicator of extrahepatic bile duct cancer}

\emph{Pathology international 2019 Feb;69(2):86-93}

PubMed: \url{https://www.ncbi.nlm.nih.gov/pubmed/?term=30729623}

Bile duct cancer is known to contain numerous fibroblasts, and reported
to recruit cancer- associated fibroblasts by secreting platelet-derived
growth factor-D (PDGF-D) which needs serine proteases, such as
matriptase, to behave as a ligand. However, their expression pattern,
and prognostic value have not been clarified. In this study, we
investigated the clinicopathological significance of PDGF-D and
matriptase expression in patients with extrahepatic bile duct cancer.
The samples were obtained from 256 patients who underwent the surgical
resection between 1991 and 2015, and the expression levels of PDGF-D and
matriptase were evaluated immunohistochemically. Staining intensities
and distribution were scored, and finally classified into low and high
expression groups in cancer cells and stroma respectively. High
expression of matriptase in the cancer stroma was detected in 91 tumors
(40\%). The high stromal matriptase expression was significantly
associated with shorter recurrence-free survival (RFS) and overall
survival (OS) (P = 0.0027 and 0.0023, respectively). Multivariate
analyses also demonstrated that the stromal matriptase expression level
was an independent influential factor in RFS (P = 0.0050) and OS
(P = 0.0093). Our findings suggest that the high stromal matriptase
expression was strongly associated with tumor progression, recurrence
and poor outcomes in patients with extrahepatic bile duct cancer.

{}

{}

\begin{center}\rule{0.5\linewidth}{\linethickness}\end{center}

Back to top

\begin{center}\rule{0.5\linewidth}{\linethickness}\end{center}

\pagebreak

\hypertarget{gallbladder}{%
\subsubsection{Gallbladder}\label{gallbladder}}

\hypertarget{staging-1}{%
\paragraph{Staging}\label{staging-1}}

Gallbladder TNM staging, Margins, Survival

\begin{center}\rule{0.5\linewidth}{\linethickness}\end{center}

 - \textbf{The optimal number of lymph nodes to evaluate among patients
undergoing surgery for gallbladder cancer: Correlating the number of
nodes removed with survival in 6531 patients}

\emph{Journal of surgical oncology 2019 Mar;():}

PubMed: \url{https://www.ncbi.nlm.nih.gov/pubmed/?term=30864246}

BACKGROUND: The aim of the current study was to identify the minimum
number and the optimal range of lymph nodes (LNs) to be examined among
patients with gallbladder cancer (GBC). METHODS: Between January 1,
2004, and December 31, 2015, patients with a diagnosis of GBC were
identified in the National Cancer Database. A machine-based learning
approach was used to identify the minimum number and range of LNs to
evaluate relative to long-term outcomes. RESULTS: Among 6531 patients
with GBC, median number of LNs evaluated was 2 (IQR:1-5); only 21.1\%
(n = 1376) of patients had 6 or more LNs evaluated. The median number of
metastatic LNs was 0 (IQR: 0-1). On multivariable analysis, evaluation
of \textless{} 4 LNs was associated with a higher hazard of death
(referent 4-7 LNs: \textless{} 4 LNs, HR = 1.27, 95\% CI, 1.16-1.40;
P \textless{} 0.001), whereas, patients who had 4 to 7 LNs
and \textgreater{} 7 LNs evaluated had comparable long-term mortality
risk (HR = 1.10, 95\%CI, 0.98-1.24; P = 0.11). There was no difference
in the proportion of patients who had at least one metastatic LN
identified per T category based on total number of nodes resected (all
P \textgreater{} 0.05). CONCLUSION: The overwhelming majority of
patients did not have the American Joint Committee on Cancer (AJCC)
recommended 6 total LN count . A machine-based learning approach
identified evaluation of 4 to 7 LNs as the LN number associated with
optimal staging and survival. While obtaining 6 LNs may be challenging,
evaluation of at least 4 LNs may be a more appropriate threshold as this
cut-off value was associated with optimal patient outcomes and staging.

{}

{}

\begin{center}\rule{0.5\linewidth}{\linethickness}\end{center}

Back to top

\begin{center}\rule{0.5\linewidth}{\linethickness}\end{center}

\pagebreak

\hypertarget{ampulla-of-vater}{%
\subsubsection{Ampulla of Vater}\label{ampulla-of-vater}}

\begin{center}\rule{0.5\linewidth}{\linethickness}\end{center}

\hypertarget{morphology-diagnostics-ihc-2}{%
\paragraph{Morphology, Diagnostics,
IHC}\label{morphology-diagnostics-ihc-2}}

Morphology, Diagnostics, IHC

\begin{center}\rule{0.5\linewidth}{\linethickness}\end{center}

 - \textbf{Identification of ampullary carcinoma mixed subtype using a
panel of six antibodies and its clinical significance}

\emph{Journal of surgical oncology 2019 Mar;119(3):295-302}

PubMed: \url{https://www.ncbi.nlm.nih.gov/pubmed/?term=30548547}

OBJECTIVES: To investigate the function of immunomarkers CK7, CK20,
CK17, CDX2, MUC1, and MUC2 in the identification of primary ampullary
carcinoma mixed subtype. METHODS: Forty-two cases of primary ampullary
carcinoma were performed by immunohistochemical studies. The correlation
between the mixed subtype and the other two subtypes and patient
survival data was analyzed using the SPSS 16.0 statistical software.
RESULTS: Among 42 cases, 12 (28.6\%) cases were classified as mixed
subtype, which showed variable expression patterns: 91.7\% (11/12) for
CK7, 83.3\% (10/12) for CK20; 66.7\% (8/12) for CK17, CDX2, and MUC1;
and 50\% (6/12) for MUC2. Ten (83.3\%) mixed types coexpressed four or
more immunomarkers. Eight (19\%) intestinal subtypes mainly showed a
positive expression of CK20, CDX2, and MUC2. Twenty-two (52.4\%)
pancreaticobiliary subtypes showed a positive expression of CK7, MUC1,
and CK17. Stages III and IV diseases in mixed subtype (25\%) and
intestinal subtype (25\%) were less than pancreaticobiliary
subtype(63.6\%) (p = 0.039). Follow-up data appeared to show a better
survival rate for patients with mixed subtype than those with
pancreaticobiliary subtypes. CONCLUSION: Immunohistochemical staining
provided a more reliable means of diagnosing mixed ampulla carcinoma.
Accurate subtyping of ampullary carcinoma is clinically important to
select effective chemotherapy regimens and to assess disease prognosis.

{}

{}

\begin{center}\rule{0.5\linewidth}{\linethickness}\end{center}

 - \textbf{Clinical relevance of pancreatobiliary and intestinal
subtypes of ampullary and duodenal adenocarcinoma: Pattern of
recurrence, chemotherapy, and survival after pancreatoduodenectomy}

\emph{Pancreatology : official journal of the International Association
of Pancreatology (IAP) \ldots{} {[}et al.{]} 2019 Mar;19(2):316-324}

PubMed: \url{https://www.ncbi.nlm.nih.gov/pubmed/?term=30713128}

BACKGROUND: The clinical relevance of the classification of ampullary
adenocarcinoma (AC) into pancreatobiliary (PB) or intestinal (Int)
subtypes has not been resolved. METHODS: Clinicopathological factors,
survival, and localization and treatment of recurrence were investigated
for patients with AC and duodenal adenocarcinoma (DC) treated by
pancreatoduodenectomy from 2000 to 2015. RESULTS: A total of 109 AC
(45\,PB, 64 Int) and 71 DC (all Int) were identified. Median overall
survival (OS) for ACPB vs DC vs ACInt was 43.6 vs 51 vs 75 months,
respectively. ACPB had significantly shorter OS than ACInt
(p\,=\,0.036). However, for AC stage (HR\,=\,2.39; 95 \%CI 1.23-4.64,
p\,=\,0.010) was the only factor associated with mortality risk in
multivariate analysis. Localization of recurrence (n\,=\,88) was
predominantly distant (ACPB 81.5\%; ACInt 92\%; DC 91.7\%, p\,=\,0.371).
Post-recurrence survival (PRS) for ACPB, ACInt and DC did not differ
(6.9 vs 9.2 vs 7.5 months, p\,=\,0.755). Best supportive care or
palliative chemotherapy were offered for recurrent disease to
44.5\%/48.1\% for ACPB, 40\%/56\% for ACInt, and 41.7\%/52.8\% for DC
(p\,=\,0.947). The choice of chemotherapy regimen varied considerably.
Five patients underwent surgical resection or ablation with curative
intent. All deaths among ACPB were caused by recurrent disease, whereas
29.4\% of ACInt and 23.1\% of DC deaths was non-cancer related or caused
by other specific cancer. CONCLUSION: ACPB, ACInt and DC have similar
recurrence patterns and PRS. The difference in survival between ACPB and
ACInt was not statistically significant when stratified by stage. The
optimal chemotherapy in patients with recurrent AC remains undefined.

{}

{}

\begin{center}\rule{0.5\linewidth}{\linethickness}\end{center}

Back to top

\begin{center}\rule{0.5\linewidth}{\linethickness}\end{center}

\pagebreak

\hypertarget{staging-2}{%
\paragraph{Staging}\label{staging-2}}

Ampulla of Vater TNM staging, Margins, Survival

\begin{center}\rule{0.5\linewidth}{\linethickness}\end{center}

 - \textbf{The Prognostic Relevance of the New 8th Edition of the Union
for International Cancer Control Classification of TNM Staging for
Ampulla of Vater Carcinoma}

\emph{Annals of surgical oncology 2019 Feb;():}

PubMed: \url{https://www.ncbi.nlm.nih.gov/pubmed/?term=30790111}

OBJECTIVE: The aim of this study was to investigate the clinical
relevance of the 8th edition of the Union for International Cancer
Control classification of TNM staging for ampulla of Vater carcinoma
(AC). METHODS: A total of 104 consecutive patients who underwent
macroscopic curative resection for AC between January 2002 and September
2017 were investigated. RESULTS: Significant differences in
recurrence-free survival (RFS) were found between T1a and T1b
(p = 0.0030), but not between T1b and T2 (p = 0.9319), T2 and T3a
(p = 0.0732), or T3a and T3b (p = 0.2118). The prognostic impact of the
depth of duodenal invasion and pancreatic invasion, which define the T
category, were evaluated. With regard to duodenal invasion, significant
differences in RFS were found between the negative and submucosa
classifications (p = 0.0012) and the muscularis propria and serosa
classifications (p = 0.0131), but not between the submucosa and
muscularis propria classifications (p = 0.6390). With regard to
pancreatic invasion, significant differences in RFS were found between
the negative and ≤ 0.5~cm classifications (p = 0.0001), and ≤ 0.5~cm
and \textgreater{} 0.5~cm classifications (p = 0.0062). A Cox
proportional hazard analysis for RFS revealed that duodenal invasion
(submucosa or muscularis propria/negative, hazard ratio {[}HR{]} 5.08;
serosa/negative, HR 7.42), and pancreatic invasion (≤ 0.5~cm/negative,
HR 8.23; \textgreater{} 0.5~cm/negative, HR 9.81) were independent
prognostic factors. An alternative new T category was proposed, based on
the HRs, as follows: T1, tumor limited to the ampulla of Vater or
sphincter of Oddi; T2, duodenal invasion (submucosa or muscularis
propria); T3, pancreatic invasion (≤ 0.5~cm) or duodenal invasion
(serosa); and T4, pancreatic invasion (\textgreater{} 0.5~cm). This
alternative T category can well classify each subgroup with prognostic
differences. CONCLUSIONS: Reconsideration of the T category based on the
prognostic impact of TNM factors, including the depth of duodenal and
pancreatic invasion, are required in the 8th edition T category.

{}

{}

\begin{center}\rule{0.5\linewidth}{\linethickness}\end{center}

 - \textbf{Staging for Ampullary Carcinoma: Is Less Actually More?}

\emph{Annals of surgical oncology 2019 Feb;():}

PubMed: \url{https://www.ncbi.nlm.nih.gov/pubmed/?term=30788628}

{}

{}

\begin{center}\rule{0.5\linewidth}{\linethickness}\end{center}

Back to top

\begin{center}\rule{0.5\linewidth}{\linethickness}\end{center}

\pagebreak

\hypertarget{neuroendocrine}{%
\subsubsection{Neuroendocrine}\label{neuroendocrine}}

\begin{center}\rule{0.5\linewidth}{\linethickness}\end{center}

\hypertarget{pannet}{%
\paragraph{PanNET}\label{pannet}}

PanNET, Pancreatic Neuroendocrine Tumors and related neuroendocrine
neoplasms

\begin{center}\rule{0.5\linewidth}{\linethickness}\end{center}

 - \textbf{Cystic pancreatic neuroendocrine tumors: A more favorable
lesion?}

\emph{Pancreatology : official journal of the International Association
of Pancreatology (IAP) \ldots{} {[}et al.{]} 2019 Mar;19(2):372-376}

PubMed: \url{https://www.ncbi.nlm.nih.gov/pubmed/?term=30704851}

BACKGROUND: Pancreatic neuroendocrine tumors (PNETs) are predominantly
solid lesions with malignant potential. Cystic PNETs are a small subset
in which data are scarce. The aim of this study was to compare clinical
and biologic differences between cystic and solid PNETs. METHODS:
Patients with PNETs undergoing pancreatectomy between 1988 and 2016\,at
a high-volume center were reviewed retrospectively. Demographic,
clinical, and histopathologic data were collected and analyzed. RESULTS:
347 patients with PNETs were identified; 27\% (n\,=\,91) were cystic.
Patients with cystic PNETs were generally older (59 vs.~55 years,
p\,=\,0.05). Cystic PNETs were more commonly non-functional (95\%
vs.~82\%, p\,=\,0.004), asymptomatic (44\% vs.~28\%, p\,=\,0.009), and
located in the pancreatic body/tail (81\% vs.~60\%,
p\,\textless{}\,0.001) than solid PNETs. Although cystic and solid PNETs
had similar sizes and pathologic stage at the time of resection, Ki-67
proliferation index (Ki-67\,≤\,9\%: 98\% vs.~85\%; p\,=\,0.007), and
histologic grade (grade I: 84\% vs.~59\%; p\,=\,0.009) had less
aggressive features in cystic PNETs. CONCLUSION: In addition to
reporting a higher than previously published incidence of cystic PNET
(27\%), this study found significant differences in multiple
clinicopathologic variables between cystic and solid PNETs. Cystic PNET
may be a distinct and possibly less aggressive subtype of PNET yet have
similar pathologic stage, recurrence, and survival to solid PNETs.
Cystic PNETs require further attention to better understand the true
natural history.

{}

{}

\begin{center}\rule{0.5\linewidth}{\linethickness}\end{center}

 - \textbf{Distinct clinicopathological and prognostic features of
insulinoma with synchronous distant metastasis}

\emph{Pancreatology : official journal of the International Association
of Pancreatology (IAP) \ldots{} {[}et al.{]} 2019 Apr;19(3):472-477}

PubMed: \url{https://www.ncbi.nlm.nih.gov/pubmed/?term=30850220}

BACKGROUND: The clinicopathological and prognostic features of
insulinoma with synchronous metastases are unclear. This study aimed to
verify the distinct clinicopathological and prognostic features of
insulinoma with synchronous distant metastasis. METHODS: Patients with
pancreatic neuroendocrine tumor (PanNET) were retrospectively enrolled
and divided into cohort 1 (Fudan University Shanghai Cancer Center) and
cohort 2 (Surveillance, Epidemiology, and End Results Program database).
Both cohorts were further divided into three subgroups: insulinoma,
nonfunctioning pancreatic neuroendocrine tumor (NF-PanNET), and
non-insulinoma functioning pancreatic neuroendocrine tumor (NiF-PanNET).
RESULTS: Cohorts 1 and 2 comprised 505 and 2761 patients (1566 M0
patients and 1195 M1 patients), respectively. In cohort 1 and cohort 2
M0 subgroup, insulinoma showed longer disease-free survival, overall
survival (OS), and disease-specific survival (DSS) than NiF-PanNET and
NF-PanNET (not reached vs.~48 and 60months, p\,\textless{}\,0.001;
183months vs.~87 and 109months, p\,\textless{}\,0.001; 247months vs.~121
and 140months, p\,=\,0.002). However, in cohort 2 M1, the mDSS for
metastatic insulinoma was shorter than that for NiF-PanNET (31months
vs.~61months, p\,=\,0.045), while the mDSS and mOS were similar to those
for NF-PanNET. The percentage of T1 and N0 patients was similar between
the metastatic insulinoma subgroup and NiF-PanNET and NF-PanNET
subgroups. The Ki-67 index and recurrence had a positive linear
relationship only for NiF-PanNET and NF-PanNET (p\,=\,0.009).
CONCLUSIONS: Insulinoma with synchronous metastasis showed
clinicopathological and prognostic characteristics similar to those of
NF-PanNET. Metastatic insulinoma had worse prognosis than non-insulinoma
F-PanNET. These findings may help in the clinical management of
metastatic insulinoma.

{}

{}

\begin{center}\rule{0.5\linewidth}{\linethickness}\end{center}

 - \textbf{Neuroendocrine Tumors (NETs) of the Minor Papilla/Ampulla:
Analysis of 16 Cases Underlines Homology With Major Ampulla NETs and
Differences From Extra-Ampullary Duodenal NETs}

\emph{The American journal of surgical pathology 2019 Mar;():}

PubMed: \url{https://www.ncbi.nlm.nih.gov/pubmed/?term=30913089}

Neuroendocrine tumors (NETs) of the minor papilla/ampulla (MIPA) are
rare and poorly studied. Only individual case reports and no
comprehensive analysis are available from the literature. We collected
16 MIPA NETs and investigated their clinicopathologic and
immunohistochemical features, including markers such as somatostatin,
pancreatic polypeptide, gastrin, serotonin, MUC1, cytokeratin 7, and
somatostatin receptors type 2A and 5. The median age at diagnosis was
57.5 years, and the female-to-male ratio was 2.2:1. The median NET size
was 1.45 cm, and most (94\%) were low-grade (G1) tumors. Similarly to
what was observed in the major ampulla, 3 histotypes were found: (i)
ampullary-type somatostatin-producing tumors (ASTs, 10 cases),
characterized by somatostatin expression in most tumor cells,
focal-to-extensive tubulo-acinar structures, often with psammoma bodies,
MUC1 reactivity, and no or rare membranous reactivity for somatostatin
receptor type 2A; (ii) gangliocytic paragangliomas (3 cases),
characterized by the coexistence of 3 tumor cell types: epithelioid,
often reactive for pancreatic polypeptide, ganglion-like cells, and S100
reactive sustentacular/stromal cells; and (iii) ordinary nonfunctioning
NETs (3 cases), resembling those more commonly observed in the
extra-ampullary duodenum. Comparable histotypes could also be recognized
among the 30 MIPA NETs from the literature. No NET-related patient death
among MIPA cases was observed during a median follow-up of 38 months;
however, MIPA ASTs showed lymph node metastases and invasion of the
duodenal muscularis propria or beyond in 44\% and 40\% of cases,
respectively. In conclusion, MIPA NETs closely resemble tumors arising
in the major ampulla, with predominance of ASTs.

{}

{}

\begin{center}\rule{0.5\linewidth}{\linethickness}\end{center}

Back to top

\begin{center}\rule{0.5\linewidth}{\linethickness}\end{center}

\pagebreak

\hypertarget{microenvironment-1}{%
\paragraph{Microenvironment}\label{microenvironment-1}}

Tumor Stroma Interactions, Microenvironment, Inflammatory Response

\begin{center}\rule{0.5\linewidth}{\linethickness}\end{center}

 - \textbf{Characterization of the Neuroendocrine Tumor Immune
Microenvironment}

\emph{Pancreas 2018 10;47(9):1123-1129}

PubMed: \url{https://www.ncbi.nlm.nih.gov/pubmed/?term=30153220}

OBJECTIVES: The immune environment and the potential for neuroendocrine
tumors (NETs) to respond to immune checkpoint inhibitors remain largely
unexplored. We assessed immune checkpoint marker expression, lymphocytic
infiltrate, and associated mutational profiles in a cohort of small
intestine and pancreatic NETs. METHODS: We assessed expression of PDCD1
(PD-1), CD274 (PD-L1), and PDCD1LG2 (PD-L2) in archival tissue from 64
small intestine (SINETs) and 31 pancreatic NETs (pNET). We additionally
assessed T-cell infiltrates, categorizing T-cell subsets based on
expression of the T-cell markers CD3, CD8, CD45RO (PTPRC), or FOXP3.
Finally, we explored associations between immune checkpoint marker
expression, lymphocytic infiltrate, and tumor mutational profiles.
RESULTS: Expression of PD-1 or PD-L1 in small intestine or pancreatic
NET was rare, whereas expression of PD-L2 was common in both NET
subtypes. T-cell infiltrates were more abundant in pNET than in SINET.
We found no clear associations between immune checkpoint marker
expression, immune infiltrates, and specific mutational profile within
each tumor type. CONCLUSIONS: Our findings provide an initial assessment
of the immune environment of well-differentiated NETs. Further studies
to define the immunologic differences between pNET and SINET, as well as
the role of PD-L2 in these tumors, are warranted.

{}

{}

\begin{center}\rule{0.5\linewidth}{\linethickness}\end{center}

Back to top

\begin{center}\rule{0.5\linewidth}{\linethickness}\end{center}

\pagebreak

\hypertarget{staging-3}{%
\paragraph{Staging}\label{staging-3}}

PanNET TNM staging, Margins, Survival

\begin{center}\rule{0.5\linewidth}{\linethickness}\end{center}

 - \textbf{A modified M-stage classification based on the metastatic
patterns of pancreatic neuroendocrine neoplasms: a population-based
study}

\emph{BMC endocrine disorders 2018 Oct;18(1):73}

PubMed: \url{https://www.ncbi.nlm.nih.gov/pubmed/?term=30340569}

BACKGROUND: The present study aims to improve the M-stage classification
of pancreatic neuroendocrine neoplasms (pNENs). METHODS: Two thousand
six hundred sixty six pNENs were extracted from the Surveillance,
Epidemiology, and End Results database to explore the metastatic
patterns of pNENs. Metastatic patterns were categorized as single, two,
or multiple (three or more) distant organ metastasis. The mean overall
survival and hazard rate of different metastatic patterns were
calculated by Kaplan-Meier and Cox proportional hazards models,
respectively. The discriminatory capability of the modified M-stage
classification was evaluated by Harrell's concordance index. RESULTS:
The overall survival time significantly decreased with an increasing
number of metastatic organs. In addition, pNENs with only liver
metastasis had better prognosis when compared to other metastatic
patterns. Thus, we modified the M-stage classification (mM-stage) as
follows: mM0-stage, tumor without metastasis; mM1-stage, tumor only
metastasized to liver; mM2-stage, tumor metastasized to other single
distant organ (lung, bone, or brain) or two distant organs; mM3-stage,
tumor metastasized to three or more distant organs. Harrell's
concordance index showed that the modified M-stage classification had
superior discriminatory capability than both the American Joint
Committee on Cancer (AJCC) and the European Neuroendocrine Tumor Society
(ENETS) M-stage classifications. CONCLUSIONS: The modified M-stage
classification is superior to both AJCC and ENETS M-stage
classifications in the prognosis of pNENs. In the future, individualized
treatment and follow-up programs should be explored for patients with
distinct metastatic patterns.

{}

{}

\begin{center}\rule{0.5\linewidth}{\linethickness}\end{center}

 - \textbf{Pancreatic neuroendocrine tumours: Grade is superior to T, N,
or M status in predicting outcome and selecting patients for
chemotherapy:A retrospective cohort study in the SEER database}

\emph{International journal of surgery (London, England) 2019 Mar;():}

PubMed: \url{https://www.ncbi.nlm.nih.gov/pubmed/?term=30872175}

BACKGROUND: Pancreatic neuroendocrine tumours (pNETs) are a rare and
heterogeneous group of tumours with an increasing incidence. Current
staging criteria for pNETs remain limited and controversial. Meanwhile,
the impact of chemotherapy on overall survival has not been fully
defined. OBJECTIVES: The current study aimed to explore epidemiologic
trends of pancreatic neuroendocrine tumours (pNETs). To determine
feasible improvements to staging criteria and investigate the
relationship between chemotherapy and survival. METHODS: A retrospective
cohort study design was used to analyse annual cancer incidence rates,
patient demographics, tumour site and stage, and treatment of pNETs.
Data were obtained from the National Cancer Institute's SEER registry
for all patients diagnosed with pNETs between January 1973 and December
2015. RESULTS: Patients diagnosed after 2010 were more likely to present
with age greater than 45 years, T0, T1 status, N0 status, M0 status, and
well differentiation. Current AJCC staging criteria was applicable to
patients with well differentiation, but not other differentiation. The
revised system, defined by Grade, T, N, and M status, could robustly
discriminate between survival curves. Chemotherapy was associated with
significantly improved survival for patients with poorly differentiated
and undifferentiated tumour grading. CONCLUSIONS: Grade is superior to
`T', `N', or `M' status in predicting outcomes and selecting patients
for chemotherapy. It is necessary and feasible to combine grade into
current staging criteria.

{}

{}

\begin{center}\rule{0.5\linewidth}{\linethickness}\end{center}

Back to top

\begin{center}\rule{0.5\linewidth}{\linethickness}\end{center}

\pagebreak

\hypertarget{cytopathology}{%
\subsection{Cytopathology}\label{cytopathology}}

\begin{center}\rule{0.5\linewidth}{\linethickness}\end{center}

\hypertarget{bile-ducts-1}{%
\subsubsection{Bile Ducts}\label{bile-ducts-1}}

\begin{center}\rule{0.5\linewidth}{\linethickness}\end{center}

 - \textbf{Bile cytology: A new scoring system for improving diagnostic
accuracy}

\emph{Diagnostic cytopathology 2019 Feb;():}

PubMed: \url{https://www.ncbi.nlm.nih.gov/pubmed/?term=30730608}

BACKGROUND: Benign and malignant cells need to be distinguished in any
cytological examination of bile. Here, we report an original scoring
system to improve the diagnostic accuracy of bile cytology. METHODS: The
study used 158 bile aspiration samples obtained for cytological
examination. Fourteen cytological findings were used to differentiate
benign and malignant samples. Statistical significance tests and
multivariate analysis were used to determine and quantify significant
findings and develop a scoring system. RESULTS: Four cytological
findings were significant in discriminating between benign and malignant
cells: abnormal chromatin, irregularly arranged nuclei, irregularly
overlapped nuclei, and irregular cluster margins. Our newly developed
scoring system based on these four cytological findings yielded
excellent results with a sensitivity of 87\%, specificity of 98\%, and
an odds ratio of 329. CONCLUSIONS: The use of our new scoring system is
expected to contribute to the diagnostic accuracy of cytological
evaluations of bile samples.

{}

{}

\begin{center}\rule{0.5\linewidth}{\linethickness}\end{center}

Back to top

\begin{center}\rule{0.5\linewidth}{\linethickness}\end{center}

\pagebreak

\hypertarget{molecular-pathology}{%
\subsection{Molecular Pathology}\label{molecular-pathology}}

\begin{center}\rule{0.5\linewidth}{\linethickness}\end{center}

\hypertarget{pancreas-1}{%
\subsubsection{Pancreas}\label{pancreas-1}}

\begin{center}\rule{0.5\linewidth}{\linethickness}\end{center}

\hypertarget{pancreas-2}{%
\paragraph{Pancreas}\label{pancreas-2}}

\begin{center}\rule{0.5\linewidth}{\linethickness}\end{center}

 - \textbf{Knockdown of KDM1B inhibits cell proliferation and induces
apoptosis of pancreatic cancer cells}

\emph{Pathology, research and practice 2019 May;215(5):1054-1060}

PubMed: \url{https://www.ncbi.nlm.nih.gov/pubmed/?term=30846414}

Pancreatic cancer (PC) is one of the common malignant tumors in
digestive tract with a high fatality rate. The oncogenic role of
lysine-specific demethylase1 (LSD1/KDM1 A) has been well recognized in
PC. While, the role of its homolog LSD2 (KDM1B) in regulating PC
progression is poorly understood. In this study, we attempted to
evaluate the functional role of KDM1B in PC cells. The expression of
KDM1B was detected by immunohistochemistry and immunoblotting in PC
tissues and cells. Lentivirus-mediated shRNA was applied to silence
KDM1B in PANC-1 and SW1990 cells. Cell proliferation was measured by MTT
and Celigo assay. Cell apoptosis was determined by both Caspase-Glo®3/7
assay and Flow cytometry. Intracellular signaling molecules were
detected using a PathScan intracellular signaling array kit. In this
study, we found KDM1B was highly expressed in PC tissues compared to
paracancerous tissues. Moreover, elevated expression of KDM1B was
detected in PC cell lines (BxPC-3, CFPAC-1, PANC-1 and SW1990) as
compared with a normal human pancreatic duct epithelial cell line
(HPDE6-C7). Further investigations revealed that KDM1B knockdown
significantly inhibited PC cell proliferation. Furthermore, the
apoptosis of PANC-1 and SW1990 cells was significantly increased after
KDM1B knockdown. Notably, the activations of p-ERK1/2, p-Smad2, p-p53,
cleaved PARP, cleaved Caspase-3, cleaved Caspase-7, p-eIF2a and Survivin
were promoted by KDM1B knockdown, while IkBa was suppressed. Taken
together, our findings provided new insights into the critical and
multifaceted roles of KDM1B in the regulation of cell proliferation and
apoptosis, and offered a potentially novel target in preventing the
progression of PC.

{}

{}

\begin{center}\rule{0.5\linewidth}{\linethickness}\end{center}

 - \textbf{Real-Time Targeted Genome Profile Analysis of Pancreatic
Ductal Adenocarcinomas Identifies Genetic Alterations That Might Be
Targeted With Existing Drugs or Used as Biomarkers}

\emph{Gastroenterology 2019 Mar;():}

PubMed: \url{https://www.ncbi.nlm.nih.gov/pubmed/?term=30836094}

BACKGROUND \& AIMS: It has been a challenge to select treatment for
patients with pancreatic ductal adenocarcinomas (PDACs) based on genome
alterations. We performed targeted genomic profile analyses of a large
number of PDACs to assess the full spectrum of actionable genomic
alterations. METHODS: We performed targeted genomic profile analyses of
3594 PDAC samples from an international cohort, including capture-based
targeted genomic profiling of as many as 315 cancer-associated genes and
intron regions of 28 genes that are rearranged in cancer cells. Tumor
mutation burden (TMB) and microsatellite instability (MSI) status were
also assessed. TMB was calculated across a 1.14-megabase region;
TMB-high was defined as ≥20 mutations/megabase. MSI-high status was
assigned based on analysis of 114 intron homopolymer loci. RESULTS:
KRAS, TP53, CDKN2A, and SMAD4 were the most frequently altered genes in
the PDAC tissues. We found KRAS mutations in 88\% of samples. Among
PDACs without mutations in KRAS, we found alterations in genes whose
products are in the mitogen-activated protein kinase signaling pathway
and are candidate drug targets (actionable targets, n~= 132; 4\%), as
well as gene fusions (n~= 51), gene amplifications (n~= 35), genes with
missense mutations (n~= 30), and genes that contain deletions (n~= 16).
Many of these encode proteins in receptor tyrosine kinase, RAS, or
mitogen-activated protein kinase signaling pathways. Aside from TP53,
alterations in genes encoding DNA damage repair proteins (BRCA and FANC)
were detected in in 14\% of PDACs. Among PDACs evaluated for MSI (n~=
2563) and TMB (n~= 1021), MSI-high and/or TMB-high phenotypes were
detected in 0.5\% of samples. Alterations in FGF23, CCND2, PIK3CA, and
FGF6 were more commonly detected in intraductal papillary mucinous
neoplasm-associated PDACs. CONCLUSIONS: In targeted genomic profile
analyses of 3594 PDACs, we found 17\% to contain genomic alterations
that might make the tumor cells susceptible to currently used anticancer
agents. We identified mutations in genes that could contribute to
progression of intraductal papillary mucinous neoplasms into
malignancies. These alterations might be used as biomarkers for early
detection.

{}

{}

\begin{center}\rule{0.5\linewidth}{\linethickness}\end{center}

 - \textbf{Transcriptomic analysis of the Aquaporin (AQP) gene family
interactome identifies a molecular panel of four prognostic markers in
patients with pancreatic ductal adenocarcinoma}

\emph{Pancreatology : official journal of the International Association
of Pancreatology (IAP) \ldots{} {[}et al.{]} 2019 Apr;19(3):436-442}

PubMed: \url{https://www.ncbi.nlm.nih.gov/pubmed/?term=30826259}

BACKGROUND: This study aimed to assess the differential gene expression
of aquaporin (AQP) gene family interactome in pancreatic ductal
adenocarcinoma (PDAC) using data mining techniques to identify novel
candidate genes intervening in the pathogenicity of PDAC. METHOD:
Transcriptome data mining techniques were used in order to construct the
interactome of the AQP gene family and to determine which genes members
are differentially expressed in PDAC as compared to controls. The same
techniques were used in order to evaluate the potential prognostic role
of the differentially expressed genes. RESULTS: Transcriptome microarray
data of four GEO datasets were incorporated, including 142 primary tumor
samples and 104 normal pancreatic tissue samples. Twenty differentially
expressed genes were identified, of which nineteen were downregulated
and one up-regulated. A molecular panel of four genes (Aquaporin 7 -
AQP7; Archain 1 - ARCN1; Exocyst Complex Component 3 - EXOC3; Coatomer
Protein Complex Subunit Epsilon - COPE) were identified as potential
prognostic markers associated with overall survival. CONCLUSION: These
outcomes should be further assessed in~vitro in order to fully
understand the role of these genes in the pathophysiological mechanism
of PDAC.

{}

{}

\begin{center}\rule{0.5\linewidth}{\linethickness}\end{center}

 - \textbf{A pipeline for rapidly generating genetically engineered
mouse models of pancreatic cancer using in vivo CRISPR-Cas9-mediated
somatic recombination}

\emph{Laboratory investigation; a journal of technical methods and
pathology 2019 Feb;():}

PubMed: \url{https://www.ncbi.nlm.nih.gov/pubmed/?term=30728464}

Genetically engineered mouse models (GEMMs) that recapitulate the major
genetic drivers in pancreatic ductal adenocarcinoma (PDAC) have provided
unprecedented insights into the pathogenesis of this lethal neoplasm.
Nonetheless, generating an autochthonous model is an expensive, time
consuming and labor intensive process, particularly when tissue specific
expression or deletion of compound alleles are involved. In addition,
many of the current PDAC GEMMs cause embryonic, pancreas-wide activation
or loss of driver alleles, neither of which reflects the cognate human
disease scenario. The advent of CRISPR/Cas9 based gene editing can
potentially circumvent many of the aforementioned shortcomings of
conventional breeding schema, but ensuring the efficiency of gene
editing in vivo remains a challenge. Here we have developed a pipeline
for generating PDAC GEMMs of complex genotypes with high efficiency
using a single ``workhorse'' mouse strain expressing Cas9 in the adult
pancreas under a p48 promoter. Using adeno-associated virus (AAV)
mediated delivery of multiplexed guide RNAs (sgRNAs) to the adult murine
pancreas of p48-Cre; LSL-Cas9 mice, we confirm our ability to express an
oncogenic Kras G12D allele through homology-directed repair (HDR), in
conjunction with CRISPR-induced disruption of cooperating alleles
(Trp53, Lkb1 and Arid1A). The resulting GEMMs demonstrate a spectrum of
precursor lesions (pancreatic intraepithelial neoplasia {[}PanIN{]} or
Intraductal papillary mucinous neoplasm {[}IPMN{]} with eventual
progression to PDAC. Next generation sequencing of the resulting murine
PDAC confirms HDR of oncogenic KrasG12D allele at the endogenous locus,
and insertion deletion (``indel'') and frameshift mutations of targeted
tumor suppressor alleles. By using a single ``workhorse'' mouse strain
and optimal AAV serotype for in vivo gene editing with combination of
driver alleles, we present a facile autochthonous platform for
interrogation of the PDAC genome.

{}

{}

\begin{center}\rule{0.5\linewidth}{\linethickness}\end{center}

Back to top

\begin{center}\rule{0.5\linewidth}{\linethickness}\end{center}

\pagebreak

\hypertarget{spn}{%
\paragraph{SPN}\label{spn}}

Solid Pseudopapillary Neoplasm

\begin{center}\rule{0.5\linewidth}{\linethickness}\end{center}

 - \textbf{Targeted next generation sequencing of pancreatic solid
pseudopapillary neoplasms show mutations in Wnt signaling pathway genes}

\emph{Pathology international 2019 Feb;():}

PubMed: \url{https://www.ncbi.nlm.nih.gov/pubmed/?term=30811747}

Solid pseudopapillary neoplasms of the pancreas are rare neoplasms that
have been shown to harbor recurrent somatic pathogenic variants in the
beta-catenin gene, CTNNB1. Here, we used targeted next generation
sequencing to analyze these tumors for other associated mutations. Six
cases of solid pseudopapillary neoplasms were studied. DNA extracted
from formalin-fixed paraffin embedded tissue blocks was analyzed using
the Ion Torrent platform, with the 50-gene Ampliseq Cancer Hotspot Panel
v2 (CHPv2), with further variant validation performed by Sanger
sequencing. Four tumors (67\%) were confirmed to harbor mutations within
CTNNB1, two with c.109T \textgreater{} G p.(Ser37Ala) and two with
c.94G \textgreater{} A p.(Asp32Asn). One case showed a frameshift
deletion in the Adenomatous Polyposis Coli gene, APC c.3964delG
p.(Glu1322Lysfs*93) with a variant allele frequency of 42.6\%. Sanger
sequencing on non-tumoral tissue confirmed the variant was somatic. The
patient with the APC mutation developed metastasis and died. In addition
to the four cases harboring CTNNB1 variants, we found a case
characterized by poor outcome, showing a rare frameshift deletion in the
APC gene. Since the APC product interacts with beta-catenin, APC
variants may, in addition to CTNNB1, contribute to the pathogenesis of
solid pseudopapillary neoplasms via the Wnt signaling pathway.

{}

{}

\begin{center}\rule{0.5\linewidth}{\linethickness}\end{center}

Back to top

\begin{center}\rule{0.5\linewidth}{\linethickness}\end{center}

\pagebreak

\hypertarget{pancreatitis-other-diseases}{%
\paragraph{Pancreatitis \& Other
Diseases}\label{pancreatitis-other-diseases}}

Molecular Studies on Pancreatitis \& Other Diseases

\begin{center}\rule{0.5\linewidth}{\linethickness}\end{center}

 - \textbf{STING signalling protects against chronic pancreatitis by
modulating Th17 response}

\emph{Gut 2019 Jan;():}

PubMed: \url{https://www.ncbi.nlm.nih.gov/pubmed/?term=30705050}

OBJECTIVE: Chronic pancreatitis (CP) is an inflammatory disease with
progressive fibrosis leading to exocrine and endocrine dysfunction.
Currently, there are no approved effective therapies for CP. Stimulator
of interferon genes (STING) signalling is a key innate immune sensor of
DNA. In this study, we evaluated the role of STING signalling in CP.
DESIGN: We used an experimental model of CP to test the effect of STING
signalling in STING wild-type and knockout mice as well as bone marrow
chimaeras (BMCs). STING was activated using a pharmacological agent.
Since we found changes in Th17 cells, we used neutralising and control
antibodies to determine the role of IL-17A. The effect of STING
signalling was further explored in IL-17A generation and we examined the
effect of IL-17A on pancreatic stellate cells (PSCs). Human pancreas
from patients with CP and without CP were also stained for IL-17A.
RESULTS: STING activation decreased CP-associated pancreatic
inflammation and fibrosis, whereas absence of STING led to worsening of
the disease. BMCs showed that leucocytes play an important role in STING
signalling-mediated amelioration of experimental CP. STING deletion was
associated with increased Th17 cell infiltration in the pancreas,
whereas STING agonist limited this Th17 response. Importantly,
anti-IL-17A antibody treatment mitigated the severity of CP in the
absence of STING signalling. STING deficiency promoted Th17 polarisation
and PSCs express functional IL-17 receptor by upregulating fibrosis
genes. Compared with tumour margins, pancreas from patients with CP had
significant increase in IL-17A+ cells. CONCLUSION: Unlike acute
pancreatitis, STING activation is protective in CP. STING signalling is
important in regulating adaptive immune responses by diminishing
generation of IL-17A during CP and presents a novel therapeutic target
for CP.

{}

{}

\begin{center}\rule{0.5\linewidth}{\linethickness}\end{center}

Back to top

\begin{center}\rule{0.5\linewidth}{\linethickness}\end{center}

\pagebreak

\hypertarget{molecular-research-on-microenvironment}{%
\paragraph{Molecular Research on
Microenvironment}\label{molecular-research-on-microenvironment}}

Tumor Stroma Interactions, Microenvironment, Inflammatory Response,
Microbiome

\begin{center}\rule{0.5\linewidth}{\linethickness}\end{center}

 - \textbf{Reliable evaluation of tumor-infiltrating lymphocytes in
pancreatic cancer tissue biopsies}

\emph{Oncotarget 2019 Feb;10(10):1149-1159}

PubMed: \url{https://www.ncbi.nlm.nih.gov/pubmed/?term=30800224}

Tumor-infiltrating lymphocytes (TILs) represent cancer microenvironment.
We previously reported TILs was prognosticators in pancreatic ductal
adenocarcinoma (PDAC) patients by immunohistochemically measuring them
in surgically-resected tissues. The aim of this study was to assess how
best to evaluate TILs in PDAC tissue biopsies. First, we showed
expression of CD3, CD4, or CD8 genes in PDAC tissue measured by
quantitative RT-PCR (RT-qPCR) was prognostic using 241
surgically-resected specimens. We assessed whether the TILs in biopsied
tissues can be effectively evaluated by comparing between
immunohistochemistry and RT-qPCR. As a study model, we sampled twenty
biopsies from surgically-resected PDAC specimen (n = 17). We
investigated the variation levels of TILs in the different biopsies from
the same specimen and evaluated using the intraclass correlation
coefficient (ICC). The ICC value was 0.58 for CD3, 0.61 for CD4, and
0.46 for CD8, respectively; these ICC values meant correlations of
``moderate'' to ``substantial'' levels. To reach ``near perfect'', 3, 3,
and 5 times biopsies were necessary for CD3, CD4, and CD8, respectively.
When ICC values of immunolabeled TILs were of ``low'', ≥6 times biopsies
were necessary to reach ``moderate'' levels. We found that TILs measured
by RT-qPCR and repeated sampling increased reliability in TILs detected
from biopsied PDAC tissues.

{}

{}

\begin{center}\rule{0.5\linewidth}{\linethickness}\end{center}

 - \textbf{Activation of IGF/IGF-IR signaling pathway fails to induce
epithelial-mesenchymal transition in pancreatic cancer cells}

\emph{Pancreatology : official journal of the International Association
of Pancreatology (IAP) \ldots{} {[}et al.{]} 2019 Mar;19(2):390-396}

PubMed: \url{https://www.ncbi.nlm.nih.gov/pubmed/?term=30799278}

BACKGROUND: Pancreatic cancer stromal cells produce various protein
factors, which presumably provide cancer cells with drug resistance and
may influence their ability to form metastasis via induction of
epithelial-mesenchymal transition (ЕМТ). The goal of our project was to
study the effects of IGF-I on expression of protein markers of
epithelial and mesenchymal differentiation, and on expression of
transcriptional regulators of EMT in pancreatic cancer cell lines.
METHODS: We used Western blot analysis to study the expression patterns
of epithelial and mesenchymal protein markers in pancreatic cancer cell
lines, which have been stimulated with IGF-I for various periods of
time. The ELISA technique was employed to determine the concentration of
IGF-I in conditioned media. Additionally, the effect of IGF-I on
proliferation of pancreatic cancer cells was measured via MTS technique.
RESULTS: We investigated the effect of IGF/IGF-IR signaling pathway
activation on expression levels of cell differentiation markers in five
pancreatic cancer cell lines (AsPC-1, BxPC-3, Capan-2, MiaPaCa-2 and
Panc1). The IGF-I stimulation led to phosphorylation of IGF-IR and
activation of PI-3K/Akt signaling cascade. At the same time our results
reveal that the activation of IGF/IGF-IR signaling pathway in pancreatic
cancer cells does not induce a significant shift in cell phenotype
towards mesenchymal differentiation and does not induce a decrease in
expression levels of epithelial protein markers. CONCLUSIONS: Our
results demonstrate that IGF-I does not function as an effective
inductor of EMT in pancreatic cancer cell lines and that stimulation of
IGF-I/IGF-IR signaling pathway does not lead to EMT associated changes
in cell differentiation.

{}

{}

\begin{center}\rule{0.5\linewidth}{\linethickness}\end{center}

 - \textbf{Signaling Networks That Control Cellular Plasticity in
Pancreatic Tumorigenesis, Progression, and Metastasis}

\emph{Gastroenterology 2019 May;156(7):2073-2084}

PubMed: \url{https://www.ncbi.nlm.nih.gov/pubmed/?term=30716326}

Pancreatic ductal adenocarcinoma is one of the deadliest cancers, and
its incidence on the rise. The major challenges in overcoming the poor
prognosis with this disease include late detection and the aggressive
biology of the disease. Intratumoral heterogeneity; presence of a
robust, reactive, and desmoplastic stroma; and the crosstalk between the
different tumor components require complete understanding of the
pancreatic tumor biology to better understand the therapeutic challenges
posed by this disease. In this review, we discuss the processes involved
during tumorigenesis encompassing the inherent plasticity of the
transformed cells, development of tumor stroma crosstalk, and enrichment
of cancer stem cell population during tumorigenesis.

{}

{}

\begin{center}\rule{0.5\linewidth}{\linethickness}\end{center}

 - \textbf{Semaphorin-5A maintains epithelial phenotype of malignant
pancreatic cancer cells}

\emph{BMC cancer 2018 Dec;18(1):1283}

PubMed: \url{https://www.ncbi.nlm.nih.gov/pubmed/?term=30577767}

BACKGROUND: Pancreatic cancer (PC) is a highly aggressive disease, and
the lethality of this disease stems from early metastatic dissemination
where surgical removal cannot provide a cure. Improvement of the
therapeutic outcome and overall survival of PC patients requires to
understand the fundamental processes that lead to metastasis such as the
gain of cellular migration ability. One such family of proteins, which
are essential players of cellular migration, is Semaphorin. Previously,
we have identified one of the Semaphorin family member, Semaphorin-5A
(SEMA5A) to be involved in organ-specific homing during PC metastasis.
We have also demonstrated that SEMA5A has a constitutive expression in
PC cell lines derived from metastatic sites in comparison with low
endogenous expression in the primary tumor-derived cell line. In this
study, we examined whether constitutive SEMA5A expression in metastatic
PC cells regulates tumor growth and metastatic potential. METHODS: We
generated SEMA5A knockdown in T3M-4 and CD18/HPAF cells and assessed
their phenotypes on in vitro motility, tumor growth, and metastatic
progression. RESULTS: In contrary to our initial expectations,
orthotopic injection of SEMA5A knockdown cells into nude mice resulted
in a significant increase in both tumor burden and liver metastases in
comparison with the Control cells. Similarly, we observed higher in
vitro migratory potential with pronounced morphological changes
associated with epithelial-mesenchymal transition (EMT), a decrease in
the expression of epithelial marker E-cadherin (E-Cad), increase in the
expression of mesenchymal markers N-cadherin (N-Cad) and Snail and the
activation of the Wnt-signaling pathway in SEMA5A knockdown cells.
Furthermore, re-establishing SEMA5A expression with a knockdown
resistant mouse Sema5A in SEMA5A knockdown cells resulted in a reversion
to the epithelial state (mesenchymal-epithelial transition; MET), as
indicated by the rescue of E-Cad expression and a decrease in N-Cad and
Snail expression. CONCLUSIONS: Collectively, our data suggest that
SEMA5A expression maintains epithelial phenotype in the metastatic
microenvironment.

{}

{}

\begin{center}\rule{0.5\linewidth}{\linethickness}\end{center}

Back to top

\begin{center}\rule{0.5\linewidth}{\linethickness}\end{center}

\pagebreak

\hypertarget{preneoplastic-and-preinvasive-lesions}{%
\paragraph{Preneoplastic and Preinvasive
Lesions}\label{preneoplastic-and-preinvasive-lesions}}

Molecular Pathology Preneoplastic and Preinvasive Lesions, PanIN, IPMN,
MCN, ICPN

\begin{center}\rule{0.5\linewidth}{\linethickness}\end{center}

 - \textbf{Prevalence of Germline Mutations Associated With Cancer Risk
in Patients With Intraductal Papillary Mucinous Neoplasms}

\emph{Gastroenterology 2019 May;156(6):1905-1913}

PubMed: \url{https://www.ncbi.nlm.nih.gov/pubmed/?term=30716324}

BACKGROUND \& AIMS: Many patients with pancreatic adenocarcinoma carry
germline mutations associated with increased risk of cancer. It is not
clear whether patients with intraductal papillary mucinous neoplasms
(IPMNs), which are precursors to some pancreatic cancers, also carry
these mutations. We assessed the prevalence of germline mutations
associated with cancer risk in patients with histologically confirmed
IPMN. METHODS: We obtained nontumor tissue samples from 315 patients
with surgically resected IPMNs from 1997 through 2017, and we sequenced
94 genes with variants associated with cancer risk. Mutations associated
with increased risk of cancer were identified and compared with
individuals from the Exome Aggregation Consortium. RESULTS: We
identified 23 patients with a germline mutation associated with cancer
risk (7.3\%; 95\% confidence interval, 4.9-10.8). Nine patients had a
germline mutation associated with pancreatic cancer susceptibility
(2.9\%; 95\% confidence interval, 1.4-5.4). More patients with IPMNs
carried germline mutations in ATM (P \textless{} .0001), PTCH1 (P
\textless{} .0001), and SUFU (P \textless{} .0001) compared with
controls. Patients with IPMNs and germline mutations associated with
pancreatic cancer were more like to have concurrent invasive pancreatic
carcinoma compared with patients with IPMNs without these mutations (P
\textless{} .0320). CONCLUSIONS: In sequence analyses of 315 patients
with surgically resected IPMNs, we found that almost 3\% to carry
mutations associated with pancreatic cancer risk. More patients with
IPMNs and germline mutations associated with pancreatic cancer had
concurrent invasive pancreatic carcinoma compared with patients with
IPMNs without these mutations. Genetic analysis of patients with IPMNs
might identify those at greatest risk for cancer.

{}

{}

\begin{center}\rule{0.5\linewidth}{\linethickness}\end{center}

Back to top

\begin{center}\rule{0.5\linewidth}{\linethickness}\end{center}

\pagebreak


\end{document}
