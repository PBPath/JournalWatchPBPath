% Options for packages loaded elsewhere
\PassOptionsToPackage{unicode}{hyperref}
\PassOptionsToPackage{hyphens}{url}
%
\documentclass[
]{article}
\usepackage{lmodern}
\usepackage{amssymb,amsmath}
\usepackage{ifxetex,ifluatex}
\ifnum 0\ifxetex 1\fi\ifluatex 1\fi=0 % if pdftex
  \usepackage[T1]{fontenc}
  \usepackage[utf8]{inputenc}
  \usepackage{textcomp} % provide euro and other symbols
\else % if luatex or xetex
  \usepackage{unicode-math}
  \defaultfontfeatures{Scale=MatchLowercase}
  \defaultfontfeatures[\rmfamily]{Ligatures=TeX,Scale=1}
\fi
% Use upquote if available, for straight quotes in verbatim environments
\IfFileExists{upquote.sty}{\usepackage{upquote}}{}
\IfFileExists{microtype.sty}{% use microtype if available
  \usepackage[]{microtype}
  \UseMicrotypeSet[protrusion]{basicmath} % disable protrusion for tt fonts
}{}
\makeatletter
\@ifundefined{KOMAClassName}{% if non-KOMA class
  \IfFileExists{parskip.sty}{%
    \usepackage{parskip}
  }{% else
    \setlength{\parindent}{0pt}
    \setlength{\parskip}{6pt plus 2pt minus 1pt}}
}{% if KOMA class
  \KOMAoptions{parskip=half}}
\makeatother
\usepackage{xcolor}
\IfFileExists{xurl.sty}{\usepackage{xurl}}{} % add URL line breaks if available
\IfFileExists{bookmark.sty}{\usepackage{bookmark}}{\usepackage{hyperref}}
\hypersetup{
  pdftitle={Pancreatobiliary Pathology Society Journal Watch},
  hidelinks,
  pdfcreator={LaTeX via pandoc}}
\urlstyle{same} % disable monospaced font for URLs
\usepackage[margin=1in]{geometry}
\usepackage{graphicx}
\makeatletter
\def\maxwidth{\ifdim\Gin@nat@width>\linewidth\linewidth\else\Gin@nat@width\fi}
\def\maxheight{\ifdim\Gin@nat@height>\textheight\textheight\else\Gin@nat@height\fi}
\makeatother
% Scale images if necessary, so that they will not overflow the page
% margins by default, and it is still possible to overwrite the defaults
% using explicit options in \includegraphics[width, height, ...]{}
\setkeys{Gin}{width=\maxwidth,height=\maxheight,keepaspectratio}
% Set default figure placement to htbp
\makeatletter
\def\fps@figure{htbp}
\makeatother
\setlength{\emergencystretch}{3em} % prevent overfull lines
\providecommand{\tightlist}{%
  \setlength{\itemsep}{0pt}\setlength{\parskip}{0pt}}
\setcounter{secnumdepth}{5}
\renewcommand{\linethickness}{0.05em}
\usepackage{booktabs}
\usepackage{sectsty} \allsectionsfont{\nohang\centering \emph}
\ifluatex
  \usepackage{selnolig}  % disable illegal ligatures
\fi

\title{Pancreatobiliary Pathology Society Journal Watch}
\usepackage{etoolbox}
\makeatletter
\providecommand{\subtitle}[1]{% add subtitle to \maketitle
  \apptocmd{\@title}{\par {\large #1 \par}}{}{}
}
\makeatother
\subtitle{April - August 2020}
\author{}
\date{\vspace{-2.5em}Last Update on 2020-08-18}

\begin{document}
\maketitle

{
\setcounter{tocdepth}{2}
\tableofcontents
}
\begin{center}\rule{0.5\linewidth}{0.5pt}\end{center}

\hypertarget{pbpath_jw}{%
\section{PBPath Journal Watch Articles}\label{pbpath_jw}}

\begin{center}\rule{0.5\linewidth}{0.5pt}\end{center}

\textbf{Welcome to the PBPath Journal Watch!}

This bi-monthly journal watch features exciting recently published
pancreas and biliary pathology articles that will provide up to date
medical knowledge in our field. These articles will be showcased in
several convenient categories, including surgical pathology,
cytopathology, molecular pathology, pancreas, gallbladder, bile ducts,
and ampulla among others. The articles in each category are in no
particular order. See the list of journals we search regularly
\href{http://pbpath.org/pbpath-journal-watch/}{here}. Previous months'
issues may be found in our
\emph{\href{http://pbpath.org/journal-watch-archive/}{archive}} and you
may see \href{http://pbpath.org/journal-watch-upcoming-issue/}{drafts of
the upcoming issue here}.

We encourage members to actively participate by recommending new
articles and providing feedback using the
\protect\hyperlink{feedback}{forms provided below.}

\textbf{We hope that you will enjoy the new PBPath Journal Watch!}

\pagebreak

\begin{center}\rule{0.5\linewidth}{0.5pt}\end{center}

\hypertarget{special}{%
\subsection{Special Section Articles from PBPath
Society}\label{special}}

\begin{center}\rule{0.5\linewidth}{0.5pt}\end{center}

\textbf{SPECIAL SECTION---CONTRIBUTIONS FROM THE PANCREATOBILIARY
PATHOLOGY SOCIETY}

\href{https://www.archivesofpathology.org/toc/arpa/144/7}{Archives of
Pathology \& Laboratory Medicine - July 2020}

\begin{itemize}
\tightlist
\item
  \textbf{Challenging Topics in Pancreatic Neoplasia }
\end{itemize}

\emph{Olca Basturk and Alyssa M. Krasinskas}

\url{https://www.archivesofpathology.org/doi/pdf/10.5858/arpa.2020-0142-ED}

\begin{itemize}
\tightlist
\item
  \textbf{Pancreatic Neoplasms With Acinar Differentiation: A Review of
  Pathologic and Molecular Features }
\end{itemize}

\emph{Elizabeth D. Thompson and Laura D. Wood}

\url{https://www.archivesofpathology.org/doi/pdf/10.5858/arpa.2019-0472-RA}

\begin{itemize}
\tightlist
\item
  \textbf{Pancreatic Neuroendocrine Neoplasms: Landscape and Horizon }
\end{itemize}

\emph{Laura H. Tang}

\url{https://www.archivesofpathology.org/doi/pdf/10.5858/arpa.2019-0654-RA}

\begin{itemize}
\tightlist
\item
  \textbf{Pancreatic Solid Pseudopapillary Neoplasm: Key Pathologic and
  Genetic Features }
\end{itemize}

\emph{Stefano La Rosa and Massimo Bongiovanni}

\url{https://www.archivesofpathology.org/doi/pdf/10.5858/arpa.2019-0473-RA}

\begin{itemize}
\tightlist
\item
  \textbf{Pathology of Treated Pancreatic Ductal Adenocarcinoma and Its
  Clinical Implications }
\end{itemize}

\emph{Teddy Sutardji Nagaria, Hua Wang, Deyali Chatterjee and Huamin
Wang}

\url{https://www.archivesofpathology.org/doi/pdf/10.5858/arpa.2019-0477-RA}

\begin{center}\rule{0.5\linewidth}{0.5pt}\end{center}

Back to top

\begin{center}\rule{0.5\linewidth}{0.5pt}\end{center}

\pagebreak

\hypertarget{pancreas}{%
\subsection{Pancreas}\label{pancreas}}

\begin{center}\rule{0.5\linewidth}{0.5pt}\end{center}

\begin{itemize}
\tightlist
\item
  \textbf{Carboxypeptidase A1 and regenerating islet-derived 1α as new
  markers for pancreatic acinar cell carcinoma}
\end{itemize}

\emph{Human pathology 2020 Jul;103():120-126}

PubMed: \url{https://www.ncbi.nlm.nih.gov/pubmed/?term=32702400}

Acinar cell carcinoma (ACC) is a rare tumor that differentiates
toward~pancreatic acinar cells and shows evidence of pancreatic enzyme
production. Mixed acinar-neuroendocrine carcinoma (MANC) is defined as
having more than 30\% of both acinar and neuroendocrine cell types as
per immunohistochemistry analysis. Trypsin is currently the most
commonly used stain for acinar differentiation. In this study, we
investigate the utility of two novel markers, carboxypeptidase A1 (CPA1)
and regenerating islet-derived 1α (REG1a), in diagnosing ACC/MANC.
Immunohistochemical staining for CPA1 and REG1a was performed on 14
cases of ACC and 5 cases of MANC~as well as on 80 other pancreatic
tumors including 20 cases each of~ductal adenocarcinoma,
well-differentiated neuroendocrine tumor, mucinous cystic neoplasm, and
solid pseudopapillary tumor. All ACCs and MANCs were positive for CPA1
(all diffuse) and REG1a (12 diffuse, 4 patchy, and 3 focal). A diffuse
or patchy staining pattern was significantly more common in ACC/MANC
cases (100\% diffuse/patchy for CPA1 and 84\% for REG1a) than in other
pancreatic tumors (5\% diffuse/patchy for CPA1 and 7.5\% for REG1a),
with a P-value of \textless0.0001 for both CPA1 and REG1a. The
sensitivity and specificity of diffuse/patchy staining for CPA1 and
REG1a in diagnosing pancreatic ACC/MANC were 100\% and 95\% for CPA1 and
84\% and 93\% for REG1a, respectively. In conclusion, CPA1 and REG1a are
sensitive markers for ACC that can be used as additional acinar cell
differentiation markers to help in the diagnosis of pancreatic ACC and
MANC. A negative result for CPA1 virtually excludes ACC/MANC.

doi: \url{https://doi.org/10.1016/j.humpath.2020.07.019}

\begin{center}\rule{0.5\linewidth}{0.5pt}\end{center}

\begin{center}\rule{0.5\linewidth}{0.5pt}\end{center}

\begin{itemize}
\tightlist
\item
  \textbf{The North American Neuroendocrine Tumor Society Consensus
  Guidelines for Surveillance and Medical Management of Pancreatic
  Neuroendocrine Tumors}
\end{itemize}

\emph{Pancreas 2020 Aug;49(7):863-881}

PubMed: \url{https://www.ncbi.nlm.nih.gov/pubmed/?term=32675783}

This article is the result of the North American Neuroendocrine Tumor
Society consensus conference on the medical management of pancreatic
neuroendocrine tumors from July 19 to 20, 2018. The guidelines panel
consisted of medical oncologists, pathologists, gastroenterologists,
endocrinologists, and radiologists. The panel reviewed a series of
questions regarding the medical management of patients with pancreatic
neuroendocrine tumors as well as questions regarding surveillance after
resection. The available literature was reviewed for each of the
question and panel members voted on controversial topics, and the
recommendations were included in a document circulated to all panel
members for a final approval.

doi: \url{https://doi.org/10.1097/MPA.0000000000001597}

\begin{center}\rule{0.5\linewidth}{0.5pt}\end{center}

\begin{center}\rule{0.5\linewidth}{0.5pt}\end{center}

\begin{itemize}
\tightlist
\item
  \textbf{Epithelial-mesenchymal transition in undifferentiated
  carcinoma of the pancreas with and without osteoclast-like giant
  cells}
\end{itemize}

\emph{Virchows Archiv : an international journal of pathology 2020
Jul;():}

PubMed: \url{https://www.ncbi.nlm.nih.gov/pubmed/?term=32661742}

Undifferentiated carcinoma (UC) and undifferentiated carcinoma with
osteoclast-like giant cells (UCOGC) are peculiar variants of pancreatic
ductal adenocarcinoma (PDAC), characterized by hypercellularity and
absence of glandular patterns. The inflammatory microenvironment is
peculiar in UCOGC, since it is dominated by macrophages and
osteoclast-like giant cells. However, from a molecular point of view,
both UC and UCOGC are very similar to conventional PDAC, sharing
alterations of the most common genetic drivers. Clinically, UC usually
show a worse prognosis, whereas UCOGC may show a better prognosis if it
is not associated with a PDAC component. To highlight potential
biological differences between these entities, we investigated the role
of the epithelial to mesenchymal transition (EMT) in UC and UCOGC.
Specifically, we analyzed the immunohistochemical expression of three
well-known EMT markers, namely Twist1, Snai2, and E-cadherin, in 16
cases of UCOGC and 10 cases of UC. We found that EMT is more frequently
activated in UC (10/10 cases) than in UCOGC (8/16 cases; p = 0.05).
Furthermore, in UCOGC, EMT was activated with a higher frequency in
cases with an associated PDAC component. Snai2 was the most frequently
and strongly expressed marker in both tumor types (10/10 UC, 8/16
UCOGC), and its expression was higher in UC than in UCOGC (mean
immunohistochemical score: 4.8 in UC vs.~2.1 in UCOGC,
p \textless{} 0.01). Our results shed new light on the biology of UC and
UCOGC: EMT appeared as a more important process in UC, and Snai2 emerged
as a central EMT effector in this setting.

doi: \url{https://doi.org/10.1007/s00428-020-02889-3}

\begin{center}\rule{0.5\linewidth}{0.5pt}\end{center}

\begin{center}\rule{0.5\linewidth}{0.5pt}\end{center}

\begin{itemize}
\tightlist
\item
  \textbf{Evaluation of Pathologic Response on Overall Survival After
  Neoadjuvant Therapy in Pancreatic Ductal Adenocarcinoma}
\end{itemize}

\emph{Pancreas 2020 Aug;49(7):897-903}

PubMed: \url{https://www.ncbi.nlm.nih.gov/pubmed/?term=32658070}

OBJECTIVES: Single-institution studies have shown improved outcomes
among patients with a pathologic complete response (pCR) following
neoadjuvant therapy. We sought to evaluate the impact of pCR and
near-complete response (nCR) on overall survival (OS) using a large
national database. METHODS: The National Cancer Database was queried for
patients given a diagnosis of pancreatic cancer from 2004 to 2014. A pCR
was defined as no tumor identified in the pancreas after surgical
resection. An nCR was defined as a primary tumor less than 1 cm without
lymph node metastases. The primary outcome was OS. RESULTS: A total of
5364 patients underwent neoadjuvant chemotherapy and/or radiation
followed by pancreatectomy. Forty-one patients (0.8\%) had a pCR, 54
(1\%) had an nCR, and the remaining 5266 (98.2\%) had an otherwise
incomplete response. Patients with pCR had a median OS of 43 months
compared with 24 months for nCR and 23 months for incomplete response (P
\textless{} 0.0001). Only pCR was associated with improved OS on
adjusted Cox regression. CONCLUSIONS: For patients given a diagnosis of
pancreatic cancer who underwent neoadjuvant treatment and surgical
resection, achieving a pCR was associated with improved OS compared with
those with residual tumor. An association between nCR and improved
survival was not observed.

doi: \url{https://doi.org/10.1097/MPA.0000000000001590}

\begin{center}\rule{0.5\linewidth}{0.5pt}\end{center}

\begin{center}\rule{0.5\linewidth}{0.5pt}\end{center}

\begin{itemize}
\tightlist
\item
  \textbf{Comparison of Different Anti-Ki67 Antibody Clones and Hotspot
  Sizes for Assessing Proliferative Index and Grading in Pancreatic
  Neuroendocrine Tumours using Manual and Image Analysis}
\end{itemize}

\emph{Histopathology 2020 Jul;():}

PubMed: \url{https://www.ncbi.nlm.nih.gov/pubmed/?term=32617996}

AIMS: Ki67 proliferative index (PI) is essential for grading
gastroenteric and pancreatic neuroendocrine tumours (GEP NETs).
Analytical and preanalytical variables can affect Ki67 PI. In contrast
to counting methodology, until now little attention has focussed on the
question of clone equivalence and effect of hotspot size on Ki67 PI in
GEP NETs. Using manual counting and image analysis, this study compared
the Ki67 PI achieved using MM1, K2 and 30-9 to MIB1, a clone which has
been validated for, and is referenced in guidelines relating to,
assessment of Ki67 PI in GEP NETs. METHODS AND RESULTS: 42 pancreatic
NETs were each immunohistochemically stained for the anti-Ki67 clones
MIB1, MM1, K2 and 30-9. Ki67 PI was calculated manually and by image
analysis, the latter using 3 different hotspot sizes. In manual
comparisons using single hotspot high power fields, non-MIB1 clones
overestimated Ki67 PI compared to MIB1, resulting in grading
discordances. Image analysis shows good agreement with manual Ki67 PI
but a tendency to overestimate absolute Ki67 PI. Increasing the size of
tumour hotspot from 500 to 2000 cells resulted in a decrease in Ki67 PI.
CONCLUSION: Different anti-Ki67 clones do not produce equivalent PIs in
GEP NETs and clone selection may therefore affect patient care.
Increasing the hotspot size decreases the Ki67 PI. Greater
standardisation in terms of antibody clone selection and hotspot size is
required for grading GEP NETs. Image analysis is an effective tool for
assisting Ki67 assessment and allows easier standardisation of the size
of the tumour hotspot.

doi: \url{https://doi.org/10.1111/his.14200}

\begin{center}\rule{0.5\linewidth}{0.5pt}\end{center}

\begin{center}\rule{0.5\linewidth}{0.5pt}\end{center}

\begin{itemize}
\tightlist
\item
  \textbf{Prevalence of histological features resembling autoimmune
  pancreatitis in neoplastic pancreas resections}
\end{itemize}

\emph{Histopathology 2020 Jul;():}

PubMed: \url{https://www.ncbi.nlm.nih.gov/pubmed/?term=32608526}

INTRODUCTION: Type 1 and type 2 autoimmune pancreatitis (AIP) can mimic
pancreatic neoplasia. Due to the small quantity of tissue in
mass-targeted pancreas biopsies, inflammatory features may raise the
differential of AIP. However, the frequency of AIP-like histology in
neoplastic pancreas is not well characterized. Therefore, the
specificity of inflammatory lesions on biopsy with respect to the
diagnosis of AIP is uncertain. MATERIALS AND METHODS: Neoplastic
pancreas resections performed at our institution between 2008 to 2019
were retrospectively reviewed. Features of AIP type 1 and 2 were
assessed in the non-neoplastic areas. If features of IgG4-associated AIP
were seen, IgG4 immunohistochemistry was performed. RESULTS: We
identified 163 neoplastic pancreas resections. Of these, 34 had one or
more types of inflammatory lesions in non-neoplastic pancreatic tissue.
Dense lymphoplasmacytic inflammation mimicking type 1 AIP was found in 6
cases with mild to moderately increased IgG4 positive plasma cells.
Neutrophilic infiltrates in small intralobular ducts were found in 20
cases. Mild extra-lobular ductitis or duct microabscess was found in 10
specimens. Marked neutrophilic duct destruction that resembled
granulocytic epithelial lesions was found in 12 cases. Some cases showed
multiple features. CONCLUSION: Approximately 20\% of neoplastic pancreas
resections showed focal areas that could raise the differential of AIP.
More cases showed neutrophilic predominant inflammation, as seen in type
2 autoimmune pancreatitis, compared to dense lymphoplasmacytic
infiltrates seen in type 1 AIP. Pathologists must be cautious when
making a diagnosis of AIP on biopsy tissue based on histological
findings alone.

doi: \url{https://doi.org/10.1111/his.14197}

\begin{center}\rule{0.5\linewidth}{0.5pt}\end{center}

\begin{center}\rule{0.5\linewidth}{0.5pt}\end{center}

\begin{itemize}
\tightlist
\item
  \textbf{Next-generation sequencing of residual cytologic fixative
  preserved DNA from pancreatic lesions: A pilot study}
\end{itemize}

\emph{Cancer cytopathology 2020 Jun;():}

PubMed: \url{https://www.ncbi.nlm.nih.gov/pubmed/?term=32598087}

BACKGROUND: Endoscopic ultrasound-guided fine needle aspiration
(EUS-FNA) is a sensitive and specific tool in the risk stratification of
pancreatic lesions, including cysts. The sensitivity and specificity of
EUS-FNA has been shown to improve when cytology is combined with
next-generation sequencing (NGS). Ideally, fresh cyst fluid is used for
NGS. In this pilot study, we explore the possibility of sequencing DNA
derived from residual alcohol-fixed pancreatic aspirates. METHODS:
Residual cytologic fixatives (n~=~42) from 39 patients who underwent
EUS-FNA for pancreatic lesions were collected along with demographics,
imaging, and laboratory studies. Samples were designated as
nonneoplastic/nonmucinous benign (NB), mucinous cyst (MC), pancreatic
ductal adenocarcinoma (PDAC), or well-differentiated neuroendocrine
tumor (NET) on the basis of cytopathologic evaluation and sequenced on
the Oncomine platform (ThermoFisher Scientific, Waltham, Massachusetts).
RESULTS: Ten of 14 (71.4\%) MCs exhibited clinically significant
variants, including KRAS, GNAS, and TP53. Ten of 15 (66.7\%) PDACs had
KRAS alterations, and 9 of 15 (60\%) showed variants in TP53. No
variants were detected in any NETs. Only 1 of 9 (11.1\%) NB aspirates
showed variants in KRAS and MAP2K. Sequencing of formalin-fixed,
paraffin-embedded tissue revealed variants identical to those detected
in fixative-derived DNA in 4 of 5 cases (80\%). CONCLUSION: Residual DNA
from alcohol-fixed aspirates are an underutilized source for NGS.
Sequencing residual fixative-derived DNA has the potential to be
integrated into the workup of pancreatic aspirates, possibly impacting
management.

doi: \url{https://doi.org/10.1002/cncy.22315}

\begin{center}\rule{0.5\linewidth}{0.5pt}\end{center}

\begin{center}\rule{0.5\linewidth}{0.5pt}\end{center}

\begin{itemize}
\tightlist
\item
  \textbf{Alterations in driver genes are predictive of survival in
  patients with resected pancreatic ductal adenocarcinoma}
\end{itemize}

\emph{Cancer 2020 Sep;126(17):3939-3949}

PubMed: \url{https://www.ncbi.nlm.nih.gov/pubmed/?term=32573775}

BACKGROUND: KRAS, TP53, CDKN2A, and SMAD4 are established driver genes
in pancreatic ductal adenocarcinoma (PDAC). This study was aimed at
determining whether the mutational status of driver genes and those
involved in DNA repair pathways are associated with clinical outcomes
for individuals who undergo resection. METHODS: Eligible individuals
were those who underwent resection of PDAC and consented to targeted
sequencing of their primary tumor via Memorial Sloan
Kettering-Integrated Mutation Profiling of Actionable Cancer Targets
(MSK-IMPACT). Genomic alterations were determined on the basis of
MSK-IMPACT results from formalin-fixed, paraffin-embedded samples.
Associations between genomic alterations and clinical outcomes were
assessed. RESULTS: Targeted sequencing was performed on 283 primary
tumors resected between 2004 and 2017. The median follow-up was 23
months among survivors. Alterations in KRAS and TP53 were associated
with worse overall survival (OS) in comparison to wild type (median for
KRAS, 38.8 months {[}95\% CI, 33.0-45.5 months{]} vs 91.0 months {[}95\%
CI, 34.8 months to not available (NA){]}; P~=~.043; median for TP53,
37.4 months {[}95\% CI, 32.1-42.8 months{]} vs 65.0 months {[}95\% CI,
33.0 months to NA{]}; P~=~.035). KRAS G12D mutations were associated
with worse OS (median, 31.6 months {[}95\% CI, 25.3-45.5 months{]} vs
39.2 months {[}95\% CI, 37.4-75.2 months{]}; P~=~.012). TP53 truncating
mutations (median, 39.6 months {[}95\% CI, 32.4-75.2 months{]} vs 33.9
months {[}95\% CI, 24.0-39.0 months{]}; P~=~.020) and those associated
with loss of heterozygosity (median, 26.6 months {[}95\% CI, 21.6-44.2
months{]} vs 39.2 months {[}95\% CI, 34.5-49.1 months{]}; P~=~.048) had
decreased OS. TP53 alterations were independently associated with OS in
a multivariate analysis (hazard ratio, 1.54; 95\% CI, 1.01-2.33;
P~=~.042). Individuals with germline alterations in homologous
recombination deficiency (HRD) genes had improved OS in comparison with
those without them (median, not reached vs 37.0 months; 95\% CI,
33.0-49.8 months; P~=~.035). CONCLUSIONS: In patients with resected
PDAC, genomic alterations in KRAS and TP53 are associated with worse
outcomes, whereas alterations in HRD genes are associated with a
favorable prognosis. Further studies are needed to better define these
alterations as biomarkers in resected PDAC.

doi: \url{https://doi.org/10.1002/cncr.33038}

\begin{center}\rule{0.5\linewidth}{0.5pt}\end{center}

\begin{center}\rule{0.5\linewidth}{0.5pt}\end{center}

\begin{itemize}
\tightlist
\item
  \textbf{Morphologic Variants of Pancreatic Neuroendocrine Tumors:
  Clinicopathologic Analysis and Prognostic Stratification}
\end{itemize}

\emph{Endocrine pathology 2020 Sep;31(3):239-253}

PubMed: \url{https://www.ncbi.nlm.nih.gov/pubmed/?term=32488621}

Better prognostication/stratification of pancreatic neuroendocrine
tumors (PanNETs) is needed. In this detailed morpheomic study of 163
resected PanNETs, 11 unusual variants, some of which were not previously
recognized, and others scarcely documented in the literature, were
identified, and their pathologic characteristics were further analyzed.
By behavior and clinicopathologic associations, these variants could be
grouped into three prognostically different categories. I.~More
aggressive (20\%). Included in this group were the variants that in
average showed higher grade and stage and adverse outcome including
oncocytic, plasmacytoid, lipid-rich and previously unrecognized hepatoid
variants, which often had a more diffuse/broad-band growth pattern, with
some also displaying discohesiveness. They were characterized by
abundant cytoplasm and often had prominent nucleoli (as seen in
metabolically active cells), thus the provisional name ``metabolic cell
phenotype.'' Because of their diversion from classical neuroendocrine
cytomorphology, these variants created challenges on original diagnostic
workup, particularly hepatoid examples, which revealed Arginase 1/Hep
Par-1 expression in 50\%. II.~Less aggressive (10\%). These cases either
showed signs of maturation, including nested growth, paraganglioid
pattern (which was previously unrecognized), and organoid PanNETs such
as ``ductulo-insular'' growth, or showed symplastic/degenerative
changes, and despite their paradoxically disconcerting histology, were
more benevolent in behavior. III. Undetermined. There were other
variants including mammary tubulolobular-like, pseudoglandular,
peliotic, and sclerotic PanNETs, which although diagnostically
challenging, their biologic significance could not be determined because
of rarity or heterogeneous characteristics. Prognostic associations:
Features that were significantly different in the more aggressive group
than the less aggressive group were median size (5.0 vs 1.6 cm, p
\textless{} 0.001), percentage of pT3+T4 cases (72\% vs 12\%, p
\textless{} 0.001), Ki67 index (5.3\% vs 2.3\%, p = 0.001), \% G2 and G3
cases (77\% vs 27\%, p \textless{} 0.001), and rate of lymph node and
distant metastasis (96\% vs 27\%, p \textless{} 0.001). In stepwise
logistic regression model using the 3 established prognosticators of T
stage, size, and grade along with morphology, only aggressive-morphology
(metabolic cell phenotype) was found to be associated with metastatic
behavior with an odds ratio of 5.9 with 95\% confidence interval (C.I.)
1.688 to 22.945 and p value 0.007. In conclusion, PanNETs display
various morphologic patterns that are not only challenging and important
diagnostically but appear to have biologic significance. Tumors with
more diffuse growth of cells with nucleoli and abundant cytoplasm and/or
discohesion (oncocytic, hepatoid, lipid-rich, plasmacytoid PanNETs),
provisionally termed ``metabolic cell phenotype,'' show aggressive
characteristics and are an independent determinant of adverse outcome
and thus may require closer post-surgical follow-up, whereas variants
with more degenerative or mature features (ductuloinsular, pleomorphic,
paraganglioma-like) appear to be more benevolent despite their more
atypical and worrisome morphology.

doi: \url{https://doi.org/10.1007/s12022-020-09628-z}

\begin{center}\rule{0.5\linewidth}{0.5pt}\end{center}

\begin{center}\rule{0.5\linewidth}{0.5pt}\end{center}

\begin{itemize}
\tightlist
\item
  \textbf{Ki-67 proliferation index in neuroendocrine tumors: Can
  augmented reality microscopy with image analysis improve scoring?}
\end{itemize}

\emph{Cancer cytopathology 2020 Aug;128(8):535-544}

PubMed: \url{https://www.ncbi.nlm.nih.gov/pubmed/?term=32401429}

BACKGROUND: The Ki-67 index is important for grading neuroendocrine
tumors (NETs) in cytology. However, different counting methods exist.
Recently, augmented reality microscopy (ARM) has enabled real-time image
analysis using glass slides. The objective of the current study was to
compare different traditional Ki-67 scoring methods in cell block
material with newer methods such as ARM. METHODS: Ki-67 immunostained
slides from 50 NETs of varying grades were retrieved (39 from the
pancreas and 11 metastases). Methods with which to quantify the Ki-67
index in up to 3 hot spots included: 1) ``eyeball'' estimation (EE); 2)
printed image manual counting (PIMC); 3) ARM with live image analysis;
and 4) image analysis using whole-slide images (WSI) (field of view
{[}FOV{]} and the entire slide). RESULTS: The Ki-67 index obtained using
the different methods varied. The pairwise kappa results varied from no
agreement for image analysis using digital image analysis WSI (FOV) and
histology to near-perfect agreement for ARM and PIMC. Using surgical
pathology as the gold standard, the EE method was found to have the
highest concordance rate (84.2\%), followed by WSI analysis of the
entire slide (73.7\%) and then both the ARM and PIMC methods (63.2\% for
both). The PIMC method was the most time-consuming whereas image
analysis using WSI (FOV) was the fastest method followed by ARM.
CONCLUSIONS: The Ki-67 index for NETs in cell block material varied by
the method used for scoring, which may affect grade. PIMC was the most
time-consuming method, and EE had the highest concordance rate. Although
real-time automated counting using image analysis demonstrated
inaccuracies, ARM streamlined and hastened the task of Ki-67
quantification in NETs.

doi: \url{https://doi.org/10.1002/cncy.22272}

\begin{center}\rule{0.5\linewidth}{0.5pt}\end{center}

\begin{center}\rule{0.5\linewidth}{0.5pt}\end{center}

\begin{itemize}
\tightlist
\item
  \textbf{Simple mucinous cysts of the pancreas have heterogeneous
  somatic mutations}
\end{itemize}

\emph{Human pathology 2020 Jul;101():1-9}

PubMed: \url{https://www.ncbi.nlm.nih.gov/pubmed/?term=32380013}

Simple mucinous cysts of the pancreas have an epithelial lining
resembling pancreatic intraepithelial neoplasia but may have a clinical
presentation similar to premalignant mucinous neoplasms such as
intraductal papillary mucinous neoplasms. Whether the epithelial lining
shares genomic alterations with other pancreatic preinvasive neoplasms
such as PanIN and intraductal papillary mucinous neoplasm has not been
determined. We performed targeted sequencing analysis using a
custom-designed MiSeq panel including the full coding regions of 18
pancreatic cancer genes on 13 clinically and pathologically
well-characterized simple mucinous cysts. We detected 59 mutations in 15
genes in the cohort, with a median of 4 mutations per cyst (range~=~0-16
mutations per cyst). The mutated genes and rate of detected mutations
were as follows: KMT2C (MLL3) (62\%), KRAS (15\%), BRAF (8\%), RNF43
(8\%), CDKN2a (8\%), TP53 (15\%), and SMAD4 (8\%). No GNAS mutations
were detected. Four cases (31\%) had no mutations detected. These
findings place the majority of simple mucinous cysts of the pancreas in
the spectrum of early, low-grade mucinous neoplasia, albeit with
a~different spectrum of genomic alterations compared with PanIN and
intraductal papillary mucinous neoplasm.

doi: \url{https://doi.org/10.1016/j.humpath.2020.04.006}

\begin{center}\rule{0.5\linewidth}{0.5pt}\end{center}

\begin{center}\rule{0.5\linewidth}{0.5pt}\end{center}

\begin{itemize}
\tightlist
\item
  \textbf{Detection of Circulating Tumor DNA in Patients with Pancreatic
  Cancer Using Digital Next-Generation Sequencing}
\end{itemize}

\emph{The Journal of molecular diagnostics : JMD 2020 Jun;22(6):748-756}

PubMed: \url{https://www.ncbi.nlm.nih.gov/pubmed/?term=32205290}

Circulating tumor DNA (ctDNA) measurements can be used to estimate tumor
burden, but avoiding false-positive results is challenging. Herein,
digital next-generation sequencing (NGS) is evaluated as a ctDNA
detection method. Plasma KRAS and GNAS hotspot mutation levels were
measured in 140 subjects, including 67 with pancreatic ductal
adenocarcinoma and 73 healthy and disease controls. To limit chemical
modifications of DNA that yield false-positive mutation calls, plasma
DNA was enzymatically pretreated, after which DNA was aliquoted for
digital detection of mutations (up to 384 aliquots/sample) by PCR and
NGS. A digital NGS score of two SDs above the mean in controls was
considered positive. Thirty-seven percent of patients with pancreatic
cancer, including 31\% of patients with stages I/II disease, had
positive KRAS codon 12 ctDNA scores; only one patient had a positive
GNAS mutation score. Two disease control patients had positive ctDNA
scores. Low-normal-range digital NGS scores at mutation hotspots were
found at similar levels in healthy and disease controls, usually at
sites of cytosine deamination, and were likely the result of chemical
modification of plasma DNA and NGS error rather than true mutations.
Digital NGS detects mutated ctDNA in patients with pancreatic cancer
with similar yield to other methods. Detection of low-level,
true-positive ctDNA is limited by frequent low-level detection of
false-positive mutation calls in plasma DNA from controls.

doi: \url{https://doi.org/10.1016/j.jmoldx.2020.02.010}

\begin{center}\rule{0.5\linewidth}{0.5pt}\end{center}

\begin{center}\rule{0.5\linewidth}{0.5pt}\end{center}

\begin{itemize}
\tightlist
\item
  \textbf{Tumor-insular Complex in Neoadjuvant Treated Pancreatic Ductal
  Adenocarcinoma Is Associated With Higher Residual Tumor}
\end{itemize}

\emph{The American journal of surgical pathology 2020 Jun;44(6):817-825}

PubMed: \url{https://www.ncbi.nlm.nih.gov/pubmed/?term=32091434}

The tumor microenvironment in pancreatic ductal adenocarcinoma (PDAC)
plays a vital role in treatment response, and therefore, patient
survival. We and others have observed an intimate association of
neoplastic ductal cells with non-neoplastic islet cells, recapitulating
the ductoinsular complex. We define this phenomenon as tumor-insular
complex (TIC). Herein, we describe the clinicopathologic characteristics
of TIC in neoadjuvant treated PDAC cases for the first time. We
retrospectively reviewed the pathology of 105 cases of neoadjuvant
treated PDAC resected at our institution. TIC was noted in 35 cases
(33.3\%), the mean tumor bed size was 2.7±1.0 cm, mean percentage of
residual tumor 40±28\% and mean Residual Tumor Index (RTI) (an index
previously established as a prognostic parameter by our group) was
1.1±1.0. TIC was significantly associated with perineural invasion
(P=0.001), higher tumor bed size (P=0.007), percentage of residual tumor
(P=0.009), RTI (P=0.001), ypT stage (P=0.045), and poor treatment
response, grouped by a previously established criteria (P=0.010). Using
our prior binary reported prognostic cutoff for RTI of ≤0.35 and
\textgreater0.35, TIC was associated with a RTI \textgreater0.35
(P=0.002). Moreover, patients who did not receive neoadjuvant radiation
were associated with a higher frequency of TIC (P=0.003). In this
cohort, RTI but not TIC was also shown to be a significant independent
prognosticator for recurrence-free survival and overall survival on
multivariate analysis. In conclusion, TIC is significantly associated
with a more aggressive neoplasm which shows a poor treatment response.
Further studies will be needed to better understand the tumor biology of
TICs.

doi: \url{https://doi.org/10.1097/PAS.0000000000001454}

\begin{center}\rule{0.5\linewidth}{0.5pt}\end{center}

\begin{center}\rule{0.5\linewidth}{0.5pt}\end{center}

\begin{itemize}
\tightlist
\item
  \textbf{Insulinoma-associated protein 1 (INSM1) is a robust marker for
  identifying and grading pancreatic neuroendocrine tumors}
\end{itemize}

\emph{Cancer cytopathology 2020 04;128(4):269-277}

PubMed: \url{https://www.ncbi.nlm.nih.gov/pubmed/?term=31977134}

BACKGROUND: Pancreatic neuroendocrine tumor (PNET) is a diagnostic
challenge with limited samples in not only identification but grading.
Prior studies have shown insulinoma-associated protein 1 (INSM1) to be a
robust marker in identifying PNETs from other solid pancreatic tumors on
resection specimens. In this study, we investigated the utility of INSM1
not only for identifying PNETs but also for grading in cell blocks (CBs)
and surgical resections (SRs). METHODS: A search for PNET cases between
2000 and 2019 identified 55 samples (26 CBs and 29 SRs) that were
further separated into high (2 CBs, 3 SRs), intermediate (4 CBs, 7 SRs),
and low (20 CBs, 19 SRs) grades based on their final pathology report
and Ki-67 level. Immunohistochemical (IHC) staining for INSM1 (C-8,
Santa Cruz Biotechnology {[}1:100{]}) was performed and quantified using
an H score of 0 to 300. Non-PNET solid pancreatic tumors were compared
and included acinar cell carcinoma, solid pseudopapillary neoplasm, and
ductal adenocarcinoma. RESULTS: All 55 cases of PNET demonstrated
nuclear INSM1 staining. The average H scores for INSM1 staining of PNET
were 254 and 252 in CB and SR, respectively. The H scores decreased with
increasing tumor grade, with low-grade (G1), intermediate-grade (G2),
and high-grade (G3) tumors showing average INSM1 H scores of 229 and
253, 266 and 253, and 30 and 33 in both CB and SR, respectively.
CONCLUSION: IHC with INSM1 plays a role in identifying and potentially
grading PNETs.

doi: \url{https://doi.org/10.1002/cncy.22242}

\begin{center}\rule{0.5\linewidth}{0.5pt}\end{center}

\begin{center}\rule{0.5\linewidth}{0.5pt}\end{center}

\begin{itemize}
\tightlist
\item
  \textbf{Microscopic size measurements in post-neoadjuvant therapy
  resections of pancreatic ductal adenocarcinoma (PDAC) predict patient
  outcomes}
\end{itemize}

\emph{Histopathology 2020 Jul;77(1):144-155}

PubMed: \url{https://www.ncbi.nlm.nih.gov/pubmed/?term=31965618}

AIMS: Pancreatic ductal adenocarcinomas (PDACs) are increasingly being
treated with neoadjuvant therapy. However, the American Joint Committee
on Cancer (AJCC) 8th edition T staging based on tumour size does not
reflect treatment effect, which often results in multiple, small foci of
residual tumour in a background of mass-forming fibrosis. Thus, we
evaluated the performance of AJCC 8th edition T staging in predicting
patient outcomes by the use of a microscopic tumour size measurement
method. METHODS AND RESULTS: One hundred and six post-neoadjuvant
therapy pancreatectomies were reviewed, and all individual tumour foci
were measured. T stages based on gross size with microscopic adjustment
(GS) and the largest single microscopic focus size (MFS) were examined
in association with clinicopathological variables and patient outcomes.
Sixty-three of 106 (59\%) were locally advanced; 78\% received
FOLFIRINOX treatment. The average GS and MFS were 25~mm and 11~mm,
respectively; nine cases each were classified as T0, 35 and 85 cases as
T1, 42 and 12 cases as T2, and 20 and 0 cases as T3, based on the GS and
the MFS, respectively. Higher GS-based and MFS-based T stages were
significantly associated with higher tumour regression grade,
lymphovascular and perineural invasion, and higher N stage. Furthermore,
higher MFS-based T stage was significantly associated with shorter
disease-free survival (DFS) (P~\textless~0.001) and shorter overall
survival (OS) (P~=~0.002). GS was significantly associated with OS
(P~=~0.046), but not with DFS. CONCLUSIONS: In post-neoadjuvant therapy
PDAC resections, MFS-based T staging is superior to GS-based T staging
for predicting patient outcomes, suggesting that microscopic
measurements have clinical utility beyond the conventional use of GS
measurements alone.

doi: \url{https://doi.org/10.1111/his.14067}

\begin{center}\rule{0.5\linewidth}{0.5pt}\end{center}

\begin{center}\rule{0.5\linewidth}{0.5pt}\end{center}

\begin{itemize}
\tightlist
\item
  \textbf{Clinicopathologic and Prognostic Significance of Gallbladder
  and Cystic Duct Invasion in Distal Bile Duct Carcinoma}
\end{itemize}

\emph{Archives of pathology \& laboratory medicine 2020
Jun;144(6):755-763}

PubMed: \url{https://www.ncbi.nlm.nih.gov/pubmed/?term=31755778}

CONTEXT.---: The roles of the gallbladder and cystic duct (CD) invasions
in distal bile duct carcinoma (DBDC) have not been well elucidated.
OBJECTIVE.---: To define the characteristics and prognostic significance
of gallbladder or CD invasions in patients with DBDC. DESIGN.---: Organ
invasion patterns with clinicopathologic features were assessed in 258
resected DBDCs. RESULTS.---: CD invasions (N = 31) were associated with
frequent concomitant pancreatic and/or duodenal invasions (23 of 31,
74\%) and showed stromal infiltration (16 of 31, 52\%) and intraductal
cancerization (15 of 31, 48\%) patterns. In only 2 cases, invasions with
intraductal cancerization were observed in the gallbladder neck.
Conversely, all pancreatic (N = 175) and duodenal (83) invasions
developed through stromal infiltration. CD invasions were associated
with larger tumor size (P = .001), bile duct margin positivity (P =
.001), perineural invasions (P = .04), and higher N categories (P =
.007). Patients with pancreatic or duodenal invasions had significantly
lower survival rates than those without pancreatic (median, 31.0 versus
93.9 months) or duodenal (27.5 versus 56.8 months, P \textless{} .001,
both) invasions. However, those with gallbladder or CD invasions did not
have different survival times (P = .13). Patients with concomitant
gallbladder/CD and pancreatic/duodenal invasions demonstrated
significantly lower survival rates than those without organ invasions (P
\textless{} .001). CONCLUSIONS.---: Gallbladder invasions were rare in
DBDCs as neck invasions with intraductal cancerization. CD invasions
occurred by stromal infiltrations and intraductal cancerization, whereas
all pancreatic and duodenal invasions had stromal infiltration patterns.
Gallbladder and/or CD invasions did not affect survival rates of
patients with DBDC, while pancreatic and duodenal invasions affected
survival rates. Therefore, these differences in survival rates may
originate from the different invasive patterns of DBDCs.

doi: \url{https://doi.org/10.5858/arpa.2019-0218-OA}

\begin{center}\rule{0.5\linewidth}{0.5pt}\end{center}

\begin{center}\rule{0.5\linewidth}{0.5pt}\end{center}

\begin{itemize}
\tightlist
\item
  \textbf{Recurrent Rearrangements in PRKACA and PRKACB in Intraductal
  Oncocytic Papillary Neoplasms of the Pancreas and~Bile Duct}
\end{itemize}

\emph{Gastroenterology 2020 02;158(3):573-582.e2}

PubMed: \url{https://www.ncbi.nlm.nih.gov/pubmed/?term=31678302}

BACKGROUND \& AIMS: Intraductal oncocytic papillary neoplasms (IOPNs) of
the pancreas and bile duct contain epithelial cells with numerous, large
mitochondria and are cystic precursors to pancreatic ductal
adenocarcinoma (PDAC) and cholangiocarcinoma (CCA), respectively.
However, IOPNs do not have the genomic alterations found in other
pancreatobiliary neoplasms. In fact, no recurrent genomic alterations
have been described in IOPNs. PDACs without activating mutations in KRAS
contain gene rearrangements, so we investigated whether IOPNs have
recurrent fusions in genes. METHODS: We analyzed 20 resected pancreatic
IOPNs and 3 resected biliary IOPNs using a broad RNA-based targeted
sequencing panel to detect cancer-related fusion genes. Four invasive
PDACs and 2 intrahepatic CCAs from the same patients as the IOPNs, were
also available for analysis. Samples of pancreatic cyst fluid (n~= 5,
collected before surgery) and bile duct brushings (n~= 2) were analyzed
for translocations. For comparison, we analyzed pancreatobiliary lesions
from 126 patients without IOPN (controls). RESULTS: All IOPNs evaluated
were found to have recurring fusions of ATP1B1-PRKACB (n~= 13),
DNAJB1-PRKACA (n~= 6), or ATP1B1-PRKACA (n~= 4). These fusions also were
found in corresponding invasive PDACs and intrahepatic CCAs, as well as
in matched pancreatic cyst fluid and bile duct brushings. These gene
rearrangements were absent from all 126 control pancreatobiliary
lesions. CONCLUSIONS: We identified fusions in PRKACA and PRKACB genes
in pancreatic and biliary IOPNs, as well as in PDACs and pancreatic cyst
fluid and bile duct cells from the same patients. We did not identify
these gene fusions in 126 control pancreatobiliary lesions. These
fusions might be used to identify patients at~risk~for~IOPNs and their
associated invasive carcinomas.

doi: \url{https://doi.org/10.1053/j.gastro.2019.10.028}

\begin{center}\rule{0.5\linewidth}{0.5pt}\end{center}

\begin{center}\rule{0.5\linewidth}{0.5pt}\end{center}

\begin{itemize}
\tightlist
\item
  \textbf{DNAJB1-PRKACA fusions occur in oncocytic pancreatic and
  biliary neoplasms and are not specific for fibrolamellar
  hepatocellular carcinoma}
\end{itemize}

\emph{Modern pathology : an official journal of the United States and
Canadian Academy of Pathology, Inc 2020 04;33(4):648-656}

PubMed: \url{https://www.ncbi.nlm.nih.gov/pubmed/?term=31676785}

Recently discovered DNAJB1-PRKACA oncogenic fusions have been considered
diagnostic for fibrolamellar hepatocellular carcinoma. In this study, we
describe six pancreatobiliary neoplasms with PRKACA fusions, five of
which harbor the DNAJB1-PRKACA fusion. All neoplasms were subjected to a
hybridization capture-based next-generation sequencing assay
(MSK-IMPACT), which enables the identification of sequence mutations,
copy number alterations, and selected structural rearrangements
involving ≥410 genes (n = 6) and/or to a custom targeted, RNA-based
panel (MSK-Fusion) that utilizes Archer Anchored Multiplex PCR
technology and next-generation sequencing to detect gene fusions in 62
genes (n = 2). Selected neoplasms also underwent FISH analysis, albumin
mRNA in-situ hybridization, and arginase-1 immunohistochemical labeling
(n = 3). Five neoplasms were pancreatic, and one arose in the
intrahepatic bile ducts. All revealed at least focal oncocytic
morphology: three cases were diagnosed as intraductal oncocytic
papillary neoplasms, and three as intraductal papillary mucinous
neoplasms with mixed oncocytic and pancreatobiliary or gastric features.
Four cases had an invasive carcinoma component composed of oncocytic
cells. Five cases revealed DNAJB1-PRKACA fusions and one revealed an
ATP1B1-PRKACA fusion. None of the cases tested were positive for albumin
or arginase-1. Our data prove that DNAJB1-PRKACA fusion is neither
exclusive nor diagnostic for fibrolamellar hepatocellular carcinoma, and
caution should be exercised in diagnosing liver tumors with
DNAJB1-PRKACA fusions as fibrolamellar hepatocellular carcinoma,
particularly if a pancreatic lesion is present. Moreover, considering
DNAJB1-PRKACA fusions lead to upregulated protein kinase activity and
that this upregulated protein kinase activity has a significant role in
tumorigenesis of fibrolamellar hepatocellular carcinoma, protein kinase
inhibition could have therapeutic potential in the treatment of these
pancreatobiliary neoplasms as well, once a suitable drug is developed.

doi: \url{https://doi.org/10.1038/s41379-019-0398-2}

\begin{center}\rule{0.5\linewidth}{0.5pt}\end{center}

\begin{center}\rule{0.5\linewidth}{0.5pt}\end{center}

\begin{itemize}
\tightlist
\item
  \textbf{RET gene rearrangements occur in a subset of pancreatic acinar
  cell carcinomas}
\end{itemize}

\emph{Modern pathology : an official journal of the United States and
Canadian Academy of Pathology, Inc 2020 04;33(4):657-664}

PubMed: \url{https://www.ncbi.nlm.nih.gov/pubmed/?term=31558784}

Pancreatic acinar cell carcinoma is relatively rare (1 to 2\% of
pancreatic malignancies) but may be under-recognized. In contrast to
pancreatic ductal adenocarcinoma, most acinar cell carcinomas lack
mutations in KRAS, DPC, CDKN2A or TP53, but appear to have a high
incidence of gene rearrangements, with up to 20\% reported to be driven
by BRAF fusions. With the development of a new class of RET-specific
tyrosine kinase inhibitors, which appear to have particularly strong
activity against RET gene rearranged tumours, there is now considerable
interest in identifying RET gene rearrangements across a wide range of
cancers. RET rearrangements have been reported to occur at a very low
incidence (\textless1\%) in all pancreatic carcinomas. We postulated
that given its unique molecular profile, RET gene rearrangements may be
common in acinar cell carcinomas. We performed fluorescent in-situ
hybridization (FISH) studies on a cohort of 40 acinar cell spectrum
tumours comprising 36 pure acinar cell carcinomas, three
pancreatoblastomas and one mixed acinar-pancreatic neuroendocrine
tumour. RET gene rearrangements were identified in 3 (7.5\%) cases and
BRAF gene rearrangements in 5 (12.5\%). All gene rearranged tumours were
pure acinar cell carcinomas. Our findings indicate that amongst all
pancreatic carcinomas, acinar carcinomas are highly enriched for
potentially actionable gene rearrangements in RET or BRAF. FISH testing
is inexpensive and readily available in the routine clinical setting and
may have a role in the assessment of all acinar cell carcinomas-at this
stage to recruit patients for clinical trials of new targeted therapies,
but perhaps in the near future as part of routine care.

doi: \url{https://doi.org/10.1038/s41379-019-0373-y}

\begin{center}\rule{0.5\linewidth}{0.5pt}\end{center}

\begin{center}\rule{0.5\linewidth}{0.5pt}\end{center}

\begin{itemize}
\tightlist
\item
  \textbf{Whole-genome sequencing reveals distinct genetic bases for
  insulinomas and non-functional pancreatic neuroendocrine tumours:
  leading to a new classification system}
\end{itemize}

\emph{Gut 2020 May;69(5):877-887}

PubMed: \url{https://www.ncbi.nlm.nih.gov/pubmed/?term=31462556}

OBJECTIVE: Insulinomas and non-functional pancreatic neuroendocrine
tumours (NF-PanNETs) have distinctive clinical presentations but share
similar pathological features. Their genetic bases have not been
comprehensively compared. Herein, we used whole-genome/whole-exome
sequencing (WGS/WES) to identify genetic differences between insulinomas
and NF-PanNETs. DESIGN: The mutational profiles and copy-number
variation (CNV) patterns of 211 PanNETs, including 84 insulinomas and
127 NF-PanNETs, were obtained from WGS/WES data provided by Peking Union
Medical College Hospital and the International Cancer Genome Consortium.
Insulinoma RNA sequencing and immunohistochemistry data were assayed.
RESULTS: PanNETs were categorised based on CNV patterns: amplification,
copy neutral and deletion. Insulinomas had CNV amplifications and copy
neutral and lacked CNV deletions. CNV-neutral insulinomas exhibited an
elevated rate of YY1 mutations. In contrast, NF-PanNETs had all three
CNV patterns, and NF-PanNETs with CNV deletions had a high rate of
loss-of-function mutations of tumour suppressor genes. NF-PanNETs with
CNV alterations (amplification and deletion) had an elevated risk of
relapse, and additional DAXX/ATRX mutations could predict an increased
relapse risk in the first 2-year period. CONCLUSION: These WGS/WES data
allowed a comprehensive assessment of genetic differences between
insulinomas and NF-PanNETs, reclassifying these tumours into novel
molecular subtypes. We also proposed a novel relapse risk stratification
system using CNV patterns and DAXX/ATRX mutations.

doi: \url{https://doi.org/10.1136/gutjnl-2018-317233}

\begin{center}\rule{0.5\linewidth}{0.5pt}\end{center}

\begin{center}\rule{0.5\linewidth}{0.5pt}\end{center}

\begin{itemize}
\tightlist
\item
  \textbf{Sclerosing epithelioid mesenchymal neoplasm of the
  pancreas~-~a proposed new entity}
\end{itemize}

\emph{Modern pathology : an official journal of the United States and
Canadian Academy of Pathology, Inc 2020 03;33(3):456-467}

PubMed: \url{https://www.ncbi.nlm.nih.gov/pubmed/?term=31383964}

We have encountered pancreatic tumors with unique histologic features,
which do not conform to any of the known tumors of the pancreas or other
anatomical sites. We aimed to define their clinicopathologic features
and whether they are characterized by recurrent molecular signatures.
Eight cases were identified; studied histologically and by
immunohistochemistry. Selected cases were also subjected to whole-exome
sequencing (WES; n = 4), RNA-sequencing (n = 6), Archer FusionPlex assay
(n = 5), methylation profiling using the Illumina MethylationEPIC (850k)
array platform (n = 6), and TERT promoter sequencing (n = 5). Six
neoplasms occurred in females. The mean age was 43 years (range: 26-75).
Five occurred in the head/neck of the pancreas. All patients were
treated surgically; none received neoadjuvant/adjuvant therapy. All
patients are free of disease after 53 months of median follow-up (range:
8-94). The tumors were well-circumscribed, and the median size was
1.8 cm (range: 1.3-5.8). Microscopically, the unencapsulated tumors had
a geographic pattern of epithelioid cell nests alternating with spindle
cell fascicles. Some areas showed dense fibrosis, in which enmeshed
tumor cells imparted a slit-like pattern. The predominant epithelioid
cells had scant cytoplasm and round-oval nuclei with open chromatin. The
spindle cells displayed irregular, hyperchromatic nuclei. Mitoses were
rare. No lymph node metastases were identified. All tumors were positive
for vimentin, CD99 and cytokeratin (patchy), while negative for markers
of solid pseudopapillary neoplasm, neuroendocrine, acinar,
myogenic/rhabdoid, vascular, melanocytic, or lymphoid differentiation,
gastrointestinal stromal tumor as well as MUC4. Whole-exome sequencing
revealed no recurrent somatic mutations or amplifications/homozygous
deletions in any known oncogenes or tumor suppressor genes.
RNA-sequencing and the Archer FusionPlex assay did not detect any
recurrent likely pathogenic gene fusions. Single sample gene set
enrichment analysis revealed that these tumors display a likely
mesenchymal transcriptomic program. Unsupervised analysis (t-SNE) of
their methylation profiles against a set of different mesenchymal
neoplasms demonstrated a distinct methylation pattern. Here, we describe
pancreatic neoplasms with unique morphologic/immunophenotypic features
and a distinct methylation pattern, along with a lack of abnormalities
in any of key genetic drivers, supporting that these neoplasms represent
a novel entity with an indolent clinical course. Given their mesenchymal
transcriptomic features, we propose the designation of ``sclerosing
epithelioid mesenchymal neoplasm'' of the pancreas.

doi: \url{https://doi.org/10.1038/s41379-019-0334-5}

\begin{center}\rule{0.5\linewidth}{0.5pt}\end{center}

\begin{center}\rule{0.5\linewidth}{0.5pt}\end{center}

\begin{itemize}
\tightlist
\item
  \textbf{New Nodal Staging for Primary Pancreatic Neuroendocrine
  Tumors: A Multi-institutional and National Data Analysis}
\end{itemize}

\emph{Annals of surgery 2019 Jul;():}

PubMed: \url{https://www.ncbi.nlm.nih.gov/pubmed/?term=31356277}

OBJECTIVE: To determine the prognostic role of metastatic lymph node
(LN) number and the minimal number of LNs for optimal staging of
patients with pancreatic neuroendocrine tumors (pNETs). BACKGROUND:
Prognosis relative to number of LN metastasis (LNM), and minimal number
of LNs needed to evaluate for accurate staging, have been poorly defined
for pNETs. METHODS: Number of LNM and total number of LN evaluated
(TNLE) were assessed relative to recurrence-free survival (RFS) and
overall survival (OS) in a multi-institutional database. External
validation was performed using Surveillance, Epidemiology and End
Results (SEER) registry. RESULTS: Among 854 patients who underwent
resection, 233 (27.3\%) had at least 1 LNM. Patients with 1, 2, or 3 LNM
had a comparable worse RFS versus patients with no nodal metastasis
(5-year RFS, 1 LNM 65.6\%, 2 LNM 68.2\%, 3 LNM 63.2\% vs 0 LNM 82.6\%;
all P \textless{} 0.001). In contrast, patients with ≥4 LNM (proposed
N2) had a worse RFS versus patients who either had 1 to 3 LNM (proposed
N1) or node-negative disease (5-year RFS, ≥4 LNM 43.5\% vs 1-3 LNM
66.3\%, 0 LNM 82.6\%; all P \textless{} 0.05) {[}C-statistics area under
the curve (AUC) 0.650{]}. TNLE ≥8 had the highest discriminatory power
relative to RFS (AUC 0.713) and OS (AUC 0.726) among patients who had 1
to 3 LNM, and patients who had ≥4 LNM in the multi-institutional and
SEER database (n = 2764). CONCLUSIONS: Regional lymphadenectomy of at
least 8 lymph nodes was necessary to stage patients accurately. The
proposed nodal staging of N0, N1, and N2 optimally staged patients.

doi: \url{https://doi.org/10.1097/SLA.0000000000003478}

\begin{center}\rule{0.5\linewidth}{0.5pt}\end{center}

\begin{center}\rule{0.5\linewidth}{0.5pt}\end{center}

\begin{itemize}
\tightlist
\item
  \textbf{GNAS but Not Extended RAS Mutations Spectrum are Associated
  with a Better Prognosis in Intraductal Pancreatic Mucinous Neoplasms}
\end{itemize}

\emph{Annals of surgical oncology 2019 Aug;26(8):2640-2650}

PubMed: \url{https://www.ncbi.nlm.nih.gov/pubmed/?term=31025231}

BACKGROUND: The management of intraductal papillary mucinous neoplasms
(IPMNs) is mainly based on imaging features and clinical symptoms, and
remains challenging. OBJECTIVE: The aim of this study was to assess
GNAS, RAS family (KRAS, NRAS and HRAS), BRAF, and PIK3CA mutation status
in resected IPMNs and correlate it with clinicopathological
characteristics and patient survival. METHODS: Overall, 149 consecutive
unselected patients who underwent pancreatectomy for IPMNs were
included. After dissection from formalin-fixed and paraffin-embedded
tumors, GNAS mutational screening was assessed by allelic discrimination
using Taqman® probes and confirmed by SNaPshot analysis. RAS family,
BRAF, and PIK3CA mutational screening was assessed by high resolution
melt and Sanger sequencing. RESULTS: Gastric- and intestinal-type IPMNs
were the most frequent lesions (52\% and 41\%, respectively).
Intestinal-type IPMNs were more frequently associated high-grade
dysplasia (49\%) and were the only IPMNs associated with colloid-type
carcinoma. All pancreatobiliary IPMNs were invasive lesions, located in
the main pancreatic duct. GNAS-activating mutations were strongly
associated with the intestinal phenotype (p \textless{} 10-4), while RAS
pathway mutations were not associated with any particular phenotype.
Mutations within other members of the epidermal growth factor receptor
(EGFR) pathway were very rare (2\%). GNAS-mutated IPMNs were rarely
invasive (11\%) and almost exclusively (83\%) of the colloid type. For
invasive lesions, multivariate analyses determined that only node
negativity was associated with improved cancer-specific survival, but,
in univariate analysis, GNAS mutation was associated with prolonged
survival. CONCLUSION: In patients selected for surgery, GNAS mutation
analysis and tumor phenotype can help to better predict patient
prognosis. In the near future, a more precise mutational analysis of
IPMNs might help to better tailor their management.

doi: \url{https://doi.org/10.1245/s10434-019-07389-6}

\begin{center}\rule{0.5\linewidth}{0.5pt}\end{center}

\begin{center}\rule{0.5\linewidth}{0.5pt}\end{center}

Back to top

\begin{center}\rule{0.5\linewidth}{0.5pt}\end{center}

\pagebreak

\hypertarget{gallbladder}{%
\subsection{Gallbladder}\label{gallbladder}}

\begin{center}\rule{0.5\linewidth}{0.5pt}\end{center}

\begin{itemize}
\tightlist
\item
  \textbf{Neuroendocrine Carcinomas of the Gallbladder: A
  Clinicopathologic and Immunohistochemical Analysis of 34 Resected
  Cases}
\end{itemize}

\emph{The American journal of surgical pathology 2020 Jul;():}

PubMed: \url{https://www.ncbi.nlm.nih.gov/pubmed/?term=32739935}

Neuroendocrine carcinoma (NEC) is an aggressive malignant tumor that
rarely arises from the gallbladder. Here, we investigated the
clinicopathologic and immunohistochemical characteristics of 34 NECs of
the gallbladder. The patients were predominantly women (68\%) with a
median age of 63 years (range, 37 to 82 y). NECs frequently occurred in
the fundus (44\%) as mass-forming lesions (66\%). Histologically, 17
tumors were of small cell type, and another 17 were of large cell type.
Twenty-three cases (68\%) were associated with biliary intraepithelial
neoplasia (38\%) and intracholecystic papillary neoplasm (29\%). The
majority of tumors exhibited a diffuse growth pattern (74\%), followed
by organoid (24\%) or scirrhous (2\%) growth patterns. Histologic
features related to neuroendocrine differentiation, such as nuclear
molding (56\%), perilobular pseudopalisading (18\%), and rosette
formation (15\%), were identified. Immunohistochemically, cytokeratin 7
and 20 were expressed in 19 (56\%) and 8 (24\%) cases, respectively.
Loss of Rb1 expression and concomitant overexpression of p16 were
observed in 25 (74\%) cases. No BRAF mutations were identified in any of
the 34 NECs. For survival analysis, the 1-, 3-, and 5-year overall
survival rates were 64\%, 35\%, and 19\%, respectively. In a
multivariate analysis, the receipt of adjuvant chemoradiation therapy
was identified as the only independent prognostic factor associated with
the overall survival rate. The 1- and 3-year overall survival rates of
patients with NECs were poorer for patients with poorly differentiated
adenocarcinoma of the gallbladder (P\textless0.001). The complete
resection and application of postoperative adjuvant therapy may
influence a better clinical outcome in patients with NEC of the
gallbladder.

doi: \url{https://doi.org/10.1097/PAS.0000000000001536}

\begin{center}\rule{0.5\linewidth}{0.5pt}\end{center}

\begin{center}\rule{0.5\linewidth}{0.5pt}\end{center}

\begin{itemize}
\tightlist
\item
  \textbf{Intracholecystic tubular non-mucinous neoplasm (ICTN) of the
  gallbladder: a clinicopathologically distinct, invasion-resistant
  entity}
\end{itemize}

\emph{Virchows Archiv : an international journal of pathology 2020
Jul;():}

PubMed: \url{https://www.ncbi.nlm.nih.gov/pubmed/?term=32691142}

Preinvasive tumor-forming gallbladder neoplasms that are composed of
small, non-mucinous tubules with complex architecture remain a poorly
characterized group. Here, we evaluated the clinicopathological
characteristics of this entity. Twenty-eight examples were analyzed.
Tumors were invariably pedunculated polyps with thin stalks, often
presented as loosely attached intraluminal nodules, with cauliflower
architecture (akin to cholesterol polyps) comprised of compact,
back-to-back acinar-like, small tubular units with minimal/no cytoplasm
showing variable complexity, creating a picture distinct from the other
tubular type dysplasia in the gallbladder. Their limited stroma showed
distinctive amorphous amyloid-like hyalinization (39\%). While some had
round nuclei with single prominent nucleoli, others exhibited slightly
more elongated nuclei with washed out chromatin~reminiscent of papillary
thyroid carcinoma. Squamoid/meningothelial-like morules (71\%) and
subtle neuroendocrine cell clusters (39\%) were frequent. The level of
cytoarchitectural atypia qualified as high-grade dysplasia (HGD) in all
cases, but none were invasive. The background mucosa showed no
dysplasia, but cholesterolosis. The majority (n = 8/12) showed diffuse
MUC6 expression and lacked MUC5AC expression. Based on these
observations, 635 gallbladder carcinomas were re-analyzed for
residual/adjacent lesions with entity-defining characteristics disclosed
here, and none could be identified. Preinvasive tubular non-mucinous
neoplasm of the gallbladder, which we propose to classify as
intracholecystic tubular non-mucinous neoplasm, is a
clinicopathologically discrete entity, which tends to occur in uninjured
gallbladders and in association with cholesterol polyps. By being
tubular, non-mucinous and MUC6-positive, it is akin to intraductal
tubulopapillary neoplasms of pancreatobiliary tract, but it is~also
different in many other aspects. Although their cytoarchitectural
complexity warrants an HGD/carcinoma classification, they do not show
invasion and their distinct characteristics warrant their separate
classification.

doi: \url{https://doi.org/10.1007/s00428-020-02877-7}

\begin{center}\rule{0.5\linewidth}{0.5pt}\end{center}

\begin{center}\rule{0.5\linewidth}{0.5pt}\end{center}

\begin{itemize}
\tightlist
\item
  \textbf{Long-term outcomes of surgical resection for T1b gallbladder
  cancer: an institutional evaluation}
\end{itemize}

\emph{BMC cancer 2020 Jan;20(1):20}

PubMed: \url{https://www.ncbi.nlm.nih.gov/pubmed/?term=31907021}

BACKGROUND: There is no comprehensive agreement concerning the overall
performance of radical resection for T1b gallbladder cancer (GBC). This
research focused on addressing whether T1b GBC may spread
loco-regionally and whether radical resection is necessary. METHODS: A
retrospective analysis was conducted of 1032 patients with GBC who
underwent surgical resection at our centre and its affiliated
institutions between January 1982 and December 2018. A total of 47
patients with T1b GBC, 29 (62\%) of whom underwent simple
cholecystectomy and 18 (38\%) of whom underwent radical resection with
regional lymph node dissection, were enrolled in the study. RESULTS: GBC
was diagnosed pre-operatively in 16 patients (34\%), whereas 31 patients
(66\%) had incidental GBC. There was no blood venous or perineural
invasion in any patient on histology evaluation, except for lymphatic
vessel invasion in a single patient. There were no metastases in any
analysed lymph nodes. The open surgical approach was more prevalent
among the 18 patients who underwent radical resection (open in all 18
patients) than among the 29 patients who underwent simple
cholecystectomy (open in 21; laparoscopic in 8) (P = 0.017). The
cumulative 10- and 20-year overall survival rates were 65 and 25\%,
respectively. The outcome following simple cholecystectomy (10-year
overall survival rate of 66\%) was akin to that following radical
resection (64\%, P = 0.618). The cumulative 10- and 20-year
disease-specific survival rates were 93 and 93\%, respectively. The
outcome following simple cholecystectomy (10-year disease-specific
survival rate of 100\%) was equivalent to that following radical
resection (that of 86\%, P = 0.151). While age (\textgreater{} 70 years,
hazard ratio 5.285, P = 0.003) and gender (female, hazard ratio 0.272,
P = 0.007) had a strong effect on patient overall survival, surgical
procedure (simple cholecystectomy vs.~radical resection) and surgical
approach (open vs.~laparoscopic) did not. CONCLUSIONS: Most T1b GBCs
represent local disease. As pre-operative diagnosis, including tumour
penetration of T1b GBC, is difficult, the decision of radical resection
is justified. Additional radical resection is not required following
simple cholecystectomy provided that the penetration depth is restricted
towards the muscular layer and that surgical margins are uninvolved.

doi: \url{https://doi.org/10.1186/s12885-019-6507-2}

\begin{center}\rule{0.5\linewidth}{0.5pt}\end{center}

\begin{center}\rule{0.5\linewidth}{0.5pt}\end{center}

\begin{itemize}
\tightlist
\item
  \textbf{Non-neoplastic Polyps of the Gallbladder: A Clinicopathologic
  Analysis of 447 Cases}
\end{itemize}

\emph{The American journal of surgical pathology 2020 04;44(4):467-476}

PubMed: \url{https://www.ncbi.nlm.nih.gov/pubmed/?term=31725469}

There is no systematic histopathologic analysis of non-neoplastic polyps
in the gallbladder. In this study, in addition to a computer search for
cases designated as ``polyp,'' a systematic review of 2533 consecutive
routinely sampled archival and 203 totally submitted prospective
cholecystectomies were analyzed for \textgreater2 mm polyps (cut-off was
based on radiologic sensitivity). A total of 447 non-neoplastic polyps
were identified. The frequency was 3\% in archival cases and 5\% in
totally submitted cases. Only 21 (5\%) were ≥1 cm. The average age was
52 years, and the female to male ratio was 3.1. Two distinct categories
were delineated: (1) injury-related polyps (n=273): (a)
Fibro(myo)glandular polyps (n=214) were small (mean=0.4 cm),
broad-based, often multiple (45\%), almost always (98\%)
gallstone-associated, and were composed of a mixture of
(myo)fibroblastic tissue/lobular glandular units with chronic
cholecystitis. Dysplasia seen in 9\% seemed to be secondary involvement.
(b) Metaplastic pyloric glands forming polypoid collections (n=42). (c)
Inflammatory-type polyps associated with acute/subacute injury (11
granulation tissue, 3 xanthogranulomatous, 3 lymphoid). (2) Cholesterol
polyps (n=174) occurred in uninjured gallbladders, revealing a very thin
stalk, edematous cores devoid of glands but with cholesterol-laden
macrophages in 85\%, and cholesterolosis in the uninvolved mucosa in
60\%. Focal low-grade dysplasia was seen in 3\%, always confined to the
polyp, unaccompanied by carcinoma. In conclusion, non-neoplastic polyps
are seen in 3\% of cholecystectomies and are often small. Injury-related
fibromyoglandular polyps are the most common. Cholesterol polyps have
distinctive cauliflower architecture, often in a background of uninjured
gallbladders with cholesterolosis and may lack the cholesterol-laden
macrophages in the polyp itself. Although dysplastic changes can involve
non-neoplastic polyps, they do not seem to be the cause of invasive
carcinoma by themselves.

doi: \url{https://doi.org/10.1097/PAS.0000000000001405}

\begin{center}\rule{0.5\linewidth}{0.5pt}\end{center}

\begin{center}\rule{0.5\linewidth}{0.5pt}\end{center}

Back to top

\begin{center}\rule{0.5\linewidth}{0.5pt}\end{center}

\pagebreak

\hypertarget{bileducts}{%
\subsection{Bile Ducts}\label{bileducts}}

\begin{center}\rule{0.5\linewidth}{0.5pt}\end{center}

\begin{itemize}
\tightlist
\item
  \textbf{Biliary intraductal tubule-forming neoplasm: a whole exome
  sequencing study of MUC5AC-positive and -negative cases}
\end{itemize}

\emph{Histopathology 2020 Jun;76(7):1005-1012}

PubMed: \url{https://www.ncbi.nlm.nih.gov/pubmed/?term=32181510}

AIMS: Biliary intraductal tubular neoplasms that are non-mucinous and
negative for mucin 5AC (MUC5AC) are called intraductal tubulopapillary
neoplasms (ITPNs). Intraductal tubular neoplasms with mucinous cytoplasm
and MUC5AC positivity also occur and their nature remains unclear,
although some pathologists may classify these as `intraductal papillary
neoplasms of the bile duct (IPNBs) of gastric type'. This study aimed to
elucidate genetic features of biliary intraductal tubular neoplasms.
METHODS AND RESULTS: Six resected cases of biliary intraductal neoplasm
with~\textgreater70\% tubular configuration were characterised by
clinicopathological examination and whole exome sequencing, and the
findings obtained were compared between MUC5AC-negative (n~=~2) and
-positive cases (n~=~4). The intraductal tumours consisted of the
pancreatobiliary-type epithelium with high-grade dysplasia arranged in
back-to-back tubules. Both of the two MUC5AC-negative cases were
non-invasive neoplasms and developed in the liver, whereas all
MUC5AC-positive cases had invasive carcinoma and were present in the
intrahepatic (n~=~2), perihilar (n~=~1) and distal bile ducts (n~=~1).
In an exome-sequencing study, MUC5AC-negative cases harboured mutations
in CTNNB1, SF3B1, BAP1 and BRCA1 (one case each). KRAS mutations were
observed in three of four MUC5AC-positive cases (75\%) but none of the
MUC5AC-negative neoplasms. Compared to published data, known driver
genes of other intraductal neoplasms of the pancreatobiliary system
(e.g.~APC, CTNNB1, STK11, GNAS and PIK3CA) were wild-type in all but one
MUC5AC-negative case with CTNNB1 mutation. Chromatin modifiers (ARID1A,
BAP1 and KMT2C) were also altered in MUC5AC-positive cases, similar to
usual cholangiocarcinomas. CONCLUSIONS: This exome-sequencing study
suggested that MUC5AC-negative biliary ITPNs are genetically distinct
from pancreatic ITPNs and IPNBs. They may also biologically differ from
MUC5AC-positive tubular neoplasms despite morphological resemblance.

doi: \url{https://doi.org/10.1111/his.14103}

\begin{center}\rule{0.5\linewidth}{0.5pt}\end{center}

\begin{center}\rule{0.5\linewidth}{0.5pt}\end{center}

\begin{itemize}
\tightlist
\item
  \textbf{Intraductal papillary neoplasms of the bile duct consist of
  two distinct types specifically associated with clinicopathological
  features and molecular phenotypes}
\end{itemize}

\emph{The Journal of pathology 2020 May;251(1):38-48}

PubMed: \url{https://www.ncbi.nlm.nih.gov/pubmed/?term=32100878}

Intraductal papillary neoplasm of the bile duct (IPNB) is a grossly
visible papillary biliary neoplasm with morphological variations and
occasional invasion. Recently a new classification of IPNB into type 1
and type 2 was proposed in which the type 1 IPNBs consist of fine
papillary neoplastic glands and the type 2 IPNBs consist of complex
branching glands, seldom with foci of solid-tubular components. However,
clinicopathological and molecular characteristics of these types of
IPNBs are yet to be identified. We aimed to uncover clinicopathological
and molecular characteristics of the types of IPNBs. Thirty-six IPNBs
were studied retrospectively. Clinicopathological features as well as
molecular alterations of 31 genes were evaluated by means of targeted
next-generation sequencing and immunohistochemical examination of
expression of mucin and cancer-associated molecules. The 36 IPNBs were
classified into 22 of type 1 and 14 of type 2. The type 1 IPNBs were
associated with a non-invasive phenotype, intestinal and oncocytic
subtypes, development in the intrahepatic bile duct, overt mucin
production, and a relatively good prognosis. The type 2 IPNBs were
associated with an invasive phenotype, the pancreatobiliary subtype,
development within the extrahepatic bile duct, and worse prognosis
compared with the type 1 IPNBs. In the molecular analysis, recurrent
mutations were found in TP53 (34.3\%), KRAS (31.4\%), STK11 (25.7\%),
CTNNB1 (17.1\%), APC (14.3\%), SMAD4 (14.3\%), GNAS (11.4\%), PBRM1
(11.4\%), ELF3 (8.6\%), KMT2C (8.6\%), NF1 (8.6\%), PIK3CA (8.6\%),
ARID1A (5.7\%), ARID2 (5.7\%), BAP1 (5.7\%), BRAF (5.7\%), EPHA6
(5.7\%), ERBB2 (5.7\%), ERBB3 (5.7\%), KMT2D (5.7\%), and RNF43 (5.7\%).
Mutations in KRAS and GNAS were enriched in the type 1 IPNBs, whereas
mutations in TP53, SMAD4, and KMT2C were enriched in the type 2 IPNBs.
These results indicate that IPNBs consist of two distinct types of
neoplasms specifically associated with clinicopathological features and
molecular phenotypes. © 2020 Pathological Society of Great Britain and
Ireland. Published by John Wiley \& Sons, Ltd.

doi: \url{https://doi.org/10.1002/path.5398}

\begin{center}\rule{0.5\linewidth}{0.5pt}\end{center}

\begin{center}\rule{0.5\linewidth}{0.5pt}\end{center}

\begin{itemize}
\tightlist
\item
  \textbf{Testing for ROS1, ALK, MET, and HER2 rearrangements and
  amplifications in a large series of biliary tract adenocarcinomas}
\end{itemize}

\emph{Virchows Archiv : an international journal of pathology 2020
Jul;477(1):33-45}

PubMed: \url{https://www.ncbi.nlm.nih.gov/pubmed/?term=32447492}

Biliary tract carcinomas are divided into intrahepatic, perihilar,
distal extrahepatic cholangiocarcinomas, and gallbladder
adenocarcinomas. Therapies targeting ROS1, ALK, MET, and HER2
alterations are currently evaluated in clinical trials. We assessed ROS1
and ALK translocations/amplifications as well as MET and HER2
amplifications for each tumor subtype by fluorescent in situ
hybridization (FISH) and immunohistochemistry (IHC) in 73 intrahepatic,
40 perihilar bile duct, 36 distal extrahepatic cholangiocarcinomas, and
45 gallbladder adenocarcinomas (n = 194). By FISH, we detected
targetable alterations in 5.2\% of cases (n = 10): HER2 and MET
amplifications were found in 4.1\% (n = 8) and 1.0\% (n = 2),
respectively. The HER2-amplified cases were mostly gallbladder
adenocarcinomas (n = 5). The MET- and HER2-amplified cases were all
positive by IHC. Fourteen cases without MET amplification were positive
by IHC, whereas HER2 over-expression was detected by IHC only in
HER2-amplified cases. We detected no ALK or ROS1 translocation or
amplification. Several alterations were consistent with aneuploidy: 24
cases showed only one copy of ROS1 gene, 4 cases displayed a profile of
chromosomal instability, and an over-representation of centromeric
alpha-satellite sequences was found in five cases. We confirm a
relatively high rate of HER2 amplifications in gallbladder
adenocarcinomas and the efficacy of IHC to screen these cases. Our
results also suggest the value of IHC to screen MET amplification.
Contrary to initial publications, ROS1 rearrangements seem to be very
rare in biliary tract adenocarcinomas. We confirm a relatively high
frequency of aneuploidy and chromosomal instability and reveal the
over-representation of centromeric alpha-satellite sequences in
intrahepatic cholangiocarcinomas.

doi: \url{https://doi.org/10.1007/s00428-020-02822-8}

\begin{center}\rule{0.5\linewidth}{0.5pt}\end{center}

\begin{center}\rule{0.5\linewidth}{0.5pt}\end{center}

Back to top

\begin{center}\rule{0.5\linewidth}{0.5pt}\end{center}

\pagebreak

\hypertarget{ampulla}{%
\subsection{Ampulla}\label{ampulla}}

\begin{center}\rule{0.5\linewidth}{0.5pt}\end{center}

\begin{itemize}
\tightlist
\item
  \textbf{Poorly Cohesive (Signet Ring Cell) Carcinoma of the Ampulla of
  Vater}
\end{itemize}

\emph{International journal of surgical pathology 2020
May;28(3):236-244}

PubMed: \url{https://www.ncbi.nlm.nih.gov/pubmed/?term=31612756}

In the ampulla of Vater, carcinomas with
``diffuse-infiltrative''/``signet ring cell'' morphology, designated as
``poorly cohesive carcinoma'' (PCC) in the WHO classification, are very
rare and poorly characterized. Nine cases with a classical PCC
morphology constituting \textgreater50\% of the tumor were identified.
Mean age was 64.8 years (vs 64.6 in ampullary carcinomas {[}ACs{]}) and
6 were males, 3 females. The mean invasive tumor size was 2.5 cm (vs 1.9
in ACs). Other morphologic patterns displayed included cord-like
infiltration (n=2), plasmacytoid cells (n=2), and microglandular
component (n=4), including goblet cell adenocarcinoma-like foci. None of
the cases were associated with dysplasia. By immunohistochemistry, the
carcinomas did not show intestinal differentiation (CDX2 0/9, CK20 1/9,
MUC2 3/9), MUC1 was positive in 4/9, MUC5AC was positive in 7/8.
E-cadherin loss was noted in 4/9. All cases were advanced stage
(6/9-pT3, 3/9-pT4) (vs 43\% in ACs). Lymph node metastases were
identified in 44\% (vs 45\% in AC). Six patients (67\%) died of disease
at a median of 25 months, 3 were alive at 13, 15, and 60 months. Overall
median survival was significantly worse than that of intestinal-type ACs
(26 vs 122 months, P = .006) and trended toward worse than
pancreatobiliary type (26 vs 42 months, P = .1). In conclusion, PCCs
constitute 2.45\% of all ACs. These present as advanced tumors and
express upper-gastrointestinal immunoprofile with frequent MUC5AC
labeling, which may be helpful in identifying subtle infiltration in the
surface mucosa since MUC5AC is not expressed in the ampullary mucosa.
Patients have poor prognosis.

doi: \url{https://doi.org/10.1177/1066896919880968}

\begin{center}\rule{0.5\linewidth}{0.5pt}\end{center}

\begin{center}\rule{0.5\linewidth}{0.5pt}\end{center}

Back to top

\begin{center}\rule{0.5\linewidth}{0.5pt}\end{center}

\pagebreak

\hypertarget{feedback}{%
\section{Feedback}\label{feedback}}

\href{https://docs.google.com/forms/d/e/1FAIpQLSeD3Z9J6Y7eMmiyM12f_SfAmHUlykb1zxZcwO6lg7cebGYQIQ/viewform}{Google
Feedback Form}

\hypertarget{disqus_thread}{}

Please enable JavaScript to view the comments powered by Disqus.

\begin{center}\rule{0.5\linewidth}{0.5pt}\end{center}

Back to top

\begin{center}\rule{0.5\linewidth}{0.5pt}\end{center}

\pagebreak

\end{document}
