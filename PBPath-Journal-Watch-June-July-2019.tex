\documentclass[]{article}
\usepackage{lmodern}
\usepackage{amssymb,amsmath}
\usepackage{ifxetex,ifluatex}
\usepackage{fixltx2e} % provides \textsubscript
\ifnum 0\ifxetex 1\fi\ifluatex 1\fi=0 % if pdftex
  \usepackage[T1]{fontenc}
  \usepackage[utf8]{inputenc}
\else % if luatex or xelatex
  \ifxetex
    \usepackage{mathspec}
  \else
    \usepackage{fontspec}
  \fi
  \defaultfontfeatures{Ligatures=TeX,Scale=MatchLowercase}
\fi
% use upquote if available, for straight quotes in verbatim environments
\IfFileExists{upquote.sty}{\usepackage{upquote}}{}
% use microtype if available
\IfFileExists{microtype.sty}{%
\usepackage{microtype}
\UseMicrotypeSet[protrusion]{basicmath} % disable protrusion for tt fonts
}{}
\usepackage[margin=1in]{geometry}
\usepackage{hyperref}
\hypersetup{unicode=true,
            pdftitle={Pancreatobiliary Pathology Society Journal Watch},
            pdfborder={0 0 0},
            breaklinks=true}
\urlstyle{same}  % don't use monospace font for urls
\usepackage{graphicx,grffile}
\makeatletter
\def\maxwidth{\ifdim\Gin@nat@width>\linewidth\linewidth\else\Gin@nat@width\fi}
\def\maxheight{\ifdim\Gin@nat@height>\textheight\textheight\else\Gin@nat@height\fi}
\makeatother
% Scale images if necessary, so that they will not overflow the page
% margins by default, and it is still possible to overwrite the defaults
% using explicit options in \includegraphics[width, height, ...]{}
\setkeys{Gin}{width=\maxwidth,height=\maxheight,keepaspectratio}
\IfFileExists{parskip.sty}{%
\usepackage{parskip}
}{% else
\setlength{\parindent}{0pt}
\setlength{\parskip}{6pt plus 2pt minus 1pt}
}
\setlength{\emergencystretch}{3em}  % prevent overfull lines
\providecommand{\tightlist}{%
  \setlength{\itemsep}{0pt}\setlength{\parskip}{0pt}}
\setcounter{secnumdepth}{0}
% Redefines (sub)paragraphs to behave more like sections
\ifx\paragraph\undefined\else
\let\oldparagraph\paragraph
\renewcommand{\paragraph}[1]{\oldparagraph{#1}\mbox{}}
\fi
\ifx\subparagraph\undefined\else
\let\oldsubparagraph\subparagraph
\renewcommand{\subparagraph}[1]{\oldsubparagraph{#1}\mbox{}}
\fi

%%% Use protect on footnotes to avoid problems with footnotes in titles
\let\rmarkdownfootnote\footnote%
\def\footnote{\protect\rmarkdownfootnote}

%%% Change title format to be more compact
\usepackage{titling}

% Create subtitle command for use in maketitle
\providecommand{\subtitle}[1]{
  \posttitle{
    \begin{center}\large#1\end{center}
    }
}

\setlength{\droptitle}{-2em}

  \title{Pancreatobiliary Pathology Society Journal Watch}
    \pretitle{\vspace{\droptitle}\centering\huge}
  \posttitle{\par}
  \subtitle{June July 2019}
  \author{}
    \preauthor{}\postauthor{}
      \predate{\centering\large\emph}
  \postdate{\par}
    \date{Last Update on 2019-08-26}


\begin{document}
\maketitle

{
\setcounter{tocdepth}{5}
\tableofcontents
}
\begin{center}\rule{0.5\linewidth}{\linethickness}\end{center}

\hypertarget{pbpath-journal-watch-articles}{%
\section{PBPath Journal Watch
Articles}\label{pbpath-journal-watch-articles}}

\begin{center}\rule{0.5\linewidth}{\linethickness}\end{center}

\textbf{Wellcome to the PBPath Journal Watch!}

This bi-monthly journal watch features exciting recently published
pancreas and biliary pathology articles that will provide up to date
medical knowledge in our field. These articles will be showcased in
several convenient categories, including surgical pathology,
cytopathology, and molecular pathology among others. The articles in
each category are in no particular order. See the list of journals we
search regularly \href{http://pbpath.org/pbpath-journal-watch/}{here}.
Previous months' issues may be found in our
\emph{\href{http://pbpath.org/journal-watch-archive/}{archive}} and you
may see \href{http://pbpath.org/journal-watch-upcoming-issue/}{the
drafts of upcoming issue here}.

We encourage members to actively participate by recommending new
articles and providing feedback using the forms provided below.

\textbf{We hope that you will enjoy the new PBPath Journal Watch!}

\textbf{Feedback}

Please send your feedbacks using the forms below the page.

\begin{center}\rule{0.5\linewidth}{\linethickness}\end{center}

Back to top

\begin{center}\rule{0.5\linewidth}{\linethickness}\end{center}

\pagebreak

\hypertarget{surgical-pathology}{%
\subsection{Surgical Pathology}\label{surgical-pathology}}

\begin{center}\rule{0.5\linewidth}{\linethickness}\end{center}

\hypertarget{pancreas}{%
\subsubsection{Pancreas}\label{pancreas}}

\begin{center}\rule{0.5\linewidth}{\linethickness}\end{center}

\hypertarget{morphology-diagnostics-ihc}{%
\paragraph{Morphology, Diagnostics,
IHC}\label{morphology-diagnostics-ihc}}

Morphology, Diagnostics, IHC

\begin{center}\rule{0.5\linewidth}{\linethickness}\end{center}

 - \textbf{Pancreatic schwannoma, an extremely rare and challenging
entity: Report of two cases and review of literature}

\emph{Pancreatology : official journal of the International Association
of Pancreatology (IAP) \ldots{} {[}et al.{]} 2019 Jul;19(5):729-737}

PubMed: \url{https://www.ncbi.nlm.nih.gov/pubmed/?term=31153779}

Pancreatic schwannoma is a rare benign tumor, for which the preoperative
and intraoperative definitive diagnosis is quite challenging. We present
the clinical, radiological and pathologic features of two primary
pancreatic schwannomas identified in our pathology database over a
period of 30\,years\,at our tertiary care hospital. To better understand
the clinico-pathological and radiological features of this entity, we
provide a comprehensive review of 73 cases described in the English
literature, along with our two cases. This review will especially focus
on preoperative and intraoperative diagnosis to assess their accuracy
for pancreatic schwannoma. The three most common preoperative diagnoses
based on imaging for pancreatic schwannomas were cystic neoplasm (56\%),
pancreatic neuroendocrine tumor (29\%) and mucinous cystic neoplasm
(26\%). Imaging could not definitely diagnose pancreatic schwannoma in
any of the reported cases. To obtain a definite diagnosis before
surgery, 25 cases underwent imaging-guided fine-needle aspiration
(FNA)/biopsy, of which 60\% were correctly reported as benign with
definite diagnosis of pancreatic schwannoma in 48\%. A higher diagnostic
accuracy was observed in biopsies (71\%) than FNA (37\%). In addition,
an intraoperative frozen section was carried out in 15 cases, and 47\%
were correctly diagnosed. Despite relatively low accuracy, preoperative
histological assessment can be helpful in surgical managment. A core
tissue specimen is recommended to improve the diagnostic accuracy in
this setting.

{}

{}

\begin{center}\rule{0.5\linewidth}{\linethickness}\end{center}

 - \textbf{Lipase hypersecretion syndrome: A distinct form of
paraneoplastic syndrome specific to pancreatic acinar carcinomas}

\emph{Seminars in diagnostic pathology 2019 Jul;36(4):240-245}

PubMed: \url{https://www.ncbi.nlm.nih.gov/pubmed/?term=31300257}

Lipase hypersecretion syndrome (LHS) is a paraneoplastic syndrome seen
exclusively as a result of pancreatic acinar cell carcinoma (ACC). In
LHS, acinar enzymes (lipase, trypsin and chymotrypsin) which are
normally secreted to the duodenum for digestive purposes, are instead
released to the blood by the carcinoma cells. In a way, it is
``endocrine-ization'' of an ``exocrine'' function. These circulating
enzymes, especially lipase, exerts its digestive action on other
tissues, especially on the subcutaneous tissues in the pressure poins of
legs, creating a picture often mistaken as erythema nodosum or rheumatic
nodules. The bone and joints may also be effected, which mostly appears
to be secondary to the complications and super-infection of the skin
lesions. Eosinophilia also often accompanies this syndrome. The accurate
diagnosis of LHS requires the identification of the pancreatic primary
as well as its correct classification as acinar because a variety of
pancreatic tumors can be associated with skin lesions, ranging from rare
metastasis of adenocarcinoma to the necrolytic migratory erythema caused
by glucagon-producing neuroendocrine tumors. Towards this differential,
the diagnostic characteristics of acinar cell carcinomas that have been
better elucidated in the past decade often need to be employed in
increasingly smaller specimens and the liver, especially since most LHS
cases also have liver metastasis (presumably due to the by-pass of the
``first-pass'' liver metabolism phenomenon). ACC (and LHS) occur in
patients in their 60's. The pancreatic mass is often large, round,
demarcated and closely resemble neuroendocrine and solid-pseudopapillary
neoplasms but are more atypical/proliferative, and commonly show single
prominent nucleoli and a distinctive chromophilia. Immunostaining with
trypsin/chymotrypsin, negativity of beta-catenin help in the
differential; as a caveat, neuroendocrine differentiation is common in
ACCs. In conclusion, LHS is a rare type of paraneoplastic syndrome
specific to ACC. The accurate diagnosis requires attention to their
subtle diagnostic characteristics.

{}

{}

\begin{center}\rule{0.5\linewidth}{\linethickness}\end{center}

Back to top

\begin{center}\rule{0.5\linewidth}{\linethickness}\end{center}

\pagebreak

\hypertarget{staging}{%
\paragraph{Staging}\label{staging}}

Pancreas TNM staging, Margins, Survival

\begin{center}\rule{0.5\linewidth}{\linethickness}\end{center}

 - \textbf{Pancreatic ductal adenocarcinoma and paraaortic lymph nodes
metastases: The accuracy of intraoperative frozen section}

\emph{Pancreatology : official journal of the International Association
of Pancreatology (IAP) \ldots{} {[}et al.{]} 2019 Jul;19(5):710-715}

PubMed: \url{https://www.ncbi.nlm.nih.gov/pubmed/?term=31174978}

BACKGROUND: Pancreatoduodenectomy for pancreatic ductal adenocarcinoma
(PDAC) with paraaortic lymph nodes metastases (PALN~+) is associated
with poor survival. Still, there are no current guidelines advocating
systematic detection of PALN+. METHODS: All consecutive patients who
underwent surgical exploration/resection with concurrent paraaortic
(group 16) lymphadenectomy for PDAC between 2009 and 2016 were
considered for inclusion. Resection was systematically aborted in case
of intraoperative PALN~+~detection. Diagnostic performance of
preoperative imaging upon blind review and intraoperative PALN
dissection with frozen section (FS) for PALN detection were evaluated.
Additionally, the prognostic significance of PALN~+~on overall survival
(OS) was analyzed. RESULTS: Over the study period, among 129 patients
undergoing surgery for PDAC, 113 had intraoperative PALN dissection with
FS analysis. Median number of resected PALN was 3 (range, 1-15).
Overall, PALN+ was found in 19 patients (16.8\%). Upon blind review,
preoperative imaging performed poorly for PALN~+~detection with a low
agreement between imaging and final pathology (Kappa-Cohen
index\textless{}0.2). In contrast, PALN FS showed high detection
performances and strong agreement with final pathology (Kappa-Cohen
index~=~0.783, 95\%CI 0.779-0.867, p~\textless{}~0.001). Regarding
survival outcomes, there was no difference between patients with PALN+
and patients not resected in the setting of liver metastases or locally
unresectable disease found at exploration (p~=~0.708). CONCLUSIONS:
Before PD for PDAC, intraoperative PALN dissection and FS analysis
yields accurate PALN assessment and allows appropriate patient
selection. This should be routinely performed and aborting resection
should be strongly considered in case of PALN+.

{}

{}

\begin{center}\rule{0.5\linewidth}{\linethickness}\end{center}

Back to top

\begin{center}\rule{0.5\linewidth}{\linethickness}\end{center}

\pagebreak

\hypertarget{preneoplastic-and-preinvasive-lesions}{%
\paragraph{Preneoplastic and Preinvasive
Lesions}\label{preneoplastic-and-preinvasive-lesions}}

Preneoplastic and Preinvasive Lesions, PanIN, IPMN, MCN, ICPN

\begin{center}\rule{0.5\linewidth}{\linethickness}\end{center}

 - \textbf{Pancreatic Cysts and Intraductal Papillary Mucinous Neoplasm
in Autosomal Dominant Polycystic Kidney Disease}

\emph{Pancreas 2020 05;48(5):698-705}

PubMed: \url{https://www.ncbi.nlm.nih.gov/pubmed/?term=31091218}

OBJECTIVES: Pancreatic lesions in autosomal dominant polycystic kidney
disease (ADPKD) are primarily cysts. They are increasingly recognized,
with isolated reports of intraductal papillary mucinous neoplasia
(IPMN). METHODS: Retrospective study to determine prevalence, number,
size, and location of pancreatic abnormalities using abdominal magnetic
resonance imaging (MRI) of genotyped ADPKD patients (seen February 1998
to October 2013) and compared with age- and sex-matched non-ADPKD
controls. We evaluated presentation, investigation, and management of
all IPMNs among individuals with ADPKD (January 1997 to December 2016).
RESULTS: Abdominal MRIs were examined for 271 genotyped ADPKD patients.
A pancreatic cyst lesion (PCL) was detected in 52 patients (19\%; 95\%
confidence interval, 15\%-23\%). Thirty-seven (71\%) had a solitary PCL;
15 (28\%) had multiple. Pancreatic cyst lesion prevalence did not differ
by genotype. Intraductal papillary mucinous neoplasia was detected in
1\% of ADPKD cases. Among 12 IPMN patients (7 branch duct; 5 main duct
or mixed type) monitored for about 140 months, 2 with main duct IPMNs
required Whipple resection, and 1 patient died of complications from
small-bowel obstruction after declining surgical intervention.
CONCLUSIONS: With MRI, PCLs were detected in 19\% and IPMNs in 1\% of
271 ADPKD patients with proven mutations, without difference across
genotypes. Pancreatic cyst lesions were asymptomatic and remained stable
in size.

{}

{}

\begin{center}\rule{0.5\linewidth}{\linethickness}\end{center}

Back to top

\begin{center}\rule{0.5\linewidth}{\linethickness}\end{center}

\pagebreak

\hypertarget{microenvironment}{%
\paragraph{Microenvironment}\label{microenvironment}}

Tumor Stroma Interactions, Microenvironment, Inflammatory Response

\begin{center}\rule{0.5\linewidth}{\linethickness}\end{center}

 - \textbf{Stromal hyaluronan accumulation is associated with low tumor
grade and nodal metastases in pancreatic ductal adenocarcinoma}

\emph{Human pathology 2019 Aug;90():37-44}

PubMed: \url{https://www.ncbi.nlm.nih.gov/pubmed/?term=31121193}

Pancreatic ductal adenocarcinoma is an aggressive malignancy
characterized by abundant desmoplastic stroma. Hyaluronan is a prominent
stromal component of pancreatic ductal adenocarcinoma and is associated
with unique clinical-pathological profiles in other tumor types. The
current study aimed to delineate clinical and pathological features
associated with hyaluronan accumulation in pancreatic ductal
adenocarcinoma using a novel hyaluronan-binding assay currently being
used in a clinical trial targeting hyaluronan. Sixty-four
formalin-fixed, paraffin-embedded samples of pancreatic ductal
adenocarcinomas from 49 patients treated at a single tertiary care
hospital were stained. Fifty-two percent of tumors had high levels of
hyaluronan. High levels were associated with low tumor grade and lymph
node metastases, novel associations not previously seen in pancreatic
ductal adenocarcinoma. This study has elucidated a novel
clinical-pathological profile in pancreatic ductal adenocarcinomas using
a new assay, suggesting hyaluronan may act as a biomarker for a subset
of pancreatic tumors that could be targeted by hyaluronan-degrading
agents.

{}

{}

\begin{center}\rule{0.5\linewidth}{\linethickness}\end{center}

 - \textbf{Quasimesenchymal phenotype predicts systemic metastasis in
pancreatic ductal adenocarcinoma}

\emph{Modern pathology : an official journal of the United States and
Canadian Academy of Pathology, Inc 2019 Jun;32(6):844-854}

PubMed: \url{https://www.ncbi.nlm.nih.gov/pubmed/?term=30683911}

Metastasis following surgical resection is a leading cause of mortality
in pancreatic ductal adenocarcinoma. Epithelial-mesenchymal transition
is thought to play an important role in metastasis, although its
clinical relevance in metastasis remains uncertain. We evaluated a panel
of RNA in-situ hybridization probes for epithelial-mesenchymal
transition-related genes expressed in circulating tumor cells. We
assessed the predictive value of this panel for metastasis in pancreatic
ductal adenocarcinoma and, to determine if the phenotype is
generalizable between cancers, in colonic adenocarcinoma. One hundred
fifty-eight pancreatic ductal adenocarcinomas and 205 colonic
adenocarcinomas were classified as epithelial or quasimesenchymal
phenotype using dual colorimetric RNA-in-situ hybridization. SMAD4
expression on pancreatic ductal adenocarcinomas was assessed by
immunohistochemistry. Pancreatic ductal adenocarcinomas with
quasimesenchymal phenotype had a significantly shorter disease-specific
survival (P = 0.031) and metastasis-free survival (P = 0.0001) than
those with an epithelial phenotype. Pancreatic ductal adenocarcinomas
with SMAD4 loss also had lower disease-specific survival (P = 0.041) and
metastasis-free survival (P = 0.001) than those with intact SMAD4.
However, the quasimesenchymal phenotype proved a more robust predictor
of metastases-area under the curve for quasimesenchymal = 0.8;
SMAD4 = 0.6. The quasimesenchymal phenotype also predicted
metastasis-free survival (P = 0.004) in colonic adenocarcinoma.
Epithelial-mesenchymal transition defined two phenotypes with distinct
metastatic capabilities-epithelial phenotype tumors with predominantly
organ-confined disease and quasimesenchymal phenotype with high risk of
metastatic disease in two epithelial malignancies. Collectively, this
work validates the relevance of epithelial-mesenchymal transition in
human gastrointestinal tumors.

{}

{}

\begin{center}\rule{0.5\linewidth}{\linethickness}\end{center}

 - \textbf{The Importance of a Conjoint Analysis of Tumor-Associated
Macrophages and Immune Checkpoints in Pancreatic Cancer}

\emph{Pancreas 2019 Aug;48(7):904-912}

PubMed: \url{https://www.ncbi.nlm.nih.gov/pubmed/?term=31268976}

OBJECTIVES: Tumor-associated macrophages are dominant players in
establishing the inmmunosuppressive microenvironment in pancreatic
ductal adenocarcinoma (PDAC). Immune checkpoint inhibitor monotherapy
has achieved limited clinical effectiveness. To date, the interaction of
macrophages and checkpoint regulators and their correlation with
clinicopathologic characteristics in PDAC have been largely unavailable.
METHODS: Macrophages and immune checkpoint expression were assessed by
immunohistochemistry from 80 PDAC samples. Clinicopathologic features
and the prognostic value of each marker were evaluated. In vitro changes
in the expression of immune markers in cocultured macrophages and PDAC
cells were detected by Western blot and immunosorbance assays. RESULTS:
The macrophages marker CD163 and the checkpoint marker programmed
death-ligand 1 (PD-L1) remained as the independent prognostic factors
for overall survival (hazard ratio, 2.543; P = 0.017 and hazard ratio,
2.389; P = 0.021). Furthermore, integrated analysis of CD163 and PD-L1
served as more optimal indicators of survival (P = 0.000). In vitro
coculture of macrophages and PDAC cells significantly increased the
expression of CD163 and PD-L1, compared with monocultured counterpart (P
\textless{} 0.05). CONCLUSIONS: Combined analysis of CD163 and PD-L1 was
enhanced indicators of survival in PDAC patients. The interaction of
macrophages and immune checkpoints implied the value of the combination
therapy.

{}

{}

\begin{center}\rule{0.5\linewidth}{\linethickness}\end{center}

Back to top

\begin{center}\rule{0.5\linewidth}{\linethickness}\end{center}

\pagebreak

\hypertarget{spn}{%
\paragraph{SPN}\label{spn}}

Solid Pseudopapillary Neoplasm

\begin{center}\rule{0.5\linewidth}{\linethickness}\end{center}

 - \textbf{Solid-pseudopapillary Neoplasms of the Pancreas is still an
Enigma: a Clinicopathological Review}

\emph{Pathology oncology research : POR 2019 Jun;():}

PubMed: \url{https://www.ncbi.nlm.nih.gov/pubmed/?term=31209654}

The solid-pseudopapillary neoplasm of the pancreas is a rare but
enigmatic entity occurring mainly in young women. Since the first
description by V. Frantz in 1959 the terminology of this tumor has
continuously changed but it has remained simply descriptive, because the
exact histogenesis is still obscure. Although in majority of cases the
survival is excellent, nevertheless, the expected prognosis is not
exactly predictable. In this review the authors aim to summarize its
clinico-pathological features, the expected biological behavior, the
molecular alterations, the immune phenotype and discuss the putative
histogenesis. From diagnostic point of view, the salient histological
characteristic findings are analyzed that would help to differentiate it
from other, look-alike pancreatic tumors, and suggestions are made about
the desirable content of the histological report.

{}

{}

\begin{center}\rule{0.5\linewidth}{\linethickness}\end{center}

Back to top

\begin{center}\rule{0.5\linewidth}{\linethickness}\end{center}

\pagebreak

\hypertarget{pancreatitis-other-diseases}{%
\paragraph{Pancreatitis \& Other
Diseases}\label{pancreatitis-other-diseases}}

\begin{center}\rule{0.5\linewidth}{\linethickness}\end{center}

 - \textbf{Autoimmune Pancreatitis Type 2: Diagnostic Utility of PD-L1
Immunohistochemistry}

\emph{The American journal of surgical pathology 2019 Jul;43(7):898-906}

PubMed: \url{https://www.ncbi.nlm.nih.gov/pubmed/?term=31135488}

BACKGROUND: Autoimmune pancreatitis (AIP) encompasses a heterogenous
disease group that includes IgG4-related type 1 AIP and non-IgG4-related
type 2 AIP. Clinically and on imaging, type 2 AIP mimics type 1 AIP,
other forms of chronic pancreatitis and pancreatic ductal adenocarcinoma
(PDAC); therefore, discriminatory markers may aid proper diagnosis.
Herein, we examine the expression of PD-L1 and indoleamine
2,3-dioxygenase (IDO1) as a diagnostic tool to distinguish type 2 AIP
from other forms of pancreatitis and PDAC. DESIGN: We evaluated 35
pancreatectomy specimens diagnosed with type 2 AIP and potential mimics
of this disease including type 1 AIP (n=14), chronic pancreatitis-not
otherwise specified (n=10), groove pancreatitis (n=14), and PDAC
(n=278). We scored inflammatory infiltrates, fibrosis and atrophy and
performed immunohistochemical staining for PD-L1 and IDO1. We validated
our findings on a series of endoscopic ultrasound-guided biopsies from
patients with suspected type 2 AIP and inflammatory and neoplastic
mimics of this disease (n=37). RESULTS: The mean age of patients with
type 2 AIP was 50 years with a F:M ratio of 1.2:1. Patients with type 2
AIP showed pancreatic ductal staining for PD-L1 and IDO1 in 69\% (24/35)
and 60\% (15/25) of cases, respectively. PD-L1 reactivity was noted in
3\% of patients with other forms of chronic pancreatitis and 3\% of
PDACs; notably, peritumoral ducts and acini were negative. Eight of 9
endoscopic ultrasound-guided biopsies with pancreatic ductal epithelium
from patients with type 2 AIP were positive for PD-L1, while the
inflammatory and neoplastic mimics were negative. Collectively, the
sensitivity and specificity of PD-L1 as a marker of type 2 AIP was 70\%
and 99\%, respectively. CONCLUSIONS: Ductal PD-L1 reactivity has the
potential to distinguish type 2 AIP from other forms of pancreatitis and
PDAC.

{}

{}

\begin{center}\rule{0.5\linewidth}{\linethickness}\end{center}

 - \textbf{Pancreatobiliary Versus Head and Neck Manifestations in
Immunoglobulin G4-related Disease: Distinct Subsets of the Same
Disease?}

\emph{Pancreas 2019 Jul;48(6):799-804}

PubMed: \url{https://www.ncbi.nlm.nih.gov/pubmed/?term=31210660}

OBJECTIVES: We compared the clinical profiles and organ manifestations
of the commonly encountered immunoglobulin G4-related diseases
(IgG4-RDs) on either side of the diaphragm: head and neck (HN) versus
pancreatobiliary (PB) in IgG4-RD. METHODS: From the Mayo Clinic,
Rochester, database, we identified 53 HN and 88 PB IgG4-RD based on the
first affected organ manifestation. RESULTS: Compared with HN IgG4-RD,
subjects with PB IgG4-RD were likely to be older (median, 64.8 vs 50.2
years; P \textless{} 0.0001), male (83\% vs 60.4\%; P = 0.003), and with
a shorter duration of follow-up (24.4 vs 48.7 months; P \textless{}
0.0001). In HN versus PB-IgG4-RD orbital, lacrimal gland, submandibular,
parotid gland, asthma, and sinusitis manifestations were more common
(77\% vs 4.5\%, 21\% vs 0\%, 32\% vs 8\%, 13\% vs 0\%, 36\% vs 9\%, and
51\% vs 6.8\%; P \textless{} 0.0001, respectively), whereas lung
manifestations were similar (13.2\% vs 5.6\%; P = 0.12). In contrast, in
PB versus HN IgG4-RD, pancreas and biliary were more frequent (98.8\% vs
15\%, 56.8\% vs 3.7\%; P \textless{} 0.0001), whereas renal lesions were
similar (12.5\% vs 7.5\%; P = 0.36). CONCLUSION: Pancreatobiliary and HN
IgG4-RD have distinct clinical profiles. Proximity matters in other
organ involvement in IgG4-RD, and organs involved tend to cluster close
to each.

{}

{}

\begin{center}\rule{0.5\linewidth}{\linethickness}\end{center}

Back to top

\begin{center}\rule{0.5\linewidth}{\linethickness}\end{center}

\pagebreak

\hypertarget{bile-ducts}{%
\subsubsection{Bile Ducts}\label{bile-ducts}}

\begin{center}\rule{0.5\linewidth}{\linethickness}\end{center}

\hypertarget{morphology-diagnostics-ihc-1}{%
\paragraph{Morphology, Diagnostics,
IHC}\label{morphology-diagnostics-ihc-1}}

Morphology, Diagnostics, IHC

\begin{center}\rule{0.5\linewidth}{\linethickness}\end{center}

 - \textbf{The Pathologic and Genetic Characteristics of the Intestinal
Subtype of Intraductal Papillary Neoplasms of the Bile Duct}

\emph{The American journal of surgical pathology 2019
Sep;43(9):1212-1220}

PubMed: \url{https://www.ncbi.nlm.nih.gov/pubmed/?term=31166202}

The present study aimed to identify the pathologic and genetic
characteristics of intestinal subtype of intraductal papillary neoplasm
of the bile duct (iIPNB) showing columnar cells with pseudostratified,
cigar-shaped nuclei, and basophilic or amphophilic cytoplasm with the
diffuse immunohistochemical expression of CK20 and/or CDX2. A total of
34 cases of iIPNB were pathologically examined according to their
anatomic location (the bile duct) and were then compared with the
intestinal subtype of intraductal papillary mucinous neoplasm (iIPMN) of
the pancreas (n=22). Mutations of 26 somatic genes were examined in
formalin-fixed paraffin-embedded tissue specimens from 21 cases of iIPNB
using the TruSight Tumor 26 gene panel and next-generation sequencing.
iIPNB cases were divided into intrahepatic (n=6) and extrahepatic (n=28)
categories. Intrahepatic IPNBs showed a less-complicated
villous-papillary pattern, while extrahepatic IPNBs showed a papillary
pattern with tubular and/or villous components and predominant
high-grade dysplasia with complicated architectures. MUC5AC was
frequently and extensively expressed in intrahepatic iIPNBs and iIPMNs
but not in extrahepatic iIPNBs. CD10 was frequently expressed in
extrahepatic IPNBs but not in intrahepatic iIPNBs or iIPMN. Genetic
mutations of TP53 and PIK3CA, which were infrequent or absent in iIPMNs,
were frequently detected in extrahepatic iIPNBs, while KRAS and GNAS,
which were commonly observed in iIPMNs, were frequently detected in
intrahepatic iIPNBs. Intrahepatic iIPNBs showed villous-papillary growth
with features reminiscent of iIPMNs, while extrahepatic iIPNBs showed
papillary growth with tubular and/or villous components, complicated
histology and variable differences from iIPMNs, suggesting differences
in the tumorigenesis of iIPNBs along the biliary tree.

{}

{}

\begin{center}\rule{0.5\linewidth}{\linethickness}\end{center}

 - \textbf{Programmed cell death ligand-1 (PD-L1) expression in
extrahepatic biliary tract cancers: a comparative study using 22C3,
SP263 and E1L3N anti-PD-L1 antibodies}

\emph{Histopathology 2019 May;():}

PubMed: \url{https://www.ncbi.nlm.nih.gov/pubmed/?term=31081949}

AIMS: Pembrolizumab has shown promising results for patients with
programmed cell death ligand-1 (PD-L1)-positive advanced biliary tract
cancer in an ongoing clinical trial. However, data on PD-L1 expression
in bile duct cancers is limited, and the frequency of PD-L1 positivity
varies, which may be partly due to the assay used. The aim of this study
was to evaluate PD-L1 expression status in bile duct cancers by using
22C3, SP263 and E1L3N antibodies. METHODS AND RESULTS: We evaluated
PD-L1 expression in tissue microarrays of 183 extrahepatic bile duct
cancers, including 89 perihilar and 94 distal bile duct cancers, by
using 22C3, SP263 and E1L3N. When the 22C3 assay was used, tumoral PD-L1
was shown to be expressed in 16.9\% of cases at a 1\% threshold. When
the SP263 and E1L3N assays were used, tumoral PD-L1 was shown to be
expressed in 26\% and 7.1\% of cases, respectively. When whole tissue
sections were examined, 59.6\% of PD-L1-positive cases showed a low
percentage (\textless{}10\%) of positive tumour cells. Tumoral PD-L1
positivity was associated with poor histological differentiation
(P~=~0.017) and the biliary epithelial phenotype (P~=~0.041). High
tumoral PD-L1 expression (≥10\%) was associated with worse overall
survival (OS) and disease-free survival (DFS) (OS, P~=~0.012; DFS,
P~=~0.042). CONCLUSIONS: PD-L1 was expressed in a small subset of
patients with bile duct cancer, and the percentage of positive tumour
cells was low in PD-L1-positive cases. The SP263 assay showed the
highest PD-L1 positivity in both tumour cells and immune cells, followed
by the 22C3 and E1L3N assays. High PD-L1 expression was associated with
a poor prognosis in extrahepatic bile duct cancer patients.

{}

{}

\begin{center}\rule{0.5\linewidth}{\linethickness}\end{center}

 - \textbf{Smoking, Alcohol, and Biliary Tract Cancer Risk: A Pooling
Project of 26 Prospective Studies}

\emph{Journal of the National Cancer Institute 2019 May;():}

PubMed: \url{https://www.ncbi.nlm.nih.gov/pubmed/?term=31127946}

BACKGROUND: Tobacco and alcohol are well-established risk factors for
numerous cancers, yet their relationship to biliary tract cancers
remains unclear. METHODS: We pooled data from 26 prospective studies to
evaluate associations of cigarette smoking and alcohol consumption with
biliary tract cancer risk. Study-specific hazard ratios (HRs) and 95\%
confidence intervals (CIs) for associations with smoking and alcohol
consumption were calculated. Random effects meta-analysis produced
summary estimates. All statistical tests were two-sided. RESULTS: Over a
period of 38,369,156 person-years of follow-up, 1,391 gallbladder, 758
intrahepatic bile duct, 1,208 extrahepatic bile duct, and 623 ampulla of
Vater cancer cases were identified. Ever, former, and current smoking
were associated with increased extrahepatic bile duct and ampulla of
Vater cancers risk (e.g., current versus never smokers hazard ratio
{[}HR{]} = 1.69, 95\% confidence interval {[}CI{]} = 1.34 to 2.13 and
2.22, 95\%CI = 1.69 to 2.92, respectively), with dose-response effects
for smoking pack-years, duration, and intensity (all
P-trend\textless{}0.01). Current smoking and smoking intensity were also
associated with intrahepatic bile duct cancer (e.g., \textgreater{}40
cigarettes/day versus never smokers HR = 2.15, 95\%CI: 1.15 to 4.00;
P-trend=0.001). No convincing association was observed between smoking
and gallbladder cancer. Alcohol consumption was only associated with
intrahepatic bile duct cancer, with increased risk for individuals
consuming ≥5 versus 0 drinks/day (HR = 2.35, 95\%CI = 1.46 to 3.78;
P-trend=0.04). There was evidence of statistical heterogeneity between
several cancer sites, particularly between gallbladder cancer and the
other biliary tract cancers. CONCLUSIONS: Smoking appears to increase
the risk of developing all biliary tract cancers except gallbladder
cancer. Alcohol may increase the risk of intrahepatic bile duct cancer.
Findings highlight etiologic heterogeneity across the biliary tract.

{}

{}

\begin{center}\rule{0.5\linewidth}{\linethickness}\end{center}

 - \textbf{Distinct histomorphological features are associated with IDH1
mutation in intrahepatic Cholangiocarcinoma}

\emph{Human pathology 2019 May;():}

PubMed: \url{https://www.ncbi.nlm.nih.gov/pubmed/?term=31121195}

Intrahepatic cholangiocarcinoma has known histological heterogeneity.
Mutations in IDH1 (mIDH1) define a molecular subclass of intrahepatic
cholangiocarcinoma and IDH-targeted therapies are in development.
Characterizing mIDH1 ICC histomorphology is of clinical interest for
efficient identification. Resected ICCs with targeted next generation
sequencing by MSK-IMPACT were selected. Clinical data were obtained. By
slide review, blinded to IDH status, data were collected for histology
type, mucin production, necrosis, fibrosis, cytoplasm cell shape (low
cuboidal, plump cuboidal/polygonal, and columnar), and architectural
pattern (anastomosing, tubular, compact tubular, and solid). A tumor was
considered architecturally heterogeneous if no dominant pattern
represented ≥75\% of the tumor. Parameters were compared between
mIDH1and IDH wild type controls. In the examined cohort (113 ICC: 29
mIDH1 and 84 IDH wild type), all IDH1 mutant tumors were of small duct
type histology, thus analysis was limited to 101 small duct type tumors.
mIDH1cases were more likely to have plump cuboidal/ polygonal shape
(P=.014) and geographic-type fibrosis (P=.005) while IDH1 wild type were
more likely to have low cuboidal shape (P=.005). Both groups were
predominantly architecturally heterogeneous with no significant
difference in the distribution of architectural patterns. Plump
cuboidal/polygonal cell shape and a geographic-type pattern of
intra-tumoral fibrosis are more often seen in mIDH1compared to IDH wild
type tumors, however IDH1 mutation is not associated with a distinct
histoarchitectural pattern.

{}

{}

\begin{center}\rule{0.5\linewidth}{\linethickness}\end{center}

Back to top

\begin{center}\rule{0.5\linewidth}{\linethickness}\end{center}

\pagebreak

\hypertarget{gallbladder}{%
\subsubsection{Gallbladder}\label{gallbladder}}

\begin{center}\rule{0.5\linewidth}{\linethickness}\end{center}

\hypertarget{morphology-diagnostics-ihc-2}{%
\paragraph{Morphology, Diagnostics,
IHC}\label{morphology-diagnostics-ihc-2}}

Morphology, Diagnostics, IHC

\begin{center}\rule{0.5\linewidth}{\linethickness}\end{center}

 - \textbf{Intracholecystic papillary-tubular neoplasms of the
gallbladder - A clinicopathological study of 36 cases}

\emph{Annals of diagnostic pathology 2019 Jun;40():88-93}

PubMed: \url{https://www.ncbi.nlm.nih.gov/pubmed/?term=31077876}

Intracholecystic papillary-tubular neoplasms (ICPNs) account for
\textless{}0.5\% of all cholecystectomies. There is a lack of
significant published data from the Indian subcontinent on ICPN to the
best of our knowledge. The objective of the current study was to
describe the clinicopathological features of ICPN of gallbladder from
the departmental archives during a 5.5-year period. We also aimed to
classify them into various histological subtypes and to correlate the
clinicopathological parameters of ICPN with invasive adenocarcinoma.
This study included 36 cases diagnosed over a period of 5.5\,years
(2013-2018). Clinical, radiological and histopathological data were
analyzed in detail. The incidence of ICPN was 0.8\%. The mean age of
patients was 45.7\,years with a female to male ratio of 1.3:1. Biliary
phenotype was associated with invasion (p\,≤0.001). Papillary pattern
was present in 15 cases (41.6\%) and was associated with invasion
(p\,≤0.001). High grade dysplasia was seen in 34 cases (94.4\%), of
which invasion was seen in 18 cases (50\%). One case in our study also
had synchronous common bile duct carcinoma. Majority (92\%) of the
patients were alive and well at the end of available follow-up (mean of
7\,months and 25\,days). ICPNs are mass forming neoplasms of the
gallbladder with a slight female predominance. Biliary phenotype has an
aggressive course, often associated with an invasive adenocarcinoma
component. Papillary configuration of the lesion is significantly
associated with an invasive component. Diligent follow-up of these
lesions is warranted as they can be associated with other malignancies
of the biliary system.

{}

{}

\begin{center}\rule{0.5\linewidth}{\linethickness}\end{center}

 - \textbf{Sarcomatoid carcinomas of the gallbladder: clinicopathologic
characteristics}

\emph{Virchows Archiv : an international journal of pathology 2019
Jul;475(1):59-66}

PubMed: \url{https://www.ncbi.nlm.nih.gov/pubmed/?term=31177317}

Sarcomatoid carcinomas recently came into the spotlight through genetic
profiling studies and also as a distinct model of epithelial-mesenchymal
transition. The literature on sarcomatoid carcinomas of gallbladder is
limited. In this study, 656 gallbladder carcinomas (GBC) were reviewed.
Eleven (1.7\%) with a sarcomatoid component were identified and analyzed
in comparison with ordinary GBC (O-GBC). Patients included 9 females and
2 males (F/M = 4.5 vs.~3.9) with a mean age-at-diagnosis of 71 (vs.~64).
The median tumor size was 4.6~cm (vs.~2.5; P = 0.01). Nine patients
(84\%) presented with advanced stage (pT3/4) tumor (vs.~48\%). An
adenocarcinoma component constituting 1-75\% of the tumor was present in
nine, and eight had surface dysplasia/CIS; either in situ or invasive
carcinoma was present in all cases. An intracholecystic
papillary-tubular neoplasm was identified in one. Seven showed
pleomorphic-sarcomatoid pattern, and four showed subtle/bland elongated
spindle cells. Three had an angiosarcomatoid pattern. Two had
heterologous elements. One showed few osteoclast-like giant cells, only
adjacent to osteoid. Immunohistochemically, vimentin, was positive in
six of six; P53 expression was \textgreater{} 60\% in six of six,
keratins in six of seven, and p63 in two of six. Actin, desmin, and S100
were negative. The median Ki67 index was 40\%. In the follow-up, one
died peri-operatively, eight died of disease within 3 to 8~months
(vs.~26~months median survival for O-GBC), and two were alive at 9 and
15~months. The behavior overall was worse than ordinary adenocarcinomas
in general but was not different when grade and stage were matched. In
summary, sarcomatoid component is identified in \textless{} 2\% of GBC.
Unlike sarcomatoid carcinomas in the remainder of pancreatobiliary
tract, these are seldom of the ``osteoclastic'' type and patients
present with large/advanced stage tumors. Limited data suggests that
these tumors are aggressive with rapid mortality unlike pancreatic
osteoclastic ones which often have indolent behavior.

{}

{}

\begin{center}\rule{0.5\linewidth}{\linethickness}\end{center}

 - \textbf{Epithelial Inclusions in Gallbladder May Mimic Parasite
Infection}

\emph{American journal of clinical pathology 2019 Aug;152(3):399-402}

PubMed: \url{https://www.ncbi.nlm.nih.gov/pubmed/?term=31189015}

{}

{}

\begin{center}\rule{0.5\linewidth}{\linethickness}\end{center}

Back to top

\begin{center}\rule{0.5\linewidth}{\linethickness}\end{center}

\pagebreak

\hypertarget{neuroendocrine}{%
\subsubsection{Neuroendocrine}\label{neuroendocrine}}

\begin{center}\rule{0.5\linewidth}{\linethickness}\end{center}

\hypertarget{pannet}{%
\paragraph{PanNET}\label{pannet}}

PanNET, Pancreatic Neuroendocrine Tumors and related neuroendocrine
neoplasms

\begin{center}\rule{0.5\linewidth}{\linethickness}\end{center}

 - \textbf{Insulinoma-associated protein 1 expression in primary and
metastatic neuroendocrine neoplasms of the gastrointestinal and
pancreaticobiliary tracts}

\emph{Histopathology 2019 May;():}

PubMed: \url{https://www.ncbi.nlm.nih.gov/pubmed/?term=31077609}

AIMS: Insulinoma-associated protein 1 (INSM1) is a transcription factor
that is expressed in developing and mature neuroendocrine tissue. Recent
studies have shown that INSM1 is a sensitive marker for neuroendocrine
tumours. The aims of this study were to evaluate INSM1 expression in
primary gastroenteropancreatic neuroendocrine neoplasms (GEP-NENs) and
in their known metastases, in order to assess its sensitivity as
compared with chromogranin-A (CgA) and synaptophysin (SYN), and to
evaluate any change in expression between primary and metastatic
disease. METHODS AND RESULTS: We identified 30 patients with primary
GEP-NEN. Liver metastatic tissue was available for 26 patients; two
patients had two metachronous metastatic foci, yielding a total of 28
metastatic cases. An additional two and seven non-paired cases of
primary and metastatic grade 3 GEP-NEN, respectively, were included. To
assess specificity, we evaluated the expression of these markers in
other primary tumours (colorectal adenocarcinoma, acinar cell carcinoma,
solid pseudopapillary neoplasm, cholangiocarcinoma, and hepatocellular
carcinoma) and metastatic tumours in the liver (adrenal cortical, breast
and prostate carcinomas) that may present as differential diagnoses. In
our cohort, all of the primary GEP-NENs and 94\% of the metastatic
GEP-NENs expressed INSM1. INSM1 showed similar sensitivity to SYN and
higher sensitivity than CgA in both primary and metastatic neoplasms.
INSM1 has comparable specificity to CgA, and higher specificity than
SYN. CONCLUSIONS: The nuclear reactivity and the high sensitivity and
specificity of INSM1 make it a preferred neuroendocrine marker. In
conclusion, INSM1 can be used as a single first-line marker for primary
and metastatic GEP-NEN.

{}

{}

\begin{center}\rule{0.5\linewidth}{\linethickness}\end{center}

 - \textbf{Is the Real Prevalence of Pancreatic Neuroendocrine Tumors
Underestimated? A Retrospective Study on a Large Series of Pancreatic
Specimens}

\emph{Neuroendocrinology 2019 05;109(2):165-170}

PubMed: \url{https://www.ncbi.nlm.nih.gov/pubmed/?term=31117106}

BACKGROUND/AIMS: The annual incidence of pancreatic neuroendocrine
tumors (PanNET) has been estimated to be around 0.8/100,000 inhabitants.
The aim of this study was to determine the frequency of incidental
histological diagnosis of PanNET in pancreatic specimen evaluation for a
purpose other other than PanNET diagnosis. METHODS: One thousand
seventy-four histopathological examinations of pancreatic specimens
performed in 3 centers in Italy were retrospectively reviewed. All cases
with a main pathological diagnosis of PanNET were excluded. RESULTS: An
incidental associated diagnosis of PanNET was made in 41 specimens
(4\%). Among those 41 cases, 29 (71\%) had a largest diameter
\textless{}5 mm (microadenoma), whereas the other 12 (29\%) had a
maximum size ≥5 mm (median diameter of the whole series = 3 mm, range
1-15). The association with a main diagnosis of intraductal papillary
mucinous neoplasms (IPMN) was significantly higher for patients who had
an incidental PanNET (p = 0.048). There was no association between
incidental diagnosis of PanNET and age, gender, BMI, smoking habit,
diabetes, and type of operation. CONCLUSIONS: The frequency of
incidental histological diagnosis of PanNET is considerably high,
suggesting that their real prevalence is probably underestimated. The
present study suggests a possible correlation between the incidental
occurrence of PanNET and IPMN.

{}

{}

\begin{center}\rule{0.5\linewidth}{\linethickness}\end{center}

 - \textbf{Diagnostic and Management Challenges in Vasoactive Intestinal
Peptide Secreting Tumors: A Series of 15 Patients}

\emph{Pancreas 2019 Aug;48(7):934-942}

PubMed: \url{https://www.ncbi.nlm.nih.gov/pubmed/?term=31268974}

OBJECTIVES: Vasoactive intestinal peptide-secreting tumors (VIPomas) are
rare functioning neuroendocrine tumors often characterized by a
difficult-to-control secretory syndrome and high potential to develop
metastases. We hereby present the characteristics of 15 cases of VIPomas
and provide a recent literature review. METHODS: This was a
retrospective data analysis of 15 patients with VIPoma from 3 different
centers and literature research through PubMed database during the last
10 years. RESULTS: Fifteen patients with VIPomas (9 with hepatic
metastases at diagnosis) with watery diarrhea and raised VIP levels were
studied. Ten patients (67\%) had grade 2 tumors, 6 of 15 had localized
disease and underwent potentially curative surgery, whereas the
remaining 9 received multiple systemic therapies; 3 patients died during
follow-up. The median overall survival was 71 months (range, 41-154
months). Patients who were treated with curative surgery (n = 7) had
longer median overall survival compared with patients who were treated
with other therapeutic modalities (44 vs 33 months). CONCLUSIONS: The
management of VIPomas is challenging requiring the application of
multiple treatment modalities. Patients who underwent surgical treatment
with curative intent appear to have higher survival rate. Central
registration and larger prospective studies are required to evaluate the
effect of currently employed therapies in these patients.

{}

{}

\begin{center}\rule{0.5\linewidth}{\linethickness}\end{center}

 - \textbf{Ki-67 Index of 5\% is Better Than 2\% in Stratifying G1 and
G2 of the World Health Organization Grading System in Pancreatic
Neuroendocrine Tumors}

\emph{Pancreas 2019 Jul;48(6):795-798}

PubMed: \url{https://www.ncbi.nlm.nih.gov/pubmed/?term=31210659}

OBJECTIVE: The World Health Organization (WHO) grading system for the
stratification of G1 and G2 pancreatic neuroendocrine tumors (pNETs)
using an optimal Ki-67 index cutoff is still controversial. The present
study aimed at finding one optimal Ki-67 cutoff value that distinguishes
G1 and G2 tumors by analyzing the prognosis of patients with pNET in our
center. METHODS: Data from 84 patients with pNET undergoing surgical
resection in The First Affiliated Hospital of Sun Yat-sen University
between March 2003 and October 2015 were retrospectively analyzed.
RESULTS: The 5-year overall survival rate was 74.2\%. Univariate
analysis revealed that functional secretion, WHO grade, and TNM stage
were significantly associated with long-term survival (all P \textless{}
0.05). Multivariate analysis demonstrated that WHO grade (P = 0.023) and
TNM stage (P = 0.040) were independent prognostic factors. The receiver
operating characteristic curve showed that the Ki-67 index of 5\% had
the best predictive ability (76.7\%) for 5-year survival with a hazard
ratio of 44.7. The hazard ratio was only 8.14 when the Ki-67 index
cutoff was 2\%. CONCLUSIONS: TNM stage and WHO grade were independent
prognostic factors of pNETs. A Ki-67 index of 5\% is better than 2\% in
stratifying G1 and G2 pNET tumors.

{}

{}

\begin{center}\rule{0.5\linewidth}{\linethickness}\end{center}

Back to top

\begin{center}\rule{0.5\linewidth}{\linethickness}\end{center}

\pagebreak

\hypertarget{staging-1}{%
\paragraph{Staging}\label{staging-1}}

PanNET TNM staging, Margins, Survival

\begin{center}\rule{0.5\linewidth}{\linethickness}\end{center}

 - \textbf{Prognostic Validity of the American Joint Committee on Cancer
Eighth Edition TNM Staging System for Surgically Treated and
Well-Differentiated Pancreatic Neuroendocrine Tumors: A Comprehensive
Analysis of 254 Consecutive Patients From a Large Chinese Institution}

\emph{Pancreas 2019 5;48(5):613-621}

PubMed: \url{https://www.ncbi.nlm.nih.gov/pubmed/?term=31091206}

OBJECTIVES: We aimed to validate the novel American Joint Committee on
Cancer (AJCC) eighth edition staging manual for well-differentiated
(G1/G2) pancreatic neuroendocrine tumors (pNETs). METHODS: Data of
eligible patients were retrospectively collected, grouped, and analyzed
by applying the new AJCC system. RESULTS: According to the AJCC eighth
staging manual for pNETs, 93, 66, 53, and 42 patients had stage I, II,
III, and IV disease, respectively, with estimated 5-year overall
survival (OS) rates of 96.9\%, 92.8\%, 48.4\%, and 16.8\% (P \textless{}
0.005), respectively. A total of 57, 28, 20, and 17 patients with G1
pNETs and 36, 38, 33, and 25 ones with G2 tumors were defined by the new
AJCC system as having stage I, II, III, and IV disease, respectively.
The estimated 5-year OS for stage I, II, III and IV disease was 100.0\%,
97.1\%, 52.5\%, and 18.2\%, respectively, for G1 pNETs (P \textless{}
0.005) and 94.2\%, 90.3\%, 38.7\%, and 12.7\%, respectively, for G2
tumors (P \textless{} 0.005). The novel AJCC classification, tumor
grading, and radical resection were all prognostic predictors for OS in
patients with pNETs. CONCLUSIONS: The new AJCC eighth staging system for
well-differentiated pNETs was prognostic and might be adopted in
clinical practice.

{}

{}

\begin{center}\rule{0.5\linewidth}{\linethickness}\end{center}

\begin{itemize}
\tightlist
\item
  Significance of Lymph Node Metastasis in Resectable
  Well-differentiated Pancreatic Neuroendocrine Tumor.
\end{itemize}

\url{https://www.ncbi.nlm.nih.gov/pubmed/31268980}

\begin{center}\rule{0.5\linewidth}{\linethickness}\end{center}

Back to top

\begin{center}\rule{0.5\linewidth}{\linethickness}\end{center}

\pagebreak

\hypertarget{cytopathology}{%
\subsection{Cytopathology}\label{cytopathology}}

\begin{center}\rule{0.5\linewidth}{\linethickness}\end{center}

\hypertarget{pancreas-1}{%
\subsubsection{Pancreas}\label{pancreas-1}}

\begin{center}\rule{0.5\linewidth}{\linethickness}\end{center}

 - \textbf{Diagnostic Efficacy of Liquid-Based Cytology in Endoscopic
Ultrasound-Guided Fine Needle Aspiration for Pancreatic Mass Lesions
During the Learning Curve: A Retrospective Study}

\emph{Pancreas 2019 5;48(5):686-689}

PubMed: \url{https://www.ncbi.nlm.nih.gov/pubmed/?term=31091216}

OBJECTIVES: The diagnostic yield of endoscopic ultrasound-guided fine
needle aspiration (EUS-FNA) cytology widely varies depending on the
treatment method used. Liquid-based cytology (LBC) has gained popularity
in the gynecological field because of its efficacy in collection of
target cells and simplicity in the manipulation of specimens. Since the
introduction of EUS-FNA at our institution, we have used LBC for the
diagnosis of pancreatic mass lesions. This study aims to investigate the
diagnostic efficacy of EUS-FNA with LBC in patients with pancreatic mass
lesions during the learning curve for EUS-FNA. METHODS: In this study,
we retrospectively enrolled 222 patients with pancreatic mass lesions
who were diagnosed using EUS-FNA with LBC between 2011 and 2016. The
diagnostic yields for EUS-FNA with LBC for pancreatic mass lesions were
evaluated. RESULTS: The diagnostic sensitivity, specificity, and
accuracy for malignancy were found to be 93.9\%, 95.1\%, and 94.1\%,
respectively. CONCLUSIONS: This study suggests that EUS-FNA with LBC for
specimens provides good diagnostic efficacy in patients with pancreatic
mass lesions even during the learning curve for EUS-FNA.

{}

{}

\begin{center}\rule{0.5\linewidth}{\linethickness}\end{center}

 - \textbf{Ultrasound-guided fine needle aspiration cytology in the
diagnosis of hepatic and pancreatic perivascular epithelioid cell
tumors: A case series}

\emph{Diagnostic cytopathology 2019 Apr;47(4):315-319}

PubMed: \url{https://www.ncbi.nlm.nih.gov/pubmed/?term=30417984}

Perivascular epithelioid cell tumors (PEComas) are rare mesenchymal
tumors that can affect any part of the body. They can be sporadic or
arise in the setting of tuberous sclerosis (TSC). In this article, we
report a series of three hepatic and two pancreatic PEComas diagnosed
preoperatively with ultrasound-guided fine needle aspiration (FNA). All
patients were female (age range 28-70), had no personal history of TSC
and presented with a single, localized painless mass. Rapid on-site
evaluation (ROSE) of cytologic samples was performed for all cases to
evaluate for cellular content and adequacy of specimens. Direct smears
and cell block preparations revealed a proliferation of medium to large
polygonal epithelioid cells, with abundant eosinophilic and vacuolated
cytoplasm, arranged in sheets and nests. On immunohistochemistry (IHC),
neoplastic cells showed co-expression of melanocytic and smooth muscle
markers and a diagnosis of PEComa was rendered. PEComas of the pancreas
and liver are rare neoplasms, but should always be considered when
examining ``clear cell'' neoplasms, especially in young female patients.
If good quality cytologic samples are obtained by FNA, a correct
diagnosis can be achieved with the help of IHC. This is of particular
importance in order to plan adequate surgical strategy and to avoid
overtreatment.

{}

{}

\begin{center}\rule{0.5\linewidth}{\linethickness}\end{center}

\begin{itemize}
\tightlist
\item
  Cytohistological diagnosis of pancreatic serous cystadenoma: a
  multimodal approach.
\end{itemize}

\url{https://www.ncbi.nlm.nih.gov/pubmed/31235542}

\begin{center}\rule{0.5\linewidth}{\linethickness}\end{center}

Back to top

\begin{center}\rule{0.5\linewidth}{\linethickness}\end{center}

\pagebreak

\hypertarget{ampulla-duodenum}{%
\subsubsection{Ampulla Duodenum}\label{ampulla-duodenum}}

\begin{center}\rule{0.5\linewidth}{\linethickness}\end{center}

 - \textbf{Cytologic features of pancreatobiliary neoplasm of duodenum}

\emph{Diagnostic cytopathology 2019 Jul;():}

PubMed: \url{https://www.ncbi.nlm.nih.gov/pubmed/?term=31276311}

Small intestinal adenomas are uncommon. Majority of these occur in the
region of the ampulla of Vater. Adenomas of the ampulla can be further
subdivided into two types-intestinal and pancreatobiliary. While
intestinal adenomas are more frequent, pancreatobiliary adenomas are
rare. There is limited literature regarding the role of endoscopic
ultrasound-guided fine needle aspiration (EUS-FNA) in the diagnosis of
ampullary/peri-ampullary neoplasms. Here, we describe the cytologic
features of a pancreatobiliary neoplasm of the duodenum that was sampled
by EUS-FNA. The aspirate was cellular and revealed cells with moderately
abundant oncocytic cytoplasm. The nuclei were round with fine chromatin
and focally prominent nucleoli. Although the concurrent biopsy showed no
high-grade dysplasia or invasive carcinoma, the EUS and imaging findings
were highly suspicious for invasion. A broad differential diagnosis is
under consideration for a duodenal mass that encompasses neoplasms of
the biliary tract, pancreas, duodenum, and ampulla of Vater. To our
knowledge, cytologic features of a pancreatobiliary neoplasm of the
duodenum have not been previously reported. Our case highlights the
features seen on cytology with histologic correlation in the hopes of
elucidating features to better characterize these lesions.

{}

{}

\begin{center}\rule{0.5\linewidth}{\linethickness}\end{center}

Back to top

\begin{center}\rule{0.5\linewidth}{\linethickness}\end{center}

\pagebreak

\hypertarget{molecular-pathology}{%
\subsection{Molecular Pathology}\label{molecular-pathology}}

\begin{center}\rule{0.5\linewidth}{\linethickness}\end{center}

\hypertarget{pancreas-2}{%
\subsubsection{Pancreas}\label{pancreas-2}}

\begin{center}\rule{0.5\linewidth}{\linethickness}\end{center}

\hypertarget{pancreas-3}{%
\paragraph{Pancreas}\label{pancreas-3}}

\begin{center}\rule{0.5\linewidth}{\linethickness}\end{center}

 - \textbf{Morphological classification of pancreatic ductal
adenocarcinoma that predicts molecular subtypes and correlates with
clinical outcome}

\emph{Gut 2019 Jun;():}

PubMed: \url{https://www.ncbi.nlm.nih.gov/pubmed/?term=31201285}

INTRODUCTION: Transcriptional analyses have identified several distinct
molecular subtypes in pancreatic ductal adenocarcinoma (PDAC) that have
prognostic and potential therapeutic significance. However, to date, an
indepth, clinicomorphological correlation of these molecular subtypes
has not been performed. We sought to identify specific morphological
patterns to compare with known molecular subtypes, interrogate their
biological significance, and furthermore reappraise the current grading
system in PDAC. DESIGN: We first assessed 86 primary, chemotherapy-naive
PDAC resection specimens with matched RNA-Seq data for specific,
reproducible morphological patterns. Differential expression was applied
to the gene expression data using the morphological features. We next
compared the differentially expressed gene signatures with previously
published molecular subtypes. Overall survival (OS) was correlated with
the morphological and molecular subtypes. RESULTS: We identified four
morphological patterns that segregated into two components (`gland
forming' and `non-gland forming') based on the presence/absence of
well-formed glands. A morphological cut-off (≥40\% `non-gland forming')
was established using RNA-Seq data, which identified two groups (A and
B) with gene signatures that correlated with known molecular subtypes.
There was a significant difference in OS between the groups. The
morphological groups remained significantly prognostic within cancers
that were moderately differentiated and classified as `classical' using
RNA-Seq. CONCLUSION: Our study has demonstrated that PDACs can be
morphologically classified into distinct and biologically relevant
categories which predict known molecular subtypes. These results provide
the basis for an improved taxonomy of PDAC, which may lend itself to
future treatment strategies and the development of deep learning models.

{}

{}

\begin{center}\rule{0.5\linewidth}{\linethickness}\end{center}

 - \textbf{NRG1 Gene Fusions Are Recurrent, Clinically Actionable Gene
Rearrangements in KRAS Wild-Type Pancreatic Ductal Adenocarcinoma}

\emph{Clinical cancer research : an official journal of the American
Association for Cancer Research 2019 Aug;25(15):4674-4681}

PubMed: \url{https://www.ncbi.nlm.nih.gov/pubmed/?term=31068372}

PURPOSE: Gene fusions involving neuregulin 1 (NRG1) have been noted in
multiple cancer types and have potential therapeutic implications.
Although varying results have been reported in other cancer types, the
efficacy of the HER-family kinase inhibitor afatinib in the treatment of
NRG1 fusion-positive pancreatic ductal adenocarcinoma is not fully
understood. EXPERIMENTAL DESIGN: Forty-seven patients with pancreatic
ductal adenocarcinoma received comprehensive whole-genome and
transcriptome sequencing and analysis. Two patients with gene fusions
involving NRG1 received afatinib treatment, with response measured by
pretreatment and posttreatment PET/CT imaging. RESULTS: Three of 47
(6\%) patients with advanced pancreatic ductal adenocarcinoma were
identified as KRAS wild type by whole-genome sequencing. All KRAS
wild-type tumors were positive for gene fusions involving the ERBB3
ligand NRG1. Two of 3 patients with NRG1 fusion-positive tumors were
treated with afatinib and demonstrated a significant and rapid response
while on therapy. CONCLUSIONS: This work adds to a growing body of
evidence that NRG1 gene fusions are recurrent, therapeutically
actionable genomic events in pancreatic cancers. Based on the clinical
outcomes described here, patients with KRAS wild-type tumors harboring
NRG1 gene fusions may benefit from treatment with afatinib.See related
commentary by Aguirre, p.~4589.

{}

{}

\begin{center}\rule{0.5\linewidth}{\linethickness}\end{center}

 - \textbf{Oncogenic NRG1 Fusions: A New Hope for Targeted Therapy in
Pancreatic Cancer}

\emph{Clinical cancer research : an official journal of the American
Association for Cancer Research 2019 Aug;25(15):4589-4591}

PubMed: \url{https://www.ncbi.nlm.nih.gov/pubmed/?term=31164372}

Approximately 8\%-10\% of pancreatic ductal adenocarcinoma cases are
KRAS wild type. In a subset of these tumors, NRG1 gene fusions have been
identified as targetable oncogenic drivers, a discovery that highlights
the importance of deep molecular characterization for KRAS wild-type
pancreatic cancers and provides a novel treatment strategy in this
disease.See related article by Jones et al., p.~4674.

{}

{}

\begin{center}\rule{0.5\linewidth}{\linethickness}\end{center}

 - \textbf{Maintenance Olaparib for Germline BRCA-Mutated Metastatic
Pancreatic Cancer}

\emph{The New England journal of medicine 2019 07;381(4):317-327}

PubMed: \url{https://www.ncbi.nlm.nih.gov/pubmed/?term=31157963}

BACKGROUND: Patients with a germline BRCA1 or BRCA2 mutation make up a
small subgroup of those with metastatic pancreatic cancer. The
poly(adenosine diphosphate-ribose) polymerase (PARP) inhibitor olaparib
has had antitumor activity in this population. METHODS: We conducted a
randomized, double-blind, placebo-controlled, phase 3 trial to evaluate
the efficacy of olaparib as maintenance therapy in patients who had a
germline BRCA1 or BRCA2 mutation and metastatic pancreatic cancer and
disease that had not progressed during first-line platinum-based
chemotherapy. Patients were randomly assigned, in a 3:2 ratio, to
receive maintenance olaparib tablets (300 mg twice daily) or placebo.
The primary end point was progression-free survival, which was assessed
by blinded independent central review. RESULTS: Of the 3315 patients who
underwent screening, 154 underwent randomization and were assigned to a
trial intervention (92 to receive olaparib and 62 to receive placebo).
The median progression-free survival was significantly longer in the
olaparib group than in the placebo group (7.4 months vs.~3.8 months;
hazard ratio for disease progression or death, 0.53; 95\% confidence
interval {[}CI{]}, 0.35 to 0.82; P = 0.004). An interim analysis of
overall survival, at a data maturity of 46\%, showed no difference
between the olaparib and placebo groups (median, 18.9 months vs.~18.1
months; hazard ratio for death, 0.91; 95\% CI, 0.56 to 1.46; P = 0.68).
There was no significant between-group difference in health-related
quality of life, as indicated by the overall change from baseline in the
global quality-of-life score (on a 100-point scale, with higher scores
indicating better quality of life) based on the European Organization
for Research and Treatment of Cancer Quality of Life Questionnaire
(between-group difference, -2.47 points; 95\% CI, -7.27 to 2.33). The
incidence of grade 3 or higher adverse events was 40\% in the olaparib
group and 23\% in the placebo group (between-group difference, 16
percentage points; 95\% CI, -0.02 to 31); 5\% and 2\% of the patients,
respectively, discontinued the trial intervention because of an adverse
event. CONCLUSIONS: Among patients with a germline BRCA mutation and
metastatic pancreatic cancer, progression-free survival was longer with
maintenance olaparib than with placebo. (Funded by AstraZeneca and
others; POLO ClinicalTrials.gov number, NCT02184195.).

{}

{}

\begin{center}\rule{0.5\linewidth}{\linethickness}\end{center}

 - \textbf{Circulating Tumor DNA as a Clinical Test in Resected
Pancreatic Cancer}

\emph{Clinical cancer research : an official journal of the American
Association for Cancer Research 2019 Aug;25(16):4973-4984}

PubMed: \url{https://www.ncbi.nlm.nih.gov/pubmed/?term=31142500}

PURPOSE: In research settings, circulating tumor DNA (ctDNA) shows
promise as a tumor-specific biomarker for pancreatic ductal
adenocarcinoma (PDAC). This study aims to perform analytical and
clinical validation of a KRAS ctDNA assay in a Clinical Laboratory
Improvement Amendments (CLIA) and College of American
Pathology-certified clinical laboratory. EXPERIMENTAL DESIGN:
Digital-droplet PCR was used to detect the major PDAC-associated somatic
KRAS mutations (G12D, G12V, G12R, and Q61H) in liquid biopsies. For
clinical validation, 290 preoperative and longitudinal postoperative
plasma samples were collected from 59 patients with PDAC. The utility of
ctDNA status to predict PDAC recurrence during follow-up was assessed.
RESULTS: ctDNA was detected preoperatively in 29 (49\%) patients and was
an independent predictor of decreased recurrence-free survival (RFS) and
overall survival (OS). Patients who had neoadjuvant chemotherapy were
less likely to have preoperative ctDNA than were chemo-naïve patients
(21\% vs.~69\%; P \textless{} 0.001). ctDNA levels dropped significantly
after tumor resection. Persistence of ctDNA in the immediate
postoperative period was associated with a high rate of recurrence and
poor median RFS (5 months). ctDNA detected during follow-up predicted
clinical recurrence {[}sensitivity 90\% (95\% confidence interval (CI),
74\%-98\%), specificity 88\% (95\% CI, 62\%-98\%){]} with a median lead
time of 84 days (interquartile range, 25-146). Detection of ctDNA during
postpancreatectomy follow-up was associated with a median OS of 17
months, while median OS was not yet reached at 30 months for patients
without ctDNA (P = 0.011). CONCLUSIONS: Measurement of KRAS ctDNA in a
CLIA laboratory setting can be used to predict recurrence and survival
in patients with PDAC.

{}

{}

\begin{center}\rule{0.5\linewidth}{\linethickness}\end{center}

 - \textbf{Next-Generation Sequencing in Pancreatic Cancer}

\emph{Pancreas 2019 Jul;48(6):739-748}

PubMed: \url{https://www.ncbi.nlm.nih.gov/pubmed/?term=31206465}

Pancreatic ductal adenocarcinoma (PDAC) is lethal, and the majority of
patients present with locally advanced or metastatic disease that is not
amenable to cure. Thus, with surgical resection being the only curative
modality, it is critical that disease is identified at an earlier stage
to allow the appropriate therapy to be applied. Unfortunately, a
specific biomarker for early diagnosis has not yet been identified;
hence, no screening process exists. Recently, high-throughput screening
and next-generation sequencing (NGS) have led to the identification of
novel biomarkers for many disease processes, and work has commenced in
PDAC. Genomic data generated by NGS not only have the potential to
assist clinicians in early diagnosis and screening, especially in
high-risk populations, but also may eventually allow the development of
personalized treatment programs with targeted therapies, given the large
number of gene mutations seen in PDAC. This review introduces the basic
concepts of NGS and provides a comprehensive review of the current
understanding of genetics in PDAC as related to discoveries made using
NGS.

{}

{}

\begin{center}\rule{0.5\linewidth}{\linethickness}\end{center}

 - \textbf{p110γ deficiency protects against pancreatic carcinogenesis
yet predisposes to diet-induced hepatotoxicity}

\emph{Proceedings of the National Academy of Sciences of the United
States of America 2019 Jul;116(29):14724-14733}

PubMed: \url{https://www.ncbi.nlm.nih.gov/pubmed/?term=31266893}

Pancreatic ductal adenocarcinoma (PDAC) is notorious for its poor
survival and resistance to conventional therapies. PI3K signaling is
implicated in both disease initiation and progression, and specific
inhibitors of selected PI3K p110 isoforms for managing solid tumors are
emerging. We demonstrate that increased activation of PI3K signals
cooperates with oncogenic Kras to promote aggressive PDAC in vivo. The
p110γ isoform is overexpressed in tumor tissue and promotes
carcinogenesis via canonical AKT signaling. Its selective blockade
sensitizes tumor cells to gemcitabine in vitro, and genetic ablation of
p110γ protects against Kras-induced tumorigenesis. Diet/obesity was
identified as a crucial means of p110 subunit up-regulation, and in the
setting of a high-fat diet, p110γ ablation failed to protect against
tumor development, showing increased activation of pAKT and hepatic
damage. These observations suggest that a careful and judicious approach
should be considered when targeting p110γ for therapy, particularly in
obese patients.

{}

{}

\begin{center}\rule{0.5\linewidth}{\linethickness}\end{center}

 - \textbf{Endogenous Gastrin Collaborates With Mutant KRAS in
Pancreatic Carcinogenesis}

\emph{Pancreas 2019 Aug;48(7):894-903}

PubMed: \url{https://www.ncbi.nlm.nih.gov/pubmed/?term=31268978}

OBJECTIVE: The KRAS gene is the most frequently mutated gene in
pancreatic cancer, and no successful anti-Ras therapy has been
developed. Gastrin has been shown to stimulate pancreatic cancer in an
autocrine fashion. We hypothesized that reactivation of the peptide
gastrin collaborates with KRAS during pancreatic carcinogenesis.
METHODS: LSL-Kras; P48-Cre (KC) mutant KRAS transgenic mice were crossed
with gastrin-KO (GKO) mice to develop GKO/KC mice. Pancreata were
examined for 8 months for stage of pancreatic intraepithelial neoplasia
lesions, inflammation, fibrosis, gastrin peptide, and microRNA
expression. Pancreatic intraepithelial neoplasias from mice were
collected by laser capture microdissection and subjected to
reverse-phase protein microarray, for gastrin and protein kinases
associated with signal transduction. Gastrin mRNA was measured by RNAseq
in human pancreatic cancer tissues and compared to that in normal
pancreas. RESULTS: In the absence of gastrin, PanIN progression,
inflammation, and fibrosis were significantly decreased and signal
transduction was reversed to the canonical pathway with decreased KRAS.
Gastrin re-expression in the PanINs was mediated by miR-27a. Gastrin
mRNA expression was significantly increased in human pancreatic cancer
samples compared to normal human pancreas controls. CONCLUSIONS: This
study supports the mitogenic role of gastrin in activation of KRAS
during pancreatic carcinogenesis.

{}

{}

\begin{center}\rule{0.5\linewidth}{\linethickness}\end{center}

 - \textbf{Switchable CAR-T cells mediate remission in metastatic
pancreatic ductal adenocarcinoma}

\emph{Gut 2019 06;68(6):1052-1064}

PubMed: \url{https://www.ncbi.nlm.nih.gov/pubmed/?term=30121627}

OBJECTIVE: Pancreatic ductal adenocarcinoma (PDAC) is a disease of unmet
medical need. While immunotherapy with chimeric antigen receptor T
(CAR-T) cells has shown much promise in haematological malignancies,
their efficacy for solid tumours is challenged by the lack of
tumour-specific antigens required to avoid on-target, off-tumour
effects. Switchable CAR-T cells whereby activity of the CAR-T cell is
controlled by dosage of a tumour antigen-specific recombinant Fab-based
`switch' to afford a fully tunable response may overcome this
translational barrier. DESIGN: In this present study, we have used
conventional and switchable CAR-T cells to target the antigen HER2,
which is upregulated on tumour cells, but also present at low levels on
normal human tissue. We used patient-derived xenograft models derived
from patients with stage IV PDAC that mimic the most aggressive features
of PDAC, including severe liver and lung metastases. RESULTS: Switchable
CAR-T cells followed by administration of the switch directed against
human epidermal growth factor receptor 2 (HER2)-induced complete
remission in difficult-to-treat, patient-derived advanced pancreatic
tumour models. Switchable HER2 CAR-T cells were as effective as
conventional HER2 CAR-T cells in vivo testing a range of different CAR-T
cell doses. CONCLUSION: These results suggest that a switchable CAR-T
system is efficacious against aggressive and disseminated tumours
derived from patients with advanced PDAC while affording the potential
safety of a control switch.

{}

{}

\begin{center}\rule{0.5\linewidth}{\linethickness}\end{center}

\begin{itemize}
\tightlist
\item
  Experimental microdissection enables functional harmonisation of
  pancreatic cancer subtypes.
\end{itemize}

\url{https://www.ncbi.nlm.nih.gov/pubmed/30658994}

\begin{center}\rule{0.5\linewidth}{\linethickness}\end{center}

\begin{itemize}
\tightlist
\item
  Establishment and Analysis of a 3D Co-Culture Spheroid Model of
  Pancreatic Adenocarcinoma for Application in Drug Discovery.
\end{itemize}

\url{https://www.ncbi.nlm.nih.gov/pubmed/30912022}

\begin{center}\rule{0.5\linewidth}{\linethickness}\end{center}

Back to top

\begin{center}\rule{0.5\linewidth}{\linethickness}\end{center}

\pagebreak

\hypertarget{spn-1}{%
\paragraph{SPN}\label{spn-1}}

Solid Pseudopapillary Neoplasm

\begin{center}\rule{0.5\linewidth}{\linethickness}\end{center}

 - \textbf{Targeted next generation sequencing of pancreatic solid
pseudopapillary neoplasms show mutations in Wnt signaling pathway genes}

\emph{Pathology international 2019 Apr;69(4):193-201}

PubMed: \url{https://www.ncbi.nlm.nih.gov/pubmed/?term=30811747}

Solid pseudopapillary neoplasms of the pancreas are rare neoplasms that
have been shown to harbor recurrent somatic pathogenic variants in the
beta-catenin gene, CTNNB1. Here, we used targeted next generation
sequencing to analyze these tumors for other associated mutations. Six
cases of solid pseudopapillary neoplasms were studied. DNA extracted
from formalin-fixed paraffin embedded tissue blocks was analyzed using
the Ion Torrent platform, with the 50-gene Ampliseq Cancer Hotspot Panel
v2 (CHPv2), with further variant validation performed by Sanger
sequencing. Four tumors (67\%) were confirmed to harbor mutations within
CTNNB1, two with c.109T \textgreater{} G p.(Ser37Ala) and two with
c.94G \textgreater{} A p.(Asp32Asn). One case showed a frameshift
deletion in the Adenomatous Polyposis Coli gene, APC c.3964delG
p.(Glu1322Lysfs*93) with a variant allele frequency of 42.6\%. Sanger
sequencing on non-tumoral tissue confirmed the variant was somatic. The
patient with the APC mutation developed metastasis and died. In addition
to the four cases harboring CTNNB1 variants, we found a case
characterized by poor outcome, showing a rare frameshift deletion in the
APC gene. Since the APC product interacts with beta-catenin, APC
variants may, in addition to CTNNB1, contribute to the pathogenesis of
solid pseudopapillary neoplasms via the Wnt signaling pathway.

{}

{}

\begin{center}\rule{0.5\linewidth}{\linethickness}\end{center}

Back to top

\begin{center}\rule{0.5\linewidth}{\linethickness}\end{center}

\pagebreak

\hypertarget{pancreatitis-other-diseases-1}{%
\paragraph{Pancreatitis \& Other
Diseases}\label{pancreatitis-other-diseases-1}}

Molecular Studies on Pancreatitis \& Other Diseases

\begin{center}\rule{0.5\linewidth}{\linethickness}\end{center}

 - \textbf{Heme Oxygenase-1 Polymorphism Is Associated With the
Development of Necrotic Acute Pancreatitis Via Vascular Cell Adhesion
Molecule-1 and the E-Selectin Expression Regulation Pathway}

\emph{Pancreas 2019 Jul;48(6):787-791}

PubMed: \url{https://www.ncbi.nlm.nih.gov/pubmed/?term=31210657}

OBJECTIVES: Severe acute pancreatitis can lead to systemic
complications. Here, we explore the mechanisms based on our previous
study associated with the deregulation of heme oxygenase-1 (HO-1) and
development of severe acute pancreatitis. METHODS: Acute pancreatitis
patients (n = 135) and age- and sex-matched healthy controls (n = 108)
were studied. The polymerase chain reaction products were analyzed with
an ABI 3130 genetic analyzer and GeneMapper software. A short allele was
defined ≤27 dinucleotide (GT) repeats, whereas a long allele was defined
\textgreater{}27 GT. Levels of 12 different cytokines in blood serum
were measured by enzyme-linked immunosorbent assay. All samples in this
study were consistently stored in -80°C. RESULTS: Patients with the long
long genotype expressed E-selectin and vascular cell adhesion molecule-1
at statistically significantly higher levels in serum compared with
short short genotype or short long genotypes. Vascular cell adhesion
molecule-1 and E-selectin serum levels significantly correlate with the
total allele length of the HO-1 promoter region. CONCLUSION:
Polymorphism of the GT repeats in the HO-1 promoter region may be a risk
factor for developing acute necrotizing pancreatitis due to deregulation
of the immune response.

{}

{}

\begin{center}\rule{0.5\linewidth}{\linethickness}\end{center}

Back to top

\begin{center}\rule{0.5\linewidth}{\linethickness}\end{center}

\pagebreak

\hypertarget{molecular-research-on-microenvironment}{%
\paragraph{Molecular Research on
Microenvironment}\label{molecular-research-on-microenvironment}}

Tumor Stroma Interactions, Microenvironment, Inflammatory Response,
Microbiome

\begin{center}\rule{0.5\linewidth}{\linethickness}\end{center}

 - \textbf{Rethinking T Cells in Pancreas Cancer}

\emph{Clinical cancer research : an official journal of the American
Association for Cancer Research 2019 Jul;25(13):3747-3749}

PubMed: \url{https://www.ncbi.nlm.nih.gov/pubmed/?term=31072844}

Patients with pancreatic ductal adenocarcinoma do not benefit from
checkpoint blockade. However, human tumors harbor evidence of adaptive
immunity in clonally expanded T-cell populations. Immune intact modeling
of human tumors identifies stromal sequestration as a mechanism of
immune escape. Targeting the stroma combined with checkpoint blockade
unleashes antitumor immunity.See related article by Seo et al., p.~3934.

{}

{}

\begin{center}\rule{0.5\linewidth}{\linethickness}\end{center}

 - \textbf{Vasohibin-2 plays an essential role in metastasis of
pancreatic ductal adenocarcinoma}

\emph{Cancer science 2019 Jul;110(7):2296-2308}

PubMed: \url{https://www.ncbi.nlm.nih.gov/pubmed/?term=31074083}

Vasohibin-2 (VASH2) is expressed in various cancers and promotes their
progression. We recently reported that pancreatic cancer patients with
higher VASH2 expression show poorer prognosis. Herein, we sought to
characterize the role of VASH2 in pancreatic cancer. We used
LSL-KrasG12D ; LSL-Trp53R172H ; Pdx-1-Cre (KPC) mice, a mouse model of
pancreatic ductal adenocarcinoma (PDAC), and cells isolated from them
(KPC cells). Knockdown of Vash2 from PDAC cells did not affect their
proliferation, but decreased their migration. When Vash2-knockdown PDAC
cells were orthotopically inoculated, liver metastasis and peritoneal
dissemination were reduced, and the survival period was significantly
prolonged. When KPC mice were crossed with Vash2-deficient mice,
metastasis was significantly decreased in Vash2-deficient KPC mice.
VASH2 was recently identified to have tubulin carboxypeptidase activity.
VASH2 knockdown decreased, whereas VASH2 overexpression increased
tubulin detyrosination of PDAC cells, and tubulin carboxypeptidase (TCP)
inhibitor parthenolide inhibited VASH2-induced cell migration. We next
clarified its role in the tumor microenvironment. Tumor angiogenesis was
significantly abrogated in~vivo when VASH2 was knocked down or deleted.
We further examined genes downregulated by Vash2 knockdown in KPC cells,
and found chemokines and cytokines that were responsible for the
recruitment of myeloid derived suppressor cells (MDSC). Indeed, MDSC
were accumulated in PDAC of KPC mice, and they were significantly
decreased in Vash2-deficient KPC mice. These findings suggest that VASH2
plays an essential role in the metastasis of PDAC with multiple effects
on both cancer cells and the tumor microenvironment, including tubulin
detyrosination, tumor angiogenesis and evasion of tumor immunity.

{}

{}

\begin{center}\rule{0.5\linewidth}{\linethickness}\end{center}

 - \textbf{Development of resistance to FAK inhibition in pancreatic
cancer is linked to stromal depletion}

\emph{Gut 2019 May;():}

PubMed: \url{https://www.ncbi.nlm.nih.gov/pubmed/?term=31076405}

OBJECTIVE: We investigated how pancreatic cancer developed resistance to
focal adhesion kinase (FAK) inhibition over time. DESIGN: Pancreatic
ductal adenocarcinoma (PDAC) tumours from KPC mice (p48-CRE;
LSL-KRasG12D/wt; p53flox/wt) treated with FAK inhibitor were analysed
for the activation of a compensatory survival pathway in resistant
tumours. We identified pathways involved in the regulation of signal
transducer and activator of transcription 3 (STAT3) signalling on FAK
inhibition by gene set enrichment analysis and verified these outcomes
by RNA interference studies. We also tested combinatorial approaches
targeting FAK and STAT3 in syngeneic transplantable mouse models of PDAC
and KPC mice. RESULTS: In KPC mice, the expression levels of
phosphorylated STAT3 (pSTAT3) were increased in PDAC cells as they
progressed on FAK inhibitor therapy. This progression corresponded to
decreased collagen density, lowered numbers of SMA+ fibroblasts and
downregulation of the transforming growth factor beta (TGF-β)/SMAD
signalling pathway in FAK inhibitor-treated PDAC tumours. Furthermore,
TGF-β production by fibroblasts in vitro drives repression of STAT3
signalling and enhanced responsiveness to FAK inhibitor therapy.
Knockdown of SMAD3 in pancreatic cancer cells abolished the inhibitory
effects of TGF-β on pSTAT3. We further found that tumour-intrinsic STAT3
regulates the durability of the antiproliferative activity of FAK
inhibitor, and combinatorial targeting of FAK and Janus kinase/STAT3 act
synergistically to suppress pancreatic cancer progression in mouse
models. CONCLUSION: Stromal depletion by FAK inhibitor therapy leads to
eventual treatment resistance through the activation of STAT3
signalling. These data suggest that, similar to tumour-targeted
therapies, resistance mechanisms to therapies targeting stromal
desmoplasia may be critical to treatment durability.

{}

{}

\begin{center}\rule{0.5\linewidth}{\linethickness}\end{center}

 - \textbf{MicroRNA modulated networks of adaptive and innate immune
response in pancreatic ductal adenocarcinoma}

\emph{PloS one 2019 05;14(5):e0217421}

PubMed: \url{https://www.ncbi.nlm.nih.gov/pubmed/?term=31150430}

Despite progress in treatment strategies, only \textasciitilde{}24\% of
pancreatic ductal adenocarcinoma (PDAC) patients survive \textgreater{}1
year. Our goal was to elucidate deregulated pathways modulated by
microRNAs (miRNAs) in PDAC and Vater ampulla (AMP) cancers. Global miRNA
expression was identified in 19 PDAC, 6 AMP and 25 paired,
histologically normal pancreatic tissues using the GeneChip 4.0 miRNA
arrays. Computational approaches were used for miRNA target
prediction/identification of miRNA-regulated pathways. Target gene
expression was validated in 178 pancreatic cancer and 4 pancreatic
normal tissues from The Cancer Genome Atlas (TCGA). 20 miRNAs were
significantly deregulated (FC≥2 and p\textless{}0.05) (15 down- and 5
up-regulated) in PDAC. miR-216 family (miR-216a-3p, miR-216a-5p,
miR-216b-3p and miR-216b-5p) was consistently down-regulated in PDAC.
miRNA-modulated pathways are associated with innate and adaptive immune
system responses in PDAC. AMP cancers showed 8 down- and 1 up-regulated
miRNAs (FDR p\textless{}0.05). Most enriched pathways (p\textless{}0.01)
were RAS and Nerve Growth Factor signaling. PDAC and AMP display
different global miRNA expression profiles and miRNA regulated
networks/tumorigenesis pathways. The immune response was enriched in
PDAC, suggesting the existence of immune checkpoint pathways more
relevant to PDAC than AMP.

{}

{}

\begin{center}\rule{0.5\linewidth}{\linethickness}\end{center}

 - \textbf{Stromal Microenvironment Shapes the Intratumoral Architecture
of Pancreatic Cancer}

\emph{Cell 2019 Jun;178(1):160-175.e27}

PubMed: \url{https://www.ncbi.nlm.nih.gov/pubmed/?term=31155233}

Single-cell technologies have described heterogeneity across tissues,
but the spatial distribution and forces that drive single-cell
phenotypes have not been well defined. Combining single-cell RNA and
protein analytics in studying the role of stromal cancer-associated
fibroblasts (CAFs) in modulating heterogeneity in pancreatic cancer
(pancreatic ductal adenocarcinoma {[}PDAC{]}) model systems, we have
identified significant single-cell population shifts toward invasive
epithelial-to-mesenchymal transition (EMT) and proliferative (PRO)
phenotypes linked with mitogen-activated protein kinase (MAPK) and
signal transducer and activator of transcription 3 (STAT3) signaling.
Using high-content digital imaging of RNA in situ hybridization in 195
PDAC tumors, we quantified these EMT and PRO subpopulations in 319,626
individual cancer cells that can be classified within the context of
distinct tumor gland ``units.'' Tumor gland typing provided an
additional layer of intratumoral heterogeneity that was associated with
differences in stromal abundance and clinical outcomes. This
demonstrates the impact of the stroma in shaping tumor architecture by
altering inherent patterns of tumor glands in human PDAC.

{}

{}

\begin{center}\rule{0.5\linewidth}{\linethickness}\end{center}

 - \textbf{Metastasis of pancreatic cancer: An uninflamed liver
micromilieu controls cell growth and cancer stem cell properties by
oxidative phosphorylation in pancreatic ductal epithelial cells}

\emph{Cancer letters 2019 Jul;453():95-106}

PubMed: \url{https://www.ncbi.nlm.nih.gov/pubmed/?term=30930235}

Pancreatic ductal adenocarcinoma (PDAC) is commonly diagnosed when liver
metastases already emerged. We recently demonstrated that hepatic
stromal cells determine the dormancy status along with cancer stem cell
(CSC) properties of pancreatic ductal epithelial cells (PDECs) during
metastasis. This study investigated the influence of the hepatic
microenvironment - and its inflammatory status - on metabolic
alterations and how these impact cell growth and CSC-characteristics of
PDECs. Coculture with hepatic stellate cells (HSCs), simulating a
physiological liver stroma, but not with hepatic myofibroblasts (HMFs)
representing liver inflammation promoted expression of Succinate
Dehydrogenase subunit B (SDHB) and an oxidative metabolism along with a
quiescent phenotype in PDECs. SiRNA-mediated SDHB knockdown increased
cell growth and CSC-properties. Moreover, liver micrometastases of tumor
bearing KPC mice strongly expressed SDHB while expression of the
CSC-marker Nestin was exclusively found in macrometastases.
Consistently, RNA-sequencing and in silico modeling revealed
significantly altered metabolic fluxes and enhanced SDH activity
predominantly in premalignant PDECs in the presence of HSC compared to
HMF. Overall, these data emphasize that the hepatic microenvironment
determines the metabolism of disseminated PDECs thereby controlling cell
growth and CSC-properties during liver metastasis.

{}

{}

\begin{center}\rule{0.5\linewidth}{\linethickness}\end{center}

Back to top

\begin{center}\rule{0.5\linewidth}{\linethickness}\end{center}

\pagebreak

\hypertarget{preneoplastic-and-preinvasive-lesions-1}{%
\paragraph{Preneoplastic and Preinvasive
Lesions}\label{preneoplastic-and-preinvasive-lesions-1}}

Molecular Pathology Preneoplastic and Preinvasive Lesions, PanIN, IPMN,
MCN, ICPN, IOPN

\begin{center}\rule{0.5\linewidth}{\linethickness}\end{center}

 - \textbf{Single-Cell Transcriptomics of Pancreatic Cancer Precursors
Demonstrates Epithelial and Microenvironmental Heterogeneity as an Early
Event in Neoplastic Progression}

\emph{Clinical cancer research : an official journal of the American
Association for Cancer Research 2019 Apr;25(7):2194-2205}

PubMed: \url{https://www.ncbi.nlm.nih.gov/pubmed/?term=30385653}

PURPOSE: Early detection of pancreatic ductal adenocarcinoma (PDAC)
remains elusive. Precursor lesions of PDAC, specifically intraductal
papillary mucinous neoplasms (IPMNs), represent a bona fide pathway to
invasive neoplasia, although the molecular correlates of progression
remain to be fully elucidated. Single-cell transcriptomics provides a
unique avenue for dissecting both the epithelial and microenvironmental
heterogeneities that accompany multistep progression from noninvasive
IPMNs to PDAC. EXPERIMENTAL DESIGN: Single-cell RNA sequencing was
performed through droplet-based sequencing on 5,403 cells from 2
low-grade IPMNs (LGD-IPMNs), 2 high-grade IPMNs (HGD-IPMN), and 2 PDACs
(all surgically resected). RESULTS: Analysis of single-cell
transcriptomes revealed heterogeneous alterations within the epithelium
and the tumor microenvironment during the progression of noninvasive
dysplasia to invasive cancer. Although HGD-IPMNs expressed many core
signaling pathways described in PDAC, LGD-IPMNs harbored subsets of
single cells with a transcriptomic profile that overlapped with invasive
cancer. Notably, a proinflammatory immune component was readily seen in
low-grade IPMNs, composed of cytotoxic T cells, activated T-helper
cells, and dendritic cells, which was progressively depleted during
neoplastic progression, accompanied by infiltration of myeloid-derived
suppressor cells. Finally, stromal myofibroblast populations were
heterogeneous and acquired a previously described tumor-promoting and
immune-evading phenotype during invasive carcinogenesis. CONCLUSIONS:
This study demonstrates the ability to perform high-resolution profiling
of the transcriptomic changes that occur during multistep progression of
cystic PDAC precursors to cancer. Notably, single-cell analysis provides
an unparalleled insight into both the epithelial and microenvironmental
heterogeneities that accompany early cancer pathogenesis and might be a
useful substrate to identify targets for cancer interception.See related
commentary by Hernandez-Barco et al., p.~2027.

{}

{}

\begin{center}\rule{0.5\linewidth}{\linethickness}\end{center}

 - \textbf{Multiple KRAS mutations in the non-mucinous epithelial lining
in the majority of mucinous cystic neoplasms of the pancreas}

\emph{Histopathology 2019 May;():}

PubMed: \url{https://www.ncbi.nlm.nih.gov/pubmed/?term=31077597}

AIMS: Mucinous cystic neoplasms (MCNs) of the pancreas are cystic
neoplasms lined by mucinous lining epithelium (MLE) with associated
ovarian-type stroma. Although a non-MLE (NMLE) can be observed in some
MCNs, whether cystic neoplasms with ovarian-type stroma and NMLE should
be classified as MCNs or separately designated is debated. METHODS AND
RESULTS: To test this, NMLEs were defined as flat or cuboidal epithelial
cells without intracytoplasmic mucin. A total of 112 MCNs were reviewed,
and the epithelium was classified as NMLE or MLE. A total of 110 females
and two males with a mean age of 46.5~±~12.3~years were included in this
study. At least focal NMLE was noted in 76.8\% (86/112) of MCNs. The
mean percentage of the neoplastic epithelium that was NMLE in these 86
cases was 46\%. NMLE was predominant (\textgreater{}50\%) in 38.4\%
(43/112) of cases. MCNs with NMLE were smaller (42~±~21~mm) than those
with MLE (60~±~36~mm, P~\textless{}~0.001), and all NMLEs had low-grade
dysplasia. Twelve MCNs with NMLE or MLE were selected for KRAS mutation
analysis with droplet digital polymerase chain reaction after laser
capture microdissection. All 12 MCNs showed multiple types of KRAS
mutation, which were detected in 92\% (11/12) of NMLE foci and 89\%
(8/9) of MLE foci. Predominant NMLE was common in small MCNs with
low-grade dysplasia. CONCLUSIONS: Clonal KRAS mutations were observed in
both NMLE and MLE, supporting the hypothesis that MCNs with NMLE should
be classified as MCNs.

{}

{}

\begin{center}\rule{0.5\linewidth}{\linethickness}\end{center}

 - \textbf{Cross Validation of the Monoclonal Antibody Das-1 in
Identification of High-Risk Mucinous Pancreatic Cystic Lesions}

\emph{Gastroenterology 2019 Sep;157(3):720-730.e2}

PubMed: \url{https://www.ncbi.nlm.nih.gov/pubmed/?term=31175863}

BACKGROUND \& AIMS: Although pancreatic cystic lesions (PCLs) are
frequently and incidentally detected, it is a challenge to determine
their risk of malignancy. In immunohistochemical and enzyme-linked
immunosorbent assay (ELISA) analyses of tissue and cyst fluid from
pancreatic intraductal papillary mucinous neoplasms, the monoclonal
antibody Das-1 identifies those at risk for malignancy with high levels
of specificity and sensitivity. We aimed to validate the ability of
Das-1 to identify high-risk PCLs in comparison to clinical guidelines
and clinical features, using samples from a multicenter cohort. METHODS:
We obtained cyst fluid samples of 169 PCLs (90 intraductal papillary
mucinous neoplasms, 43 mucinous cystic neoplasms, and 36 non-mucinous
cysts) from patients undergoing surgery at 4 tertiary referral centers
(January 2010 through June 2017). Histology findings from surgical
samples, analyzed independently and centrally re-reviewed in a blinded
manner, were used as the reference standard. High-risk PCLs were those
with invasive carcinomas, high-grade dysplasia, or intestinal-type
intraductal papillary mucinous neoplasms with intermediate-grade
dysplasia. An ELISA with Das-1 was performed in parallel using banked
cyst fluid samples. We evaluated the biomarker's performance, generated
area under the curve values, and conducted multivariate logistic
regression using clinical and pathology features. RESULTS: The ELISA for
Das-1 identified high-risk PCLs with 88\% sensitivity, 99\% specificity,
and 95\% accuracy, at a cutoff optical density value of 0.104. In
10-fold cross-validation analysis with 100 replications, Das-1
identified high-risk PCLs with 88\% sensitivity and 98\% specificity.
The Sendai, Fukuoka, and American Gastroenterological Association
guideline criteria identified high-risk PCLs with 46\%, 52\%, and 74\%
accuracy (P for comparison to Das-1 ELISA \textless{}.001). When we
controlled for Das-1 in multivariate regression, main pancreatic duct
dilation \textgreater{}5 mm (odds ratio, 14.98; 95\% confidence
interval, 2.63-108; P \textless{} .0012), main pancreatic duct dilation
≥1 cm (odds ratio, 47.9; 95\% confidence interval, 6.39-490; P
\textless{} .0001), and jaundice (odds ratio, 6.16; 95\% confidence
interval, 1.08-36.7; P~= .0397) were significantly associated with
high-risk PCLs. CONCLUSIONS: We validated the ability of an ELISA with
the monoclonal antibody Das-1 to detect PCLs at risk for malignancy with
high levels of sensitivity and specificity. This biomarker might be used
in conjunction with clinical guidelines to identify patients at risk for
malignancy.

{}

{}

\begin{center}\rule{0.5\linewidth}{\linethickness}\end{center}

 - \textbf{Intraductal Papillary Mucinous Neoplasms Arise from Multiple
Independent Clones, Each With Distinct Mutations}

\emph{Gastroenterology 2019 Jun;():}

PubMed: \url{https://www.ncbi.nlm.nih.gov/pubmed/?term=31175866}

BACKGROUND \& AIMS: Intraductal papillary mucinous neoplasms (IPMNs) are
lesions that can progress to invasive pancreatic cancer and an important
system for studies of pancreatic tumorigenesis. We performed
comprehensive genomic analyses of entire IPMNs to determine the
diversity of somatic mutations in genes that promote tumorigenesis.
METHODS: We microdissected neoplastic tissues from 6-24 regions each of
20 resected IPMNs, resulting in 227 neoplastic samples that were
analyzed by capture-based targeted sequencing. Somatic mutations in
genes associated with pancreatic tumorigenesis were assessed across
entire IPMN lesions, and the resulting data were supported by
evolutionary modeling, whole-exome sequencing, and in situ detection of
mutations. RESULTS: We found a high prevalence of heterogeneity among
mutations in IPMNs. Heterogeneity in mutations in KRAS and GNAS was
significantly more prevalent in IPMNs with low-grade dysplasia than in
IPMNs with high-grade dysplasia (P\textless{}.02). Whole-exome
sequencing confirmed that IPMNs contained multiple independent clones,
each with distinct mutations, as originally indicated by targeted
sequencing and evolutionary modeling. We also found evidence for
convergent evolution of mutations in RNF43 and TP53, which are acquired
during later stages of tumorigenesis. CONCLUSIONS: In an analysis of the
heterogeneity of mutations throughout IPMNs, we found that early-stage
IPMNs contain multiple independent clones, each with distinct mutations,
indicating their polyclonal origin. These findings challenge the model
in which pancreatic neoplasms arise from a single clone. Increasing our
understanding of the mechanisms of IPMN polyclonality could lead to
strategies to identify patients at increased risk for pancreatic cancer.

{}

{}

\begin{center}\rule{0.5\linewidth}{\linethickness}\end{center}

 - \textbf{Pancreatobiliary-type intraductal papillary mucinous neoplasm
of the pancreas may have two subtypes with distinct clinicopathological
and genetic features}

\emph{Human pathology 2019 Jun;():}

PubMed: \url{https://www.ncbi.nlm.nih.gov/pubmed/?term=31175918}

We recently experienced cases of pancreatobiliary-type intraductal
papillary mucinous neoplasms (PB-IPMNs) with imaging features resembling
pancreatic ductal adenocarcinomas (PDACs), and histological appearance
of purely pancreatobiliary morphology and highly distorted papillary
growth, which led to the present study aiming to systematically
investigate PB-IPMNs in comparison with PDACs. Surgical cases of
PB-IPMNs (n=31) and PDACs (n=24) were examined. PB-IPMNs were classified
into monotypic tumors (n=12; 39\%) consisting of entirely high-grade
pancreatobiliary-type neoplastic cells and polytypic cases (n=19; 61\%)
associated with components of low-grade dysplasia and/or other
histological types (e.g., gastric, intestinal, or oncocytic types).
Clinically, monotypic PB-IPMNs less commonly had dilatation of the
ampullary orifice (0\% vs.~74\%) and mucin hypersecretion (17\%
vs.~89\%) than polytypic cases. In most cases of monotypic PB-IPMNs,
cystic dilatation of the lesional ducts were less obvious on imaging;
therefore, 33\% were radiologically diagnosed as PDACs. Histologically,
intraductal tumors in monotypic cases showed a highly complex papillary
architecture with tubular/cribriform glands and irregular branching, and
all these cases were associated with invasive malignancy. GNAS mutations
were detected in polytypic PB-IPMNs (6/19; 32\%), but there were no GNAS
mutations in monotypic cases. The recurrence-free survival of patients
with monotypic PB-IPMN or PDAC was similar and significantly worse than
that of patients with polytypic PB-IPMN. In conclusion, some cases of
monotypic PB-IPMNs lacked the classic characteristics of IPMNs and
shared features with PDACs, raising the possibility that these cases may
be better classified as a papillary variant of PDACs rather than IPMNs.

{}

{}

\begin{center}\rule{0.5\linewidth}{\linethickness}\end{center}

 - \textbf{KRAS in Cyst Fluid Obtained by Endoscopic
Ultrasound-Fine-Needle Aspiration in Pancreatic Cystic Lesions: A
Systematic Review and Meta-analysis}

\emph{Pancreas 2019 Jul;48(6):749-758}

PubMed: \url{https://www.ncbi.nlm.nih.gov/pubmed/?term=31206466}

To evaluate the diagnostic accuracy of KRAS mutation in pancreatic
cystic fluid and compare it with carcinoembryonic antigen and cytology,
we identified studies with cyst fluid obtained by endoscopic ultrasound
prior to surgery. We classified cysts as malignant, premalignant, and
benign. A random-effects model was used for quantitative meta-analysis.
Pooled sensitivities, specificities, and summary receiver operating
characteristic curve analysis were conducted. We analyzed 16 studies,
with 3429 patients, including 731 referred for surgery. Carcinoembryonic
antigen was better for clinically significant cysts (premalignant and
malignant) with sensitivity = 0.58 (95\% confidence interval {[}CI{]},
0.53-0.65), specificity = 0.9 (95\% CI, 0.76-0.97), and area under the
curve (AUC) = 0.69. Cytology performed better in malignant cysts, with
sensitivity = 0.37 (95\% CI, 0.27-0.48), specificity = 0.96 (95\% CI,
0.93-0.98), and AUC = 0.78. Isolated, KRAS mutation failed the diagnosis
of malignant and significant cysts, with sensitivities = 0.43 (95\% CI,
0.34-0.43) and 0.46 (95\% CI, 0.42-0.51), specificities = 0.62 (95\% CI,
0.56-0.68) and 0.97 (95\% CI, 0.92-0.99), and AUCs = 0.56 and 0.53,
respectively. Carcinoembryonic antigen and cytology are more accurate
than KRAS. Additional studies are lacking to recommend KRAS as a single
diagnostic test.

{}

{}

\begin{center}\rule{0.5\linewidth}{\linethickness}\end{center}

Back to top

\begin{center}\rule{0.5\linewidth}{\linethickness}\end{center}

\pagebreak

\hypertarget{neuroendocrine-1}{%
\subsubsection{Neuroendocrine}\label{neuroendocrine-1}}

\begin{center}\rule{0.5\linewidth}{\linethickness}\end{center}

 - \textbf{Microenvironment and tumor inflammatory features improve
prognostic prediction in gastro-entero-pancreatic neuroendocrine
neoplasms}

\emph{The journal of pathology. Clinical research 2019 May;():}

PubMed: \url{https://www.ncbi.nlm.nih.gov/pubmed/?term=31136102}

Microenvironment-related immune and inflammatory markers, when combined
with established Ki-67 and morphology parameters, can improve prognostic
prediction in gastro-entero-pancreatic neuroendocrine neoplasms
(GEP-NENs). Therefore, we evaluate the prognostic value of
microenvironment and tumor inflammatory features (MoTIFs) in GEP-NENs.
For this purpose, formalin-fixed paraffin-embedded tissue sections from
350 patients were profiled by immunohistochemistry for immune,
inflammatory, angiogenesis, proliferation, NEN-, and fibroblast-related
markers. A total of 314 patients were used to generate overall survival
(OS) and disease-free survival (DFS) MoTIFs prognostic indices (PIs).
PIs and additional variables were assessed using Cox models to generate
nomograms for predicting 5-year OS and DFS. A total of 36 patients were
used for external validation of PIs and nomograms' prognostic
segregations. From our analysis, G1/G2 versus G3 GEP-NENs showed
phenotypic divergence with immune-inflammatory markers. HLA, CD3, CD8,
and PD-1/PD-L1 IHC expression separated G3 into two sub-categories with
high versus low adaptive immunity-related features. MoTIFs PI for OS
based on COX-2Tumor(T) \textgreater{} 4, PD-1Stromal(S) \textgreater{}
0, CD8S \textless{} 1, and HLA-IS \textless{} 1 was associated with
worst survival (hazard ratio {[}HR{]} 2.50; 95\% confidence interval
{[}CI{]}, 2.12-2.96; p \textless{} 0.0001). MoTIFs PI for DFS was based
on COX-2T \textgreater{} 4, PD-1S \textgreater{} 4, HLA-IS \textless{}
1, HLA-IT \textless{} 2, HLA-DRS \textless{} 6 (HR 1.77; 95\% CI,
1.58-1.99; p \textless{} 0.0001). Two nomograms were developed including
morphology (HR 4.83; 95\% CI, 2.30-10.15; p \textless{} 0.001) and Ki-67
(HR 11.32; 95\% CI, 5.28-24.24; p \textless{} 0.001) for OS, and
morphology (PI = 0: HR 10.23; 95\% CI, 5.67-18.47; PI = 5: HR 2.87; 95\%
CI, 1.21-6.81; p \textless{} 0.001) and MoTIFs PI for DFS in
well-differentiated GEP-NENs (HR 6.21; 95\% CI, 2.52-13.31; p
\textless{} 0.001). We conclude that G1/G2 to G3 transition is
associated with immune-inflammatory profile changes; in fact, MoTIFs
combined with morphology and Ki-67 improve 5-year DFS prediction in
GEP-NENs. The immune context of a subset of G3 poorly differentiated
tumors is consistent with activation of adaptive immunity, suggesting a
potential for responsiveness to immunotherapy targeting immune
checkpoints.

{}

{}

\begin{center}\rule{0.5\linewidth}{\linethickness}\end{center}

Back to top

\begin{center}\rule{0.5\linewidth}{\linethickness}\end{center}

\pagebreak

\hypertarget{feedback}{%
\section{Feedback}\label{feedback}}

Please send your feedbacks using the forms below:

\href{https://docs.google.com/forms/d/e/1FAIpQLSeD3Z9J6Y7eMmiyM12f_SfAmHUlykb1zxZcwO6lg7cebGYQIQ/viewform}{Google
Feedback Form}

\hypertarget{disqus_thread}{}

Please enable JavaScript to view the comments powered by Disqus.

\begin{center}\rule{0.5\linewidth}{\linethickness}\end{center}

Back to top

\begin{center}\rule{0.5\linewidth}{\linethickness}\end{center}

\pagebreak


\end{document}
